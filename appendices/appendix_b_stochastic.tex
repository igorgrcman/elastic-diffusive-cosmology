\chapter{From Stochastic Mechanics to Quantum Mechanics}
\label{app:stochastic}

\begin{center}
\textit{Detailed Derivation of the Schrödinger Equation from Diffusion}
\end{center}

\vspace{1em}

This appendix provides the complete mathematical derivation connecting the Langevin equation for diffusive motion to the Schrödinger equation of quantum mechanics, following Nelson's stochastic mechanics (1966, 1985). The main text (Chapter \ref{ch:quantum_constants}) presents the physical picture; here we supply the mathematical rigor.

\section{The Langevin Equation and Fokker-Planck}
\label{app:langevin_fp}

\subsection{Setup}

Consider a particle of mass $m$ at position $\mathbf{x}(t)$ subject to:
\begin{itemize}
    \item A deterministic force: $\mathbf{F} = -\nabla V$
    \item Viscous drag: $-\gamma \mathbf{v}$
    \item Stochastic force: $\boldsymbol{\xi}(t)$ (white noise)
\end{itemize}

The \textbf{Langevin equation} is:
\begin{equation}
m\frac{d\mathbf{v}}{dt} = -\nabla V - \gamma \mathbf{v} + \boldsymbol{\xi}(t)
\label{app:langevin}
\end{equation}

The stochastic force satisfies:
\begin{align}
\langle \xi_i(t) \rangle &= 0 \\
\langle \xi_i(t) \xi_j(t') \rangle &= 2\gamma E_{\text{fluct}} \cdot \delta_{ij} \delta(t - t')
\label{app:noise_correlation}
\end{align}

In thermal systems, $E_{\text{fluct}} = k_B T$. In EDC, $E_{\text{fluct}} = \sigma \ell^2$ (elastic energy).

\subsection{Overdamped Limit}

In the \textbf{overdamped limit} ($\gamma \gg m/\tau$ where $\tau$ is the observation timescale), inertia is negligible:
\begin{equation}
\gamma \mathbf{v} = -\nabla V + \boldsymbol{\xi}(t)
\end{equation}

The velocity becomes:
\begin{equation}
\mathbf{v} = -\frac{1}{\gamma}\nabla V + \frac{1}{\gamma}\boldsymbol{\xi}(t)
\label{app:overdamped_v}
\end{equation}

\subsection{Fokker-Planck Equation}

For a stochastic process $d\mathbf{x} = \mathbf{A}(\mathbf{x})dt + \mathbf{B}(\mathbf{x})d\mathbf{W}$, where $d\mathbf{W}$ is a Wiener process, the probability density $\rho(\mathbf{x}, t)$ evolves according to the \textbf{Fokker-Planck equation}:
\begin{equation}
\frac{\partial \rho}{\partial t} = -\nabla \cdot (\mathbf{A} \rho) + D \nabla^2 \rho
\label{app:fokker_planck}
\end{equation}

where the diffusion coefficient is:
\begin{equation}
D = \frac{E_{\text{fluct}}}{\gamma}
\label{app:diffusion_def}
\end{equation}

For the overdamped Langevin equation with $\mathbf{A} = -\nabla V / \gamma$:
\begin{equation}
\frac{\partial \rho}{\partial t} = \frac{1}{\gamma}\nabla \cdot (\rho \nabla V) + D \nabla^2 \rho
\label{app:fp_with_potential}
\end{equation}

\section{Nelson's Stochastic Mechanics}
\label{app:nelson}

\subsection{The Key Insight}

Edward Nelson (1966) showed that quantum mechanics can be formulated as a theory of \textit{conservative diffusion}—diffusion without energy dissipation. The crucial step is to decompose motion into forward and backward components.

\subsection{Forward and Backward Derivatives}

Define the \textbf{forward derivative}:
\begin{equation}
D_+ x(t) = \lim_{\Delta t \to 0^+} \left\langle \frac{x(t + \Delta t) - x(t)}{\Delta t} \right\rangle
\end{equation}

and the \textbf{backward derivative}:
\begin{equation}
D_- x(t) = \lim_{\Delta t \to 0^+} \left\langle \frac{x(t) - x(t - \Delta t)}{\Delta t} \right\rangle
\end{equation}

For a diffusion process, these are generally \textit{different} due to the irregularity of Brownian paths.

\subsection{Current and Osmotic Velocities}

Define:
\begin{align}
\mathbf{v} &= \frac{1}{2}(D_+ \mathbf{x} + D_- \mathbf{x}) \quad \text{(current velocity)} \\
\mathbf{u} &= \frac{1}{2}(D_+ \mathbf{x} - D_- \mathbf{x}) \quad \text{(osmotic velocity)}
\end{align}

The \textbf{current velocity} $\mathbf{v}$ describes the net flow of probability. The \textbf{osmotic velocity} $\mathbf{u}$ describes diffusive spreading.

For a diffusion process with coefficient $D$:
\begin{equation}
\mathbf{u} = D \nabla \ln \rho
\label{app:osmotic}
\end{equation}

where $\rho(\mathbf{x}, t)$ is the probability density.

\subsection{The Stochastic Newton Equation}

Nelson postulates that the \textbf{mean acceleration} satisfies Newton's law:
\begin{equation}
m \cdot \frac{1}{2}(D_+ D_- + D_- D_+) \mathbf{x} = -\nabla V
\label{app:stochastic_newton}
\end{equation}

This can be written in terms of $\mathbf{v}$ and $\mathbf{u}$:
\begin{equation}
m\left[\frac{\partial \mathbf{v}}{\partial t} + (\mathbf{v} \cdot \nabla)\mathbf{v} - (\mathbf{u} \cdot \nabla)\mathbf{u} - D\nabla^2 \mathbf{u}\right] = -\nabla V
\label{app:nelson_newton}
\end{equation}

\subsection{Continuity Equation}

The probability density satisfies:
\begin{equation}
\frac{\partial \rho}{\partial t} + \nabla \cdot (\rho \mathbf{v}) = 0
\label{app:continuity}
\end{equation}

\section{Derivation of the Schrödinger Equation}
\label{app:schrodinger_derivation}

\subsection{The Madelung Transformation}

Define:
\begin{align}
\rho &= |\psi|^2 = \psi^* \psi \\
\mathbf{v} &= \frac{1}{m}\nabla S
\end{align}

where $S$ is a phase function. Write the wave function as:
\begin{equation}
\psi = \sqrt{\rho} \cdot e^{iS/\hbar}
\label{app:madelung}
\end{equation}

From equation \eqref{app:osmotic}:
\begin{equation}
\mathbf{u} = D \nabla \ln \rho = \frac{D}{2} \nabla \ln |\psi|^2 = D \frac{\nabla |\psi|}{|\psi|}
\end{equation}

\subsection{The Critical Identification}

The key step is to set:
\begin{equation}
\boxed{D = \frac{\hbar}{2m}}
\label{app:d_identification}
\end{equation}

Then:
\begin{equation}
\mathbf{u} = \frac{\hbar}{2m} \nabla \ln \rho
\end{equation}

\subsection{Derivation}

Substituting into Nelson's equation \eqref{app:nelson_newton} and using the continuity equation \eqref{app:continuity}, after substantial algebra (see Nelson 1966 or Fritsche \& Haugk 2003), one obtains two coupled equations:

\textbf{Continuity:}
\begin{equation}
\frac{\partial \rho}{\partial t} + \frac{1}{m}\nabla \cdot (\rho \nabla S) = 0
\end{equation}

\textbf{Hamilton-Jacobi with quantum potential:}
\begin{equation}
\frac{\partial S}{\partial t} + \frac{(\nabla S)^2}{2m} + V - \frac{\hbar^2}{2m}\frac{\nabla^2 \sqrt{\rho}}{\sqrt{\rho}} = 0
\label{app:quantum_hj}
\end{equation}

The last term is the \textbf{quantum potential}:
\begin{equation}
Q = -\frac{\hbar^2}{2m}\frac{\nabla^2 \sqrt{\rho}}{\sqrt{\rho}}
\end{equation}

\subsection{Recovery of Schrödinger}

Using the Madelung transformation \eqref{app:madelung}, the two real equations combine into a single complex equation. Define:
\begin{equation}
\psi = \sqrt{\rho} \cdot e^{iS/\hbar}
\end{equation}

Then:
\begin{align}
\nabla \psi &= \left(\frac{\nabla \rho}{2\rho} + \frac{i}{\hbar}\nabla S\right)\psi \\
\nabla^2 \psi &= \left[\frac{\nabla^2 \sqrt{\rho}}{\sqrt{\rho}} + \frac{2i}{\hbar}\frac{\nabla \sqrt{\rho}}{\sqrt{\rho}} \cdot \nabla S + \frac{i}{\hbar}\nabla^2 S - \frac{(\nabla S)^2}{\hbar^2}\right]\psi
\end{align}

Multiplying the continuity equation by $i\hbar/(2\rho)$ and adding to equation \eqref{app:quantum_hj}, one obtains:
\begin{equation}
\boxed{i\hbar \frac{\partial \psi}{\partial t} = -\frac{\hbar^2}{2m}\nabla^2 \psi + V\psi}
\label{app:schrodinger_final}
\end{equation}

This is the \textbf{Schrödinger equation}.

\section{Physical Interpretation in EDC}
\label{app:edc_interpretation}

\subsection{Origin of Diffusion}

In Nelson's original work, the diffusion coefficient $D = \hbar/(2m)$ was assumed. In EDC, it is \textit{derived}:
\begin{equation}
D = \frac{E_{\text{fluct}}}{\gamma} = \frac{\sigma \ell^2}{\beta \eta_{\text{bulk}} \ell} = \frac{\sigma \ell}{\beta \eta_{\text{bulk}}}
\end{equation}

For this to equal $\hbar/(2m)$ for all particles (universality), we require $m \cdot \ell = \text{const}$, which leads to the Compton relation and the geometric formula $\hbar = \sigma R_\xi^3/c$.

\subsection{Why Is There No Dissipation?}

In ordinary Brownian motion, diffusion is accompanied by energy dissipation (friction heats the medium). In EDC, the fluctuations are \textit{elastic}, not thermal. The membrane stores and returns energy without loss—this is ``conservative diffusion.''

Mathematically, the Fokker-Planck equation for conservative diffusion has a special structure that preserves total energy. This is encoded in Nelson's requirement that mean acceleration equals $-\nabla V/m$, not $-\nabla V/m - \gamma \mathbf{v}/m$.

\subsection{The Quantum Potential}

The quantum potential $Q = -(\hbar^2/2m)\nabla^2\sqrt{\rho}/\sqrt{\rho}$ has a clear interpretation in EDC: it is the \textbf{back-reaction of the membrane's elastic deformation} on the vortex.

Where probability density is high (many vortices), the membrane is more deformed. This creates a ``pressure'' that affects vortex motion—the quantum potential.

\section{Summary}
\label{app:summary}

The derivation proceeds as follows:
\begin{enumerate}
    \item Vortices on the membrane experience stochastic forces from the viscous Bulk.
    \item This creates diffusive motion with coefficient $D = \sigma \ell / (\beta \eta_{\text{bulk}})$.
    \item Universality of $\hbar$ requires $m \cdot \ell = \hbar/c$, fixing $D = \hbar/(2m)$.
    \item Nelson's stochastic mechanics shows that diffusion with $D = \hbar/(2m)$ is mathematically equivalent to the Schrödinger equation.
    \item The quantum potential arises from elastic back-reaction of the membrane.
\end{enumerate}

\textbf{Conclusion:} Quantum mechanics is not fundamental. It is the effective description of classical stochastic dynamics on an elastic membrane in a viscous higher-dimensional fluid.

\vspace{2em}

\noindent\textbf{Key References:}
\begin{itemize}
    \item E. Nelson, ``Derivation of the Schrödinger Equation from Newtonian Mechanics,'' \textit{Phys. Rev.} \textbf{150}, 1079 (1966).
    \item E. Nelson, \textit{Quantum Fluctuations}, Princeton University Press (1985).
    \item L. Fritsche and M. Haugk, ``A New Look at the Derivation of the Schrödinger Equation from Newtonian Mechanics,'' \textit{Ann. Phys.} \textbf{12}, 371 (2003).
\end{itemize}

