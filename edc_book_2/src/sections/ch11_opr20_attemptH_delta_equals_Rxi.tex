%!TEX root = ../EDC_Part_II_Weak_Sector.tex
% ==============================================================================
% OPR-20 Attempt H: Derive $\delta$ = R_ξ from Part I Brane Microphysics
% Status: [Def]+[Dc] (definitional identification based on physical criterion)
% ==============================================================================

\subsection{Attempt H: Thick-Brane Microphysics and the \texorpdfstring{$\delta = R_\xi$}{delta = Rxi} Gate}
\label{sec:ch11_opr20_attemptH}

Attempt~G identified a natural Robin parameter $\alpha = 2\pi$ emerging from the
ratio $\alpha = \ell/\delta$ when $\delta = R_\xi$. This places the eigenvalue
$x_1 \approx 2.4$ inside the broad target range $[2.3, 2.8]$ established in
Attempt~F---\emph{without needle tuning}. The remaining gate is to justify the
identification $\delta = R_\xi$ from the microphysics of the thick brane.

\begin{tcolorbox}[colback=gray!5!white, colframe=gray!60!black,
    title=\textbf{Attempt H Goal}]
Establish that the effective boundary-layer thickness $\delta$ entering the Robin
boundary condition equals the diffusion scale $R_\xi$ from Part~I membrane physics.

\medskip
\textbf{NOT claimed:}
\begin{itemize}[nosep]
    \item Derivation of $R_\xi$ itself from first principles (that is Part~I)
    \item Independent derivation of $M_W$ or weak scale
    \item Full BVP solution for mediator wavefunction
\end{itemize}

\textbf{Claimed:}
\begin{itemize}[nosep]
    \item $\delta = R_\xi$ follows from the physical criterion that the boundary
          layer thickness equals the characteristic field relaxation scale
    \item This is a \emph{definitional identification} \textbf{[Def]}, not a bare postulate \tagP{}
\end{itemize}
\end{tcolorbox}

% ------------------------------------------------------------------------------
\subsubsection{H.1: Definition of the Boundary Layer Thickness $\delta$}
\label{sec:attemptH_delta_def}

The Robin boundary condition $f'(\xi_0) + \alpha f(\xi_0) = 0$ emerges from the
variational principle when the brane has \emph{finite thickness}. Consider a
5D bulk scalar field $\phi(x^\mu, \xi)$ with action:
\begin{equation}
    S = \int d^4x \int_0^\ell d\xi \left[
        \frac{1}{2}(\partial_\xi \phi)^2 + \frac{1}{2}m_5^2 \phi^2
    \right]
    + S_{\text{boundary}}
    \label{eq:attemptH_bulk_action}
\end{equation}
where $S_{\text{boundary}}$ includes contributions from the brane layer.

\paragraph{Thick-brane smoothing.}
A sharp (infinitely thin) brane imposes a delta-function BC. A physically
realistic \emph{thick brane} of finite thickness $\delta$ introduces a gradient
penalty across the transition zone. Following the standard thick-brane
literature, the effective boundary action is:
\begin{equation}
    S_{\text{bdy}} = \int d^4x \left[
        -\frac{\tau}{2} \phi^2(0) + \frac{\lambda}{2} \phi(0) \partial_\xi \phi(0)
    \right]
    \label{eq:attemptH_bdy_action}
\end{equation}
where $\tau$ and $\lambda$ are dimensionful coefficients determined by the
brane structure.

\paragraph{Derivation of Robin BC.}
Varying $S + S_{\text{bdy}}$ with respect to $\phi$ at the boundary yields \tagDc{}:
\begin{equation}
    \partial_\xi \phi(0) + \alpha \phi(0) = 0,
    \quad
    \text{with} \quad \alpha = \frac{\tau}{1 + \lambda}
    \label{eq:attemptH_robin_from_action}
\end{equation}
The Robin parameter $\alpha$ has dimensions $[\alpha] = 1/\text{length}$.

\paragraph{Dimensional analysis.}
The only length scale characterizing the thick-brane transition is the
boundary-layer thickness $\delta$. On dimensional grounds:
\begin{equation}
    \alpha_{\text{phys}} \sim \frac{c_{\text{geom}}}{\delta}
    \label{eq:attemptH_alpha_dimensional}
\end{equation}
where $c_{\text{geom}}$ is a dimensionless geometric factor of order unity.
Converting to the dimensionless solver convention ($\alpha = \alpha_{\text{phys}} \cdot \ell$):
\begin{equation}
    \boxed{
    \alpha = c_{\text{geom}} \cdot \frac{\ell}{\delta}
    }
    \label{eq:attemptH_alpha_ell_delta}
\end{equation}
This structure is \tagDc{}. The question becomes: what sets $\delta$?

% ------------------------------------------------------------------------------
\subsubsection{H.2: Definition of R$_\xi$ from Part I}
\label{sec:attemptH_Rxi_def}

The scale $R_\xi$ is defined in Part~I (Framework~v2.0) as the \emph{correlation
length} of the diffusive/frozen membrane regime:

\begin{tcolorbox}[colback=blue!5!white, colframe=blue!50!black,
    title=\textbf{Definition (Part I): R$_\xi$ as Correlation Length}]
In the frozen regime, membrane fluctuations are correlated over a characteristic
scale $R_\xi$:
\begin{equation}
    \langle \phi(x) \phi(x') \rangle \sim e^{-|x - x'|/R_\xi}
    \quad \text{for } |x - x'| \gg R_\xi
    \label{eq:attemptH_Rxi_correlation}
\end{equation}
where $\phi$ denotes the membrane displacement field.

\medskip
\textbf{Physical interpretation:}
\begin{itemize}[nosep]
    \item $R_\xi$ is the \emph{diffusion length} over which Plenum energy spreads
          before the frozen boundary decouples
    \item Equivalently, the \emph{screening length} for bulk perturbations
    \item Sets the compactification circumference: $\ell = 2\pi R_\xi$ [Dc]
\end{itemize}

\textbf{Numerical value (Part I):} $R_\xi \sim 10^{-3}$ fm $= 10^{-18}$ m

\textbf{Status:} \tagP{} from Part~I diffusion physics (not derived here)
\end{tcolorbox}

% ------------------------------------------------------------------------------
\subsubsection{H.3: Physical Argument for $\delta$ = R$_\xi$}
\label{sec:attemptH_derivation}

The key observation is that both $\delta$ and $R_\xi$ characterize \emph{field
relaxation over a transition zone}:

\begin{itemize}
    \item \textbf{$\delta$ (boundary layer):} Thickness over which the Robin BC
          ``smooths out'' a sharp junction. Fields relax from bulk to brane
          behavior over this scale.

    \item \textbf{R$_\xi$ (correlation length):} Scale over which membrane
          fluctuations decay. Fields lose coherence over this distance.
\end{itemize}

\paragraph{Physical criterion.}
The boundary-layer thickness $\delta$ is the scale over which fields transition
from bulk-dominated to brane-dominated behavior. In the thick-brane model, this
transition is controlled by diffusion---the same process that sets $R_\xi$.

\begin{tcolorbox}[colback=yellow!5!white, colframe=yellow!60!black,
    title=\textbf{Identification: $\delta$ = R$_\xi$ [Def]}]
\textbf{Statement:} The effective boundary-layer thickness $\delta$ equals the
diffusion/correlation scale $R_\xi$:
\begin{equation}
    \boxed{\delta = R_\xi}
    \label{eq:attemptH_delta_Rxi}
\end{equation}

\textbf{Physical justification:}
\begin{enumerate}[nosep]
    \item The thick brane has finite width characterized by the membrane
          fluctuation scale.
    \item Field modes relax across the brane on the diffusion timescale
          corresponding to length $R_\xi$.
    \item There is no other intrinsic length available: $R_\xi$ is the
          \emph{only} sub-electroweak scale from Part~I physics.
    \item The 1/$e$ decay convention for correlation functions
          (Eq.~\ref{eq:attemptH_Rxi_correlation}) naturally defines the
          boundary-layer extent.
\end{enumerate}

\textbf{Epistemic status:} \textbf{[Def]} (definitional identification based on
physical criterion). Not \tagP{} (bare postulate) because the criterion
``boundary layer = relaxation scale'' is physics-based.
\end{tcolorbox}

\paragraph{\texorpdfstring{Consequence for $\alpha$.}{Consequence for alpha.}}
Substituting $\delta = R_\xi$ and $\ell = 2\pi R_\xi$ (from Part~I) into
Eq.~(\ref{eq:attemptH_alpha_ell_delta}) with $c_{\text{geom}} = 1$:
\begin{equation}
    \alpha = \frac{\ell}{\delta} = \frac{2\pi R_\xi}{R_\xi} = 2\pi
    \label{eq:attemptH_alpha_2pi}
\end{equation}
This is the \emph{natural} Robin parameter---no tuning required.

% ------------------------------------------------------------------------------
\subsubsection{H.4: Numerical Verification}
\label{sec:attemptH_numerics}

Using the BVP solver from Attempt~F at $\alpha = 2\pi$:

\begin{center}
\begin{tabular}{lccc}
\toprule
\textbf{Quantity} & \textbf{Formula} & \textbf{Value} & \textbf{Status} \\
\midrule
Robin parameter & $\alpha = \ell/\delta = 2\pi$ & 6.28 & \tagDc{}+\textbf{[Def]} \\
Ground state & $x_1$ from BVP & 2.41 & \tagDc{} \\
Target range & $[2.3, 2.8]$ & \checkmark (in range) & \\
\addlinespace
Circumference & $\ell = 2\pi\sqrt{2} R_\xi$ & $8.89 \times 10^{-3}$ fm & \tagDc{} \\
Mediator mass & $m_\phi = x_1/\ell$ & 53.5 GeV & \tagDc{}+\tagP{} \\
\bottomrule
\end{tabular}
\end{center}

\paragraph{Diagnostic comparison [BL].}
The predicted mediator mass $m_\phi \approx 54$ GeV is 33\% below the SM $W$
boson mass $M_W = 80.4$ GeV. This is \emph{not} fed back into the derivation;
the comparison is purely diagnostic.

\paragraph{No-smuggling verification.}
\begin{itemize}[nosep]
    \item[$\checkmark$] $\alpha = 2\pi$ from $\ell/\delta$ with geometric constants
    \item[$\checkmark$] $\delta = R_\xi$ from diffusion physics (Part~I)
    \item[$\checkmark$] $R_\xi \sim 10^{-3}$ fm from Part~I (not from weak scale)
    \item[$\times$] No $M_W$, $G_F$, $g_2$, or $v$ used as inputs
\end{itemize}

% ------------------------------------------------------------------------------
\subsubsection{H.5: Candidate Alternatives to $\delta$ = R$_\xi$}
\label{sec:attemptH_alternatives}

\begin{table}[ht]
\centering
\caption{Alternative $\delta$ candidates and their implications}
\label{tab:attemptH_delta_candidates}
\small
\begin{tabular}{llccl}
\toprule
\textbf{Candidate} & \textbf{$\delta$} & \textbf{$\alpha$} & \textbf{$x_1$} & \textbf{Status} \\
\midrule
A: Diffusion scale & $R_\xi$ & $2\pi \approx 6.3$ & 2.41 & \textbf{[Def] preferred} \\
B: Half-diffusion & $R_\xi/2$ & $4\pi \approx 12.6$ & 2.69 & [P] (no physical basis) \\
C: Double-diffusion & $2R_\xi$ & $\pi \approx 3.1$ & 1.91 & [P] (below target) \\
D: Compton (electron) & $\bar{\lambda}_e$ & $\ell/\bar{\lambda}_e$ & varies & [P] (introduces $m_e$) \\
\bottomrule
\end{tabular}
\end{table}

\textbf{Why A is preferred:}
\begin{itemize}[nosep]
    \item Candidate A ($\delta = R_\xi$) uses the \emph{only} intrinsic length
          from Part~I membrane physics.
    \item Candidates B and C have no physical motivation; the factors $1/2$
          or $2$ are arbitrary.
    \item Candidate D introduces the electron mass scale, which is not
          available at this level of the derivation (would be circular).
\end{itemize}

% ------------------------------------------------------------------------------
\subsubsection{H.6: Epistemic Summary and OPR-20 Upgrade}
\label{sec:attemptH_epistemic}

\begin{tcolorbox}[colback=blue!5!white, colframe=blue!50!black,
    title=\textbf{Attempt H: Component Status}]

\begin{center}
\begin{tabular}{lll}
\toprule
\textbf{Component} & \textbf{Status} & \textbf{Note} \\
\midrule
Robin BC from action & \tagDc{} & Standard thick-brane variation \\
$\alpha \sim \ell/\delta$ structure & \tagDc{} & Dimensional analysis \\
$\ell = 2\pi R_\xi$ & \tagDc{} & Circumference (Part I, Attempt E) \\
$R_\xi$ correlation length & \tagP{} & Part I diffusion physics \\
$\delta = R_\xi$ identification & \textbf{[Def]} & Physical criterion: relaxation scale \\
$\alpha = 2\pi$ (natural) & \tagDc{}+\textbf{[Def]} & Follows from above \\
$m_\phi \approx 54$ GeV & \tagDc{}+\tagP{} & Uses $R_\xi$ value [P] \\
\bottomrule
\end{tabular}
\end{center}
\end{tcolorbox}

\begin{tcolorbox}[
    colback=yellow!5!white,
    colframe=yellow!60!black,
    title=\textbf{OPR-20 Attempt H: Stoplight Verdict}
]
\begin{center}
\textbf{\large YELLOW [Dc]+[Def]+[P] (Gate Partially Closed)}
\end{center}

\medskip
\textbf{What Attempt H achieved:}
\begin{itemize}[nosep]
    \item Established $\delta = R_\xi$ via physical criterion (not bare postulate)
    \item Upgraded $\delta$ identification from [P] to [Def]
    \item Verified $\alpha = 2\pi$ produces $x_1 = 2.41$ (in target range)
    \item No SM inputs used; no-smuggling verified
\end{itemize}

\textbf{What remains [P]:}
\begin{itemize}[nosep]
    \item $R_\xi \sim 10^{-3}$ fm value (from Part I, not derived here)
    \item Numeric $m_\phi \approx 54$ GeV depends on $R_\xi$ value
\end{itemize}

\textbf{Upgrade achieved:}
\begin{quote}
OPR-20b ($\alpha$ provenance): \textbf{[OPEN] $\to$ YELLOW [Def]+[P]}

The $\delta = R_\xi$ gate is now definitionally closed. The remaining
dependence on $R_\xi$ value is traced to Part~I and is explicitly tagged.
\end{quote}
\end{tcolorbox}

\paragraph{Comparison to other attempts.}
\begin{center}
\begin{tabular}{lccl}
\toprule
\textbf{Attempt} & \textbf{$x_1$} & \textbf{$m_\phi$ (GeV)} & \textbf{Key finding} \\
\midrule
C/D (DD BC) & $\pi \approx 3.14$ & 70 & Best geometric factor $2\pi\sqrt{2}$ \\
E (NN BC) & $\pi/2 \approx 1.57$ & 35 & 2$\pi$ factor upgraded [Dc] \\
G (Robin $\alpha = 2\pi$) & 2.41 & 54 & Natural $\alpha$ from $\ell/\delta$ \\
\textbf{H ($\delta = R_\xi$)} & \textbf{2.41} & \textbf{54} & \textbf{$\delta$ gate closed [Def]} \\
\bottomrule
\end{tabular}
\end{center}

% ------------------------------------------------------------------------------
\subsubsection{H.7: The 33\% Discrepancy and Future Work}
\label{sec:attemptH_discrepancy}

The predicted $m_\phi \approx 54$ GeV is 33\% below $M_W = 80.4$ GeV. This
discrepancy could arise from:

\begin{enumerate}
    \item \textbf{BC choice (OPR-20a):} If the mediator is actually a KK mode
          with Dirichlet BC ($x_1 = \pi$), then $m_\phi \approx 70$ GeV (12\%
          discrepancy). See Attempt~G\_BC.

    \item \textbf{$R_\xi$ value:} The Part~I value $R_\xi \sim 10^{-3}$ fm may
          need refinement. A value $R_\xi \sim 0.67 \times 10^{-3}$ fm would
          give $m_\phi = 80$ GeV with Robin BC.

    \item \textbf{Additional geometric factors:} The factor $c_{\text{geom}} = 1$
          in Eq.~(\ref{eq:attemptH_alpha_dimensional}) may receive corrections
          from detailed junction physics.

    \item \textbf{Quantum corrections:} The tree-level analysis ignores
          wavefunction renormalization effects that could shift $x_1$.
\end{enumerate}

\textbf{Important:} The 33\% discrepancy is \emph{not} fatal. It is:
\begin{itemize}[nosep]
    \item Within dimensional analysis uncertainty (factor $\sim 2$)
    \item Improvable by refining $R_\xi$ or BC choice
    \item Not a structural failure of the derivation chain
\end{itemize}

% ------------------------------------------------------------------------------
\subsubsection{H.8: What Mathematical Result is Missing}
\label{sec:attemptH_missing_math}

The identification $\delta = R_\xi$ currently rests on a \emph{physical criterion}
(matching the boundary-layer scale to the diffusion/correlation scale). A rigorous
derivation would require \textbf{matched asymptotic analysis} or equivalent.

\begin{tcolorbox}[colback=red!5!white, colframe=red!50!black,
    title=\textbf{Lemma Stub: $\delta = R_\xi$ from Boundary Layer Analysis}]
\label{box:lemma_stub_delta_Rxi}

\textbf{Status:} \tagOPEN{} (statement only; proof not completed)

\medskip
\textbf{Desired statement:}
Let $\phi(\xi)$ satisfy the thick-brane field equation with diffusive dynamics
characterized by correlation length $R_\xi$. In the limit where the brane
transition region has width $\delta \ll \ell$, the effective Robin BC
$\phi'(0) + \alpha\phi(0) = 0$ has parameter:
\[
    \alpha = \frac{c}{\delta} + O(\delta/\ell)
\]
where $c$ is a geometric factor of order unity, and the boundary-layer thickness
$\delta$ satisfies $\delta = R_\xi$ from the matching condition.

\medskip
\textbf{Required mathematical ingredients:}
\begin{enumerate}[nosep]
    \item \textbf{Inner expansion:} Rescale $\xi = \delta \zeta$ and solve in the
          boundary layer where $\zeta = O(1)$.
    \item \textbf{Outer expansion:} Solve in the bulk where $\xi = O(\ell)$.
    \item \textbf{Matching condition:} Require the inner and outer solutions to
          agree in the overlap region $\delta \ll \xi \ll \ell$. This fixes $\delta$
          in terms of the diffusion physics.
    \item \textbf{Identification:} Show that the matching condition yields
          $\delta = R_\xi$ (the correlation length).
\end{enumerate}

\medskip
\textbf{Why this is non-trivial:}
The matching requires knowing the explicit form of the diffusion-driven
field profile in the boundary layer, which depends on the microphysics of
the brane-bulk interface. Without this, $\delta = R_\xi$ remains at [Def]
status rather than [Dc].
\end{tcolorbox}

\paragraph{Path to closure.}
To upgrade from [Def] to [Dc]:
\begin{enumerate}[nosep]
    \item Specify the field equation in the boundary layer (from 5D action)
    \item Solve the inner problem with diffusive/correlation physics
    \item Perform asymptotic matching to derive $\delta$
    \item Verify $\delta = R_\xi$ emerges from the matching, not assumed
\end{enumerate}

% ------------------------------------------------------------------------------
\subsubsection{H.9: Fail-Safe Narrative}
\label{sec:attemptH_failsafe}

\begin{tcolorbox}[colback=green!5!white, colframe=green!50!black,
    title=\textbf{Fail-Safe: Even Without $\delta = R_\xi$, the Structure Remains Valid}]
\label{box:opr20_failsafe}

The $\delta = R_\xi$ identification is important but \textbf{not structurally
essential}. If this identification fails or is revised, the EDC closure spine
remains valid with the following modifications:

\medskip
\textbf{What survives without $\delta = R_\xi$:}
\begin{itemize}[nosep]
    \item \textbf{BVP framework:} The Sturm--Liouville structure remains correct;
          only the numerical value of $\alpha$ changes.
    \item \textbf{Dimensional analysis:} $\alpha \sim \ell/\delta$ remains valid
          for \emph{some} boundary-layer scale $\delta$.
    \item \textbf{Generation counting:} $N_{\text{bound}}$ is determined by $V(\xi)$
          and BCs, independent of the specific $\delta$ value.
    \item \textbf{Framework 2.0 logic:} The ``5D cause $\to$ brane process $\to$ 3D
          shadow'' flow is unaffected.
\end{itemize}

\medskip
\textbf{What changes without $\delta = R_\xi$:}
\begin{itemize}[nosep]
    \item \textbf{Numerical predictions:} The specific value $\alpha = 2\pi$ and
          resulting $m_\phi \approx 54$ GeV would need revision.
    \item \textbf{Part I connection:} The link between weak-scale BVP and
          membrane diffusion physics would be weaker.
    \item \textbf{Epistemic status:} OPR-20b would remain [P] rather than
          upgrading to [Def].
\end{itemize}

\medskip
\textbf{Bottom line:} $\delta = R_\xi$ is a \emph{microphysical identification},
not a structural postulate. The closure spine (BVP $\to$ spectrum $\to$ observables)
is robust against its revision. The worst case is that $\delta$ becomes an additional
[P] parameter rather than being derived from Part~I.
\end{tcolorbox}