% ==============================================================================
% Unified Master Figure: One Pipeline, Many Ontologies
% ==============================================================================

\subsection{Master Diagram: One Interface Pipeline, Multiple Ontological Sources}
\label{sec:master_diagram}

Figure~\ref{fig:master_pipeline} summarizes the EDC weak-sector story in a single view.
It shows:
\begin{enumerate}[nosep]
  \item[(i)] The \emph{common} Absorption--Dissipation--Release pipeline
  \item[(ii)] The \emph{ontology-dependent} origin of pumping/excitation
  \item[(iii)] The \emph{kinematic gates} that determine which observer-facing
        outputs are allowed
\end{enumerate}

\begin{figure}[ht]
\centering
% figures/fig_master_weak_pipeline.tex
% Master unified weak-sector pipeline diagram
\begin{tikzpicture}[scale=0.90, transform shape]

% Load styles

% ─────────────────────────────────────────────────────────────────────────────
% Background regions
% ─────────────────────────────────────────────────────────────────────────────
\fill[gray!12] (0,0) rectangle (12.5,-2.6);
\fill[blue!8] (0,-2.6) rectangle (12.5,-5.8);
\fill[green!8] (0,-5.8) rectangle (12.5,-8.6);

% Region labels
\node[font=\scriptsize, gray!70!black] at (0.8,-0.3) {5D Bulk};
\node[font=\scriptsize, blue!50!black] at (0.8,-2.9) {Thick Brane};
\node[font=\scriptsize, green!50!black] at (0.8,-6.1) {3D Outputs};

% ─────────────────────────────────────────────────────────────────────────────
% Ontology sources (top row)
% ─────────────────────────────────────────────────────────────────────────────
\node[bulk box, text width=2.4cm] (Nsrc) at (2.0,-1.3)
  {Neutron\\{\tiny bulk-core}};
\node[process box, text width=2.4cm] (Msrc) at (5.0,-1.3)
  {Muon/Tau\\{\tiny brane-dominant}};
\node[process box, text width=2.4cm] (Psrc) at (8.0,-1.3)
  {Pion\\{\tiny junction-pair}};
\node[output box, text width=2.4cm] (Vsrc) at (11.0,-1.3)
  {Neutrino\\{\tiny edge mode}};

% ─────────────────────────────────────────────────────────────────────────────
% Pipeline stages (middle row)
% ─────────────────────────────────────────────────────────────────────────────
\node[brane box, text width=2.8cm] (Abs) at (2.5,-4.0)
  {Absorption\\{\tiny $E_{\text{brane}}$}};
\node[brane box, text width=2.8cm] (Dis) at (6.0,-4.0)
  {Dissipation\\{\tiny $\{\phi_k\}$}};
\node[gate box, text width=3.2cm, minimum height=1.0cm] (Rel) at (9.8,-4.0)
  {Release\\$\mathcal{P}_{\text{frozen}}$};

% ─────────────────────────────────────────────────────────────────────────────
% Connect sources to pipeline
% ─────────────────────────────────────────────────────────────────────────────
\draw[edc flow] (Nsrc.south) -- ++(0,-0.4) -| (Abs.north);
\draw[edc flow] (Msrc.south) -- ++(0,-0.4) -| (Abs.north);
\draw[edc flow] (Psrc.south) -- ++(0,-0.6) -| (Dis.north);

% Pipeline flow
\draw[edc flow] (Abs.east) -- (Dis.west);
\draw[edc flow] (Dis.east) -- (Rel.west);

% ─────────────────────────────────────────────────────────────────────────────
% Outputs (bottom row)
% ─────────────────────────────────────────────────────────────────────────────
\node[output box, text width=2.2cm] (eout) at (3.5,-7.2)
  {$e^-$\\{\tiny charged}};
\node[output box, text width=2.2cm] (nuout) at (6.5,-7.2)
  {$\nu,\bar\nu$\\{\tiny ledger}};
\node[output box, text width=2.2cm] (pout) at (9.5,-7.2)
  {$p$\\{\tiny anchor}};

% Dashed for rare/composite
\node[rectangle, draw=orange!50, dashed, rounded corners=2pt,
      fill=orange!5, text width=1.6cm, font=\tiny, align=center] (hadout) at (11.5,-7.2)
  {hadrons\\(rare)};

% Connect release to outputs
\draw[edc arrow] (Rel.south) -- ++(0,-0.5) -| (eout.north);
\draw[edc arrow] (Rel.south) -- ++(0,-0.5) -| (nuout.north);
\draw[edc arrow] (Rel.south) -- ++(0,-0.5) -| (pout.north);
\draw[edc dashed] (Rel.south) -- ++(0,-0.5) -| (hadout.north);

% Neutrino direct connection (edge mode → output)
\draw[edc dashed, purple!50] (Vsrc.south) -- ++(0,-3.0) -| (nuout.north east);

% ─────────────────────────────────────────────────────────────────────────────
% Q-gate annotation
% ─────────────────────────────────────────────────────────────────────────────
\node[rectangle, draw=red!40, fill=red!5, rounded corners=2pt,
      text width=5.0cm, font=\tiny, align=left] at (2.5,-5.4)
  {\textbf{Energy gate:} $Q > m_{\text{products}}$\\
   Neutron: $Q = 0.78$ MeV $\Rightarrow$ $e$ only\\
   ($\mu$ forbidden: $m_\mu = 106$ MeV $\gg Q$)};

\end{tikzpicture}

\caption{\textbf{Unified weak-sector pipeline.}
All cases share the same interface skeleton (Absorption $\to$ Dissipation $\to$ Release),
but differ in their ontology (bulk-core junction vs brane-dominant defect vs composite
junction-pair vs edge mode). The energy gate $\mathcal{P}_{\text{energy}}$ enforces
kinematic allowance: channels whose rest-mass threshold exceeds the available $Q$-value
are forbidden at leading order (e.g., neutron cannot emit $\mu$ because
$m_\mu \approx 106$ MeV $\gg Q_n \approx 0.78$ MeV).}
\label{fig:master_pipeline}
\end{figure}

\subsection{Kinematic Gates and Output Allowance}
\label{sec:kinematic_gates}

\subsubsection{Why ``Forbidden'' Means Kinematically Closed}

Throughout this chapter, ``forbidden'' is used in the strict kinematic sense:
the channel is closed because the available $Q$-value is below the rest-mass
threshold required to create the product. \textbf{No additional dynamical
assumption is needed for such a closure.}

This is important: when we say ``$\mathcal{P}_{\text{energy}}$ forbids the
$\mu$ channel,'' we mean that the energy gate simply blocks a rest-mass
threshold that is orders of magnitude too high. This is not a metaphysical
prohibition---it is arithmetic.

\subsubsection{Gate Summary Table}

\begin{table}[ht]
\centering
\caption{\textbf{Kinematic gate summary across weak-sector case studies.}
``Allowed'' means kinematically open at leading order; branching ratios are
not derived here. All values are \tagBL{}.}
\label{tab:gates}
\begin{tabular}{llll}
\toprule
\textbf{Case} & \textbf{Available scale} & \textbf{Threshold test} &
\textbf{Allowed output} \\
\midrule
Neutron $n \to p + \cdots$ & $Q_n \approx 0.782$ MeV &
$m_\ell c^2 \le Q_n$ & $e^-$ only; $\mu$ forbidden \\
Muon $\mu \to \cdots$ & $m_\mu c^2 \approx 106$ MeV &
$m_\ell c^2 \le m_\mu c^2$ & $e^-$ (lightest defect) \\
Tau $\tau \to \cdots$ & $m_\tau c^2 \approx 1777$ MeV &
multiple thresholds & $e^-$, $\mu^-$, hadrons \\
Pion $\pi \to \ell\nu$ & $m_\pi c^2 \approx 140$ MeV &
helicity/BC suppression & $\mu$ dominates; $e$ suppressed \\
\bottomrule
\end{tabular}
\end{table}

\subsubsection{The Neutron Example in Full Sentences}

In neutron beta decay, the total energy available to the leptonic sector is the
$Q$-value:
\begin{equation}
Q_n = (m_n - m_p - m_e)c^2 \approx 0.782~\text{MeV},
\label{eq:Qn_value}
\end{equation}
so any channel requiring production of a heavier charged lepton is closed.

Because $m_\mu c^2 \approx 105.7$ MeV $\gg Q_n$, a neutron \emph{cannot} produce
a muon in ordinary decay. This is the precise meaning of ``$\mathcal{P}_{\text{energy}}$
forbids the $\mu$ channel'': the energy gate blocks a rest-mass threshold that is
more than 100 times too high.

\begin{tcolorbox}[mechanism, title={Q-Gate for Neutron Decay}]
\textbf{Kinematic gate} \tagBL{}/\tagDc{}:
\begin{align}
Q_\beta(e) &= m_n - m_p - m_e = 1.293 - 0.511 = 0.782~\text{MeV} > 0
  && \Rightarrow \text{OPEN} \\
Q_\beta(\mu) &= m_n - m_p - m_\mu = 1.293 - 105.66 = -104.4~\text{MeV} < 0
  && \Rightarrow \text{CLOSED}
\end{align}

The electron channel is kinematically allowed; the muon channel is kinematically
forbidden. This is baseline physics, not an EDC-specific claim.
\end{tcolorbox}

\subsubsection{What the Gates Tell Us}

The gate structure demonstrates that:
\begin{enumerate}[nosep]
  \item \textbf{Channel selection is kinematic}: Neutron $\to$ electron (not muon)
        because $Q_\beta(\mu) < 0$.
  \item \textbf{Mode overlap matters separately}: Muon $\to$ leptons only because
        $\mathcal{P}_{\text{mode}}$ forbids hadronic channels (not kinematics).
  \item \textbf{Chirality suppression is real but distinct}: Pion $\to$ muon
        dominates over electron by $(m_\mu/m_e)^2$ due to
        $\mathcal{P}_{\text{chir}}$ (boundary conditions).
  \item \textbf{Electron stability is structural}: No lower charged mode exists,
        so all gates are blocked.
\end{enumerate}

These are \emph{facts} that EDC must be consistent with; they are not EDC-derived
claims. The EDC contribution is to interpret these gates as projections in the
thick-brane interface picture.

