% ==============================================================================
% Attempt 3B: Can 1/α Arise from 5D Electrostatics Without 5D Magnetism?
% Status: Analysis complete — no robust mechanism found
% ==============================================================================

\subsection{\texorpdfstring{Attempt 3B: EM-Sector Origin of $1/\alpha$}{Attempt 3B: EM-Sector Origin of 1/alpha}}
\label{sec:attempt3B_em_options}

\begin{tcolorbox}[edcGuardrail, title=\textbf{Epistemic Status: Systematic Audit}]
This section documents a systematic test of whether the factor $1/\alpha$ in the
candidate ratio $m_\mu/m_e \approx (3/2)/\alpha$ can be derived from the electromagnetic
sector of EDC, consistent with Book~I (Framework v2.0). We tested six options and
found \textbf{no robust mechanism}.
\end{tcolorbox}

\subsubsection{Book I Baseline}

From Framework v2.0, the relevant established results are:
\begin{itemize}[nosep]
    \item $\alpha = (4\pi + 5/6)/(6\pi^5) = 1/137.027$ \tagDc{}%
          \footnote{Term status: $4\pi$ is \tagDc{}; $5/6$ is motivated \tagP{} (derivation pending, see \S\ref{sec:frozen_regime}). Whole expression is \tagDc{} (conditional), not \tagDer{}.}
    \item $m_p/m_e = 6\pi^5 = 1836.12$ \tagDer{}
    \item $m_\mu/m_e = (3/2)(1 + \alpha^{-1})$ \tagI{} (0.14\% accuracy)
    \item EM sector: Kaluza-Klein mechanism (charge = winding) \tagDer{}
\end{itemize}

\paragraph{Critical observation.}
In Book~I, $\alpha^{-1}$ can be written as:
\begin{equation}
    \alpha^{-1} = \frac{6\pi^5}{4\pi + 5/6} = \frac{m_p/m_e}{4\pi + 5/6}
\end{equation}
This means the $1/\alpha$ in the muon formula is \textbf{not} an independent
EM mechanism---it is a \emph{derived consequence} of how $\alpha$ relates to
the proton/electron mass ratio.

\subsubsection{Options Matrix (Stoplight Verdicts)}

We tested six physically motivated routes for obtaining $1/\alpha$ from EM-sector physics:

\begin{center}
\small
\begin{tabular}{p{2.8cm}cccp{4cm}}
\toprule
\textbf{Option} & \textbf{1/$\alpha$?} & \textbf{4$\pi$ closed?} & \textbf{Stoplight} & \textbf{Failure mode} \\
\midrule
O1: Full 5D gauge $A_M$ & No & N/A & \textcolor{red!80!black}{\textbf{RED}} & Self-energy $\propto \alpha$, not $1/\alpha$ \\
O2: E-only static & No & N/A & \textcolor{red!80!black}{\textbf{RED}} & Same as O1 \\
O3: Scalar $\Phi$ model & No & N/A & \textcolor{red!80!black}{\textbf{RED}} & Scalar doesn't invert coupling \\
O4: Susceptibility $\chi$ & Maybe & No & \textcolor{orange!80!black}{\textbf{YELLOW}} & Requires $\chi \propto e^2$ (unproven) \\
O5: Kinematic B & No & N/A & \textcolor{red!80!black}{\textbf{RED}} & Kinematic, not coupling factor \\
O6: Gauge stiffness & No & N/A & \textcolor{red!80!black}{\textbf{RED}} & Stiffness $1/e^2$ cancels source $e^2$ \\
\bottomrule
\end{tabular}
\end{center}

\paragraph{Stoplight criteria (for transparency):}
\begin{itemize}[nosep]
    \item \textcolor{green!50!black}{\textbf{GREEN}}: Robust $1/\alpha$ mechanism, convention-independent, no free parameters
    \item \textcolor{orange!80!black}{\textbf{YELLOW}}: Plausible pathway but requires additional assumption not yet derived
    \item \textcolor{red!80!black}{\textbf{RED}}: Produces $\alpha$ (not $1/\alpha$), or fails for structural reasons
\end{itemize}

\subsubsection{Why Standard EM Self-Energy Fails}

For a localized charged defect, the electrostatic self-energy scales as:
\begin{equation}
    E_{\text{self}} \propto \frac{e^2}{4\pi\varepsilon_0 r} \propto \alpha \times \frac{\hbar c}{r}
\end{equation}
This gives $\alpha$ dependence, \textbf{not} $1/\alpha$. To obtain an inverse-coupling
dependence, we would need an energy that scales as $1/e^2$, such as:
\begin{itemize}[nosep]
    \item Gauge field ``stiffness'' (kinetic term coefficient $\propto 1/e^2$)
    \item Dielectric ``restoring force'' if susceptibility $\chi \propto e^2$
\end{itemize}
However, in physical energies, the gauge stiffness cancels with the source strength,
and $\chi \propto e^2$ has not been derived from the EDC action.

\subsubsection{The YELLOW Pathway: Polarization Model}

Option O4 remains open as a \textbf{potential} route:

\begin{tcolorbox}[colback=orange!5, colframe=orange!50!black,
    title=\textbf{Potential Mechanism (Not Yet Derived)}]
If the brane interface has a polarization response with susceptibility $\chi$, and if
$\chi \propto e^2$ can be derived from the EDC action, then the polarization energy:
\begin{equation}
    E_{\text{pol}} = \frac{1}{2\chi} \int (\delta P)^2 \, d^3x \propto \frac{1}{e^2} \propto \frac{1}{\alpha}
\end{equation}
would provide an inverse-coupling mechanism.

\textbf{Status:} (open) --- requires derivation of $\chi \propto e^2$ from brane physics.
\end{tcolorbox}

\subsubsection{Verdict}

\begin{tcolorbox}[colback=red!5, colframe=red!50!black,
    title=\textbf{Attempt 3B Verdict: No Robust Mechanism Found}]
No EM-sector option produces a clean $1/\alpha$ mechanism consistent with Book~I.

\textbf{Key insight:} The $1/\alpha$ in the Framework muon formula is not an independent
EM mechanism---it arises from the \emph{definition} of $\alpha$ in terms of geometric
quantities. The muon formula:
\begin{equation}
    \frac{m_\mu}{m_e} = \frac{3}{2}\left(1 + \alpha^{-1}\right)
    = \frac{3}{2}\left(1 + \frac{m_p/m_e}{4\pi + 5/6}\right)
\end{equation}
is a relation involving the proton mass scale, not an independent EM coupling inversion.

\textbf{Recommendation:} Keep the muon formula as \tagI{} (identified), not \tagDer{}.
Redirect effort to other derivation targets (e.g., V--A structure, Koide constraint).
\end{tcolorbox}

\paragraph{Open problems (status: open).}
\begin{enumerate}[nosep]
    \item Derive $\chi \propto e^2$ for brane polarizability from the EDC action
    \item Derive the $(3/2)$ factor from oscillator spectrum in the $\xi$-dimension
    \item Prove that muon wavefunction extension samples baryon configuration space
\end{enumerate}
