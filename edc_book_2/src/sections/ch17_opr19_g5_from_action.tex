%!TEX root = ../EDC_Part_II_Weak_Sector_rebuild.tex
% ==============================================================================
% Chapter 17: OPR-19 — The 5D Gauge Coupling g₅ from the Action
% Status: CONDITIONAL [Dc] — reduction formula derived, warp/domain remain [P]
% Version: 1.0 (2026-01-25)
% ==============================================================================
%
% MISSION (OPR-19):
% Derive the canonical relation between the 5D gauge coupling g₅ and the
% effective 4D coupling g₄ by explicit dimensional reduction from the 5D action.
%
% ABSOLUTE CONSTRAINTS:
% - NO SM observables as inputs (M_W, G_F, v, sin²θ_W)
% - Every step must be explicitly computed
% - All rescalings tracked
% - Scale Taxonomy respected (Δ, δ, ℓ, R_ξ distinct)
%
% ==============================================================================

\chapter{OPR-19: The 5D Gauge Coupling from First Principles}
\label{sec:ch17_opr19}

% ==============================================================================
% READER'S MAP
% ==============================================================================

\begin{tcolorbox}[colback=blue!3!white, colframe=blue!50!black,
    title=\textbf{Reader's Map: Chapter 17}]
\label{box:ch17_reader_map}

\textbf{Goal:} Derive the canonical relation between the 5D gauge coupling $g_5$
and the effective 4D coupling $g_4$ by explicit dimensional reduction.

\textbf{Inputs:}
\begin{itemize}[nosep]
    \item A stated 5D metric ansatz (warped geometry) \tagP{}
    \item The 5D gauge kinetic term with proper measure $\sqrt{-G}$
    \item No SM observables used as anchors
\end{itemize}

\textbf{Output:}
\begin{itemize}[nosep]
    \item Boxed formula: $1/g_4^2 = (1/g_5^2) \int d\xi\,|f(\xi)|^2$
    \item Weight function $W(\xi)$ derived (not assumed)
    \item All rescalings explicit and tracked
\end{itemize}

\textbf{Status:} CONDITIONAL \tagDc{} if $W(\xi)$ is derived and normalization
is explicit; warp factor and domain remain \tagP{}.

\textbf{Cross-references:}
\begin{itemize}[nosep]
    \item OPR-21 (Ch14): Fermion BVP uses same metric
    \item OPR-20 (Ch13): Mediator mass uses effective coupling
    \item OPR-04 (Ch16): Scale Taxonomy for $\ell$ definition
\end{itemize}
\end{tcolorbox}

% ==============================================================================
% SECTION 1: ACTION AND CONVENTIONS
% ==============================================================================

\section{Action and Conventions}
\label{sec:ch17_action}

\subsection{The 5D Gauge Action}
\label{subsec:ch17_gauge_action}

We start from the 5D gauge field action with canonical normalization:
\begin{equation}
    S_{\text{gauge}}^{(5D)} = -\frac{1}{4g_5^2} \int d^5x \sqrt{-G} \,
    G^{MA} G^{NB} F_{MN} F_{AB}
    \label{eq:ch17:5D_action}
\end{equation}
where:
\begin{itemize}[nosep]
    \item $G_{AB}$ is the 5D metric with signature $(-,+,+,+,+)$
    \item Capital indices $M, N, A, B \in \{0,1,2,3,5\}$ run over 5D
    \item Greek indices $\mu, \nu \in \{0,1,2,3\}$ run over 4D
    \item $F_{MN} = \partial_M A_N - \partial_N A_M$ is the field strength
    \item $g_5$ is the 5D gauge coupling
\end{itemize}

\textbf{Epistemic status:} \tagM{} --- standard gauge field theory definition.

\subsection{Warped Metric Ansatz}
\label{subsec:ch17_metric}

We adopt the warped metric:
\begin{equation}
    ds^2 = G_{AB} dx^A dx^B = e^{2A(\xi)} \eta_{\mu\nu} dx^\mu dx^\nu + d\xi^2
    \label{eq:ch17:metric}
\end{equation}
where:
\begin{itemize}[nosep]
    \item $A(\xi)$ is the warp factor (function of the transverse coordinate only)
    \item $\eta_{\mu\nu} = \text{diag}(-1,+1,+1,+1)$ is the 4D Minkowski metric
    \item $\xi \in [0, \ell]$ is the transverse (fifth) coordinate
    \item $\ell$ is the domain size (see Scale Taxonomy, Chapter~\ref{sec:ch16_reader_map})
\end{itemize}

\textbf{Epistemic status:} \tagP{} --- ansatz. The warp factor $A(\xi)$ and domain
$\ell$ are postulated, not derived from first principles.

\paragraph{Metric components.}
From~(\ref{eq:ch17:metric}):
\begin{align}
    G_{\mu\nu} &= e^{2A(\xi)} \eta_{\mu\nu}, \quad
    G_{55} = 1, \quad
    G_{\mu 5} = 0 \\
    G^{\mu\nu} &= e^{-2A(\xi)} \eta^{\mu\nu}, \quad
    G^{55} = 1, \quad
    G^{\mu 5} = 0
    \label{eq:ch17:metric_components}
\end{align}

\paragraph{Metric determinant.}
\begin{equation}
    G = \det(G_{AB}) = e^{8A(\xi)} \cdot \det(\eta_{\mu\nu}) \cdot 1 = -e^{8A(\xi)}
\end{equation}
\begin{equation}
    \boxed{\sqrt{-G} = e^{4A(\xi)}}
    \label{eq:ch17:sqrt_G}
\end{equation}

\textbf{Epistemic status:} \tagDc{} --- direct computation from the metric ansatz.

\subsection{Units and Conversion}
\label{subsec:ch17_units}

\begin{tcolorbox}[colback=yellow!5!white, colframe=orange!50!black,
    title=\textbf{Unit Conventions}]
\begin{itemize}[nosep]
    \item Natural units: $\hbar = c = 1$
    \item Length conversion: $1 \text{ fm} = 5.0677 \text{ GeV}^{-1}$ \tagBL{}
    \item Dimension of $g_5$: $[g_5] = [\text{length}]^{1/2}$ (see~\S\ref{subsec:ch17_dimensions})
    \item Dimension of $g_4$: $[g_4] = 1$ (dimensionless)
\end{itemize}
\end{tcolorbox}

% ==============================================================================
% SECTION 2: MODE DECOMPOSITION
% ==============================================================================

\section{Mode Decomposition and Gauge Choice}
\label{sec:ch17_decomposition}

\subsection{Kaluza-Klein Ansatz}
\label{subsec:ch17_kk_ansatz}

Decompose the 5D gauge field into 4D modes:
\begin{equation}
    A_\mu(x,\xi) = \sum_n a_\mu^{(n)}(x) f_n(\xi)
    \label{eq:ch17:kk_decomposition}
\end{equation}
where:
\begin{itemize}[nosep]
    \item $a_\mu^{(n)}(x)$ are 4D gauge fields (KK modes)
    \item $f_n(\xi)$ are mode profiles in the transverse direction
\end{itemize}

\paragraph{Gauge choice.}
We work in unitary gauge with $A_5 = 0$. This eliminates the scalar
component that would arise from the fifth component.

\textbf{Epistemic status:} \tagDc{} --- standard KK decomposition with gauge choice.

\subsection{Field Strength Components}
\label{subsec:ch17_field_strength}

The field strength $F_{MN} = \partial_M A_N - \partial_N A_M$ has components:
\begin{align}
    F_{\mu\nu} &= \partial_\mu A_\nu - \partial_\nu A_\mu
    = \sum_n f_{\mu\nu}^{(n)}(x) f_n(\xi) \label{eq:ch17:Fmunu} \\
    F_{\mu 5} &= \partial_\mu A_5 - \partial_\xi A_\mu
    = -\sum_n a_\mu^{(n)}(x) f_n'(\xi) \label{eq:ch17:Fmu5}
\end{align}
where $f_{\mu\nu}^{(n)} = \partial_\mu a_\nu^{(n)} - \partial_\nu a_\mu^{(n)}$
is the 4D field strength.

\textbf{Epistemic status:} \tagDc{} --- direct computation.

% ==============================================================================
% SECTION 3: DIMENSIONAL REDUCTION
% ==============================================================================

\section{Dimensional Reduction of the Kinetic Term}
\label{sec:ch17_reduction}

\subsection{Expansion of $F_{MN} F^{MN}$}
\label{subsec:ch17_expansion}

The contraction $F_{MN} F^{MN}$ decomposes as:
\begin{equation}
    F_{MN} F^{MN} = F_{\mu\nu} F^{\mu\nu} + 2 F_{\mu 5} F^{\mu 5}
    \label{eq:ch17:F_squared}
\end{equation}

\subsection{The $F_{\mu\nu} F^{\mu\nu}$ Term}
\label{subsec:ch17_Fmunu_term}

\paragraph{Step 1: Index contraction.}
\begin{equation}
    F_{\mu\nu} F^{\mu\nu} = G^{\mu\alpha} G^{\nu\beta} F_{\mu\nu} F_{\alpha\beta}
    = e^{-2A} \eta^{\mu\alpha} \cdot e^{-2A} \eta^{\nu\beta} \cdot F_{\mu\nu} F_{\alpha\beta}
\end{equation}
\begin{equation}
    = e^{-4A(\xi)} \, f_{\mu\nu}^{(n)} f^{(n)\mu\nu} |f_n(\xi)|^2
    \label{eq:ch17:Fmunu_contraction}
\end{equation}

\paragraph{Step 2: Combine with $\sqrt{-G}$.}
\begin{equation}
    \sqrt{-G} \cdot F_{\mu\nu} F^{\mu\nu}
    = e^{4A} \cdot e^{-4A} \cdot f_{\mu\nu}^{(n)} f^{(n)\mu\nu} |f_n(\xi)|^2
\end{equation}

\begin{tcolorbox}[colback=green!5!white, colframe=green!50!black,
    title=\textbf{Critical Result: Warp Factor Cancellation}]
\label{box:ch17_warp_cancel}
\begin{equation}
    \boxed{\sqrt{-G} \cdot F_{\mu\nu} F^{\mu\nu} = f_{\mu\nu}^{(n)} f^{(n)\mu\nu} |f_n(\xi)|^2}
    \label{eq:ch17:warp_cancel}
\end{equation}

\textbf{The warp factors $e^{4A}$ from $\sqrt{-G}$ and $e^{-4A}$ from the
index contractions exactly cancel!}

This is non-trivial: the 4D gauge kinetic term has a \emph{flat measure}
in the $\xi$-integral, regardless of the warp factor profile.

\textbf{Epistemic status:} \tagDc{} --- explicit computation with metric factors.
\end{tcolorbox}

\subsection{The $F_{\mu 5} F^{\mu 5}$ Term}
\label{subsec:ch17_Fmu5_term}

For completeness:
\begin{equation}
    F_{\mu 5} F^{\mu 5} = G^{\mu\alpha} G^{55} F_{\mu 5} F_{\alpha 5}
    = e^{-2A} \cdot 1 \cdot a_\mu^{(n)} a^{(n)\mu} |f_n'(\xi)|^2
\end{equation}
\begin{equation}
    \sqrt{-G} \cdot F_{\mu 5} F^{\mu 5} = e^{4A} \cdot e^{-2A} \cdot a_\mu^{(n)} a^{(n)\mu} |f_n'(\xi)|^2
    = e^{2A(\xi)} \cdot a_\mu^{(n)} a^{(n)\mu} |f_n'(\xi)|^2
    \label{eq:ch17:Fmu5_term}
\end{equation}

This term contributes to the mass sector (involving $f_n'$), not to the
kinetic normalization. The warp factor $e^{2A}$ remains in this term.

% ==============================================================================
% SECTION 4: EFFECTIVE 4D COUPLING
% ==============================================================================

\section{Canonical Normalization and Definition of $g_4$}
\label{sec:ch17_normalization}

\subsection{The 4D Effective Action}
\label{subsec:ch17_4d_action}

Integrating~(\ref{eq:ch17:5D_action}) over $\xi$ using~(\ref{eq:ch17:warp_cancel}):
\begin{equation}
    S_{\text{gauge}}^{(4D)} = -\frac{1}{4g_5^2} \int d^4x
    \left( \int_0^\ell d\xi \, |f_n(\xi)|^2 \right) f_{\mu\nu}^{(n)} f^{(n)\mu\nu}
    + \text{(mass terms)}
    \label{eq:ch17:4D_action}
\end{equation}

\subsection{Definition of the Effective 4D Coupling}
\label{subsec:ch17_g4_def}

For canonical 4D normalization $S = -\frac{1}{4g_4^2} \int d^4x \, f_{\mu\nu} f^{\mu\nu}$,
we identify:

\begin{tcolorbox}[colback=red!5!white, colframe=red!50!black,
    title=\textbf{Result: $g_5 \to g_4$ Reduction Formula}]
\label{box:ch17_result}
\begin{equation}
    \boxed{\frac{1}{g_{4,n}^2} = \frac{1}{g_5^2} \int_0^\ell d\xi \, |f_n(\xi)|^2}
    \label{eq:ch17:g4_formula}
\end{equation}

\textbf{Weight function:} $W(\xi) = 1$ (flat measure, due to warp cancellation).

\textbf{Conditions:}
\begin{itemize}[nosep]
    \item Warped metric ansatz~(\ref{eq:ch17:metric}) \tagP{}
    \item Domain $\xi \in [0, \ell]$ with $\ell$ postulated \tagP{}
    \item Unitary gauge $A_5 = 0$ \tagDc{}
    \item No brane-localized kinetic terms \tagP{}
\end{itemize}

\textbf{Epistemic status:} CONDITIONAL \tagDc{} --- formula derived; $A(\xi)$ and $\ell$ remain \tagP{}.
\end{tcolorbox}

\subsection{Special Cases}
\label{subsec:ch17_special_cases}

\paragraph{Case A: Flat zero mode.}
If $f_0(\xi) = 1/\sqrt{\ell}$ (uniform, normalized):
\begin{equation}
    \int_0^\ell d\xi \, |f_0|^2 = \frac{1}{\ell} \cdot \ell = 1
\end{equation}
\begin{equation}
    \boxed{g_{4,0} = g_5 \quad \text{(flat zero mode)}}
    \label{eq:ch17:flat_mode}
\end{equation}

\paragraph{Case B: Localized mode.}
If $f_n(\xi)$ is localized with effective width $\delta_{\text{eff}} \ll \ell$:
\begin{equation}
    \int_0^\ell d\xi \, |f_n|^2 \sim \delta_{\text{eff}}
    \quad \Rightarrow \quad
    g_{4,n}^2 \sim \frac{g_5^2}{\delta_{\text{eff}}}
    \label{eq:ch17:localized_mode}
\end{equation}
The effective 4D coupling is \emph{enhanced} for localized modes.

\begin{tcolorbox}[colback=blue!5!white, colframe=blue!50!black,
    title=\textbf{Remark: Normalization Convention}]
\textbf{This chapter adopts the convention} $\int_0^\ell d\xi \, |f_n|^2 = 1$
for normalized modes. Under this choice:
\begin{itemize}[nosep]
    \item Flat zero mode: $f_0 = 1/\sqrt{\ell}$ gives $g_{4,0} = g_5$.
\end{itemize}

\textbf{Alternative convention.} If one instead uses unnormalized $f_0 = 1$
(constant), then $\int_0^\ell d\xi \, |f_0|^2 = \ell$, yielding:
\begin{equation*}
    \frac{1}{g_4^2} = \frac{\ell}{g_5^2}
    \quad \Rightarrow \quad
    g_4 = \frac{g_5}{\sqrt{\ell}}
\end{equation*}

\textbf{Physical observables are invariant}: the different conventions
correspond to different field redefinitions that compensate the coupling
change. The key structural result---\emph{warp factors cancel in the kinetic
term}---is independent of normalization choice.
\end{tcolorbox}

% ==============================================================================
% SECTION 5: DIMENSIONAL ANALYSIS
% ==============================================================================

\section{Dimensional Analysis}
\label{sec:ch17_dimensions}
\label{subsec:ch17_dimensions}

\paragraph{Dimensions of $g_5$.}
From the action~(\ref{eq:ch17:5D_action}):
\begin{itemize}[nosep]
    \item $[S] = 1$ (dimensionless)
    \item $[\int d^5x] = L^5$
    \item $[F^2] = L^{-4}$
    \item $[\sqrt{-G}] = 1$
\end{itemize}
Therefore:
\begin{equation}
    [g_5^2] = L^{5-4} = L, \quad [g_5] = L^{1/2}
    \label{eq:ch17:dim_g5}
\end{equation}

\paragraph{Dimensions of $g_4$.}
The 4D gauge coupling is dimensionless:
\begin{equation}
    [g_4] = 1
    \label{eq:ch17:dim_g4}
\end{equation}

\paragraph{Consistency check.}
From~(\ref{eq:ch17:g4_formula}) with $[f_n] = L^{-1/2}$ for proper normalization:
\begin{equation}
    [1/g_4^2] = [1/g_5^2] \cdot [L] \cdot [L^{-1}] = L^{-1} \cdot L \cdot L^{-1} = L^{-1}
\end{equation}
This requires $[g_4^2] = L$, which contradicts~(\ref{eq:ch17:dim_g4}).

\textbf{Resolution:} If $f_n$ is dimensionless (normalized so $\int |f_n|^2 d\xi$
has dimension $L$), then:
\begin{equation}
    [1/g_4^2] = L^{-1} \cdot L = 1 \quad \checkmark
\end{equation}

% ==============================================================================
% SECTION 6: NO-SMUGGLING CHECKLIST
% ==============================================================================

\section{Epistemic and No-Smuggling Checklist}
\label{sec:ch17_checklist}

\begin{tcolorbox}[colback=green!3!white, colframe=green!50!black,
    title=\textbf{No-Smuggling Checklist (OPR-19)}]
\label{box:ch17_no_smuggling}
\begin{itemize}
    \item[\ding{51}] Start from explicit 5D gauge action with $\sqrt{-G}$
    \item[\ding{51}] Derive reduction weight $W(\xi)$ from metric (not assumed)
    \item[\ding{51}] Show all field rescalings for canonical 4D normalization
    \item[\ding{51}] No SM observables used as inputs ($M_W, G_F, v, \sin^2\theta_W$ absent)
    \item[\ding{51}] Scale Taxonomy respected: $\Delta, \delta, \ell, R_\xi$ not identified implicitly
    \item[\ding{51}] Dimensional analysis verified
\end{itemize}
\end{tcolorbox}

\begin{tcolorbox}[colback=yellow!5!white, colframe=orange!50!black,
    title=\textbf{Assumptions Made (Explicit)}]
\label{box:ch17_assumptions}
\begin{enumerate}[nosep]
    \item[(A-19-1)] Warped metric ansatz~(\ref{eq:ch17:metric}) \tagP{}
    \item[(A-19-2)] Domain $\xi \in [0, \ell]$ with $\ell$ postulated \tagP{}
    \item[(A-19-3)] Unitary gauge $A_5 = 0$ \tagDc{}
    \item[(A-19-4)] Mode profiles $f_n(\xi)$ satisfy appropriate BVP \tagP{}/\tagDc{}
    \item[(A-19-5)] No brane-localized kinetic terms \tagP{}
\end{enumerate}

\textbf{Scale Taxonomy note:} The domain size $\ell$ is from the Scale Taxonomy
(Chapter~\ref{sec:ch16_reader_map}). If $\ell = n\Delta$ is assumed, this
requires citing assumption (A3).
\end{tcolorbox}

% ==============================================================================
% SECTION 7: SUMMARY AND STATUS
% ==============================================================================

\section{Summary and Closure Status}
\label{sec:ch17_summary}

\begin{tcolorbox}[colback=blue!3!white, colframe=blue!50!black,
    title=\textbf{OPR-19 Summary Box}]
\label{box:ch17_summary}

\textbf{Main Result:}
\begin{equation}
    \frac{1}{g_{4,n}^2} = \frac{1}{g_5^2} \int_0^\ell d\xi \, |f_n(\xi)|^2
    \tag{\ref{eq:ch17:g4_formula}}
\end{equation}

\textbf{Key Insight:} Warp factors cancel for the 4D gauge kinetic term,
yielding a flat measure $W(\xi) = 1$.

\textbf{Special Case:} For flat zero mode, $g_{4,0} = g_5$.

\textbf{Closure Status:}
\begin{center}
\begin{tabular}{lll}
\toprule
\textbf{Component} & \textbf{Status} & \textbf{Reference} \\
\midrule
Reduction formula & DERIVED \tagDc{} & Eq.~(\ref{eq:ch17:g4_formula}) \\
Warp cancellation & DERIVED \tagDc{} & Eq.~(\ref{eq:ch17:warp_cancel}) \\
Dimensional analysis & VERIFIED \tagM{} & \S\ref{sec:ch17_dimensions} \\
Warp factor $A(\xi)$ & POSTULATED \tagP{} & Eq.~(\ref{eq:ch17:metric}) \\
Domain $\ell$ & POSTULATED \tagP{} & Scale Taxonomy \\
Mode profiles $f_n$ & CONDITIONAL \tagDc{} & Depends on BC \\
\midrule
\textbf{Overall OPR-19} & \textbf{CONDITIONAL} \tagDc{} & \\
\bottomrule
\end{tabular}
\end{center}

\textbf{What remains [P]:}
\begin{itemize}[nosep]
    \item Warp factor $A(\xi)$ --- not derived from brane-bulk matching
    \item Domain size $\ell$ --- not derived from first principles
    \item Mode normalization --- depends on boundary conditions
\end{itemize}
\end{tcolorbox}

% ==============================================================================
% CROSS-REFERENCES TO OTHER CHAPTERS
% ==============================================================================

\paragraph{Cross-references.}
\begin{itemize}[nosep]
    \item \textbf{OPR-21} (Chapter~\ref{ch:bvp_master_key}): Fermion BVP uses same metric;
          fermion normalization has different weight due to spinor structure.
    \item \textbf{OPR-20} (Chapter~13): Mediator mass eigenvalue problem uses
          the effective coupling derived here.
    \item \textbf{OPR-04} (Chapter~\ref{sec:ch16_reader_map}): Scale Taxonomy
          defines $\ell$ and distinguishes it from $\Delta$, $\delta$, $R_\xi$.
\end{itemize}

% ==============================================================================
% STOPLIGHT VERDICT (2026-01-29)
% ==============================================================================
\subsection{OPR-19 Stoplight Verdict}
\label{subsec:opr19_stoplight}

\begin{tcolorbox}[colback=yellow!10!white, colframe=orange!60!black,
    title=\textbf{OPR-19: Stoplight Verdict}]

\textbf{Status: \textcolor{YellowOrange}{YELLOW} [Dc]}

\textbf{What is derived:}
\begin{itemize}[nosep]
\item $g_4^2 = g_5^2/\ell$ dimensional reduction formula
\item Warp factor cancellation mechanism
\item Mode normalization structure
\end{itemize}

\textbf{Blockers to GREEN:}
\begin{itemize}[nosep]
\item Warp factor $A(\xi)$ not derived from brane-bulk matching
\item Domain size $\ell$ not derived from first principles
\item Boundary conditions remain [P]
\end{itemize}

\textbf{Next upgrade:} Derive $\ell$ from Israel junction or stability criterion.

See \S\ref{sec:gate_registry} for consolidated gate registry.
\end{tcolorbox}
