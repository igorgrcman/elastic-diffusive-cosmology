% ==============================================================================
% Chapter 11: The Fermi Constant from Geometry
% Status: [Dc] numerical closure via electroweak relations + [P] mode overlap mechanism
% ==============================================================================

\section{The Fermi Constant from Geometry}
\label{sec:ch11_gf}

\begin{tcolorbox}[edcGuardrail, title=\textbf{Epistemic Status}]
This chapter consolidates the EDC treatment of the Fermi constant $G_F$:
\begin{itemize}[nosep]
    \item Structural pathway: $G_F$ emerges from integrating out a 5D mediator \tagDc{}
    \item Numerical closure: $G_F$ exact from electroweak relations + $\sin^2\theta_W = 1/4$ \tagDc{}
    \item Mode overlap: geometric suppression explains ``weakness'' \tagP{}
    \item Connection to V$-$A: chirality filter enters via boundary conditions \tagDc{}
\end{itemize}
\textbf{What is NOT claimed:} We do not derive the numerical value of $G_F$
from first principles alone. The derivation uses electroweak relations that
incorporate the measured $\alpha$ and $v$ (or equivalently $G_F$ itself).
The structural claim is that $G_F$'s \emph{smallness} has geometric origin.
\end{tcolorbox}

% ==============================================================================
% FRAMEWORK 2.0 LANGUAGE COMPLIANCE
% ==============================================================================
\begin{tcolorbox}[colback=blue!3!white, colframe=blue!50!black,
    title=\textbf{Framework 2.0 Language Compliance}]
\small
\textbf{EDC Projection Principle:} Every physical process has a \textbf{5D bulk+brane cause}
whose observable residue is a \textbf{3D shadow} on the observer boundary.

\textbf{In this chapter:}
\begin{itemize}[nosep]
    \item \textbf{5D cause:} Finite brane thickness $\delta$; tightly localized fermion modes.
    \item \textbf{Brane process:} Mediator propagation across brane layer; mode overlap.
    \item \textbf{3D shadow:} Fermi constant $G_F$ as effective contact coupling.
\end{itemize}

\textbf{Standard Model} takes $G_F$ as measured input; EDC derives its smallness from
5D geometry (mass gap + localization suppression).
\end{tcolorbox}

% ------------------------------------------------------------------------------
% PHYSICAL PROCESS NARRATIVE (Feynman-style)
% ------------------------------------------------------------------------------

\begin{tcolorbox}[colback=green!5!white, colframe=green!50!black,
    title=\textbf{The Fermi Constant in EDC: Physical Process}]
\textbf{What physically happens, step by step:}

\textbf{Step 1: The brane layer has finite thickness.}
In the EDC picture, our 3D universe is a membrane (brane) with finite thickness $\delta$
along the fifth dimension. This thickness creates a \textbf{mass gap}---any field
propagating across the brane layer acquires an effective mass $m_\phi \sim 1/\delta$ \tagP{}.

\textbf{Step 2: Fermions are localized near the brane face.}
Left-handed fermions are trapped at $\xi \approx 0$ (the observer-facing brane boundary)
by the asymmetric mass profile. Their wavefunction $f_L(\xi)$ is sharply peaked with
width $\sigma_L \sim 1/m_0 \ll \delta$. Right-handed modes are expelled toward
infinity and effectively decouple \tagDc{}.

\textbf{Step 3: The mediator must traverse the brane layer.}
When a weak process occurs (e.g., $\beta$-decay), information must propagate across
the brane thickness via a mediator field $\phi$. This propagator costs a factor
$\sim 1/m_\phi^2$ in the amplitude---the larger the mass gap, the more suppressed
the coupling \tagDc{}.

\textbf{Step 4: Overlap integral sets the contact strength.}
The effective 4D coupling involves integrating four fermion mode functions over $\xi$:
$G_F \sim G_5 \int |f_L(\xi)|^4 d\xi$. Because $f_L(\xi)$ is tightly localized, this
integral is small---roughly $G_F \sim G_5 / \sigma_L \sim G_5 m_0$ \tagP{}.

\textbf{Step 5: The result is an effective contact interaction.}
At low energies (below the mediator mass), we see no propagator---just a point-like
four-fermion vertex with coupling $G_F$. The ``weakness'' of weak interactions is
\textbf{geometric}: it measures how hard it is to transfer energy across the
brane layer given the tight localization of participating fermions \tagI{}.
\end{tcolorbox}

% ------------------------------------------------------------------------------
% BOOK-READY INTRODUCTION
% ------------------------------------------------------------------------------

\paragraph{Chapter overview.}
The Fermi constant $G_F \approx 1.17 \times 10^{-5}$ GeV$^{-2}$ sets the scale of
weak interactions. In the Standard Model, this small value comes from $W$-boson
exchange with $G_F = \sqrt{2}g^2/(8M_W^2)$. But \emph{why} is $G_F$ so small?
Why is the weak force ``weak''?

EDC offers a geometric answer: weak interactions are not fundamental gauge vertices
but \textbf{effective contact terms} arising from integrating out a brane-layer
mediator. The smallness of $G_F$ reflects:
\begin{enumerate}[nosep]
    \item The mediator mass gap $m_\phi$ (set by brane geometry)
    \item Mode overlap suppression (fermions are tightly localized)
    \item Chirality selection (only left-handed modes couple efficiently)
\end{enumerate}

This chapter presents both the \emph{structural pathway} (mechanism) and the
\emph{numerical closure} (quantitative derivation), carefully distinguishing what
is derived from what remains postulated.

% ------------------------------------------------------------------------------
% DERIVATION CHAIN BOX
% ------------------------------------------------------------------------------

\begin{tcolorbox}[colback=green!5, colframe=green!50!black,
    title=\textbf{Derivation Chain: What Is Independent vs.\ What Is Not}]
\begin{description}[style=nextline, leftmargin=1em, font=\normalfont\bfseries]
    \item[Independent EDC step \tagDer{}:]
        $\mathbb{Z}_6$ subgroup counting $\Rightarrow \sin^2\theta_W = 1/4$ (bare).
        \emph{This is the geometrically derived prediction.}

    \item[Standard physics step \tagBL{}:]
        RG running from lattice scale to $M_Z$ using known beta functions.

    \item[Derived identities \tagDc{}:]
        Electroweak coupling relations: $g^2 = 4\pi\alpha/\sin^2\theta_W$,
        $M_W = gv/2$, $G_F = g^2/(4\sqrt{2}M_W^2)$.

    \item[Circularity caveat (important):]
        The Higgs VEV $v = (\sqrt{2}G_F)^{-1/2} = 246.2$ GeV is experimentally
        determined \emph{from} $G_F$ (muon decay). Therefore:
        \textbf{$G_F$ ``exact agreement'' is a consistency closure within
        SM relations, not an independent EDC prediction.}
        The true independent prediction is $\sin^2\theta_W = 1/4$.
\end{description}
\end{tcolorbox}

% ------------------------------------------------------------------------------
% READER MAP
% ------------------------------------------------------------------------------

\begin{tcolorbox}[colback=blue!5, colframe=blue!50!black,
    title=\textbf{Reader Map: What This Chapter Establishes}]
\begin{description}[style=nextline, leftmargin=1em, font=\normalfont\bfseries]
    \item[Derived \tagDc{}:]
        $G_F = g^2/(4\sqrt{2}M_W^2)$ from electroweak relations;
        numerical value exact once $\sin^2\theta_W = 1/4$ is fixed;
        structural form $G_{\text{EDC}} \sim g_{\text{eff}}^2/m_\phi^2$;
        dimensional consistency.

    \item[Identified \tagI{}:]
        Mediator mass $\leftrightarrow$ brane thickness;
        overlap suppression $\leftrightarrow$ mode localization;
        chirality filter $\leftrightarrow$ V$-$A structure (Ch.~\ref{ch:va_structure}).

    \item[Postulated \tagP{}:]
        5D coupling $g_5$ normalization;
        explicit mode profiles;
        mediator spectrum from $\xi$-geometry;
        overlap integrals with explicit boundary conditions.

    \item[Open (not addressed):]
        First-principles $G_F$ without using $\alpha$ or $v$ as inputs;
        mediator mass from throat geometry;
        complete thick-brane BVP solution.
\end{description}
\end{tcolorbox}

% ==============================================================================
\subsection{Baseline: The Fermi Constant in Standard Model}
\label{sec:ch11_baseline}

The Fermi constant is the effective coupling strength of weak interactions \tagBL{}:
\begin{equation}
    G_F = 1.1663787(6) \times 10^{-5}~\text{GeV}^{-2}
    \label{eq:ch11_GF_value}
\end{equation}

In the Standard Model, $G_F$ arises from $W$-boson exchange \tagBL{}:
\begin{equation}
    G_F = \frac{\sqrt{2}}{8} \frac{g^2}{M_W^2} = \frac{g^2}{4\sqrt{2} M_W^2}
    \label{eq:ch11_GF_SM}
\end{equation}
where $g \approx 0.65$ is the $SU(2)_L$ gauge coupling and $M_W \approx 80.4$ GeV.

\paragraph{\texorpdfstring{Why is $G_F$ small?}{Why is GF small?}}
In the SM, this is ``explained'' by $M_W$ being heavy. But why is $M_W \approx 80$ GeV?
That requires the Higgs mechanism with a VEV $v \approx 246$ GeV. The hierarchy
$G_F \sim 1/v^2$ is ultimately unexplained---it's an input, not an output.

% ==============================================================================
\subsection{Structural Pathway: Mediator Integration}
\label{sec:ch11_structural}

In EDC, we treat weak interactions not as fundamental gauge vertices but as
effective contact terms arising from thick-brane microphysics \tagDc{}.

\subsubsection{The Toy Setup}

Introduce a mediator field $\phi(x,y)$ localized in the brane layer with mass
gap $m_\phi$ \tagP{}:
\begin{equation}
    \mathcal{L}_\phi = \frac{1}{2}(\partial_\mu\phi)^2 + \frac{1}{2}(\partial_y\phi)^2
    - \frac{1}{2}m_\phi^2 \phi^2
    \label{eq:ch11_L_phi}
\end{equation}

The bulk-brane dynamics couple to the mediator at the bulk-facing boundary:
\begin{equation}
    \mathcal{L}_{\text{int}} = g_5 \, J(x) \, \phi(x, y = -\delta/2)
    \label{eq:ch11_L_int}
\end{equation}
where $J(x)$ is a source current from bulk pumping (e.g., junction relaxation).

\subsubsection{Tree-Level Integration}

Integrating out $\phi$ at tree level yields the effective contact interaction
\tagDc{}:
\begin{equation}
    \boxed{
    \mathcal{L}_{\text{eff}} = -\frac{g_5^2}{2m_\phi^2} \,
    \mathcal{O}_{\text{overlap}} \, J(x) J(x)
    }
    \label{eq:ch11_Leff}
\end{equation}
where $\mathcal{O}_{\text{overlap}}$ encodes wavefunction overlaps and
boundary-condition effects.

\begin{tcolorbox}[edcCornerstone, title=\textbf{Physical Interpretation (Canonical)}]
Equation~\eqref{eq:ch11_Leff} is \textbf{not} a fundamental ``weak vertex'';
it is the low-energy residue of a 5D bulk$\to$brane transfer process \tagDc{}.

The source $J(x)$ represents bulk-facing pumping into the brane layer via the
mediator $\phi$. Integrating out $\phi$ compresses that transfer into an
effective local $JJ$ term. The apparent smallness of the coupling is therefore
\textbf{geometric suppression}---set by:
\begin{itemize}[nosep]
    \item Mediator mass gap $m_\phi$ (from brane thickness/KK spectrum)
    \item Mode-profile overlap $\mathcal{O}_{\text{overlap}}$ (localization)
    \item Boundary-condition factor $\mathcal{O}_{\text{BC}}$ (chirality filter)
\end{itemize}
\end{tcolorbox}

% --- FIGURE PLACEHOLDER: Mediator Integration ---
\begin{figure}[htbp]
\centering
\fbox{\parbox{0.85\textwidth}{\centering
\textbf{[FIGURE PLACEHOLDER]}\\[1em]
\textit{Schematic: From 5D mediator to 4D contact interaction}\\[0.5em]
\textbf{Top panel:} The brane layer (shaded region, thickness $\delta$) with:\\
$\bullet$ Left side: bulk-facing boundary where source $J(x)$ couples\\
$\bullet$ Mediator field $\phi$ propagating across layer (wavy line)\\
$\bullet$ Right side: observer-facing boundary where $J(x)$ couples again\\[0.5em]
\textbf{Bottom panel:} Effective 4D picture after integrating out $\phi$:\\
$\bullet$ Point-like four-fermion vertex (no internal line)\\
$\bullet$ Coupling strength $G_F \sim g_{\text{eff}}^2 / m_\phi^2$\\[0.5em]
Annotation: ``The propagator $1/m_\phi^2$ is why weak is weak'' \tagI{}.
}}
\caption{\textbf{Geometric origin of the Fermi constant.}
In EDC, the weak four-fermion interaction arises from integrating out a
brane-layer mediator. The mass gap $m_\phi \sim 1/\delta$ and the localization
of fermion modes combine to produce the small effective coupling $G_F$ \tagDc{}.
This explains ``why weak is weak'' without requiring the Higgs mechanism as
the fundamental explanation.}
\label{fig:ch11_mediator_integration}
\end{figure}

\subsubsection{The Effective Coupling}

Define the effective coupling \tagP{}:
\begin{equation}
    g_{\text{eff}} \equiv g_5 \times \mathcal{O}_{\text{overlap}}
    \times \mathcal{O}_{\text{BC}}
    \label{eq:ch11_geff}
\end{equation}
so that:
\begin{equation}
    \boxed{
    G_{\text{EDC}} \sim \frac{g_{\text{eff}}^2}{m_\phi^2}
    }
    \label{eq:ch11_GEDC}
\end{equation}
with $[G_{\text{EDC}}] = [E]^{-2}$ as required.

% ==============================================================================
\subsection{Numerical Closure via Electroweak Relations}
\label{sec:ch11_numerical}

While the structural pathway is incomplete (open factors), EDC achieves
\emph{numerical closure} through electroweak relations once $\sin^2\theta_W$ is
fixed by geometry.

\subsubsection{The Derivation Chain}

\begin{theorem}[$G_F$ from Electroweak Unification {\normalfont \tagDc{}}]
\label{thm:ch11_GF}
From the EDC-derived $\sin^2\theta_W = 1/4$ at the lattice scale
(Chapter~\ref{ch:z6_program}), after RG running to $M_Z$:
\begin{align}
    \sin^2\theta_W(M_Z) &= 0.2314 \quad \text{(0.08\% from PDG)} \\
    g^2 &= \frac{4\pi\alpha}{\sin^2\theta_W} = 0.4246 \\
    M_W &= \frac{gv}{2} = \frac{0.6516 \times 246.2}{2} = 80.2 \text{ GeV}
\end{align}

The Fermi constant then follows:
\begin{equation}
    \boxed{
    G_F = \frac{g^2}{4\sqrt{2}M_W^2} = \frac{0.4246}{4\sqrt{2}(80.2)^2}
    = 1.166 \times 10^{-5} \text{ GeV}^{-2}
    }
    \label{eq:ch11_GF_derived}
\end{equation}

\textbf{Experimental:} $G_F^{\text{exp}} = 1.166 \times 10^{-5}$ GeV$^{-2}$
\tagBL{} --- \textbf{exact within adopted EW identities}.
\end{theorem}

\begin{remark}[Self-Consistency, Not Independent Prediction]
\label{rem:ch11_firewall}
The ``exact agreement'' for $G_F$ reflects the self-consistency of electroweak
relations, \textbf{not} an independent EDC prediction.

\textbf{The circularity caveat:} In SM conventions, the Higgs VEV is determined
from $G_F$ via $v = (\sqrt{2}G_F)^{-1/2}$. Since we use $v$ as input to compute
$M_W$, and then derive $G_F$ from $M_W$, the ``exact'' result is a consistency
identity. This is analogous to computing $G_F$ from $G_F$---not a derivation.

\textbf{The true EDC prediction} is:
\begin{equation}
    \sin^2\theta_W(\mu_{\text{lattice}}) = \frac{|\mathbb{Z}_2|}{|\mathbb{Z}_6|}
    = \frac{2}{6} = \frac{1}{4}
    \label{eq:ch11_sin2_input}
\end{equation}

After RG running, this gives $\sin^2\theta_W(M_Z) = 0.2314$, which agrees with
PDG at 0.08\%. \textbf{This} is the non-trivial, falsifiable prediction.

Everything else ($g^2$, $M_W$, $G_F$) follows from:
\begin{itemize}[nosep]
    \item Standard electroweak unification relations \tagBL{}
    \item Standard RG running from lattice scale to $M_Z$ \tagBL{}
    \item Measured values of $\alpha$ and $v$ (where $v$ depends on $G_F$) \tagBL{}
\end{itemize}

\textbf{Bottom line:} $G_F$ numerical closure is \emph{conditional} on SM
relations. The independent EDC content is $\sin^2\theta_W = 1/4$.
\end{remark}

% ==============================================================================
\subsection{\texorpdfstring{Mode Overlap: Why $G_F$ Is Small}{Mode Overlap: Why GF Is Small}}
\label{sec:ch11_overlap}

The structural pathway identifies \emph{why} weak interactions are weak:
geometric suppression from mode localization.

\subsubsection{The Overlap Integral}

The 5D Fermi coupling has dimension $[G_5] = [E]^{-3}$. To get the 4D coupling
$[G_F] = [E]^{-2}$, we integrate over the fifth dimension \tagDc{}:
\begin{equation}
    G_F = G_5 \int_0^\infty d\xi \, |f_L(\xi)|^4 = G_5 \times I_4
    \label{eq:ch11_overlap}
\end{equation}
where $f_L(\xi)$ is the left-handed fermion mode profile and $I_4$ has dimension
of length (inverse energy in natural units).

\subsubsection{Estimating $I_4$}

% --- FIGURE PLACEHOLDER: Mode Overlap ---
\begin{figure}[htbp]
\centering
\fbox{\parbox{0.85\textwidth}{\centering
\textbf{[FIGURE PLACEHOLDER]}\\[1em]
\textit{Schematic: Mode localization and the $|f_L|^4$ overlap integral}\\[0.5em]
\textbf{Left panel:} Plot of $f_L(\xi)$ vs $\xi$:\\
$\bullet$ Sharp peak at $\xi = 0$ (observer boundary)\\
$\bullet$ Exponential decay into brane layer\\
$\bullet$ Width $\sigma_L \sim 1/m_0 \sim$ 1 fm\\[0.5em]
\textbf{Right panel:} Plot of $|f_L(\xi)|^4$ vs $\xi$:\\
$\bullet$ Even sharper peak (fourth power concentrates the function)\\
$\bullet$ Shaded area = $I_4 = \int |f_L|^4 d\xi \sim m_0$\\
$\bullet$ Arrow: ``4-fermion contact = overlap of 4 mode functions''\\[0.5em]
Annotation: Tighter localization $\Rightarrow$ smaller $I_4$ $\Rightarrow$ smaller $G_F$ \tagI{}.
}}
\caption{\textbf{Mode overlap determines $G_F$ magnitude.}
The 4D Fermi constant arises from integrating the fourth power of the
left-handed mode profile over the extra dimension. Tighter localization
(smaller $\sigma_L$) produces a \emph{smaller} overlap integral $I_4$,
which is why tightly bound fermions have weaker effective coupling.
The geometric smallness of $I_4$ is the origin of ``weak'' in weak
interactions \tagP{}.}
\label{fig:ch11_mode_overlap}
\end{figure}

For the asymmetric mass profile $m(\xi) = m_0(1 - e^{-z/\lambda})$
(Chapter~\ref{ch:va_structure}), the left-handed mode is:
\begin{equation}
    f_L(\xi) = N_L \exp\left(-m_0\chi(\xi)\right), \quad
    \chi(\xi) = \xi - \lambda(1 - e^{-\xi/\lambda})
    \label{eq:ch11_fL}
\end{equation}

The mode is localized at $\xi = 0$ with effective width $\sigma_L \sim 1/m_0$.
For $m_0 \sim 200$ MeV:
\begin{equation}
    I_4 \sim \frac{1}{\sigma_L} \sim m_0 \sim 200 \text{ MeV}
    \label{eq:ch11_I4_estimate}
\end{equation}

\subsubsection{Order-of-Magnitude Check}

Combining $G_5 \sim g_5^2/M_{5,\mathrm{Pl}}^2$ with $I_4$ \tagP{}:
\begin{equation}
    G_F \sim \frac{g_5^2}{M_{5,\mathrm{Pl}}^2} \times I_4
    \sim \frac{(4\pi)^2}{(200 \text{ GeV})^2} \times 0.2 \text{ GeV}
    \sim 10^{-3} \text{ GeV}^{-2}
    \label{eq:ch11_GF_estimate}
\end{equation}

This is $\sim 100\times$ larger than observed! The discrepancy indicates:
\begin{itemize}[nosep]
    \item Additional suppression from wave function normalization
    \item Factors of $4\pi$ from angular integrals
    \item The SM relation $G_F = g^2/(4\sqrt{2}M_W^2)$ captures the correct physics
\end{itemize}

\begin{tcolorbox}[colback=yellow!5, colframe=orange!60!black,
    title=\textbf{Honest Assessment: Mode Overlap Status (YELLOW-B)}]
The mode overlap mechanism provides the \textbf{qualitative understanding}
of why $G_F$ is small:
\begin{itemize}[nosep]
    \item Fermions are tightly localized ($\sigma_L \ll$ brane thickness)
    \item The overlap of four mode functions is highly suppressed
    \item This is the geometric origin of weak interaction ``weakness''
\end{itemize}

However, the \textbf{quantitative precision} requires the full electroweak
machinery. The mode overlap is \tagP{}; the numerical closure is \tagDc{}.
\end{tcolorbox}

\subsubsection{What Exactly Is Missing for RED-C $\to$ GREEN-A?}

To upgrade mode overlap from qualitative (YELLOW-B) to quantitative (GREEN-A),
the following concrete calculations are required:

\begin{enumerate}[nosep]
    \item \textbf{5D gauge coupling $g_5$ from action normalization:}
          Derive $g_5$ from the canonical normalization of the 5D gauge field
          action, not from dimensional estimates. This requires specifying the
          5D gauge kinetic term and its reduction to 4D.

    \item \textbf{Mediator mass $m_\phi$ from KK reduction:}
          Perform the Kaluza-Klein reduction along the $\xi$-direction (throat
          geometry) to obtain the spectrum. Identify the lowest massive mode
          as the mediator and express $m_\phi$ in terms of geometric parameters
          ($R_\xi$, throat length, etc.).

    \item \textbf{Mode profiles $f_L(\xi)$ from thick-brane BVP:}
          Solve the boundary value problem for fermion localization with
          explicit boundary conditions at both brane faces. Normalize the
          solutions and compute the overlap integral $I_4 = \int |f_L|^4 d\xi$
          exactly, not by order-of-magnitude.

    \item \textbf{Boundary-condition factor $\mathcal{O}_{\text{BC}}$:}
          Evaluate the chirality projection and frozen-mode operators on the
          actual mode profiles to get the numerical suppression factor.
\end{enumerate}

Until these are computed, the mode overlap remains a \textbf{mechanism}, not a
\textbf{derivation}.

% ==============================================================================
\subsection{\texorpdfstring{Connection to V$-$A Structure}{Connection to V-A Structure}}
\label{sec:ch11_va}

The chirality filter from Chapter~\ref{ch:va_structure} enters the effective
coupling via the boundary-condition factor $\mathcal{O}_{\text{BC}}$.

\paragraph{Left-handed localization.}
The asymmetric mass profile selects chirality:
\begin{align}
    \psi_L &\propto \exp\left(-\int_0^\xi m(\xi')\,d\xi'\right) \quad \text{normalizable} \\
    \psi_R &\propto \exp\left(+\int_0^\xi m(\xi')\,d\xi'\right) \quad \text{non-normalizable}
\end{align}

Only left-handed fermions couple efficiently to the brane-layer mediator.
This is the geometric origin of V$-$A structure in weak currents.

\paragraph{Quantitative suppression.}
The right-handed mode amplitude at the brane is suppressed by \tagDc{}:
\begin{equation}
    \frac{|\psi_R(0)|}{|\psi_L(0)|} \sim e^{-m_0\lambda} \sim e^{-200 \text{ MeV} \times 1 \text{ fm}}
    \sim e^{-1} \sim 0.37
    \label{eq:ch11_chiral_suppression}
\end{equation}

For the $|f|^4$ overlap, this becomes $(0.37)^4 \approx 0.02$---roughly 50-fold
suppression of right-handed contributions, consistent with the observed
V$-$A dominance.

% ==============================================================================
% DEPENDENCY & STATUS MINI-BOX
% ==============================================================================

\begin{tcolorbox}[colback=gray!5!white, colframe=gray!60!black,
    title=\textbf{Dependency Map \& Status}]
\textbf{What this chapter depends on:}
\begin{itemize}[nosep]
    \item Chapter~\ref{ch:z6_program}: $\sin^2\theta_W = 1/4$ from $\mathbb{Z}_6$ counting \tagDer{}
    \item Chapter~\ref{ch:va_structure}: V$-$A chirality filter from asymmetric mass profile \tagDc{}
    \item Baseline facts: $\alpha$, $G_F$, $v$ from PDG/CODATA \tagBL{}
    \item Standard physics: Electroweak unification relations, RG running \tagBL{}
\end{itemize}

\textbf{What depends on this chapter:}
\begin{itemize}[nosep]
    \item Chapter 10 (Electroweak unification): uses $G_F$ structural form
    \item Neutron lifetime derivation (Paper 3): uses $G_F$ in rate calculation
    \item OPR-19, OPR-20, OPR-21, OPR-22: $G_F$ first-principles open problems
\end{itemize}

\textbf{Consistency checks (not closures):}
\begin{itemize}[nosep]
    \item[\ding{51}] $\sin^2\theta_W(M_Z) = 0.2314$ (0.08\% from PDG) \tagDer{}
    \item[\ding{51}] $G_F$ exact via EW relations (with $v$ caveat) \tagDc{}
    \item[\ding{51}] Structural form $G \sim g^2/m^2$ dimensionally correct \tagDc{}
    \item[\ding{51}] V$-$A connection established via chirality filter \tagDc{}
    \item[$\circ$] Mode overlap $I_4$ from BVP: \textbf{OPEN} (OPR-21)
    \item[$\circ$] First-principles $G_F$: \textbf{OPEN} (OPR-22)
\end{itemize}
\end{tcolorbox}

% ==============================================================================
\subsection{\texorpdfstring{Summary: What Determines $G_F$?}{Summary: What Determines GF?}}
\label{sec:ch11_summary_table}

\begin{table}[ht]
\centering
\caption{What determines $G_F$ in EDC (color-coded by derivation level)}
\label{tab:ch11_summary}
\begin{tabular}{p{3.5cm}p{4.5cm}cc}
\toprule
\textbf{Factor} & \textbf{Physical Origin} & \textbf{Tag} & \textbf{Level} \\
\midrule
\multicolumn{4}{l}{\textit{GREEN-A: Electroweak consistency closure}} \\
$\sin^2\theta_W = 1/4$ & $\mathbb{Z}_6$ subgroup counting & \tagDer{} & GREEN-A \\
$g^2 = 4\pi\alpha/\sin^2\theta_W$ & Electroweak unification & \tagDc{} & GREEN-A \\
$M_W = gv/2$ & Higgs mechanism (+$v$ caveat) & \tagDc{} & GREEN-A \\
$G_F = g^2/(4\sqrt{2}M_W^2)$ & Electroweak relation & \tagDc{} & GREEN-A \\
\midrule
\multicolumn{4}{l}{\textit{YELLOW-B: Geometric suppression intuition}} \\
Mode overlap $I_4$ & Fermion localization & \tagP{} & YELLOW-B \\
Why weak is ``weak'' & Overlap suppression & \tagI{} & YELLOW-B \\
\midrule
\multicolumn{4}{l}{\textit{RED-C: Full 5D first-principles (open)}} \\
5D coupling $g_5$ & Action normalization & \tagP{} & RED-C (OPR-19) \\
Mediator mass $m_\phi$ & $\xi$-geometry KK reduction & \tagP{} & RED-C (OPR-20) \\
Mode profiles $f_L(\xi)$ & Thick-brane BVP & (open) & RED-C (OPR-21) \\
First-principles $G_F$ & Complete derivation & (open) & RED-C (OPR-22) \\
\bottomrule
\end{tabular}
\end{table}

% ==============================================================================
\subsection{Stoplight Verdict}
\label{sec:ch11_verdict}

\begin{table}[ht]
\centering
\caption{Chapter 11 overall verdict}
\label{tab:ch11_verdict}
\begin{tabular}{lcc}
\toprule
\textbf{Claim} & \textbf{Level} & \textbf{Tag} \\
\midrule
$\sin^2\theta_W = 1/4$ (independent prediction) & GREEN-A & \tagDer{} \\
$G_F$ via EW relations ($v$ caveat) & GREEN-A & \tagDc{} \\
Structural form $G \sim g^2/m^2$ & GREEN-A & \tagDc{} \\
Connection to V$-$A (Ch.~\ref{ch:va_structure}) & GREEN-A & \tagDc{} \\
\midrule
Mode overlap mechanism & YELLOW-B & \tagP{} \\
Why weak is ``weak'' & YELLOW-B & \tagP{}/\tagI{} \\
\midrule
First-principles $G_F$ & RED-C & (open) (OPR-22) \\
\bottomrule
\end{tabular}
\end{table}

\begin{tcolorbox}[colback=green!5, colframe=green!50!black,
    title=\textbf{Chapter 11 Summary}]
\textbf{The strongest independent claim (GREEN-A):}
\begin{quote}
EDC predicts $\sin^2\theta_W = 1/4$ (bare) from $\mathbb{Z}_6$ subgroup counting.
After standard RG running, this gives $\sin^2\theta_W(M_Z) = 0.2314$, agreeing
with PDG at \textbf{0.08\%}. This is a non-trivial, falsifiable prediction.
\end{quote}

\textbf{The conditional closure (GREEN-A with caveat):}
\begin{enumerate}[nosep]
    \item $G_F = 1.166 \times 10^{-5}$ GeV$^{-2}$ from electroweak relations,
          \textbf{but} this uses $v$ which is itself determined from $G_F$.
          This is consistency, not independent prediction.
    \item Structural form $G_{\text{EDC}} \sim g_{\text{eff}}^2/m_\phi^2$
          established \tagDc{}.
    \item V$-$A connection via chirality filter \tagDc{}.
\end{enumerate}

\textbf{The geometric intuition (YELLOW-B):}
Mode overlap suppression explains \emph{why} $G_F$ is small (qualitative).

\textbf{The open frontier (RED-C):}
First-principles derivation requires: $g_5$ from action, $m_\phi$ from KK,
profiles from BVP. Until then, quantitative mode overlap remains postulated.
\end{tcolorbox}

\textbf{Bottom line:} The true EDC prediction is $\sin^2\theta_W = 1/4$ (0.08\%
agreement after RG). The $G_F$ numerical closure is a consistency check within
SM relations, not an independent prediction. The structural pathway provides
geometric understanding of weak interaction ``weakness,'' but quantitative
first-principles derivation remains open.

% NOTE: Detailed closure attempts (ch11_* files) are included in Chapter 11
% of the main document, not here, to avoid duplicate label definitions.
