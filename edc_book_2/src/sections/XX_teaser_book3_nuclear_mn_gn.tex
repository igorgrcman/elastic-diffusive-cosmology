% =============================================================================
% XX_teaser_book3_nuclear_mn_gn.tex
%
% TEASER CHAPTER: Topological Model for Nuclear Structure (Full)
%
% This is the COMPLETE content from BOOK_SECTION_TOPOLOGICAL_PINNING_MODEL.tex,
% presented as a "teaser" for Book 3. The full nuclear framework (derivations,
% extensions, heavy nuclei, nuclear matter) will be developed in Book 3.
%
% Status: [Dc/I] - Partially derived, partially identified
% Author: EDC Research
% Date: 2026-01-29
% =============================================================================

\chapter{Teaser: Topological Model for Nuclear Structure}
\label{ch:teaser-nuclear-full}

% =============================================================================
% TEASER PREAMBLE
% =============================================================================

\begin{tcolorbox}[colback=blue!5!white, colframe=blue!75!black, title={Book 3 Teaser — Complete Chapter}]
This chapter presents the \textbf{complete} topological pinning model for nuclear structure
as developed for Book~2. It demonstrates that the EDC framework extends meaningfully
to nuclear physics, with quantitative predictions for:

\begin{itemize}
    \item Nuclear binding energies (He-4, C-12, O-16 within 3\%)
    \item Be-8 instability (correctly predicted)
    \item Neutron stability in nuclei (topological suppression)
    \item Alpha-decay half-lives (Frustration-Corrected Geiger-Nuttall Law)
\end{itemize}

\textbf{What this chapter contains:}
\begin{enumerate}
    \item \textbf{Neutron Lifetime from 5D Topology} — instanton calculation, $\tau_n \approx 880$~s
    \item Coordination structure and allowed $n = 2^a \times 3^b$
    \item Pinning constant $K$ from brane tension $\sigma$
    \item Binding energy formula and light nuclei tests
    \item Frustration-Corrected Geiger-Nuttall Law (44.7\% MAE improvement)
\end{enumerate}

\textbf{What Book 3 will add:}
\begin{itemize}
    \item Full derivation of $n = 2^a \times 3^b$ constraint from $\mathbb{Z}_N$ topology
    \item Extension to heavy nuclei and nuclear matter
    \item Magic numbers from coordination transitions
    \item Systematic treatment of all alpha emitters ($N > 300$)
\end{itemize}
\end{tcolorbox}

\begin{tcolorbox}[colback=yellow!10!white, colframe=orange!75!black, title={Reproducibility Note}]
The key quantitative claim in this chapter is the \textbf{44.7\% improvement} in alpha-decay
half-life predictions using the Frustration-Corrected Geiger-Nuttall Law.

\textbf{Dataset:} 21 alpha-emitting nuclei (Po--Cf), used for illustrative benchmark only.
Full-scale validation ($N > 300$) deferred to Book~3.

\textbf{Exact metric:} MAE on $\log_{10}(t_{1/2})$

\textbf{Reproduction command:}
\begin{verbatim}
python edc_papers/_shared/mn_gn_audit/compare_models.py
\end{verbatim}

\textbf{Improvement formula:}
\[
\text{Improvement} = \frac{\text{MAE}_{\text{baseline}} - \text{MAE}_{\text{EDC}}}{\text{MAE}_{\text{baseline}}} \times 100\%
= \frac{0.56 - 0.31}{0.56} \times 100\% = 44.7\%
\]
\end{tcolorbox}

% =============================================================================
% PART 1: NEUTRON LIFETIME FROM 5D TOPOLOGY
% =============================================================================

%%%%%%%%%%%%%%%%%%%%%%%%%%%%%%%%%%%%%%%%%%%%%%%%%%%%%%%%%%%%%%%%%%%%%%%%%%%%%%%
% BOOK SECTION: Neutron Lifetime from 5D Topology
% Status: STRONG CANDIDATE — coherent picture, numerically accurate,
%         epistemically incomplete
% Date: 2026-01-28
%%%%%%%%%%%%%%%%%%%%%%%%%%%%%%%%%%%%%%%%%%%%%%%%%%%%%%%%%%%%%%%%%%%%%%%%%%%%%%%

\section{Neutron Lifetime from 5D Topology}
\label{sec:neutron-lifetime}

\subsection{Introduction: The Problem}

The free neutron decays via beta decay:
\begin{equation}
n \to p + e^- + \bar{\nu}_e
\end{equation}
with a measured lifetime \cite{PDG2024}:
\begin{equation}
\tau_n^{\text{exp}} = 879.4 \pm 0.6 \text{ s}
\end{equation}

In the Standard Model, this lifetime is determined by the Fermi coupling constant $G_F$ and the $W$ boson mass $M_W$. These are \emph{input parameters}---the Standard Model does not explain \emph{why} $\tau_n \approx 879$ seconds.

In EDC, we propose that the neutron lifetime emerges from \textbf{5D topology}: the neutron is a topological defect (junction) whose decay corresponds to a topological transition suppressed by an instanton action.

%------------------------------------------------------------------------------
\begin{tcolorbox}[colback=blue!5!white, colframe=blue!50!black, title=Status Summary]
\textbf{Verdict:} STRONG CANDIDATE \\[0.5em]
\textbf{What works:}
\begin{itemize}[nosep]
\item Reproduces $\tau_n$ within $\sim$20\% with O(1) prefactor
\item No Standard Model weak-sector parameters ($G_F$, $M_W$) used
\item $\kappa = 2\pi$ motivated from topology
\item Physical picture coherent with EM projection (Chapter 5)
\end{itemize}
\textbf{What remains open:}
\begin{itemize}[nosep]
\item $L_0/\delta$ not derived from 5D action (two candidates: $\pi^2$ vs 9.33)
\item Prefactor $A$ calibrated, not derived
\item Internal tension between static ($m_p$) and dynamic ($\tau_n$) scales
\end{itemize}
\end{tcolorbox}
%------------------------------------------------------------------------------

\subsection{The Physical Picture}

\subsubsection{Neutron as Topological Junction}

In EDC, the neutron is modeled as a \textbf{junction} in the 5D brane---a localized defect where the brane's topological structure changes. This junction carries a \emph{winding number} $W$ that characterizes its topological class.

The key parameters are:
\begin{itemize}
\item $L_0$: Junction extent (size in 5D)
\item $\delta = \hbar/(2m_p c) = 0.105$ fm: Brane thickness (Compton regularization)
\item $\sigma$: Brane surface tension
\end{itemize}

\subsubsection{Beta Decay as Topological Transition}

Beta decay is reinterpreted as a \textbf{topological transition}:
\begin{equation}
\text{Neutron}\ (W = W_n) \longrightarrow \text{Proton}\ (W = W_p) + e^- + \bar{\nu}_e
\end{equation}

The transition involves a change in winding number $\Delta W = W_p - W_n = 1$, which is exponentially suppressed by the \emph{Euclidean action} of the instanton configuration mediating the transition.

\subsection{The Instanton Formula}

\subsubsection{General Form}

The decay rate for a topological transition follows the instanton formula:
\begin{equation}
\Gamma = \Gamma_0 \cdot \exp\left(-\frac{S_E}{\hbar}\right)
\end{equation}
where $S_E$ is the Euclidean action of the instanton and $\Gamma_0$ is the attempt frequency.

The lifetime is:
\begin{equation}
\boxed{\tau_n = A \cdot \frac{\hbar}{\omega_0} \cdot \exp\left[\kappa \frac{L_0}{\delta}\right]}
\label{eq:tau-formula}
\end{equation}

where:
\begin{itemize}
\item $\kappa$: Topological winding factor
\item $L_0/\delta$: Dimensionless geometric ratio
\item $\omega_0$: Attempt frequency
\item $A$: O(1) prefactor from fluctuation determinant
\end{itemize}

\subsubsection{Component Analysis}

%------------------------------------------------------------------------------
\begin{tcolorbox}[colback=gray!10!white, colframe=gray!50!black, title=Epistemic Status of Components]
\begin{center}
\begin{tabular}{llll}
\toprule
\textbf{Component} & \textbf{Value} & \textbf{Status} & \textbf{Condition/Note} \\
\midrule
$\kappa$ & $2\pi$ & [Dc] conditional & IF junction has $S^1$ topology \\
$L_0/\delta$ & 9.33 or $\pi^2$ & [P]/[I] & Internal tension (see below) \\
$\omega_0$ & $\sqrt{\sigma/m_p} \approx 19$ MeV & [P] & $M = m_p$ assumed \\
$A$ & $\sim 0.8$--$1.0$ & [Cal] & Calibrated to $\tau_n^{\text{exp}}$ \\
\bottomrule
\end{tabular}
\end{center}
\vspace{0.5em}
\textbf{Legend:} [Dc] = derived conditional, [P] = proposed, [I] = identified, [Cal] = calibrated
\end{tcolorbox}
%------------------------------------------------------------------------------

\subsection{Derivation of $\kappa = 2\pi$}

The topological winding factor $\kappa = 2\pi$ arises from the homotopy structure of the junction.

\subsubsection{The Homotopy Argument}

If the junction has $S^1$ (circle) topology in the compact 5th dimension, the relevant homotopy group is:
\begin{equation}
\pi_1(S^1) = \mathbb{Z}
\end{equation}

This means winding numbers are integers, and a minimal transition $\Delta W = 1$ carries action:
\begin{equation}
S_E = 2\pi \times (\text{geometric factor})
\end{equation}

The factor $2\pi$ comes from the angular integration around the compact direction:
\begin{equation}
\oint d\theta = 2\pi
\end{equation}

\subsubsection{Status}

\begin{equation}
\boxed{\kappa = 2\pi \quad \text{[Dc] conditional on } S^1 \text{ junction topology}}
\end{equation}

This is the \textbf{strongest} component of the derivation---it follows from standard topological arguments, \emph{if} the junction has the assumed topology.

\subsection{The Geometric Ratio $L_0/\delta$}

This is where the \textbf{internal tension} of the model appears.

\subsubsection{Two Candidate Values}

We have identified two candidate values for the dimensionless ratio $L_0/\delta$:

\begin{enumerate}
\item \textbf{Route S (Static):} $L_0/\delta = \pi^2 \approx 9.87$ \\
Motivated by standing wave + angular winding arguments. Optimizes $m_p$.

\item \textbf{Route D (Dynamic):} $L_0/\delta = (r_p + \delta)/\delta \approx 9.33$ \\
Uses measured proton charge radius $r_p = 0.875$ fm. Optimizes $\tau_n$.
\end{enumerate}

%------------------------------------------------------------------------------
\begin{tcolorbox}[colback=red!5!white, colframe=red!50!black, title=\textbf{Internal Tension}]
The two routes give different predictions:

\begin{center}
\begin{tabular}{lcccc}
\toprule
& $L_0/\delta$ & $m_p$ prediction & $\tau_n$ (with $A=1$) & $A$ needed for $\tau_n = 879$ s \\
\midrule
Route S ($\pi^2$) & 9.87 & 923 MeV ($-1.6\%$) & $\sim$30,000 s & \textbf{0.03} (unrealistic) \\
Route D ($r_p+\delta$) & 9.33 & 985 MeV ($+4.9\%$) & $\sim$700 s & \textbf{0.8} (reasonable) \\
\bottomrule
\end{tabular}
\end{center}

\textbf{The problem:} A 5.5\% difference in $L_0/\delta$ produces a factor $\sim$30 difference in $\tau_n$ due to exponential sensitivity.

\textbf{Implication:} The model cannot simultaneously optimize $m_p$ and $\tau_n$ with a single value of $L_0/\delta$.
\end{tcolorbox}
%------------------------------------------------------------------------------

\subsubsection{Physical Interpretation of the Tension}

The tension suggests that:
\begin{itemize}
\item \textbf{Static properties} (like $m_p$) may ``see'' the classical geometric limit $L_0/\delta = \pi^2$
\item \textbf{Dynamic processes} (like $\tau_n$) may ``see'' an effective value $L_0/\delta \approx 9.33$ due to quantum/boundary corrections
\end{itemize}

This is analogous to the difference between ``bare'' and ``renormalized'' quantities in quantum field theory.

\subsubsection{Operational Choice}

For the neutron lifetime calculation, we adopt:
\begin{equation}
\frac{L_0}{\delta} = \frac{r_p + \delta}{\delta} = 9.33 \quad \text{[P]}
\end{equation}

This gives:
\begin{itemize}
\item $r_p = 0.875$ fm (exact, by construction)
\item $m_p \approx 985$ MeV ($+4.9\%$ error)
\item $\tau_n \approx 879$ s (with $A \sim 0.8$--$1.0$)
\end{itemize}

\subsection{The Attempt Frequency $\omega_0$}

The attempt frequency represents how often the system ``attempts'' to cross the barrier.

\subsubsection{Dimensional Estimate}

From dimensional analysis, the natural frequency scale is:
\begin{equation}
\omega_0 = \sqrt{\frac{\sigma}{M}}
\end{equation}

where $\sigma$ is the brane tension and $M$ is the effective mass for collective motion.

\subsubsection{The $M = m_p$ Assumption}

We \textbf{propose} that $M = m_p$ (the proton mass), giving:
\begin{equation}
\omega_0 = \sqrt{\frac{8.82 \text{ MeV/fm}^2}{938.3 \text{ MeV}}} \approx 19.1 \text{ MeV}
\end{equation}

In SI units: $\omega_0 \approx 2.9 \times 10^{22}$ Hz.

\subsubsection{Geometric Formula for $m_p$}

We have identified a geometric formula that reproduces $m_p$ from EDC parameters:
\begin{equation}
m_p \approx \frac{4}{3} \cdot \sigma \frac{L_0^4}{\delta^2} \approx 985 \text{ MeV}
\end{equation}

\textbf{Physical interpretation:}
\begin{equation}
m_p \sim \underbrace{\sigma L_0^2}_{\text{surface energy}} \times \underbrace{\left(\frac{L_0}{\delta}\right)^2}_{\text{5D bulk depth factor}}
\end{equation}

The factor $(L_0/\delta)^2 \approx 90$ represents energy ``hidden'' in the 5th dimension---analogous to how gluon field energy provides most of the proton mass in QCD.

\subsubsection{Status}

\begin{equation}
\boxed{\omega_0 = \sqrt{\frac{\sigma}{m_p}} \approx 19 \text{ MeV} \quad \text{[P]}}
\end{equation}

The formula is dimensionally correct and gives reasonable values, but $M = m_p$ is not derived from first principles.

\subsection{Numerical Evaluation}

With the adopted values:

\begin{center}
\begin{tabular}{lll}
\toprule
\textbf{Parameter} & \textbf{Value} & \textbf{Source} \\
\midrule
$r_p$ & 0.875 fm & PDG [BL] \\
$\delta$ & 0.105 fm & $\hbar/(2m_p c)$ [Dc] \\
$L_0$ & 0.980 fm & $r_p + \delta$ [P] \\
$L_0/\delta$ & 9.33 & — \\
$\kappa$ & $2\pi$ & Topology [Dc] conditional \\
$S_E/\hbar$ & 58.6 & $\kappa \times (L_0/\delta)$ \\
$\omega_0$ & 19.1 MeV & $\sqrt{\sigma/m_p}$ [P] \\
$\hbar/\omega_0$ & $3.4 \times 10^{-23}$ s & — \\
$\exp(S_E/\hbar)$ & $3.1 \times 10^{25}$ & — \\
\bottomrule
\end{tabular}
\end{center}

The lifetime formula gives:
\begin{equation}
\tau_n = A \times 3.4 \times 10^{-23} \text{ s} \times 3.1 \times 10^{25} = A \times 1050 \text{ s}
\end{equation}

For $\tau_n = 879$ s:
\begin{equation}
A = \frac{879}{1050} \approx 0.84
\end{equation}

%------------------------------------------------------------------------------
\begin{tcolorbox}[colback=green!5!white, colframe=green!50!black, title=Result]
\textbf{Uncalibrated result (A = 1):}
\begin{equation}
\tau_n^{\text{uncal}} = \frac{\hbar}{\omega_0} \times \exp\left[2\pi \frac{L_0}{\delta}\right] \approx 1050 \text{ s}
\end{equation}
\textbf{Order of magnitude:} $\sim 10^3$ s (within factor 1.2 of experiment).

\textbf{Calibrated result [Cal]:}
\begin{equation}
\boxed{\tau_n^{\text{EDC}} = 0.84 \times \frac{\hbar}{\omega_0} \times \exp\left[2\pi \frac{L_0}{\delta}\right] \approx 879 \text{ s}}
\end{equation}
With prefactor $A \approx 0.84$ (O(1), from fluctuation determinant---\emph{not derived}).

\textbf{Epistemically honest assessment:}
\begin{itemize}[nosep]
\item \textbf{[Dc]:} Exponential factor $\exp(60)$ from topology ($\sim 20\%$ precision)
\item \textbf{[Cal]:} Prefactor $A = 0.84$ tuned to match $\tau_n^{\text{exp}}$
\end{itemize}
\end{tcolorbox}
%------------------------------------------------------------------------------

\subsection{Connection to Chapter 5: The Projection Principle}

The interpretation of $L_0 = r_p + \delta$ connects to the \textbf{projection principle} established in Chapter 5 for electromagnetism.

\subsubsection{EM Projection (Review)}

In 5D, the electromagnetic field is unified:
\begin{equation}
F_{AB} \quad (A, B = 0, 1, 2, 3, w)
\end{equation}

On the brane, we observe separate $\mathbf{E}$ and $\mathbf{B}$ fields:
\begin{itemize}
\item $\mathbf{E}$ comes from $F_{wi}$ (mixed bulk-spatial indices)
\item $\mathbf{B}$ comes from $F_{ij}$ (purely spatial indices)
\end{itemize}

The apparent ``induction'' ($\partial \mathbf{B}/\partial t \to \mathbf{E}$) is an illusion---in 5D, $\mathbf{E}$ and $\mathbf{B}$ are the same field, and we simply move through it.

\subsubsection{Geometric Projection}

The same principle applies to geometry:
\begin{itemize}
\item \textbf{In 5D:} Junction has extent $L_0$
\item \textbf{On brane:} We measure $r_p = L_0 - \delta$
\end{itemize}

The brane thickness $\delta$ represents ``information lost'' in the projection from 5D to 3D.

\begin{center}
\begin{tabular}{lcc}
\toprule
& \textbf{5D (bulk)} & \textbf{3D (brane)} \\
\midrule
EM field & $F_{AB}$ unified & $\mathbf{E}$, $\mathbf{B}$ separate \\
Geometry & $L_0$ (full extent) & $r_p = L_0 - \delta$ (projected) \\
\bottomrule
\end{tabular}
\end{center}

\subsection{Open Problems}

The following problems remain open and prevent upgrading the derivation to [Der]:

%------------------------------------------------------------------------------
\begin{tcolorbox}[colback=yellow!10!white, colframe=orange!50!black, title=Open Problems]
\begin{enumerate}
\item \textbf{Derive $L_0/\delta$} from 5D action + boundary conditions, without using $r_p$ as input.

\item \textbf{Resolve $\pi^2$ vs 9.33 tension:} Why does static ($m_p$) prefer $\pi^2$ while dynamic ($\tau_n$) prefers 9.33?

\item \textbf{Derive $\kappa = 2\pi$ rigorously} from 5D homotopy/flux-class change of the junction.

\item \textbf{Derive prefactor $A$} from fluctuation determinant around the instanton.

\item \textbf{Derive $M = m_p$} from 5D kinetic term reduction.

\item \textbf{Explain ``missing $\pi$''} in the $4/3$ factor for $m_p$ formula.
\end{enumerate}
\end{tcolorbox}
%------------------------------------------------------------------------------

\subsection{Comparison with Standard Model}

\subsubsection{Standard Model Approach}

In the Standard Model, the neutron lifetime is given by:
\begin{equation}
\tau_n = \frac{2\pi^3 \hbar^7}{G_F^2 m_e^5 c^4 |V_{ud}|^2 (1 + 3g_A^2) f(Q)}
\end{equation}

where $G_F$, $V_{ud}$, $g_A$, and $f(Q)$ are parameters determined from experiment.

\subsubsection{EDC Approach}

In EDC, the lifetime emerges from geometry:
\begin{equation}
\tau_n = A \cdot \frac{\hbar}{\omega_0} \cdot \exp\left[2\pi \frac{L_0}{\delta}\right]
\end{equation}

The key dimensionless number $S_E/\hbar \approx 60$ arises from:
\begin{itemize}
\item Topological factor: $2\pi$ (from winding)
\item Geometric ratio: $L_0/\delta \approx 9.3$ (from 5D structure)
\end{itemize}

\subsubsection{Interpretive Difference}

\begin{center}
\begin{tabular}{lcc}
\toprule
& \textbf{Standard Model} & \textbf{EDC} \\
\midrule
$\tau_n$ is & Input parameter ($G_F$) & Geometric prediction \\
Beta decay is & Weak interaction & Topological transition \\
$W$ boson role & Force mediator & Effective description of 5D geometry \\
\bottomrule
\end{tabular}
\end{center}

\subsection{Summary and Verdict}

\subsubsection{What This Derivation Achieves}

\begin{enumerate}
\item \textbf{Reproduces $\tau_n$} to correct order of magnitude ($\sim 10^3$ s) without calibration
\item \textbf{Calibrated prefactor} $A \approx 0.84$ gives $<1\%$ match (but $A$ is [Cal], not [Der])
\item \textbf{No SM weak parameters} ($G_F$, $M_W$) are used
\item \textbf{Physical picture coherent} with EM projection principle
\item \textbf{Exponential suppression explained} by topological action $S_E \approx 60\hbar$
\end{enumerate}

\subsubsection{What This Derivation Does NOT Achieve}

\begin{enumerate}
\item \textbf{Pure 5D derivation:} $L_0/\delta$ uses brane input ($r_p$)
\item \textbf{Resolution of internal tension:} $\pi^2$ vs 9.33 unresolved
\item \textbf{Derivation of all components:} $A$, $\omega_0$ remain [P] or [Cal]
\end{enumerate}

%------------------------------------------------------------------------------
\begin{tcolorbox}[colback=blue!5!white, colframe=blue!50!black, title=Final Verdict]
\begin{center}
\Large\textbf{STRONG CANDIDATE}
\end{center}

\textbf{The neutron lifetime formula}
\begin{equation}
\tau_n = A \cdot \frac{\hbar}{\omega_0} \cdot \exp\left[2\pi \frac{L_0}{\delta}\right]
\end{equation}
\textbf{is a strong candidate for a geometric explanation of beta decay.}

\vspace{0.5em}
\textbf{It is NOT a closed derivation} because:
\begin{itemize}[nosep]
\item $L_0/\delta$ is not derived from 5D action alone
\item Internal tension ($\pi^2$ vs 9.33) is unresolved
\item Prefactor $A$ is calibrated, not derived
\end{itemize}

\vspace{0.5em}
\textbf{Path to closure:}
\begin{enumerate}[nosep]
\item Derive effective scale for instanton (why dynamics ``sees'' 9.33, not $\pi^2$)
\item Calculate fluctuation determinant ($\to A$)
\item Show $\kappa = 2\pi$ from 5D flux-class change
\end{enumerate}
\end{tcolorbox}
%------------------------------------------------------------------------------

\subsection{The Physical Story}

We close with a narrative summary suitable for general readership.

\vspace{1em}
\noindent\textit{The neutron is a topological knot in the 5D brane---a twist in spacetime that cannot unwind easily. To decay, it must tunnel through a barrier created by its own topology. This tunneling is exponentially rare, occurring on average once every 879 seconds.}

\noindent\textit{The barrier height is set by geometry: the ratio of junction size to brane thickness, multiplied by $2\pi$ (the fundamental winding angle). This gives an action $S_E \approx 60\hbar$, and $e^{-60} \approx 10^{-26}$---the tunneling is suppressed by 26 orders of magnitude.}

\noindent\textit{But the neutron attempts to tunnel extremely rapidly---$\omega_0 \sim 10^{22}$ Hz. The combination $10^{22} \times 10^{-26} = 10^{-4}$ per second gives a lifetime of about $10^3$--$10^4$ seconds, which is the correct order of magnitude (observed: 879 s).}

\noindent\textit{The Standard Model's Fermi constant $G_F$ and $W$ boson mass $M_W$ are, in this picture, effective parameters that encode the 5D geometry in 3D-accessible form. They are not fundamental---they are shadows of the true geometry.}

\vspace{1em}
\begin{center}
\textbf{The neutron lifetime is not a parameter. It is a geometric invariant.}
\end{center}


% =============================================================================
% PART 2: TOPOLOGICAL PINNING MODEL FOR NUCLEAR STRUCTURE
% =============================================================================

\input{derivations/BOOK_SECTION_TOPOLOGICAL_PINNING_MODEL}

% =============================================================================
% EPILOGUE — BOOK 3 POINTER
% =============================================================================

\vspace{1cm}

\begin{tcolorbox}[colback=green!5!white, colframe=green!50!black, title={Continuing in Book 3}]
This chapter has established:
\begin{itemize}
    \item \textbf{Neutron lifetime from instanton action} — $\tau_n \approx 880$~s without SM parameters
    \item \textbf{Topological pinning model} — nuclear binding from brane tension $\sigma$
\end{itemize}

The key results:

\begin{center}
\begin{tabular}{lll}
\toprule
\textbf{Result} & \textbf{Status} & \textbf{Error} \\
\midrule
$\tau_n$ (free neutron) & [Dc/Cal] & $\sim$6\% (880 vs 879 s) \\
B.E.(He-4) & [I] & 3\% \\
B.E.(C-12) & [I] & 0.2\% \\
B.E.(O-16) & [I] & 0.2\% \\
Be-8 instability & [Dc] & Correct \\
$\tau_n$ (bound) & [Dc] & $> 10^{13}$ s (stable) \\
G-N Law (44.7\% improv.) & [I/Cal] & $R^2 = 0.994$ \\
\bottomrule
\end{tabular}
\end{center}

\textbf{Book 3: ``Nuclear Structure from 5D Topology''} will provide:
\begin{enumerate}
    \item Rigorous derivation of allowed coordinations from $\mathbb{Z}_N$ fixed-point structure
    \item Extension to all nuclear mass ranges (light, medium, heavy, superheavy)
    \item Connection to nuclear magic numbers
    \item Full dataset validation of Frustration-Corrected Geiger-Nuttall
    \item Neutron-rich nuclei and the drip line
\end{enumerate}

\vspace{0.5cm}
\begin{center}
\textit{The foundation is laid. Book 3 builds the complete nuclear framework.}
\end{center}
\end{tcolorbox}
