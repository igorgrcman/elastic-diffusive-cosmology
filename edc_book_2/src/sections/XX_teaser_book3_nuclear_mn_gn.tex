% =============================================================================
% XX_teaser_book3_nuclear_mn_gn.tex
%
% Teaser Chapter: Nuclear Applications Preview
%
% This chapter provides a preview of Book 3 results, specifically the
% Frustration-Corrected Geiger-Nuttall Law for alpha-decay half-lives.
%
% Status: [I/Cal] - Pattern identified, coefficients calibrated
% Author: EDC Research
% Date: 2026-01-29
% =============================================================================

\chapter{Preview: Nuclear Applications (Book 3)}
\label{ch:teaser-nuclear}

\begin{tcolorbox}[colback=blue!5!white, colframe=blue!75!black, title=Chapter Status]
This chapter is a \textbf{teaser} for the full nuclear framework developed in Book~3.
It presents one key result---the Frustration-Corrected Geiger-Nuttall Law---to
demonstrate the predictive power of the EDC topological model.
\end{tcolorbox}

% =============================================================================
\section{Motivation: From Weak Sector to Nuclear Structure}
\label{sec:nuclear-motivation}

The weak sector developed in this book (Part II) establishes that particle masses
and mixing matrices emerge from the geometric structure of the 5D bulk.
Book~3 extends this framework to nuclear structure, addressing:

\begin{itemize}
    \item Binding energies of light nuclei
    \item Nuclear stability and magic numbers
    \item Alpha-decay half-lives
    \item Neutron lifetime in bound vs.\ free states
\end{itemize}

Here we preview one striking result: a \textbf{44.7\% improvement} in predicting
alpha-decay half-lives using a \emph{frustration-corrected} Geiger-Nuttall law.

% =============================================================================
\section{Topological Frustration in Nuclear Matter}
\label{sec:frustration-concept}

\subsection{The Key Insight: $n = 43$ is Forbidden}

The EDC topological pinning model (Book 3, Chapter 3) predicts that only certain
coordination numbers are \emph{allowed} for stable nuclear configurations:
\begin{equation}
    n_{\text{allowed}} = 2^a \times 3^b \qquad (a, b \geq 0)
    \label{eq:allowed-coordinations}
\end{equation}

This gives: $n = 1, 2, 3, 4, 6, 8, 9, 12, 16, 18, 24, 27, 32, 36, \ldots$

Numerical optimization of nuclear matter (Fermi gas with pinning corrections)
predicts an \emph{optimal} coordination of approximately $n \approx 43$.

\textbf{But 43 is a prime number $> 3$, making it topologically forbidden.}

This creates \emph{geometric frustration}: heavy nuclei cannot achieve their
energetically optimal configuration, leading to stress that drives alpha decay.

% =============================================================================
\section{The Frustration-Corrected Geiger-Nuttall Law}
\label{sec:frustration-gn}

\begin{tcolorbox}[colback=green!5!white, colframe=green!75!black,
                  title={Frustration-Corrected Geiger-Nuttall Law [I/Cal]}]
\begin{equation}
    \log_{10}(t_{1/2}) = a \cdot \frac{Z}{\sqrt{Q_\alpha}} + c \cdot \varepsilon_f + b
    \label{eq:frustration-gn}
\end{equation}
where:
\begin{itemize}
    \item $Z$ = atomic number of daughter nucleus
    \item $Q_\alpha$ = alpha-decay energy (MeV)
    \item $\varepsilon_f(A)$ = frustration energy per nucleon (MeV)
\end{itemize}

\textbf{Fitted parameters:}
\begin{align*}
    a &= 1.63 && \text{(Geiger-Nuttall coefficient)} \\
    c &= -2.40 && \text{(frustration coefficient; negative = faster decay)} \\
    b &= -42.1 && \text{(intercept)}
\end{align*}
\end{tcolorbox}

\subsection{The Frustration Energy}

The frustration energy interpolates between the alpha-cluster regime ($n = 6$)
and the bulk limit ($n \to 43$):
\begin{equation}
    n_{\text{eff}}(A) = 6 + 37\left(1 - e^{-(A-20)/80}\right)
    \label{eq:n-eff-interpolation}
\end{equation}

The frustration is the energy mismatch between the effective coordination and
the nearest allowed coordination:
\begin{equation}
    \varepsilon_f(A) = \left| E/A(n_{\text{eff}}) - E/A(n_{\text{allowed}}) \right|
    \label{eq:frustration-energy}
\end{equation}

% =============================================================================
\section{Result: 44.7\% Improvement Over Standard Geiger-Nuttall}
\label{sec:gn-result}

\begin{tcolorbox}[colback=yellow!10!white, colframe=orange!75!black,
                  title={Result with Full Provenance}]
\textbf{Dataset:} 21 alpha-emitting nuclei (Po to Cf), mass numbers 210--252

\textbf{Metric:} Mean Absolute Error on $\log_{10}(t_{1/2})$

\textbf{Comparison:}
\begin{center}
\begin{tabular}{lcc}
\toprule
Model & MAE & $R^2$ \\
\midrule
Standard Geiger-Nuttall & 0.56 & 0.982 \\
Frustration-Corrected   & 0.31 & 0.994 \\
\midrule
\textbf{Improvement}    & \textbf{44.7\%} & +1.2 pp \\
\bottomrule
\end{tabular}
\end{center}

\textbf{Formula:} $\text{Improvement} = \dfrac{\text{MAE}_{\text{baseline}} - \text{MAE}_{\text{EDC}}}{\text{MAE}_{\text{baseline}}} \times 100\%$

\textbf{Baseline:} Standard Geiger-Nuttall law, refitted to same 21 nuclei.

\textbf{Reproducibility:} \\
\texttt{python src/derivations/frustration\_geiger\_nuttall.py}
\end{tcolorbox}

\begin{tcolorbox}[colback=red!5!white, colframe=red!50!black, title={Important Caveat}]
\textbf{Small curated dataset (N=21)} used only for illustrative benchmark.
Full-scale validation across all known alpha emitters ($N > 300$) is deferred to Book~3,
where the frustration mechanism and topological derivation of $\mathbb{M}_n$ coordinations
are presented in full.

This teaser demonstrates the \emph{form} of the frustration-corrected law and the
\emph{direction} of improvement; quantitative claims about predictive accuracy
require the full dataset analysis in Book~3.
\end{tcolorbox}

\subsection{Physical Interpretation}

The negative frustration coefficient ($c = -2.40$) confirms the physical picture:
\begin{itemize}
    \item Higher frustration energy $\varepsilon_f$ reduces the predicted half-life
    \item Heavy nuclei with $A > 200$ have significant frustration
    \item The frustration ``assists'' alpha emission by destabilizing the nucleus
\end{itemize}

% =============================================================================
\section{Epistemic Status}
\label{sec:epistemic-status}

\begin{tcolorbox}[colback=gray!10!white, colframe=gray!75!black, title={Epistemic Tags}]
\begin{itemize}
    \item[\textbf{[I]}] \textbf{Identified:} Correlation between frustration energy and half-life improvement
    \item[\textbf{[Cal]}] \textbf{Calibrated:} Parameters $a$, $c$, $b$ fitted to 21 nuclei
    \item[\textbf{[P]}] \textbf{Proposed:} The $n_{\text{eff}}(A)$ interpolation is phenomenological
\end{itemize}

\textbf{What is derived (Book 3):}
\begin{itemize}
    \item Allowed coordinations $n = 2^a \times 3^b$ from $\mathbb{Z}_N$ topology [Dc]
    \item Optimal $n \approx 43$ from Fermi gas + pinning energy minimization [Dc]
\end{itemize}

\textbf{What is calibrated:}
\begin{itemize}
    \item Coefficients $a$, $c$, $b$ from least-squares fit [Cal]
    \item Interpolation scale $(A-20)/80$ is phenomenological [P]
\end{itemize}
\end{tcolorbox}

% =============================================================================
\section{What Book 3 Will Cover}
\label{sec:book3-preview}

The full nuclear framework in Book~3 includes:

\begin{enumerate}
    \item \textbf{Topological pinning model} with derivation of $n = 2^a \times 3^b$ constraint
    \item \textbf{Light nuclei binding energies:} He-4, C-12, O-16 within 3\%
    \item \textbf{Be-8 instability:} Correctly predicted from coordination mismatch
    \item \textbf{Nuclear magic numbers:} Connection to allowed coordinations
    \item \textbf{Neutron lifetime:} Why bound neutrons are stable (topological suppression)
    \item \textbf{Alpha decay systematics:} Full derivation of frustration correction
\end{enumerate}

\vspace{1cm}

\begin{center}
\fbox{\parbox{0.9\textwidth}{
\textbf{Why is this a teaser?}

This chapter presents only the \emph{benchmark result} (44.7\% MAE improvement)
and the \emph{functional form} of the frustration-corrected Geiger-Nuttall law.

The underlying \textbf{topological mechanism}---why coordination number $\mathbb{M}_n = 43$
is forbidden (prime $> 3$) and how this generates geometric frustration---requires
the full development of the $\mathbb{Z}_N$ pinning model.

\medskip
\textit{Full treatment in Book 3: ``Nuclear Structure from 5D Topology.''}
}}
\end{center}
