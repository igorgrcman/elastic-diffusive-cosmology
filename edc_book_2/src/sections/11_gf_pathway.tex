% ==============================================================================
% Section 1.10: Structural Pathway to G_F (Overview)
% Full treatment in Chapter: The Fermi Constant from Geometry
% ==============================================================================

\section{\texorpdfstring{Structural Pathway to $G_F$ (Overview)}{Structural Pathway to GF (Overview)}}
\label{sec:gf_pathway}

This section provides a brief overview of how the effective coupling strength
emerges in EDC. For the complete treatment including numerical derivation and
mode overlap analysis, see Chapter~\ref{ch:gf_derivation}.

\paragraph{The central question.}
The Fermi constant $G_F = 1.17 \times 10^{-5}$ GeV$^{-2}$ \tagBL{} sets the scale
of weak interactions. Why is this value so small? Why is the weak force ``weak''?

\paragraph{EDC answer.}
In EDC, weak interactions are not fundamental gauge vertices but effective contact
terms arising from integrating out a brane-layer mediator \tagDc{}:
\begin{equation}
G_{\text{EDC}} \sim \frac{g_{\text{eff}}^2}{m_\phi^2}
\label{eq:gf_overview}
\end{equation}

The smallness reflects geometric suppression:
\begin{itemize}[nosep]
    \item Mediator mass gap $m_\phi$ (from brane geometry)
    \item Mode overlap suppression (fermion localization)
    \item Chirality selection (V$-$A structure)
\end{itemize}

\paragraph{Numerical closure.}
EDC achieves exact numerical agreement for $G_F$ through electroweak relations
once $\sin^2\theta_W = 1/4$ is derived from $\mathbb{Z}_6$ geometry. See
Chapter~\ref{ch:gf_derivation} for the complete derivation chain.

\begin{tcolorbox}[colback=blue!5, colframe=blue!50!black,
    title=\textbf{Forward Reference}]
The structural pathway and numerical derivation are consolidated in
\textbf{Chapter~\ref{ch:gf_derivation}: The Fermi Constant from Geometry}.
That chapter provides:
\begin{itemize}[nosep]
    \item Complete derivation: $G_F$ exact from electroweak relations
    \item Mode overlap mechanism: why weak is ``weak''
    \item Connection to V$-$A structure (Chapter~\ref{ch:va_structure})
    \item Honest assessment of what remains open
\end{itemize}
\end{tcolorbox}

