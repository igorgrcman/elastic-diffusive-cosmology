% ==============================================================================
% Section: Neutron Dual-Route Proof (Route A + Route B + Convergence)
% Status: Effective 1D model; full 5D derivation OPEN
% ==============================================================================

\subsection{Neutron Dual-Route: Metastability and Lifetime}
\label{sec:neutron_dual_route}

This section provides two independent derivation routes for the neutron's metastable
character and lifetime, analogous to the proton's dual-route proof (\S\ref{sec:proton_routeA_anchor},
\S\ref{sec:routeB_z6_to_steiner}). Both routes converge on the same physical picture:
the neutron is a metastable excitation above the proton anchor.

\begin{tcolorbox}[colback=yellow!5!white, colframe=yellow!50!black,
                  title={\textbf{5D Forensic Audit Statement}}]
\textbf{Current status:} All quantitative neutron results ($\tau_n$, $V_B$, $q_n$) in this
chapter use the \textbf{effective 1D WKB/bridge model}---a 5D-\emph{induced} but not
5D-\emph{derived} framework.

\textbf{What is OPEN:}
\begin{itemize}[nosep]
  \item Derive $S_{\mathrm{eff}}[q]$ from full 5D action (bulk + brane + GHY + Israel)
  \item Derive effective mass $M(q)$ and potential $V(q)$ from 5D geometry
  \item Derive barrier height $V_B$ from first principles (currently [Cal])
  \item Connect $\Gamma_0$ prefactor to 5D mode spectrum
\end{itemize}

\textbf{What this section does:} Establishes the \emph{structural} picture (Route A) and
the \emph{effective lifetime} (Route B) with honest epistemic tags.
\end{tcolorbox}

% ==============================================================================
% ROUTE A: 5D Structural / Metastability Statement
% ==============================================================================

\subsubsection{Neutron Metastable Bridge --- Route A (5D Structural Statement)}
\label{subsec:neutron_routeA_structural}

Route A establishes the neutron as a metastable configuration in the same topological
sector as the proton, without importing any external microscopic decay model.

\paragraph{Epistemic status for Route A.}
\begin{itemize}[nosep]
  \item \textbf{[Der]} Proton anchor is local minimum (Corollary~\ref{cor:proton_minimum})
  \item \textbf{[M]+[P]} Topological sector preserved during $\tau_{\mathrm{obs}}$
  \item \textbf{[Dc]} Neutron has $E(q_n) > E(0)$ (conditional on potential structure)
  \item \textbf{[P]} Specific form of $V(q)$ near $q_n$ (not derived from 5D action)
\end{itemize}

\paragraph{Configuration picture.}

The proton Y-junction is established as a local energy minimum within its topological
sector (Corollary~\ref{cor:proton_minimum}, \S\ref{sec:proton_routeA_anchor}). The neutron
is modeled as a configuration in the \emph{same topological class} but displaced from
the Steiner minimum:
\begin{equation}
\label{eq:neutron_config_space}
\Psi_n \in \mathcal{C}_Y, \quad q(\Psi_n) = q_n > 0, \quad q(\Psi_p) = 0.
\end{equation}

The collective coordinate $q$ parametrizes the deviation from the proton anchor
(consistent with \S\ref{sec:case_neutron}).
Geometrically, $q$ measures the displacement of the Y-junction node into the bulk
relative to the brane-anchored ring (the ring is the intersection of flux sheets
with the brane surface). At $q = 0$, the junction is Steiner-balanced: the proton anchor.
At $q > 0$, the junction is lifted, incurring extra worldsheet length and hence extra energy.

% --- Lemma A1 ---
\begin{edcLemmaBox}{Topological Sector Preservation}{[M]+[P]}
\label{lem:A1_topo_sector}
During the observational timescale $\tau_{\mathrm{obs}} \lesssim \tau_n \approx 880\,\mathrm{s}$,
the neutron configuration remains in the Y-junction topological sector $\mathcal{C}_Y$.
No topology-changing processes (flux tube reconnection, pair creation from vacuum)
occur at the relevant energy scales.

\textbf{Consequence:} The neutron can only evolve \emph{within} $\mathcal{C}_Y$, meaning
relaxation toward the proton anchor ($q \to 0$) is the only available decay channel
at low energies.
\end{edcLemmaBox}

\begin{proof}[Justification]
Topology-changing processes require energy $E_{\mathrm{top}} \gtrsim \sigma \cdot a^2$
to create/annihilate flux tube segments, where $\sigma$ is membrane tension and $a$ is
core size. We assume $E_{\mathrm{top}}$ is parametrically larger than
$\Delta m_{np} c^2 \approx 1.3\,\mathrm{MeV}$ [P]. Hence the topological sector is effectively frozen during neutron decay.
This is a physical argument [P] based on scale separation, not a rigorous derivation. \hfill $\square$
\end{proof}

% --- Lemma A2 ---
\begin{edcLemmaBox}{Energy Functional in Thin-String Limit}{[Der]+[P]}
\label{lem:A2_energy_functional}
In the frozen (thin-string) limit, the energy functional for a Y-junction configuration
with collective coordinate $q$ takes the form:
\begin{equation}
\label{eq:energy_functional_q}
E[q] = E_0 + V(q), \quad V(0) = 0, \quad V''(0) > 0,
\end{equation}
where $E_0$ is the proton ground-state energy and $V(q)$ is the excitation potential.

\textbf{Derived [Der]:} The dominant contribution $E \propto \tau L$ follows from
Nambu--Goto (Lemma~\ref{lem:nambu_goto}).

\textbf{Assumed [P]:} The specific shape of $V(q)$ near $q_n$, including the existence
of a local maximum (barrier) between $q_n$ and $q = 0$.
\end{edcLemmaBox}

% --- Proposition A3 ---
\begin{edcPropositionBox}{Neutron as Metastable Local Minimum}{[Dc]+[P]}
\label{prop:A3_metastable}
If the potential $V(q)$ has a local minimum at $q_n > 0$ separated from the global
minimum at $q = 0$ by a barrier of height $V_B > 0$, then:
\begin{enumerate}[nosep]
  \item The neutron configuration $\Psi_n$ is a \emph{metastable} state
  \item The neutron can decay to the proton via barrier penetration (quantum tunneling)
        or thermal activation
  \item The decay does not require topology change---only relaxation within $\mathcal{C}_Y$
\end{enumerate}

\textbf{Status:} [Dc] conditional on the existence of such a barrier; the barrier itself
is currently [P] (postulated based on physical reasoning, not derived from 5D action).
\end{edcPropositionBox}

% --- Remark A (BC clarification) ---
\begin{tcolorbox}[colback=yellow!5!white, colframe=yellow!50!black,
                  title={\textbf{Remark A: What Route A Does NOT Claim}}]
\label{rem:routeA_bc_disclaimer}
\begin{itemize}[nosep]
  \item We do \textbf{not} claim that boundary conditions create attraction or generate
        the barrier $V_B$.
  \item BC (Neumann, Robin, Dirichlet) provide the \emph{scale} $\delta$ (brane thickness) and affect
        the \emph{mode spectrum}---they do not determine the sign of $V'(q)$.
  \item The barrier mechanism is \textbf{topological/geometric} (core repulsion + logarithmic
        growth at large separation---EDC geometric containment), consistent with the proton
        anchor derivation.
  \item The linearized BC analysis gives $V'_{\mathrm{lin}}(d) > 0$ for all BC choices
        (see \texttt{aside\_frozen\_brane\_bc\_v1/}).
\end{itemize}
\end{tcolorbox}

% --- Route A Status Map ---
\paragraph{Route A status map.}

\begin{center}
\begin{tabular}{llll}
\toprule
\textbf{Step} & \textbf{Claim} & \textbf{Tag} & \textbf{Source} \\
\midrule
A0 & Proton is local minimum at $q=0$ & [Dc] & Cor.~\ref{cor:proton_minimum} \\
A1 & Topological sector preserved & [M]+[P] & Lemma~\ref{lem:A1_topo_sector} \\
A2 & $E[q] = E_0 + V(q)$ structure & [Der]+[P] & Lemma~\ref{lem:A2_energy_functional} \\
A3 & Neutron is metastable at $q_n > 0$ & [Dc]+[P] & Prop.~\ref{prop:A3_metastable} \\
\midrule
--- & Specific $V(q)$ shape & [P] & Not derived \\
--- & Barrier height $V_B$ & [P]/[Cal] & \S\ref{subsec:neutron_routeB_wkb} \\
--- & BC do not create barrier & [Der] & aside\_frozen\_brane\_bc\_v1 \\
\bottomrule
\end{tabular}
\end{center}

% ==============================================================================
% ROUTE B: Effective 1D WKB/Bridge Lifetime
% ==============================================================================

\subsubsection{Neutron Lifetime --- Route B (Effective 1D WKB/Bridge)}
\label{subsec:neutron_routeB_wkb}

Route B provides the quantitative lifetime calculation using an effective 1D model.
This is \emph{not} the full 5D derivation---it is a 5D-induced effective description.

\begin{tcolorbox}[colback=red!5!white, colframe=red!50!black,
                  title={\textbf{Forensic Audit Declaration}}]
This section uses the \textbf{effective 1D WKB/bridge model}. The connection to full 5D
dynamics (derive $M(q)$, $V(q)$, $V_B$ from $S_{5D}$) is \textbf{OPEN}.

Current numbers are calibrated [Cal] to reproduce $\tau_n \approx 879\,\mathrm{s}$,
not derived [Der] from first principles.
\end{tcolorbox}

\paragraph{Scope and interpretive stance.}
EDC is \emph{not} presented as a replacement for established 3D effective descriptions.
Instead, EDC is a 5D geometric \emph{why}-framework: it aims to account for \emph{why}
stable 3D structures and timescales exist by identifying the underlying topological and
energetic constraints in the thick-brane/bulk setting.
In this section we therefore use 3D-measured quantities only as \textbf{baseline benchmarks}
(e.g.\ the empirical timescale $\tau_n^{\mathrm{BL}}$), while the derivation focuses on
the EDC mechanism: configuration-coordinate trapping and relaxation/tunneling in~$q$.

% --- Lemma B1 ---
\begin{edcLemmaBox}{Effective Action Ansatz}{[P]+[Dc]}
\label{lem:B1_effective_action}
The neutron's bulk-core dynamics are modeled by an effective 1D action:
\begin{equation}
\label{eq:effective_action}
S_{\mathrm{eff}}[q] = \int dt \left( \frac{1}{2} M(q) \dot{q}^2 - V(q) \right),
\end{equation}
where:
\begin{itemize}[nosep]
  \item $q(t)$ is the collective coordinate (junction displacement from Steiner)
  \item $M(q)$ is an effective mass (currently [P]---not derived from 5D)
  \item $V(q)$ is the effective potential with minimum at $q_n$ and barrier $V_B$
\end{itemize}

\textbf{Physical motivation [Dc]:} This form follows from standard reduction of
field theory to collective coordinates when a single slow mode dominates.

\textbf{Not derived [P]:} The specific functions $M(q)$, $V(q)$ from 5D geometry.
\end{edcLemmaBox}

% --- Lemma B2 ---
\begin{edcLemmaBox}{WKB Tunneling Rate}{[M]}
\label{lem:B2_wkb}
For a particle in a metastable potential well, the quantum tunneling decay rate is:
\begin{equation}
\label{eq:wkb_rate}
\Gamma = \Gamma_0 \exp\left( -\frac{2}{\hbar} \int_{q_n}^{q_{\mathrm{exit}}}
         \sqrt{2M(q)(V(q) - E_n)} \, dq \right),
\end{equation}
where $\Gamma_0$ is a prefactor depending on attempt frequency, and the integral
is over the classically forbidden region.

\textbf{Status:} This is standard quantum mechanics [M], not EDC-specific.
\end{edcLemmaBox}

% --- Proposition B3 ---
\begin{edcPropositionBox}{Calibrated Barrier Height}{[Cal]}
\label{prop:B3_calibration}
Using the effective 1D model with calibrated parameters:
\begin{equation}
\label{eq:calibration}
V_B \approx 2.6\,\mathrm{MeV}, \quad \Gamma_0 \sim \mathcal{O}(10^{15}\,\mathrm{s}^{-1}),
\end{equation}
the WKB formula reproduces the observed neutron lifetime:
\begin{equation}
\tau_n = \Gamma^{-1} \approx 879\,\mathrm{s} \quad \text{[BL]}.
\end{equation}

\textbf{Critical status:} This is \textbf{[Cal]} (calibration), not \textbf{[Der]} (derivation).
The values of $V_B$ and $\Gamma_0$ are chosen to fit $\tau_n$, not derived from 5D action.
\end{edcPropositionBox}

% --- Remark B (OPEN items) ---
\begin{tcolorbox}[colback=blue!5!white, colframe=blue!50!black,
                  title={\textbf{Remark B: What Is OPEN (5D Derivation Roadmap)}}]
\label{rem:routeB_open}
The following derivations are \textbf{not yet implemented}:
\begin{enumerate}[nosep]
  \item \textbf{Derive $V(q)$ from 5D action:} $S_{5D} = S_{\mathrm{bulk}} + S_{\mathrm{brane}}
        + S_{\mathrm{GHY}} \to V(q)$
  \item \textbf{Derive $M(q)$ from 5D geometry:} Identify the kinetic term's origin in
        bulk field fluctuations
  \item \textbf{Derive $V_B$ from first principles:} Currently [Cal]; target is [Der]
  \item \textbf{Derive $\Gamma_0$ from mode spectrum:} Connect prefactor to thick-brane
        mode structure
  \item \textbf{Validate WKB approximation:} Check that $V_B \gg \hbar\omega$ for
        semiclassical regime
\end{enumerate}

\textbf{See also:} Open Problem RT-CH5-001 (derive $\tau_n$ from 5D).
\end{tcolorbox}

% --- Bridge note ---
\paragraph{Connection to existing neutron package.}

The effective action \eqref{eq:effective_action} is consistent with the damped oscillator
and pumping picture in \S\ref{sec:case_neutron}:
\begin{itemize}[nosep]
  \item The ``charging'' power $\Pi_{\mathrm{pump}} = -\dot{q} \partial_q V$ matches
        the pumping picture in the neutron chapter
  \item The collective coordinate $q$ is the same variable used throughout the neutron chapter
  \item The trigger condition $q(t_\star) \approx 0$ corresponds to reaching the proton anchor
\end{itemize}

% --- Route B Status Map ---
\paragraph{Route B status map.}

\begin{center}
\begin{tabular}{llll}
\toprule
\textbf{Step} & \textbf{Claim} & \textbf{Tag} & \textbf{Source} \\
\midrule
B1 & Effective action form & [P]+[Dc] & Lemma~\ref{lem:B1_effective_action} \\
B2 & WKB tunneling formula & [M] & Lemma~\ref{lem:B2_wkb} \\
B3 & $V_B \approx 2.6$ MeV reproduces $\tau_n$ & [Cal] & Prop.~\ref{prop:B3_calibration} \\
\midrule
--- & $V(q)$ from 5D action & OPEN & --- \\
--- & $M(q)$ from 5D geometry & OPEN & --- \\
--- & $V_B$ from first principles & OPEN & --- \\
--- & $\Gamma_0$ from mode spectrum & OPEN & --- \\
\bottomrule
\end{tabular}
\end{center}

% ==============================================================================
% CONVERGENCE: Route A ∩ Route B
% ==============================================================================

\subsubsection{Convergence: Structural Metastability and Effective Lifetime}
\label{subsec:neutron_convergence}

Both routes agree that the neutron is a metastable excitation above the proton anchor.
Route A provides the \emph{structural} (5D topological-geometric) picture; Route B provides
the \emph{effective} (1D WKB) quantitative lifetime. We do not claim to derive the microscopic
decay channel; here ``decay'' is modeled as tunneling/relaxation in the EDC configuration coordinate~$q$.

\paragraph{Shared objects.}

Both routes use:
\begin{itemize}[nosep]
  \item \textbf{Collective coordinate $q$:} Neutron at $q_n > 0$, proton at $q = 0$
  \item \textbf{Barrier concept:} Separation between neutron minimum and proton anchor
  \item \textbf{Proton anchor:} Terminal state of decay (established via dual-route in
        \S\ref{sec:proton_routeA_anchor}, \S\ref{sec:routeB_z6_to_steiner})
  \item \textbf{Topological sector:} Y-junction class $\mathcal{C}_Y$ preserved throughout
\end{itemize}

\paragraph{Route comparison.}

\begin{center}
\begin{tabular}{lll}
\toprule
\textbf{Aspect} & \textbf{Route A (Structural)} & \textbf{Route B (Effective)} \\
\midrule
Framework & 5D topological-geometric & Effective 1D WKB \\
Primary output & Metastability existence & Lifetime $\tau_n$ \\
Key assumption & Barrier exists [P] & $V_B$, $M(q)$ values [Cal] \\
Quantitative? & No (qualitative structure) & Yes ($\tau_n \approx 879$ s) \\
5D-derived? & Partially [Dc] & Not yet (OPEN) \\
Imports external decay microphysics? & No & No \\
\bottomrule
\end{tabular}
\end{center}

% --- What we claim / don't claim ---
\begin{tcolorbox}[colback=green!5!white, colframe=green!60!black,
                  title={\textbf{Convergence Statement: What We Claim and What We Don't}}]

\textbf{What we DO claim:}
\begin{itemize}[nosep]
  \item Neutron is a metastable Y-junction configuration with $q_n > 0$ [Dc]+[P]
  \item Decay is relaxation toward proton anchor ($q \to 0$) within topological sector [Dc]
  \item Effective 1D WKB model reproduces $\tau_n \approx 879$ s with calibrated $V_B$ [Cal]
  \item No external decay-channel microphysics is required for this picture
\end{itemize}

\textbf{What we do NOT claim:}
\begin{itemize}[nosep]
  \item We do \textbf{not} derive $\tau_n$ from 5D action (OPEN)
  \item We do \textbf{not} derive $V_B$, $M(q)$, $V(q)$ from first principles (OPEN)
  \item We do \textbf{not} claim BC generate attraction or create the barrier
  \item We do \textbf{not} import any external microscopic mechanism language; the model
        is purely an effective barrier/tunneling description in~$q$
  \item We do \textbf{not} claim the effective model is the final 5D derivation
\end{itemize}

\textbf{Path forward:} Derive $S_{\mathrm{eff}}[q]$ from 5D action to convert [Cal] $\to$ [Der].
\end{tcolorbox}

% --- Final status summary ---
\paragraph{Combined epistemic status.}

\begin{center}
\begin{tabular}{lll}
\toprule
\textbf{Claim} & \textbf{Tag} & \textbf{Route} \\
\midrule
Neutron in same topo sector as proton & [M]+[P] & A \\
Neutron has $E > E_{\mathrm{proton}}$ & [Dc] & A \\
Metastable local minimum exists & [Dc]+[P] & A \\
Decay via barrier penetration & [Dc] & A+B \\
$\tau_n \approx 879$ s reproduced & [Cal] & B \\
\midrule
BC do not generate attraction & [Der] & A (audit) \\
Full 5D derivation of $V_B$, $M$, $V$ & OPEN & --- \\
\bottomrule
\end{tabular}
\end{center}

