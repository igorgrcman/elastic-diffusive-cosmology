%!TEX root = ../EDC_Part_II_Weak_Sector.tex
% ============================================================================
% PMNS Attempt 4: Menu Sweep (Rank-2 + Double-Path)
% Status: YELLOW [I/Cal] — Structure identified, parameters not derived
% ============================================================================

\subsection{Attempt 4: Rank-2 Baseline and Double-Path Mechanisms}
\label{sec:pmns_attempt4}

Attempts~2 and~3 established that (i)~$\theta_{23}$ emerges from $\mathbb{Z}_6$
geometry~\tagDc{}, and (ii)~discrete phases alone cannot produce the asymmetric
PMNS pattern (large $\theta_{12}$, $\theta_{23}$; small $\theta_{13}$). Attempt~4
tests \emph{structured perturbative approaches} that may achieve this pattern
while preserving the geometric $\theta_{23}$.

\subsubsection{Mechanisms Tested}

\paragraph{A4-1: Rank-2 baseline + reactor perturbation.}
Construct the PMNS matrix as a product of rotations:
\begin{equation}
  U_{\text{PMNS}} = R_{23}(\theta_{23}^0) \cdot R_{13}(\varepsilon) \cdot R_{12}(\theta_{12}^0)
  \label{eq:a41_construction}
\end{equation}
where $\theta_{23}^0$ is fixed from $\mathbb{Z}_6$ geometry~\tagDc{}, $\theta_{12}^0$
is a discrete candidate, and $\varepsilon$ is a small ``reactor'' perturbation coupling
generations~1 and~3.

\paragraph{A4-2: Double-path mixing.}
Construct an effective mixing generator as a sum of two contributions:
\begin{equation}
  H = H_1 + r \, e^{i\phi} H_2
  \label{eq:a42_construction}
\end{equation}
with $H_1$ encoding (2--3) mixing and $H_2$ encoding (1--2) or (1--3) structure.
The PMNS matrix is then $U = \exp(iH)$.

\paragraph{A4-3: Flavor-dependent localization (bonus).}
Allow different penetration depths $\kappa_\alpha$ for each flavor, testing whether
asymmetric localization can produce the PMNS pattern.

\subsubsection{Track Definitions}

\begin{description}[leftmargin=1em]
  \item[Track A (discrete only):] No continuous parameters fitted. All inputs
    from discrete sets or prior geometric derivations.
  \item[Track B (one calibration):] Exactly one parameter $[Cal]$ fitted to
    match a target observable.
\end{description}

\subsubsection{Results}

\begin{table}[ht]
\centering
\caption{PMNS Attempt 4: Stoplight summary}
\label{tab:pmns_attempt4}
\begin{tabular}{llccccc}
\toprule
\textbf{Model} & \textbf{Track} & \textbf{sin$^2\theta_{12}$} & \textbf{sin$^2\theta_{23}$} &
\textbf{sin$^2\theta_{13}$} & \textbf{$[Cal]$} & \textbf{Overall} \\
\midrule
\textbf{PDG 2024} & $[$BL$]$ & 0.307 & 0.546 & 0.022 & — & — \\
\midrule
A4-1 & A & \textcolor{green!60!black}{0.308} & \textcolor{green!60!black}{0.564} &
  \textcolor{green!60!black}{0.022} & None$^*$ & \textcolor{green!60!black}{\textbf{GREEN}} \\
A4-1 & B & \textcolor{green!60!black}{0.308} & \textcolor{green!60!black}{0.564} &
  \textcolor{green!60!black}{0.022} & $\varepsilon$ & \textcolor{green!60!black}{\textbf{GREEN}} \\
\addlinespace
A4-2 & A & \textcolor{red}{0.124} & \textcolor{red}{0.959} &
  \textcolor{green!60!black}{0.019} & None & \textcolor{orange}{YELLOW} \\
A4-2 & B & \textcolor{red}{0.142} & \textcolor{red}{0.966} &
  \textcolor{green!60!black}{0.022} & $r$ & \textcolor{orange}{YELLOW} \\
\addlinespace
A4-3 & A & \textcolor{orange}{0.219} & \textcolor{orange}{0.602} &
  \textcolor{red}{0.000} & None & \textcolor{red}{RED} \\
A4-3 & B & \textcolor{green!60!black}{0.281} & \textcolor{orange}{0.624} &
  \textcolor{red}{0.007} & $\kappa_e$ & \textcolor{orange}{YELLOW} \\
\bottomrule
\end{tabular}

\vspace{0.5em}
{\footnotesize $^*$Track A uses $\theta_{12}^0 = 33.7°$ from discrete set;
see epistemic note below.}
\end{table}

\paragraph{Best configuration (A4-1).}
The rank-2 construction with:
\begin{align}
  \theta_{23}^0 &= \arcsin\sqrt{0.564} \approx 48.7° \quad \text{(from geometry [Dc])} \\
  \theta_{12}^0 &= 33.7° \quad \text{(discrete candidate [I])} \\
  \varepsilon &= 0.15 \text{ rad} \quad \text{(Track A) or calibrated [Cal]}
\end{align}
achieves all three angles within 3\% of PDG values:
\begin{center}
\begin{tabular}{lccl}
\toprule
Angle & Model & PDG & Status \\
\midrule
$\sin^2\theta_{12}$ & 0.308 & 0.307 & \textcolor{green!60!black}{GREEN} (0.3\%) \\
$\sin^2\theta_{23}$ & 0.564 & 0.546 & \textcolor{green!60!black}{GREEN} (3\%) \\
$\sin^2\theta_{13}$ & 0.022 & 0.022 & \textcolor{green!60!black}{GREEN} ($<1$\%) \\
\bottomrule
\end{tabular}
\end{center}

\subsubsection{Epistemic Assessment}

\begin{tcolorbox}[colback=yellow!5!white, colframe=yellow!60!black, title=Epistemic Warning]
\textbf{Track A achieves GREEN numerically, but:}
\begin{itemize}[nosep]
  \item $\theta_{23}^0 = 48.7°$: \textbf{Derived [Dc]} from $\mathbb{Z}_6$ geometry (Attempt~2)
  \item $\theta_{12}^0 = 33.7°$: \textbf{Identified [I]} — matches PDG exactly, but no geometric
    derivation exists. The value $33.7° = \arcsin\sqrt{0.307}$ is the observed solar angle;
    using it as input is identification, not derivation.
  \item $\varepsilon = 0.15$ rad: \textbf{Effectively calibrated} — this value produces
    $\sin^2\theta_{13} = 0.022$. No geometric origin for $\varepsilon \approx \sqrt{0.022}$
    has been identified.
\end{itemize}

\textbf{Honest classification:}
\begin{itemize}[nosep]
  \item $\theta_{23}$: GREEN [Dc]
  \item $\theta_{12}$: YELLOW [I] — structure works, value not derived
  \item $\theta_{13}$: YELLOW [I/Cal] — controlled by $\varepsilon$, value not derived
\end{itemize}
\end{tcolorbox}

\subsubsection{What Attempt 4 Establishes}

\begin{enumerate}
  \item \textbf{Structure identified:} The asymmetric PMNS pattern
    ($\theta_{23} \approx 45°$, $\theta_{12} \approx 34°$, $\theta_{13} \approx 8.5°$)
    can be produced by a rank-2 baseline with small reactor perturbation.

  \item \textbf{$\theta_{23}$ preserved:} The geometric derivation from Attempt~2
    ($\sin^2\theta_{23} = 0.564$) survives in the A4-1 construction.

  \item \textbf{$\theta_{12}$ requires input:} No discrete geometric angle naturally
    produces $\theta_{12} \approx 33.7°$. The value $54.7°$ (tribimaximal magic angle)
    gives $\sin^2\theta_{12} = 0.67$, far from 0.307.

  \item \textbf{$\theta_{13}$ controllable:} The reactor angle is set by $\varepsilon$;
    $\varepsilon \approx 0.15$ rad gives the observed value. A geometric origin for
    this perturbation is needed.

  \item \textbf{Double-path fails:} A4-2 breaks $\theta_{23}$, producing values near
    maximal ($\sin^2\theta_{23} \approx 0.96$) regardless of parameters.

  \item \textbf{Flavor-$\kappa$ insufficient:} A4-3 cannot simultaneously fit all three
    angles even with calibration.
\end{enumerate}

\subsubsection{Comparison with CKM}

The CKM matrix was derived~\tagDc{} from the same exponential overlap framework that
failed for PMNS (Attempts~2--3). Why does CKM work but PMNS fail?

\begin{center}
\begin{tabular}{lcc}
\toprule
\textbf{Property} & \textbf{CKM (quarks)} & \textbf{PMNS (leptons)} \\
\midrule
Mixing pattern & Hierarchical ($\lambda$, $\lambda^2$, $\lambda^3$) & Asymmetric (large-large-small) \\
Largest angle & $\theta_{12}^q \approx 13°$ & $\theta_{23}^\ell \approx 45°$ \\
Smallest angle & $\theta_{13}^q \approx 0.2°$ & $\theta_{13}^\ell \approx 8.5°$ \\
Overlap success & YES (exponential hierarchy) & PARTIAL ($\theta_{23}$ only) \\
\bottomrule
\end{tabular}
\end{center}

\textbf{Interpretation:} The exponential overlap model naturally produces
\emph{hierarchical} patterns (all small or exponentially ordered). CKM is
hierarchical; PMNS is not. The PMNS asymmetry (two large angles, one small)
requires a different mechanism—Attempt~4 identifies the \emph{structure}
(rank-2 + perturbation) but not the \emph{geometric origin} of the parameters.

\subsubsection{Verdict}

\begin{tcolorbox}[colback=green!5!white, colframe=green!50!black, title=Attempt 4 Verdict: Structure Identified]
\textbf{A4-1 (Rank-2 + $\varepsilon$): GREEN numerically, YELLOW epistemically.}

The construction $U = R_{23}(\theta_{23}^0) \cdot R_{13}(\varepsilon) \cdot R_{12}(\theta_{12}^0)$
with $\theta_{23}^0$ from geometry and appropriate $(\theta_{12}^0, \varepsilon)$
reproduces all three PMNS angles within 3\%.

\textbf{What is closed:}
\begin{itemize}[nosep]
  \item $\theta_{23}$: derived from $\mathbb{Z}_6$ geometry [Dc]
  \item PMNS structure: rank-2 baseline + reactor perturbation [I]
\end{itemize}

\textbf{What remains open:}
\begin{itemize}[nosep]
  \item Geometric origin of $\theta_{12}^0 \approx 33.7°$
  \item Geometric origin of $\varepsilon \approx 0.15$ rad
  \item CP phase $\delta_{\text{PMNS}}$ (not addressed)
\end{itemize}

\textbf{OPR-13 status:} Remains \textbf{YELLOW [Dc/I]}. $\theta_{23}$ derived;
$\theta_{12}$, $\theta_{13}$ structure identified but values require derivation.
\end{tcolorbox}

\paragraph{Code.} \texttt{code/pmns\_attempt4\_menu\_sweep.py}

\paragraph{Output.} \texttt{code/output/pmns\_attempt4\_results.txt}
