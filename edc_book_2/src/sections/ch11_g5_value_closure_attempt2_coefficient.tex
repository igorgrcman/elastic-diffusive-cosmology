%!TEX root = ../EDC_Part_II_Weak_Sector.tex
% ==============================================================================
% Chapter 11: Coefficient Provenance for G1 (OPR-19 Attempt 2)
% Status: Coefficient candidates surveyed [P]; closure still RED-C [OPEN]
% ==============================================================================

\subsection{Coefficient Provenance: Attempt 2}
\label{sec:ch11_coefficient_attempt}

\subsubsection{Context from Attempt 1}

The value closure attempt (\S\ref{sec:ch11_value_closure_attempt}) identified G1
as the most promising SM-free formula for the gauge coupling:
\begin{equation}
    \boxed{
    g^2_{\text{G1}} = 4\pi \times \frac{\sigma r_e^3}{\hbar c} \approx 0.373
    }
    \label{eq:ch11_g1_recap}
\end{equation}
This is 11\% below the SM value $g_2^2 \approx 0.42$ (computed from
$g_2^2 = 4\pi\alpha/\sin^2\theta_W$). The numeric proximity is encouraging,
but two questions remain open:

\begin{enumerate}[nosep]
    \item \textbf{Why $4\pi$?} The coefficient $4\pi$ appears without derivation.
    \item \textbf{Why $r_e^3$?} The dimensional structure $\sigma r_e^3/(\hbar c)$
          combines membrane tension with a \emph{volume} scaling.
\end{enumerate}

\paragraph{Goal of this section.}
Survey candidate origins for the $4\pi$ coefficient and $r_e^3$ dimensional
structure, determining which (if any) can be derived from EDC geometry.

\begin{tcolorbox}[colback=blue!5, colframe=blue!60!black,
    title=\textbf{Executive Summary: Coefficient Provenance}]
\textbf{Question:} Can we derive the $4\pi$ coefficient in G1 from first principles?

\textbf{Result:} Multiple geometric origins exist for $4\pi$, but \textbf{none
uniquely selects it} over alternatives like $2\pi$, $\pi$, or $4\pi/3$.
The choice remains \tagP{}.

\textbf{Key insight:} The $4\pi$ factor is consistent with:
\begin{itemize}[nosep]
    \item Solid angle integration: $\int d\Omega = 4\pi$
    \item Sphere area normalization: $A = 4\pi r_e^2$
    \item Flux quantization through $S^2$ (Dirac monopole)
\end{itemize}
But without a specific physical mechanism linking the coupling to one of these,
the coefficient is postulated, not derived.

\textbf{Status:} OPR-19 remains \textbf{RED-C [OPEN]}. G1 numeric closeness (11\%)
is a consistency check; coefficient provenance is the closure gate.
\end{tcolorbox}

% ------------------------------------------------------------------------------
\subsubsection{No-Smuggling Checklist}
\label{sec:ch11_coefficient_guardrails}

\begin{tcolorbox}[colback=red!5!white, colframe=red!60!black,
    title=\textbf{No-Smuggling Checklist (Attempt 2)}]
\textbf{Forbidden:}
\begin{itemize}[nosep]
    \item[\ding{55}] Using SM $g_2^2 \approx 0.42$ to calibrate the coefficient
    \item[\ding{55}] Choosing coefficient to match SM a posteriori
    \item[\ding{55}] Importing $M_W$, $G_F$, or $v = 246$ GeV
\end{itemize}

\textbf{Allowed:}
\begin{itemize}[nosep]
    \item[\ding{51}] $\sigma r_e^2 = 5.86$ MeV \tagDc{} (from $\mathbb{Z}_6$ geometry)
    \item[\ding{51}] $r_e = 1$ fm \tagP{} (lattice spacing postulate)
    \item[\ding{51}] Geometric factors from sphere/angular integrals \tagDc{}
    \item[\ding{51}] Action normalization conventions \tagBL{}
\end{itemize}

\textbf{Procedure:} Coefficient candidates are derived from allowed geometric
structures. A posteriori comparison to SM is a \emph{consistency check}, not calibration.
\end{tcolorbox}

% ------------------------------------------------------------------------------
\subsubsection{Coefficient Provenance Candidates}
\label{sec:ch11_coeff_table}

\begin{table}[ht]
\centering
\caption{Coefficient provenance candidates for G1}
\label{tab:ch11_coeff_candidates}
\small
\begin{tabular}{p{3.5cm}p{4.5cm}cccc}
\toprule
\textbf{Candidate} & \textbf{Origin} & \textbf{Coeff.} & \textbf{$g^2$} & \textbf{Derived?} & \textbf{Tag} \\
\midrule
\multicolumn{6}{l}{\textit{Angular / surface integrals}} \\
$4\pi$ (solid angle) & $\int d\Omega = 4\pi$ & $4\pi$ & 0.373 & \tagDc{} & candidate \\
$4\pi r_e^2$ (sphere area) & Area normalization & $4\pi$ & 0.373 & \tagDc{} & candidate \\
$2\pi$ (circle) & $\oint d\theta = 2\pi$ & $2\pi$ & 0.187 & \tagDc{} & too small \\
\addlinespace
\multicolumn{6}{l}{\textit{Volume factors}} \\
$4\pi/3$ (sphere volume) & $V = \frac{4}{3}\pi r^3$ & $4\pi/3$ & 0.124 & \tagDc{} & too small \\
$8\pi/3$ (hemisphere) & Half volume $\times 2$ & $8\pi/3$ & 0.249 & \tagP{} & possible \\
\addlinespace
\multicolumn{6}{l}{\textit{Action normalization}} \\
$1/4$ (gauge action) & $-\frac{1}{4}F_{\mu\nu}F^{\mu\nu}$ & varies & — & \tagBL{} & absorbed \\
$1/(4\pi)$ (SI/Heaviside) & Unit system & varies & — & \tagBL{} & convention \\
\addlinespace
\multicolumn{6}{l}{\textit{Flux / topological}} \\
$4\pi$ (Dirac monopole) & Flux quantization & $4\pi$ & 0.373 & \tagDc{} & if monopole \\
$\pi$ (half flux) & Fractional charge & $\pi$ & 0.093 & \tagP{} & too small \\
\addlinespace
\multicolumn{6}{l}{\textit{Composite factors}} \\
$\sqrt{2} \times 2\pi$ & If $\mathbb{Z}_2$ enhances & $\sqrt{2} \times 2\pi$ & 0.264 & \tagP{} & intermediate \\
$\sqrt{3} \times 2\pi$ & If $\mathbb{Z}_3$ enhances & $\sqrt{3} \times 2\pi$ & 0.323 & \tagP{} & closer \\
\bottomrule
\end{tabular}
\end{table}

% ------------------------------------------------------------------------------
\subsubsection{Analysis of Leading Candidates}
\label{sec:ch11_coeff_analysis}

\paragraph{Candidate 1: Solid angle integration.}

The most natural origin for $4\pi$ is the solid angle:
\begin{equation}
    \int_{S^2} d\Omega = \int_0^\pi \sin\theta\, d\theta \int_0^{2\pi} d\phi = 4\pi
    \label{eq:ch11_solid_angle}
\end{equation}
This would appear if the coupling arises from integrating a flux or field
strength over a spherical surface centered on a defect:
\begin{equation}
    g^2 \propto \int_{S^2} \mathcal{F} \cdot d\mathbf{A} = 4\pi \mathcal{F}_0 r_e^2
    \label{eq:ch11_flux_integral}
\end{equation}
With $\mathcal{F}_0 \sim \sigma r_e / (\hbar c)$, this gives:
\begin{equation}
    g^2 = 4\pi \times \frac{\sigma r_e^3}{\hbar c} \quad \checkmark
\end{equation}

\textbf{Assessment:} The solid angle origin is geometrically natural and gives
the correct coefficient. However, the physical mechanism (what flux? through what
surface?) is not specified. \textbf{Status:} \tagDc{} for the factor $4\pi$;
\tagP{} for the physical interpretation.

\paragraph{Candidate 2: Sphere area normalization.}

If the coupling is defined per unit area of a brane defect with radius $r_e$:
\begin{equation}
    g^2 = \frac{\text{(effective coupling)} \times A_{S^2}}{(\hbar c)}
    = \frac{\sigma r_e \times 4\pi r_e^2}{\hbar c}
    = \frac{4\pi \sigma r_e^3}{\hbar c} \quad \checkmark
\end{equation}
This matches G1. The interpretation is that the coupling strength is proportional
to the surface area of the topological defect.

\textbf{Assessment:} Dimensionally and numerically consistent. Requires postulating
that the relevant area is $4\pi r_e^2$ rather than, say, the hexagonal cell area
from $\mathbb{Z}_6$. \textbf{Status:} \tagP{} (specific surface choice not derived).

\paragraph{Candidate 3: Volume vs. area.}

Why $r_e^3$ instead of $r_e^2$? The dimensional structure $\sigma r_e^3/(\hbar c)$
is dimensionless:
\begin{align}
    [\sigma r_e^2] &= [E] \\
    [\sigma r_e^3 / (\hbar c)] &= [E \cdot L / (E \cdot L)] = [E]^0 \quad \checkmark
\end{align}
The factor $r_e^3$ could arise from:
\begin{itemize}[nosep]
    \item Area ($4\pi r_e^2$) times radius ($r_e$) for flux-line counting
    \item Volume integration $\int r^2 dr$ with cutoff at $r_e$
    \item Overlap integral involving brane thickness
\end{itemize}
Without a specific calculation, the $r_e^3$ structure is \tagP{}.

\paragraph{Sensitivity to coefficient choice.}

\begin{table}[ht]
\centering
\caption{$g^2$ values for different coefficients (using $\sigma r_e^3/(\hbar c) = 0.0297$)}
\label{tab:ch11_g2_sweep}
\small
\begin{tabular}{lccl}
\toprule
\textbf{Coefficient} & \textbf{Value} & \textbf{$g^2$} & \textbf{Status vs SM ($g_2^2 \approx 0.42$)} \\
\midrule
$\pi$ & 3.14 & 0.093 & 78\% below \\
$2\pi$ & 6.28 & 0.187 & 55\% below \\
$\sqrt{3} \times 2\pi$ & 10.88 & 0.323 & 23\% below \\
$4\pi$ & 12.57 & 0.373 & \textbf{11\% below} \\
$4\pi/\alpha^{1/2}$ & 14.02 & 0.416 & 1\% below (tuned) \\
$\sqrt{2} \times 4\pi$ & 17.77 & 0.528 & 26\% above \\
\bottomrule
\end{tabular}
\end{table}

\paragraph{Key observation.}
The coefficient $4\pi$ is the unique ``nice'' geometric factor (solid angle, sphere
area) that gives $g^2$ within 20\% of the SM value. Smaller factors ($\pi$, $2\pi$)
underpredict; larger factors ($4\pi\sqrt{2}$, $8\pi$) overpredict.

This is \emph{suggestive} but not \emph{derived}. We cannot claim that the membrane
geometry ``predicts'' $4\pi$ without showing which physical calculation produces it.

% ------------------------------------------------------------------------------
\subsubsection{Mini-Derivation Sketch: Flux Through Defect Sphere}
\label{sec:ch11_mini_derivation}

\paragraph{Setup.}
Consider a spherical brane defect of radius $r_e$ embedded in 5D. The defect carries
surface tension $\sigma$. The 4D effective coupling is determined by the flux of a
5D gauge field through the defect boundary.

\paragraph{Flux quantization ansatz.}
If the gauge field is sourced by the defect with total flux:
\begin{equation}
    \Phi = \int_{S^2(r_e)} \mathbf{B} \cdot d\mathbf{A} = g_0 \times 4\pi r_e^2
    \label{eq:ch11_flux}
\end{equation}
where $g_0$ is a ``bare'' coupling per unit area, then the effective 4D coupling is:
\begin{equation}
    g_{\text{eff}}^2 = g_0^2 \times (4\pi r_e^2)^2 / \text{(normalization)}
\end{equation}

\paragraph{Tension-coupling relation.}
Postulate that $g_0 \propto \sigma^{1/2}$, giving:
\begin{equation}
    g^2 \sim \sigma \times (\text{geometric factor}) \times \frac{r_e^3}{\hbar c}
\end{equation}
The geometric factor from the spherical integration is $4\pi$.

\textbf{Status of this derivation:} The steps are plausible but contain postulates:
\begin{itemize}[nosep]
    \item[$\checkmark$] Spherical geometry at $r_e$ scale is \tagDc{} (from defect picture)
    \item[\ding{55}] $g_0 \propto \sigma^{1/2}$ is \tagP{} (tension-coupling relation)
    \item[\ding{55}] Specific normalization factor is \tagP{}
\end{itemize}

\paragraph{Verdict on mini-derivation.}
The flux picture motivates $4\pi$ but does not uniquely derive it. The tension-coupling
relation $g_0 \propto \sigma^{1/2}$ would need to come from a microscopic action principle.

% ------------------------------------------------------------------------------
\subsubsection{Closure Path Forward}
\label{sec:ch11_coeff_closure_path}

\paragraph{What would close OPR-19.}
Any of the following would upgrade the coefficient from \tagP{} to \tagDc{}:

\begin{enumerate}[nosep]
    \item \textbf{Derive $g^2$ from 5D action.}
          Start from the 5D gauge action $S = \int d^5x\, (-\frac{1}{4g_5^2}) F^2$,
          perform KK reduction on a finite interval $[0, \ell]$, and show that
          the 4D coupling $g_4^2$ equals $4\pi \sigma r_e^3/(\hbar c)$.

    \item \textbf{Derive from brane-localized action.}
          If the gauge field is brane-localized with action
          $S_{\text{brane}} = \int d^4x\, \delta(z) (-\frac{1}{4g_b^2}) F^2$,
          derive $g_b^2$ from membrane tension.

    \item \textbf{Show $4\pi$ from flux quantization.}
          If the defect carries quantized flux (Dirac-like), derive the flux
          quantum from EDC topology and show it gives $4\pi \sigma r_e^3/(\hbar c)$.

    \item \textbf{Derive from $\mathbb{Z}_6$ lattice.}
          If the hexagonal lattice structure determines the coefficient, compute
          the relevant geometric integral over the $\mathbb{Z}_6$ cell.
\end{enumerate}

\paragraph{Recommended next action.}
\begin{itemize}[nosep]
    \item \textbf{5D reduction:} Perform explicit KK reduction of 5D gauge action
          with EDC brane profile; extract coefficient.
    \item \textbf{$\mathbb{Z}_6$ geometry:} Compute overlap integrals over hexagonal
          cell; check if $4\pi$ or variant emerges.
    \item \textbf{Tension-coupling:} Derive relation between $\sigma$ and gauge
          coupling from brane action principle.
\end{itemize}

% ------------------------------------------------------------------------------
\subsubsection{Attempt 2 Verdict}
\label{sec:ch11_coeff_verdict}

\begin{tcolorbox}[colback=green!5, colframe=green!50!black,
    title=\textbf{OPR-19 Coefficient Provenance: Status}]

\textbf{Before this section:}
\begin{quote}
OPR-19: RED-C [OPEN] --- G1 formula $g^2 = 4\pi \sigma r_e^3/(\hbar c)$ gives
0.37 (11\% from SM); coefficient ``$4\pi$'' unexplained.
\end{quote}

\textbf{After this section:}
\begin{quote}
OPR-19: \textbf{RED-C [OPEN]} --- Coefficient candidates enumerated; no unique
derivation.
\begin{itemize}[nosep]
    \item $4\pi$ consistent with solid angle, sphere area, or flux quantization
    \item All alternatives ($\pi$, $2\pi$, $4\pi/3$) give $g^2$ outside 20\% of SM
    \item Tension-coupling relation $g^2 \propto \sigma$ is \tagP{}
\end{itemize}
\end{quote}

\medskip
\noindent\fbox{\parbox{0.94\textwidth}{\small
\textbf{Honest verdict:} The $4\pi$ coefficient is \emph{geometrically natural}
(solid angle, sphere area) and \emph{numerically successful} (11\% from SM), but
it is not \emph{uniquely derived}.

\textbf{Progress:} We have established that:
\begin{itemize}[nosep]
    \item G1's numeric proximity is a consistency check, not a derivation
    \item The coefficient must come from spherical/angular geometry
    \item The closure gate is the tension-coupling relation from action principle
\end{itemize}

\textbf{Remaining gap:} Derive $g^2 = 4\pi \sigma r_e^3/(\hbar c)$ from a 5D action
or brane-localized field theory.}}
\end{tcolorbox}

