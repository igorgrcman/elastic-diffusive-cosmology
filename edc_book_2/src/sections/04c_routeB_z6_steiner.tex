% ==============================================================================
% Section: Route B --- Z6 Crystallization to Steiner 120°
% Status: Independent derivation path, complementary to Route A
% ==============================================================================

\subsection{Route B: $\mathbb{Z}_6$ Crystallization $\to$ Steiner 120°}
\label{sec:routeB_z6_to_steiner}

This section provides an \emph{independent} derivation of the proton's 120° Y-junction
geometry via the $\mathbb{Z}_6$ crystallization program. Route B is complementary to
Route A (\S\ref{sec:proton_routeA_anchor}); both converge on the same physical result.

% ------------------------------------------------------------------------------
% Postulate P2
% ------------------------------------------------------------------------------

\begin{edcPostulateBox}{Flux Tube Interactions (P2)}{[P]}
\label{post:P2_flux_tubes}
Flux tubes (topological defects) in the thick-brane have:
\begin{enumerate}[nosep]
  \item \textbf{Short-range repulsion:} Core overlap creates energy divergence
        ($V_{\mathrm{core}} \to +\infty$ as $d \to 0$)
  \item \textbf{Long-range confinement:} Logarithmic or linear growth
        ($V_{\mathrm{lin}} \to +\infty$ as $d \to \infty$)
\end{enumerate}

The combined potential $V(d) = V_{\mathrm{core}}(d) + V_{\mathrm{lin}}(d)$ has a minimum
at some characteristic distance $d_0 > 0$.
\end{edcPostulateBox}

\begin{tcolorbox}[colback=yellow!5!white, colframe=yellow!50!black,
                  title={\textbf{Remark: What We Do NOT Claim}}]
\label{rem:routeB_BC_disclaimer}
\textbf{Important clarification:}
\begin{itemize}[nosep]
  \item We do \textbf{not} derive attraction from $\xi$-boundary conditions.
  \item Linearized BC analysis gives $V'_{\mathrm{lin}}(d) > 0$ for all BC
        (Neumann, Robin, Dirichlet)---see \texttt{aside\_frozen\_brane\_bc\_v1/}.
  \item BC provide the scale $\delta$ (brane thickness) and mode spectrum,
        \textbf{not} force sign reversal.
  \item The minimum mechanism is \textbf{topological} (core repulsion + log growth),
        not BC-induced.
  \item The specific location $d_0 \sim \delta$ remains [P] or [Cal], not [Der].
\end{itemize}
\end{tcolorbox}

% ------------------------------------------------------------------------------
% Lemma B1: Kepler-Hales
% ------------------------------------------------------------------------------

\begin{edcLemmaBox}{Kepler--Hales Optimal Packing (2D)}{[M]}
\label{lem:B1_kepler_hales}
The densest packing of equal circles in 2D is the hexagonal lattice, with packing
fraction $\pi/(2\sqrt{3}) \approx 90.69\%$. This is a corollary of the Kepler
conjecture (Hales, 2005).
\end{edcLemmaBox}

\begin{proof}[Reference]
Classical result; see Hales (2005) for the 3D proof and its 2D corollary.
The hexagonal lattice uniquely maximizes packing density among all periodic
arrangements of equal disks. \hfill $\square$
\end{proof}

% ------------------------------------------------------------------------------
% Lemma B2: Hexagonal crystallization
% ------------------------------------------------------------------------------

\begin{edcLemmaBox}{Hexagonal Crystallization}{[Dc]}
\label{lem:B2_hex_crystallization}
Given P2 (Postulate~\ref{post:P2_flux_tubes}) and energy minimization with stable
separation $d_0$, the equilibrium configuration of flux tubes on the 2D brane
crystallizes to a hexagonal lattice.
\end{edcLemmaBox}

\begin{proof}
By P2, flux tubes have a potential $V(d)$ with minimum at $d_0$. In equilibrium,
each tube seeks to minimize total energy. By Lemma~\ref{lem:B1_kepler_hales},
hexagonal packing maximizes coordination number (6) at fixed spacing $d_0$.
If $V(d_0) < 0$ (net binding), hexagonal has lowest energy per particle:
$\epsilon_{\mathrm{hex}} = 3V(d_0) < \epsilon_{\mathrm{square}} = 2V(d_0)$.
Crystallization to hexagonal follows. \hfill $\square$
\end{proof}

% ------------------------------------------------------------------------------
% Lemma B3: Z6 symmetry implies equal tensions
% ------------------------------------------------------------------------------

\begin{edcLemmaBox}{$\mathbb{Z}_6$ Symmetry $\to$ Equal Effective Tensions}{[Dc]}
\label{lem:B3_equal_tensions}
The hexagonal lattice has $\mathbb{Z}_6$ rotational symmetry. At a Y-junction
(vertex where three lattice edges meet), the $\mathbb{Z}_3 \subset \mathbb{Z}_6$
subgroup permutes the three edges. This implies equal effective tensions along
all three branches.
\end{edcLemmaBox}

\begin{proof}
The hexagonal lattice is invariant under 60° rotations ($\mathbb{Z}_6$).
A Y-junction vertex has three edges separated by 120°. The $\mathbb{Z}_3$
subgroup (rotations by 120°) cyclically permutes these edges. Since the
energy functional respects lattice symmetry, edge tensions must be equal:
$\tau_1 = \tau_2 = \tau_3 = \tau$. \hfill $\square$
\end{proof}

% ------------------------------------------------------------------------------
% Theorem B: Steiner from equal tensions
% ------------------------------------------------------------------------------

\begin{edcTheoremBox}{Equal-Tension Y-Junction $\to$ Steiner 120°}{[M]}
\label{thm:B_steiner_equal_tension}
An equilibrium Y-junction with three branches of equal tension $\tau$ meeting
at a point has angles of 120° between adjacent branches.
\end{edcTheoremBox}

\begin{proof}
Force balance at the junction: each branch exerts tension $\tau$ along its
direction. For equilibrium, $\sum_i \vec{\tau}_i = 0$. Three equal-magnitude
vectors summing to zero must have 120° between adjacent pairs (by symmetry
and vector addition). This is the Steiner condition. \hfill $\square$
\end{proof}

% ------------------------------------------------------------------------------
% Corollary B: Proton via Route B
% ------------------------------------------------------------------------------

\begin{edcCorollaryBox}{Proton Y-Junction via Route B}{[Dc]}
\label{cor:B_proton_steiner}
The proton Y-junction exhibits 120° Steiner geometry via the following chain:
\begin{enumerate}[nosep]
  \item \textbf{[P]} Postulate P2: Flux tubes have repulsion + confinement with minimum at $d_0$
  \item \textbf{[M]} Kepler--Hales: Optimal 2D packing is hexagonal
  \item \textbf{[Dc]} Crystallization: Flux tubes form hexagonal lattice
  \item \textbf{[Dc]} $\mathbb{Z}_6$ symmetry: Equal tensions at Y-junctions
  \item \textbf{[M]} Steiner theorem: Equal tensions $\Rightarrow$ 120° angles
\end{enumerate}
\end{edcCorollaryBox}

% ==============================================================================
% CONVERGENCE STATEMENT
% ==============================================================================

\subsubsection{Convergence Statement: Route A and Route B}
\label{sec:convergence_routes}

\begin{tcolorbox}[colback=green!5!white, colframe=green!60!black,
                  title={\textbf{Convergence: Two Routes to Steiner 120°}}]

\textbf{Route A} (\S\ref{sec:proton_routeA_anchor}): Topology + Nambu--Goto + Steiner
\begin{center}
$\pi_1$ protection [M]+[P] $\to$ $E = \tau L$ [Der] $\to$ Steiner 120° [M] $\to$ Proton minimum [Dc]
\end{center}

\textbf{Route B} (\S\ref{sec:routeB_z6_to_steiner}): P2 + Crystallization + Steiner
\begin{center}
P2 [P] $\to$ Kepler--Hales [M] $\to$ Hex lattice [Dc] $\to$ Equal tensions [Dc] $\to$ Steiner 120° [M]
\end{center}

\vspace{0.5em}
\textbf{Common terminus:} Both routes use the Steiner theorem [M] as the final
mathematical step. The 120° result is \emph{overdetermined}.

\textbf{Physics enters differently:}
\begin{itemize}[nosep]
  \item Route A: Topological sector + Nambu--Goto action (frozen worldsheets)
  \item Route B: P2 postulate + crystallization dynamics (equilibrium lattice)
\end{itemize}

\textbf{Implication:} The proton's 120° geometry is robust---it follows from
\emph{either} minimal surface physics (Route A) \emph{or} optimal packing physics (Route B).
Neither route requires BC-induced attraction (which does not exist).

\end{tcolorbox}

% ------------------------------------------------------------------------------
% Status summary
% ------------------------------------------------------------------------------

\subsubsection{Epistemic Status of Route B Claims}

\begin{center}
\begin{tabular}{llll}
\toprule
\textbf{Claim} & \textbf{Status} & \textbf{Label} & \textbf{Dependency} \\
\midrule
Flux tube repulsion + confinement & [P] & Postulate~\ref{post:P2_flux_tubes} & P2 \\
Kepler--Hales 2D packing & [M] & Lemma~\ref{lem:B1_kepler_hales} & Classical \\
Hexagonal crystallization & [Dc] & Lemma~\ref{lem:B2_hex_crystallization} & P2 + [M] \\
Equal tensions from $\mathbb{Z}_6$ & [Dc] & Lemma~\ref{lem:B3_equal_tensions} & Hex lattice \\
Steiner 120° from equal tensions & [M] & Theorem~\ref{thm:B_steiner_equal_tension} & Classical \\
Proton 120° via Route B & [Dc] & Corollary~\ref{cor:B_proton_steiner} & Full chain \\
\midrule
BC create attraction & \textbf{FALSE} & Remark~\ref{rem:routeB_BC_disclaimer} & Audit result \\
$d_0 \sim \delta$ location & [P]/[Cal] & --- & Not derived \\
\bottomrule
\end{tabular}
\end{center}

