% ==============================================================================
% Section 1.1: Epistemic Framework and How to Read Chapter 1 (short)
% ==============================================================================
% NOTE: This is the SHORT chapter-level version. The full Reader Contract
% and Epistemic Standard are in the book-level frontmatter.
% ==============================================================================

\section{Epistemic Framework and Chapter Guide}
\label{sec:reader_contract}

This chapter follows the EDC Epistemic Standard (Framework v2.0). Each claim
carries one evidence label: \tagBL{} baseline, \tagDer{} derived, \tagDc{} derived
conditional, \tagI{} identified, \tagCal{} calibrated, \tagP{} proposed, \tagM{} math.
Canonical definitions: Preface.

\paragraph{Reading map.}
\begin{itemize}[nosep]
  \item \textbf{Section~\ref{sec:unified_pipeline}} defines the canonical weak-sector
        pipeline used throughout this Part.
  \item \textbf{Sections~\ref{sec:case_neutron}--\ref{sec:case_neutrino}} apply that
        pipeline to concrete decay cases, specifying only case-dependent inputs.
  \item \textbf{Section~\ref{sec:gf_pathway}} explains the structural pathway toward
        an effective coupling ($G_F$).
  \item \textbf{Section~\ref{sec:epistemic_map}} collects all unresolved items
        (Consolidated Open Problems).
\end{itemize}

\paragraph{One-line rule.}
If a concept is defined elsewhere, we reference it rather than re-explain.
Pipeline = \S\ref{sec:unified_pipeline};
SM$\leftrightarrow$EDC bridge = \S\ref{sec:gf_pathway};
Open problems = \S\ref{sec:epistemic_map}.

