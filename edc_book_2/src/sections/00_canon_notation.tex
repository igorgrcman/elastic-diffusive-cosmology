% ==============================================================================
% CANONICAL NOTATION POLICY
% ==============================================================================
% This file establishes the canonical symbol conventions for EDC Part II,
% aligned with Paper 2 (DOI: 10.5281/zenodo.18211854) published canon.
% ==============================================================================

\section*{Notation Convention}
\addcontentsline{toc}{section}{Notation Convention}
\label{sec:notation_convention}

\begin{tcolorbox}[colback=cyan!5!white, colframe=cyan!50!black,
    title=\textbf{Canonical 5D Coordinate Symbols}]

\textbf{Physical 5D depth coordinate:} $\xi$ (xi)
\begin{itemize}[nosep]
    \item Domain: $\xi \in [0, \infty)$ (half-line) or $\xi \in [0, \ell]$ (compact)
    \item Units: Length
    \item Usage: $V(\xi)$, $\psi(\xi)$, $f(\xi)$, $\dfrac{d}{d\xi}$, $\displaystyle\int d\xi$
    \item Boundary: $\xi = 0$ is the observer brane
\end{itemize}

\medskip

\textbf{Dimensionless depth coordinate:} $\tilde{\xi}$ (xi-tilde)
\begin{itemize}[nosep]
    \item Definition: $\tilde{\xi} := \xi/\ell$ where $\ell$ is the KK scale
    \item Domain: $\tilde{\xi} \in [0, 1]$
    \item Usage: Rescaled BVP problems, $\tilde{V}(\tilde{\xi}) := \ell^2 V(\ell\tilde{\xi})$
\end{itemize}

\medskip

\textbf{Reserved (not used for 5D depth):} $z$, $\zeta$, $y$

\end{tcolorbox}

\begin{tcolorbox}[colback=gray!5!white, colframe=gray!50!black,
    title=\textbf{Cross-Document Compatibility}]
This notation aligns Part~II with the published canon in Paper~2
(DOI:~10.5281/zenodo.18211854), which defines the extra dimension
via $\xi \cong S^1$ in Postulate~3.

The correspondence with earlier draft notation is:
\[
    \xi_{\text{Part II}} \equiv \xi_{\text{Paper 2}} \equiv z_{\text{earlier drafts}}
\]
\end{tcolorbox}
