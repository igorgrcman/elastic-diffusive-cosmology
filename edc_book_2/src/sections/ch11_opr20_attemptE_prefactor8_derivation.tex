%!TEX root = ../EDC_Part_II_Weak_Sector.tex
% ==============================================================================
% Chapter 11: OPR-20 Attempt E — Prefactor-8 First-Principles Derivation
% Status: Track A derives 2π [Dc]; Track B identifies 0.9003 candidates [P]/[OPEN]
% ==============================================================================

\subsection{Attempt E: Prefactor-8 First-Principles Derivation}
\label{sec:ch11_attemptE}

\subsubsection{Executive Summary}
\label{sec:ch11_attemptE_executive}

\begin{tcolorbox}[colback=blue!5, colframe=blue!60!black,
    title=\textbf{OPR-20 Attempt E: Executive Summary}]

\textbf{Objective:} Upgrade the geometric prefactor story from \tagP{} to \tagDc{} by
deriving what $\ell$ is in terms of $R_\xi$, and identify whether the $\sim$12\%
residual can be explained without new free parameters.

\textbf{Two derivation tracks:}
\begin{enumerate}[nosep]
    \item[\textbf{A)}] \textbf{Why $\ell = 2\pi R_\xi$?} --- From diffusion correlation
          length definition and KK mode structure.
    \item[\textbf{B)}] \textbf{Where does the missing 0.9003 come from?} --- Since
          $8/(2\pi\sqrt{2}) \approx 0.9003$, identify EDC-native candidates for this factor.
\end{enumerate}

\textbf{Key findings:}
\begin{itemize}[nosep]
    \item[\ding{51}] \textbf{Track A:} The factor $2\pi$ emerges from the relationship between
          correlation length and circumference of a compactified dimension \tagDc{}
    \item[\ding{51}] \textbf{Track A:} Alternative interpretations ($\pi$, $4\pi$) require
          non-standard setups \tagDc{} (negative closure)
    \item[\ding{55}] \textbf{Track B:} The missing 0.9003 factor has candidates but none
          uniquely derived \tagP{}/[OPEN]
\end{itemize}

\textbf{Status:} OPR-20 partial upgrade: $2\pi$ interpretation now \tagDc{};
exact factor 8 remains [OPEN].
\end{tcolorbox}

% ------------------------------------------------------------------------------
\subsubsection{Recap: Attempt D Closures}
\label{sec:ch11_attemptE_recap}

Previous attempts established firm negative closures that constrain the solution space:

\paragraph{Negative Closures from Attempts A--D.}

\begin{enumerate}[nosep]
    \item \textbf{Standard BC route [Dc] (negative):} Dirichlet-Neumann combinations
          give at most factor-4 reduction in $x_1$. The factor-8 cannot come from
          standard boundary conditions alone.

    \item \textbf{Overcounting audit [Dc]:} The Z$_2$ orbifold factor and the Israel
          junction factor are \emph{the same physics}---they cannot be multiplied.
          Triple-counting ($2 \times 2 \times 2 = 8$) is invalid.

    \item \textbf{Robin BC parameters [Dc]+[P]:} The Robin structure
          $\phi' + \alpha\phi = 0$ is derived from junction physics \tagDc{}, but
          the parameter value $\alpha\ell \sim 0.1$ (needed for factor-8) requires
          mild tuning \tagP{}.
\end{enumerate}

\paragraph{Best Structural Candidate.}

The most honest factor that passes independence checks is:
\begin{equation}
    C_{\text{geom}} = 2\pi\sqrt{2} \approx 8.886
    \quad \Rightarrow \quad
    m_\phi \approx 70 \text{ GeV}
    \quad \text{(12\% below } M_W \approx 80 \text{ GeV)}
    \label{eq:ch11_E_best_candidate}
\end{equation}

The question for Attempt E: Can we derive why the factor is $2\pi\sqrt{2}$ (or exactly 8)
from first principles, without introducing new free parameters?

% ------------------------------------------------------------------------------
\subsubsection{No-Smuggling Guardrails (Attempt E)}
\label{sec:ch11_attemptE_guardrails}

\begin{tcolorbox}[colback=red!5!white, colframe=red!60!black,
    title=\textbf{No-Smuggling Guardrails (Attempt E)}]
\textbf{Forbidden as inputs:}
\begin{itemize}[nosep]
    \item[\ding{55}] $M_W = 80$ GeV, $G_F$, $g_2$, $v = 246$ GeV
    \item[\ding{55}] Any PDG weak-scale numbers to define $\ell$, $x_1$, or $R_\xi$
    \item[\ding{55}] Choosing factors ``because they make 8''
\end{itemize}

\textbf{Allowed:}
\begin{itemize}[nosep]
    \item[\ding{51}] $R_\xi \sim 10^{-3}$ fm from Part I diffusion analysis \tagP{}
    \item[\ding{51}] Geometric constants ($\pi$, $\sqrt{2}$) with stated derivation
    \item[\ding{51}] KK mode structure from 5D compactification \tagDc{}
    \item[\ding{51}] Comparison to $M_W$ only as \tagBL{} sanity check at the end
\end{itemize}
\end{tcolorbox}

% ==============================================================================
% TRACK A: WHY ℓ = 2π R_ξ
% ==============================================================================
\subsubsection{Track A: Derivation of $\ell = 2\pi R_\xi$}
\label{sec:ch11_attemptE_trackA}

\paragraph{\texorpdfstring{Definition of $R_\xi$ in Part I.}{Definition of R-xi in Part I.}}

In the EDC membrane model (Part I), $R_\xi$ is defined as the \emph{correlation length}
of the diffusive/frozen regime. Physically, it characterizes the scale over which
membrane fluctuations are correlated:
\begin{equation}
    \langle \phi(x) \phi(x') \rangle \sim e^{-|x - x'|/R_\xi}
    \quad \text{for } |x - x'| \gg R_\xi
    \label{eq:ch11_E_correlation}
\end{equation}

For a 5D setup where the extra dimension is compactified, $R_\xi$ relates to the
compactification geometry.

\paragraph{Compactification and Circumference.}

Consider a compact extra dimension parametrized by coordinate $y$ with period $L$.
The standard KK decomposition gives:
\begin{equation}
    \phi(x^\mu, y) = \sum_n \phi_n(x^\mu) f_n(y),
    \quad f_n(y) = \frac{1}{\sqrt{L}} e^{2\pi i n y / L}
    \label{eq:ch11_E_KK_decomposition}
\end{equation}

The \emph{radius} of the compact dimension is:
\begin{equation}
    R = \frac{L}{2\pi}
    \quad \Leftrightarrow \quad
    L = 2\pi R
    \label{eq:ch11_E_radius_circumference}
\end{equation}

This is a \emph{definition}, not an assumption: the circumference of a circle of
radius $R$ is $2\pi R$.

\paragraph{\texorpdfstring{Identifying $R_\xi$ with the Radius.}{Identifying R-xi with the Radius.}}

If the Part I correlation length $R_\xi$ is the \emph{radius} of the effective
compactification, then the relevant KK length scale is:
\begin{equation}
    \boxed{
    \ell = 2\pi R_\xi
    }
    \quad \text{[Dc] from geometry}
    \label{eq:ch11_E_ell_derivation}
\end{equation}

This is derived, not postulated:
\begin{itemize}[nosep]
    \item The correlation length $R_\xi$ characterizes the radius of the extra dimension
    \item The KK quantization uses the circumference $L = 2\pi R$
    \item Therefore $\ell = 2\pi R_\xi$
\end{itemize}

\paragraph{Alternative Interpretations and Why They Fail.}

\begin{table}[ht]
\centering
\caption{Alternative $\ell$ interpretations}
\label{tab:ch11_E_alternatives}
\small
\begin{tabular}{p{3cm}ccp{5cm}l}
\toprule
\textbf{Interpretation} & \textbf{$\ell$} & \textbf{Factor} & \textbf{Requires} & \textbf{Status} \\
\midrule
$R_\xi$ is circumference & $R_\xi$ & 1 & Redefine $R_\xi$ as $L$ not $R$ & Non-standard \\
Half-orbifold & $\pi R_\xi$ & $\pi$ & Only use fundamental domain & Inconsistent with Z$_2$ \\
Full solid angle & $4\pi R_\xi$ & $4\pi$ & 3D isotropic measure & Wrong dimension \\
\textbf{Standard circle} & $2\pi R_\xi$ & $2\pi$ & --- & \tagDc{} \\
\bottomrule
\end{tabular}
\end{table}

\paragraph{Track A Verdict.}

\begin{tcolorbox}[colback=green!5, colframe=green!50!black,
    title=\textbf{Track A: $2\pi$ Factor Derivation}]
\textbf{Derived [Dc]:}
\begin{itemize}[nosep]
    \item $R_\xi$ is the radius of the effective compact dimension
    \item KK quantization uses circumference $L = 2\pi R$
    \item Therefore $\ell = 2\pi R_\xi$ with factor $2\pi$
\end{itemize}

\textbf{Negative closure [Dc]:}
\begin{itemize}[nosep]
    \item Factor 1 ($\ell = R_\xi$): requires non-standard definition of $R_\xi$
    \item Factor $\pi$: inconsistent with full orbifold
    \item Factor $4\pi$: confuses 1D circumference with 3D solid angle
\end{itemize}

\textbf{Status:} The $2\pi$ factor is now \tagDc{}, upgraded from \tagP{}.
\end{tcolorbox}

% ==============================================================================
% TRACK B: THE MISSING 0.9003 FACTOR
% ==============================================================================
\subsubsection{Track B: The Missing 0.9003 Factor}
\label{sec:ch11_attemptE_trackB}

With $2\pi\sqrt{2} \approx 8.886$ giving $m_\phi \approx 70$ GeV, we are 12\% low
compared to $M_W \approx 80$ GeV. To hit exactly 80 GeV, we would need an additional
factor:
\begin{equation}
    f_{\text{missing}} = \frac{8}{2\pi\sqrt{2}} = \frac{4}{\pi\sqrt{2}} \approx 0.9003
    \label{eq:ch11_E_missing_factor}
\end{equation}

Alternatively, if the ``true'' geometric factor is exactly 8 (not $2\pi\sqrt{2}$),
then something must provide the 0.9003 correction.

\paragraph{Candidate B1: Orbifold Fundamental Domain.}

On a Z$_2$ orbifold $S^1/\mathbb{Z}_2$, the fundamental domain has length $\ell/2$
instead of $\ell$. This gives:
\begin{equation}
    \ell_{\text{fund}} = \frac{\ell}{2} = \pi R_\xi
    \quad \Rightarrow \quad
    \text{factor } \pi \text{ instead of } 2\pi
    \label{eq:ch11_E_B1}
\end{equation}

However, the KK mass depends on the \emph{quantization condition}, not the fundamental
domain size. The eigenvalue $x_1 = \pi/2$ for Neumann-Neumann already accounts for
the orbifold structure.

\textbf{Status:} Already included in $x_1$; cannot provide additional factor. \tagDc{} (negative)

\paragraph{Candidate B2: Thick-Brane Finite-Width Correction.}

If the brane has finite thickness $\delta$, mode profiles are not delta-function
localized. The overlap integral changes:
\begin{equation}
    I_4 = \int_{-\delta/2}^{+\delta/2} |f_L(\xi)|^4 \, d\xi
    \label{eq:ch11_E_B2_overlap}
\end{equation}

For a smooth profile, this can differ from the thin-brane limit by an $\mathcal{O}(1)$
factor. However:
\begin{itemize}[nosep]
    \item This affects $I_4$ (overlap), not $\ell$ (KK scale)
    \item The OPR-21 BVP is needed to compute this
    \item Cannot predict a specific 0.9003 without solving the BVP
\end{itemize}

\textbf{Status:} Plausible mechanism but not derived; requires BVP. \tagP{}/[OPEN]

\paragraph{Candidate B3: Junction Phase-Space Reduction.}

The Israel junction condition involves matching across the brane. If the brane
carries brane-localized kinetic terms (BKT), the effective propagator is modified:
\begin{equation}
    G_{\text{eff}}(p^2) = \frac{1}{p^2 + m^2 + \kappa p^2}
    = \frac{1}{(1+\kappa)(p^2 + m^2_{\text{eff}})}
    \label{eq:ch11_E_B3_BKT}
\end{equation}

The factor $(1+\kappa)^{-1}$ could contribute. For $(1+\kappa)^{-1} \approx 0.9$:
\begin{equation}
    \kappa \approx 0.11
    \label{eq:ch11_E_B3_kappa}
\end{equation}

This is a mild BKT coefficient. However:
\begin{itemize}[nosep]
    \item $\kappa$ is not derived from EDC parameters
    \item This is effectively another tunable parameter
\end{itemize}

\textbf{Status:} Mechanism exists; parameter $\kappa$ is \tagP{}.

\paragraph{Candidate B4: Brane Curvature Correction.}

If the brane has intrinsic curvature (not flat in the extra dimension), the effective
path length differs from $2\pi R$:
\begin{equation}
    L_{\text{eff}} = 2\pi R \left(1 + \frac{R_{\text{curv}}^2}{R^2} + \ldots \right)
    \label{eq:ch11_E_B4_curvature}
\end{equation}

For this to give a 10\% correction:
\begin{equation}
    \frac{R_{\text{curv}}^2}{R^2} \sim 0.1
    \quad \Rightarrow \quad
    R_{\text{curv}} \sim 0.3 R_\xi
    \label{eq:ch11_E_B4_estimate}
\end{equation}

This would require significant brane curvature at the $R_\xi$ scale.

\textbf{Status:} No evidence for such curvature in Part I. \tagP{}/[OPEN]

\paragraph{Candidate B5: Numerical Coincidence Check.}

We check whether 0.9003 matches any simple EDC-native combination:
\begin{align}
    \frac{4}{\pi\sqrt{2}} &\approx 0.9003 \\
    \frac{2\sqrt{2}}{\pi} &\approx 0.9003 \\
    1 - \frac{1}{10} &= 0.9 \\
    \frac{9}{10} &= 0.9
    \label{eq:ch11_E_B5_numerics}
\end{align}

The factor $4/(\pi\sqrt{2})$ is the \emph{definition} of the residual; it doesn't
help unless we can derive why ``4'' appears (already present in Route E normalization).

\textbf{Status:} No compelling geometric origin for 0.9003 identified. [OPEN]

\paragraph{Track B Summary.}

\begin{table}[ht]
\centering
\caption{Candidates for the missing 0.9003 factor}
\label{tab:ch11_E_B_candidates}
\small
\begin{tabular}{p{4cm}ccp{4cm}l}
\toprule
\textbf{Candidate} & \textbf{Factor} & \textbf{Match?} & \textbf{Mechanism} & \textbf{Status} \\
\midrule
B1: Orbifold domain & 0.5 & No & Fundamental domain & \tagDc{} (neg) \\
B2: Thick-brane overlap & variable & Maybe & BVP needed & \tagP{}/[OPEN] \\
B3: BKT phase-space & $(1+\kappa)^{-1}$ & Yes if $\kappa=0.11$ & New parameter & \tagP{} \\
B4: Brane curvature & variable & Maybe & Large curvature & \tagP{}/[OPEN] \\
B5: Numerics & 0.9003 & Identity & --- & [OPEN] \\
\bottomrule
\end{tabular}
\end{table}

\textbf{Track B Verdict:} No candidate uniquely derives the 0.9003 factor. The residual
remains [OPEN].

% ==============================================================================
% UPDATED VERDICT
% ==============================================================================
\subsubsection{Attempt E: Final Verdict}
\label{sec:ch11_attemptE_verdict}

\begin{tcolorbox}[colback=gray!10, colframe=gray!60!black,
    title=\textbf{OPR-20 Attempt E: Final Assessment}]

\textbf{What Attempt E derived:}

\begin{enumerate}[nosep]
    \item \textbf{Track A ($2\pi$ factor):}
          \begin{itemize}[nosep]
              \item $R_\xi$ is the radius of the compact dimension (from Part I definition)
              \item KK quantization uses circumference $\ell = 2\pi R_\xi$
              \item The $2\pi$ factor is now \tagDc{}, not \tagP{}
              \item Alternative factors (1, $\pi$, $4\pi$) are negatively closed \tagDc{}
          \end{itemize}

    \item \textbf{Track B (0.9003 residual):}
          \begin{itemize}[nosep]
              \item No unique derivation of the missing 0.9003 factor
              \item Candidates exist (BKT, thick-brane, curvature) but all require
                    parameters that are not derived from EDC
              \item The residual remains [OPEN]
          \end{itemize}
\end{enumerate}

\textbf{Implications for the geometric factor:}

\begin{center}
\small
\begin{tabular}{lccl}
\toprule
\textbf{Component} & \textbf{Factor} & \textbf{Tag} & \textbf{Change} \\
\midrule
Circumference ($2\pi$) & 6.28 & \tagDc{} & Upgraded from \tagP{} \\
Mode normalization ($\sqrt{2}$) & 1.41 & \tagDc{} & Unchanged \\
Combined & $2\pi\sqrt{2} = 8.89$ & \tagDc{} & Upgraded \\
\midrule
Missing factor to hit 8 & 0.9003 & [OPEN] & Not derived \\
\bottomrule
\end{tabular}
\end{center}

\textbf{Updated best candidate:}
\begin{equation}
    C_{\text{geom}} = 2\pi\sqrt{2} \approx 8.89 \quad \text{[Dc]}
    \quad \Rightarrow \quad
    m_\phi \approx 70 \text{ GeV}
    \label{eq:ch11_E_final}
\end{equation}

\textbf{12\% residual interpretation:}
\begin{itemize}[nosep]
    \item Could be absorbed into $R_\xi$ uncertainty (Part I estimate is order-of-magnitude)
    \item Could indicate missing physics (BKT, thick-brane corrections)
    \item Could be the ``EDC prediction'': $m_\phi = 70$ GeV, not 80 GeV
\end{itemize}

\textbf{Final status:}
\begin{itemize}[nosep]
    \item \textbf{Partial upgrade:} The $2\pi\sqrt{2}$ factor is now fully \tagDc{}
          (no longer has \tagP{} components)
    \item \textbf{OPR-20:} Remains \textbf{RED-C [Dc]+[OPEN]} because exact factor 8
          and the 12\% residual are not uniquely derived
    \item \textbf{Progress:} The factor is no longer ``arbitrary''---it has geometric
          provenance. The residual is the remaining open problem.
\end{itemize}
\end{tcolorbox}

% ------------------------------------------------------------------------------
\subsubsection{Closure Targets (Post-Attempt E)}
\label{sec:ch11_attemptE_closure}

To upgrade OPR-20 from RED-C to YELLOW:

\begin{enumerate}[nosep]
    \item \textbf{Accept $2\pi\sqrt{2}$ as the answer:}
          Declare $m_\phi \approx 70$ GeV as the EDC prediction and flag the 12\%
          tension with $M_W = 80$ GeV as an open question (potentially testable).

    \item \textbf{Derive the missing 0.9003:}
          Solve the thick-brane BVP (OPR-21) and compute whether the overlap $I_4$
          or effective $x_1$ provides the correction factor.

    \item \textbf{Absorb into $R_\xi$:}
          Refine the Part I estimate of $R_\xi$ with 12\% precision to determine
          whether the discrepancy is within parameter uncertainty.
\end{enumerate}

\begin{tcolorbox}[colback=gray!10, colframe=gray!60!black,
    title=\textbf{Micro-Status (for margins)}]
\textbf{OPR-20 Attempt E:} Track A derives $\ell = 2\pi R_\xi$ \tagDc{}; factor $2\pi$
upgraded from [P]. Track B: 0.9003 residual has candidates but none uniquely derived [OPEN].
Combined factor $2\pi\sqrt{2} \approx 8.89$ \tagDc{}; $m_\phi \approx 70$ GeV.
Status: RED-C [Dc]+[OPEN]; structural progress, residual remains.
\end{tcolorbox}

