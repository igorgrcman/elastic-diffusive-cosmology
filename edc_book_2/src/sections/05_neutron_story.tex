% ==============================================================================
% Section: Neutron as an Excited Junction on a Proton-Anchored Interface
% ==============================================================================

\subsection{Neutron as an Excited Junction on a Proton-Anchored Interface}
\label{sec:neutron_story}

\subsubsection{Why the Proton Anchor Matters for Neutron Physics (Mechanistic Continuity)}

The Proton-Anchor Stability Principle (\S\ref{sec:proton_anchor}) is not a philosophical add-on;
it is the \emph{mechanical prerequisite} for defining any long-lived observer-facing weak process.
If the brane--observer interface were not stabilized by a locally minimizing topological anchor,
then junction excitations would not admit a consistent separation between:
\begin{enumerate}[nosep]
  \item[(i)] a metastable ``ground'' configuration (proton-like anchor),
  \item[(ii)] a nearby excited configuration (neutron-like junction state), and
  \item[(iii)] an observer-facing release map that conserves the ledger.
\end{enumerate}

In EDC language: the proton provides the \emph{stable reference sector} in configuration space
against which neutron decay can be defined as a relaxation process, rather than an arbitrary fitted timer.

\subsubsection{Neutron Ontology: A Bulk-Core Excitation Above the Anchored Junction}

\begin{edcDefinitionBox}{Neutron state as a junction excitation}{[Dc]}
We treat the neutron as a bulk-core junction excitation: a configuration $\Psi_n$ in the same broad
topological family as the anchored Y-junction, but displaced in the bulk-facing degrees of freedom.
The proton corresponds to a locally minimizing configuration $\Psi_p$ (anchor),
while the neutron corresponds to a higher-energy configuration $\Psi_n$ that can relax toward $\Psi_p$.
\end{edcDefinitionBox}

Mechanistically, the decay is not ``a weak vertex'' acting in 3D; it is the \emph{completion of a 5D relaxation}:
a bulk-facing excitation relaxes toward the proton anchor while transferring (pumping) energy into the brane layer,
which then redistributes that energy and finally releases allowed 3D outputs.

\subsubsection{The Three-Phase Pipeline: Absorption $\to$ Dissipation $\to$ Release}

We reuse the unified pipeline established in the weak program:
\begin{equation}
\label{eq:unified_pipeline_neutron}
\text{Bulk relaxation} \;\Rightarrow\;
\underbrace{\Pi_{\mathrm{pump}}}_{\text{absorption into brane}}
\;\Rightarrow\;
\underbrace{\Gamma_{\mathrm{eff}}}_{\text{brane-layer dissipation}}
\;\Rightarrow\;
\underbrace{\mathcal{P}_{\mathrm{frozen}}}_{\text{release as 3D outputs}}.
\end{equation}

\paragraph{Absorption (charging) \tagDc{}.}
During relaxation, the bulk-facing configuration does work on brane-layer degrees of freedom.
In the 1D effective description (junction coordinate $q(t)$), the instantaneous pumping power is
\begin{equation}
\label{eq:pump_power_neutron}
\Pi_{\mathrm{pump}}(t) \;\equiv\; -\dot q(t)\,\partial_q V(q),
\qquad [\Pi_{\mathrm{pump}}]=\text{energy/time}.
\end{equation}
The integrated absorbed energy is then
\begin{equation}
\label{eq:charging_integral_neutron}
\Delta E_{\mathrm{brane}}
\;\equiv\;
\int_{t_0}^{t_\star} \Pi_{\mathrm{pump}}(t)\,dt,
\end{equation}
where $t_\star$ denotes the end of the pumping regime (defined below).

\paragraph{Dissipation (redistribution) \tagDc{}/\tagP{}.}
The brane does not instantly emit 3D particles; instead it redistributes absorbed energy into brane-layer modes.
We parameterize the release-rate scale by an effective dissipation rate $\Gamma_{\mathrm{eff}}$:
\begin{equation}
\label{eq:release_power_neutron}
\Pi_{\mathrm{release}}(t) \;\equiv\; \Gamma_{\mathrm{eff}}\,E_{\mathrm{brane}}(t),
\qquad [\Pi_{\mathrm{release}}]=\text{energy/time}.
\end{equation}
\textbf{Important:} $\Gamma_{\mathrm{eff}}$ is \emph{not} tuned to the neutron lifetime; it is a placeholder until derived from
thick-brane microphysics (open).

\paragraph{Release (observer-facing projection)/\tagDc{}.}
Once the system enters the frozen/output regime, brane-layer modes are projected into \emph{allowed} 3D outputs:
\begin{equation}
\label{eq:release_map_neutron}
\Delta E_{\mathrm{brane}}
\;\xrightarrow{\;\mathcal{P}_{\mathrm{frozen}}\;}
\{e^-,\bar\nu_e\} + \text{(recoil)} + \text{(soft)}.
\end{equation}
This is not creation ``from nothing''; it is an observer-facing boundary projection of permitted outputs
consistent with the ledger and selection rules.

\subsubsection{Trigger Is a Boundary Condition, Not a Timer}

\begin{edcDefinitionBox}{Regime switch / trigger (neutron)}{[Dc]}
We define the end of the pumping regime by a practical asymptotic condition:
\begin{equation}
\label{eq:trigger_neutron}
q(t_\star)\approx 0
\;\land\;
\Xi(t_\star)\ll 1,
\qquad
\Xi(t)\equiv \frac{\Pi_{\mathrm{pump}}(t)}{\Pi_{\mathrm{release}}(t)}.
\end{equation}
Here $q\approx 0$ denotes proximity to the anchored junction geometry (Steiner-like closure),
and $\Xi\ll 1$ encodes that pumping has become negligible relative to brane redistribution/release.
\end{edcDefinitionBox}

\subsubsection{Why Only the Electron Channel Is Allowed (Not ``Forbidden'' by Fiat)}

A common confusion is the sentence ``the muon channel is forbidden''.
What actually happens is \emph{kinematic closure} under the EDC energy ledger.

\begin{edcDefinitionBox}{Kinematic allowance criterion}{[BL]}
In any observer-facing release, a charged lepton $\ell^-$ can appear only if the available Q-value satisfies
\begin{equation}
\label{eq:Q_value_condition}
Q_{n\to p} \;\ge\; m_\ell c^2 \;+\; E_{\mathrm{recoil}} \;+\; E_{\mathrm{soft}},
\end{equation}
where $Q_{n\to p} \equiv (m_n - m_p)c^2$ is the measured Q-value \tagBL{}, and the other terms are non-negative.
\end{edcDefinitionBox}

For neutron decay, the empirical Q-value is small:
\begin{equation}
\label{eq:Q_numeric_neutron}
Q_{n\to p} \approx 1.293~\mathrm{MeV}\quad\text{\tagBL{}},
\end{equation}
whereas the muon rest energy is
\begin{equation}
\label{eq:muon_threshold}
m_\mu c^2 \approx 105.7~\mathrm{MeV}\quad\text{\tagBL{}}.
\end{equation}
Since $Q_{n\to p}\ll m_\mu c^2$, the inequality \eqref{eq:Q_value_condition} cannot be satisfied for $\ell=\mu$,
even before accounting for recoil/soft losses.
Therefore, the muon channel is not ``dynamically suppressed'': it is \emph{kinematically closed}.

In contrast, the electron threshold is compatible with the small Q-value:
$m_e c^2\approx 0.511~\mathrm{MeV}$ \tagBL{}, leaving room for neutrino energy and recoil.

\begin{tcolorbox}[mechanism, title={Kinematic Gate Summary (Neutron)}]
\textbf{Electron channel} \tagBL{}/\tagDc{}:
\[
Q_{n\to p} - m_e c^2 \approx 1.293 - 0.511 = 0.782~\text{MeV} > 0 \quad \Rightarrow \quad \text{OPEN}
\]

\textbf{Muon channel} \tagBL{}/\tagDc{}:
\[
Q_{n\to p} - m_\mu c^2 \approx 1.293 - 105.7 = -104.4~\text{MeV} < 0 \quad \Rightarrow \quad \text{CLOSED}
\]

This is arithmetic, not a law of nature that ``forbids'' muons. The muon simply cannot be produced
because there is not enough energy available.
\end{tcolorbox}

\subsubsection{Neutron Ledger Closure: Where the Energy Goes}

\begin{edcLedgerBox}{Neutron decay bookkeeping}{[Dc]}
The brane receives energy from bulk relaxation (absorption) and must return it via allowed outputs (release):
\begin{equation}
\label{eq:ledger_neutron}
\Delta E_{\mathrm{brane}}
=
Q_{n\to p} - E_{e^-} - E_{\bar\nu_e} - E_{\mathrm{recoil}} - E_{\mathrm{soft}} - E_{\mathrm{bulk,res}}.
\end{equation}
Each term is non-negative except $E_{\mathrm{bulk,res}}$, which parameterizes any residual bulk-side mismatch (open).
Ledger closure means the RHS remains consistent with measured Q and observed spectra, without tuning.
\end{edcLedgerBox}

\subsubsection{Process Diagram: Neutron Decay Mechanism}

\begin{figure}[ht]
\centering
% figures/fig_neutron_process_pipeline.tex
% Neutron decay process pipeline diagram
\begin{tikzpicture}[scale=0.88, transform shape]

% Load styles
% tikz_style_edc.tex — Reusable TikZ styles for EDC papers
% Version 1.0 — 2026-01-20
% Include via: % tikz_style_edc.tex — Reusable TikZ styles for EDC papers
% Version 1.0 — 2026-01-20
% Include via: \input{tikz_style_edc}

% ============================================================
% REQUIRED LIBRARIES (must be loaded in main document)
% ============================================================
% \usetikzlibrary{calc,angles,quotes,decorations.markings,decorations.pathmorphing,positioning}

% ============================================================
% POSITIONING DEFAULTS
% ============================================================
\tikzset{
    % Default node distances for horizontal/vertical layouts
    edc node distance/.style={node distance=1.6cm and 2.0cm},
    % Compact variant for dense diagrams
    edc compact/.style={node distance=1.2cm and 1.5cm},
    % Spread variant for clarity
    edc spread/.style={node distance=2.0cm and 2.5cm},
}

% ============================================================
% COLOR PALETTE (consistent with epistemic tags)
% ============================================================
\definecolor{edcBulk}{RGB}{220,50,50}        % Red tones for bulk/5D
\definecolor{edcBrane}{RGB}{50,150,50}       % Green tones for brane-layer
\definecolor{edcOutput}{RGB}{50,100,200}     % Blue tones for 3D outputs
\definecolor{edcNeutral}{RGB}{100,100,100}   % Gray for neutral/annotations

% ============================================================
% BOX STYLES
% ============================================================
\tikzset{
    % Generic EDC box (base style)
    edc box/.style={
        rectangle,
        draw,
        rounded corners=3pt,
        minimum width=2.2cm,
        minimum height=0.8cm,
        align=center,
        font=\small,
        inner sep=4pt,
    },
    % Bulk-core box (red family)
    bulk box/.style={
        edc box,
        fill=red!10,
        draw=edcBulk!70!black,
        text=black,
    },
    % Brane-layer box (green family)
    brane box/.style={
        edc box,
        fill=green!10,
        draw=edcBrane!70!black,
        text=black,
    },
    % 3D output box (blue family)
    output box/.style={
        edc box,
        fill=blue!10,
        draw=edcOutput!70!black,
        text=black,
    },
    % Neutral/process box
    process box/.style={
        edc box,
        fill=gray!10,
        draw=gray!60!black,
        text=black,
    },
    % Label-only box (no background)
    label box/.style={
        rectangle,
        rounded corners=2pt,
        draw=gray!40,
        fill=white,
        inner sep=2pt,
        font=\scriptsize,
    },
}

% ============================================================
% ARROW STYLES
% ============================================================
\tikzset{
    % Standard thick arrow
    edc arrow/.style={
        ->,
        >=stealth,
        thick,
    },
    % Emphasized arrow (for main flow)
    edc flow/.style={
        ->,
        >=stealth,
        very thick,
        line width=1.2pt,
    },
    % Dashed arrow (for optional/weak connections)
    edc dashed/.style={
        ->,
        >=stealth,
        thick,
        dashed,
    },
    % Double arrow (for bidirectional)
    edc bidir/.style={
        <->,
        >=stealth,
        thick,
    },
}

% ============================================================
% REGION STYLES (for background fills)
% ============================================================
\tikzset{
    % Bulk region (5D)
    bulk region/.style={
        fill=blue!8,
    },
    % Brane layer region
    brane region/.style={
        fill=yellow!25,
    },
    % Observer/3D region
    observer region/.style={
        fill=green!8,
    },
}

% ============================================================
% LABEL STYLES
% ============================================================
\tikzset{
    % Phase label (below nodes)
    phase label/.style={
        font=\scriptsize\itshape,
        text=black!70,
    },
    % Equation label (for inline math)
    eq label/.style={
        font=\scriptsize,
        fill=white,
        inner sep=1pt,
    },
    % Section annotation
    section label/.style={
        font=\footnotesize\bfseries,
        text=black,
    },
}

% ============================================================
% JUNCTION/PARTICLE STYLES
% ============================================================
\tikzset{
    % Y-junction point
    junction point/.style={
        circle,
        fill=red!60!black,
        minimum size=4pt,
        inner sep=0pt,
    },
    % Flux tube arm
    flux arm/.style={
        thick,
        blue!60!black,
    },
    % Particle dot (electron, etc.)
    particle/.style={
        circle,
        fill=black,
        minimum size=5pt,
        inner sep=0pt,
    },
    % Neutrino (smaller, gray)
    neutrino/.style={
        circle,
        fill=gray,
        minimum size=4pt,
        inner sep=0pt,
    },
}

% ============================================================
% SPRING DECORATION (for mechanical models)
% ============================================================
\tikzset{
    spring/.style={
        thick,
        decorate,
        decoration={
            coil,
            aspect=0.5,
            segment length=2mm,
            amplitude=2mm,
        },
    },
    % Wave decoration (for field modes)
    wave field/.style={
        thick,
        decorate,
        decoration={
            snake,
            amplitude=2pt,
            segment length=8pt,
        },
    },
}

% ============================================================
% BOUNDARY STYLES
% ============================================================
\tikzset{
    % Bulk-facing boundary (dashed red)
    bulk boundary/.style={
        very thick,
        red!70!black,
        dashed,
    },
    % Observer-facing boundary (solid green)
    observer boundary/.style={
        thick,
        green!50!black,
    },
    % Brane edge (orange)
    brane edge/.style={
        thick,
        orange!70!black,
    },
}

% ============================================================
% CONVENIENCE COMMANDS
% ============================================================
% Arrow label (above)
\newcommand{\arrlabel}[1]{\scriptsize #1}
% Arrow label (below)
\newcommand{\arrlabelb}[1]{\scriptsize #1}

% ============================================================
% END OF STYLE FILE
% ============================================================


% ============================================================
% REQUIRED LIBRARIES (must be loaded in main document)
% ============================================================
% \usetikzlibrary{calc,angles,quotes,decorations.markings,decorations.pathmorphing,positioning}

% ============================================================
% POSITIONING DEFAULTS
% ============================================================
\tikzset{
    % Default node distances for horizontal/vertical layouts
    edc node distance/.style={node distance=1.6cm and 2.0cm},
    % Compact variant for dense diagrams
    edc compact/.style={node distance=1.2cm and 1.5cm},
    % Spread variant for clarity
    edc spread/.style={node distance=2.0cm and 2.5cm},
}

% ============================================================
% COLOR PALETTE (consistent with epistemic tags)
% ============================================================
\definecolor{edcBulk}{RGB}{220,50,50}        % Red tones for bulk/5D
\definecolor{edcBrane}{RGB}{50,150,50}       % Green tones for brane-layer
\definecolor{edcOutput}{RGB}{50,100,200}     % Blue tones for 3D outputs
\definecolor{edcNeutral}{RGB}{100,100,100}   % Gray for neutral/annotations

% ============================================================
% BOX STYLES
% ============================================================
\tikzset{
    % Generic EDC box (base style)
    edc box/.style={
        rectangle,
        draw,
        rounded corners=3pt,
        minimum width=2.2cm,
        minimum height=0.8cm,
        align=center,
        font=\small,
        inner sep=4pt,
    },
    % Bulk-core box (red family)
    bulk box/.style={
        edc box,
        fill=red!10,
        draw=edcBulk!70!black,
        text=black,
    },
    % Brane-layer box (green family)
    brane box/.style={
        edc box,
        fill=green!10,
        draw=edcBrane!70!black,
        text=black,
    },
    % 3D output box (blue family)
    output box/.style={
        edc box,
        fill=blue!10,
        draw=edcOutput!70!black,
        text=black,
    },
    % Neutral/process box
    process box/.style={
        edc box,
        fill=gray!10,
        draw=gray!60!black,
        text=black,
    },
    % Label-only box (no background)
    label box/.style={
        rectangle,
        rounded corners=2pt,
        draw=gray!40,
        fill=white,
        inner sep=2pt,
        font=\scriptsize,
    },
}

% ============================================================
% ARROW STYLES
% ============================================================
\tikzset{
    % Standard thick arrow
    edc arrow/.style={
        ->,
        >=stealth,
        thick,
    },
    % Emphasized arrow (for main flow)
    edc flow/.style={
        ->,
        >=stealth,
        very thick,
        line width=1.2pt,
    },
    % Dashed arrow (for optional/weak connections)
    edc dashed/.style={
        ->,
        >=stealth,
        thick,
        dashed,
    },
    % Double arrow (for bidirectional)
    edc bidir/.style={
        <->,
        >=stealth,
        thick,
    },
}

% ============================================================
% REGION STYLES (for background fills)
% ============================================================
\tikzset{
    % Bulk region (5D)
    bulk region/.style={
        fill=blue!8,
    },
    % Brane layer region
    brane region/.style={
        fill=yellow!25,
    },
    % Observer/3D region
    observer region/.style={
        fill=green!8,
    },
}

% ============================================================
% LABEL STYLES
% ============================================================
\tikzset{
    % Phase label (below nodes)
    phase label/.style={
        font=\scriptsize\itshape,
        text=black!70,
    },
    % Equation label (for inline math)
    eq label/.style={
        font=\scriptsize,
        fill=white,
        inner sep=1pt,
    },
    % Section annotation
    section label/.style={
        font=\footnotesize\bfseries,
        text=black,
    },
}

% ============================================================
% JUNCTION/PARTICLE STYLES
% ============================================================
\tikzset{
    % Y-junction point
    junction point/.style={
        circle,
        fill=red!60!black,
        minimum size=4pt,
        inner sep=0pt,
    },
    % Flux tube arm
    flux arm/.style={
        thick,
        blue!60!black,
    },
    % Particle dot (electron, etc.)
    particle/.style={
        circle,
        fill=black,
        minimum size=5pt,
        inner sep=0pt,
    },
    % Neutrino (smaller, gray)
    neutrino/.style={
        circle,
        fill=gray,
        minimum size=4pt,
        inner sep=0pt,
    },
}

% ============================================================
% SPRING DECORATION (for mechanical models)
% ============================================================
\tikzset{
    spring/.style={
        thick,
        decorate,
        decoration={
            coil,
            aspect=0.5,
            segment length=2mm,
            amplitude=2mm,
        },
    },
    % Wave decoration (for field modes)
    wave field/.style={
        thick,
        decorate,
        decoration={
            snake,
            amplitude=2pt,
            segment length=8pt,
        },
    },
}

% ============================================================
% BOUNDARY STYLES
% ============================================================
\tikzset{
    % Bulk-facing boundary (dashed red)
    bulk boundary/.style={
        very thick,
        red!70!black,
        dashed,
    },
    % Observer-facing boundary (solid green)
    observer boundary/.style={
        thick,
        green!50!black,
    },
    % Brane edge (orange)
    brane edge/.style={
        thick,
        orange!70!black,
    },
}

% ============================================================
% CONVENIENCE COMMANDS
% ============================================================
% Arrow label (above)
\newcommand{\arrlabel}[1]{\scriptsize #1}
% Arrow label (below)
\newcommand{\arrlabelb}[1]{\scriptsize #1}

% ============================================================
% END OF STYLE FILE
% ============================================================


% ─────────────────────────────────────────────────────────────────────────────
% Background regions
% ─────────────────────────────────────────────────────────────────────────────
\fill[gray!12] (-5.5,2.0) rectangle (5.5,3.5);
\fill[blue!8] (-5.5,0.3) rectangle (5.5,2.0);
\fill[green!8] (-5.5,-1.2) rectangle (5.5,0.3);

% Region labels
\node[font=\scriptsize, gray!70!black] at (-4.8,3.2) {Bulk / Plenum};
\node[font=\scriptsize, blue!50!black] at (-4.8,1.7) {Thick brane layer};
\node[font=\scriptsize, green!50!black] at (-4.8,0.0) {3D outputs};

% ─────────────────────────────────────────────────────────────────────────────
% Bulk layer nodes
% ─────────────────────────────────────────────────────────────────────────────
\node[bulk box, text width=2.4cm] (nexc) at (-3.5,2.7)
  {Neutron $\Psi_n$\\{\tiny bulk-core junction}};

\node[bulk box, text width=2.4cm] (relax) at (0,2.7)
  {Relaxation\\{\tiny $q(t)\to 0$}};

\node[output box, text width=2.4cm] (anchor) at (3.5,2.7)
  {Proton $\Psi_p$\\{\tiny anchor (minimum)}};

% ─────────────────────────────────────────────────────────────────────────────
% Brane layer nodes
% ─────────────────────────────────────────────────────────────────────────────
\node[brane box, text width=2.4cm] (pump) at (-3.5,1.15)
  {Absorption\\{\tiny $\Pi_{\mathrm{pump}}$}};

\node[brane box, text width=2.4cm] (diss) at (0,1.15)
  {Dissipation\\{\tiny $\Gamma_{\mathrm{eff}} E_{\mathrm{brane}}$}};

\node[gate box, text width=2.4cm, minimum height=0.8cm] (frozen) at (3.5,1.15)
  {$\mathcal{P}_{\mathrm{frozen}}$\\{\tiny release gate}};

% ─────────────────────────────────────────────────────────────────────────────
% Output layer nodes
% ─────────────────────────────────────────────────────────────────────────────
\node[output box, text width=1.8cm] (eout) at (1.5,-0.45)
  {$e^-$\\{\tiny charged}};

\node[output box, text width=1.8cm] (nuout) at (3.5,-0.45)
  {$\bar\nu_e$\\{\tiny ledger}};

\node[rectangle, draw=gray!50, dashed, rounded corners=2pt,
      fill=gray!5, text width=1.8cm, font=\tiny, align=center] (recoil) at (5.2,-0.45)
  {recoil/soft};

% ─────────────────────────────────────────────────────────────────────────────
% Arrows - bulk flow
% ─────────────────────────────────────────────────────────────────────────────
\draw[edc flow] (nexc) -- (relax);
\draw[edc flow] (relax) -- (anchor);

% Arrow from relaxation down to pump (energy transfer)
\draw[edc arrow] (relax.south) -- (pump.north);

% Brane flow
\draw[edc flow] (pump) -- (diss);
\draw[edc flow] (diss) -- (frozen);

% Release to outputs
\draw[edc arrow] (frozen.south) -- ++(0,-0.3) -| (eout.north);
\draw[edc arrow] (frozen.south) -- ++(0,-0.3) -| (nuout.north);
\draw[edc dashed] (frozen.south) -- ++(0,-0.3) -| (recoil.north);

% ─────────────────────────────────────────────────────────────────────────────
% Annotations
% ─────────────────────────────────────────────────────────────────────────────

% Trigger condition
\node[rectangle, draw=purple!40, fill=purple!5, rounded corners=2pt,
      font=\tiny, align=center, text width=3.2cm] at (0,0.35)
  {Trigger: $q(t_\star)\approx 0$, $\Xi\ll 1$};

% Kinematic gate annotation
\node[rectangle, draw=red!40, fill=red!5, rounded corners=2pt,
      font=\tiny, align=left, text width=4.5cm] at (-3.0,-0.5)
  {\textbf{Kinematic gate:}\\
   $Q_{n\to p} = 1.29$ MeV\\
   $\Rightarrow$ $e^-$ allowed ($m_e = 0.51$ MeV)\\
   $\Rightarrow$ $\mu^-$ closed ($m_\mu = 106$ MeV)};

\end{tikzpicture}

\caption{\textbf{Mechanistic narrative of neutron decay in EDC.}
The proton anchor provides the stable endpoint; the neutron is an excited bulk-core junction.
Relaxation pumps energy into the brane (absorption), the brane redistributes it (dissipation),
and the frozen observer-facing projection releases allowed 3D outputs (electron + antineutrino + recoil/soft)
consistent with the ledger. The muon channel is kinematically closed because $Q_{n\to p} \ll m_\mu c^2$.}
\label{fig:neutron_process_pipeline}
\end{figure}

\subsubsection{OPEN Items Needed for Full Proof-Grade Closure}

This section provides a mechanism-complete narrative, but the following remain open:
\begin{itemize}[nosep]
\item \textbf{OPEN-N(geom):} Explicit 5D functional $\mathcal{E}[\Psi]$ and BCs specialized to the neutron excitation.
\item \textbf{OPEN-N(modes):} Brane-layer mode spectrum and quantitative $\Gamma_{\mathrm{eff}}$ from microphysics.
\item \textbf{OPEN-N(map):} Evaluation of $\mathcal{P}_{\mathrm{frozen}}$ including $\mathcal{P}_{\mathrm{chir}}$ contribution.
\end{itemize}

\subsubsection{Falsifiability Hooks}

\begin{tcolorbox}[falsifiability]
\begin{itemize}[nosep]
  \item If the neutron decay spectrum is inconsistent with a relaxation-to-anchor mechanism, the model fails.
  \item If the lifetime cannot be connected to $\Gamma_{\mathrm{eff}}$ without ad hoc tuning, the framework
        is incomplete.
  \item If a muon is observed in neutron decay (at rates incompatible with higher-order processes),
        the kinematic gate interpretation fails.
  \item If the ledger does not close (missing energy beyond experimental uncertainty), the mechanism fails.
\end{itemize}
\end{tcolorbox}

