% ch20_epistemic_summary_closure_status.tex
% CH20: Epistemic Summary & Closure Status
% Purpose: Reader's Checkpoint and Zenodo-ready epistemic ledger
% Status: META (summary/index, not new derivation)
% Created: 2026-01-26
% Scope guard: This chapter summarizes existing results; it does NOT introduce new physics.

\chapter{Epistemic Summary \& Closure Status}
\label{ch:ch20_epistemic_summary}

%%%%%%%%%%%%%%%%%%%%%%%%%%%%%%%%%%%%%%%%%%%%%%%%%%%%%%%%%%%%%%%%%%%%%%%%%%%%%%%
% SCOPE GUARD BOX
%%%%%%%%%%%%%%%%%%%%%%%%%%%%%%%%%%%%%%%%%%%%%%%%%%%%%%%%%%%%%%%%%%%%%%%%%%%%%%%

\begin{tcolorbox}[colback=yellow!5!white, colframe=yellow!70!black,
    title=\textbf{Scope Guard: What This Chapter Is (and Is Not)}]

\textbf{Purpose}: This chapter is a \emph{summary}, \emph{index}, and \emph{navigation aid}
for Part~II. It consolidates the epistemic status of all claims into a single checkpoint.

\textbf{What this chapter does}:
\begin{itemize}[nosep]
    \item Summarizes parameter inputs and their status ([P], [Dc], [M], [BL])
    \item Maps the dependency chain from primitives to derived outputs
    \item Lists canonical results with pointers to their derivation locations
    \item Catalogues open problems with priority and blocking status
    \item Clarifies what Part~II claims and what it does \emph{not} claim
\end{itemize}

\textbf{What this chapter does NOT do}:
\begin{itemize}[nosep]
    \item Introduce new physical assumptions or postulates
    \item Derive new results or modify existing derivations
    \item Use any Standard Model observable as a derivation input
\end{itemize}

\textbf{Audit evidence}: \texttt{audit/evidence/CH20\_EPISTEMIC\_SUMMARY\_AUDIT.md}
\end{tcolorbox}

%%%%%%%%%%%%%%%%%%%%%%%%%%%%%%%%%%%%%%%%%%%%%%%%%%%%%%%%%%%%%%%%%%%%%%%%%%%%%%%
% SECTION 20.1: READER'S MAP
%%%%%%%%%%%%%%%%%%%%%%%%%%%%%%%%%%%%%%%%%%%%%%%%%%%%%%%%%%%%%%%%%%%%%%%%%%%%%%%

\section{Reader's Map}
\label{sec:ch20_readers_map}

\begin{tcolorbox}[colback=green!5!white, colframe=green!50!black,
    title=\textbf{Quick Navigation: The Green Path}]

\textbf{If you want to understand the core derivation chain with minimal detours}:

\medskip
\begin{center}
\begin{tikzpicture}[node distance=0.8cm, auto,
    block/.style={rectangle, draw, fill=green!10, text width=3cm, text centered, rounded corners, minimum height=1cm},
    arrow/.style={->, thick, >=stealth}]

    \node [block] (ch14) {Ch14: BVP + Robin BC\\Box~14.2};
    \node [block, right=of ch14] (ch17) {Ch17: OPR-19\\$g_5 \to g_4$};
    \node [block, right=of ch17] (ch18) {Ch18: OPR-20\\$m_1 = x_1/\ell$};
    \node [block, right=of ch18] (ch19) {Ch19: OPR-22\\$G_{\text{eff}}$};

    \draw [arrow] (ch14) -- (ch17);
    \draw [arrow] (ch17) -- (ch18);
    \draw [arrow] (ch18) -- (ch19);
\end{tikzpicture}
\end{center}

\medskip
\textbf{Key distinction}:
\begin{itemize}[nosep]
    \item \colorbox{green!15}{\textbf{Physical (canonical)}}: Domain wall $V_L = M^2 - M'$, $\mu$-window $[13, 17]$
    \item \textbf{Toy benchmark}: P\"oschl--Teller $V = -V_0\,\text{sech}^2$, $\mu$-window $[15, 18]$
\end{itemize}

\textbf{The physical path is canonical}. Toy benchmarks are for code validation only.

See Box~14.1 (\texttt{\textbackslash ref\{box:ch14\_canonical\_physical\_path\}}) for the full reader route map.
\end{tcolorbox}

%%%%%%%%%%%%%%%%%%%%%%%%%%%%%%%%%%%%%%%%%%%%%%%%%%%%%%%%%%%%%%%%%%%%%%%%%%%%%%%
% SECTION 20.2: PARAMETER LEDGER
%%%%%%%%%%%%%%%%%%%%%%%%%%%%%%%%%%%%%%%%%%%%%%%%%%%%%%%%%%%%%%%%%%%%%%%%%%%%%%%

\section{Parameter Ledger}
\label{sec:ch20_parameter_ledger}

\begin{tcolorbox}[colback=blue!5!white, colframe=blue!50!black,
    title=\textbf{Primitive Inputs: What Enters the Framework}]

The following parameters are \textbf{inputs} to the Part~II derivation chain.
None are derived from Standard Model observables.

\medskip
\begin{center}
\renewcommand{\arraystretch}{1.4}
\begin{tabular}{llp{4cm}l}
\toprule
\textbf{Symbol} & \textbf{Status} & \textbf{Physical Meaning} & \textbf{First Appears} \\
\midrule
$g_5$ & \tagP{} & 5D gauge coupling & Ch17 (OPR-19) \\
$\sigma$ & \tagP{} & Membrane tension & Ch15 (OPR-01) \\
$\Delta$ & \tagP{} & Domain-wall kink width & Ch16 (OPR-04) \\
$\ell$ & \tagP{} & Sturm--Liouville domain size & Ch14 (OPR-21) \\
$\rho = \Delta/\ell$ & \tagP{} & Wall-to-domain ratio & Ch14 Box~14.2 \\
$\hat\kappa$ & \tagP{} & Robin BC parameter (dimensionless) & Ch14 \S14.4 \\
$y$ & \tagP{} & Yukawa coupling (domain wall) & Ch15 (OPR-01) \\
\bottomrule
\end{tabular}
\end{center}

\medskip
\textbf{Key relations}:
\begin{itemize}[nosep]
    \item $\mu = M_0 \ell$ is the dimensionless control parameter (OPR-21)
    \item $M_0 = (\sqrt{3}/2)\, y\, \sqrt{\sigma\Delta}$ from OPR-01 \tagDc{}
    \item The $\mu$-window for $N_{\text{bound}} = 3$ is \textbf{shape-dependent}: $\mu_3(V, \hat\kappa, \rho)$
\end{itemize}

\end{tcolorbox}

\paragraph{Scale Taxonomy Reference.}
The distinction between $\Delta$ (kink width), $\delta$ (boundary-layer scale), $\ell$ (domain size),
and $R_\xi$ (diffusion scale) is critical. See Chapter~16, \S\ref{sec:ch16_reader_map} for the
canonical Scale Taxonomy.

\begin{tcolorbox}[colback=red!5!white, colframe=red!50!black,
    title=\textbf{Common Pitfall: Do Not Confuse These Scales}]
\begin{center}
\begin{tabular}{lll}
\toprule
\textbf{Symbol} & \textbf{Role} & \textbf{Typical Order} \\
\midrule
$\Delta$ & Scalar kink width & Profile microphysics \\
$\delta$ & Boundary-layer / Robin BC & Transport regularization \\
$\ell$ & BVP domain & Sturm--Liouville interval \\
$R_\xi$ & Diffusion / correlation & $\sim \hbar c / M_Z$ \\
\bottomrule
\end{tabular}
\end{center}
Under assumptions (A1)--(A3), these may be related, but \textbf{no such relation is assumed by default}.
See OPR-04 for the conditional tension analysis.
\end{tcolorbox}

%%%%%%%%%%%%%%%%%%%%%%%%%%%%%%%%%%%%%%%%%%%%%%%%%%%%%%%%%%%%%%%%%%%%%%%%%%%%%%%
% SECTION 20.3: DEPENDENCY GRAPH
%%%%%%%%%%%%%%%%%%%%%%%%%%%%%%%%%%%%%%%%%%%%%%%%%%%%%%%%%%%%%%%%%%%%%%%%%%%%%%%

\section{Dependency Graph}
\label{sec:ch20_dependency_graph}

The derivation chain flows from [P] primitives through conditional [Dc] results to the
effective weak-sector outputs.

\begin{center}
\begin{tikzpicture}[node distance=1.2cm, auto,
    prim/.style={rectangle, draw, fill=yellow!20, text width=2.5cm, text centered, rounded corners, minimum height=0.8cm, font=\small},
    deriv/.style={rectangle, draw, fill=green!20, text width=3cm, text centered, rounded corners, minimum height=0.8cm, font=\small},
    output/.style={rectangle, draw, fill=blue!20, text width=3cm, text centered, rounded corners, minimum height=0.8cm, font=\small},
    arrow/.style={->, thick, >=stealth}]

    % Primitives
    \node [prim] (g5) {$g_5$ [P]};
    \node [prim, below=0.5cm of g5] (sigma) {$\sigma, \Delta, y$ [P]};
    \node [prim, below=0.5cm of sigma] (ell) {$\ell, \rho, \hat\kappa$ [P]};

    % Derived intermediates
    \node [deriv, right=2cm of g5] (opr19) {OPR-19\\$g_{4,n} = g_5 f_n(0)$};
    \node [deriv, right=2cm of sigma] (opr01) {OPR-01\\$M_0 = f(\sigma,\Delta,y)$};
    \node [deriv, right=2cm of ell] (opr21) {OPR-21\\$V_L, N_{\text{bound}}, f_n(\xi)$};

    % More derived
    \node [deriv, right=2cm of opr19] (opr20) {OPR-20\\$m_1 = x_1/\ell$};
    \node [deriv, below=0.3cm of opr20] (f10) {$|f_1(0)|^2$\\from BVP};

    % Output
    \node [output, right=2cm of opr20] (geff) {OPR-22\\$G_{\text{eff}} = \frac{g_5^2 \ell |f_1(0)|^2}{2 x_1^2}$};

    % Arrows
    \draw [arrow] (g5) -- (opr19);
    \draw [arrow] (sigma) -- (opr01);
    \draw [arrow] (ell) -- (opr21);
    \draw [arrow] (opr01) -- (opr21);
    \draw [arrow] (opr19) -- (opr20);
    \draw [arrow] (opr21) -- (opr20);
    \draw [arrow] (opr21) -- (f10);
    \draw [arrow] (opr20) -- (geff);
    \draw [arrow] (f10) -- (geff);
    \draw [arrow] (opr19) to[bend left=20] (geff);
\end{tikzpicture}
\end{center}

\paragraph{Chain Status Summary.}
\begin{itemize}[nosep]
    \item \textbf{OPR-19} (Ch17): $g_5 \to g_4$ reduction --- CONDITIONAL [Dc]
    \item \textbf{OPR-20} (Ch18): $m_1 = x_1/\ell$ eigenvalue structure --- CONDITIONAL [Dc]
    \item \textbf{OPR-21} (Ch14): BVP mode profiles, $N_{\text{bound}}$ --- CONDITIONAL [Dc]
    \item \textbf{OPR-22} (Ch19): $G_{\text{eff}}$ from mode overlap --- CONDITIONAL [Dc]
    \item \textbf{OPR-01} (Ch15): $M_0 = f(\sigma, \Delta, y)$ anchor --- CONDITIONAL [Dc]
\end{itemize}

All results are \tagDc{} (conditional on [P] primitives), not \tagDer{} (unconditionally derived).

%%%%%%%%%%%%%%%%%%%%%%%%%%%%%%%%%%%%%%%%%%%%%%%%%%%%%%%%%%%%%%%%%%%%%%%%%%%%%%%
% SECTION 20.4: CANONICAL RESULTS SUMMARY
%%%%%%%%%%%%%%%%%%%%%%%%%%%%%%%%%%%%%%%%%%%%%%%%%%%%%%%%%%%%%%%%%%%%%%%%%%%%%%%

\section{Canonical Results Summary}
\label{sec:ch20_canonical_results}

\begin{tcolorbox}[colback=green!5!white, colframe=green!50!black,
    title=\textbf{Key Outputs (Derived Conditional on [P] Inputs)}]

\begin{center}
\renewcommand{\arraystretch}{1.4}
\begin{tabular}{p{4cm}lp{5cm}}
\toprule
\textbf{Result} & \textbf{Status} & \textbf{Reference} \\
\midrule
$g_{4,n} = g_5 f_n(0)$ & \tagDc{} & Ch17, Eq.~(17.x), OPR-19 \\
$m_n = x_n / \ell$ & \tagDc{} & Ch18, Eq.~(18.x), OPR-20 \\
$x_n = x_n(\hat\kappa, V)$ & \tagDc{} & BVP eigenvalue (shape-dependent) \\
$C_{\text{eff}} = g_5^2 \ell / x_1^2$ & \tagDc{} & Ch18, Eq.~(18.x), OPR-20 \\
$G_{\text{eff}} = \frac{1}{2} C_{\text{eff}} |f_1(0)|^2$ & \tagDc{} & Ch19, Eq.~(19.x), OPR-22 \\
$M_0^2 = \frac{3y^2}{4} \sigma \Delta$ & \tagDc{} & Ch15, Eq.~(15.x), OPR-01 \\
$V_L = M^2 - M'$ (physical) & \tagDc{} & Ch14, from 5D Dirac \\
Robin BC: $f'(0) + \kappa f(0) = 0$ & \tagDc{} & Ch14, from Israel junction \\
\bottomrule
\end{tabular}
\end{center}

\end{tcolorbox}

\paragraph{Green Path Family (Physical BC).}
From OPEN-22-4b-R-PHYS (Box~14.2):

\begin{center}
\renewcommand{\arraystretch}{1.3}
\begin{tabular}{llcc}
\toprule
\textbf{Path} & \textbf{BC Type} & \textbf{$N_{\text{bound}}=3$ Window} & \textbf{Status} \\
\midrule
\rowcolor{green!15}
Green-A & Neumann ($\hat\kappa = 0$) & $\mu \in [13.0, 15.6]$ & CANONICAL \\
\rowcolor{green!10}
Green-B & Robin ($\hat\kappa = 0.5$) & $\mu \in [13.0, 15.2]$ & CANONICAL \\
\rowcolor{green!10}
Green-B & Robin ($\hat\kappa = 1.0$) & $\mu \in [13.0, 14.8]$ & CANONICAL \\
\rowcolor{green!10}
Green-B & Robin ($\hat\kappa = 2.0$) & $\mu \in [13.0, 13.6]$ & CANONICAL (narrow) \\
\bottomrule
\end{tabular}
\end{center}

\textbf{Scope guard}: Cross-$\hat\kappa$ comparisons are only valid within the common $\mu$-overlap
and at fixed $\rho$. The $\mu$-window \textbf{narrows} with increasing $\hat\kappa$.

%%%%%%%%%%%%%%%%%%%%%%%%%%%%%%%%%%%%%%%%%%%%%%%%%%%%%%%%%%%%%%%%%%%%%%%%%%%%%%%
% SECTION 20.5: OPEN PROBLEMS REGISTER
%%%%%%%%%%%%%%%%%%%%%%%%%%%%%%%%%%%%%%%%%%%%%%%%%%%%%%%%%%%%%%%%%%%%%%%%%%%%%%%

\section{Open Problems Register (Snapshot)}
\label{sec:ch20_open_problems}

\begin{tcolorbox}[colback=red!5!white, colframe=red!50!black,
    title=\textbf{Blocking for ``No [P]'' Book (Part III Campaign)}]

These items must be resolved to upgrade Part~II from [Dc] to [Der]:

\begin{enumerate}[nosep]
    \item \textbf{Derive $g_5$} from UV completion or 5D action normalization (OPEN-22-2)
    \item \textbf{Derive $\sigma$} from cosmological/gravitational first principles
    \item \textbf{Derive $\Delta, \ell, \rho$} from junction stability or brane microphysics (OPR-04)
    \item \textbf{Derive $y$} from gauge embedding or naturalness
    \item \textbf{Derive $V(\xi)$} entirely from 5D action (not ansatz)
\end{enumerate}

\textbf{Status}: All above are \textbf{Part III scope}---not addressed in this book.
\end{tcolorbox}

\begin{tcolorbox}[colback=yellow!5!white, colframe=yellow!50!black,
    title=\textbf{Non-Blocking for Part II Zenodo (Scope-Guarded)}]

These items are documented as open but do not block Part~II publication:

\begin{itemize}[nosep]
    \item \textbf{OPR-03}: $\pi_1(M^5) = \mathbb{Z}_3$ topology closure
    \item \textbf{OPR-13}: PMNS angles ($\theta_{23}$ derived, $\theta_{12}/\theta_{13}$ identified)
    \item \textbf{OPR-14}: CP phase $\delta$ derivation
    \item \textbf{OPR-15}: Dirac/Majorana determination
    \item \textbf{OPR-16}: Pion mass/lifetime
    \item \textbf{OPR-17}: SU(2)$_L$ gauge embedding
    \item \textbf{OPR-18}: CKM/PMNS from overlaps
\end{itemize}

Full registry: \texttt{canon/opr/OPR\_REGISTRY.md}
\end{tcolorbox}

%%%%%%%%%%%%%%%%%%%%%%%%%%%%%%%%%%%%%%%%%%%%%%%%%%%%%%%%%%%%%%%%%%%%%%%%%%%%%%%
% SECTION 20.6: WHAT THIS BOOK CLAIMS
%%%%%%%%%%%%%%%%%%%%%%%%%%%%%%%%%%%%%%%%%%%%%%%%%%%%%%%%%%%%%%%%%%%%%%%%%%%%%%%

\section{What This Book Claims (and What It Does Not)}
\label{sec:ch20_claims}

\begin{center}
\renewcommand{\arraystretch}{1.4}
\begin{tabular}{p{6.5cm}|p{6.5cm}}
\toprule
\textbf{Claims Within Scope (Part II)} & \textbf{Not Claimed Here} \\
\midrule
EFT structure: $G_{\text{eff}}$ from 5D mediator exchange [Dc] &
Baryogenesis / Sakharov conditions \\[4pt]

V--A chirality from geometric asymmetry [Dc] &
Matter--antimatter asymmetry origin \\[4pt]

$N_{\text{bound}} = 3$ as geometric possibility [Dc] &
Why parameters fall in 3-gen window \\[4pt]

Robin BC structure from Israel junction [Dc] &
Actual value of $\kappa$ from microphysics \\[4pt]

$M_0 = f(\sigma, \Delta, y)$ from kink theory [Dc] &
First-principles derivation of $\sigma$ \\[4pt]

$\sin^2\theta_W = 1/4$ from $\mathbb{Z}_6$ counting [Der] &
CP violation mechanism \\[4pt]

Shape-dependent $\mu$-windows for $N_{\text{bound}}$ [Dc] &
Inflow/outflow thermodynamics (Part I) \\[4pt]

Green Path family (Neumann + Robin) [Dc] &
Neutrino mass absolute scale \\
\bottomrule
\end{tabular}
\end{center}

\begin{tcolorbox}[colback=gray!5!white, colframe=gray!50!black,
    title=\textbf{Explicit Non-Treatment Notice}]
The following topics are \textbf{not treated} in Part~II and should not be inferred from any statement herein:
\begin{itemize}[nosep]
    \item \textbf{Baryogenesis}: No Sakharov conditions analysis
    \item \textbf{Antimatter}: No matter--antimatter asymmetry mechanism
    \item \textbf{Dark matter}: No Plenum-based DM candidate
    \item \textbf{Quantum gravity}: No graviton KK modes or gravitational corrections
\end{itemize}
These topics may be addressed in future work (Part~III or beyond).
\end{tcolorbox}

%%%%%%%%%%%%%%%%%%%%%%%%%%%%%%%%%%%%%%%%%%%%%%%%%%%%%%%%%%%%%%%%%%%%%%%%%%%%%%%
% SECTION 20.7: REPRO & AUDIT POINTERS
%%%%%%%%%%%%%%%%%%%%%%%%%%%%%%%%%%%%%%%%%%%%%%%%%%%%%%%%%%%%%%%%%%%%%%%%%%%%%%%

\section{Reproducibility \& Audit Pointers}
\label{sec:ch20_repro}

\paragraph{Key Sanity Scripts.}
The following scripts reproduce numerical results cited in Part~II:

\begin{center}
\renewcommand{\arraystretch}{1.3}
\begin{tabular}{lp{6cm}}
\toprule
\textbf{Script} & \textbf{Validates} \\
\midrule
\texttt{code/opr19\_g5\_sanity.py} & OPR-19 dimensional consistency \\
\texttt{code/opr20\_mediator\_mass\_sanity.py} & OPR-20 eigenvalue structure \\
\texttt{code/opr22\_geff\_sanity.py} & OPR-22 $G_{\text{eff}}$ formula \\
\texttt{code/open22\_4bR\_phys\_robin\_sweep.py} & Green Path Robin BC sweep \\
\texttt{code/opr21\_bvp\_physical\_run.py} & Physical domain wall BVP \\
\texttt{repro/scripts/repro\_sin2\_z6\_verify.py} & $\sin^2\theta_W = 1/4$ from $\mathbb{Z}_6$ \\
\bottomrule
\end{tabular}
\end{center}

\paragraph{Audit Evidence Files.}
\begin{itemize}[nosep]
    \item \texttt{audit/evidence/OPR19\_G5\_DERIVATION\_REPORT.md}
    \item \texttt{audit/evidence/OPR20\_MEDIATOR\_MASS\_DERIVATION\_REPORT.md}
    \item \texttt{audit/evidence/OPR22\_GEFF\_DERIVATION\_REPORT.md}
    \item \texttt{audit/evidence/OPEN22\_4bR\_PHYSICAL\_ROBIN\_AUDIT.md}
    \item \texttt{audit/evidence/CH20\_EPISTEMIC\_SUMMARY\_AUDIT.md} (this chapter)
\end{itemize}

\paragraph{What Is Reproducible Without New Inputs.}
Given the [P] primitives listed in \S\ref{sec:ch20_parameter_ledger}, all [Dc] results
in \S\ref{sec:ch20_canonical_results} can be reproduced by:
\begin{enumerate}[nosep]
    \item Running the BVP solver with specified $(V, \ell, \rho, \hat\kappa)$
    \item Extracting eigenvalues $x_n$ and mode profiles $f_n(\xi)$
    \item Computing $G_{\text{eff}}$ via the OPR-22 formula
\end{enumerate}
No Standard Model observable ($M_W$, $G_F$, $v$, $\sin^2\theta_W$) is used as input.

%%%%%%%%%%%%%%%%%%%%%%%%%%%%%%%%%%%%%%%%%%%%%%%%%%%%%%%%%%%%%%%%%%%%%%%%%%%%%%%
% CLOSING BOX
%%%%%%%%%%%%%%%%%%%%%%%%%%%%%%%%%%%%%%%%%%%%%%%%%%%%%%%%%%%%%%%%%%%%%%%%%%%%%%%

\begin{tcolorbox}[colback=green!5!white, colframe=green!50!black,
    title=\textbf{Epistemic Checkpoint: Part II Status}]

\textbf{What Part~II achieves}:
\begin{itemize}[nosep]
    \item Complete derivation chain from [P] primitives to $G_{\text{eff}}$
    \item All intermediate steps tagged with epistemic status
    \item No Standard Model observables used as derivation inputs
    \item All claims scope-guarded with explicit conditions
\end{itemize}

\textbf{What remains for Part~III}:
\begin{itemize}[nosep]
    \item Derive [P] primitives from deeper principles
    \item Upgrade [Dc] $\to$ [Der] by closing parameter dependencies
    \item Address baryogenesis, CP violation, neutrino masses
\end{itemize}

\textbf{Publication readiness}: Part~II is ready for Zenodo archival as a
\emph{conditional derivation framework} with transparent epistemic tagging.
\end{tcolorbox}
