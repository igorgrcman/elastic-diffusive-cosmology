% ==============================================================================
% Section 1.11: Epistemic Landscape and Open Problems
% ==============================================================================

\section{Epistemic Landscape and Consolidated Open Problems}
\label{sec:epistemic_map}

This section consolidates the epistemic status of all claims made in this chapter,
along with all open problems requiring future work. Rather than scattering ``What
Remains Open'' subsections throughout, we gather them here for clarity.

% ==============================================================================
% FRAMEWORK 2.0 LANGUAGE COMPLIANCE
% ==============================================================================
\begin{tcolorbox}[breakable, enhanced, colback=blue!3!white, colframe=blue!50!black,
    title=\textbf{Framework 2.0 Language Compliance}, width=\linewidth]
\small
\textbf{EDC Projection Principle:} Every physical process has a \textbf{5D bulk+brane cause}
whose observable residue is a \textbf{3D shadow} on the observer boundary.

\textbf{In this chapter:}
\begin{itemize}[nosep]
    \item \textbf{Baseline facts [BL]:} 3D shadows that any valid model must reproduce.
    \item \textbf{Derived-conditional [Dc]:} 3D predictions from 5D mechanisms.
    \item \textbf{Open problems [OPEN]:} 5D mechanisms not yet fully derived.
\end{itemize}

\textbf{Standard Model observables} (masses, lifetimes, couplings) are the 3D targets;
EDC success is measured by deriving these shadows from 5D causes without circularity.
\end{tcolorbox}

% =========================
% Part II — Book-ready box
% =========================
\begin{tcolorbox}[breakable, enhanced, colback=white, colframe=black, boxrule=0.5pt,
    title={\textbf{Part II (Weak Sector) --- Status Map (as of 2026-01-22)}}, width=\linewidth]
\small
\textit{PMNS angles:} $\theta_{23}$ \textbf{GREEN [Dc]} (Z$_6$ geometry);
$\theta_{13}$ \textbf{YELLOW [BL$\to$Dc]} via $\varepsilon=\lambda/\sqrt{2}$ (15\% off);
$\theta_{12}$ \textbf{YELLOW [Dc]} via $\arctan(1/\sqrt{2})$ (8.6\% off).
\textit{CKM CP:} Phase Cancellation Theorem \textbf{GREEN [Dc]} (pure Z$_3$ gives $J=0$);
Z$_2$ selection yields $\delta\simeq60^\circ$ \textbf{YELLOW [Dc]+[I]} (5° from PDG) with $J\simeq2.9\times10^{-5}$.
\textit{SU(2)$_L$:} minimal ``where/how'' embedding \textbf{YELLOW [P]} (brane-localized; $g_{\rm eff}\simeq g_2$ up to brane terms).
\textit{$G_F$:} sanity skeleton + \textbf{closure spine derived} ($G_{\text{eff}} = g_5^2 \ell |f_1(0)|^2/(2x_1^2)$ [Dc], OPR-22; no-smuggling guardrails explicit); numeric closure \textbf{YELLOW [Dc]+[OPEN]} pending BVP.
Value attempt: G1 ($g^2 = 4\pi\sigma r_e^3/\hbar c$) gives 11\% from SM [P]; $\ell$ candidates surveyed---no SM-free closure.
\textbf{Hard blockers (P1):} OPR-02/21 thick-brane BVP (master key for $I_4$ + parity selection). \textit{(OPR-19/20/22 now have derivation spines [Dc].)}
\textit{Latest:} OPR-20 Attempt G: Derive $\alpha$ from brane physics; \textbf{natural $\alpha = 2\pi$} from $\alpha = \ell/\delta$ with $\delta = R_\xi$ [Dc]+[P]; falls inside target range [5.5, 15] without tuning; $\delta = R_\xi$ identification [P], not yet derived from microphysics. Attempt G\_BC: BC provenance audit; orbifold parity $\to$ BC mapping [BL]; C/D vs E discrepancy resolved (BC choice, not error); \textbf{OPR-20 split into OPR-20a (BC) + OPR-20b ($\alpha$)}; canonical baseline $x_1 = \pi$ [P] (12\% from $M_W$, vs 56\% for NN). \textbf{Attempt H:} $\delta = R_\xi$ identification (boundary layer = relaxation scale); $\alpha = 2\pi \to x_1 = 2.41 \to m_\phi = 54$ GeV (33\% from $M_W$, no SM input). \textbf{H2-Hard (2026-01-23):} Route A BLOCKED (BL theorem missing); Route B PARTIAL ($\delta = R_\xi$ step [P]); convergence CANNOT TEST. \textbf{OPR-20b: [P]+[OPEN]} (downgraded from [Def]+[P]). Gates OPR-20c/d/e/f promoted. Status: RED-C [Dc]+[P]+[OPEN].

\medskip
\textbf{Verdict (Part II):} Flavor numerics largely locked; gauge ``where/how'' fixed [P]; $G_F$ attack-surface mapped---true closure awaits BVP + $(g_5,m_\phi)$ from 5D first principles (no SM-circularity).
\end{tcolorbox}

% ==============================================================================
% PRE-PUBLICATION WARNING (2026-01-29)
% ==============================================================================
\begin{tcolorbox}[colback=red!10!white, colframe=red!70!black,
    title=\textbf{Editorial Warning: Pre-Publication Review Required}]
\textbf{Book 2 is not publication-final.} Before any public release:
\begin{itemize}[nosep]
\item Full narrative/claim audit required
\item Decide what to publish vs.\ internal working notes
\item Ensure strict epistemic tagging ([Der], [Dc], [I], [P], [Cal], [BL])
\item Remove internal-only notes and debug boxes
\end{itemize}
See \texttt{edc\_book\_2/docs/PREPUBLICATION\_REVIEW\_WARNING.md} for checklist.
\end{tcolorbox}

% ==============================================================================
% k-channel Cross-Validation Box (2026-01-29)
% ==============================================================================
% k-channel Spin-Chain Cross-Validation Box
% File: edc_papers/_shared/boxes/kchannel_spinchain_crossval_box.tex
% Created: 2026-01-29
% Purpose: Book-ready summary of independent k-channel cross-validation
%
% EPISTEMIC STATUS: [Der] for mathematical results; cross-validation only

\begin{tcolorbox}[colback=cyan!5!white, colframe=cyan!50!black,
                  title=\textbf{k-channel Cross-Validation (Independent Analog Test)}]
\label{box:kchannel-spinchain-xval}

\textbf{Definition} (averaging correction only):
\begin{equation}
k(N) := \frac{\langle O \rangle_{\text{disc}}}{\langle O \rangle_{\text{cont}}}
= 1 + \frac{a}{c} \quad \text{[Der]}
\end{equation}
Applies \textbf{only} to averaging observables (discrete sampling vs continuum average)
with Z$_N$-symmetric anisotropy $O(\theta) = c + a\cos(N\theta)$.
Under equal-corner-share ($a/c = 1/N$): $k(N) = 1 + 1/N$.
See Box~\ref{box:zn-kchannel-robustness} for applicability rules.

\vspace{0.3em}
\textbf{Cross-Validation} (spin-chain exact diagonalization):
\begin{itemize}[nosep,leftmargin=1.5em]
\item Model: XX spin chain with periodic BC, ground state local energy $o_n$
\item Test: $R = O_{\text{disc}}/O_{\text{cont}}$ for $N = 3, 4, 5, 6, 8, 10, 12$
\item Result: $R = 1 + 1/N$ at machine precision ($<10^{-15}$) for all $N$
\end{itemize}

\vspace{0.3em}
\textbf{VALIDATES} [Der]:
\begin{itemize}[nosep,leftmargin=1.5em]
\item[\checkmark] Averaging mechanism ($R = 1 + a/c$ formula)
\item[\checkmark] $N \ne 6$ generality (tested $N = 3, 4, 5, 8, 10, 12$)
\item[\checkmark] Numerical reproducibility (independent physical system)
\end{itemize}

\vspace{0.3em}
\textbf{DOES NOT VALIDATE}:
\begin{itemize}[nosep,leftmargin=1.5em]
\item[$\times$] EDC sector predictions (pion, $N_{\text{cell}}$, nuclear physics)
\item[$\times$] Physical origin of equal-corner-share ($a/c = 1/N$) in EDC
\item[$\times$] Any claim that spin chains are described by EDC
\end{itemize}

\vspace{0.3em}
\textit{Use this box as epistemic guardrail: k-channel is a correction channel
(discrete$\to$continuum averaging), not a universal multiplier.}

\vspace{0.2em}
\textit{Source: \texttt{spin\_chain\_kchannel\_ed\_test.py}; doc: \texttt{SPIN\_CHAIN\_KCHANNEL\_CROSSVALIDATION.md}}

\end{tcolorbox}


% ==============================================================================
% G_F BVP All-Gates Pass + Physical Priors Box (2026-01-29)
% ==============================================================================
% ═══════════════════════════════════════════════════════════════════════════════
% G_F BVP ALL-GATES PASS + PHYSICAL PRIORS BOX
% ═══════════════════════════════════════════════════════════════════════════════
%
% File: gf_bvp_allgates_physical_priors_box.tex
% Created: 2026-01-29
% Issue: OPR-21b/c — BVP pipeline results with physical priors
% Status: Framework GREEN; values YELLOW (BVP-gated, tuned priors)
%
% Usage: % ═══════════════════════════════════════════════════════════════════════════════
% G_F BVP ALL-GATES PASS + PHYSICAL PRIORS BOX
% ═══════════════════════════════════════════════════════════════════════════════
%
% File: gf_bvp_allgates_physical_priors_box.tex
% Created: 2026-01-29
% Issue: OPR-21b/c — BVP pipeline results with physical priors
% Status: Framework GREEN; values YELLOW (BVP-gated, tuned priors)
%
% Usage: % ═══════════════════════════════════════════════════════════════════════════════
% G_F BVP ALL-GATES PASS + PHYSICAL PRIORS BOX
% ═══════════════════════════════════════════════════════════════════════════════
%
% File: gf_bvp_allgates_physical_priors_box.tex
% Created: 2026-01-29
% Issue: OPR-21b/c — BVP pipeline results with physical priors
% Status: Framework GREEN; values YELLOW (BVP-gated, tuned priors)
%
% Usage: \input{../../edc_papers/_shared/boxes/gf_bvp_allgates_physical_priors_box}
%
% Dependencies: tcolorbox, booktabs, enumitem
%
% ═══════════════════════════════════════════════════════════════════════════════

\begin{tcolorbox}[breakable, enhanced, colback=green!5!white, colframe=green!60!black,
    title={\textbf{BVP Pipeline: All Gates Pass (OPR-21b/c)}}, width=\linewidth]

\textbf{Result:} Three independent gates PASS simultaneously at parameters found via scan.
This match is \textbf{YELLOW} until priors ($\delta$, $d_{\text{LR}}$, fw) are derived from 5D action.

\vspace{0.5em}
\textbf{Why non-trivial:} At baseline parameters, Gate~1 failed by factor $38\times$.
After parameter scan, $X_{\text{EDC}}/X_{\text{target}} = 1.045$ --- a $36.8\times$ improvement.
Gate~1 ($I_4$ overlap) is the \textbf{primary falsification channel}: if mode-overlap mechanism
cannot produce the required suppression, the entire $G_F$ derivation path fails.

\vspace{0.5em}
\begin{center}
\renewcommand{\arraystretch}{1.2}
\begin{tabular}{llcc}
\toprule
\textbf{Gate} & \textbf{Test} & \textbf{Value} & \textbf{Status} \\
\midrule
Gate 1 & $I_4$ overlap window & $2.08 \times 10^{-3}$ GeV & \textcolor{OliveGreen}{\textbf{PASS}} (was $38\times$ off) \\
Gate 2 & $M_{\text{eff}}$ mass scaling & $2.43$ GeV & \textcolor{OliveGreen}{\textbf{PASS}} \\
Gate 3 & $g_{\text{eff}}^2$ coupling & $0.196$ & \textcolor{OliveGreen}{\textbf{PASS}} \\
\midrule
\multicolumn{2}{l}{\textbf{Target match:}} & \multicolumn{2}{l}{$X_{\text{EDC}} / X_{\text{target}} = 1.045$ (\textbf{4.5\% error})} \\
\bottomrule
\end{tabular}
\end{center}

\vspace{0.5em}
\textbf{Tuned parameters} \tagDc{}+\tagCal{}:
\begin{center}
\begin{tabular}{lccl}
\toprule
\textbf{Parameter} & \textbf{Value} & \textbf{Physical Length} & \textbf{Interpretation} \\
\midrule
$\delta$ (brane thickness) & $0.533\,\text{GeV}^{-1}$ & $0.105\,\text{fm}$ & $= \hbar/(2m_p)$ \tagDc{} \\
LR\_separation & $8.0\,\delta$ & $0.84\,\text{fm}$ & $\approx r_p$ (suggestive) \\
fermion\_width & $0.8\,\delta$ & $0.085\,\text{fm}$ & $\approx 0.4\,\lambda_N$ \\
\bottomrule
\end{tabular}
\end{center}

\vspace{0.5em}
\textbf{Sensitivity decomposition} \tagDer{}:
\begin{itemize}[nosep]
    \item LR\_separation elasticity: $-6.5$ (dominant, exponential control)
    \item fermion\_width elasticity: $+1.3$ (secondary, polynomial control)
\end{itemize}
Physical interpretation: LR controls the exponential overlap suppression; fw fine-tunes the residual.

\vspace{0.5em}
\textbf{Coincidence flag:} $d_{\text{LR}} = 8\delta = 0.84\,\text{fm} \approx r_p = 0.84\,\text{fm}$.\\
\textit{Marked as suggestive, NOT derived. Would require deriving $d_{\text{LR}}$ from chiral localization.}

\vspace{0.5em}
\textbf{Potential-shape robustness:} Gaussian, RS-like, and tanh backgrounds all produce wells with depth $\sim 1/\delta^2$ and width $\sim \delta$. Specific shape is modeling choice \tagDc{}, not uniquely determined.

\end{tcolorbox}

% ═══════════════════════════════════════════════════════════════════════════════
% EPISTEMIC GUARDRAIL
% ═══════════════════════════════════════════════════════════════════════════════

\begin{tcolorbox}[colback=yellow!10!white, colframe=yellow!70!black,
    title={\textbf{Epistemic Status: Framework GREEN, Values YELLOW}}]

\textbf{What this validates:}
\begin{itemize}[nosep]
    \item Mode-overlap mechanism can produce the required $G_F$ suppression \checkmark
    \item Three independent consistency gates are satisfied simultaneously \checkmark
    \item Parameter space exists where BVP solution matches target \checkmark
    \item Sensitivity is understood: LR dominant (exponential), fw secondary \checkmark
\end{itemize}

\textbf{What this does NOT validate:}
\begin{itemize}[nosep]
    \item Physical priors are not derived from 5D action (tuned) $\times$
    \item The $d_{\text{LR}} \approx r_p$ coincidence is not explained $\times$
    \item Potential shape is ansatz, not derived from bulk dynamics $\times$
    \item $\delta = \hbar/(2m_p)$ is postulated, not derived $\times$
\end{itemize}

\vspace{0.5em}
\textbf{Big Guardrail:}
\begin{center}
\fbox{\parbox{0.9\linewidth}{\centering
\textbf{Framework: GREEN} (mechanism works)\\
\textbf{Values: YELLOW} (BVP-gated, tuned priors --- not yet derived)
}}
\end{center}

\vspace{0.5em}
\textbf{Upgrade path YELLOW $\to$ GREEN:}
\begin{enumerate}[nosep]
    \item Derive $\delta = \hbar/(2m_p)$ from 5D action (no tuning)
    \item Derive $d_{\text{LR}}$ from chiral localization / junction geometry
    \item Derive fw from BVP stability or eigenvalue constraints
\end{enumerate}

\textbf{Cross-references:}
\begin{itemize}[nosep]
    \item Sensitivity analysis: \texttt{docs/GF\_BVP\_TUNING\_DECOMPOSITION.md}
    \item Physical priors: \texttt{docs/GF\_BVP\_PHYSICAL\_PRIORS.md}
    \item Gate report: \texttt{docs/GF\_BVP\_GATE\_REPORT.md}
    \item Defense notes: \texttt{docs/GF\_BVP\_DEFENSE\_NOTES.md}
\end{itemize}

\end{tcolorbox}

% ═══════════════════════════════════════════════════════════════════════════════
% END OF BOX
% ═══════════════════════════════════════════════════════════════════════════════

%
% Dependencies: tcolorbox, booktabs, enumitem
%
% ═══════════════════════════════════════════════════════════════════════════════

\begin{tcolorbox}[breakable, enhanced, colback=green!5!white, colframe=green!60!black,
    title={\textbf{BVP Pipeline: All Gates Pass (OPR-21b/c)}}, width=\linewidth]

\textbf{Result:} Three independent gates PASS simultaneously at parameters found via scan.
This match is \textbf{YELLOW} until priors ($\delta$, $d_{\text{LR}}$, fw) are derived from 5D action.

\vspace{0.5em}
\textbf{Why non-trivial:} At baseline parameters, Gate~1 failed by factor $38\times$.
After parameter scan, $X_{\text{EDC}}/X_{\text{target}} = 1.045$ --- a $36.8\times$ improvement.
Gate~1 ($I_4$ overlap) is the \textbf{primary falsification channel}: if mode-overlap mechanism
cannot produce the required suppression, the entire $G_F$ derivation path fails.

\vspace{0.5em}
\begin{center}
\renewcommand{\arraystretch}{1.2}
\begin{tabular}{llcc}
\toprule
\textbf{Gate} & \textbf{Test} & \textbf{Value} & \textbf{Status} \\
\midrule
Gate 1 & $I_4$ overlap window & $2.08 \times 10^{-3}$ GeV & \textcolor{OliveGreen}{\textbf{PASS}} (was $38\times$ off) \\
Gate 2 & $M_{\text{eff}}$ mass scaling & $2.43$ GeV & \textcolor{OliveGreen}{\textbf{PASS}} \\
Gate 3 & $g_{\text{eff}}^2$ coupling & $0.196$ & \textcolor{OliveGreen}{\textbf{PASS}} \\
\midrule
\multicolumn{2}{l}{\textbf{Target match:}} & \multicolumn{2}{l}{$X_{\text{EDC}} / X_{\text{target}} = 1.045$ (\textbf{4.5\% error})} \\
\bottomrule
\end{tabular}
\end{center}

\vspace{0.5em}
\textbf{Tuned parameters} \tagDc{}+\tagCal{}:
\begin{center}
\begin{tabular}{lccl}
\toprule
\textbf{Parameter} & \textbf{Value} & \textbf{Physical Length} & \textbf{Interpretation} \\
\midrule
$\delta$ (brane thickness) & $0.533\,\text{GeV}^{-1}$ & $0.105\,\text{fm}$ & $= \hbar/(2m_p)$ \tagDc{} \\
LR\_separation & $8.0\,\delta$ & $0.84\,\text{fm}$ & $\approx r_p$ (suggestive) \\
fermion\_width & $0.8\,\delta$ & $0.085\,\text{fm}$ & $\approx 0.4\,\lambda_N$ \\
\bottomrule
\end{tabular}
\end{center}

\vspace{0.5em}
\textbf{Sensitivity decomposition} \tagDer{}:
\begin{itemize}[nosep]
    \item LR\_separation elasticity: $-6.5$ (dominant, exponential control)
    \item fermion\_width elasticity: $+1.3$ (secondary, polynomial control)
\end{itemize}
Physical interpretation: LR controls the exponential overlap suppression; fw fine-tunes the residual.

\vspace{0.5em}
\textbf{Coincidence flag:} $d_{\text{LR}} = 8\delta = 0.84\,\text{fm} \approx r_p = 0.84\,\text{fm}$.\\
\textit{Marked as suggestive, NOT derived. Would require deriving $d_{\text{LR}}$ from chiral localization.}

\vspace{0.5em}
\textbf{Potential-shape robustness:} Gaussian, RS-like, and tanh backgrounds all produce wells with depth $\sim 1/\delta^2$ and width $\sim \delta$. Specific shape is modeling choice \tagDc{}, not uniquely determined.

\end{tcolorbox}

% ═══════════════════════════════════════════════════════════════════════════════
% EPISTEMIC GUARDRAIL
% ═══════════════════════════════════════════════════════════════════════════════

\begin{tcolorbox}[colback=yellow!10!white, colframe=yellow!70!black,
    title={\textbf{Epistemic Status: Framework GREEN, Values YELLOW}}]

\textbf{What this validates:}
\begin{itemize}[nosep]
    \item Mode-overlap mechanism can produce the required $G_F$ suppression \checkmark
    \item Three independent consistency gates are satisfied simultaneously \checkmark
    \item Parameter space exists where BVP solution matches target \checkmark
    \item Sensitivity is understood: LR dominant (exponential), fw secondary \checkmark
\end{itemize}

\textbf{What this does NOT validate:}
\begin{itemize}[nosep]
    \item Physical priors are not derived from 5D action (tuned) $\times$
    \item The $d_{\text{LR}} \approx r_p$ coincidence is not explained $\times$
    \item Potential shape is ansatz, not derived from bulk dynamics $\times$
    \item $\delta = \hbar/(2m_p)$ is postulated, not derived $\times$
\end{itemize}

\vspace{0.5em}
\textbf{Big Guardrail:}
\begin{center}
\fbox{\parbox{0.9\linewidth}{\centering
\textbf{Framework: GREEN} (mechanism works)\\
\textbf{Values: YELLOW} (BVP-gated, tuned priors --- not yet derived)
}}
\end{center}

\vspace{0.5em}
\textbf{Upgrade path YELLOW $\to$ GREEN:}
\begin{enumerate}[nosep]
    \item Derive $\delta = \hbar/(2m_p)$ from 5D action (no tuning)
    \item Derive $d_{\text{LR}}$ from chiral localization / junction geometry
    \item Derive fw from BVP stability or eigenvalue constraints
\end{enumerate}

\textbf{Cross-references:}
\begin{itemize}[nosep]
    \item Sensitivity analysis: \texttt{docs/GF\_BVP\_TUNING\_DECOMPOSITION.md}
    \item Physical priors: \texttt{docs/GF\_BVP\_PHYSICAL\_PRIORS.md}
    \item Gate report: \texttt{docs/GF\_BVP\_GATE\_REPORT.md}
    \item Defense notes: \texttt{docs/GF\_BVP\_DEFENSE\_NOTES.md}
\end{itemize}

\end{tcolorbox}

% ═══════════════════════════════════════════════════════════════════════════════
% END OF BOX
% ═══════════════════════════════════════════════════════════════════════════════

%
% Dependencies: tcolorbox, booktabs, enumitem
%
% ═══════════════════════════════════════════════════════════════════════════════

\begin{tcolorbox}[breakable, enhanced, colback=green!5!white, colframe=green!60!black,
    title={\textbf{BVP Pipeline: All Gates Pass (OPR-21b/c)}}, width=\linewidth]

\textbf{Result:} Three independent gates PASS simultaneously at parameters found via scan.
This match is \textbf{YELLOW} until priors ($\delta$, $d_{\text{LR}}$, fw) are derived from 5D action.

\vspace{0.5em}
\textbf{Why non-trivial:} At baseline parameters, Gate~1 failed by factor $38\times$.
After parameter scan, $X_{\text{EDC}}/X_{\text{target}} = 1.045$ --- a $36.8\times$ improvement.
Gate~1 ($I_4$ overlap) is the \textbf{primary falsification channel}: if mode-overlap mechanism
cannot produce the required suppression, the entire $G_F$ derivation path fails.

\vspace{0.5em}
\begin{center}
\renewcommand{\arraystretch}{1.2}
\begin{tabular}{llcc}
\toprule
\textbf{Gate} & \textbf{Test} & \textbf{Value} & \textbf{Status} \\
\midrule
Gate 1 & $I_4$ overlap window & $2.08 \times 10^{-3}$ GeV & \textcolor{OliveGreen}{\textbf{PASS}} (was $38\times$ off) \\
Gate 2 & $M_{\text{eff}}$ mass scaling & $2.43$ GeV & \textcolor{OliveGreen}{\textbf{PASS}} \\
Gate 3 & $g_{\text{eff}}^2$ coupling & $0.196$ & \textcolor{OliveGreen}{\textbf{PASS}} \\
\midrule
\multicolumn{2}{l}{\textbf{Target match:}} & \multicolumn{2}{l}{$X_{\text{EDC}} / X_{\text{target}} = 1.045$ (\textbf{4.5\% error})} \\
\bottomrule
\end{tabular}
\end{center}

\vspace{0.5em}
\textbf{Tuned parameters} \tagDc{}+\tagCal{}:
\begin{center}
\begin{tabular}{lccl}
\toprule
\textbf{Parameter} & \textbf{Value} & \textbf{Physical Length} & \textbf{Interpretation} \\
\midrule
$\delta$ (brane thickness) & $0.533\,\text{GeV}^{-1}$ & $0.105\,\text{fm}$ & $= \hbar/(2m_p)$ \tagDc{} \\
LR\_separation & $8.0\,\delta$ & $0.84\,\text{fm}$ & $\approx r_p$ (suggestive) \\
fermion\_width & $0.8\,\delta$ & $0.085\,\text{fm}$ & $\approx 0.4\,\lambda_N$ \\
\bottomrule
\end{tabular}
\end{center}

\vspace{0.5em}
\textbf{Sensitivity decomposition} \tagDer{}:
\begin{itemize}[nosep]
    \item LR\_separation elasticity: $-6.5$ (dominant, exponential control)
    \item fermion\_width elasticity: $+1.3$ (secondary, polynomial control)
\end{itemize}
Physical interpretation: LR controls the exponential overlap suppression; fw fine-tunes the residual.

\vspace{0.5em}
\textbf{Coincidence flag:} $d_{\text{LR}} = 8\delta = 0.84\,\text{fm} \approx r_p = 0.84\,\text{fm}$.\\
\textit{Marked as suggestive, NOT derived. Would require deriving $d_{\text{LR}}$ from chiral localization.}

\vspace{0.5em}
\textbf{Potential-shape robustness:} Gaussian, RS-like, and tanh backgrounds all produce wells with depth $\sim 1/\delta^2$ and width $\sim \delta$. Specific shape is modeling choice \tagDc{}, not uniquely determined.

\end{tcolorbox}

% ═══════════════════════════════════════════════════════════════════════════════
% EPISTEMIC GUARDRAIL
% ═══════════════════════════════════════════════════════════════════════════════

\begin{tcolorbox}[colback=yellow!10!white, colframe=yellow!70!black,
    title={\textbf{Epistemic Status: Framework GREEN, Values YELLOW}}]

\textbf{What this validates:}
\begin{itemize}[nosep]
    \item Mode-overlap mechanism can produce the required $G_F$ suppression \checkmark
    \item Three independent consistency gates are satisfied simultaneously \checkmark
    \item Parameter space exists where BVP solution matches target \checkmark
    \item Sensitivity is understood: LR dominant (exponential), fw secondary \checkmark
\end{itemize}

\textbf{What this does NOT validate:}
\begin{itemize}[nosep]
    \item Physical priors are not derived from 5D action (tuned) $\times$
    \item The $d_{\text{LR}} \approx r_p$ coincidence is not explained $\times$
    \item Potential shape is ansatz, not derived from bulk dynamics $\times$
    \item $\delta = \hbar/(2m_p)$ is postulated, not derived $\times$
\end{itemize}

\vspace{0.5em}
\textbf{Big Guardrail:}
\begin{center}
\fbox{\parbox{0.9\linewidth}{\centering
\textbf{Framework: GREEN} (mechanism works)\\
\textbf{Values: YELLOW} (BVP-gated, tuned priors --- not yet derived)
}}
\end{center}

\vspace{0.5em}
\textbf{Upgrade path YELLOW $\to$ GREEN:}
\begin{enumerate}[nosep]
    \item Derive $\delta = \hbar/(2m_p)$ from 5D action (no tuning)
    \item Derive $d_{\text{LR}}$ from chiral localization / junction geometry
    \item Derive fw from BVP stability or eigenvalue constraints
\end{enumerate}

\textbf{Cross-references:}
\begin{itemize}[nosep]
    \item Sensitivity analysis: \texttt{docs/GF\_BVP\_TUNING\_DECOMPOSITION.md}
    \item Physical priors: \texttt{docs/GF\_BVP\_PHYSICAL\_PRIORS.md}
    \item Gate report: \texttt{docs/GF\_BVP\_GATE\_REPORT.md}
    \item Defense notes: \texttt{docs/GF\_BVP\_DEFENSE\_NOTES.md}
\end{itemize}

\end{tcolorbox}

% ═══════════════════════════════════════════════════════════════════════════════
% END OF BOX
% ═══════════════════════════════════════════════════════════════════════════════


\subsection{Quantitative Summary: Thresholds and Gates}

Before cataloging the epistemic status of each claim, we present the quantitative
data that underlies the case studies. This table is \textbf{not} EDC-specific;
it is baseline physics \tagBL{} that any framework must reproduce.

\subsubsection{Q-Gates and Kinematic Thresholds}

\begin{center}
\begin{tabular}{llccc}
\toprule
\textbf{Decay} & \textbf{Channel} & \textbf{$Q$-value} & \textbf{Gate} & \textbf{Status} \\
\midrule
\multirow{2}{*}{Neutron} & $n \to p + e^- + \bar\nu_e$ &
  $+0.782$ MeV & $\mathcal{P}_{\text{energy}}$ & OPEN \\
& $n \to p + \mu^- + \bar\nu_\mu$ &
  $-104.4$ MeV & $\mathcal{P}_{\text{energy}}$ & CLOSED \\
\addlinespace
\multirow{2}{*}{Muon} & $\mu^- \to e^- + \bar\nu_e + \nu_\mu$ &
  $+105.1$ MeV & $\mathcal{P}_{\text{energy}}$ & OPEN \\
& $\mu^- \to \text{hadrons}$ &
  --- & $\mathcal{P}_{\text{mode}}$ & FORBIDDEN \\
\addlinespace
\multirow{2}{*}{Tau} & $\tau^- \to e^-/\mu^- + \nu\bar\nu$ &
  $+1776/1671$ MeV & $\mathcal{P}_{\text{energy}}$ & OPEN \\
& $\tau^- \to \text{hadrons} + \nu_\tau$ &
  $+1637$ MeV & $\mathcal{P}_{\text{mode}}$ & OPEN \\
\addlinespace
\multirow{2}{*}{Pion} & $\pi^+ \to \mu^+ + \nu_\mu$ &
  $+33.9$ MeV & $\mathcal{P}_{\text{chir}}$ & OPEN \\
& $\pi^+ \to e^+ + \nu_e$ &
  $+139.1$ MeV & $\mathcal{P}_{\text{chir}}$ & SUPPRESSED \\
\addlinespace
Electron & $e^- \to X$ & --- & No lower mode & BLOCKED \\
\bottomrule
\end{tabular}
\end{center}

\paragraph{Reading the table.}
\begin{itemize}[nosep]
  \item $Q > 0$: kinematically allowed (energy available for products)
  \item $Q < 0$: kinematically forbidden (would violate energy conservation)
  \item SUPPRESSED: allowed but with reduced amplitude (helicity suppression)
  \item FORBIDDEN: blocked by mode mismatch, not kinematics
  \item BLOCKED: no decay channel exists
\end{itemize}

\subsubsection{Mass and Lifetime Data}

\begin{center}
\begin{tabular}{lcccc}
\toprule
\textbf{Particle} & \textbf{Mass (MeV)} & \textbf{Lifetime} &
\textbf{Ontology} & \textbf{Dominant Gate} \\
\midrule
Neutron & $939.565$ & $879.4$ s & Bulk-core junction &
$\mathcal{P}_{\text{energy}}$ \\
Muon & $105.66$ & $2.20~\mu$s & Brane-dominant &
$\mathcal{P}_{\text{mode}}$ \\
Tau & $1776.9$ & $0.290$ ps & Brane-dominant &
$\mathcal{P}_{\text{energy}}$ \\
Pion & $139.57$ & $26.0$ ns & Junction-pair &
$\mathcal{P}_{\text{chir}}$ \\
Electron & $0.511$ & $> 10^{28}$ yr & Brane defect (ground) &
None (stable) \\
Neutrino & $< 10^{-6}$ & Stable & Edge mode &
Overlap suppression \\
\bottomrule
\end{tabular}
\end{center}

All values are \tagBL{} (PDG 2024). The ``Ontology'' and ``Dominant Gate'' columns
are EDC interpretations \tagP{}/\tagDc{}.

\subsubsection{What the Table Shows}

This quantitative summary demonstrates that:
\begin{enumerate}[nosep]
  \item \textbf{Channel selection is kinematic}: Neutron $\to$ electron (not muon)
        because $Q_\beta(\mu) < 0$.
  \item \textbf{Mode overlap matters}: Muon $\to$ leptons only because mode mismatch
        forbids hadronic channels.
  \item \textbf{Chirality suppression is real}: Pion $\to$ muon dominates over
        electron by $(m_\mu/m_e)^2 \approx 4 \times 10^4$.
  \item \textbf{Electron stability is structural}: No lower charged mode exists.
\end{enumerate}

These are \emph{facts} that EDC must be consistent with; they are not EDC-derived
claims.

\vspace{1em}

This section provides a comprehensive summary of the epistemic status of each
claim made in this chapter. The goal is transparency: the reader should know
exactly what is established, what is structural interpretation, and what
remains to be computed.

\subsection{The Five Categories}

Throughout this chapter, we have used the following epistemic tags:

\begin{center}
\begin{tabular}{clp{8cm}}
\toprule
\textbf{Tag} & \textbf{Status} & \textbf{Meaning} \\
\midrule
\tagBL{} & Baseline & Established experimental fact or Standard Model result \\
 & Definition & Terminological convention adopted in this work \\
\tagP{} & Postulate & Structural assumption or hypothesis \\
\tagDc{} & Deduction & Derived from postulates via explicit reasoning \\
(open) & Open & Requires further work; not yet computed or proven \\
\bottomrule
\end{tabular}
\end{center}

\subsection{Baseline Facts: The 3D Shadows EDC Must Reproduce}

The following are empirical facts (3D shadows) that EDC must be consistent with.
These are \emph{targets} for derivation, not sources for input:

\subsubsection{Particle Properties}

\begin{center}
\begin{tabular}{lll}
\toprule
\textbf{Quantity} & \textbf{Value} & \textbf{Source} \\
\midrule
Neutron mass & $m_n = 939.565$ MeV & PDG \\
Proton mass & $m_p = 938.272$ MeV & PDG \\
Electron mass & $m_e = 0.511$ MeV & PDG \\
Muon mass & $m_\mu = 105.66$ MeV & PDG \\
Tau mass & $m_\tau = 1776.9$ MeV & PDG \\
Pion mass & $m_{\pi^\pm} = 139.57$ MeV & PDG \\
\bottomrule
\end{tabular}
\end{center}

\subsubsection{Lifetimes}

\begin{center}
\begin{tabular}{lll}
\toprule
\textbf{Particle} & \textbf{Lifetime} & \textbf{Source} \\
\midrule
Neutron & $\tau_n \approx 880$ s & PDG \\
Muon & $\tau_\mu \approx 2.2 \times 10^{-6}$ s & PDG \\
Tau & $\tau_\tau \approx 2.9 \times 10^{-13}$ s & PDG \\
Pion & $\tau_\pi \approx 2.6 \times 10^{-8}$ s & PDG \\
Electron & $> 10^{28}$ years & PDG (limit) \\
\bottomrule
\end{tabular}
\end{center}

\subsubsection{Decay Channels and Branching Ratios}

\begin{center}
\begin{tabular}{lll}
\toprule
\textbf{Decay} & \textbf{Branching Ratio} & \textbf{Status} \\
\midrule
$n \to p + e^- + \bar\nu_e$ & $\approx 100\%$ & \tagBL{} \\
$\mu^- \to e^- + \bar\nu_e + \nu_\mu$ & $\approx 100\%$ & \tagBL{} \\
$\tau^- \to e^- + \bar\nu_e + \nu_\tau$ & $\approx 17.8\%$ & \tagBL{} \\
$\tau^- \to \mu^- + \bar\nu_\mu + \nu_\tau$ & $\approx 17.4\%$ & \tagBL{} \\
$\tau^- \to \text{hadrons} + \nu_\tau$ & $\approx 64.8\%$ & \tagBL{} \\
$\pi^+ \to \mu^+ + \nu_\mu$ & $\approx 99.99\%$ & \tagBL{} \\
$\pi^+ \to e^+ + \nu_e$ & $\approx 0.012\%$ & \tagBL{} \\
\bottomrule
\end{tabular}
\end{center}

\subsubsection{Coupling Constants}

\begin{center}
\begin{tabular}{lll}
\toprule
\textbf{Quantity} & \textbf{Value} & \textbf{Source} \\
\midrule
Fermi constant & $G_F = 1.166 \times 10^{-5}~\text{GeV}^{-2}$ & PDG \\
$W$ boson mass & $M_W = 80.4$ GeV & PDG \\
Fine structure const. & $\alpha \approx 1/137$ & CODATA \\
\bottomrule
\end{tabular}
\end{center}

\subsection{Postulates (Structural Assumptions)}

The following are hypotheses that define the EDC framework:

\begin{center}
\begin{tabular}{p{4cm}p{9cm}}
\toprule
\textbf{Postulate} & \textbf{Statement} \\
\midrule
Thick brane & The 3D universe is a finite-thickness layer in 5D \\
Bulk-core particles & Neutron, proton have 5D bulk structure \\
Brane-dominant modes & Leptons are excitations of the brane layer \\
Edge modes & Neutrinos are localized at the bulk-brane interface \\
Frozen projection & Observer-facing boundary is quasi-static \\
Pipeline structure & Weak decays proceed via absorption-dissipation-release \\
Mode overlap & Branching ratios depend on wavefunction overlaps \\
Chirality projection & Boundary conditions select helicity \\
\bottomrule
\end{tabular}
\end{center}

\subsection{Deductions (What Follows from Postulates)}

The following claims are derived from the postulates:

\subsubsection{Qualitative Deductions}

\begin{center}
\begin{tabular}{p{5cm}p{8cm}}
\toprule
\textbf{Claim} & \textbf{Derivation Path} \\
\midrule
Neutron decays to electron (not muon) & Kinematic threshold: $Q_\beta(\mu) < 0$ \\
Electron is stable & No lower-lying charged mode exists \\
Muon decay is purely leptonic & Mode mismatch with hadrons \\
Tau has hadronic channels & Higher mode energy opens thresholds \\
Neutrinos interact weakly & Edge-mode localization suppresses overlap \\
\bottomrule
\end{tabular}
\end{center}

\subsubsection{Quantitative Deductions}

\begin{center}
\begin{tabular}{p{4cm}p{5cm}p{4cm}}
\toprule
\textbf{Quantity} & \textbf{EDC Expression} & \textbf{Status} \\
\midrule
$Q_\beta(e)$ value & $m_n - m_p - m_e = 0.782$ MeV & \tagDc{} (arithmetic) \\
$Q_\beta(\mu)$ sign & $< 0$ (channel closed) & \tagDc{} \\
$R_{e/\mu}$ scaling & $\propto (m_e/m_\mu)^2$ & \tagBL{} + \tagP{} \\
\bottomrule
\end{tabular}
\end{center}

\subsection{Open Problems (What Remains to Be Done)}

The following require further work:

\subsubsection{Critical Open Problems}

\begin{center}
\begin{tabular}{p{5cm}p{8cm}}
\toprule
\textbf{Problem} & \textbf{What Is Needed} \\
\midrule
Neutron lifetime value & Compute tunneling rate from 5D junction dynamics \\
$G_F$ derivation & Compute overlap integral in thick-brane background \\
Helicity suppression factor & Solve Dirac equation with boundary conditions \\
Mode spectrum & Solve thick-brane eigenvalue problem \\
Neutrino mass & Compute edge-mode energy \\
\bottomrule
\end{tabular}
\end{center}

\subsubsection{Important but Non-Critical}

\begin{center}
\begin{tabular}{p{5cm}p{8cm}}
\toprule
\textbf{Problem} & \textbf{What Is Needed} \\
\midrule
Tau branching ratios & Compute mode overlaps for hadronic channels \\
$\mu/\tau$ lifetime ratio & Connect to mode energy differences \\
Generation structure & Explain three lepton generations from geometry \\
Neutrino mixing ($\theta_{12}$, $\theta_{13}$) & Structure identified (Attempt~4); geometric origin of $\theta_{12}^0$, $\varepsilon$ needed \\
\bottomrule
\end{tabular}
\end{center}

\subsection{Visual Summary: The Epistemic Landscape}

\begin{center}
\begin{tikzpicture}[scale=0.9]

% Baseline region
\fill[green!15] (-5,0) rectangle (5,1.5);
\node[font=\small\bfseries, green!50!black] at (0,1.2) {BASELINE (Established)};
\node[font=\scriptsize, align=center] at (-2.5,0.5) {Masses, lifetimes,\\branching ratios};
\node[font=\scriptsize, align=center] at (2.5,0.5) {$G_F$, $M_W$,\\SM formulas};

% Postulate region
\fill[yellow!20] (-5,1.7) rectangle (5,3.2);
\node[font=\small\bfseries, yellow!60!black] at (0,2.9) {POSTULATES (Structural Assumptions)};
\node[font=\scriptsize, align=center] at (-2.5,2.2) {Thick brane,\\bulk-core particles};
\node[font=\scriptsize, align=center] at (2.5,2.2) {Frozen projection,\\pipeline structure};

% Deduction region
\fill[blue!15] (-5,3.4) rectangle (5,4.9);
\node[font=\small\bfseries, blue!50!black] at (0,4.6) {DEDUCTIONS (Derived from Postulates)};
\node[font=\scriptsize, align=center] at (-2.5,3.9) {Channel selection,\\stability conditions};
\node[font=\scriptsize, align=center] at (2.5,3.9) {Qualitative patterns,\\threshold effects};

% Open region
\fill[red!10] (-5,5.1) rectangle (5,6.6);
\node[font=\small\bfseries, red!50!black] at (0,6.3) {OPEN (To Be Computed)};
\node[font=\scriptsize, align=center] at (-2.5,5.6) {Lifetime values,\\$G_F$ derivation};
\node[font=\scriptsize, align=center] at (2.5,5.6) {Mode spectrum,\\overlap integrals};

% Arrows showing logical flow
\draw[-{Stealth}, thick, gray] (5.5,0.75) -- (5.5,2.45);
\draw[-{Stealth}, thick, gray] (5.5,2.45) -- (5.5,4.15);
\draw[-{Stealth}, thick, gray] (5.5,4.15) -- (5.5,5.85);
\node[font=\tiny, rotate=90] at (5.9,3.3) {logical dependence};

\end{tikzpicture}
\end{center}

\subsection{What This Chapter Does and Does Not Claim}

\begin{tcolorbox}[readerContract, title={Final Epistemic Statement}]
\textbf{This chapter claims}:
\begin{itemize}[nosep]
  \item A coherent structural interpretation of weak decays in thick-brane geometry
  \item Qualitative explanations for channel selection rules
  \item A well-posed framework for quantitative computation
  \item Explicit falsifiability conditions for each claim
\end{itemize}

\textbf{This chapter does not claim}:
\begin{itemize}[nosep]
  \item First-principles derivation of lifetime values
  \item Explicit computation of branching ratios
  \item Derivation of $G_F$ from the 5D action
  \item Complete solution of the mode spectrum
\end{itemize}

The gap between ``structural interpretation'' and ``derived result'' is
substantial. Closing this gap is the research program.
\end{tcolorbox}

\vspace{0.5em}
\noindent\textit{For readers who want a forensic trail (what was tested, what was rejected,
and which equations support it), a supplementary \textbf{Meta Documentation Pack} for Part~II
contains a Claim Ledger, Decision Log, Research Timeline, and Evidence Map, and can be
optionally included at compile time.}

% ==============================================================================
\subsection{Open Problems Register (OPR)}
\label{sec:opr}

\begin{tcolorbox}[breakable, enhanced, colback=yellow!4, colframe=yellow!60!black,
    title=\textbf{Open Status (Part II)}, width=\linewidth]
\textbf{Part~II is not closed.} The weak-sector narrative is now internally
consistent, but several results remain explicitly \emph{open}: (i)~full
$\mathrm{SU}(2)_L$ embedding, (ii)~PMNS angles beyond the $\mathbb{Z}_3$
baseline, (iii)~the CKM apex $(\bar\rho, \bar\eta)$ and a $\delta$ refinement
beyond the discrete-phase estimate, and (iv)~a first-principles bridge for
$(g_5, m_\phi)$ and the BVP/KK spectrum. These are tracked in the Open
Problems Register and addressed as a staged research program.
\end{tcolorbox}

\subsubsection{Priority-1 Open Problems}

The following items block major claims and define the immediate research program:

\begin{table}[ht]
\centering
\caption{Open Problems Register: Priority-1 items}
\label{tab:opr_p1}
\small\begin{tabular}{@{}lp{4cm}lp{4cm}@{}}
\toprule
\textbf{ID} & \textbf{Item} & \textbf{Status} & \textbf{Next Action} \\
\midrule
OPR-02 & KK tower truncation (N$_{\text{gen}}$) & RED-C [P] & BVP spectral count + robustness (\S\ref{subsubsec:bvp_ngen}) \\
OPR-05a & PMNS $\theta_{23}$ & \textcolor{OliveGreen}{\textbf{GREEN}} [Dc] & Closed: $\sin^2\theta_{23} = 0.564$ (3\%) \\
OPR-05b & PMNS $\theta_{13}$/$\varepsilon$ & \textcolor{YellowOrange}{\textbf{YELLOW}} [BL$\to$Dc] & $\varepsilon = \lambda/\sqrt{2}$, 15\% from PDG \\
OPR-05c & PMNS $\theta_{12}$ & \textcolor{YellowOrange}{\textbf{YELLOW}} [Dc] & $\arctan(1/\sqrt{2}) = 35.26°$ (8.6\% off) \\
OPR-11 & CKM $(\bar\rho, \bar\eta)$ & \textcolor{YellowOrange}{\textbf{YELLOW}} [Dc]+[P] & Odd sign-flip rule [Dc]; brane-parity [P]; BVP needed \\
OPR-12 & CP phase $\delta$ & \textcolor{YellowOrange}{\textbf{YELLOW}} [Dc]+[I] & $\delta = 60°$ (5° from PDG); Phase Cancellation Thm \\
OPR-17 & SU(2)$_L$ gauge embedding & \textcolor{YellowOrange}{\textbf{YELLOW}} [P] & Where/how fixed; origin+masses OPEN \\
OPR-22 & $G_F$ first-principles & \textcolor{YellowOrange}{\textbf{YELLOW}} [Dc]+[OPEN] & Closure spine [Dc]; no-smuggling guardrails; numerics [OPEN] \\
\bottomrule
\end{tabular}
\end{table}

\subsubsection{State of Play by Sector}

\paragraph{CKM/CP (Chapter~\ref{sec:ch7_ckm}).}
\begin{itemize}[nosep]
    \item \textcolor{OliveGreen}{\textbf{Closed:}} Magnitude hierarchy $\lambda$, $\lambda^2$, $\lambda^3$ from overlap localization \tagDc{}
    \item \textcolor{OliveGreen}{\textbf{Closed:}} Phase Cancellation Theorem: pure $\mathbb{Z}_3$ gives $J = 0$ \tagDc{}
    \item \textcolor{YellowOrange}{\textbf{Partial:}} $\delta = 60°$ (5° from PDG) via $\mathbb{Z}_2$ sign selection; $J = 2.9 \times 10^{-5}$ (6\%) \tagDc{}+\tagI{}
    \item \textcolor{YellowOrange}{\textbf{Partial:}} $\mathbb{Z}_2$ parity: odd sign-flip rule \tagDc{}, brane-reflection parity \tagP{}; specific element (BVP) open
    \item \textcolor{BrickRed}{\textbf{Open:}} BVP profile parities; residual 5° in $\delta$
\end{itemize}

\paragraph{PMNS/Neutrinos (Chapter~\ref{sec:ch6_pmns}).}
\begin{itemize}[nosep]
    \item \textcolor{OliveGreen}{\textbf{Closed:}} DFT baseline falsified; $\theta_{23} \approx 45°$ derived from $\mathbb{Z}_6$ geometry (3\% accuracy) \tagDc{}
    \item \textcolor{YellowOrange}{\textbf{Partial:}} $\varepsilon = \lambda/\sqrt{2}$ (Attempt~4.1) predicts reactor scale (15\%); $\theta_{12} = \arctan(1/\sqrt{2})$ (Attempt~4.2) provides geometric origin (8.6\%)---both [Dc], no PDG-smuggling
    \item \textcolor{BrickRed}{\textbf{Open:}} CP phase; Dirac vs Majorana; T1 vs T2 selection for $\theta_{12}$
\end{itemize}

\paragraph{\texorpdfstring{$G_F$ and Weak Coupling (Chapter~\ref{sec:gf_pathway}).}{GF and Weak Coupling.}}
\begin{itemize}[nosep]
    \item \textcolor{OliveGreen}{\textbf{Closed:}} Numerical pathway consistent with EW relations \tagDc{}
    \item \textcolor{OliveGreen}{\textbf{Closed:}} Canonical $g_5$ normalization: $g_4 = g_5$ with orthonormal modes \tagDc{}
    \item \textcolor{OliveGreen}{\textbf{Closed:}} KK spectrum form: $m_\phi = x_1/\ell$ from eigenvalue equation \tagDc{}
    \item \textcolor{YellowOrange}{\textbf{Partial:}} Mode overlap mechanism identified \tagP{}
    \item \textcolor{YellowOrange}{\textbf{Partial:}} Closure spine: $G_{\text{eff}} = g_5^2 \ell |f_1(0)|^2 / (2x_1^2)$ \tagDc{} (OPR-22); no-smuggling guardrails explicit
    \item \textcolor{BrickRed}{\textbf{Open:}} $g_5$ value; $\ell$ from membrane; exact BVP solutions for $I_4$
\end{itemize}

\subsubsection{Highest-Value Closure Targets}

\begin{enumerate}
    \item \textbf{Thick-brane BVP solver} --- appears in 5 OPR items; unlocks generation
          counting, pion decay, $G_F$ derivation, neutrino masses
    \item \textbf{CP mechanism refinement} --- appears in 4 OPR items; Z$_6$ = Z$_2 \times$ Z$_3$
          analysis may close both CKM and PMNS phase sectors
    \item \textbf{KK reduction} --- needed for $g_5$, $m_\phi$; quantitative $G_F$
\end{enumerate}

\paragraph{Philosophy.}
Open problems are not weaknesses---they are precisely enumerated targets for future
work. A theory with well-defined gaps is stronger than one with hidden assumptions.
The OPR ensures that every ``(open)'' marker in this text has a tracking entry,
a priority, and a concrete next action.

\subsubsection{OPR Closure Plan}

\begin{tcolorbox}[breakable, enhanced, colback=blue!5, colframe=blue!60!black,
    title=\textbf{OPR Closure Plan: Prioritized Research Program}, width=\linewidth]
The following staged approach addresses OPR items in order of maximum payoff:

\textbf{P1-A: CP Sector Closure (OPR-06, OPR-11, OPR-12)}
\begin{itemize}[nosep]
    \item \emph{Status:} Attempt~4 + Z$_2$ parity origin completed; $\delta = 60°$ (5° from PDG); OPR-11 upgraded to YELLOW [Dc]+[P]
    \item \emph{Closed:} Odd sign-flip rule \tagDc{}; brane-reflection parity mechanism \tagP{}
    \item \emph{Remaining:} Specific element ($V_{cb}$ or $V_{ub}$) from BVP profile parities; residual 5° in $\delta$
    \item \emph{Payoff:} CKM sector structurally closed; PMNS CP phase (OPR-06) uses same $\mathbb{Z}_6$ framework
\end{itemize}

\textbf{P1-B: PMNS Mixing (OPR-05a/b/c, OPR-08) --- ALL ANGLES GEOMETRIC}
\begin{itemize}[nosep]
    \item \emph{Status:} Attempts~2--4.2 completed; all three angles now have geometric mechanisms, none calibrated to PDG
    \item \emph{Result (Attempt~2):} $\theta_{23}$ GREEN (3\%) from $\mathbb{Z}_6$ geometry \tagDc{}
    \item \emph{Result (Attempt~3):} Discrete $\mathbb{Z}_6$ phases insufficient for $\theta_{12}$, $\theta_{13}$
    \item \emph{Result (Attempt~4):} Rank-2 structure $U = R_{23} \cdot R_{13}(\varepsilon) \cdot R_{12}$ achieves GREEN
    \item \emph{Result (Attempt~4.1):} $\varepsilon = \lambda/\sqrt{2}$ predicts $\sin^2\theta_{13} = 0.025$ (15\% from PDG) \emph{without fitting} [BL$\to$Dc]
    \item \emph{Result (Attempt~4.2):} $\theta_{12} = \arctan(1/\sqrt{2}) = 35.26°$ provides geometric origin ($8.6\%$ from PDG) \tagDc{}---unified $\sqrt{2}$ factor with $\varepsilon$
    \item \emph{Key insight:} Same $\sqrt{2}$ appears in both $\varepsilon$ and $\theta_{12}$---suggests unified geometric origin
    \item \emph{Remaining:} Selection rule T1 vs T2; CP phase
\end{itemize}
\noindent\fbox{\parbox{0.94\linewidth}{\small
\textbf{PMNS closure:} OPR-05a GREEN ($\theta_{23}$ from $\mathbb{Z}_6$, 3\%), OPR-05b YELLOW ($\varepsilon = \lambda/\sqrt{2} \to \theta_{13}$, 15\%), OPR-05c YELLOW ($\theta_{12} = \arctan(1/\sqrt{2})$, 8.6\%)---all geometric [Dc], none calibrated to PDG.}}

\textbf{P1-C: Thick-Brane BVP (OPR-02, OPR-09, OPR-14, OPR-21) --- WORK PACKAGE DEFINED}
\begin{itemize}[nosep]
    \item \emph{Status:} BVP Work Package defined (\S\ref{sec:ch12_bvp_workpackage}); solver skeleton demonstrates bound states; acceptance criteria + failure modes documented
    \item \emph{Closure pack:} Formal BVP statement, output definitions, and closure criteria in \S\ref{ch:bvp_master_key}
    \item \emph{Established:} Dimensionless BVP formulation; $I_4$ overlap integral computed; grid convergence for Neumann/Mixed BCs
    \item \emph{Open:} Physical potential $V(\xi)$ from membrane params $(\sigma, r_e)$; BC justification from physics
    \item \emph{Payoff:} Unlocks generation counting, mode profiles, overlap integrals
\end{itemize}

\textbf{P2: $G_F$ Chain (OPR-19--22) --- CLOSURE SPINE DERIVED [Dc]+[OPEN]}
\begin{itemize}[nosep]
    \item \emph{Status:} Full closure plan complete (\S\ref{sec:ch11_full_closure}); \textbf{OPR-22 upgraded to YELLOW [Dc]+[OPEN]}
    \item \emph{Closed:} Closure spine $G_{\text{eff}} = g_5^2 \ell |f_1(0)|^2 / (2x_1^2)$ [Dc] (OPR-22); no-smuggling guardrails; attack-surface map
    \item \emph{Closed (earlier):} $g_4 = g_5$ canonical normalization [Dc]; $m_\phi = x_1/\ell$ KK form [Dc]
    \item \emph{Value attempt:} G1 ($g^2 = 4\pi\sigma r_e^3/\hbar c \approx 0.37$) is 11\% from SM---\textbf{[P]} promising.
          L1/L2 for $\ell$ surveyed: no SM-free candidate reproduces weak scale without tuning. \textbf{Status unchanged: RED-C [OPEN]}.
    \item \emph{Open:} $g_5$ value (OPR-19: coefficient provenance); $\ell$ from membrane (OPR-20: $f_{\rm geom}$); $I_4$ from BVP profiles (OPR-21)
    \item \emph{4$\pi$ derivation (OPR-19):} Dual-route derivation via Gauss's law + isotropy both yield $C = 4\pi$; alternatives require breaking conventions or spherical symmetry. $g^2 = 0.373$ (6\% from SM). \textbf{Status: YELLOW [Dc]+[P]}.
    \item \emph{Suppression attempt (OPR-20):} Candidate A predicts $m_\phi \approx 620$ GeV (8$\times$ overshoot). \textbf{Factor-8 forensic:} BC route CLOSED [Dc] (max factor-4). \textbf{Attempt C (geometric):} Best derived factor $2\pi\sqrt{2} \approx 8.89$ [Dc]+[P]. \textbf{Attempt D:} $R_\xi$ interpretation [P]; Robin structure [Dc], params [P]; overcounting audit confirms Z$_2 \equiv$ Israel junction, factor-8 naive combinations INVALID. \textbf{Attempt E:} 2$\pi$ factor upgraded [P]$\to$[Dc] (circumference); missing 0.9003 residual remains [OPEN]. \textbf{Attempt F:} Mediator BVP + junction$\to$Robin [Dc]; \textbf{broad region} $\alpha \in [5.5, 15]$ (47\% of range) gives $x_1 \approx 2.5$---NOT needle-tuned. \textbf{Attempt G:} Derive $\alpha$ from brane physics; \textbf{natural $\alpha = 2\pi \approx 6.3$} from $\alpha = \ell/\delta$ with $\delta = R_\xi$ [Dc]+[P]; falls inside target without tuning; upgrade condition: derive $\delta = R_\xi$. \textbf{Attempt G\_BC:} BC provenance audit; orbifold parity $\to$ BC mapping [BL]; \textbf{OPR-20 split} into OPR-20a (BC: field identity) + OPR-20b ($\alpha$: microphysics); baseline $x_1 = \pi$ [P]. \textbf{Attempt H:} $\delta = R_\xi$ closed [Def] (boundary layer = relaxation scale); $\alpha = 2\pi$ natural; $x_1 = 2.41 \to m_\phi = 54$ GeV. \textbf{OPR-20b: YELLOW [Def]+[P]}. \textbf{Attempt H1 (OPR-20a):} Mediator identity analysis; \textbf{A$_5$ ruled out [Dc]} (profile vanishes at boundary $\to$ coupling suppressed); shortlist: KK $A_\mu$ ($n=1$) or brane scalar; discriminant: KK tower vs single mediator. \textbf{OPR-20a: YELLOW [Dc]+[P]}. \textbf{Attempt H2-plus (OPR-20b stricter audit):} Applied mega-prompt criteria requiring two-route convergence. \textbf{Findings:} R$_\xi$ is EW-constrained (M$_Z$), not action-derived; ``unique scale'' claim lacks formal proof; Route A (diffusion$\to$BL theorem) BLOCKED; Route B partial ($\delta = R_\xi$ step is [P]). \textbf{OPR-20b: [Def]+[P] $\to$ [P]+[OPEN] (downgrade).} \textbf{Attempt H2-Hard (2026-01-23):} Rigorous two-route audit per H2-Hard mission. \textbf{Route A (Diffusion $\to$ BL theorem):} BLOCKED---matched asymptotic analysis not performed; transverse diffusion equation missing. \textbf{Route B (Junction $\to$ Robin $\to$ $\delta$):} PARTIAL---Robin structure [Dc], $\alpha \sim \ell/\delta$ [Dc], but $\delta = R_\xi$ step is [P] (physical plausibility, not derivation). \textbf{Convergence:} CANNOT TEST with only one incomplete route. \textbf{Failure modes:} FM-H2-5 (diffusion absent) and FM-H2-6 (EW circularity) partially triggered. \textbf{Single Sentence Rule verdict:} Book must use ``We postulate $\delta \equiv R_\xi$; deriving this identity remains open.'' [P]+[OPEN]. \textbf{Gates promoted:} OPR-20c (R$_\xi$ from action), OPR-20d (BL theorem), OPR-20e (unique scale), OPR-20f ($\delta$-robustness). Status RED-C [Dc]+[P]+[OPEN].
    \item \emph{Payoff:} First-principles $G_F$ without SM circularity---framework non-circular by construction
\end{itemize}
\noindent\emph{G$_F$ chain:} OPR-19/20/22 now have derivation spines [Dc]. Value closure attempt (\S\ref{sec:ch11_value_closure_attempt}) identified G1 as promising coupling formula; $\ell$ requires explaining $f_{\rm geom} \sim 10^{-3}$ or deriving $M_W$ first. True independent prediction is $\sin^2\theta_W = 1/4$; numerical $G_F$ uses SM relations (consistency, not derivation). No hidden tuning.

\textbf{P3: Gauge Structure (OPR-17) --- WHERE/HOW FIXED}
\begin{itemize}[nosep]
    \item \emph{Status:} YELLOW [P] --- Brane-localized SU(2)$_L$ embedding; overlap coupling $g_{\text{eff}} \simeq g_2$
    \item \emph{Closed:} Where gauge fields live (boundary); how they couple (overlap integral)
    \item \emph{Open:} Gauge symmetry origin; W$^\pm$/Z$^0$ mass generation
    \item \emph{Next:} Derive why SU(2)$_L$ symmetry exists; Higgs mechanism in 5D
\end{itemize}

\textbf{Summary:} 24 OPR items total. P1 items (A+B+C) close flavor physics.
P2 closes weak coupling quantitatively. P3 completes gauge structure.
\end{tcolorbox}

% NOTE: BVP Work Package is included as Chapter 14 in the main document,
% not here, to avoid duplicate label definitions.
% See \S\ref{sec:ch12_bvp_workpackage} for the full work package.

