%!TEX root = ../EDC_Part_II_Weak_Sector.tex
% ==============================================================================
% Chapter 11: g_5 and ell Value Closure Attempt (OPR-19 + OPR-20)
% Status: RED-C [OPEN] — Candidate formulas presented; no closure achieved
% ==============================================================================

\subsection{Value Closure Attempt: \texorpdfstring{$g_5$}{g5} and \texorpdfstring{$\ell$}{l} from Membrane Parameters}
\label{sec:ch11_value_closure_attempt}

This subsection attempts to close the \emph{numeric values} of $g_5$ and $\ell$
from membrane parameters $(\sigma, r_e)$. We present two candidate paths for each,
analyze their consistency, and give an honest verdict on closure status.

\begin{tcolorbox}[edcGuardrail, title=\textbf{Scope and Guardrails}]
\textbf{Does:}
\begin{itemize}[nosep]
    \item Survey candidate formulas for $g_5$ and $\ell$ from existing EDC derivations
    \item Check dimensional consistency for each candidate
    \item Compute numeric predictions and compare to required values
\end{itemize}

\textbf{Does NOT:}
\begin{itemize}[nosep]
    \item Claim closure where none exists
    \item Use $M_W$, $G_F$, or $v = 246$ GeV to fix parameters (forbidden)
    \item Introduce new postulates without explicit tagging
\end{itemize}
\end{tcolorbox}

% ------------------------------------------------------------------------------
\subsubsection{Normalization Audit: Current State}
\label{sec:ch11_norm_audit}

\paragraph{Forensic summary.}
A codebase audit of existing $g_5$, $g_4$, and $\ell$ definitions reveals:

\begin{table}[ht]
\centering
\caption{Normalization conventions in current documentation}
\label{tab:ch11_norm_audit}
\small
\begin{tabular}{lll}
\toprule
\textbf{Symbol} & \textbf{Convention} & \textbf{Reference} \\
\midrule
$g_5$ & 5D coupling, $[g_5] = [E]^{-1/2}$ & Eq.~\eqref{eq:ch11_5d_gauge_action} \\
$g_4 = g_5$ & With orthonormal KK modes $\int \chi^2 d\xi = 1$ & Eq.~\eqref{eq:ch11_g4_g5_relation} \\
$g_4 = g_5/\sqrt{\ell}$ & Alternative: if $\int \chi^2 d\xi = \ell$ & \S\ref{sec:ch11_g5_canonical} \\
\addlinespace
$\ell$ & Brane layer thickness, $[\ell] = [E]^{-1}$ & \S\ref{sec:ch11_kk_spectrum} \\
$m_\phi = x_1/\ell$ & Mediator mass from KK eigenvalue & Eq.~\eqref{eq:ch11_mphi_scale} \\
\bottomrule
\end{tabular}
\end{table}

\paragraph{Key finding.}
With orthonormal modes, $g_4 = g_5$ \emph{numerically}. This is \tagDc{} from the
5D action. However, the \emph{value} of $g_5$ remains \textbf{[OPEN]}: we have the
coupling's dimension but not its magnitude from first principles.

% ------------------------------------------------------------------------------
\subsubsection{Candidate L1: Brane Thickness from Weak Scale Matching}
\label{sec:ch11_ell_L1}

\paragraph{The identification.}
If the mediator mass is to match the $W$ boson mass scale \tagI{}:
\begin{equation}
    m_\phi \sim M_W \quad\Rightarrow\quad
    \ell_{\text{L1}} = \frac{x_1}{M_W} \sim \frac{\pi}{80\text{ GeV}} \approx 0.04 \text{ fm}
    \label{eq:ch11_ell_L1}
\end{equation}

\begin{tcolorbox}[colback=red!5!white, colframe=red!60!black,
    title=\textbf{L1 Verdict: Identification, Not Derivation}]
\textbf{Status:} \tagI{} (calibration)

\textbf{Problem:} Using $M_W = 80$ GeV as input \emph{imports the weak scale}.
This is explicitly forbidden by the no-smuggling guardrails.

\textbf{Would be [Dc] if:} We could derive $M_W$ from $(\sigma, r_e)$ independently.
Currently, that derivation does not exist.
\end{tcolorbox}

% ------------------------------------------------------------------------------
\subsubsection{Candidate L2: Brane Thickness from Membrane Parameters}
\label{sec:ch11_ell_L2}

\paragraph{Dimensional analysis.}
From EDC membrane parameters, the available length scales are:
\begin{itemize}[nosep]
    \item $r_e \approx 1$ fm (lattice spacing) \tagP{}
    \item $\lambda_\sigma \equiv \hbar c / (\sigma r_e^2) = 197.3 / 5.856 \approx 34$ fm
          (energy-to-length conversion)
    \item $R_\xi \sim 10^{-3}$ fm (diffusion correlation length from Part I) \tagP{}
\end{itemize}

\paragraph{Candidate formula.}
A natural ansatz for brane thickness from membrane tension \tagP{}:
\begin{equation}
    \ell_{\text{L2}} = \frac{\hbar c}{\sigma r_e^2} \times f_{\text{geom}}
    \label{eq:ch11_ell_L2}
\end{equation}
where $f_{\text{geom}}$ is a dimensionless geometric factor.

\paragraph{Numeric requirement.}
To match the weak scale ($m_\phi \sim 80$ GeV with $x_1 \sim \pi$):
\begin{equation}
    \ell_{\text{required}} \sim \frac{\pi}{80 \text{ GeV}} \approx 0.04 \text{ fm}
    \quad\Rightarrow\quad
    f_{\text{geom}} \approx \frac{0.04}{34} \approx 1.2 \times 10^{-3}
    \label{eq:ch11_f_geom_required}
\end{equation}

\begin{tcolorbox}[colback=yellow!5!white, colframe=yellow!60!black,
    title=\textbf{L2 Verdict: Plausible Form, Open Coefficient}]
\textbf{Status:} \tagP{} (postulated form)

\textbf{What is closed:}
\begin{itemize}[nosep]
    \item Dimensional structure: $[\ell] = [\hbar c / (\sigma r_e^2)] = [E]^{-1}$ \checkmark
    \item Membrane origin: uses only $(\sigma, r_e)$, no weak-sector input
\end{itemize}

\textbf{What is open:}
\begin{itemize}[nosep]
    \item The geometric factor $f_{\text{geom}} \approx 10^{-3}$ is unexplained
    \item Without deriving $f_{\text{geom}}$, this is calibration in disguise
\end{itemize}

\textbf{Would be [Dc] if:} $f_{\text{geom}}$ were derived from brane geometry
(e.g., ratio of curvature scales, topological factors).
\end{tcolorbox}

\paragraph{\texorpdfstring{Alternative: Direct $R_\xi$ identification.}{Alternative: Direct R-xi identification.}}
Another candidate sets $\ell \sim R_\xi$ directly, where $R_\xi \sim 10^{-3}$ fm
is the diffusion correlation length from Part I. This gives:
\begin{equation}
    \ell \sim R_\xi \sim 10^{-3} \text{ fm}
    \quad\Rightarrow\quad
    m_\phi \sim \frac{\pi}{\ell} \sim 600 \text{ GeV}
    \label{eq:ch11_ell_Rxi}
\end{equation}
This overshoots $M_W$ by a factor $\sim 8$. The discrepancy suggests either
(a) $R_\xi$ is not the correct scale, or (b) additional geometric factors apply.

% ------------------------------------------------------------------------------
\subsubsection{Candidate G1: $g^2$ from Membrane Tension (Existing Derivation)}
\label{sec:ch11_g5_G1}

\paragraph{The formula.}
An existing derivation in Part II (\S\ref{sec:ch3_weak_coupling}) proposes:
\begin{equation}
    \boxed{
    g^2 = 4\pi \times \frac{\sigma r_e^3}{\hbar c} \approx 0.37
    }
    \label{eq:ch11_g2_membrane}
\end{equation}

\paragraph{Numeric evaluation.}
\begin{align}
    \sigma r_e^2 &= 5.856 \text{ MeV} \tagDc{} \\
    r_e &= 1 \text{ fm} \tagP{} \\
    \hbar c &= 197.3 \text{ MeV}\cdot\text{fm} \tagBL{} \\
    \frac{\sigma r_e^3}{\hbar c} &= \frac{5.856 \times 1}{197.3} \approx 0.0297
\end{align}
Therefore:
\begin{equation}
    g^2_{\text{G1}} = 4\pi \times 0.0297 \approx 0.373
    \label{eq:ch11_g2_G1_value}
\end{equation}

\paragraph{Comparison to SM.}
The SM weak coupling at low energy: $g_2^2 = 4\pi\alpha/\sin^2\theta_W \approx 0.42$.
The G1 prediction is 11\% below SM.

\begin{tcolorbox}[colback=green!5!white, colframe=green!50!black,
    title=\textbf{G1 Verdict: Promising but Requires Justification}]
\textbf{Status:} \tagP{} (postulated, pending derivation audit)

\textbf{What is closed:}
\begin{itemize}[nosep]
    \item Dimensional consistency: $[g^2] = [\sigma r_e^3 / \hbar c] = [E]^0$ \checkmark
    \item No weak-sector input: uses only $(\sigma, r_e)$
    \item Numeric proximity: 11\% from SM value
\end{itemize}

\textbf{What is open:}
\begin{itemize}[nosep]
    \item Why $4\pi$? The coefficient needs derivation from first principles
    \item Is this $g_4^2$ or $g_5^2$? With orthonormal modes, $g_4 = g_5$
    \item RG running: does this predict bare or renormalized coupling?
\end{itemize}

\textbf{Would be [Dc] if:} The $4\pi$ factor were derived from loop integration
or geometric normalization.
\end{tcolorbox}

% ------------------------------------------------------------------------------
\subsubsection{Candidate G2: $g_5^2$ from Inverse Tension}
\label{sec:ch11_g5_G2}

\paragraph{Alternative dimensional ansatz.}
Another candidate from exploratory notes:
\begin{equation}
    g_5^2 \sim \frac{(\hbar c)^2}{\sigma r_e^2}
    \label{eq:ch11_g5_G2}
\end{equation}

\paragraph{Numeric evaluation.}
\begin{equation}
    g_5^2 \sim \frac{(197.3)^2}{5.856} \approx 6650 \text{ MeV}\cdot\text{fm}^2
    = 6650 \text{ (MeV)}^{-1} \times (\hbar c)^2
\end{equation}

\paragraph{Dimensional check.}
$[g_5^2] = [(\hbar c)^2 / E] = [E]^{-1}$, which matches $[g_5]^2 = [E]^{-1}$. \checkmark

\begin{tcolorbox}[colback=red!5!white, colframe=red!60!black,
    title=\textbf{G2 Verdict: Dimensional Match, Numeric Mismatch}]
\textbf{Status:} \tagP{} (exploratory, not closed)

\textbf{Problem:} The G2 formula gives a very large coupling strength. When
combined with the $G_F$ closure spine:
\begin{equation}
    G_F \sim \frac{g_5^2 \ell^2 I_4}{x_1^2}
    \sim \frac{6650 \times (0.04)^2 \times I_4}{\pi^2}
    \sim 1.1 \times I_4 \text{ GeV}^{-2}
\end{equation}
To get $G_F \sim 10^{-5}$ GeV$^{-2}$, we would need $I_4 \sim 10^{-5}$ GeV,
which is unrealistically small (sub-eV localization scale).

\textbf{Verdict:} G2 is dimensionally valid but numerically inconsistent.
Not pursued further.
\end{tcolorbox}

% ------------------------------------------------------------------------------
\subsubsection{Combined Closure Test: G1 + L1}
\label{sec:ch11_closure_test}

Using the most promising candidates (G1 for $g^2$, L1 for $\ell$):

\paragraph{Input values.}
\begin{align}
    g^2 &= 0.373 \quad \text{(G1)} \\
    \ell &= 0.04 \text{ fm} \approx 0.2 \text{ GeV}^{-1} \quad \text{(L1, via $M_W$)} \\
    x_1 &= \pi \quad \text{(Dirichlet BC)} \\
    I_4 &= ? \quad \text{(requires BVP solution)}
\end{align}

\paragraph{Required overlap integral.}
From the closure spine Eq.~\eqref{eq:ch11_closure_spine}:
\begin{equation}
    G_F = \frac{g^2 \ell^2 I_4}{x_1^2}
    \quad\Rightarrow\quad
    I_4 = \frac{G_F \, x_1^2}{g^2 \ell^2}
\end{equation}
With $G_F = 1.17 \times 10^{-5}$ GeV$^{-2}$:
\begin{equation}
    I_4 = \frac{1.17 \times 10^{-5} \times \pi^2}{0.373 \times (0.2)^2}
    \approx \frac{1.15 \times 10^{-4}}{0.015}
    \approx 7.7 \times 10^{-3} \text{ GeV}
    \label{eq:ch11_I4_required}
\end{equation}

\paragraph{Physical interpretation.}
The required $I_4 \approx 8$ MeV has the \emph{same dimension as a localization scale}
(both are Energy in natural units). For an exponential profile $f_L(\xi) = \sqrt{2m_0}\,e^{-m_0\xi}$,
the exact relation is $I_4 = m_0$ (see \S\ref{sec:ch3_electroweak}), so:
\begin{equation}
    m_0 \equiv I_4 \approx 8 \text{ MeV} \quad \Rightarrow \quad
    \text{localization length} \sim m_0^{-1} \approx 25 \text{ fm}
\end{equation}
This is consistent with tight localization within the brane layer ($\delta \sim R_\xi \sim 10^{-3}$ fm).

\begin{tcolorbox}[colback=yellow!5!white, colframe=yellow!60!black,
    title=\textbf{Combined Test Verdict: Consistency Check Passes}]
\textbf{Status:} Numerically consistent \emph{but} relies on L1 (identification).

The required $I_4 \approx 8$ MeV is physically reasonable for tightly
localized fermion modes. However:
\begin{itemize}[nosep]
    \item L1 uses $M_W$ as input (forbidden for first-principles claim)
    \item $I_4$ is not computed, only back-solved
    \item This is a \emph{consistency check}, not a derivation
\end{itemize}
\end{tcolorbox}

% ------------------------------------------------------------------------------
\subsubsection{No-Smuggling Audit}
\label{sec:ch11_no_smuggling_value}

\begin{table}[ht]
\centering
\caption{No-smuggling audit for value closure candidates}
\label{tab:ch11_no_smuggling_value}
\small
\begin{tabular}{lcl}
\toprule
\textbf{Candidate} & \textbf{SM-Free?} & \textbf{Issue} \\
\midrule
L1: $\ell = \pi/M_W$ & \textcolor{red}{\ding{55}} & Uses $M_W$ as input \\
L2: $\ell = (\hbar c/\sigma r_e^2) \times f_{\text{geom}}$ & \textcolor{yellow!80!black}{$\sim$} & $f_{\text{geom}}$ undetermined \\
G1: $g^2 = 4\pi \sigma r_e^3/\hbar c$ & \textcolor{green!60!black}{\ding{51}} & No SM input; $4\pi$ unexplained \\
G2: $g_5^2 = (\hbar c)^2/\sigma r_e^2$ & \textcolor{green!60!black}{\ding{51}} & No SM input; but numerically fails \\
\bottomrule
\end{tabular}
\end{table}

\paragraph{Honest assessment.}
Only G1 passes the no-smuggling test \emph{and} gives a plausible numeric value.
For $\ell$, no SM-free candidate currently yields the weak scale without
additional unexplained factors.

% ------------------------------------------------------------------------------
\subsubsection{OPR-19/20 Value Closure: Summary}
\label{sec:ch11_value_closure_summary}

\begin{tcolorbox}[colback=red!5!white, colframe=red!60!black,
    title=\textbf{OPR-19/20 Value Closure Status: RED-C [OPEN]}]

\textbf{Before this section:}
\begin{quote}
OPR-19/20: RED-C --- Forms derived ($g_4 = g_5$, $m_\phi = x_1/\ell$); values open.
\end{quote}

\textbf{After this section:}
\begin{quote}
OPR-19/20: \textbf{RED-C} [OPEN] --- Value candidates surveyed; no closure achieved.
\begin{itemize}[nosep]
    \item \textbf{G1 ($g^2 = 4\pi\sigma r_e^3/\hbar c$):} Promising; 11\% from SM;
          coefficient $4\pi$ needs derivation. Status: \tagP{}.
    \item \textbf{L1 ($\ell = \pi/M_W$):} Uses SM input; forbidden for first-principles.
          Status: \tagI{}.
    \item \textbf{L2 ($\ell \sim \hbar c/\sigma r_e^2 \times f$):} SM-free form but
          coefficient $f \sim 10^{-3}$ unexplained. Status: \tagP{}.
\end{itemize}
\end{quote}

\medskip
\noindent\fbox{\parbox{0.94\textwidth}{\small
\textbf{Honest verdict:} The value closure attempt has not succeeded. G1 provides
a plausible membrane-based coupling, but $\ell$ remains tied to SM identification
or unexplained geometric factors. Full closure requires:
\begin{enumerate}[nosep]
    \item Derive the $4\pi$ coefficient in G1 from first principles
    \item Derive $\ell$ from membrane geometry without importing $M_W$
    \item Or: derive $M_W$ itself from $(\sigma, r_e)$
\end{enumerate}}}
\end{tcolorbox}

% ------------------------------------------------------------------------------
\subsubsection{What Would Constitute Closure}
\label{sec:ch11_what_would_close}

\begin{table}[ht]
\centering
\caption{Closure requirements for OPR-19/20 values}
\label{tab:ch11_closure_requirements}
\begin{tabular}{lll}
\toprule
\textbf{Item} & \textbf{Currently} & \textbf{To Close} \\
\midrule
$g^2$ value & $4\pi\sigma r_e^3/\hbar c$ \tagP{} & Derive $4\pi$ from loop/geometry \\
$\ell$ value & $\pi/M_W$ \tagI{} & Derive from $(\sigma, r_e)$ only \\
$f_{\text{geom}}$ & Fitted $\approx 10^{-3}$ & Derive from brane curvature \\
BCs (N/D/mixed) & Assumed & Derive from consistency \\
\bottomrule
\end{tabular}
\end{table}

\paragraph{Critical path.}
The most promising route to closure is:
\begin{enumerate}[nosep]
    \item Use G1 as the $g^2$ formula (already close to SM)
    \item Derive $f_{\text{geom}}$ from the ratio $R_\xi / r_e$ or similar
    \item Alternatively: solve the full BVP (OPR-21) and extract $I_4$, then
          back-check consistency without using $M_W$
\end{enumerate}

\paragraph{Failure mode.}
If the geometric factor $f_{\text{geom}}$ turns out to require fine-tuning
(i.e., no natural origin from brane parameters), then the weak scale would
remain an \emph{environmental} feature of EDC rather than a derived prediction.
This would be an honest limitation, not a flaw in methodology.

