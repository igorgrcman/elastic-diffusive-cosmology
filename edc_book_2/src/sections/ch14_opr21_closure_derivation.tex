%!TEX root = ../EDC_Part_II_Weak_Sector_rebuild.tex
% ==============================================================================
% OPR-21 CLOSURE: FROM 5D DIRAC + ISRAEL JUNCTION TO PHYSICAL BVP
% The complete derivation chain with no step-skipping
% Status: CONDITIONAL [Dc] — structure derived, parameters [P]
% ==============================================================================

\subsection{OPR-21 Closure: From 5D Dirac + Israel Junction to a Physical BVP}
\label{subsec:opr21_closure}

% ------------------------------------------------------------------------------
% EPISTEMIC CONTRACT
% ------------------------------------------------------------------------------

\begin{tcolorbox}[edcGuardrail, title=\textbf{Epistemic Contract: Learning-Style Derivation}]
This section provides a \emph{non-retroactive} derivation chain:
\[
\text{(5D Dirac)} \;\Rightarrow\; \text{(1D mode equations + potentials)} \;\Rightarrow\;
\text{(BCs from junction)} \;\Rightarrow\; \text{(Sturm--Liouville BVP)} \;\Rightarrow\;
N_{\text{bound}}.
\]
The goal is to prevent ``step-skipping'' and later ``patching by memory.''

\textbf{Status convention:}
\begin{itemize}[nosep]
    \item \tagM{}: Mathematical identities (no physics input)
    \item \tagDc{}: Statements that follow \emph{given stated assumptions}
    \item \tagP{}: Profile choices and parameter values (not yet derived from EDC action)
\end{itemize}

\textbf{Evidence files:}
\begin{itemize}[nosep]
    \item \texttt{audit/evidence/OPR21\_VEFF\_DERIVATION\_REPORT.md}
    \item \texttt{audit/evidence/OPR21\_BC\_ISRAEL\_REPORT.md}
    \item \texttt{code/opr21\_bvp\_physical\_run.py}
\end{itemize}
\end{tcolorbox}

% ==============================================================================
% CANONICAL PHYSICAL PATH BOX (OPR-21R / PHYSICAL PATH LOCK)
% ==============================================================================

\begin{tcolorbox}[colback=green!5!white, colframe=green!50!black,
    title=\textbf{Canonical Physical Path (WD) --- Reader Route Map}]
\label{box:ch14_canonical_physical_path}

\textbf{This chapter uses two potential families}. The \textbf{canonical physical path}
is the domain-wall family; the Pöschl--Teller serves only as a benchmark.

\medskip
\begin{center}
\renewcommand{\arraystretch}{1.3}
\begin{tabular}{llcc}
\toprule
\textbf{Path} & \textbf{Potential Family} & \textbf{$\mu_3$ Window} & \textbf{Status} \\
\midrule
\rowcolor{green!10}
\textbf{Physical (WD)} & Domain wall: $V_L = M^2 - M'$ & $[13, 17]$ & \tagDc{} \\
Benchmark (toy) & Pöschl--Teller: $V = -V_0\,\text{sech}^2$ & $[15, 18]$ & \tagM{} \\
\bottomrule
\end{tabular}
\end{center}

\medskip
\textbf{Key principles (OPR-21R):}
\begin{itemize}[nosep]
    \item The three-generation window $\mu_3 = \mu_3(V, \kappa, \rho)$ is \textbf{shape-dependent}.
    \item The deprecated ``universal'' $[25, 35)$ window is a toy artifact --- do \emph{not} use.
    \item All downstream numerics (OPR-22, G$_{\text{eff}}$) should use the \textbf{physical} path.
\end{itemize}

\textbf{If you read only one path}: follow the \colorbox{green!10}{green-highlighted} physical (WD) row.
\end{tcolorbox}

% ==============================================================================
\subsubsection{Step 1: Geometric Setup (Domain and Brane)}
\label{subsubsec:opr21_step1}

\paragraph{Metric ansatz.}
Consider a warped 5D background with one transverse coordinate $\xi$ and a brane
at $\xi = 0$. The metric takes the form \tagP{}:
\begin{equation}
    \boxed{
    ds^2 = e^{2A(\xi)}\eta_{\mu\nu}dx^\mu dx^\nu + d\xi^2, \qquad \xi \in [0, \infty)
    }
    \label{eq:opr21:metric}
\end{equation}
where:
\begin{itemize}[nosep]
    \item $x^\mu$ ($\mu = 0,1,2,3$) are 4D Minkowski coordinates
    \item $\eta_{\mu\nu} = \text{diag}(-1, +1, +1, +1)$ is the 4D Minkowski metric
    \item $A(\xi)$ is the warp factor (dimensionless function of $\xi$)
    \item The brane is located at $\xi = 0$
\end{itemize}

\paragraph{Bulk mass profile.}
We allow a $\xi$-dependent bulk mass $M(\xi)$ for the fermion sector. The profile
form is \tagP{} unless derived from the EDC action (see remaining items in
\S\ref{subsubsec:opr21_remaining}).

\paragraph{Sign convention.}
$A(\xi) > 0$ corresponds to expanding 4D sections as $\xi$ increases (away from brane).

% ==============================================================================
\subsubsection{Step 2: 5D Dirac Equation and Variable Separation}
\label{subsubsec:opr21_step2}

\paragraph{5D Dirac equation.}
The 5D Dirac equation in the warped background is \tagDc{}:
\begin{equation}
    \left[i\Gamma^A D_A - M(\xi)\right] \Psi = 0
    \label{eq:opr21:5ddirac}
\end{equation}
where $\Gamma^A$ are 5D gamma matrices and $D_A$ is the covariant derivative
including spin connection contributions.

\paragraph{Gamma matrix choice.}
For the warped metric~(\ref{eq:opr21:metric}), we use \tagM{}:
\begin{equation}
    \Gamma^\mu = e^{-A(\xi)} \gamma^\mu, \qquad \Gamma^5 = -i\gamma^5
\end{equation}
where $\gamma^\mu$ are standard 4D Dirac matrices and $\gamma^5 = i\gamma^0\gamma^1\gamma^2\gamma^3$.
Convention: $\gamma^5$ eigenvalues $\pm 1$ define chirality (L: $-1$, R: $+1$).

\paragraph{Spin connection.}
The non-vanishing spin connection components for the warped metric are:
\begin{equation}
    \omega_\mu^{a5} = -A'(\xi) e^A e_\mu^a
\end{equation}
where prime denotes $\partial_\xi$ and $e_\mu^a = e^A \delta_\mu^a$ is the vielbein.

\paragraph{Chiral decomposition ansatz.}
Separate the 5D spinor into 4D chiral components and $\xi$-profiles \tagDc{}:
\begin{equation}
    \Psi(x^\mu, \xi) = \psi_L(x^\mu) f_L(\xi) + \psi_R(x^\mu) f_R(\xi)
    \label{eq:opr21:chiral_decomp}
\end{equation}
where $\gamma^5 \psi_{L,R} = \mp \psi_{L,R}$.

\paragraph{4D mass equation.}
For 4D massive modes with mass $m$:
\begin{equation}
    i\gamma^\mu \partial_\mu \psi_{L,R} = m\, \psi_{R,L}
\end{equation}
This couples left and right chiralities as required for massive 4D fermions.

\paragraph{Coupled first-order system.}
Substituting ansatz~(\ref{eq:opr21:chiral_decomp}) into the 5D Dirac equation and
using the 4D mass equation yields coupled equations for the profiles \tagDc{}:
\begin{align}
    \left[\partial_\xi + \bigl(M(\xi) + 2A'(\xi)\bigr)\right] f_L(\xi) &= +m\, e^{-A} f_R(\xi)
    \label{eq:opr21:coupled_L} \\
    \left[\partial_\xi - \bigl(M(\xi) + 2A'(\xi)\bigr)\right] f_R(\xi) &= -m\, e^{-A} f_L(\xi)
    \label{eq:opr21:coupled_R}
\end{align}

\textbf{Key observation:} The warp factor $A(\xi)$ enters through the combination
$\Sigma(\xi) \equiv M(\xi) + 2A'(\xi)$.

% ==============================================================================
\subsubsection{Step 3: Schr\"odinger Form and Partner Potentials (Core Mechanism)}
\label{subsubsec:opr21_step3}

\paragraph{Reduction to second order.}
Define rescaled profiles $F_{L,R} = e^{2A} f_{L,R}$. Eliminating $F_R$ (or $F_L$)
from the coupled system~(\ref{eq:opr21:coupled_L})--(\ref{eq:opr21:coupled_R})
yields Schr\"odinger-type equations \tagDc{}:
\begin{align}
    \left[-\frac{d^2}{d\xi^2} + V_L(\xi)\right] F_L(\xi) &= m^2 F_L(\xi)
    \label{eq:opr21:schrodinger_L} \\
    \left[-\frac{d^2}{d\xi^2} + V_R(\xi)\right] F_R(\xi) &= m^2 F_R(\xi)
    \label{eq:opr21:schrodinger_R}
\end{align}

\paragraph{Partner potentials.}
The effective potentials are \tagDc{}:
\begin{align}
    \boxed{V_L(\xi) = \bigl(M + 2A'\bigr)^2 - \bigl(M + 2A'\bigr)'}
    \label{eq:opr21:VL} \\[6pt]
    \boxed{V_R(\xi) = \bigl(M + 2A'\bigr)^2 + \bigl(M + 2A'\bigr)'}
    \label{eq:opr21:VR}
\end{align}
Expanding explicitly:
\begin{align}
    V_L &= M^2 + 4MA' + 4(A')^2 - M' - 2A'' \\
    V_R &= M^2 + 4MA' + 4(A')^2 + M' + 2A''
\end{align}

\paragraph{Chirality asymmetry.}
The difference between left and right potentials is \tagDc{}:
\begin{equation}
    \boxed{
    V_R(\xi) - V_L(\xi) = 2\bigl(M(\xi) + 2A'(\xi)\bigr)'
    }
    \label{eq:opr21:chirality_gap}
\end{equation}

\paragraph{Flat-space reduction.}
For $A' = 0$ (unwarped extra dimension), this simplifies to \tagM{}:
\begin{equation}
    V_L = M^2 - M', \qquad V_R = M^2 + M'
    \label{eq:opr21:flat_potentials}
\end{equation}
This is the standard domain-wall result. For a domain wall mass profile
$M(\xi) = M_0 \tanh(\xi/a)$, we recover the P\"oschl--Teller-type potential:
\begin{equation}
    V_L = M_0^2 - \frac{M_0}{a}\,\text{sech}^2(\xi/a), \qquad
    V_R = M_0^2 + \frac{M_0}{a}\,\text{sech}^2(\xi/a)
\end{equation}
The left-handed mode sees an attractive well; the right-handed mode sees a barrier.

\paragraph{Supersymmetric QM form.}
With superpotential $\Sigma(\xi) = M(\xi) + 2A'(\xi)$, the potentials take the
supersymmetric quantum mechanics form \tagM{}:
\begin{equation}
    V_L = \Sigma^2 - \Sigma', \qquad V_R = \Sigma^2 + \Sigma'
\end{equation}
This structure guarantees that $V_L$ and $V_R$ have the same spectrum above
their respective thresholds, with the zero modes (if any) differing by chirality.

\begin{tcolorbox}[colback=green!5!white, colframe=green!50!black,
    title=\textbf{Interpretation: Geometric Origin of V--A}]
\label{box:opr21_va_origin}

\textbf{Conditional statement} \tagDc{}:
Given a monotone $M(\xi)$ (domain-wall profile) and the 5D Dirac reduction above,
Eq.~(\ref{eq:opr21:chirality_gap}) shows that left/right sectors experience
different effective barriers/wells.

This \emph{geometric asymmetry} is the origin of chiral selection (V--A) in the
EDC thick-brane picture. The left-handed mode is localized near the brane where
$\Sigma$ is smallest; the right-handed mode is pushed away.

\textbf{Caveat:} This interpretation is conditional on the profile $M(\xi)$ and
warp factor $A(\xi)$ being correctly derived from the EDC action.
\end{tcolorbox}

% ==============================================================================
\subsubsection{Step 4: Boundary Conditions from Israel Junction (Robin BC)}
\label{subsubsec:opr21_step4}

\paragraph{The variational principle approach.}
To derive BCs from physics (not choose ad hoc), we include a brane-localized
fermion mass term at $\xi = 0$ and apply the variational principle.

\paragraph{Brane-localized action.}
The brane-localized action at $\xi = 0$ is \tagP{}:
\begin{equation}
    S_{\text{brane}} = -\int d^4x \sqrt{-g_4} \left[m_b \bar{\Psi}\Psi + \cdots\right]_{\xi=0}
    \label{eq:opr21:brane_action}
\end{equation}
where $m_b$ is a brane-localized mass parameter and $\sqrt{-g_4} = e^{4A(0)}$.

\paragraph{Boundary variation.}
The total variation at $\xi = 0$ (bulk + brane) is:
\begin{equation}
    \delta S\big|_{\xi=0} = \int d^4x \, e^{4A(0)} \left[
        \bar{\Psi}\Gamma^5 \delta\Psi - m_b \bar{\Psi}\delta\Psi
    \right]_{\xi=0}
\end{equation}
For this to vanish for arbitrary $\delta\Psi$, we need the boundary condition.

\paragraph{Derived Robin BC.}
The variational principle combined with $Z_2$-symmetric configuration yields
the Robin boundary condition \tagDc{}:
\begin{equation}
    \boxed{
    f'(0) + \kappa\, f(0) = 0, \qquad \kappa = \frac{m_b}{2}
    }
    \label{eq:opr21:robin_bc}
\end{equation}

\textbf{Derivation status:} Eq.~(\ref{eq:opr21:robin_bc}) is \emph{not} an assumption---it
is the stationary-action condition \emph{given the stated brane term}.
The structure $\kappa = m_b/2$ is derived \tagDc{}.
The value of $m_b$ remains \tagP{} (linked to OPR-01: $\sigma$ anchor).

\paragraph{Self-adjointness verification.}
For real brane mass $m_b \in \mathbb{R}$, the Robin parameters $(\alpha, \beta) = (\kappa, 1)$
are real, satisfying the self-adjointness criterion (Theorem~\ref{thm:bvp:selfadjoint}).

\paragraph{Special cases.}
\begin{center}
\begin{tabular}{lll}
\toprule
\textbf{$\kappa$ value} & \textbf{BC type} & \textbf{Physical meaning} \\
\midrule
$\kappa = 0$ & Neumann & No brane mass ($m_b = 0$) \\
$\kappa \to \infty$ & Dirichlet & Infinite brane mass (hard wall) \\
$\kappa > 0$ finite & Robin & Finite brane mass (physical case) \\
\bottomrule
\end{tabular}
\end{center}

% ==============================================================================
\subsubsection{Step 5: Sturm--Liouville Statement and Bound States}
\label{subsubsec:opr21_step5}

\paragraph{The eigenproblem.}
Equations~(\ref{eq:opr21:schrodinger_L})--(\ref{eq:opr21:schrodinger_R})
with BC~(\ref{eq:opr21:robin_bc}) define a Sturm--Liouville eigenvalue problem
on $\xi \in [0, \infty)$:
\begin{quote}
Find $m^2$ such that $f(\xi)$ satisfies:
\begin{enumerate}[nosep]
    \item The differential equation $[-d^2/d\xi^2 + V(\xi)]f = m^2 f$
    \item The Robin BC at $\xi = 0$: $f'(0) + \kappa f(0) = 0$
    \item Normalizability at infinity: $\int_0^\infty |f(\xi)|^2 d\xi < \infty$
\end{enumerate}
\end{quote}

\paragraph{Definition of bound states.}
A \emph{bound state} is an eigenfunction normalizable under the $L^2$ inner product
on $[0, \infty)$ \tagM{}. For potentials approaching a constant $V_\infty = M_0^2$
at infinity, bound states have $m^2 < V_\infty$.

\paragraph{Generation count.}
We denote by $N_{\text{bound}}$ the number of bound eigenmodes:
\begin{equation}
    N_{\text{bound}} = \#\{n : m_n^2 < V_\infty \text{ and } f_n \in L^2[0,\infty)\}
    \label{eq:opr21:nbound_def}
\end{equation}
If $N_{\text{bound}} = 3$ emerges robustly, this provides a geometric explanation
for three fermion generations (OPR-02).

% ==============================================================================
\subsubsection{Step 6: Dimensionless Control Parameter}
\label{subsubsec:opr21_step6}

\paragraph{Parameter definition.}
For numerical scanning, we introduce the dimensionless control parameter \tagP{}:
\begin{equation}
    \boxed{\mu := M_0 \ell}
    \label{eq:opr21:mu_def}
\end{equation}
where:
\begin{itemize}[nosep]
    \item $M_0$ is the characteristic bulk mass scale in the profile $M(\xi)$
    \item $\ell$ is the domain/transition length scale
\end{itemize}

\paragraph{Physical interpretation.}
The parameter $\mu$ is a ``compressed'' representation of the physics:
\begin{itemize}[nosep]
    \item $\mu \ll 1$: Shallow well, few or no bound states
    \item $\mu \sim \mathcal{O}(1\text{--}10)$: Intermediate regime
    \item $\mu \gg 1$: Deep well, many bound states
\end{itemize}

\paragraph{Epistemic status.}
At this stage, $\mu$ is \tagP{} postulated. Full closure of OPR-21 requires
deriving $(M_0, \ell)$ from EDC parameters $(\sigma, \Delta, r_e)$---see
\S\ref{subsubsec:opr21_remaining}.

% ==============================================================================
% OPR-01 BRIDGE: Injecting the σ→M₀ anchor
% ==============================================================================
\subsubsection{From $\mu = M_0\ell$ to Geometry: The $\sigma \to M_0$ Anchor (OPR-01)}
\label{subsubsec:opr21_mu_bridge}

\begin{tcolorbox}[colback=blue!3!white, colframe=blue!50!black,
    title=\textbf{OPR-01 Bridge: Parameter Reduction}]
\textbf{Goal:} Express $\mu$ in terms of primitive EDC parameters by
substituting the OPR-01 result $M_0 = f(\sigma, \Delta, y)$.

\textbf{Cross-reference:} See \S\ref{sec:ch15_opr01} for the full derivation
of the $\sigma \to M_0$ relation from domain-wall kink theory.
\end{tcolorbox}

\paragraph{Step-by-step substitution.}
From OPR-01 (\S\ref{sec:ch15_opr01}), the domain-wall kink derivation yields \tagDc{}:
\begin{equation}
    M_0 = \frac{\sqrt{3}}{2}\, y \, \sqrt{\sigma \Delta}
    \label{eq:opr21:M0_from_opr01}
\end{equation}
where the inputs are:
\begin{itemize}[nosep]
    \item $\sigma$: membrane tension \tagP{}
    \item $\Delta$: domain-wall thickness \tagP{}
    \item $y$: Yukawa coupling (dimensionless) \tagP{}
\end{itemize}
The derivation uses scalar kink theory \tagM{} combined with a Yukawa coupling
ansatz \tagP{} --- hence the result is \tagDc{} (conditional on these assumptions).

\paragraph{Expressing $\mu$ in geometric terms.}
Substituting~(\ref{eq:opr21:M0_from_opr01}) into the $\mu$-definition~(\ref{eq:opr21:mu_def}):
\begin{equation}
    \boxed{\mu = M_0 \ell = \frac{\sqrt{3}}{2}\, y \, \sqrt{\sigma \Delta} \cdot \ell}
    \label{eq:opr21:mu_geometric}
\end{equation}

If the domain size is proportional to wall thickness, $\ell = n \Delta$ for some
dimensionless $n$ (assumption (A3)) \tagP{}, then:
\begin{equation}
    \mu = \frac{\sqrt{3}}{2}\, y \, n \, \sqrt{\sigma \Delta^3}
    \label{eq:opr21:mu_sigmaDelta}
\end{equation}

\paragraph{Parameter-count ledger.}
\begin{tcolorbox}[colback=yellow!5!white, colframe=orange!50!black,
    title=\textbf{Parameter Ledger: Before and After OPR-01}]
\label{box:opr21_parameter_ledger}
\textbf{Before OPR-01:}
\begin{center}
\begin{tabular}{lcl}
\toprule
\textbf{Parameter} & \textbf{Status} & \textbf{Note} \\
\midrule
$M_0$ & \tagP{} & Bulk mass amplitude (independent) \\
$\ell$ & \tagP{} & Domain size \\
$\Delta$ & \tagP{} & Wall thickness \\
$\sigma$ & \tagP{} & Membrane tension \\
$y$ & \tagP{} & Yukawa coupling \\
\midrule
\multicolumn{3}{l}{\textit{5 independent [P] parameters controlling $\mu = M_0\ell$}} \\
\bottomrule
\end{tabular}
\end{center}

\textbf{After OPR-01:}
\begin{center}
\begin{tabular}{lcl}
\toprule
\textbf{Parameter} & \textbf{Status} & \textbf{Note} \\
\midrule
$M_0$ & \tagDc{} & $= (\sqrt{3}/2)\, y\, \sqrt{\sigma\Delta}$ (now derived) \\
$\ell$ (or $n = \ell/\Delta$) & \tagP{} & Domain-size ratio \\
$\Delta$ & \tagP{} & Wall thickness \\
$\sigma$ & \tagP{} & Membrane tension \\
$y$ & \tagP{} & Yukawa coupling \\
\midrule
\multicolumn{3}{l}{\textit{4 independent [P] parameters: $(\sigma, \Delta, y, n)$}} \\
\bottomrule
\end{tabular}
\end{center}

\textbf{Reduction:} One parameter ($M_0$) removed from the free set.
$\mu$ is now a \emph{derived function} of $(\sigma, \Delta, y, n)$, not an
independent knob.
\end{tcolorbox}

\paragraph{Three-generation constraint in geometric terms.}
From Step~8 (\S\ref{subsubsec:opr21_step8}), the numerical BVP scan shows
$N_{\text{bound}} = 3$ for $\mu \in [\mu_3^-, \mu_3^+](V)$.
\textbf{Important (OPR-21R):} This window is \emph{shape-dependent}.
For the physical domain wall $V_L = M^2 - M'$: $[\mu_3^-, \mu_3^+] \approx [13, 17]$ \tagDc{}.
Substituting~(\ref{eq:opr21:mu_sigmaDelta}):
\begin{equation}
    13 \leq \frac{\sqrt{3}}{2}\, y \, n \, \sqrt{\sigma \Delta^3} < 17
    \quad \text{(physical domain wall)}
    \label{eq:opr21:Nbound3_geometric}
\end{equation}

Solving for $\sigma \Delta^3$ (assuming $y = 1$, $n = 4$ as reference values):
\begin{equation}
    \boxed{\sigma \Delta^3 \in [14, 24] \quad \text{(physical V, for } y=1, n=4 \text{)}}
    \label{eq:opr21:sigmaD3_constraint}
\end{equation}
\emph{Note: The old value $[52, 102]$ used the deprecated $[25, 35)$ toy benchmark.}

This is a \emph{consistency constraint} \tagDc{}: if EDC parameters $(\sigma, \Delta, y, n)$
satisfy~(\ref{eq:opr21:sigmaD3_constraint}), the BVP admits exactly three bound states.

\paragraph{What remains open for full closure.}
\begin{tcolorbox}[colback=red!5!white, colframe=red!50!black,
    title=\textbf{Remaining Open Items (after OPR-01)}]
\begin{enumerate}[nosep]
    \item \textbf{Derive $\sigma$} from cosmological or gravitational input
          (OPR-01 uses $\sigma$ as primitive \tagP{})
    \item \textbf{Derive $\Delta$} from junction stability or brane microphysics
          (OPR-04 candidate route)
    \item \textbf{Derive $y$} from gauge embedding or naturalness arguments
          (currently order-1 ansatz \tagP{})
    \item \textbf{Derive $n = \ell/\Delta$} from domain-size energetics
          (boundary-layer scale \tagP{})
\end{enumerate}
\textbf{Implication:} Full \tagDer{} closure requires closing all four items above.
Current status remains \tagDc{} conditional on the [P] choices.
\end{tcolorbox}

% ==============================================================================
\subsubsection{Step 7: Numerical Solver and Reproducibility}
\label{subsubsec:opr21_step7}

\paragraph{Solver implementation.}
We solve the Sturm--Liouville problem numerically using finite differences
on a truncated domain $\xi \in [0, \ell]$.

\paragraph{Method summary.}
\begin{enumerate}[nosep]
    \item Discretize $\xi$ on uniform grid with $N$ points
    \item Build tridiagonal Hamiltonian matrix: $H = -d^2/d\xi^2 + V(\xi)$
    \item Incorporate Robin BC via ghost point method
    \item Solve eigenvalue problem using standard routines (scipy.linalg)
    \item Count eigenvalues below threshold $V_\infty = M_0^2$
\end{enumerate}

\paragraph{Reproducibility instructions.}
\begin{tcolorbox}[colback=gray!5!white, colframe=gray!50!black,
    title=\textbf{Reproducibility: Numerical Pipeline}]
\textbf{Script:} \texttt{code/opr21\_bvp\_physical\_run.py}

\textbf{Command:}
\begin{verbatim}
cd edc_book_2/code && python3 opr21_bvp_physical_run.py
\end{verbatim}

\textbf{Outputs:}
\begin{itemize}[nosep]
    \item \texttt{code/output/opr21\_physical\_summary.json}
    \item \texttt{code/output/opr21\_physical\_phase\_diagram.md}
    \item \texttt{code/output/opr21\_physical\_robustness\_table.md}
\end{itemize}

\textbf{Note:} This section reports \emph{exactly} what the machine outputs.
It does \emph{not} claim analytic closure from numerics.
\end{tcolorbox}

% ==============================================================================
\subsubsection{Step 8: Numerical Results (Scan, Transitions, Three-Mode Window)}
\label{subsubsec:opr21_step8}

\paragraph{Parameter scan.}
The scan covers $\mu \in [0.5, 100]$ for the domain-wall potential
$M(\xi) = M_0 \tanh((\xi - \ell/2)/\Delta)$ with wall-to-domain ratio
$\rho := \Delta/\ell = 0.1$.%
\footnote{We use $\rho$ for this dimensionless ratio to avoid confusion with
the boundary-layer scale $\delta$; see Scale Taxonomy, \S\ref{sec:ch16_reader_map}.}

\paragraph{Observed transition bands.}
\begin{center}
\begin{tabular}{ccc}
\toprule
\textbf{Transition} & \textbf{$\mu$ range} & \textbf{$N_{\text{bound}}$ change} \\
\midrule
$0 \to 1$ & $\mu \sim [2, 3]$ & First bound state appears \\
$1 \to 2$ & $\mu \sim [10, 15]$ & Second bound state \\
$2 \to 3$ & $\mu \sim [20, 25]$ & Third bound state \\
$3 \to 4$ & $\mu \sim [35, 40]$ & Fourth bound state \\
\bottomrule
\end{tabular}
\end{center}

\paragraph{Three-bound-state window.}
The scan reports a \textbf{three-generation window}:
\begin{equation}
    \boxed{N_{\text{bound}} = 3 \quad \text{for} \quad \mu \in [25, 35) \quad \text{(TOY BENCHMARK)}}
    \label{eq:opr21:three_gen_window}
\end{equation}
This window is flagged as ``PROMISING'' and stable in the tested regime.

\begin{tcolorbox}[colback=yellow!10!white, colframe=yellow!70!black,
    title=\textbf{OPR-21R Update: $\mu$-Window is Shape-Dependent}]
\label{box:opr21_mu_shape_dependent}

\textbf{CRITICAL}: The $[25, 35)$ window is a \textbf{toy benchmark}, not universal.

The three-generation condition $N_{\text{bound}} = 3$ is achieved at a shape-dependent
critical value $\mu_3(V, \kappa, \rho)$:

\begin{center}
\begin{tabular}{lcc}
\toprule
\textbf{Potential} & $\mu_3$ & $N_{\text{bound}}=3$ window \\
\midrule
Toy (P\"oschl-Teller) & 15 & $[15, 18]$ \\
Physical (Domain Wall) & 13 & $[13, 17]$ \\
\bottomrule
\end{tabular}
\end{center}

\textbf{Correct statement}: ``Three generations require $\mu \in [\mu_3^-, \mu_3^+]$
where the window depends on $V(\xi)$.''

See \texttt{audit/evidence/OPR21R\_MU\_WINDOW\_SHAPE\_DEPENDENCE\_REPORT.md}
and \texttt{code/opr21r\_mu3\_scan.py} for full analysis.

\end{tcolorbox}

\begin{tcolorbox}[colback=red!3!white, colframe=red!50!black,
    title=\textbf{Critical Clarification: What $\mu$ Constrains}]
\textbf{The OPR-21 constraint is $\mu = M_0 \ell$, NOT $M_0 \Delta$.}

\medskip
The domain size $\ell$ (Sturm--Liouville interval) is \emph{a priori} distinct
from the kink width $\Delta$ (scalar wall microphysics). If one assumes
$\ell = n\Delta$ with $n \sim \mathcal{O}(1)$, this conflates two independent
scales. But $\ell$ could be much larger than $\Delta$ (e.g., $\ell \gg \Delta$
if the spectral domain extends beyond the kink core).

\medskip
\textbf{See Chapter~16, \S\ref{sec:ch16_reader_map} for the canonical Scale
Taxonomy distinguishing $\Delta$, $\delta$, $\ell$, and $R_\xi$.}
\end{tcolorbox}

\begin{tcolorbox}[colback=green!5!white, colframe=green!50!black,
    title=\textbf{Interpretation: Three Generations from Thick-Brane BVP}]
\label{box:opr21_three_gen}

\textbf{Conditional statement} \tagDc{}:
Given the derived $V_L$ structure~(\ref{eq:opr21:VL}) and derived Robin BC~(\ref{eq:opr21:robin_bc}),
a thick-brane BVP can naturally produce $N_{\text{bound}} = 3$ without SM observable input.

\textbf{What this is:}
\begin{itemize}[nosep]
    \item A physically realized possibility (not a forced fit)
    \item Proof that the thick-brane mechanism \emph{can} yield three generations
    \item A shape-dependent window: $\mu \in [\mu_3^-, \mu_3^+](V)$ (see Box~\ref{box:opr21_mu_shape_dependent})
\end{itemize}

\textbf{What this is NOT:}
\begin{itemize}[nosep]
    \item A derivation of \emph{why} $\mu$ falls in this range (that requires OPR-01)
    \item A derivation of generation masses/mixings (that requires CKM/PMNS overlaps)
    \item Calibrated to any SM observable
\end{itemize}
\end{tcolorbox}

% ==============================================================================
\subsubsection{Step 9: Robustness}
\label{subsubsec:opr21_step9}

\paragraph{BC variation scan.}
Robustness tests vary the Robin parameter $\kappa$ across the range $[0, 2]$
and domain size by factors $[0.8, 1.5]$.

\paragraph{Result.}
The count $N_{\text{bound}}$ is \textbf{stable} across BC variations in the
target regime (shape-dependent; see Box~\ref{box:opr21_mu_shape_dependent}).
This addresses the ``knife-edge tuning'' objection at the level of mode count.

\paragraph{Grid convergence.}
With 1000 grid points, eigenvalue variation is $< 0.01\%$, confirming numerical
convergence.

% ==============================================================================
\subsubsection{What This Unlocks (and What Remains OPEN)}
\label{subsubsec:opr21_remaining}

\paragraph{Upgraded by this closure.}
\tagDc{} (conditional on derivation chain above):
\begin{itemize}[nosep]
    \item \textbf{V--A origin:} Chirality asymmetry follows from
          $V_R - V_L = 2(M + 2A')'$ once the 5D reduction is accepted.
    \item \textbf{Three generations:} Bound-state count becomes a physically
          realized possibility in the BVP (mode count, not yet masses/mixings).
    \item \textbf{Robin BC structure:} The form $\kappa = m_b/2$ is derived,
          not chosen ad hoc.
\end{itemize}

\paragraph{Still required for full closure.}
\begin{enumerate}[nosep]
    \item \textbf{Derive $M_0$ from $\sigma$} (membrane tension)---linked to OPR-01
    \item \textbf{Derive $\Delta$} from transition-layer physics (profile microphysics)
    \item \textbf{Derive $\ell$} from the domain-size principle or other EDC input
    \item \textbf{Show $\mu \in [\mu_3^-, \mu_3^+](V)$} follows from EDC parameters
          \emph{without calibration} (shape-dependent window)
\end{enumerate}

\begin{tcolorbox}[colback=red!5!white, colframe=red!50!black,
    title=\textbf{No-Smuggling Checklist (OPR-21)}]
\label{box:opr21_no_smuggling}

\begin{enumerate}[nosep]
    \item[$\checkmark$] Are any SM observables ($M_W$, $G_F$, $v$, $\sin^2\theta_W$)
          used to set $\mu$? \textbf{NO}---$\mu$ is scanned, not calibrated.

    \item[$\checkmark$] Is the domain $\xi \in [0, \infty)$ stated wherever a
          half-line integral/BC is used? \textbf{YES}---see Step~1.

    \item[$\checkmark$] Are $V_L, V_R$ traced to the 5D Dirac reduction, not guessed?
          \textbf{YES}---derived in Step~3.

    \item[$\checkmark$] Is Robin BC traced to the brane term via variation, not
          imposed post hoc? \textbf{YES}---derived in Step~4.

    \item[$\checkmark$] Are numerical results reported as numerical (not promoted
          to analytic certainty)? \textbf{YES}---Step~7 explicitly states this.
\end{enumerate}
\end{tcolorbox}

% ==============================================================================
\subsubsection{OPR-21 Status Summary}
\label{subsubsec:opr21_status_summary}

\begin{center}
\begin{tabular}{lll}
\toprule
\textbf{Component} & \textbf{Status} & \textbf{Evidence} \\
\midrule
$V_L, V_R$ structure & DERIVED \tagDc{} & Eqs.~(\ref{eq:opr21:VL})--(\ref{eq:opr21:VR}) \\
Robin BC structure & DERIVED \tagDc{} & Eq.~(\ref{eq:opr21:robin_bc}) \\
Chirality asymmetry & DERIVED \tagDc{} & Eq.~(\ref{eq:opr21:chirality_gap}) \\
$N_{\text{bound}} = 3$ window & COMPUTED & Eq.~(\ref{eq:opr21:three_gen_window}) \\
$M_0 = f(\sigma, \Delta, y)$ & \tagDc{} DERIVED & OPR-01 CLOSED (\S\ref{sec:ch15_opr01}) \\
$(\sigma, \Delta, \ell, y)$ values & \tagP{} POSTULATED & Remaining primitives \\
\midrule
\textbf{Overall OPR-21} & \textbf{CONDITIONAL} \tagDc{} & Structure derived, parameters open \\
\bottomrule
\end{tabular}
\end{center}

% ==============================================================================
% PHYSICAL μ-SWEEP RESULTS BOX (OPEN-22-4b)
% ==============================================================================

\begin{tcolorbox}[colback=blue!5!white, colframe=blue!50!black,
    title=\textbf{Box 14.2: Physical $\mu$-Sweep Results --- Green Path Family}]
\label{box:ch14_physical_mu_sweep}

\textbf{Physical domain wall potential}: $V_L = M^2 - M'$ \tagDc{}

\textbf{NON-UNIVERSALITY}: All bands below are \textbf{CONDITIONAL} on:
\begin{itemize}[nosep]
    \item Potential family: $V_L = M^2 - M'$ (domain wall from 5D Dirac)
    \item Wall-to-domain ratio: $\rho = \Delta/\ell = 0.2$
\end{itemize}
\emph{Different shapes or $\rho$ values give different windows. No universal claims.}

\medskip
\textbf{Green Path = Physical BC Family} (OPEN-22-4b-R-PHYS):
\begin{center}
\renewcommand{\arraystretch}{1.3}
\begin{tabular}{llccc}
\toprule
\textbf{Path} & \textbf{BC Type} & \textbf{$N_{\text{bound}}=3$ Window} & \textbf{Gates} & \textbf{Status} \\
\midrule
\rowcolor{green!15}
\textbf{Green-A} & Neumann ($\hat\kappa = 0$) & $\mu \in [13.0, 15.6]$ & 5/5 PASS & \textbf{CANONICAL} \\
\rowcolor{green!10}
\textbf{Green-B} & Robin ($\hat\kappa = 0.5$) & $\mu \in [13.0, 15.2]$ & 5/5 PASS & CANONICAL \\
\rowcolor{green!10}
\textbf{Green-B} & Robin ($\hat\kappa = 1.0$) & $\mu \in [13.0, 14.8]$ & 5/5 PASS & CANONICAL \\
\rowcolor{green!10}
\textbf{Green-B} & Robin ($\hat\kappa = 2.0$) & $\mu \in [13.0, 13.6]$ & 5/5 PASS & CANONICAL (narrow) \\
\rowcolor{red!10}
--- & Robin ($\hat\kappa \geq 3$) & --- & FAIL & $N_{\text{bound}} \geq 4$ \\
\bottomrule
\end{tabular}
\end{center}

\medskip
\textbf{Green-B Gate Summary} (Robin BC $\hat\kappa > 0$):
\begin{center}
\renewcommand{\arraystretch}{1.2}
\begin{tabular}{llc}
\toprule
\textbf{Gate} & \textbf{Criterion} & \textbf{Status} \\
\midrule
METHOD & FEM weak formulation (no FD ghost-point) & \checkmark PASS \\
CONVERGENCE & Max drift $<1\%$ (0.22\% achieved) & \checkmark PASS \\
SPECTRUM & $N_{\text{bound}} = 3$ for $\hat\kappa \in \{0.5, 1, 2\}$ & \checkmark PASS \\
CONTINUITY & $\hat\kappa \to 0$ reproduces Neumann within 5\% & \checkmark PASS \\
NO-SMUGGLING & No SM observables used & \checkmark PASS \\
\bottomrule
\end{tabular}
\end{center}

\medskip
\textbf{Detailed Results} ($\mu = 14.0$, $\rho = 0.2$):
\begin{center}
\renewcommand{\arraystretch}{1.2}
\begin{tabular}{cccc}
\toprule
$\hat\kappa$ & $|f_1(0)|^2$ & $G_{\text{eff}}/(g_5^2\ell)$ & $N_{\text{bound}}$ \\
\midrule
\rowcolor{green!15}
0.0 (Neumann) & $7.12 \times 10^{-4}$ & $7.92 \times 10^{-3}$ & 3 \\
\rowcolor{green!10}
0.5 & $7.68 \times 10^{-4}$ & $8.93 \times 10^{-3}$ & 3 \\
\rowcolor{green!10}
1.0 & $8.30 \times 10^{-4}$ & $1.01 \times 10^{-2}$ & 3 \\
2.0 & $9.79 \times 10^{-4}$ & $3.12 \times 10^{-2}$ & 4 \\
\bottomrule
\end{tabular}
\end{center}

\medskip
\textbf{Key findings}:
\begin{itemize}[nosep]
    \item \colorbox{green!15}{Green-A (Neumann)}: Primary canonical path, widest $\mu$-window
    \item \colorbox{green!10}{Green-B (Robin $\hat\kappa \leq 2$)}: Canonical family members, narrower windows
    \item $|f_1(0)|^2$ \textbf{increases} with $\hat\kappa$ (Robin does NOT decouple modes from brane)
    \item $\hat\kappa \geq 3$: Incompatible with 3-generation constraint in $\mu \in [13, 17]$
\end{itemize}

\textbf{Evidence}: \texttt{code/open22\_4bR\_phys\_robin\_sweep.py},
\texttt{audit/evidence/OPEN22\_4bR\_PHYSICAL\_ROBIN\_AUDIT.md}

\textbf{No-smuggling}: \checkmark No SM observables ($M_W$, $G_F$, $v$, $\sin^2\theta_W$) used as inputs.
\end{tcolorbox}

% ==============================================================================
% ROBIN BC CORRECTION NOTE (OPEN-22-4b-FD)
% ==============================================================================

\begin{tcolorbox}[colback=green!5!white, colframe=green!50!black,
    title=\textbf{Correction Note: Robin BC Now Canonical (OPEN-22-4b-R-PHYS DONE)}]
\label{box:ch14_robin_correction}

Earlier Robin-BC results showing apparent decoupling ($|f_1(0)|^2 \to 0$ for $\kappa > 0$)
were an artifact of an incorrect FD ghost-point discretization. A weak-form FEM
implementation confirms that $\hat{\kappa}$ \textbf{does} affect both eigenvalues and
brane overlap.

\textbf{OPEN-22-4b-R-PHYS Resolution} (2026-01-26):
\begin{itemize}[nosep]
    \item Robin BC ($\hat\kappa \leq 2$) now part of \textbf{Green Path} canonical family
    \item All 5 gates PASS: METHOD, CONVERGENCE, SPECTRUM, CONTINUITY, NO-SMUGGLING
    \item $N_{\text{bound}} = 3$ achieved for $\hat\kappa \in \{0, 0.5, 1.0, 2.0\}$
    \item Higher $\hat\kappa$ ($\geq 3$) gives $N_{\text{bound}} \geq 4$ (incompatible)
\end{itemize}

\textbf{Evidence}: \texttt{audit/evidence/OPEN22\_4bR\_PHYSICAL\_ROBIN\_AUDIT.md},
\texttt{code/open22\_4bR\_phys\_robin\_sweep.py}
\end{tcolorbox}
