% ==============================================================================
% Section 1.7: Case Study — Charged Leptons (Muon, Tau, Electron)
% ==============================================================================

\section{Case Study: Charged Leptons}
\label{sec:case_leptons}

The charged leptons---electron, muon, and tau---share a common EDC ontology as
\emph{brane-dominant excitations}. This section treats them together to highlight
both their universality (same pipeline mechanism) and their differences (mode index,
decay channels, lifetime hierarchy).

\subsection{Muon Decay: Brane-Dominant Mode Relaxation}
\label{subsec:muon_story}
\label{sec:case_muon}  % alias for cross-references

% --- AT-A-GLANCE BOX (KB-CANON-002) ---
\begin{edcAtAGlance}{Muon Decay}
  \edcBaseline{
    Decay: $\mu^- \to e^- + \bar\nu_e + \nu_\mu$ with $\tau_\mu = 2.197 \times 10^{-6}$ s\\
    Purely leptonic (no hadrons, no quark flavor change)\\
    Energy: $m_\mu c^2 = 105.7$ MeV available\\
    Coupling: Same V$-$A structure, $G_F$ from $W$ exchange
  }
  \edcEDCView{
    Muon = brane-dominant excitation (NOT a bulk junction like neutron)\\
    Decay is internal mode relaxation within brane layer\\
    No bulk$\to$brane pumping---energy already in brane\\
    Chiral filter $\mathcal{P}_{\mathrm{chir}}$ selects helicity at boundary
  }
  \edcKeyInsight{
    The muon is a ``clean room'' test: same pipeline as neutron but without
    bulk-core complications. If EDC works here, it's not tuned to baryons.
    The V$-$A structure may be geometric (boundary orientation), not intrinsic.
  }
  \edcFalsifiable{
    \textbullet\ If $\mu \to e\gamma$ observed at significant rate (LFV)\\
    \textbullet\ If chirality pattern cannot arise from boundary geometry\\
    \textbullet\ If ledger doesn't close (energy to unknown channel)
  }
\end{edcAtAGlance}

\medskip

% ==============================================================================
% MOTIVATION: WHY MUON DECAY?
% ==============================================================================

\subsubsection{Motivation: Why Muon Decay Matters for EDC}

\begin{tcolorbox}[edcCornerstone, title=\textbf{Cornerstone: Muon as Brane Tomography}]
Muon decay ($\mu^- \to e^- + \nu_\mu + \bar{\nu}_e$) is a \emph{purely leptonic}
weak process. It involves no baryonic topology, no quark flavor transitions,
and no hadronic complications. If the thick-brane microphysics pipeline
works for muon decay, this provides strong evidence that the framework is
not merely ``tuned'' to neutron phenomenology.
\end{tcolorbox}

\paragraph{Strategic value.}
The neutron decay analysis in \S\ref{sec:case_neutron}
established the bulk$\to$brane$\to$3D pipeline for a \emph{junction state}.
The neutron's complexity (three-arm Y-junction, baryonic topology, quark
flavor change) means multiple mechanisms operate simultaneously.

Muon decay strips away these complications:
\begin{itemize}[nosep]
    \item No bulk-core junction geometry \tagP{}
    \item No baryonic number conservation requirement
    \item Pure brane-layer dynamics \tagP{}
    \item Same final output structure: $e^-$ + neutrinos
\end{itemize}

\textbf{Physical Narration:}
\begin{enumerate}[nosep]
    \item \textbf{5D cause:} The muon, as a brane-dominant excitation, occupies
          an unstable mode within the brane layer.
    \item \textbf{Brane response:} Mode redistribution occurs entirely within
          the brane layer—no bulk junction relaxation.
    \item \textbf{3D output:} The frozen projection emits $e^-$, $\nu_\mu$,
          $\bar{\nu}_e$ as allowed outputs.
\end{enumerate}

% ==============================================================================
% THICK-BRANE PIPELINE FOR BRANE-DOMINANT EXCITATIONS
% ==============================================================================

\subsubsection{Three-Layer Ontology (Review)}

The thick-brane framework defines three layers:

\begin{edcDefinitionBox}{Three-Layer Ontology}{}
\begin{enumerate}[nosep]
    \item \textbf{Bulk-core:} 5D interior, $y < -\delta/2$
    \item \textbf{Brane-layer:} Transition region, $y \in [-\delta/2, +\delta/2]$
    \item \textbf{3D outputs:} Observer-facing boundary, $y = +\delta/2$
\end{enumerate}
where $y$ is the extra-dimensional coordinate and $\delta$ is the brane thickness.
\end{edcDefinitionBox}

\subsubsection{Muon as Brane-Dominant Excitation}

Unlike the neutron, the muon does not require a bulk-core junction ontology.
In the EDC Weak Program, the muon is treated as a \emph{brane-dominant excitation}:
a localized brane-layer defect/mode that stores energy primarily in the brane subsystem,
and relaxes via the same three-phase pipeline (absorption $\to$ dissipation $\to$ release),
but with the \emph{bulk trigger removed at leading order}.

\begin{edcPostulateBox}{Muon Ontology}{[P]}
The muon $\mu^-$ is a \emph{brane-dominant excitation}: a localized,
metastable mode whose primary degrees of freedom reside within the
brane layer, not in the bulk-core.
\end{edcPostulateBox}

\textbf{Physical Narration:}
\begin{itemize}[nosep]
    \item \textbf{5D cause:} Unlike the neutron (bulk junction displaced from
          Steiner minimum), the muon is a \emph{brane-layer eigenmode} that
          happens to be unstable.
    \item \textbf{Brane response:} The instability triggers mode redistribution
          \emph{within} the brane layer—energy flows to lower-mass modes
          ($e^-$, neutrinos).
    \item \textbf{3D output:} The frozen projection maps these modes to
          observable particles.
\end{itemize}

\begin{figure}[htbp]
\centering
\begin{tikzpicture}[scale=0.9]
    % Background regions
    \fill[bulk region] (-4,-2) rectangle (-1.5,2);
    \fill[brane region] (-1.5,-2) rectangle (1.5,2);
    \fill[observer region] (1.5,-2) rectangle (4,2);

    % Labels
    \node[section label] at (-2.75,2.4) {\textbf{Bulk-Core}};
    \node[section label] at (0,2.4) {\textbf{Brane-Layer}};
    \node[section label] at (2.75,2.4) {\textbf{3D Outputs}};

    % Boundaries
    \draw[bulk boundary] (-1.5,-2) -- (-1.5,2);
    \draw[observer boundary] (1.5,-2) -- (1.5,2);

    % Muon in brane layer (not in bulk!)
    \node[circle, fill=purple!60, minimum size=12pt, inner sep=0pt] (mu) at (0,0) {};
    \node[below=0.15cm of mu, font=\footnotesize] {$\mu^-$};

    % Annotation: muon is brane-localized
    \draw[dashed, purple!60!black, thick] (-1.3,0) -- (1.3,0);
    \node[font=\scriptsize\itshape, text=purple!60!black] at (0,-0.8) {brane-localized mode};

    % Contrast: neutron would have bulk core
    \node[font=\scriptsize, text=gray] at (-2.75,0) {(no bulk core)};

    % Output particles
    \node[particle] (e) at (3,0.8) {};
    \node[neutrino] (nu1) at (3,0) {};
    \node[neutrino] (nu2) at (3,-0.8) {};
    \node[right=0.1cm of e, font=\scriptsize] {$e^-$};
    \node[right=0.1cm of nu1, font=\scriptsize] {$\nu_\mu$};
    \node[right=0.1cm of nu2, font=\scriptsize] {$\bar{\nu}_e$};

    % Flow arrow
    \draw[edc flow, purple!60!black] (0.3,0) -- (1.3,0);
    \node[font=\scriptsize, above] at (0.8,0.1) {$\mathcal{P}_{\mathrm{frozen}}$};
\end{tikzpicture}
\caption{Muon as brane-dominant excitation. Unlike the neutron (bulk junction),
the muon's degrees of freedom are localized within the brane layer itself.
Decay proceeds via internal mode redistribution, not bulk relaxation.}
\label{fig:muon-ontology}
\end{figure}

This makes $\mu$-decay a clean test of the brane mechanism because:
\begin{enumerate}[nosep]
  \item[(i)] there is no ambiguity of bulk-core topology,
  \item[(ii)] the output channel is experimentally sharp, and
  \item[(iii)] the chirality structure is a strong selection signature.
\end{enumerate}

% ==============================================================================
% CONTRAST WITH NEUTRON
% ==============================================================================

\subsubsection{Contrast with Neutron: Ontology Comparison}

\begin{table}[htbp]
\centering
\caption{Neutron vs.\ Muon in thick-brane ontology \tagBL{}/\tagP{}}
\label{tab:neutron-muon-contrast}
\begin{tabular}{lcc}
\toprule
\textbf{Property} & \textbf{Neutron} & \textbf{Muon} \\
\midrule
Primary location & Bulk-core (junction) & Brane-layer \\
Decay trigger & Junction relaxation & Mode instability \\
Baryonic topology & Yes (3-arm junction) & No \\
Quark flavor change & Yes ($d \to u$) & No \\
Output particles & $p + e^- + \bar{\nu}_e$ & $e^- + \nu_\mu + \bar{\nu}_e$ \\
Lifetime & 878.4~s \tagBL{} & $2.197 \times 10^{-6}$~s \tagBL{} \\
Bulk$\to$brane pumping & Yes & No (suppressed) \\
\bottomrule
\end{tabular}
\end{table}

% ==============================================================================
% PIPELINE FOR MUON DECAY
% ==============================================================================

\subsubsection{Pipeline for Muon Decay}

We reuse the unified pipeline:
\begin{equation}
\label{eq:mu_pipeline}
\Psi_\mu
\;\Rightarrow\;
E_{\mathrm{brane}} \text{ (stored)}
\;\Rightarrow\;
\Gamma_{\mathrm{eff}} \text{ (redistribution)}
\;\Rightarrow\;
\mathcal{P}_{\mathrm{frozen}} \text{ (3D outputs)}.
\end{equation}

\paragraph{Absorption / charging \tagDc{}.}
For a brane-dominant excitation, the ``absorption'' stage is not pumping from bulk,
but simply the existence of stored brane energy in the excited configuration:
\begin{equation}
\label{eq:mu_brane_energy}
E_{\mathrm{brane}}(t_0) \approx m_\mu c^2 \approx 105.7~\text{MeV} \quad \text{\tagBL{}},
\end{equation}
up to small corrections (soft, recoil, residual leakage) (open).

\paragraph{Dissipation (mode redistribution) \tagDc{}/\tagP{}.}
We use the same phenomenological release-rate definition:
\begin{equation}
\label{eq:mu_release_power}
\Pi_{\mathrm{release}}(t) \equiv \Gamma_{\mathrm{eff}}\,E_{\mathrm{brane}}(t),
\end{equation}
where $\Gamma_{\mathrm{eff}}$ must ultimately be derived from thick-brane microphysics (open).

\paragraph{Release map (allowed outputs)/\tagDc{}.}
The frozen projection maps brane-layer modes into allowed 3D outputs:
\begin{equation}
\label{eq:mu_release}
E_{\mathrm{brane}}
\;\xrightarrow{\;\mathcal{P}_{\mathrm{frozen}}\;}
e^- + \bar\nu_e + \nu_\mu + \text{(soft/recoil)}.
\end{equation}

% ==============================================================================
% ENERGY BOOKKEEPING LEDGER
% ==============================================================================

\subsubsection{Energy Bookkeeping Ledger}

\begin{tcolorbox}[edcCanonical, title=\textbf{Canonical: Energy Conservation in Muon Decay}]
The total energy released in muon decay must be accounted for:
\begin{equation}
    m_\mu c^2 = E_{e^-} + E_{\nu_\mu} + E_{\bar{\nu}_e} + E_{\mathrm{other}}
    \label{eq:mu_ledger_full}
\end{equation}
where $E_{\mathrm{other}}$ captures any residual energy (recoil, soft radiation, etc.).
\end{tcolorbox}

Following the pattern established for neutron decay, we decompose the energy flow:

\begin{edcDefinitionBox}{Energy Partition for Muon Decay}{}
\begin{align}
    \Delta E_{\mathrm{brane}} &= E_{e^-} + E_{\nu_\mu} + E_{\bar{\nu}_e} \tag{released to 3D} \\
    E_{\mathrm{other}} &= E_{\mathrm{soft}} + E_{\mathrm{brane\,residual}} \tag{not frozen}
\end{align}
\end{edcDefinitionBox}

\textbf{Physical Narration:}
\begin{enumerate}[nosep]
    \item \textbf{5D cause:} The muon mode (energy $m_\mu c^2$) becomes unstable.
    \item \textbf{Brane response:} Energy redistributes into brane-layer modes
          compatible with the frozen projection.
    \item \textbf{3D output:} Modes satisfying the frozen criterion
          ($\hbar\omega \gg E_{\mathrm{env}}$) project to observable particles.
\end{enumerate}

\paragraph{Key difference from neutron.}
In neutron decay, there is a \emph{bulk-core contribution}:
\[
    \Delta E_{\mathrm{bulk}} \to \Delta E_{\mathrm{brane}} + E_{\mathrm{residual}}
\]
For the muon, this bulk term is absent \tagP{}:
\[
    \Delta E_{\mu\,\mathrm{mode}} \to \Delta E_{\mathrm{brane}} + E_{\mathrm{other}}
\]
The muon's energy is already ``in'' the brane layer—no bulk$\to$brane pumping
is required.

\begin{tcolorbox}[edcGuardrail, title=\textbf{Epistemic Guardrail: Suppressed Bulk Leakage}]
\textbf{Suppressed bulk leakage:} In muon decay, we postulate \tagP{} that
leakage of energy back into the bulk-core is suppressed by the muon's
brane-localized mode structure. At leading order, bulk leakage is
treated as negligible. This is consistent with the muon being brane-dominant:
its degrees of freedom are localized within the brane layer.
\end{tcolorbox}

% ==============================================================================
% SELECTION RULES AND ALLOWED OUTPUTS
% ==============================================================================

\subsubsection{Selection Rules and Allowed Outputs}

\paragraph{\texorpdfstring{Why $e^- + \nu_\mu + \bar{\nu}_e$?}{Why e- + nu-mu + nu-e-bar?}}
The observed final state of muon decay is:
\[
    \mu^- \to e^- + \nu_\mu + \bar{\nu}_e
\]
This is the dominant decay channel \tagBL{}.
Rare radiative modes (e.g., $\mu \to e\nu\bar{\nu}\gamma$) exist but
are suppressed below $\mathcal{O}(10^{-2})$ of the total width.

In the thick-brane framework, this is not arbitrary: it is the
\emph{only allowed output configuration} satisfying:
\begin{enumerate}[nosep]
    \item \textbf{Charge conservation:} $Q_{\mu} = Q_{e} = -1$
    \item \textbf{Lepton number conservation:} $L_\mu = 1$, $L_e = 0$ initially;
          final state has $L_\mu = 1$ (via $\nu_\mu$) and $L_e = 0$
          (via $e^- + \bar{\nu}_e$ pair)
    \item \textbf{Energy threshold:} $m_\mu > m_e$ (no other charged leptons allowed)
    \item \textbf{Frozen projection compatibility:} All outputs must satisfy
          the frozen criterion $\hbar\omega \gg E_{\mathrm{env}}$
\end{enumerate}

\begin{edcDefinitionBox}{Allowed Output Set for Muon Decay}{[Dc]}
The allowed output set for muon decay is:
\[
    \mathcal{A}_\mu = \{e^-, \nu_\mu, \bar{\nu}_e\}
\]
Any other configuration (e.g., $\mu^- \to e^- + \gamma$, $\mu^- \to e^- + e^+ + e^-$)
is either forbidden by conservation laws or suppressed below $10^{-12}$ \tagBL{}.
\end{edcDefinitionBox}

\textbf{Physical Narration:}
\begin{itemize}[nosep]
    \item \textbf{5D cause:} Muon mode instability initiates redistribution.
    \item \textbf{Brane response:} Only mode combinations in $\mathcal{A}_\mu$
          can form—the brane layer's mode spectrum constrains the possibilities.
    \item \textbf{3D output:} The frozen projection maps these to $e^-$,
          $\nu_\mu$, $\bar{\nu}_e$.
\end{itemize}

% ==============================================================================
% RARE AND FORBIDDEN CHANNELS
% ==============================================================================

\subsubsection{Rare and Forbidden Channels}

\begin{table}[htbp]
\centering
\caption{Rare/forbidden muon decay channels: experimental status vs.\ EDC interpretation}
\label{tab:muon-forbidden}
\begin{tabular}{lccc}
\toprule
\textbf{Channel} & \textbf{Exp.\ limit} & \textbf{Status} & \textbf{EDC interpretation} \\
\midrule
$\mu^- \to e^- + \gamma$ & $< 4.2 \times 10^{-13}$ & \tagBL{} & LFV; selection rule violation \tagP{} \\
$\mu^- \to e^- + e^+ + e^-$ & $< 1.0 \times 10^{-12}$ & \tagBL{} & Mode mismatch hypothesis \tagP{} \\
$\mu^- \to e^- + \nu_e + \bar{\nu}_\mu$ & Not observed & \tagBL{} & Wrong lepton numbers \tagDc{} \\
\bottomrule
\end{tabular}
\end{table}

\textbf{Note:} The EDC interpretations for LFV channels are hypotheses \tagP{};
we propose that these channels violate selection rules emergent from the brane mode
spectrum, but the precise mechanism remains to be derived.

% ==============================================================================
% CHIRAL FILTER MECHANISM
% ==============================================================================

\subsubsection{Chiral Filter as Boundary Projection, Not a Fundamental Vertex}

A key empirical signature is the V--A chirality structure of weak outputs \tagBL{}.
In EDC we do not postulate a fundamental 3D ``weak vertex''.
Instead we hypothesize that chirality selection arises from a boundary/projection operator:
\begin{equation}
\label{eq:Pfrozen_factorization_mu}
\mathcal{P}_{\mathrm{frozen}}
=
\mathcal{P}_{\mathrm{energy}}\circ
\mathcal{P}_{\mathrm{mode}}\circ
\mathcal{P}_{\mathrm{chir}}.
\end{equation}

\begin{edcDefinitionBox}{Chiral Filter Decomposition}{[P]}
The frozen projection operator decomposes as:
\begin{equation}
    \mathcal{P}_{\mathrm{frozen}} = \mathcal{P}_{\mathrm{energy}} \circ
    \mathcal{P}_{\mathrm{mode}} \circ \mathcal{P}_{\mathrm{chir}}
    \label{eq:chiral-decomposition}
\end{equation}
where:
\begin{itemize}[nosep]
    \item $\mathcal{P}_{\mathrm{energy}}$: Selects modes with $\hbar\omega \gg E_{\mathrm{env}}$ (frozen criterion)
    \item $\mathcal{P}_{\mathrm{mode}}$: Selects modes compatible with allowed output set $\mathcal{A}$
    \item $\mathcal{P}_{\mathrm{chir}}$: Selects preferred chirality/helicity channel
\end{itemize}
\end{edcDefinitionBox}

\begin{tcolorbox}[mechanism, title={Chiral Filter Mechanism (Hypothesis)}]
\textbf{Physical picture:} The observer-facing boundary of the brane
($y = +\delta/2$) has a preferred orientation defined by its normal
vector $\hat{n}$. This orientation, combined with the brane-layer's
mode spectrum, may act as a \emph{chiral filter}: modes of one handedness
could couple more strongly to the 3D projection than the other.

\medskip
\textbf{Hypothesis} \tagP{}\textbf{:} We propose that the weak interaction's
``maximal parity violation'' is \emph{consistent with} geometric asymmetry
of the brane boundary, rather than requiring an intrinsic vertex asymmetry.
The derivation of $\mathcal{P}_{\mathrm{chir}}$ from the 5D action and
boundary conditions remains (open).
\end{tcolorbox}

\textbf{Physical Narration:}
\begin{enumerate}[nosep]
    \item \textbf{5D cause:} The brane has two faces: bulk-facing and observer-facing.
    \item \textbf{Brane response:} Modes in the brane layer can be decomposed
          into chirality eigenstates. The boundary conditions at $y = +\delta/2$
          favor one chirality.
    \item \textbf{3D output:} Observers detect only the ``passed'' chirality—the
          ``blocked'' component remains in the brane or is reflected.
\end{enumerate}

% ==============================================================================
% CHIRALITY SELECTION IN MUON DECAY
% ==============================================================================

\subsubsection{Chirality Selection in Muon Decay}

In Standard Model language, muon decay produces:
\begin{itemize}[nosep]
    \item $e^-$: Left-handed (predominantly)
    \item $\nu_\mu$: Left-handed
    \item $\bar{\nu}_e$: Right-handed
\end{itemize}

In the EDC framework, this pattern arises from $\mathcal{P}_{\mathrm{chir}}$:

\begin{equation}
    \mathcal{P}_{\mathrm{chir}}: \phi_{\mathrm{brane}} \mapsto
    \begin{cases}
        \phi_L & \text{(neutrinos)} \\
        \bar{\phi}_R & \text{(antineutrinos)} \\
        \text{mixed} & \text{(charged leptons, mass-dependent)}
    \end{cases}
    \label{eq:chiral-selection}
\end{equation}

\begin{tcolorbox}[edcWarning, title=\textbf{Non-Overclaim Reminder: Chiral Filter Status}]
\textbf{Status:} The chiral filter mechanism is \tagP{}.
We have not derived the specific form of $\mathcal{P}_{\mathrm{chir}}$ from
the 5D action. The claim is that such an operator \emph{exists} and
\emph{can be geometric in origin}—not that we have constructed it explicitly.
\end{tcolorbox}

% ==============================================================================
% FORMAL SKETCH: BOUNDARY AS CHIRAL PROJECTOR
% ==============================================================================

\subsubsection{Formal Sketch: Boundary as Chiral Projector}

We outline how boundary conditions at $y = +\delta/2$ could induce chirality
selection \tagP{}. This is a \emph{sketch}, not a derivation.

\textbf{Setup:} Let $\hat{n}$ be the outward normal to the observer-facing
boundary. A brane-layer spinor field $\psi(x,y)$ satisfies boundary conditions
at $y = +\delta/2$ of the schematic form:
\begin{equation}
    (1 - i\gamma^5 \hat{n}\cdot\gamma)\,\psi\big|_{y=+\delta/2} = 0
    \label{eq:bc-sketch}
\end{equation}
This type of condition (analogous to MIT bag boundary conditions) projects
out one chirality component at the boundary.

\textbf{Consequence:} Modes that ``pass through'' to the 3D side satisfy
a chirality constraint imposed by the boundary geometry, not by the bulk
Lagrangian. The normal vector $\hat{n}$ breaks parity at the boundary.

\textbf{Status:} Deriving Eq.~\eqref{eq:bc-sketch} from the full 5D action
with thick-brane profile remains (open). The above is a plausibility
argument, not a proof.

% ==============================================================================
% ENERGY FLOW DIAGRAM
% ==============================================================================

\subsubsection{Process Diagram: Muon Decay}

\begin{figure}[htbp]
\centering
\begin{tikzpicture}[scale=0.88]

% Nodes
\node[draw, fill=purple!10, rounded corners, minimum width=2.5cm, minimum height=1cm, align=center] (mu) at (0,0) {$\mu^-$ mode\\(brane-layer)};
\node[draw, fill=green!10, rounded corners, minimum width=2cm, minimum height=1cm, right=1.8cm of mu, align=center] (abs) {Absorption/\\Redistribution};
\node[draw, fill=teal!10, rounded corners, minimum width=2cm, minimum height=1cm, right=1.8cm of abs, align=center] (diss) {Dissipation/\\Mode population};
\node[draw, fill=blue!10, rounded corners, minimum width=2.2cm, minimum height=1cm, right=1.8cm of diss, align=center] (out) {3D Outputs\\$e^-, \nu_\mu, \bar{\nu}_e$};

% Arrows with labels
\draw[->, thick, purple!60!black] (mu) -- node[above, font=\scriptsize] {instability} (abs);
\draw[->, thick, green!50!black] (abs) -- node[above, font=\scriptsize] {$\Gamma_{\mathrm{eff}}$} (diss);
\draw[->, thick, blue!60!black] (diss) -- node[above, font=\scriptsize] {$\mathcal{P}_{\mathrm{frozen}}$} (out);

% Phase labels below
\node[font=\scriptsize, gray] at (0,-1) {Initial state};
\node[font=\scriptsize, gray] at ($(abs.south) + (0,-0.3)$) {Charging};
\node[font=\scriptsize, gray] at ($(diss.south) + (0,-0.3)$) {Mode spectrum};
\node[font=\scriptsize, gray] at ($(out.south) + (0,-0.3)$) {Observation};

% Chiral filter annotation
\draw[dashed, red!60!black] ($(diss.east)!0.5!(out.west)$) ++(0,-0.8) -- ++(0,1.6);
\node[font=\scriptsize, text=red!60!black, below] at ($(diss.east)!0.5!(out.west) + (0,-1.0)$) {$\mathcal{P}_{\mathrm{chir}}$};

% Ledger closure annotation
\node[draw, fill=gray!5, rounded corners, font=\footnotesize] (ledger) at (5,-2.2) {Ledger: $m_\mu c^2 = E_e + E_{\nu_\mu} + E_{\bar{\nu}_e} + E_{\mathrm{other}}$};
\draw[dashed, gray] (ledger.west) -- ++(-1,0);
\draw[dashed, gray] (ledger.east) -- ++(1,0);

\end{tikzpicture}
\caption{\textbf{Muon decay in EDC as brane-dominant relaxation.}
Stored brane energy redistributes (dissipation) and is released via frozen projection
into allowed outputs, with chirality selection implemented as a boundary operator.
Unlike neutron decay, there is no bulk trigger---the muon is a clean test of the
brane-layer mechanism. The chiral filter $\mathcal{P}_{\mathrm{chir}}$ selects
the helicity of outputs at the observer-facing boundary.}
\label{fig:muon_process_pipeline}
\end{figure}

% ==============================================================================
% WHY MUON IS A CLEAN UNIVERSALITY TEST
% ==============================================================================

\subsubsection{Why Muon Is a Clean Universality Test}

The muon decay channel tests whether:
\begin{itemize}[nosep]
  \item The same $\mathcal{P}_{\mathrm{frozen}}$ operator applies to brane-dominant excitations
        (not just bulk-core junctions)
  \item The chirality filter produces V--A structure without vertex tuning
  \item Ledger closure works for purely brane-layer relaxation
\end{itemize}

If these conditions hold, the weak-sector brane interface is a \emph{universal mechanism},
not a special case of neutron physics.

The fact that the same framework accommodates both neutron (bulk junction)
and muon (brane-dominant) decay without contradiction is a non-trivial
consistency check. The chiral filter mechanism, while still (open),
provides a geometric interpretation of weak interaction parity violation
that does not invoke intrinsic vertex asymmetry.

% ==============================================================================
% FALSIFIABILITY HOOKS
% ==============================================================================

\subsubsection{Falsifiability Hooks}

\begin{tcolorbox}[falsifiability, title=\textbf{Falsifiability: What Would Break This Model?}]
The thick-brane tomography picture for muon decay makes the following
structural predictions. Violation of any would require fundamental revision:

\begin{enumerate}[nosep]
    \item \textbf{Ledger must close:} If energy accounting shows a deficit
          that cannot be attributed to $E_{\mathrm{other}}$ (soft radiation,
          brane residual), the model fails.

    \item \textbf{Only allowed outputs:} If muon decay produced particles
          outside $\mathcal{A}_\mu = \{e^-, \nu_\mu, \bar{\nu}_e\}$ at
          observable rates, the selection rule mechanism would be falsified.

    \item \textbf{Chirality must be geometric:} If the chirality pattern
          could not be explained by boundary geometry (i.e., required
          intrinsic vertex asymmetry), the chiral filter picture fails.

    \item \textbf{No bulk escape:} If evidence emerged that muon decay
          deposits energy into bulk modes (not brane or 3D), the
          brane-dominant ontology would be falsified.

    \item \textbf{Same pipeline, different particle:} If a structurally
          similar process (e.g., $\tau$ decay) could \emph{not} be
          accommodated by the same framework, the model loses generality.

    \item \textbf{Lifetime must remain [BL]:} If the model claimed to
          \emph{predict} $\tau_\mu$ from geometric parameters alone
          (without fitting), and the prediction disagreed with experiment,
          the quantitative mapping would be falsified.
\end{enumerate}
\end{tcolorbox}

% ==============================================================================
% WHAT WE DO NOT CLAIM
% ==============================================================================

\subsubsection{Epistemic Status Summary}

\begin{tcolorbox}[edcWarning, title=\textbf{Non-Overclaim Reminder: What We Do Not Claim}]
\begin{itemize}[nosep]
    \item $\tau_\mu = 2.197 \times 10^{-6}$~s is \tagBL{} (baseline, not derived)
    \item The precise form of $\mathcal{P}_{\mathrm{chir}}$ is (open)
    \item The brane-dominant ontology for muon is \tagP{} (postulated, not derived)
    \item The relationship between $\Gamma_{\mathrm{eff}}$ and microphysics is (open)
\end{itemize}
\end{tcolorbox}

% ==============================================================================
% CANONICAL GLOSSARY
% ==============================================================================

\subsubsection{Canonical Glossary for Muon Decay}

\begin{tcolorbox}[edcCanonical, title=\textbf{Canonical Terms: Muon Decay Pipeline}]
\begin{description}[nosep, leftmargin=!, labelwidth=4cm]
\item[Brane-dominant excitation] Mode localized in brane layer (not bulk-core junction)
\item[$\mathcal{A}_\mu$] Allowed output set: $\{e^-, \nu_\mu, \bar{\nu}_e\}$
\item[$\mathcal{P}_{\mathrm{chir}}$] Chiral filter component of frozen projection
\item[Mode instability] Decay trigger for brane-dominant excitations
\item[Bulk leakage] Energy flow from brane back to bulk (suppressed for muon)
\item[Energy partition] Decomposition: $\Delta E_{\mathrm{brane}} + E_{\mathrm{other}}$
\item[MIT bag BC] Boundary condition type for chiral projection sketch
\end{description}
\end{tcolorbox}


