%!TEX root = ../EDC_Part_II_Weak_Sector.tex
% ==============================================================================
% Chapter 11: ell Suppression Mechanism Attempt (OPR-20 Follow-up)
% Status: Candidates proposed [P]; value closure still RED-C [OPEN]
% ==============================================================================

\subsection{Suppression Mechanism for \texorpdfstring{$\ell$}{l}: Attempt A2}
\label{sec:ch11_suppression_attempt}

The previous section (\S\ref{sec:ch11_value_closure_attempt}) identified the core
bottleneck: \textbf{no SM-free candidate for $\ell$ reproduces the weak scale}.
The ``natural'' membrane scale $\hbar c / (\sigma r_e^2) \approx 34$ fm is $\sim 10^3$
times larger than the required $\ell \sim 0.04$ fm.

This section isolates the missing small factor as the \textbf{true target} and
proposes two candidate mechanisms that could explain why $\ell \ll r_e$ without
importing SM weak-scale inputs.

\begin{tcolorbox}[colback=blue!5, colframe=blue!60!black,
    title=\textbf{Executive Summary: Suppression Mechanism Candidates}]
\textbf{Goal:} Derive or motivate a dimensionless suppression $f_{\text{geom}} \ll 1$
such that $\ell = (\text{natural scale}) \times f_{\text{geom}}$ gives the weak scale.

\textbf{Candidates:}
\begin{itemize}[nosep]
    \item \textbf{A (Bulk diffusion scale):} $f_A = R_\xi / r_e \sim 10^{-3}$ from
          Part I diffusion correlation length. Status: \tagP{} (postulated ratio).
    \item \textbf{B (Brane kinetic suppression):} $f_B = g_5^2 \kappa$ from
          brane kinetic term modification of effective scale. Status: \tagP{} (mechanism proposed).
\end{itemize}

\textbf{Honest verdict:} Both candidates provide \emph{plausible} mechanisms for
$f_{\text{geom}} \sim 10^{-3}$ without using SM inputs. However, neither achieves
\tagDc{} status---the underlying physics (why $R_\xi \ll r_e$, or what sets $\kappa$)
remains \textbf{[OPEN]}.

\textbf{Value closure:} Still \textbf{RED-C [OPEN]}. These candidates identify
\emph{where} the small factor could come from, not \emph{why} it has its value.
\end{tcolorbox}

% ------------------------------------------------------------------------------
\subsubsection{The Target: A Dimensionless Suppression Factor}
\label{sec:ch11_suppression_target}

\paragraph{What we need.}
From \S\ref{sec:ch11_value_closure_attempt}, the required suppression is:
\begin{equation}
    f_{\text{geom}} \equiv \frac{\ell_{\text{required}}}{\ell_{\text{natural}}}
    = \frac{0.04 \text{ fm}}{34 \text{ fm}}
    \approx 1.2 \times 10^{-3}
    \label{eq:ch11_fgeom_target}
\end{equation}
where $\ell_{\text{natural}} = \hbar c / (\sigma r_e^2)$ is the membrane-derived scale.

\paragraph{No-smuggling constraint.}
The factor $f_{\text{geom}}$ must be derived from quantities that do \emph{not}
contain the weak scale:

\begin{tcolorbox}[colback=red!5!white, colframe=red!60!black,
    title=\textbf{No-Smuggling Guardrails: Forbidden Inputs}]
\begin{itemize}[nosep]
    \item $M_W = 80$ GeV (would make $f_{\text{geom}}$ circular)
    \item $G_F = 1.17 \times 10^{-5}$ GeV$^{-2}$ (target of derivation)
    \item $v = 246$ GeV (defined via $G_F$)
    \item $\sin^2\theta_W = 0.23$ from experiment (use only EDC-derived value if needed)
\end{itemize}

\textbf{Allowed:}
\begin{itemize}[nosep]
    \item $\sigma r_e^2 = 5.86$ MeV \tagDc{} (from $\mathbb{Z}_6$ geometry)
    \item $r_e = 1$ fm \tagP{} (lattice spacing postulate)
    \item $R_\xi \sim 10^{-3}$ fm \tagP{} (from Part I diffusion physics)
    \item $\hbar c = 197.3$ MeV$\cdot$fm \tagBL{}
\end{itemize}
\end{tcolorbox}

% ------------------------------------------------------------------------------
\subsubsection{Candidate A: Bulk Diffusion Scale Ratio}
\label{sec:ch11_candidate_A}

\paragraph{Physical picture.}
In EDC, the bulk has two characteristic length scales:
\begin{itemize}[nosep]
    \item $r_e \approx 1$ fm: the lattice spacing / topological knot radius (brane structure)
    \item $R_\xi \sim 10^{-3}$ fm: the diffusion correlation length (bulk dynamics)
\end{itemize}
The ratio $R_\xi / r_e$ represents how ``thin'' the bulk diffusion layer is compared
to the brane lattice structure.

\paragraph{Candidate formula.}
\begin{equation}
    \boxed{
    f_A = \frac{R_\xi}{r_e} \approx 10^{-3}
    }
    \label{eq:ch11_fA}
\end{equation}
This gives:
\begin{equation}
    \ell_A = r_e \times f_A = R_\xi \approx 10^{-3} \text{ fm}
    \label{eq:ch11_ellA}
\end{equation}

\paragraph{Resulting mediator mass.}
With $\ell = R_\xi = 10^{-3}$ fm and $x_1 = \pi$ (Dirichlet BCs):
\begin{equation}
    m_\phi^A = \frac{x_1}{\ell_A} = \frac{\pi \times \hbar c}{10^{-3} \text{ fm}}
    = \frac{\pi \times 197.3 \text{ MeV}}{10^{-3}}
    \approx 620 \text{ GeV}
    \label{eq:ch11_mphi_A}
\end{equation}

\paragraph{Assessment.}
The Candidate A prediction $m_\phi \approx 620$ GeV overshoots $M_W = 80$ GeV by
a factor of $\sim 8$. This could indicate:
\begin{enumerate}[nosep]
    \item $x_1 \neq \pi$ (different boundary conditions could give $x_1 \approx \pi/8$)
    \item $R_\xi$ should be $\sim 8 \times 10^{-3}$ fm (factor 8 larger)
    \item An additional geometric factor is needed
\end{enumerate}

\begin{tcolorbox}[colback=yellow!5!white, colframe=yellow!60!black,
    title=\textbf{Candidate A Verdict}]
\textbf{Status:} \tagP{} (postulated)

\textbf{No-smuggling check:}
\begin{itemize}[nosep]
    \item[\ding{51}] Uses only $R_\xi$, $r_e$ (no SM weak scale)
    \item[\ding{51}] Dimensionless ratio: $[R_\xi / r_e] = [E]^0$ \checkmark
    \item[\ding{55}] Overshoots $M_W$ by factor $\sim 8$
\end{itemize}

\textbf{What would upgrade to [Dc]:}
\begin{itemize}[nosep]
    \item Derive \emph{why} $R_\xi \ll r_e$ from first-principles diffusion physics
    \item Or: show $R_\xi / r_e$ is fixed by bulk-brane consistency conditions
\end{itemize}

\textbf{Falsification:} If BVP solver with physical $V(\xi)$ gives $m_\phi$ inconsistent
with $\sim R_\xi^{-1}$, Candidate A fails.
\end{tcolorbox}

% ------------------------------------------------------------------------------
\subsubsection{Candidate B: Brane Kinetic Term Suppression}
\label{sec:ch11_candidate_B}

\paragraph{Physical picture.}
Brane-localized gauge kinetic terms (BKTs) can modify the effective 4D coupling
and, consequently, the effective mediator scale. From \S\ref{sec:ch11_g5_canonical}:
\begin{equation}
    S_{\text{brane}} = -\frac{\kappa}{4} \int d^4x \; F_{\mu\nu} F^{\mu\nu} \Big|_{\xi = 0}
    \quad\Rightarrow\quad
    \frac{1}{g_{\text{eff}}^2} = \frac{1}{g_5^2} + \kappa
    \label{eq:ch11_bkt_recap}
\end{equation}

\paragraph{Mechanism: BKT-induced scale shift.}
If $\kappa \gg g_5^{-2}$, the BKT dominates and the effective coupling becomes:
\begin{equation}
    g_{\text{eff}}^2 \approx \frac{1}{\kappa}
    \label{eq:ch11_geff_bkt}
\end{equation}
This modifies the low-energy physics by introducing a new scale $\kappa^{-1/2}$.

\paragraph{Candidate formula.}
Propose that $\kappa$ is set by the membrane tension \tagP{}:
\begin{equation}
    \kappa = \frac{c_\kappa}{\sigma r_e^2}
    \label{eq:ch11_kappa_membrane}
\end{equation}
where $c_\kappa$ is a dimensionless coefficient to be determined.

\paragraph{Effective suppression.}
The BKT introduces an effective length scale:
\begin{equation}
    \ell_{\text{eff}}^B = \sqrt{\frac{\kappa}{g_5^{-2}}} \times r_e
    = \sqrt{g_5^2 \kappa} \times r_e
    \label{eq:ch11_ell_effB}
\end{equation}
With $g_5^2 \sim g^2 \sim 0.4$ (from G1) and $\kappa \sim 1/(\sigma r_e^2) \sim 0.17$ MeV$^{-1}$:
\begin{equation}
    f_B = \sqrt{g_5^2 \kappa} = \sqrt{0.4 \times 0.17 \text{ MeV}^{-1}}
    \approx \sqrt{0.07 \text{ MeV}^{-1}}
\end{equation}
This does not directly give a dimensionless suppression without additional structure.

\paragraph{Revised approach: Mode normalization leakage.}
An alternative BKT mechanism: the presence of brane kinetic terms modifies the
KK mode normalization, effectively shifting the eigenvalue:
\begin{equation}
    x_1^{\text{eff}} = x_1 \times (1 + \kappa g_5^2)
    \label{eq:ch11_x1_eff}
\end{equation}
For $\kappa g_5^2 \gg 1$, this could give $x_1^{\text{eff}} \gg x_1$, reducing
$m_\phi = x_1^{\text{eff}} / \ell$ effectively. However, this requires $\kappa$
to be large in the appropriate units.

\paragraph{Numeric estimate.}
With $\sigma r_e^2 = 5.86$ MeV and $g_5^2 \sim 0.4$:
\begin{equation}
    \kappa g_5^2 = \frac{c_\kappa \times 0.4}{5.86 \text{ MeV}}
    = \frac{0.068 \, c_\kappa}{\text{MeV}}
\end{equation}
For this to give a factor $\sim 10^{3}$ enhancement, we would need $c_\kappa \sim 10^{4}$,
which is unnaturally large.

\begin{tcolorbox}[colback=yellow!5!white, colframe=yellow!60!black,
    title=\textbf{Candidate B Verdict}]
\textbf{Status:} \tagP{} (mechanism proposed, not derived)

\textbf{No-smuggling check:}
\begin{itemize}[nosep]
    \item[\ding{51}] Uses only $\sigma r_e^2$, $g_5$ (no SM weak scale)
    \item[\ding{51}] Dimensional structure consistent
    \item[\ding{55}] Requires large coefficient $c_\kappa \sim 10^4$ (unnatural)
\end{itemize}

\textbf{What would upgrade to [Dc]:}
\begin{itemize}[nosep]
    \item Derive $\kappa$ from brane action with no free parameters
    \item Show why $\kappa g_5^2 \gg 1$ is natural in EDC
\end{itemize}

\textbf{Falsification:} If explicit brane action gives $\kappa g_5^2 \ll 1$,
Candidate B's suppression mechanism fails.
\end{tcolorbox}

% ------------------------------------------------------------------------------
\subsubsection{Comparison Table}
\label{sec:ch11_suppression_comparison}

\begin{table}[ht]
\centering
\caption{Suppression mechanism candidates for OPR-20}
\label{tab:ch11_suppression_compare}
\small
\begin{tabular}{lccccc}
\toprule
\textbf{Candidate} & \textbf{Formula} & \textbf{$f_{\text{geom}}$} & \textbf{$m_\phi$ pred.} & \textbf{SM-free?} & \textbf{Status} \\
\midrule
A (diffusion) & $R_\xi / r_e$ & $\sim 10^{-3}$ & 620 GeV & \textcolor{green!60!black}{\ding{51}} & \tagP{} \\
B (BKT) & $\sqrt{g_5^2 \kappa}$ & requires $c_\kappa \sim 10^4$ & tunable & \textcolor{green!60!black}{\ding{51}} & \tagP{} \\
\midrule
\textbf{Target} & — & $\sim 1.2 \times 10^{-3}$ & 80 GeV & — & — \\
\bottomrule
\end{tabular}
\end{table}

\paragraph{Assessment.}
\begin{itemize}[nosep]
    \item \textbf{Candidate A} is closer to the target: it naturally produces
          $f_{\text{geom}} \sim 10^{-3}$ from the $R_\xi / r_e$ ratio, though
          the resulting $m_\phi$ overshoots by a factor of 8.
    \item \textbf{Candidate B} requires an unnaturally large coefficient to achieve
          sufficient suppression, making it less compelling.
    \item \textbf{Neither} achieves \tagDc{} status: both rely on postulated inputs
          ($R_\xi$ value, or $\kappa$ form) that are not derived from first principles.
\end{itemize}

% ------------------------------------------------------------------------------
\subsubsection{Connection to BVP Work Package}
\label{sec:ch11_suppression_bvp_hook}

\paragraph{How BVP testing would validate/falsify candidates.}
The thick-brane BVP solver (OPR-21, \S\ref{sec:ch12_bvp_workpackage}) provides a
concrete test:

\begin{enumerate}[nosep]
    \item \textbf{Input:} Potential $V(\xi)$ derived from membrane parameters
          $(\sigma, r_e, R_\xi)$
    \item \textbf{Output:} KK spectrum $\{m_n\}$ and mode profiles $\{f_n(z)\}$
    \item \textbf{Test A:} If $m_1 \sim 600$--800 GeV emerges from $V(\xi)$ with
          characteristic width $\sim R_\xi$, Candidate A is validated
    \item \textbf{Test B:} If $m_1$ requires brane kinetic term $\kappa$ to match
          $M_W$, and $\kappa$ is derivable from brane action, Candidate B gains support
\end{enumerate}

\paragraph{Critical prior for BVP.}
The suppression candidates suggest that the BVP potential $V(\xi)$ should have:
\begin{equation}
    \text{width}(V) \sim R_\xi \sim 10^{-3} \text{ fm}
    \quad\text{rather than}\quad
    \text{width}(V) \sim r_e \sim 1 \text{ fm}
    \label{eq:ch11_bvp_prior}
\end{equation}
This is a \emph{testable prediction} that distinguishes the two scale hypotheses.

% ------------------------------------------------------------------------------
\subsubsection{Suppression Attempt Summary}
\label{sec:ch11_suppression_summary}

\begin{tcolorbox}[colback=green!5, colframe=green!50!black,
    title=\textbf{OPR-20 Suppression Mechanism Status}]

\textbf{Before this section:}
\begin{quote}
OPR-20: RED-C [OPEN] --- No SM-free $\ell$ candidate; suppression factor unexplained.
\end{quote}

\textbf{After this section:}
\begin{quote}
OPR-20: \textbf{RED-C [OPEN]} --- Suppression candidates proposed \tagP{};
value closure still open.
\begin{itemize}[nosep]
    \item \textbf{Candidate A} ($f = R_\xi / r_e \sim 10^{-3}$): Most promising;
          SM-free; overshoots by factor 8
    \item \textbf{Candidate B} (BKT): Requires unnatural $c_\kappa \sim 10^4$
    \item \textbf{Neither} achieves \tagDc{} status
\end{itemize}
\end{quote}

\medskip
\noindent\fbox{\parbox{0.94\textwidth}{\small
\textbf{Honest verdict:} The suppression mechanism for $\ell$ is now \emph{isolated}
as the ratio $R_\xi / r_e$. This explains \emph{where} the $\sim 10^{-3}$ factor
could come from, but not \emph{why} $R_\xi \ll r_e$ in the first place.
The underlying physics of the diffusion correlation length remains the true
open problem.

\textbf{Progress:} Candidate A gives a concrete, SM-free formula that can be
tested by the BVP solver. This is an improvement over ``unexplained $f_{\text{geom}}$.''

\textbf{Remaining gap:} Derive $R_\xi$ from first-principles EDC dynamics,
or accept it as a fundamental scale and check consistency.}}
\end{tcolorbox}

