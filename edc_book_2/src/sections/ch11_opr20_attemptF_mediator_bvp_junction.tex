%!TEX root = ../EDC_Part_II_Weak_Sector.tex
% ==============================================================================
% OPR-20 Attempt F: Mediator BVP with Junction-Derived Robin BC
% Status: [Dc] structure + [P] parameter + [OPEN] unique derivation
% ==============================================================================

\subsection{Attempt F: Mediator BVP with Junction-Derived Boundary Conditions}
\label{sec:ch11_opr20_attemptF}

Previous attempts established that the factor-8 suppression cannot come from standard
boundary condition combinations (Attempt C: max factor-4) and that naive multiplication
of $Z_2$ and Israel factors is invalid (Attempt D: overcounting audit). Attempt E showed
that the $2\pi$ factor from circumference interpretation is derivable \tagDc{} but a
residual $\sim 0.9$ factor remains unexplained. This attempt takes a different route:
\textbf{can the eigenvalue $x_1$ itself be shifted from the naive $\pi/2$ (Neumann) or
$\pi$ (Dirichlet) toward a value that naturally produces the correct suppression?}

% ------------------------------------------------------------------------------
\subsubsection{F1: Sturm--Liouville Setup}
\label{sec:attemptF_setup}

\paragraph{The mediator mode equation.}
Consider a scalar or gauge mediator $\phi(x^\mu, \xi)$ propagating in the thick-brane
background. Separating variables $\phi(x,\xi) = \varphi(x) f(\xi)$, the extra-dimensional
profile $f(\xi)$ satisfies a Schr\"odinger-type equation \tagP{}:
\begin{equation}
    \boxed{
    -\frac{d^2 f}{d\xi^2} + V(\xi) f(\xi) = m^2 f(\xi)
    }
    \label{eq:attemptF_SL}
\end{equation}
where $m^2$ is the 4D mass-squared eigenvalue and $V(\xi)$ encodes the brane geometry.
The domain is $\xi \in [0, \ell]$ with boundary conditions to be specified.

\paragraph{Dimensionless formulation \tagDc{}.}
Define dimensionless coordinate $\tilde{\xi} := \xi/\ell \in [0,1]$ and rescaled quantities:
\begin{align}
    \tilde{V}(\tilde{\xi}) &= \ell^2 V(\ell\tilde{\xi}), \label{eq:attemptF_Vtilde} \\
    \lambda &= \ell^2 m^2, \quad x = \sqrt{\lambda}. \label{eq:attemptF_lambda}
\end{align}
The eigenvalue equation becomes:
\begin{equation}
    \left[ -\frac{d^2}{d\tilde{\xi}^2} + \tilde{V}(\tilde{\xi}) \right] \tilde{f}(\tilde{\xi}) = \lambda \tilde{f}(\tilde{\xi}),
    \label{eq:attemptF_dimensionless}
\end{equation}
which is a standard Sturm--Liouville problem. The physical mass is $m = x/\ell$.

\paragraph{\texorpdfstring{Connection to $G_F$ chain.}{Connection to GF chain.}}
From the closure spine (\S\ref{sec:ch11_full_closure}):
\begin{equation}
    G_F = \frac{g_5^2 \ell^2 I_4}{x_1^2},
\end{equation}
where $x_1 = \sqrt{\lambda_1}$ is the ground-state eigenvalue. If we can show that
$x_1 \neq \pi/2$ or $\pi$ but rather some intermediate value (e.g., $x_1 \approx 2.5$),
this provides an alternative route to the weak-scale suppression without invoking
additional geometric factors.

% ------------------------------------------------------------------------------
\subsubsection{F2: Potential Menu (EDC-Motivated)}
\label{sec:attemptF_potentials}

We consider three physically motivated potential shapes, all postulated \tagP{}
but consistent with thick-brane language in the literature:

\begin{tcolorbox}[colback=gray!5!white, colframe=gray!60!black,
    title=\textbf{Potential Models [P]}]

\textbf{V1: Square well (top-hat brane core)}
\begin{equation}
    \tilde{V}(\tilde{\xi}) = \begin{cases}
        0 & \text{if } |\tilde{\xi} - 1/2| < w/2 \\
        V_0 & \text{otherwise}
    \end{cases}
    \label{eq:attemptF_V1}
\end{equation}
Represents a localized brane with sharp boundaries. Parameters: $V_0$ (shoulder height),
$w$ (core width).

\medskip

\textbf{V2: Smooth sech$^2$ profile (domain wall)}
\begin{equation}
    \tilde{V}(\tilde{\xi}) = V_0 \left[ 1 - \operatorname{sech}^2\left(\frac{\tilde{\xi} - 1/2}{w}\right) \right]
    \label{eq:attemptF_V2}
\end{equation}
Standard kink/domain-wall potential. The mode is localized in the well at $\tilde{\xi} = 1/2$.

\medskip

\textbf{V3: Gaussian core}
\begin{equation}
    \tilde{V}(\tilde{\xi}) = V_0 \left[ 1 - \exp\left(-\frac{(\tilde{\xi} - 1/2)^2}{2w^2}\right) \right]
    \label{eq:attemptF_V3}
\end{equation}
Smooth Gaussian localization. Common in braneworld phenomenology.

\medskip

\emph{All parameters are dimensionless $\mathcal{O}(1)$ and are \textbf{not} tuned to SM values.}
\end{tcolorbox}

% ------------------------------------------------------------------------------
\subsubsection{F3: Junction $\to$ Robin BC Derivation}
\label{sec:attemptF_junction_robin}

\paragraph{The Israel junction condition.}
At a thin brane located at $\xi = z_*$, the Israel junction condition relates the
discontinuity in extrinsic curvature $K_{ab}$ to the brane stress-energy $T_{ab}$ \tagDc{}:
\begin{equation}
    [K_{ab}] - h_{ab} [K] = -\kappa_5^2 T_{ab},
    \label{eq:attemptF_Israel}
\end{equation}
where $h_{ab}$ is the induced metric and brackets denote the jump across the brane.

\paragraph{Scalar field with brane kinetic term (BKT).}
For a scalar mediator $\phi$ with bulk action and a brane-localized kinetic term:
\begin{align}
    S_{\text{bulk}} &= -\frac{1}{2} \int d^5x \sqrt{-g} \, (\partial_M \phi)^2, \\
    S_{\text{brane}} &= -\frac{\lambda}{2} \int d^4x \sqrt{-h} \, (\partial_\mu \phi)^2 \quad \text{[P]},
    \label{eq:attemptF_BKT}
\end{align}
where $\lambda$ is the BKT coefficient (dimensionless). Variation of the total action
yields the matching condition at the brane \tagDc{}:
\begin{equation}
    \left. \partial_\xi \phi \right|_{\xi_*^+} - \left. \partial_\xi \phi \right|_{\xi_*^-}
    = \lambda \, \Box_4 \phi \big|_{\xi_*},
    \label{eq:attemptF_matching}
\end{equation}
where $\Box_4$ is the 4D d'Alembertian.

\paragraph{Derivation of Robin BC.}
For a mode with 4D momentum $p^\mu$ (so $\Box_4 \phi \to -p^2 \phi$), and using the
orbifold/$Z_2$ symmetry (which identifies the two sides of the brane):
\begin{equation}
    2 f'(\xi_*) = -\lambda p^2 f(\xi_*).
\end{equation}
At the boundary $\xi = 0$ (or $\xi = \ell$), this becomes a Robin condition \tagDc{}:
\begin{equation}
    \boxed{
    f'(\text{boundary}) + \alpha \, f(\text{boundary}) = 0,
    }
    \label{eq:attemptF_Robin}
\end{equation}
where the Robin parameter is:
\begin{equation}
    \alpha = \frac{\lambda p^2}{2} \quad \text{(from BKT variation)}.
    \label{eq:attemptF_alpha_BKT}
\end{equation}
For the ground-state mediator with $m^2 = p^2 \ll 1/\ell^2$, this gives $\alpha \to 0$
(Neumann-like). For excited modes or if there are additional brane contributions,
$\alpha$ can be $\mathcal{O}(1)$ or larger.

\paragraph{Alternative: tension-dominated Robin parameter.}
If the brane tension $\sigma$ contributes directly (rather than through BKT), the
Robin parameter takes the form \tagP{}:
\begin{equation}
    \alpha = \frac{\kappa_5^2 \sigma}{2} \sim \mathcal{O}(1\text{--}10),
    \label{eq:attemptF_alpha_tension}
\end{equation}
where the last estimate uses $\kappa_5^2 \sigma \ell \sim \mathcal{O}(1)$ from
gravitational self-consistency. This is the regime explored numerically below.

\begin{tcolorbox}[colback=yellow!5!white, colframe=yellow!60!black,
    title=\textbf{Epistemic Status: Junction $\to$ Robin}]
\begin{itemize}[nosep]
    \item \textbf{Structure} (Robin form $f' + \alpha f = 0$): \tagDc{} from variation
    \item \textbf{$\alpha$ coefficient}: \tagP{} --- depends on BKT/tension parameters
          not uniquely fixed by EDC action
    \item \textbf{Unique derivation of $\alpha$}: \textbf{[OPEN]}
\end{itemize}
\end{tcolorbox}

% ------------------------------------------------------------------------------
\subsubsection{F4: Numerical Experiment Protocol}
\label{sec:attemptF_numerics}

\paragraph{Acceptance criteria.}
We seek \emph{robust} regions in parameter space where:
\begin{enumerate}[nosep]
    \item The ground-state eigenvalue $x_1 = \sqrt{\lambda_1}$ falls in a target range
          (e.g., $x_1 \in [2.3, 2.8]$, which would provide the needed shift from $\pi/2$).
    \item The region is \textbf{broad}, not needle-tuned: at least 20\% of scanned
          parameter volume should hit the target.
    \item Results are grid-converged and numerically stable.
\end{enumerate}

\paragraph{Scan protocol.}
Using the solver \texttt{tools/solve\_opr20\_mediator\_bvp.py}:
\begin{itemize}[nosep]
    \item \textbf{Model:} V1 (square well) with $V_0 = 0$ (empty box, Robin BC only)
    \item \textbf{Robin parameter:} $\alpha \in [0, 10]$ with 21 grid points
    \item \textbf{Target range:} $x_1 \in [2.3, 2.8]$
    \item \textbf{Grid:} $N = 400$ interior points (convergence verified)
\end{itemize}

\paragraph{Key results.}
The scan produces the following eigenvalue dependence on $\alpha$:

\begin{center}
\small
\begin{tabular}{r|ccc}
\toprule
$\alpha$ & $x_1$ & $x_1/\pi$ & In target? \\
\midrule
0.0  & 0.00 & 0.00 & No (Neumann, constant mode) \\
1.0  & 1.31 & 0.42 & No \\
2.0  & 1.72 & 0.55 & No \\
3.0  & 1.98 & 0.63 & No \\
4.0  & 2.15 & 0.69 & No \\
5.0  & 2.29 & 0.73 & Borderline \\
6.0  & 2.39 & 0.76 & \textbf{Yes} \\
7.0  & 2.47 & 0.79 & \textbf{Yes} \\
8.0  & 2.53 & 0.81 & \textbf{Yes} \\
9.0  & 2.58 & 0.82 & \textbf{Yes} \\
10.0 & 2.63 & 0.84 & \textbf{Yes} \\
15.0 & 2.78 & 0.88 & \textbf{Yes} \\
20.0 & 2.86 & 0.91 & No (above target) \\
$\to\infty$ & $\pi$ & 1.00 & No (Dirichlet limit) \\
\bottomrule
\end{tabular}
\end{center}

\paragraph{Robustness metric.}
Of 21 scanned $\alpha$ values from 0 to 10, \textbf{10 points (47.6\%)} fall in the
target range $x_1 \in [2.3, 2.8]$. This is a \textbf{broad region}, spanning
$\alpha \approx 5.5$ to $\alpha \approx 15$.

\begin{tcolorbox}[colback=green!5!white, colframe=green!50!black,
    title=\textbf{Robustness Finding}]
The target eigenvalue range is achieved for a \textbf{continuous band} of Robin
parameters $\alpha \in [5.5, 15]$, representing $\sim$50\% of the ``natural''
$\mathcal{O}(1)$--$\mathcal{O}(10)$ regime.

\medskip
\textbf{This is NOT needle-tuned.} The structure provides a mechanism; the
parameter $\alpha$ must come from EDC brane physics \tagP{}.
\end{tcolorbox}

\paragraph{Overcounting guard.}
The Robin BC already encodes the $Z_2$ orbifold symmetry through the matching
condition~\eqref{eq:attemptF_matching}. The Israel junction is the same physics
as the $Z_2$ reflection---one must \textbf{not} multiply these factors. This was
established in Attempt~D (\S\ref{sec:ch11_attemptD}).

% ------------------------------------------------------------------------------
\subsubsection{F5: Attempt F Verdict}
\label{sec:attemptF_verdict}

\begin{tcolorbox}[colback=blue!5!white, colframe=blue!50!black,
    title=\textbf{Attempt F: What Became [Dc], [P], [OPEN]}]

\textbf{Derived [Dc]:}
\begin{itemize}[nosep]
    \item Sturm--Liouville BVP structure (Eq.~\ref{eq:attemptF_SL})
    \item Junction/BKT $\to$ Robin BC form: $f' + \alpha f = 0$ (Eq.~\ref{eq:attemptF_Robin})
    \item Eigenvalue $x_1$ shifts continuously from 0 (Neumann) to $\pi$ (Dirichlet)
          as $\alpha$ increases
    \item Overcounting guard: $Z_2 \equiv$ Israel junction (no multiplication)
\end{itemize}

\textbf{Postulated [P]:}
\begin{itemize}[nosep]
    \item Potential shape $V(\xi)$ (V1/V2/V3 menu)
    \item Robin parameter $\alpha \sim 5$--$15$ for target $x_1$ (from scan)
    \item BKT coefficient $\lambda$ or tension contribution
\end{itemize}

\textbf{Open [OPEN]:}
\begin{itemize}[nosep]
    \item Unique derivation of $\alpha$ from EDC action
    \item Why $\alpha \sim \mathcal{O}(10)$ rather than $\mathcal{O}(1)$
    \item Connection to Part~I diffusion parameters $(\sigma, r_e, R_\xi)$
\end{itemize}
\end{tcolorbox}

\begin{tcolorbox}[
    colback=red!5!white,
    colframe=red!60!black,
    title=\textbf{OPR-20 Attempt F: Stoplight Verdict}
]
\begin{center}
\textbf{\large RED-C $\to$ RED-C [Dc]+[OPEN] (No Status Change)}
\end{center}

\medskip
\textbf{What improved:}
\begin{itemize}[nosep]
    \item Junction $\to$ Robin structure is now \textbf{[Dc]}
    \item Broad parameter region exists (not needle-tuned)
    \item Clear upgrade path: derive $\alpha$ from EDC $\Rightarrow$ YELLOW
\end{itemize}

\textbf{What remains:}
\begin{itemize}[nosep]
    \item $\alpha$ value is postulated, not derived
    \item No unique EDC prediction for $\alpha \sim 5$--$15$
    \item Weak-scale suppression requires this specific range
\end{itemize}

\textbf{Upgrade condition:}
OPR-20 upgrades to \textbf{YELLOW [P]} if and when $\alpha$ is derived from the
5D action (brane tension, BKT, or diffusion parameters) without SM input.
\end{tcolorbox}

% ------------------------------------------------------------------------------
\subsubsection{Comparison to Earlier Attempts}
\label{sec:attemptF_comparison}

\begin{table}[ht]
\centering
\caption{OPR-20 closure attempts comparison}
\label{tab:attemptF_comparison}
\small
\begin{tabular}{clccl}
\toprule
\textbf{Attempt} & \textbf{Route} & \textbf{Best Factor} & \textbf{Status} & \textbf{Key Finding} \\
\midrule
C & BC combinations & $2\pi\sqrt{2} \approx 8.89$ & [Dc]+[P] & Max factor-4 from BCs \\
D & Interpretation + Robin & same & [Dc]+[P] & $Z_2 \equiv$ Israel, no multiply \\
E & Prefactor-8 from $\ell$ & $2\pi$ & [Dc] & $\ell = 2\pi R_\xi$ derivable \\
\textbf{F} & \textbf{BVP + junction Robin} & \textbf{$x_1 \approx 2.5$} & \textbf{[Dc]+[P]} & \textbf{Broad region, $\alpha \sim 6$--$15$} \\
\bottomrule
\end{tabular}
\end{table}

\paragraph{Complementarity.}
Attempts C--E focused on the geometric factor in $\ell$ or $m_\phi = x_1/\ell$.
Attempt~F focuses on the eigenvalue $x_1$ itself. Together, they establish:
\begin{itemize}[nosep]
    \item The naive box model ($x_1 = \pi/2$ or $\pi$) is not required
    \item Junction physics naturally provides Robin BCs
    \item A broad parameter region can shift $x_1$ to the needed value
    \item \textbf{But:} no single attempt uniquely derives all parameters
\end{itemize}

\paragraph{Path forward.}
The remaining gap is the value of $\alpha$. Candidate derivations:
\begin{enumerate}[nosep]
    \item \textbf{BKT from membrane stiffness:} $\lambda \sim \sigma r_e^2 / \hbar c$, giving
          $\alpha \sim \lambda m^2 / 2$. Requires knowing $m^2$ independently.
    \item \textbf{Tension-dominated:} $\alpha \sim \kappa_5^2 \sigma \ell$. Requires
          gravitational coupling $\kappa_5$ from Part~I.
    \item \textbf{Thick-brane profile matching:} $\alpha$ emerges from matching interior
          solution to exponential tails. Requires full BVP with non-zero $V(\xi)$.
\end{enumerate}
Each path is explored in the BVP Work Package (\S\ref{sec:ch12_bvp_workpackage}).

