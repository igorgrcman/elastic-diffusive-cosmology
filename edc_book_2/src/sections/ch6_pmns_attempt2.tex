% ==============================================================================
% Chapter 6 Subsection: PMNS Attempt 2 — Overlap/Localization Model
% Status: Partial success — Z6 submixing captures theta23, fails theta12/theta13
% OPR-13 status: YELLOW (upgraded from RED)
% ==============================================================================

\subsection{Attempt PMNS-2: Overlap Model for Neutrino Mixing}
\label{sec:ch6_pmns_attempt2}

\begin{tcolorbox}[edcGuardrail, title=\textbf{Purpose}]
This subsection applies the overlap/localization model (successful for CKM hierarchy
in Chapter~\ref{sec:ch7_ckm}) to the PMNS matrix. The goal is to determine whether
$\mathbb{Z}_6$ geometry can produce the observed pattern: \textbf{large}
$\theta_{12}$, $\theta_{23}$ and \textbf{small} $\theta_{13}$.
\end{tcolorbox}

% ------------------------------------------------------------------------------
\subsubsection{Model Setup}
\label{sec:ch6_overlap_setup}

\paragraph{Overlap matrix ansatz.}
Following the CKM approach, we construct an overlap matrix between flavor eigenstates
(at positions $z_\alpha$) and mass eigenstates (at positions $z_i$) \tagP{}:
\begin{equation}
    O_{\alpha i} = \exp\left( -\frac{|z_\alpha - z_i|}{2\kappa} \right)
    \label{eq:ch6_overlap_ansatz}
\end{equation}
where $\kappa$ is the localization scale. The PMNS matrix is obtained by unitarizing
$O$ via SVD decomposition: $O = U\Sigma V^T \Rightarrow U_{\text{PMNS}} = UV^T$ \tagDc{}.

\paragraph{Track A vs.\ Track B.}
\begin{itemize}[nosep]
    \item \textbf{Track A:} No free parameters. Flavor and mass positions determined
          purely by $\mathbb{Z}_3$ or $\mathbb{Z}_6$ geometry.
    \item \textbf{Track B:} One calibrated parameter (e.g., flavor weights or
          localization asymmetry).
\end{itemize}

% ------------------------------------------------------------------------------
\subsubsection{Tested Variants}
\label{sec:ch6_variants}

We test four Track A variants and three Track B variants:

\begin{table}[ht]
\centering
\caption{PMNS Attempt 2: Tested model variants}
\label{tab:ch6_variants}
\begin{tabular}{p{3.5cm}p{5.0cm}p{2.5cm}}
\toprule
\textbf{Variant} & \textbf{Description} & \textbf{Free params} \\
\midrule
\multicolumn{3}{c}{\textit{Track A (no calibration)}} \\
\midrule
DFT baseline & All $|U_{\alpha i}|^2 = 1/3$ (Attempt 1) & 0 \\
A1: Uniform $\mathbb{Z}_3$ & Flavor and mass aligned at $\mathbb{Z}_3$ positions & 0 \\
A2: $\mathbb{Z}_3$ offset & Mass basis rotated by $\pi/3$ relative to flavor & 0 \\
A3: $\mathbb{Z}_6$ submixing & Flavor at $\mathbb{Z}_3$, mass at $\mathbb{Z}_6$ subset & 0 \\
\midrule
\multicolumn{3}{c}{\textit{Track B (one calibrated parameter)}} \\
\midrule
B1: $\nu_e$ suppression & Different $\kappa$ for electron neutrino & 1 \\
B2: Hierarchical spacing & Non-uniform mass eigenstate positions & 1 \\
B3: Flavor weights & Flavor-dependent overlap weights & 1 \\
\bottomrule
\end{tabular}
\end{table}

% ------------------------------------------------------------------------------
\subsubsection{Results: Track A}
\label{sec:ch6_track_a}

The key finding is that \textbf{variant A3 ($\mathbb{Z}_6$ submixing) achieves
partial success}:

\begin{table}[ht]
\centering
\caption{Track A results vs.\ PDG 2024 \tagBL{}}
\label{tab:ch6_track_a_results}
\begin{tabular}{lcccccc}
\toprule
\textbf{Variant} & $\sin^2\theta_{12}$ & $\sin^2\theta_{23}$ & $\sin^2\theta_{13}$
    & \textbf{Score} & \textbf{Status} \\
\midrule
PDG 2024 \tagBL{} & 0.307 & 0.546 & 0.022 & --- & --- \\
\midrule
DFT baseline & 0.500 & 0.500 & 0.333 & 0.91 & \textcolor{red}{RED} \\
A1: Uniform $\mathbb{Z}_3$ & 0.000 & 0.000 & 0.000 & 1.00 & \textcolor{red}{RED} \\
A2: Offset & 0.120 & 0.120 & 0.043 & 0.76 & \textcolor{red}{RED} \\
\textbf{A3: $\mathbb{Z}_6$ submixing} & 0.137 & \textbf{0.564} & 0.008 & 0.23 & \textcolor{orange}{YELLOW} \\
\bottomrule
\end{tabular}
\end{table}

\paragraph{Key observation.}
Variant A3 produces $\sin^2\theta_{23} = 0.564$, within 3\% of the observed value
0.546 --- a \textbf{GREEN} result for atmospheric mixing \tagDc{}.

\begin{tcolorbox}[colback=green!5, colframe=green!50!black,
    title=\textbf{Success: $\theta_{23}$ from $\mathbb{Z}_6$ Geometry}]
The near-maximal atmospheric mixing angle ($\theta_{23} \approx 45°$) emerges
\textbf{naturally} from the $\mathbb{Z}_6 = \mathbb{Z}_2 \times \mathbb{Z}_3$
submixing structure without any free parameters \tagDc{}.

\textbf{Physical interpretation:} The $\mathbb{Z}_6$ structure places mass
eigenstates at finer angular resolution than $\mathbb{Z}_3$ flavor states,
producing maximal mixing in the $\mu$--$\tau$ sector.
\end{tcolorbox}

% ------------------------------------------------------------------------------
\subsubsection{Results: Track B}
\label{sec:ch6_track_b}

Counterintuitively, Track B variants perform \textbf{worse} than Track A:

\begin{table}[ht]
\centering
\caption{Track B results (best parameter values)}
\label{tab:ch6_track_b_results}
\begin{tabular}{lcccccc}
\toprule
\textbf{Variant} & \textbf{Cal.\ param} & $\sin^2\theta_{12}$ & $\sin^2\theta_{23}$
    & $\sin^2\theta_{13}$ & \textbf{Score} & \textbf{Status} \\
\midrule
B1: $\nu_e$ supp.\ & $\epsilon = 1.5$ & 0.004 & 0.000 & 0.003 & 0.95 & \textcolor{red}{RED} \\
B2: Spacing & $\delta = 0.3$ & 0.077 & 0.032 & 0.003 & 0.85 & \textcolor{red}{RED} \\
B3: Weights & $w_e = 0.1$ & 0.031 & 0.000 & 0.000 & 0.96 & \textcolor{red}{RED} \\
\bottomrule
\end{tabular}
\end{table}

\paragraph{Why Track B fails.}
Introducing localization asymmetry to suppress $\theta_{13}$ also suppresses
$\theta_{12}$ and $\theta_{23}$, destroying the large mixing that PMNS requires.
\textbf{The overlap model naturally produces either democratic mixing (all large)
or hierarchical mixing (all small), not the observed asymmetric pattern.}

% ------------------------------------------------------------------------------
\subsubsection{Contrast with CKM}
\label{sec:ch6_ckm_contrast}

\begin{table}[ht]
\centering
\caption{Why overlap works for CKM but not PMNS}
\label{tab:ch6_ckm_pmns_contrast}
\begin{tabular}{lcc}
\toprule
\textbf{Property} & \textbf{CKM (quarks)} & \textbf{PMNS (leptons)} \\
\midrule
Diagonal elements & $\sim 1$ & $\sim 0.5$ \\
Off-diagonal pattern & Small ($\lambda$, $\lambda^2$, $\lambda^3$) & Large ($\theta_{12} \sim 33°$, $\theta_{23} \sim 45°$) \\
One small element & None (all follow hierarchy) & Yes ($\theta_{13} \sim 8°$) \\
Overlap model & Natural fit (exponential suppression) & Poor fit (requires breaking) \\
\bottomrule
\end{tabular}
\end{table}

\textbf{Conclusion:} The overlap model's success for CKM (hierarchical small mixing)
does not transfer to PMNS (large mixing with asymmetric suppression).

% ------------------------------------------------------------------------------
\subsubsection{Updated Stoplight: PMNS Mechanism}
\label{sec:ch6_pmns_stoplight_attempt2}

\begin{table}[ht]
\centering
\caption{PMNS mixing audit (post-Attempt 2)}
\label{tab:ch6_pmns_stoplight_v3}
\begin{tabular}{lccl}
\toprule
\textbf{Claim} & \textbf{Status} & \textbf{Tag} & \textbf{Note} \\
\midrule
$U_{\text{PMNS}}$ exists & GREEN & \tagBL{} & Observed \\
$\mathbb{Z}_3$ DFT baseline & FALSIFIED & \tagDc{} & $\theta_{13}$ off by $\times 15$ (Attempt 1) \\
$\mathbb{Z}_6$ overlap model & YELLOW & \tagDc{} & $\theta_{23}$ correct, others fail \\
$\theta_{23}$ from geometry & \textcolor{green!50!black}{\textbf{GREEN}} & \tagDc{} & Within 3\% (A3 variant) \\
$\theta_{12}$ from geometry & RED & (open) & Factor 2 off \\
$\theta_{13}$ from geometry & YELLOW & \tagDc{} & Closer than DFT ($\times 3$ vs $\times 15$) \\
Breaking mechanism & YELLOW & \tagP{} & Now specifically identified \\
\bottomrule
\end{tabular}
\end{table}

% ------------------------------------------------------------------------------
\subsubsection{Implications for OPR-13}
\label{sec:ch6_opr05_update}

The PMNS Attempt 2 results upgrade OPR-13 from RED to \textbf{YELLOW} with a
computed partial success:

\begin{tcolorbox}[colback=orange!5, colframe=orange!50!black,
    title=\textbf{OPR-13 Status Update}]
\textbf{Before (Attempt 1):} Pure $\mathbb{Z}_3$ symmetry predicts democratic mixing,
falsified by $\theta_{13}$.

\textbf{After (Attempt 2):}
\begin{itemize}[nosep]
    \item $\mathbb{Z}_6$ submixing \textbf{derives} $\theta_{23} \approx 45°$ \tagDc{}
    \item Solar angle $\theta_{12}$ requires additional mechanism \tagP{}
    \item Reactor angle $\theta_{13}$ improved but not reproduced \tagP{}
\end{itemize}

\textbf{Required physics:} A mechanism that breaks the $\nu_e$--$\nu_1$ overlap
while preserving $\nu_\mu$--$\nu_3$ maximal mixing.
\end{tcolorbox}

% ------------------------------------------------------------------------------
\subsubsection{Summary: PMNS Attempt 2 Verdict}
\label{sec:ch6_pmns_attempt2_summary}

\begin{tcolorbox}[colback=blue!5, colframe=blue!50!black,
    title=\textbf{Attempt PMNS-2: Final Verdict}]
\textbf{Track A (no free params):}
\begin{itemize}[nosep]
    \item Best variant: A3 ($\mathbb{Z}_6$ submixing)
    \item Overall score: 0.23 (YELLOW)
    \item $\theta_{23}$: \textcolor{green!50!black}{\textbf{GREEN}} (0.564 vs 0.546)
    \item $\theta_{12}$: \textcolor{red}{RED} (0.137 vs 0.307)
    \item $\theta_{13}$: \textcolor{red}{RED} (0.008 vs 0.022)
\end{itemize}

\textbf{Track B (one Cal param):}
All variants RED. Calibration destroys large mixing structure.

\textbf{Key success:} Atmospheric mixing $\theta_{23}$ derived from geometry.

\textbf{Key failure:} Solar mixing $\theta_{12}$ not reproduced by overlap model.

\textbf{Epistemic upgrade:} OPR-13 now YELLOW with explicit breaking requirements.
\end{tcolorbox}

% ------------------------------------------------------------------------------
\subsubsection{Implications for Future Work}
\label{sec:ch6_future_work}

The partial success of $\mathbb{Z}_6$ submixing suggests that the EDC geometric
framework captures \emph{part} of the PMNS structure, but additional physics
is needed for the solar sector.

\paragraph{\texorpdfstring{Candidate mechanisms for $\theta_{12}$.}{Candidate mechanisms for theta-12.}}
\begin{enumerate}[nosep]
    \item \textbf{Non-abelian extension:} The abelian $\mathbb{Z}_6$ cannot generate
          the non-abelian structures ($A_4$, $S_4$) known to produce tri-bimaximal
          mixing. An extended symmetry at the EDC action level might be required.

    \item \textbf{Higgs profile anisotropy:} Different Higgs couplings to different
          $\mathbb{Z}_3$ sectors could break the solar mixing toward 0.31.

    \item \textbf{Charged lepton corrections:} The PMNS matrix is
          $U_{\text{PMNS}} = U_\ell^\dagger U_\nu$. If charged lepton mixing is
          non-trivial, it could contribute the missing $\theta_{12}$.
\end{enumerate}

\textbf{Status:} All candidates are \tagP{} and remain (open) for future work.

