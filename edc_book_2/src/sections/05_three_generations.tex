% ==============================================================================
% Chapter 5: Why Exactly Three Generations?
% Status: [P]/[I] — High-risk chapter; mechanisms postulated/identified, not derived
% ==============================================================================

\section{Why Exactly Three Generations?}
\label{sec:ch5_three_generations}

\begin{tcolorbox}[edcGuardrail, title=\textbf{Epistemic Status: HIGH RISK}]
This chapter addresses one of the deepest open questions in particle physics:
\emph{why does the fermion spectrum consist of exactly three generations?}

\textbf{What we have:}
\begin{itemize}[nosep]
    \item An identification linking the $\mathbb{Z}_3$ subgroup of $\mathbb{Z}_6$ to
          generation count \tagI{}
    \item Mode indices $n = 0, 1, 2$ mapped to $(e, \mu, \tau)$ with $<1\%$ mass
          error \tagI{}
    \item Multiple candidate mechanisms for truncation (none fully derived)
\end{itemize}

\textbf{What remains open:}
\begin{itemize}[nosep]
    \item A rigorous derivation of \emph{why} exactly three modes survive
    \item The connection between EDC geometry and generation structure
\end{itemize}
\end{tcolorbox}

% ==============================================================================
% FRAMEWORK 2.0 LANGUAGE COMPLIANCE
% ==============================================================================
\begin{tcolorbox}[colback=blue!3!white, colframe=blue!50!black,
    title=\textbf{Framework 2.0 Language Compliance}]
\small
\textbf{EDC Projection Principle:} Every physical process has a \textbf{5D bulk+brane cause}
whose observable residue is a \textbf{3D shadow} on the observer boundary.

\textbf{In this chapter:}
\begin{itemize}[nosep]
    \item \textbf{5D cause:} $\mathbb{Z}_6 = \mathbb{Z}_2 \times \mathbb{Z}_3$ symmetry of hexagonal flux lattice.
    \item \textbf{Brane process:} Mode localization in three angular sectors.
    \item \textbf{3D shadow:} Three generations of fermions $(e, \mu, \tau)$.
\end{itemize}

\textbf{Standard Model counts} three generations as input; EDC seeks to \emph{derive}
this count from 5D geometry. Current status: \tagI{}/[OPEN].
\end{tcolorbox}

% ------------------------------------------------------------------------------
% PHYSICAL PROCESS NARRATIVE (Feynman-style)
% ------------------------------------------------------------------------------

\begin{tcolorbox}[colback=green!5!white, colframe=green!50!black,
    title=\textbf{Physical Process Narrative: Why Three Generations Appear in 5D EDC}]
\textbf{What physically happens, step by step:}

\textbf{Step 1: The brane has internal structure.}
In EDC, our 3D universe is a membrane embedded in 5D. This membrane is not
infinitely thin---it has a finite thickness $\delta$ along the fifth dimension $\xi$,
and internal structure in the angular coordinate $\xi$. This internal structure
is the key to generation physics \tagP{}.

\textbf{Step 2: Flux tubes arrange hexagonally.}
Energy minimization forces the flux tubes threading
the brane to arrange in a hexagonal (triangular) lattice---the densest 2D
circle packing, a classical result in plane geometry \tagDc{}/\tagP{}. This lattice has $\mathbb{Z}_6$
rotational symmetry: rotate by $60°$ and the pattern looks identical \tagDc{}.

\textbf{Step 3: $\mathbb{Z}_6$ factorizes into $\mathbb{Z}_2 \times \mathbb{Z}_3$.}
Group theory tells us that $\mathbb{Z}_6 = \mathbb{Z}_2 \times \mathbb{Z}_3$ \tagM{}.
The EDC proposal is that these two factors have distinct physical roles:
\begin{itemize}[nosep]
    \item $\mathbb{Z}_2$: \emph{interpreted as} matter/antimatter distinction \tagP{}
          (this is an identification, not a derivation)
    \item $\mathbb{Z}_3$: labels three inequivalent ``sectors'' or ``channels'' \tagP{}
\end{itemize}

\textbf{Step 4: Fermions localize in one of three channels.}
If fermion wavefunctions are sensitive to the $\mathbb{Z}_3$ structure, each
fermion must ``choose'' one of three angular sectors. The three sectors become
the three generations: electron lives in sector 0, muon in sector 1, tau in
sector 2 \tagP{}.

\textbf{Step 5: Overlap between sectors controls mixing.}
A fermion localized in sector $i$ has a profile $f_i(\xi)$ peaked at angular
position $\xi_i = 2\pi i/3$. The overlap integral $\int f_i f_j d\xi$ is large
for $i = j$ (same generation) and suppressed for $i \neq j$ (different
generations). This is the geometric origin of flavor mixing hierarchies
(developed in Ch.~7 for CKM, Ch.~6 for PMNS) \tagI{}.

\textbf{Step 6: Why only three? That's the open question.}
The $\mathbb{Z}_3$ structure \emph{suggests} three generations, but does not
\emph{prove} that a fourth is impossible. A complete derivation requires showing
that higher modes ($n \geq 3$) are dynamically forbidden---either unstable,
non-normalizable, or energetically inaccessible. This calculation is \textbf{not
yet done} (OPR-02) \tagP{}/[OPEN].
\end{tcolorbox}

% ==============================================================================
\subsection{The Problem: Generation Number as Input vs.\ Output}
\label{sec:ch5_problem}

In the Standard Model, the existence of three fermion generations is an
\emph{empirical input}. The gauge structure $SU(3)_C \times SU(2)_L \times U(1)_Y$
permits any number of generations; the value $N_{\text{gen}} = 3$ is simply
observed \tagBL{}.

EDC aims to derive $N_{\text{gen}} = 3$ from geometric structure. The candidate
mechanism links generation count to the $\mathbb{Z}_3$ factor in the hexagonal
lattice symmetry:
\begin{equation}
    \mathbb{Z}_6 = \mathbb{Z}_2 \times \mathbb{Z}_3
    \quad\Longrightarrow\quad
    |\mathbb{Z}_3| = 3 \;\stackrel{?}{\longleftrightarrow}\; N_{\text{gen}} = 3
    \label{eq:ch5_z6_factor}
\end{equation}
The challenge is to make this connection rigorous, not merely suggestive.

\paragraph{Connection to Chapter 4.}
The lepton mass candidates in Ch.~4 use mode indices $n = 0, 1, 2$ for
$(e, \mu, \tau)$:
\begin{itemize}[nosep]
    \item $n = 0$: Electron (ground state)
    \item $n = 1$: Muon (first excited state)
    \item $n = 2$: Tau (second excited state)
\end{itemize}
The question becomes: \emph{why does the tower stop at $n = 2$?}

% ==============================================================================
\subsection{Toy Model: Three Localized Channels}
\label{sec:ch5_toy_model}

Before examining the formal candidate mechanisms, it helps to build intuition
with a minimal toy model.

\paragraph{The picture.}
Imagine the brane's angular coordinate $\xi$ running from $0$ to $2\pi$.
Now imagine that the effective potential $V(\xi)$ has \textbf{three equivalent
minima} located at $\xi = 0, 2\pi/3, 4\pi/3$---the vertices of an equilateral
triangle inscribed in the circle. Each minimum is a ``localization well''
where a fermion wavefunction can be trapped \tagP{}.

\paragraph{What this captures.}
\begin{itemize}[nosep]
    \item \textbf{Three and only three:} The $\mathbb{Z}_3$ symmetry of the
          potential guarantees exactly three equivalent wells. A fermion in
          well 0 is the electron, well 1 is the muon, well 2 is the tau.
    \item \textbf{Overlap suppression:} If the wells are deep and narrow,
          the wavefunction in well $i$ has negligible overlap with well $j$.
          This explains why flavor mixing (CKM/PMNS off-diagonal elements)
          is small---it requires tunneling between wells.
    \item \textbf{Mass hierarchy:} If the three wells are not exactly
          identical (small symmetry breaking), fermions in different wells
          acquire different masses. The hierarchy $m_e \ll m_\mu \ll m_\tau$
          could arise from subtle differences in well depth or shape.
\end{itemize}

\paragraph{What this ignores.}
\begin{itemize}[nosep]
    \item \textbf{Why three minima?} The toy model \emph{assumes} $V(\xi)$
          has $\mathbb{Z}_3$ symmetry; it does not derive this from the
          EDC action. That derivation requires the full thick-brane BVP
          (OPR-02).
    \item \textbf{Why no fourth well?} In principle, a potential could have
          four or more minima. The toy model cannot explain why $\mathbb{Z}_3$
          rather than $\mathbb{Z}_4$ is selected.
    \item \textbf{Radial structure:} Real fermion profiles depend on both
          $\theta$ (angular) and $\xi$ (radial/depth) coordinates. The toy model
          treats only the angular part.
\end{itemize}

\begin{tcolorbox}[colback=yellow!5!white, colframe=yellow!60!black,
    title=\textbf{Toy Model Status}]
This ``three wells'' picture is \textbf{pedagogical} \tagP{}, not derived.
It correctly captures the \emph{structure} of the generation problem (three
equivalent sectors, overlap controls mixing) but does not answer the
\emph{dynamical} question (why three?).

The toy model is useful because it:
\begin{enumerate}[nosep]
    \item Connects Ch.~5 (generation counting) to Ch.~6/7 (PMNS/CKM mixing)
    \item Shows what a successful derivation \emph{would} look like
    \item Identifies the key open question: derive $V(\xi)$ with exactly
          three minima from EDC geometry
\end{enumerate}
\end{tcolorbox}

% --- FIGURE PLACEHOLDER 1: Three-channel localization schematic ---
\begin{figure}[htbp]
\centering
\fbox{\parbox{0.85\textwidth}{\centering
\textbf{[FIGURE PLACEHOLDER]}\\[1em]
\textit{Three-channel localization schematic}\\[0.5em]
\textbf{Panel (a):} Top view of brane showing hexagonal flux tube lattice.\\
Highlight three equivalent angular sectors at $\xi = 0, 2\pi/3, 4\pi/3$.\\
Color-code: sector 0 (blue/electron), sector 1 (green/muon), sector 2 (red/tau).\\[0.5em]
\textbf{Panel (b):} Effective potential $V(\xi)$ vs angular coordinate $\xi \in [0, 2\pi]$.\\
Show three equivalent minima (wells) with $\mathbb{Z}_3$ symmetry.\\
Sketch fermion wavefunctions $f_0, f_1, f_2$ localized in each well.\\[0.5em]
\textbf{Panel (c):} Cross-section showing brane thickness $\delta$ and\\
radial coordinate $\xi$. Fermions localized near $\xi = 0$ (observer face).
}}
\caption{\textbf{Generation structure from angular localization.}
In the EDC picture, the brane's $\mathbb{Z}_6$ hexagonal symmetry contains a
$\mathbb{Z}_3$ subgroup corresponding to three equivalent angular sectors.
Fermions localize in one of these three ``channels,'' giving rise to three
generations. The potential $V(\xi)$ has three minima; the mass hierarchy
arises from small symmetry breaking \tagP{}.}
\label{fig:ch5_three_channels}
\end{figure}

% ==============================================================================
\subsection{\texorpdfstring{Candidate Mechanism A: $\mathbb{Z}_3$ from Hexagonal Symmetry}{Candidate Mechanism A: Z3 from Hexagonal Symmetry}}
\label{sec:ch5_mechanism_a}

\subsubsection{The Argument}

The EDC thick brane supports a hexagonal lattice of flux tubes (see \S\ref{sec:step3}).
Hexagonal symmetry guarantees $\mathbb{Z}_6$ rotational invariance \tagDc{}:
\begin{equation}
    \text{Hexagonal lattice} \;\xrightarrow{\text{rotation group}}\; \mathbb{Z}_6
    \label{eq:ch5_hex_to_z6}
\end{equation}

The factorization $\mathbb{Z}_6 = \mathbb{Z}_2 \times \mathbb{Z}_3$ is pure
mathematics \tagM{}. The proposal is:
\begin{itemize}[nosep]
    \item $\mathbb{Z}_2$: \emph{Interpreted as} matter/antimatter (C-parity) \tagP{}
    \item $\mathbb{Z}_3$: Generation index (three-fold rotational symmetry) \tagP{}
\end{itemize}
\emph{Note:} These identifications are structural interpretations, not derivations.

\subsubsection{Stoplight Verdict}

\begin{center}
\small
\begin{tabular}{p{3cm}ccp{5cm}}
\toprule
\textbf{Criterion} & \textbf{Met?} & \textbf{Stoplight} & \textbf{Comment} \\
\midrule
$\mathbb{Z}_6$ from hexagons & Yes & \textcolor{green!50!black}{\textbf{GREEN}} & 2D packing + energy min \tagDc{}/\tagP{} \\
$\mathbb{Z}_6 = \mathbb{Z}_2 \times \mathbb{Z}_3$ & Yes & \textcolor{green!50!black}{\textbf{GREEN}} & Pure group theory \tagM{} \\
$|\mathbb{Z}_3| = 3 \to N_{\text{gen}}$ & No & \textcolor{red!80!black}{\textbf{RED}} & Cardinality matching, not derivation (OPR-01) \\
Explains mode truncation & No & \textcolor{red!80!black}{\textbf{RED}} & Doesn't explain why $n \geq 3$ forbidden \\
\bottomrule
\end{tabular}
\end{center}

\paragraph{Failure mode.}
The $\mathbb{Z}_3$ cardinality argument is \emph{numerological identification},
not derivation. It does not explain:
\begin{enumerate}[nosep]
    \item Why fermion generations \emph{couple} to the $\mathbb{Z}_3$ factor
    \item Why higher modes ($n \geq 3$) are absent or unstable
    \item The dynamical mechanism selecting three ground states
\end{enumerate}

\begin{tcolorbox}[colback=red!5, colframe=red!50!black,
    title=\textbf{Mechanism A Verdict: YELLOW/RED}]
The $\mathbb{Z}_6 = \mathbb{Z}_2 \times \mathbb{Z}_3$ factorization is mathematically
solid \tagM{}, and its emergence from hexagonal packing is derived \tagDc{}.
However, the link to generation count is \textbf{identified}, not derived \tagI{}.
The mechanism provides a structural cue but not a proof.
\end{tcolorbox}

% ==============================================================================
\subsection{Candidate Mechanism B: KK Tower Truncation}
\label{sec:ch5_mechanism_b}

\subsubsection{The Argument}

In Kaluza-Klein theories, compactification produces an infinite tower of modes
with masses $m_n \propto n/R$, where $R$ is the compactification radius.
The proposal is that in EDC, only the first three modes survive:

\begin{itemize}[nosep]
    \item $n = 0, 1, 2$: Stable or metastable (observable)
    \item $n \geq 3$: Unstable (decay faster than observation timescale)
\end{itemize}

\subsubsection{Potential Mechanism: Barrier Tunneling}

If higher modes must tunnel through a Peierls-type barrier to decay, the
lifetime scales as \tagP{}:
\begin{equation}
    \tau_n \propto \exp\left(+\frac{S_n}{\hbar}\right)
    \quad\text{where}\quad
    S_n = \int_0^{\xi_*} \sqrt{2m(V(\xi) - E_n)} \, d\xi
    \label{eq:ch5_lifetime}
\end{equation}

For the lifetime to drop below the Planck time at $n = 3$:
\begin{equation}
    \tau_3 < t_P \;\approx\; 5.4 \times 10^{-44}~\text{s}
    \quad\Longrightarrow\quad
    S_3 < \hbar \ln(t_P / t_0)
    \label{eq:ch5_truncation_cond}
\end{equation}
where $t_0$ is some reference timescale.

\subsubsection{Stoplight Verdict}

\begin{center}
\small
\begin{tabular}{p{3cm}ccp{5cm}}
\toprule
\textbf{Criterion} & \textbf{Met?} & \textbf{Stoplight} & \textbf{Comment} \\
\midrule
KK tower exists & Yes & \textcolor{green!50!black}{\textbf{GREEN}} & Standard 5D reduction \tagBL{} \\
Barrier form specified & No & \textcolor{red!80!black}{\textbf{RED}} & $V(\xi)$ not derived from EDC action (OPR-02) \\
Lifetime calculation & No & \textcolor{red!80!black}{\textbf{RED}} & No explicit $S_n$ computed (OPR-02) \\
Cutoff at $n = 3$ & No & \textcolor{red!80!black}{\textbf{RED}} & Would need specific barrier height (OPR-02) \\
\bottomrule
\end{tabular}
\end{center}

\paragraph{Failure mode.}
The argument requires:
\begin{enumerate}[nosep]
    \item The explicit potential $V(\xi)$ from the EDC thick-brane profile
    \item Calculation of the WKB action $S_n$ for each mode
    \item Demonstration that $S_2$ is large (stable) while $S_3$ is small (unstable)
\end{enumerate}
None of these have been completed.

\begin{tcolorbox}[colback=red!5, colframe=red!50!black,
    title=\textbf{Mechanism B Verdict: RED}]
KK tower truncation is a \textbf{plausible physical mechanism} but remains
\textbf{entirely uncomputed} in the EDC context. Without an explicit calculation
showing that exactly three modes survive, this is speculation \tagP{}.
\end{tcolorbox}

% ==============================================================================
\subsection{\texorpdfstring{Candidate Mechanism C: Bulk Topology $\pi_1(\mathcal{M}^5)$}{Candidate Mechanism C: Bulk Topology pi1(M5)}}
\label{sec:ch5_mechanism_c}

\subsubsection{The Argument}

If the 5D bulk manifold $\mathcal{M}^5$ has nontrivial fundamental group:
\begin{equation}
    \pi_1(\mathcal{M}^5) = \mathbb{Z}_3
    \label{eq:ch5_pi1}
\end{equation}
then winding modes around non-contractible loops would come in three classes,
potentially corresponding to three generations.

\subsubsection{Theoretical Motivation}

In orbifold compactifications, the fundamental group can contribute discrete
symmetries. For example \tagBL{}:
\begin{itemize}[nosep]
    \item $S^1/\mathbb{Z}_2$ orbifold: $\pi_1 = \mathbb{Z}$
    \item Lens space $L(3,1)$: $\pi_1 = \mathbb{Z}_3$
    \item Calabi-Yau threefold: $\pi_1$ depends on topology
\end{itemize}

The proposal is that EDC's bulk geometry naturally has $\pi_1 = \mathbb{Z}_3$,
providing a \emph{separate} source for the three-fold structure (independent
of the brane's $\mathbb{Z}_6$).

\subsubsection{Stoplight Verdict}

\begin{center}
\small
\begin{tabular}{p{3cm}ccp{5cm}}
\toprule
\textbf{Criterion} & \textbf{Met?} & \textbf{Stoplight} & \textbf{Comment} \\
\midrule
$\mathcal{M}^5$ topology specified & No & \textcolor{red!80!black}{\textbf{RED}} & EDC bulk metric not fully constrained (OPR-03) \\
$\pi_1(\mathcal{M}^5)$ computed & No & \textcolor{red!80!black}{\textbf{RED}} & No calculation exists (OPR-03) \\
Winding $\to$ generation & No & \textcolor{red!80!black}{\textbf{RED}} & Mechanism not worked out (OPR-03) \\
Independent of $\mathbb{Z}_6$ & Unknown & \textcolor{orange!80!black}{\textbf{YELLOW}} & Could be compatible or redundant \\
\bottomrule
\end{tabular}
\end{center}

\paragraph{Failure mode.}
This mechanism requires knowing the global topology of $\mathcal{M}^5$, which is not
determined by local dynamics. The EDC framework specifies local geometry
(thick brane, Plenum flow) but not global topology.

\begin{tcolorbox}[colback=red!5, colframe=red!50!black,
    title=\textbf{Mechanism C Verdict: RED}]
Bulk topology is a \textbf{logically possible} source for generation structure,
but EDC currently provides \textbf{no constraint or calculation} of $\pi_1(\mathcal{M}^5)$.
This remains pure speculation \tagP{}.
\end{tcolorbox}

% ==============================================================================
\subsection{Synthesis: What Do We Actually Have?}
\label{sec:ch5_synthesis}

\subsubsection{Summary Table}

\begin{center}
\begin{tabular}{p{4cm}ccc}
\toprule
\textbf{Mechanism} & \textbf{Derived?} & \textbf{Explains $n \leq 2$?} & \textbf{Verdict} \\
\midrule
A: $\mathbb{Z}_3 \subset \mathbb{Z}_6$ & Partial \tagI{} & No & \textcolor{orange!80!black}{\textbf{YELLOW}} (OPR-01) \\
B: KK truncation & No \tagP{} & No & \textcolor{red!80!black}{\textbf{RED}} (OPR-02) \\
C: $\pi_1(\mathcal{M}^5) = \mathbb{Z}_3$ & No \tagP{} & No & \textcolor{red!80!black}{\textbf{RED}} (OPR-03) \\
\bottomrule
\end{tabular}
\end{center}

% --- FIGURE PLACEHOLDER 2: Overlap → mixing intuition ---
\begin{figure}[htbp]
\centering
\fbox{\parbox{0.85\textwidth}{\centering
\textbf{[FIGURE PLACEHOLDER]}\\[1em]
\textit{Overlap integrals and flavor mixing}\\[0.5em]
\textbf{Left panel:} Three wavefunctions $f_0(\xi), f_1(\xi), f_2(\xi)$\\
plotted vs angular coordinate $\xi$. Each peaked in its own well,\\
with exponentially suppressed tails extending into neighboring sectors.\\[0.5em]
\textbf{Right panel:} $3 \times 3$ overlap matrix $O_{ij} = \int f_i f_j d\xi$.\\
Diagonal elements $O_{ii} \approx 1$ (same generation, large overlap).\\
Off-diagonal elements $O_{i \neq j} \ll 1$ (different generations, small overlap).\\
Arrow: ``This matrix structure $\to$ CKM/PMNS hierarchy (Ch.~7/6)''\\[0.5em]
\textbf{Annotation:} Deeper wells $\Rightarrow$ smaller overlap $\Rightarrow$ smaller mixing.
}}
\caption{\textbf{From generation localization to flavor mixing.}
The overlap integral between wavefunctions in different angular sectors
determines the mixing matrix elements. Large separation (deep wells) produces
small off-diagonal overlaps, explaining the near-diagonal structure of CKM
and the hierarchical structure of PMNS. This figure bridges Ch.~5 (generation
counting) to Ch.~6/7 (PMNS/CKM structure) \tagI{}.}
\label{fig:ch5_overlap_mixing}
\end{figure}

\subsubsection{Honest Assessment}

\begin{tcolorbox}[colback=gray!5, colframe=gray!50!black,
    title=\textbf{Current Status: [I]/[P] — Not Derived}]
EDC does \textbf{not} currently derive $N_{\text{gen}} = 3$ from first principles.
What exists is:
\begin{enumerate}[nosep]
    \item A \textbf{numerical identification} \tagI{}: The $\mathbb{Z}_3$ cardinality
          matches the observed generation count.
    \item A \textbf{structural consistency} \tagI{}: Mode indices $n = 0, 1, 2$ give
          mass ratios within $1\%$ of experiment.
    \item \textbf{Candidate mechanisms} \tagP{}: Three pathways that \emph{could}
          provide a derivation if completed.
\end{enumerate}

What is \textbf{missing}:
\begin{itemize}[nosep]
    \item A dynamical argument for why $n \geq 3$ modes are forbidden/unstable
    \item An explicit calculation of mode lifetimes or stability conditions
    \item A derivation linking fermion wavefunctions to $\mathbb{Z}_3$ structure
\end{itemize}
\end{tcolorbox}

% ==============================================================================
\subsection{Falsifiability Clause}
\label{sec:ch5_falsifiability}

\begin{tcolorbox}[edcCanonical, title=\textbf{Falsifiability Statement}]
\textbf{Prediction:} If the $\mathbb{Z}_3$ structure underlies generation count,
then:
\begin{equation}
    N_{\text{gen}} = 3 \quad \text{exactly}
    \label{eq:ch5_prediction}
\end{equation}

\textbf{Falsification criterion:}
\begin{itemize}[nosep]
    \item \textbf{Discovery of a 4th generation fermion} $\Longrightarrow$ EDC mechanism fails
    \item Specifically: A sequential fourth charged lepton $\ell_4^-$ or fourth
          up-type quark $t'$ with standard weak couplings would invalidate the
          $\mathbb{Z}_3$ identification.
\end{itemize}

\textbf{Current experimental status} \tagBL{}:
\begin{itemize}[nosep]
    \item LEP: $N_\nu = 2.984 \pm 0.008$ from $Z$ width (light neutrinos only)
    \item LHC: No evidence for sequential 4th generation quarks up to $\sim 1$ TeV
    \item Precision EW: 4th generation with SM-like couplings strongly disfavored
\end{itemize}

\textbf{Verdict:} Current data are \textbf{consistent} with $N_{\text{gen}} = 3$,
but this is also consistent with SM (which takes 3 as input). EDC gains predictive
power only if the derivation is completed.
\end{tcolorbox}

% ==============================================================================
\subsection{Open Problems and Next Steps}
\label{sec:ch5_open}

\paragraph{Open problems (status: open).}
\begin{enumerate}
    \item \textbf{KK truncation calculation:} Compute the thick-brane potential
          $V(\xi)$ from the EDC action and determine mode lifetimes $\tau_n$.
          Show that $\tau_0, \tau_1, \tau_2 \gg t_{\text{obs}}$ while
          $\tau_3 \ll t_{\text{obs}}$.

    \item \textbf{$\mathbb{Z}_3$ coupling derivation:} Explain \emph{why}
          fermion wavefunctions couple to the three-fold rotational structure
          of the hexagonal lattice.

    \item \textbf{Bulk topology:} Constrain $\pi_1(\mathcal{M}^5)$ from EDC dynamics,
          or show that bulk topology is irrelevant for generation structure.

    \item \textbf{Generational mixing:} If generations arise from $\mathbb{Z}_3$
          structure, what determines CKM/PMNS mixing? (See Ch.~7 pathway.)
\end{enumerate}

\paragraph{Recommended path forward.}
\begin{itemize}[nosep]
    \item \textbf{Priority 1:} Complete the KK truncation calculation (Mechanism B).
          This has the clearest physics and would provide a \emph{dynamical} reason
          for three generations.
    \item \textbf{Priority 2:} Investigate whether the Koide phase $\delta \approx 0.222$
          can be linked to $\mathbb{Z}_3$ geometry (would upgrade Mechanism A).
    \item \textbf{Fallback:} If no derivation succeeds, document the identification
          \tagI{} as a structural pattern pending future theoretical development.
\end{itemize}

% ==============================================================================
% DEPENDENCY & STATUS MINI-BOX (IF/THEN)
% ==============================================================================

\begin{tcolorbox}[colback=gray!5!white, colframe=gray!60!black,
    title=\textbf{Dependency Map \& Status (IF/THEN)}]
\textbf{What this chapter depends on:}
\begin{itemize}[nosep]
    \item Chapter 2: Hexagonal packing $\Rightarrow$ $\mathbb{Z}_6$ symmetry \tagDc{}
    \item Chapter 4: Mode indices $n = 0, 1, 2$ for lepton masses \tagI{}
    \item Baseline: $N_{\text{gen}} = 3$ observed (LEP, PDG) \tagBL{}
\end{itemize}

\textbf{What depends on this chapter:}
\begin{itemize}[nosep]
    \item Chapter 6 (PMNS): Three neutrino generations for mixing matrix
    \item Chapter 7 (CKM): Three quark generations for flavor structure
    \item Chapter 8 ($G_F$): Overlap integrals use generation profiles
\end{itemize}

\textbf{IF/THEN structure:}
\begin{itemize}[nosep]
    \item[\textbf{IF}] Hexagonal packing produces $\mathbb{Z}_6$ symmetry \tagDc{}
    \item[\textbf{AND}] $\mathbb{Z}_6 = \mathbb{Z}_2 \times \mathbb{Z}_3$ factorization \tagM{}
    \item[\textbf{AND}] Fermions couple to $\mathbb{Z}_3$ angular structure \tagP{}
    \item[\textbf{THEN}] Three inequivalent localization sectors exist \tagI{}
    \item[\textbf{AND}] Each sector hosts one generation family \tagI{}
    \item[\textbf{AND}] Overlap between sectors controls mixing \tagI{}
\end{itemize}

\textbf{Critical open question [OPEN]:}
\begin{itemize}[nosep]
    \item[$\circ$] Why is $n \geq 3$ forbidden? (OPR-02: KK truncation)
    \item[$\circ$] Derive $V(\xi)$ potential from EDC action (OPR-02)
    \item[$\circ$] Show mode $n = 3$ is unstable or non-normalizable (OPR-02)
\end{itemize}

\textbf{Bottom line:}
The structural link $|\mathbb{Z}_3| = 3 \leftrightarrow N_{\text{gen}} = 3$ is
\textbf{identified} \tagI{}, not \textbf{derived}. The chapter provides a
\emph{framework} for understanding generation physics, not a \emph{proof}.
\end{tcolorbox}

% ==============================================================================
\subsection{Summary}
\label{sec:ch5_summary}

\begin{enumerate}
    \item \textbf{Problem:} The Standard Model takes $N_{\text{gen}} = 3$ as input.
          EDC seeks a geometric derivation.

    \item \textbf{Identification:} The $\mathbb{Z}_3$ factor in
          $\mathbb{Z}_6 = \mathbb{Z}_2 \times \mathbb{Z}_3$ has cardinality 3,
          matching the observed generation count \tagI{}.

    \item \textbf{Mode structure:} Lepton masses fit a tower with $n = 0, 1, 2$
          (Ch.~4), suggesting a truncation mechanism \tagI{}.

    \item \textbf{Candidate mechanisms:}
          \begin{itemize}[nosep]
              \item[(A)] $\mathbb{Z}_3$ cardinality — \textcolor{orange!80!black}{\textbf{YELLOW}} (identified, not derived)
              \item[(B)] KK truncation — \textcolor{red!80!black}{\textbf{RED}} (plausible, not computed)
              \item[(C)] Bulk topology — \textcolor{red!80!black}{\textbf{RED}} (speculative)
          \end{itemize}

    \item \textbf{Falsifiability:} Discovery of a 4th sequential generation would
          invalidate the $\mathbb{Z}_3$ identification.

    \item \textbf{Status:} Generation structure is \textbf{identified} \tagI{} and
          \textbf{postulated} \tagP{}, not \textbf{derived}. This is an open problem.
\end{enumerate}

\begin{tcolorbox}[colback=orange!5, colframe=orange!50!black,
    title=\textbf{Epistemic Audit: Chapter 5}]
\begin{center}
\small
\begin{tabular}{lll}
\toprule
\textbf{Claim} & \textbf{Status} & \textbf{Source} \\
\midrule
$N_{\text{gen}} = 3$ observed & \tagBL{} & PDG, LEP \\
$\mathbb{Z}_6$ from hexagonal packing & \tagDc{}/\tagP{} & Ch.~2 (2D packing + energy min) \\
$\mathbb{Z}_6 = \mathbb{Z}_2 \times \mathbb{Z}_3$ & \tagM{} & Group theory \\
$|\mathbb{Z}_3| = 3 \leftrightarrow N_{\text{gen}}$ & \tagI{} & Numerical matching \\
Mode indices $n = 0,1,2$ & \tagI{} & Ch.~4 mass fits \\
KK truncation mechanism & \tagP{} & Not computed \\
$\pi_1(\mathcal{M}^5) = \mathbb{Z}_3$ & \tagP{} & Not computed \\
No 4th generation prediction & \tagI{}/\tagP{} & Follows from $\mathbb{Z}_3$ \emph{if} derivation holds \\
\bottomrule
\end{tabular}
\end{center}
\end{tcolorbox}

