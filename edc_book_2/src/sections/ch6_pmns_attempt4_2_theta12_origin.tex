%!TEX root = ../EDC_Part_II_Weak_Sector.tex
% ============================================================================
% PMNS Attempt 4.2: θ₁₂ Origin Micro-Attempt
% Status: Both T1 [Dc] and T2 [BL→Dc] achieve GREEN (~8.5% error)
% ============================================================================

\subsection{Attempt 4.2: \texorpdfstring{$\theta_{12}$}{theta12} Origin Micro-Attempt}
\label{sec:pmns_attempt4_2}

With the reactor perturbation $\varepsilon = \lambda/\sqrt{2}$ closed in Attempt~4.1,
the remaining open structure in the PMNS pattern is the solar angle $\theta_{12}$.
Attempt~4.1 found that the discrete value $35°$ works well, but this was an
\emph{identified} candidate~\tagI{}, not a derived one. Attempt~4.2 tests two
geometric mechanisms that could provide a first-principles origin for $\theta_{12}$.

\subsubsection{No-Smuggling Constraint}

\begin{tcolorbox}[colback=red!5!white, colframe=red!50!black, title=No-Smuggling Protocol]
\begin{itemize}[nosep]
  \item $\theta_{12}$ is \textbf{predicted} from geometry (T1) or $\lambda$ (T2)
  \item PDG sin$^2\theta_{12} = 0.307$ is used ONLY for \emph{evaluation} AFTER prediction
  \item The PDG-exact value $\theta_{12} = 33.7°$ is \textbf{not} a candidate
\end{itemize}
\end{tcolorbox}

\subsubsection{Candidate Mechanisms}

Two geometric candidates are tested:

\paragraph{\texorpdfstring{T1: $\theta_{12} = \arctan(1/\sqrt{2})$.}{T1: theta-12 = arctan(1/sqrt(2)).}}
In $\mathbb{Z}_6$-symmetric 5D geometry, the projection from bulk to brane involves
a factor $1/\sqrt{2}$. This same $\sqrt{2}$ appeared in $\varepsilon = \lambda/\sqrt{2}$
(Attempt~4.1), suggesting a unified geometric origin.
\begin{equation}
  \theta_{12}^{(T1)} = \arctan\left(\frac{1}{\sqrt{2}}\right) \approx 35.264°
  \quad\Rightarrow\quad \sin^2\theta_{12} = \frac{1}{3} \approx 0.333
  \label{eq:theta12_T1}
\end{equation}
Epistemic tag: \textbf{\tagDc{}}---pure geometry, no external input.

\paragraph{\texorpdfstring{T2: $\theta_{12} = 45° - \arcsin(\lambda)$.}{T2: theta-12 = 45 degrees - arcsin(lambda).}}
Maximal mixing ($45°$) reduced by a Cabibbo-scale deviation, connecting PMNS
to CKM via the common Wolfenstein parameter $\lambda$.
\begin{equation}
  \theta_{12}^{(T2)} = 45° - \arcsin(\lambda) = 45° - \arcsin(0.225) \approx 32.0°
  \quad\Rightarrow\quad \sin^2\theta_{12} \approx 0.281
  \label{eq:theta12_T2}
\end{equation}
Epistemic tag: \textbf{[BL$\to$Dc]}---uses $\lambda$~\tagBL{}, transformation is geometric.

\subsubsection{Results}

\begin{table}[ht]
\centering
\caption{Attempt 4.2: $\theta_{12}$ candidates and comparison to PDG}
\label{tab:pmns_attempt4_2}
\begin{tabular}{lccccc}
\toprule
\textbf{Candidate} & \textbf{$\theta_{12}$ (deg)} & \textbf{$\sin^2\theta_{12}$} &
\textbf{PDG} & \textbf{Error} & \textbf{Status} \\
\midrule
T1: $\arctan(1/\sqrt{2})$ & $35.26$ & $0.333$ & $0.307$ & $8.6\%$ &
\textcolor{green!60!black}{\textbf{GREEN}} \\
T2: $45° - \arcsin\lambda$ & $32.00$ & $0.281$ & $0.307$ & $8.5\%$ &
\textcolor{green!60!black}{\textbf{GREEN}} \\
\addlinespace
PDG target & $33.65$ & $0.307$ & --- & --- & \tagBL{} \\
\bottomrule
\end{tabular}
\end{table}

\paragraph{Key observation: PDG sits between T1 and T2.}
\begin{itemize}[nosep]
  \item T1 (35.26°) \emph{overshoots} PDG (33.65°) by $1.6°$
  \item T2 (32.00°) \emph{undershoots} PDG by $1.7°$
  \item Both achieve GREEN status ($<10\%$ error)
  \item Neither is obviously preferred on error alone
\end{itemize}

\paragraph{Full PMNS test.}
With $\theta_{23}$ from $\mathbb{Z}_6$ geometry~\tagDc{} and $\varepsilon = \lambda/\sqrt{2}$~[BL$\to$Dc]:

\begin{center}
\begin{tabular}{lcccc}
\toprule
\textbf{Candidate} & \textbf{$\sin^2\theta_{12}$} & \textbf{$\sin^2\theta_{23}$} &
\textbf{$\sin^2\theta_{13}$} & \textbf{Overall} \\
\midrule
T1: $\arctan(1/\sqrt{2})$ & $0.333$ (8.6\%) & $0.564$ (3.3\%) & $0.025$ (14\%) &
\textcolor{green!60!black}{\textbf{GREEN}} \\
T2: $45° - \arcsin\lambda$ & $0.281$ (8.5\%) & $0.564$ (3.3\%) & $0.025$ (14\%) &
\textcolor{green!60!black}{\textbf{GREEN}} \\
\addlinespace
PDG targets & $0.307$ & $0.546$ & $0.022$ & --- \\
\bottomrule
\end{tabular}
\end{center}

Both candidates yield an overall GREEN PMNS pattern.

\subsubsection{Epistemic Comparison}

\begin{tcolorbox}[colback=blue!5!white, colframe=blue!50!black,
    title=T1 vs T2: Epistemic Trade-off]
\textbf{T1 ($\theta_{12} = \arctan(1/\sqrt{2})$):}
\begin{itemize}[nosep]
  \item Pure geometry~\tagDc{}---no external input
  \item Error: 8.6\% (slightly worse numerically)
  \item Provides geometric origin for the $35°$ discrete candidate from Attempt~4.1
  \item Connects to $\sqrt{2}$ factor in $\varepsilon = \lambda/\sqrt{2}$ (unified mechanism)
\end{itemize}

\textbf{T2 ($\theta_{12} = 45° - \arcsin\lambda$):}
\begin{itemize}[nosep]
  \item Uses $\lambda$~\tagBL{}, transformation is geometric~[BL$\to$Dc]
  \item Error: 8.5\% (marginally better numerically)
  \item Connects PMNS to CKM via Wolfenstein scale
  \item Suggests maximal mixing reduced by Cabibbo-scale deviation
\end{itemize}

\medskip
\textbf{Neither candidate dominates.} T1 has cleaner epistemic status (pure geometry),
T2 has marginally better numerical fit. Both bracket the PDG value.
\end{tcolorbox}

\subsubsection{Comparison to 35° Discrete Candidate}

Attempt~4.1 found that $\theta_{12}^0 = 35°$ (discrete, \tagI{}) achieves 7.2\% error.
T1 ($\arctan(1/\sqrt{2}) = 35.264°$) provides a geometric origin for this value:
\begin{equation}
  \arctan\left(\frac{1}{\sqrt{2}}\right) = 35.264° \approx 35°
  \label{eq:35deg_origin}
\end{equation}

This upgrades the $35°$ candidate from \tagI{} (identified) to \tagDc{} (derived):
\emph{the solar angle is geometrically constrained, not merely fit.}

\subsubsection{Verdict}

\begin{tcolorbox}[colback=green!5!white, colframe=green!50!black,
    title=Attempt 4.2 Verdict: $\theta_{12}$ Geometric Origin Established]
\textbf{Both T1 and T2 achieve GREEN ($<10\%$ error) without PDG-smuggling.}

\textbf{OPR-13c upgrade:}
\begin{itemize}[nosep]
  \item \textbf{Before:} YELLOW \tagI{} --- ``$35°$ discrete or $33.7°$ identified''
  \item \textbf{After:} YELLOW [\tagDc{}]/[BL$\to$Dc] --- Two geometric mechanisms,
        both $\sim 8.5\%$ from PDG, neither calibrated
\end{itemize}

\textbf{Recommended path:} T1 ($\arctan(1/\sqrt{2})$) is preferred epistemically
because it requires no baseline input and connects to the $\sqrt{2}$ factor
already established in $\varepsilon = \lambda/\sqrt{2}$.

\textbf{What remains open:}
\begin{itemize}[nosep]
  \item Neither candidate exactly hits PDG ($\pm 1.6°$ bracketing)
  \item Selection rule for T1 vs T2 not derived
  \item Possible: Weighted average or RG correction could close the $1.6°$ gap
\end{itemize}
\end{tcolorbox}

\begin{tcolorbox}[colback=yellow!5!white, colframe=yellow!60!black,
    title=PMNS Closure Status After Attempt 4.2]
\textbf{Complete PMNS picture (no PDG-smuggling):}
\begin{itemize}[nosep]
  \item $\theta_{23}$: GREEN \tagDc{} --- $\sin^2\theta_{23} = 0.564$ from $\mathbb{Z}_6$
        geometry (3\% from PDG)
  \item $\theta_{13}$: YELLOW [BL$\to$Dc] --- $\varepsilon = \lambda/\sqrt{2}$ predicts
        $\sin^2\theta_{13} = 0.025$ (15\% from PDG)
  \item $\theta_{12}$: \textbf{YELLOW [\tagDc{}]} --- $\arctan(1/\sqrt{2})$ predicts
        $\sin^2\theta_{12} = 0.333$ (8.6\% from PDG)
\end{itemize}

\medskip
\noindent\fbox{\parbox{0.96\textwidth}{\small
\textbf{Solar origin (Attempt 4.2):} $\theta_{12} = \arctan(1/\sqrt{2}) \approx 35.26°$
provides geometric origin for solar angle ($<10\%$ from PDG, no fit);
unified $\sqrt{2}$ factor with $\varepsilon = \lambda/\sqrt{2}$; OPR-13c
upgraded from \tagI{} to \tagDc{}.}}

\smallskip
\noindent\emph{PMNS: all three angles now have geometric mechanisms
($\theta_{23}$: $\mathbb{Z}_6$; $\theta_{13}$: $\lambda/\sqrt{2}$;
$\theta_{12}$: $\arctan(1/\sqrt{2})$)---none calibrated to PDG.}
\end{tcolorbox}

\paragraph{Code.} \texttt{code/pmns\_attempt4\_2\_theta12\_origin.py}

\paragraph{Output.} \texttt{code/output/pmns\_attempt4\_2\_results.txt}
