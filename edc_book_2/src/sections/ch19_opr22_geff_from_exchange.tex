% ch19_opr22_geff_from_exchange.tex
% OPR-22: First-Principles G_eff from 5D Mediator Exchange
% Status: CONDITIONAL [Dc]
% Created: 2026-01-25
% Branch: book2-opr22-geff-derivation-v1

\chapter{OPR-22: Effective Fermi Coupling from 5D Mediator Exchange}
\label{ch:opr22_geff}

%%%%%%%%%%%%%%%%%%%%%%%%%%%%%%%%%%%%%%%%%%%%%%%%%%%%%%%%%%%%%%%%%%%%%%%%%%%%%%%
% READER'S MAP
%%%%%%%%%%%%%%%%%%%%%%%%%%%%%%%%%%%%%%%%%%%%%%%%%%%%%%%%%%%%%%%%%%%%%%%%%%%%%%%

\begin{tcolorbox}[colback=blue!5!white, colframe=blue!50!black,
    title=\textbf{Reader's Map: What This Chapter Does and Does Not Claim}]

\textbf{Goal}: Derive the effective four-fermion contact strength $G_{\text{eff}}$ from
first principles using the 5D gauge-fermion action and KK reduction, without using
any Standard Model observables as inputs.

\medskip

\textbf{What we ASSUME} (tagged [P] or [M]):
\begin{itemize}
    \item 5D gauge action with canonical kinetic term [M]
    \item Warped metric ansatz $ds^2 = e^{2A(\xi)} \eta_{\mu\nu} dx^\mu dx^\nu + d\xi^2$ [P]
    \item Domain $\xi \in [0, \ell]$ with $\ell$ postulated [P]
    \item \textbf{Working Default}: Brane-localized fermion current at $\xi = 0$ [P]
    \item Natural normalization convention: $\int_0^\ell |f_n|^2 d\xi = \ell$ [Dc]
\end{itemize}

\textbf{What we DERIVE} (tagged [Dc]):
\begin{itemize}
    \item 4D effective coupling $g_{\text{eff},n} = g_5 f_n(0)$ for brane-localized current
    \item Four-fermion operator from integrating out first massive mode
    \item $G_{\text{eff}} = g_5^2 \ell |f_1(0)|^2 / (2 x_1^2)$ in terms of 5D parameters
    \item Connection to OPR-20 contact strength: $G_{\text{eff}} = \frac{1}{2} C_{\text{eff}} |f_1(0)|^2$
\end{itemize}

\textbf{What we do NOT claim}:
\begin{itemize}
    \item We do NOT derive $g_5$, $\ell$, $V(\xi)$, or BC parameters from first principles
    \item We do NOT use $G_F$, $M_W$, $M_Z$, $v$, or $\sin^2\theta_W$ as inputs
    \item We call the result $G_{\text{eff}}$, not $G_F$, to distinguish computed from measured
    \item Any numerical comparison to PDG values is labeled ``external comparison only''
\end{itemize}

\textbf{Dependencies}:
\begin{itemize}
    \item OPR-19 (Chapter~\ref{ch:opr19_g5}): $g_5 \to g_4$ dimensional reduction
    \item OPR-20 (Chapter~\ref{ch:opr20_mediator_mass}): Mediator mass $m_1 = x_1/\ell$
    \item OPR-21: BVP mode profiles $f_n(\xi)$ and Robin BC structure
\end{itemize}

\textbf{Status}: CONDITIONAL [Dc] --- structure derived; $g_5$, $\ell$, $V(\xi)$, $\kappa$ remain [P].

\medskip

\textbf{Reading path}:
\begin{enumerate}
    \item \textbf{First}: Invariant EFT exchange formula (Section~\ref{sec:ch19_integrate_out})
    \item \textbf{Second}: Define eigenvalue $x_1 := m_1 \ell$ (OPR-20)
    \item \textbf{Third}: Brane coupling rule $g_{4,1} = g_5 \tilde{f}_1(0)$ (Section~\ref{sec:ch19_4d_coupling})
    \item \textbf{Fourth}: Dimensional bookkeeping and connection to OPR-20 (Section~\ref{sec:ch19_dimensions})
    \item \textbf{Fifth}: Physical run path: $V(\xi) = M^2 - M'$ results (Section~\ref{sec:ch19_physical_run})
\end{enumerate}

\end{tcolorbox}

%%%%%%%%%%%%%%%%%%%%%%%%%%%%%%%%%%%%%%%%%%%%%%%%%%%%%%%%%%%%%%%%%%%%%%%%%%%%%%%
% DEPRECATION NOTE
%%%%%%%%%%%%%%%%%%%%%%%%%%%%%%%%%%%%%%%%%%%%%%%%%%%%%%%%%%%%%%%%%%%%%%%%%%%%%%%

\begin{tcolorbox}[colback=red!5!white, colframe=red!50!black,
    title=\textbf{Deprecation Note: Legacy Formula Correction}]

Earlier chapters (CH11, CH12) may contain the formula $C_{\text{eff}} = g_5^2 \ell^2 / x_1^2$.
This is \textbf{deprecated}. The correct invariant expression is:
\begin{equation}
    C_{\text{eff}} = \frac{g_5^2 \ell}{x_1^2}
\end{equation}
See OPR-20 patch (Chapter~\ref{ch:opr20_mediator_mass}) for the corrected derivation.
The present chapter uses the corrected formula throughout.

\end{tcolorbox}

% ------------------------------------------------------------------------------
% CANONICAL PATH REMINDER (PHYSICAL PATH LOCK)
% ------------------------------------------------------------------------------
\begin{tcolorbox}[colback=green!3!white, colframe=green!40!black,
    title=\textbf{Canonical Path Reminder}]
\textbf{Physical path (WD):} Domain wall $V_L = M^2 - M'$, $\mu$-window $[13, 17]$ \tagDc{}.\\
\textbf{Toy benchmark:} Pöschl--Teller, $\mu$-window $[15, 18]$ \tagM{}.\\[3pt]
All $G_{\text{eff}}$ numerics in this chapter use the \textbf{physical} path.
See Box~\ref{box:ch14_canonical_physical_path} (Chapter~14) for the full reader route map.
\end{tcolorbox}

%%%%%%%%%%%%%%%%%%%%%%%%%%%%%%%%%%%%%%%%%%%%%%%%%%%%%%%%%%%%%%%%%%%%%%%%%%%%%%%
% SECTION 1: PHYSICAL MOTIVATION
%%%%%%%%%%%%%%%%%%%%%%%%%%%%%%%%%%%%%%%%%%%%%%%%%%%%%%%%%%%%%%%%%%%%%%%%%%%%%%%

\section{Physical Motivation}
\label{sec:ch19_motivation}

In the Standard Model, the Fermi constant $G_F$ characterizes the strength of weak
interactions at low energies. It appears in the effective four-fermion Lagrangian:
\begin{equation}
    \mathcal{L}_{\text{Fermi}} = -\frac{G_F}{\sqrt{2}} \, j^\mu j_\mu
    \label{eq:ch19_fermi_lagrangian}
\end{equation}
where $j^\mu$ represents the charged weak current.

In the Standard Model, $G_F$ is computed from the W boson exchange:
\begin{equation}
    \frac{G_F}{\sqrt{2}} = \frac{g^2}{8 M_W^2}
    \label{eq:ch19_gf_sm}
\end{equation}
where $g$ is the SU(2)$_L$ gauge coupling and $M_W$ is the W boson mass.

\textbf{The EDC question}: Can we derive an analogous quantity $G_{\text{eff}}$ from the
5D geometry, using only:
\begin{itemize}
    \item The 5D gauge coupling $g_5$ [P]
    \item The domain size $\ell$ [P]
    \item The eigenvalue structure from the Sturm--Liouville problem [Dc]
    \item The mode profile evaluation at the brane [Dc]
\end{itemize}

This chapter answers yes, giving:
\begin{equation}
    \boxed{G_{\text{eff}} = \frac{g_5^2 \ell}{2 x_1^2} \cdot |f_1(0)|^2}
    \label{eq:ch19_geff_preview}
\end{equation}

%%%%%%%%%%%%%%%%%%%%%%%%%%%%%%%%%%%%%%%%%%%%%%%%%%%%%%%%%%%%%%%%%%%%%%%%%%%%%%%
% SECTION 2: 5D GAUGE-FERMION ACTION
%%%%%%%%%%%%%%%%%%%%%%%%%%%%%%%%%%%%%%%%%%%%%%%%%%%%%%%%%%%%%%%%%%%%%%%%%%%%%%%

\section{5D Gauge-Fermion Action}
\label{sec:ch19_5d_action}

\subsection{The Starting Point}

We begin with the 5D action containing a gauge field $A_M$ and fermion fields $\psi$.
In the warped metric ansatz (cf.\ OPR-19, Chapter~\ref{ch:opr19_g5}):
\begin{equation}
    ds^2 = e^{2A(\xi)} \eta_{\mu\nu} dx^\mu dx^\nu + d\xi^2
    \label{eq:ch19_warped_metric}
\end{equation}

The gauge field action is:
\begin{equation}
    S_{\text{gauge}} = -\frac{1}{4g_5^2} \int d^4x \, d\xi \, \sqrt{-G} \,
    G^{MA} G^{NB} F_{MN} F_{AB}
    \label{eq:ch19_gauge_action}
\end{equation}

The gauge-fermion interaction:
\begin{equation}
    S_{\text{int}} = \int d^4x \, d\xi \, \sqrt{-G} \, J^M A_M
    \label{eq:ch19_interaction}
\end{equation}
where $J^M$ is the 5D fermion current.

\textbf{Epistemic status}: Eq.~\eqref{eq:ch19_warped_metric} is [P];
Eqs.~\eqref{eq:ch19_gauge_action}--\eqref{eq:ch19_interaction} are [M].

\subsection{Working Default: Brane-Localized Current}
\label{sec:ch19_brane_current}

\begin{tcolorbox}[colback=yellow!10!white, colframe=yellow!50!black,
    title=\textbf{Working Default (WD-22-1): Brane-Localized Current}]

We assume the fermion current relevant for weak interactions is localized on the
brane at $\xi = 0$:
\begin{equation}
    J^\mu(x,\xi) = j^\mu(x) \, \delta(\xi)
    \label{eq:ch19_brane_current}
\end{equation}
where $j^\mu(x)$ is the 4D fermion current.

\textbf{Rationale}: This is the standard ansatz in Randall--Sundrum and related
brane-world models. It provides a clean derivation where the 4D coupling depends
on the mode profile evaluated at the brane location.

\textbf{Alternative} (OPEN-22-1): Bulk-distributed current with profile $\rho(\xi)$
leads to overlap integrals requiring OPR-21 mode profiles.

\textbf{Epistemic status}: [P] --- working hypothesis.

\end{tcolorbox}

%%%%%%%%%%%%%%%%%%%%%%%%%%%%%%%%%%%%%%%%%%%%%%%%%%%%%%%%%%%%%%%%%%%%%%%%%%%%%%%
% SECTION 3: KK MODE EXPANSION
%%%%%%%%%%%%%%%%%%%%%%%%%%%%%%%%%%%%%%%%%%%%%%%%%%%%%%%%%%%%%%%%%%%%%%%%%%%%%%%

\section{Kaluza--Klein Mode Expansion}
\label{sec:ch19_kk_expansion}

\subsection{Mode Decomposition}

Following OPR-20 (Chapter~\ref{ch:opr20_mediator_mass}), we expand the 4D components
of the gauge field:
\begin{equation}
    A_\mu(x,\xi) = \sum_{n=0}^{\infty} a_\mu^{(n)}(x) \, f_n(\xi)
    \label{eq:ch19_kk_expansion}
\end{equation}
where:
\begin{itemize}
    \item $a_\mu^{(n)}(x)$ are 4D gauge fields satisfying $(\Box + m_n^2) a_\mu^{(n)} = 0$
    \item $f_n(\xi)$ are mode profiles on $\xi \in [0, \ell]$
    \item $m_n$ are the 4D mass eigenvalues
\end{itemize}

The mode profiles satisfy the Sturm--Liouville equation (OPR-20, Eq.~20.3):
\begin{equation}
    -\frac{d^2 f_n}{d\xi^2} + V(\xi) f_n(\xi) = m_n^2 f_n(\xi)
    \label{eq:ch19_sl_equation}
\end{equation}

\textbf{Epistemic status}: [Dc] --- standard KK technique.

\subsection{Normalization Convention}
\label{sec:ch19_normalization}

\begin{tcolorbox}[colback=green!5!white, colframe=green!50!black,
    title=\textbf{Key Convention: Natural Normalization}]

Following OPR-19 (Chapter~\ref{ch:opr19_g5}), we adopt the \textbf{natural normalization}:
\begin{equation}
    \int_0^\ell d\xi \, |f_n(\xi)|^2 = \ell
    \label{eq:ch19_natural_norm}
\end{equation}

\textbf{Consequences}:
\begin{itemize}
    \item The flat zero mode (Neumann BC) has $f_0(\xi) = 1$
    \item The profile $f_n(\xi)$ is \textbf{dimensionless}
    \item The 4D coupling is $g_{4,n}^2 = g_5^2 / \ell$ (OPR-19, Eq.~19.8)
\end{itemize}

\textbf{Alternative}: Unit normalization $\int |f_n|^2 d\xi = 1$ gives $[f_n] = L^{-1/2}$.
Conversion: $\tilde{f}_n = f_n / \sqrt{\ell}$.

\end{tcolorbox}

%%%%%%%%%%%%%%%%%%%%%%%%%%%%%%%%%%%%%%%%%%%%%%%%%%%%%%%%%%%%%%%%%%%%%%%%%%%%%%%
% SECTION 4: EFFECTIVE 4D COUPLING
%%%%%%%%%%%%%%%%%%%%%%%%%%%%%%%%%%%%%%%%%%%%%%%%%%%%%%%%%%%%%%%%%%%%%%%%%%%%%%%

\section{Effective 4D Coupling to Brane Current}
\label{sec:ch19_4d_coupling}

\subsection{Substituting the KK Expansion}

Substituting Eq.~\eqref{eq:ch19_kk_expansion} and Eq.~\eqref{eq:ch19_brane_current}
into the interaction term Eq.~\eqref{eq:ch19_interaction}:

\begin{align}
    S_{\text{int}} &= \int d^4x \, d\xi \, j^\mu(x) \delta(\xi) \,
    \sum_n a_\mu^{(n)}(x) f_n(\xi) \notag \\
    &= \sum_n \int d^4x \, j^\mu(x) \, a_\mu^{(n)}(x) \,
    \underbrace{\int d\xi \, \delta(\xi) f_n(\xi)}_{= f_n(0)} \notag \\
    &= \sum_n \int d^4x \, j^\mu(x) \, a_\mu^{(n)}(x) \, f_n(0)
    \label{eq:ch19_int_expanded}
\end{align}

\subsection{Canonical Field Normalization}

To properly track dimensions, we work with the canonically normalized 5D field
$\tilde{A}_M = A_M / g_5$. The interaction becomes:
\begin{equation}
    S_{\text{int}} = g_5 \int d^4x \, d\xi \, J^M \tilde{A}_M
\end{equation}

After KK expansion, the effective 4D coupling of mode $n$ to the brane current is:

\begin{tcolorbox}[colback=orange!5!white, colframe=orange!50!black,
    title=\textbf{Key Result: Effective 4D Coupling (Brane-Localized)}]
\begin{equation}
    \boxed{g_{\text{eff},n} = g_5 \cdot f_n(0)}
    \label{eq:ch19_geff_n}
\end{equation}
where $f_n(0)$ is the mode profile evaluated at the brane ($\xi = 0$).

\textbf{Dimensional check} (natural normalization):
\begin{itemize}
    \item $[g_5] = L^{1/2}$ (from 5D action normalization)
    \item $[f_n(0)] = 1$ (dimensionless in natural normalization)
    \item $[g_{\text{eff},n}] = L^{1/2}$
\end{itemize}

\textbf{Note}: The physical observables will involve $g_{\text{eff},n}^2 / m_n^2$, which
has the correct dimension $L^2$ for a four-fermion coupling.

\textbf{Epistemic status}: [Dc] --- follows from brane localization ansatz.

\end{tcolorbox}

\subsection{Normalization Invariance}

\textbf{Lemma} (Invariance under profile rescaling):

Under the rescaling $f_n \to c \cdot f_n$:
\begin{itemize}
    \item Normalization: $\int |f_n|^2 d\xi \to c^2 \int |f_n|^2 d\xi$
    \item Brane evaluation: $f_n(0) \to c \cdot f_n(0)$
    \item Effective coupling: $g_{\text{eff},n} \to c \cdot g_{\text{eff},n}$
\end{itemize}

The physical combination:
\begin{equation}
    \frac{g_{\text{eff},n}^2}{m_n^2} = \frac{g_5^2 \, f_n(0)^2}{m_n^2}
\end{equation}
transforms as $c^2$ under rescaling, which exactly compensates the change in the
current coupling normalization. The observable $G_{\text{eff}}$ is \textbf{invariant}
when conventions are consistently applied.

\textbf{Epistemic status}: [Dc] --- algebraic verification.

%%%%%%%%%%%%%%%%%%%%%%%%%%%%%%%%%%%%%%%%%%%%%%%%%%%%%%%%%%%%%%%%%%%%%%%%%%%%%%%
% SECTION 5: INTEGRATING OUT THE MEDIATOR
%%%%%%%%%%%%%%%%%%%%%%%%%%%%%%%%%%%%%%%%%%%%%%%%%%%%%%%%%%%%%%%%%%%%%%%%%%%%%%%

\section{Integrating Out the Mediator}
\label{sec:ch19_integrate_out}

\subsection{Low-Energy Effective Theory}

At energies $E \ll m_1$ (well below the first massive mode), we can integrate out
the mediator $a_\mu^{(1)}$ to obtain a four-fermion contact interaction.

The propagator for the massive mode in momentum space:
\begin{equation}
    \langle a_\mu^{(1)}(p) a_\nu^{(1)}(-p) \rangle = \frac{-i \eta_{\mu\nu}}{p^2 - m_1^2}
    \xrightarrow{p^2 \ll m_1^2} \frac{i \eta_{\mu\nu}}{m_1^2}
    \label{eq:ch19_propagator}
\end{equation}

\subsection{Four-Fermion Operator}

The exchange diagram generates:
\begin{equation}
    \mathcal{L}_{\text{eff}} = -\frac{g_{4,1}^2}{2 m_1^2} \, (j^\mu j_\mu)
    \label{eq:ch19_leff}
\end{equation}
where $g_{4,1}$ is the \textbf{dimensionless} 4D effective coupling.

The factor of $1/2$ arises from the standard convention for four-fermion operators
(matching the Fermi theory normalization).

\begin{tcolorbox}[colback=orange!5!white, colframe=orange!50!black,
    title=\textbf{Key Result: Invariant EFT Formula}]

The effective four-fermion contact strength is:
\begin{equation}
    \boxed{G_{\text{eff}} := \frac{g_{4,1}^2}{2 m_1^2}}
    \label{eq:ch19_geff_def}
\end{equation}

This is the \textbf{invariant} result---it holds regardless of normalization convention.
What matters is that $g_{4,1}$ is the dimensionless 4D coupling.

\textbf{Epistemic status}: [Dc] --- standard effective field theory.

\end{tcolorbox}

\begin{tcolorbox}[colback=red!5!white, colframe=red!50!black,
    title=\textbf{Common Pitfall: Double Counting $|f_1(0)|^2$}]

\textbf{Warning}: Mixing a brane-localized coupling $g_{4,1} = g_5 \tilde{f}_1(0)$ with
a bulk-normalization reduction formula can double-count $|f_1(0)|^2$.

\textbf{We fix one convention and stick to it.}

In this chapter:
\begin{itemize}
    \item Brane-localized current: $g_{4,1} = g_5 \cdot \tilde{f}_1(0)$
    \item Unit normalization: $\int_0^\ell |\tilde{f}_1|^2 d\xi = 1$, so $[\tilde{f}_1] = L^{-1/2}$
    \item Result: $g_{4,1}$ is dimensionless (as required for a 4D gauge coupling)
\end{itemize}

The $|f_1(0)|^2$ factor enters \emph{once} through the brane coupling, not separately.

\end{tcolorbox}

\subsection{Careful Dimensional Analysis}
\label{sec:ch19_dimensions}

\textbf{Step 1: 5D action dimensions}

The 5D gauge action:
\begin{equation}
    S = -\frac{1}{4g_5^2} \int d^5x \, F_{MN} F^{MN}
\end{equation}

For $[S] = 1$, $[d^5x] = L^5$, $[F] = L^{-2}$:
\begin{equation}
    \left[\frac{1}{g_5^2}\right] \cdot L^5 \cdot L^{-4} = 1
    \implies [g_5^2] = L \implies [g_5] = L^{1/2}
\end{equation}

\textbf{Step 2: Canonical field}

Define $\tilde{A}_M = A_M / g_5$ so the kinetic term is canonical. Then $[\tilde{A}] = L^{-3/2}$.

\textbf{Step 3: KK decomposition}

For canonical 4D fields $[\tilde{a}_\mu^{(n)}] = L^{-1}$ and unit-normalized profiles
$\int |\tilde{f}_n|^2 d\xi = 1$ with $[\tilde{f}_n] = L^{-1/2}$:
\begin{equation}
    \tilde{A}_\mu = \sum_n \tilde{a}_\mu^{(n)} \tilde{f}_n \implies
    [\tilde{A}_\mu] = L^{-1} \cdot L^{-1/2} = L^{-3/2} \checkmark
\end{equation}

\textbf{Step 4: Brane coupling}

The interaction $S_{\text{int}} = g_5 \int d^4x \, j^\mu \tilde{a}_\mu^{(n)} \tilde{f}_n(0)$
gives effective coupling:
\begin{equation}
    g_{\text{eff},n} = g_5 \, \tilde{f}_n(0), \quad
    [g_{\text{eff},n}] = L^{1/2} \cdot L^{-1/2} = 1 \quad \text{(dimensionless)}
\end{equation}

\textbf{Step 5: Four-fermion coupling}
\begin{equation}
    G_{\text{eff}} = \frac{g_{\text{eff},1}^2}{2 m_1^2} = \frac{g_5^2 \, |\tilde{f}_1(0)|^2}{2 m_1^2}
\end{equation}
\begin{equation}
    [G_{\text{eff}}] = \frac{L \cdot L^{-1}}{L^{-2}} = \frac{1}{L^{-2}} = L^2 = \text{GeV}^{-2} \checkmark
\end{equation}

\begin{tcolorbox}[colback=green!5!white, colframe=green!50!black,
    title=\textbf{Dimensional Summary}]

\textbf{Unit normalization} ($\int |\tilde{f}_n|^2 d\xi = 1$):
\begin{itemize}
    \item $[\tilde{f}_n] = L^{-1/2}$
    \item $g_{\text{eff},n} = g_5 \tilde{f}_n(0)$ is dimensionless
    \item $G_{\text{eff}} = g_5^2 |\tilde{f}_1(0)|^2 / (2m_1^2)$ has $[G_{\text{eff}}] = L^2$
\end{itemize}

\textbf{Natural normalization} ($\int |f_n|^2 d\xi = \ell$):
\begin{itemize}
    \item $[f_n] = 1$ (dimensionless)
    \item Conversion: $\tilde{f}_n = f_n / \sqrt{\ell}$, so $|\tilde{f}_n(0)|^2 = |f_n(0)|^2 / \ell$
    \item $G_{\text{eff}} = g_5^2 |f_1(0)|^2 / (2 m_1^2 \ell)$
\end{itemize}

Using $m_1 = x_1/\ell$:
\begin{equation}
    G_{\text{eff}} = \frac{g_5^2 |f_1(0)|^2}{2 \ell} \cdot \frac{\ell^2}{x_1^2}
    = \frac{g_5^2 \ell |f_1(0)|^2}{2 x_1^2}
\end{equation}

\textbf{Dimensional check}:
$[g_5^2 \ell / x_1^2] = L \cdot L / 1 = L^2$ \checkmark

\end{tcolorbox}

%%%%%%%%%%%%%%%%%%%%%%%%%%%%%%%%%%%%%%%%%%%%%%%%%%%%%%%%%%%%%%%%%%%%%%%%%%%%%%%
% SECTION 6: FINAL FORMULAS
%%%%%%%%%%%%%%%%%%%%%%%%%%%%%%%%%%%%%%%%%%%%%%%%%%%%%%%%%%%%%%%%%%%%%%%%%%%%%%%

\section{Final Formulas for $G_{\text{eff}}$}
\label{sec:ch19_final}

\begin{tcolorbox}[colback=orange!5!white, colframe=orange!50!black,
    title=\textbf{Key Result: $G_{\text{eff}}$ in Natural Normalization}]

For brane-localized fermion current (WD-22-1) with natural normalization
($\int |f_n|^2 d\xi = \ell$):

\begin{equation}
    \boxed{G_{\text{eff}} = \frac{g_5^2 \, \ell}{2 x_1^2} \cdot |f_1(0)|^2}
    \label{eq:ch19_geff_natural}
\end{equation}

where:
\begin{itemize}
    \item $g_5$ = 5D gauge coupling with $[g_5] = L^{1/2}$ \hfill [P]
    \item $\ell$ = domain size with $[\ell] = L$ \hfill [P]
    \item $x_1 = x_1(\kappa, V)$ = first eigenvalue from BVP (dimensionless) \hfill [Dc]
    \item $f_1(0)$ = first mode profile at brane (dimensionless) \hfill [Dc]
\end{itemize}

\textbf{Dimensional check}: $[g_5^2 \ell / x_1^2] = L^2 = \text{GeV}^{-2}$ \checkmark

\textbf{Epistemic status}: [Dc] --- derived from 5D action + KK reduction + brane localization.

\end{tcolorbox}

\begin{tcolorbox}[colback=orange!5!white, colframe=orange!50!black,
    title=\textbf{Key Result: Connection to OPR-20 $C_{\text{eff}}$}]

From OPR-20 (Chapter~\ref{ch:opr20_mediator_mass}, Eq.~20.11):
\begin{equation}
    C_{\text{eff}} = \frac{g_5^2 \, \ell}{x_1^2}
\end{equation}

The relation to $G_{\text{eff}}$ is:
\begin{equation}
    \boxed{G_{\text{eff}} = \frac{1}{2} \, C_{\text{eff}} \cdot |f_1(0)|^2}
    \label{eq:ch19_geff_ceff}
\end{equation}

\textbf{Physical interpretation}:
\begin{itemize}
    \item $C_{\text{eff}}$ is the contact strength \emph{before} specifying fermion localization
    \item $|f_1(0)|^2$ is the ``wavefunction-at-brane'' factor from brane-localized current
    \item Factor of $1/2$ is the Fermi convention for four-fermion operators
\end{itemize}

\end{tcolorbox}

\begin{tcolorbox}[colback=orange!5!white, colframe=orange!50!black,
    title=\textbf{Key Result: $G_{\text{eff}}$ in Unit Normalization}]

For unit normalization ($\int |\tilde{f}_n|^2 d\xi = 1$):

\begin{equation}
    \boxed{G_{\text{eff}} = \frac{g_5^2 \, |\tilde{f}_1(0)|^2}{2 m_1^2}
    = \frac{g_5^2 \, \ell^2 \, |\tilde{f}_1(0)|^2}{2 x_1^2}}
    \label{eq:ch19_geff_unit}
\end{equation}

\textbf{Conversion}: $|\tilde{f}_1(0)|^2 = |f_1(0)|^2 / \ell$

\textbf{Consistency check}:
$\frac{g_5^2 \ell^2 |\tilde{f}_1(0)|^2}{2 x_1^2}
= \frac{g_5^2 \ell^2}{2 x_1^2} \cdot \frac{|f_1(0)|^2}{\ell}
= \frac{g_5^2 \ell |f_1(0)|^2}{2 x_1^2}$ \checkmark

\end{tcolorbox}

%%%%%%%%%%%%%%%%%%%%%%%%%%%%%%%%%%%%%%%%%%%%%%%%%%%%%%%%%%%%%%%%%%%%%%%%%%%%%%%
% SECTION 7: FACTOR AUDIT
%%%%%%%%%%%%%%%%%%%%%%%%%%%%%%%%%%%%%%%%%%%%%%%%%%%%%%%%%%%%%%%%%%%%%%%%%%%%%%%

\section{Factor Audit Table}
\label{sec:ch19_factor_audit}

\begin{table}[h]
\centering
\begin{tabular}{|c|c|c|c|}
\hline
\textbf{Factor} & \textbf{Origin} & \textbf{Expression} & \textbf{Status} \\
\hline
$g_5^2$ & 5D gauge coupling & Parameter & [P] \\
$\ell$ & Domain size & Parameter & [P] \\
$x_1$ & First eigenvalue & $x_1(\kappa, V)$ from BVP & [Dc] \\
$m_1$ & First mass & $x_1 / \ell$ & [Dc] \\
$f_1(0)$ & Mode at brane & From BVP solution & [Dc] \\
$1/2$ & Fermi convention & Standard EFT & [M] \\
$C_{\text{eff}}$ & OPR-20 contact & $g_5^2 \ell / x_1^2$ & [Dc] \\
\hline
\end{tabular}
\caption{Origin and status of each factor in $G_{\text{eff}}$.}
\label{tab:ch19_factor_audit}
\end{table}

%%%%%%%%%%%%%%%%%%%%%%%%%%%%%%%%%%%%%%%%%%%%%%%%%%%%%%%%%%%%%%%%%%%%%%%%%%%%%%%
% SECTION 8: TOY LIMIT
%%%%%%%%%%%%%%%%%%%%%%%%%%%%%%%%%%%%%%%%%%%%%%%%%%%%%%%%%%%%%%%%%%%%%%%%%%%%%%%

\section{Toy Limit: Flat Potential with Neumann BC}
\label{sec:ch19_toy}

For the toy case $V(\xi) = 0$ with Neumann BC ($\kappa_0 = \kappa_\ell = 0$):
\begin{itemize}
    \item $x_1 = \pi$ (OPR-20, Table~20.1)
    \item $f_1(\xi) = \sqrt{2} \cos(\pi \xi / \ell)$ (normalized to $\int |f_1|^2 = \ell$)
    \item $f_1(0) = \sqrt{2}$, so $|f_1(0)|^2 = 2$
\end{itemize}

\textbf{Toy formula}:
\begin{equation}
    G_{\text{eff}}^{\text{(toy)}} = \frac{g_5^2 \, \ell}{2 \pi^2} \cdot 2
    = \frac{g_5^2 \, \ell}{\pi^2}
    \label{eq:ch19_geff_toy}
\end{equation}

\textbf{Caution}: This is the toy limit only. Physical systems have $V(\xi) \neq 0$
and potentially Robin BC, giving different $x_1$ and $f_1(0)$.

%%%%%%%%%%%%%%%%%%%%%%%%%%%%%%%%%%%%%%%%%%%%%%%%%%%%%%%%%%%%%%%%%%%%%%%%%%%%%%%
% SECTION 9: PHYSICAL RUN PATH (OPEN-22-4)
%%%%%%%%%%%%%%%%%%%%%%%%%%%%%%%%%%%%%%%%%%%%%%%%%%%%%%%%%%%%%%%%%%%%%%%%%%%%%%%

\section{Physical Run Path: From OPR-01 to $G_{\text{eff}}$}
\label{sec:ch19_physical_run}

The toy limit (Section~\ref{sec:ch19_toy}) uses $V(\xi) = 0$, which is unrealistic.
The \textbf{physical} potential comes from the 5D Dirac reduction (OPR-21), with
parameters anchored to membrane tension (OPR-01).

\subsection{Physical Potential: Domain Wall from 5D Dirac}
\label{sec:ch19_physical_potential}

From OPR-21 Lemma~2, the effective potential for left-handed fermions in flat space is:
\begin{equation}
    V_L(\xi) = M(\xi)^2 - M'(\xi)
    \label{eq:ch19_vl_physical}
\end{equation}
where $M(\xi)$ is the 5D mass profile. For a domain-wall configuration:
\begin{equation}
    M(\xi) = M_0 \tanh\left(\frac{\xi - \ell/2}{\Delta}\right)
    \label{eq:ch19_mass_profile}
\end{equation}

This gives:
\begin{equation}
    V_L(\xi) = M_0^2 \tanh^2\left(\frac{\xi - \ell/2}{\Delta}\right)
    - \frac{M_0}{\Delta} \operatorname{sech}^2\left(\frac{\xi - \ell/2}{\Delta}\right)
    \label{eq:ch19_vl_explicit}
\end{equation}

\textbf{Chirality asymmetry}: The right-handed potential differs by sign of $M'$:
\begin{equation}
    V_R - V_L = 2M', \quad \text{(origin of V--A structure)}
\end{equation}

\textbf{Epistemic status}: Eq.~\eqref{eq:ch19_vl_physical} is [Dc] from OPR-21.

\subsection{$M_0$ from Membrane Tension (OPR-01)}
\label{sec:ch19_m0_anchor}

From OPR-01 Lemma~4 (scalar kink theory), the bulk mass amplitude relates to
membrane tension:
\begin{equation}
    M_0^2 = \frac{3}{4} y^2 \sigma \Delta
    \label{eq:ch19_m0_sigma}
\end{equation}
where $y$ is the Yukawa coupling and $\sigma$ is the membrane tension.

The dimensionless parameter controlling the spectrum is:
\begin{equation}
    \mu := M_0 \ell = \frac{\sqrt{3}}{2} y \, n \sqrt{\sigma \Delta^3}
    \label{eq:ch19_mu_def}
\end{equation}
where $n = \ell/\Delta$ is the domain-size ratio.

\textbf{Epistemic status}: [Dc] from OPR-01.

\begin{tcolorbox}[colback=red!5!white, colframe=red!50!black,
    title=\textbf{Common Pitfall: Scale Disambiguation}]

\textbf{Do NOT confuse}:
\begin{itemize}
    \item $\mu = M_0 \ell$ \quad (dimensionless eigenvalue parameter)
    \item $M_0 \Delta$ \quad (potential depth at wall)
    \item $\ell$ \quad (domain size; eigenvalue spacing $\sim 1/\ell$)
    \item $\Delta$ \quad (domain wall width; potential shape)
    \item $\delta$ \quad (effective penetration depth; different from $\Delta$)
    \item $R_\xi$ \quad (geometric radius; see OPR-04 taxonomy)
\end{itemize}

\textbf{Hierarchy}: Typically $\delta \ll \Delta \lesssim \ell$.

\textbf{Physical meaning}:
\begin{itemize}
    \item $\Delta$ controls how ``sharp'' the domain wall is
    \item $\ell$ controls how many bound states exist ($N_{\text{bound}} \sim M_0 \ell / \pi$)
    \item $\mu$ is the product that enters BVP eigenvalue counting
\end{itemize}

See OPR-04 (Scale Taxonomy) for the full scale disambiguation.

\end{tcolorbox}

\subsection{Physical BVP Results (OPEN-22-4)}
\label{sec:ch19_physical_results}

Running the pipeline OPR-01 $\to$ OPR-21 $\to$ OPR-22 with physical $V(\xi)$:

\begin{table}[h]
\centering
\begin{tabular}{|c|c|c|c|c|c|}
\hline
$\mu$ & $M_0$ & $N_{\text{bound}}$ & $x_1$ & $|f_1(0)|^2$ & $G_{\text{eff}}/(g_5^2\ell)$ \\
\hline
5 & 1.25 & 2 & 4.70 & 1.54 & 0.035 \\
\textbf{10} & \textbf{2.50} & \textbf{3} & \textbf{7.77} & \textbf{0.57} & \textbf{0.0048} \\
15 & 3.75 & 4 & 10.17 & 0.061 & 0.00030 \\
20 & 5.00 & 5 & 12.00 & 0.0037 & 0.000013 \\
\hline
\end{tabular}
\caption{Physical BVP scan with domain wall potential. Target regime ($N_{\text{bound}}=3$)
    highlighted. Parameters: $n=4$, $\Delta=1$, $y=1$.}
\label{tab:ch19_physical_scan}
\end{table}

\textbf{Key observation}: The domain wall potential $V = M^2 - M'$ gives $N_{\text{bound}} = 3$
at $\mu \approx 10$, \emph{not} at $\mu \in [25, 35]$ as estimated for Pöschl--Teller-like
potentials. This reflects the different spectral properties of the physical potential.

\subsection{Physical $G_{\text{eff}}$ Formula}
\label{sec:ch19_geff_physical}

In the target regime ($\mu = 10$, $N_{\text{bound}} = 3$):
\begin{align}
    x_1 &= 7.77 \notag \\
    |f_1(0)|^2 &= 0.57 \notag \\
    G_{\text{eff}} &= (g_5^2 \ell) \times \frac{0.57}{2 \times 60.4}
    = (g_5^2 \ell) \times 0.0048
    \label{eq:ch19_geff_physical}
\end{align}

\textbf{Comparison to toy limit}:
\begin{itemize}
    \item Toy ($V=0$, Neumann): $G_{\text{eff}}^{\text{(toy)}} = (g_5^2 \ell) / \pi^2 \approx 0.10 \times (g_5^2 \ell)$
    \item Physical ($\mu=10$): $G_{\text{eff}}^{\text{(phys)}} \approx 0.0048 \times (g_5^2 \ell)$
\end{itemize}

The physical potential suppresses $G_{\text{eff}}$ by a factor of $\sim 20$ relative to
the toy limit, due to (1) larger $x_1$ and (2) smaller $|f_1(0)|^2$.

\textbf{Epistemic status}: CONDITIONAL [Dc] --- pipeline operational; parameters
$(\sigma, \Delta, \ell, y, g_5)$ remain [P].

%%%%%%%%%%%%%%%%%%%%%%%%%%%%%%%%%%%%%%%%%%%%%%%%%%%%%%%%%%%%%%%%%%%%%%%%%%%%%%%
% SECTION 10: COMPARISON BOX (OPTIONAL)
%%%%%%%%%%%%%%%%%%%%%%%%%%%%%%%%%%%%%%%%%%%%%%%%%%%%%%%%%%%%%%%%%%%%%%%%%%%%%%%

\section{External Comparison (Not an Input)}
\label{sec:ch19_comparison}

\begin{tcolorbox}[colback=gray!10!white, colframe=gray!50!black,
    title=\textbf{External Comparison Box --- NOT Used as Input}]

The measured Fermi constant is (PDG 2024):
\begin{equation}
    G_F = 1.1663788(6) \times 10^{-5} \, \text{GeV}^{-2}
\end{equation}

If we assume (purely for illustration):
\begin{itemize}
    \item $g_5^2 \approx 0.2 \, \text{GeV}^{-1}$ [P]
    \item $\ell \approx 0.05 \, \text{GeV}^{-1} \approx 0.01 \, \text{fm}$ [P]
    \item $x_1 \approx \pi$ (toy limit)
    \item $|f_1(0)|^2 \approx 2$ (toy limit)
\end{itemize}

Then:
\begin{equation}
    G_{\text{eff}} \approx \frac{0.2 \times 0.05}{2 \times \pi^2} \times 2
    \approx \frac{0.02}{\pi^2} \approx 2 \times 10^{-3} \, \text{GeV}^{-2}
\end{equation}

This is $\mathcal{O}(10^2)$ times larger than $G_F$.

\textbf{Interpretation}: The toy parameters above are not physical. Achieving
$G_{\text{eff}} \sim G_F$ requires either:
\begin{itemize}
    \item Smaller $g_5^2$ (weaker 5D coupling)
    \item Smaller $\ell$ (more compact extra dimension)
    \item Larger $x_1$ (higher eigenvalue from non-trivial potential)
    \item Smaller $|f_1(0)|^2$ (mode suppressed at brane)
\end{itemize}

\textbf{This comparison is NOT used as input.} The EDC derivation stands
independently of whether $G_{\text{eff}}$ numerically matches $G_F$.

\end{tcolorbox}

%%%%%%%%%%%%%%%%%%%%%%%%%%%%%%%%%%%%%%%%%%%%%%%%%%%%%%%%%%%%%%%%%%%%%%%%%%%%%%%
% SECTION 10: SUMMARY
%%%%%%%%%%%%%%%%%%%%%%%%%%%%%%%%%%%%%%%%%%%%%%%%%%%%%%%%%%%%%%%%%%%%%%%%%%%%%%%

\section{Summary and Epistemic Status}
\label{sec:ch19_summary}

\subsection{Main Results}

\textbf{Eq.~\eqref{eq:ch19_geff_natural}}: $G_{\text{eff}}$ in natural normalization
\begin{equation}
    G_{\text{eff}} = \frac{g_5^2 \, \ell}{2 x_1^2} \cdot |f_1(0)|^2
\end{equation}

\textbf{Eq.~\eqref{eq:ch19_geff_ceff}}: Connection to OPR-20
\begin{equation}
    G_{\text{eff}} = \frac{1}{2} \, C_{\text{eff}} \cdot |f_1(0)|^2
\end{equation}

\textbf{Eq.~\eqref{eq:ch19_geff_toy}}: Toy limit ($V=0$, Neumann BC)
\begin{equation}
    G_{\text{eff}}^{\text{(toy)}} = \frac{g_5^2 \, \ell}{\pi^2}
\end{equation}

\subsection{Epistemic Status}

\begin{table}[h]
\centering
\begin{tabular}{|l|c|}
\hline
\textbf{Item} & \textbf{Status} \\
\hline
5D action to 4D EFT structure & [Dc] \\
KK mode expansion & [Dc] \\
Brane-localized current ansatz & [P] (WD) \\
Normalization conventions & [Dc] \\
$G_{\text{eff}}$ formula structure & [Dc] \\
Connection to OPR-19/20 & [Dc] \\
Dimensional verification & [M] \\
\hline
$g_5$ value & [P] \\
$\ell$ value & [P] \\
$V(\xi)$ shape & [P] \\
BC parameters $\kappa$ & [P] \\
\hline
\textbf{Overall OPR-22} & \textbf{CONDITIONAL [Dc]} \\
\hline
\end{tabular}
\caption{Epistemic status of OPR-22 components.}
\label{tab:ch19_epistemic}
\end{table}

\subsection{Open Problems}

\begin{enumerate}
    \item \textbf{OPEN-22-1}: Bulk-distributed current (overlap integral formulation)
    \item \textbf{OPEN-22-2}: Derive $g_5$ from UV completion
    \item \textbf{OPEN-22-3}: Derive $\ell$ from first principles
    \item \textbf{OPEN-22-4}: Compute $f_1(0)$ for physical $V(\xi)$
          --- \textbf{RESOLVED} (Section~\ref{sec:ch19_physical_run})
    \item \textbf{OPEN-22-5}: Brane kinetic term corrections
    \item \textbf{OPEN-22-6}: Multi-mediator corrections (KK tower sum)
\end{enumerate}

\subsection{Cross-References}

\begin{itemize}
    \item \textbf{OPR-19} (Chapter~\ref{ch:opr19_g5}): $g_5 \to g_4$ dimensional reduction
    \item \textbf{OPR-20} (Chapter~\ref{ch:opr20_mediator_mass}): Mediator mass $m_1 = x_1/\ell$
    \item \textbf{OPR-21}: BVP mode profiles (provides $f_n(\xi)$ and $f_1(0)$)
    \item \textbf{OPR-04}: Scale Taxonomy ($\ell$ vs $\Delta$ vs $\delta$ vs $R_\xi$)
\end{itemize}

%%%%%%%%%%%%%%%%%%%%%%%%%%%%%%%%%%%%%%%%%%%%%%%%%%%%%%%%%%%%%%%%%%%%%%%%%%%%%%%
% END OF CHAPTER
%%%%%%%%%%%%%%%%%%%%%%%%%%%%%%%%%%%%%%%%%%%%%%%%%%%%%%%%%%%%%%%%%%%%%%%%%%%%%%%
