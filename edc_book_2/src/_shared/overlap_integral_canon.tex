% =============================================================================
% overlap_integral_canon.tex — Canonical Definition of Overlap Integral I_4
% Created: 2026-01-29
% Purpose: Single source of truth for overlap integral definition
% =============================================================================

\begin{tcolorbox}[colback=green!5!white, colframe=green!50!black,
    title=\textbf{Canonical Definition: Overlap Integral $I_4$}]
\label{box:overlap-integral-canon}

\textbf{Definition:} The overlap integral $I_4$ measures the spatial coincidence
of fermion mode profiles across the brane thickness:
\begin{equation}
I_4 := \int_0^\ell |\psi_L(\xi)|^2 |\psi_R(\xi)|^2 \, d\xi
\label{eq:I4-canon-def}
\end{equation}
where $\psi_{L,R}(\xi)$ are the left- and right-handed fermion profiles
and $\ell$ is the brane thickness scale.

\textbf{Physical meaning:}
\begin{itemize}[nosep]
\item $I_4$ controls the effective 4D coupling strength
\item Small $I_4$ $\Rightarrow$ weak coupling (profiles don't overlap)
\item Large $I_4$ $\Rightarrow$ strong coupling (profiles coincide)
\end{itemize}

\textbf{Role in $G_F$ chain:}
\begin{equation}
G_{\text{eff}} \propto \frac{g_5^2 \cdot I_4}{m_\phi^2}
\end{equation}
The overlap integral enters the numerator; mediator mass enters denominator.
See Chapter~\ref{ch:gf_derivation} for the full derivation pathway.

\textbf{BVP connection:} $I_4$ is computed from the BVP mode profiles
(Chapter~\ref{ch:bvp_workpackage}). The value depends on:
\begin{itemize}[nosep]
\item Potential shape $V(\xi)$
\item Boundary conditions (Neumann/Robin/Dirichlet)
\item Dimensionless depth parameter $\mu$
\end{itemize}

\textbf{k-channel note:} The k-channel correction ($k(N) = 1 + 1/N$) applies
to \emph{averaging} observables, NOT to overlap integrals directly.
See Box~\ref{box:zn-kchannel-robustness} for applicability rules.

\end{tcolorbox}
