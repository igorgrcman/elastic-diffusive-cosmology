% =============================================================================
% scale_disambiguation_box.tex — δ Scale Disambiguation
% Created: 2026-01-29
% Purpose: Prevent confusion between nucleon and electroweak thickness scales
% =============================================================================

\begin{tcolorbox}[colback=blue!5!white, colframe=blue!50!black,
    title=\textbf{Scale Disambiguation: $\delta$ Symbols}]
\label{box:scale-disambiguation}

EDC uses multiple thickness scales. To prevent confusion:

\begin{center}
\begin{tabular}{@{}llll@{}}
\toprule
\textbf{Symbol} & \textbf{Value} & \textbf{Context} & \textbf{Tag} \\
\midrule
$\delta_{\text{nucl}}$ & $\approx 0.105$ fm & Nucleon/junction core & \tagI{} \\
& $= \lambda_p/2 = \hbar/(2m_p c)$ & Proton Compton scale & \\
\addlinespace
$\delta_{\text{EW}}$ & $\approx 0.002$ fm & Electroweak mediator & \tagP{} \\
& $\sim \hbar c/M_Z$ & KK/diffusion scale & \\
\addlinespace
$\ell$ & Variable & BVP interval length & \tagP{} \\
& (context-dependent) & Mode confinement scale & \\
\bottomrule
\end{tabular}
\end{center}

\textbf{Usage rule:} Never use bare $\delta$ without subscript when both scales
appear in the same derivation. In nuclear chapters (neutron lifetime, Mn model),
$\delta$ without subscript means $\delta_{\text{nucl}}$. In weak-sector BVP chapters,
specify explicitly.

\textbf{Conversion:} $\delta_{\text{nucl}}/\delta_{\text{EW}} \approx 50$
(nucleon scale $\gg$ electroweak scale).

\end{tcolorbox}
