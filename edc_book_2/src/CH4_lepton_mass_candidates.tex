% ==============================================================================
% Chapter 4: Candidate Lepton Mass Relations
% Status: [P] — Provisional hypotheses, not derivations
% ==============================================================================

\section{Candidate Lepton Mass Relations}
\label{sec:lepton_mass_candidates}

\begin{tcolorbox}[edcGuardrail, title=\textbf{Epistemic Status: Provisional [P]}]
This section records a numerically striking but \textbf{still provisional}
candidate structure for the charged-lepton masses. The goal is \emph{not}
to claim a derivation, but to document a compact set of relations that:
\begin{enumerate}[nosep]
    \item use only previously fixed EDC quantities, and
    \item reproduce the observed hierarchy to sub-percent level,
\end{enumerate}
while clearly marking the unresolved theoretical gaps.
\end{tcolorbox}

% ==============================================================================
\subsection{Electron Scale (Candidate)}
\label{sec:electron_candidate}

We observe that the dimensionally natural combination of membrane tension,
brane thickness, and the electromagnetic coupling admits a candidate
electron-mass formula:

\begin{edcPostulateBox}{Candidate Electron Mass Formula}{[P]}
\begin{equation}
    m_e \;\stackrel{[P]}{=}\; \pi\,\sqrt{\alpha\,\sigma\,\Delta\,\hbar c}
    \label{eq:electron_candidate}
\end{equation}
where:
\begin{itemize}[nosep]
    \item $\sigma = 5.86$ MeV/fm$^2$ — membrane tension \tagP{} (OPR-01, OPEN)
    \item $\Delta = 3.121 \times 10^{-3}$ fm — brane thickness \tagP{} (OPR-04, OPEN)
    \item $\alpha = 1/137.036$ — fine structure constant at Thomson limit \tagBL{}
    \item $\hbar c = 197.3$ MeV$\cdot$fm \tagBL{}
\end{itemize}
\end{edcPostulateBox}

\paragraph{Numerical evaluation.}
\begin{equation}
    m_e^{(\mathrm{EDC})} = \pi \times \sqrt{\frac{1}{137.036} \times 5.86 \times 3.121 \times 10^{-3} \times 197.3}
    = 0.508~\text{MeV}
\end{equation}

\begin{center}
\begin{tabular}{lcc}
\toprule
\textbf{Quantity} & \textbf{Value} & \textbf{Source} \\
\midrule
$m_e$ (candidate) & 0.508 MeV & Eq.~\eqref{eq:electron_candidate} \\
$m_e$ (experiment) & 0.511 MeV & \tagBL{} \\
\textbf{Deviation} & \textbf{0.6\%} & \\
\bottomrule
\end{tabular}
\end{center}

\paragraph{Theoretical gaps (open).}
\begin{enumerate}[nosep]
    \item The prefactor $\pi$ is not derived; it is recorded as a geometric
          candidate (e.g., defect symmetry, WKB quantization phase).
    \item The role of $\alpha$ in the square root is heuristic (``EM charge
          must enter'') without an explicit 5D/brane derivation.
    \item The convention $\alpha = \alpha(0)$ (Thomson limit) is assumed;
          whether this is the correct scale remains to be justified.
\end{enumerate}

% ==============================================================================
\subsection{Muon/Electron Ratio (Candidate)}
\label{sec:muon_ratio_candidate}

A second empirical regularity appears in the mass ratio:

\begin{edcPostulateBox}{Candidate Muon/Electron Ratio}{[P]}
\begin{equation}
    \frac{m_\mu}{m_e} \;\stackrel{[P]}{=}\;
    \frac{|Z_3|}{|Z_2|} \times \frac{1}{\alpha}
    = \frac{3}{2} \times \frac{1}{\alpha}
    \label{eq:muon_ratio_candidate}
\end{equation}
where $|Z_3| = 3$ and $|Z_2| = 2$ are the orders of the cyclic subgroups
in $\mathbb{Z}_6 \simeq \mathbb{Z}_2 \times \mathbb{Z}_3$.
\end{edcPostulateBox}

\paragraph{Numerical evaluation.}
\begin{equation}
    \left(\frac{m_\mu}{m_e}\right)^{(\mathrm{EDC})}
    = \frac{3}{2} \times 137.036 = 205.55
\end{equation}

\begin{center}
\begin{tabular}{lcc}
\toprule
\textbf{Quantity} & \textbf{Value} & \textbf{Source} \\
\midrule
$m_\mu/m_e$ (candidate) & 205.55 & Eq.~\eqref{eq:muon_ratio_candidate} \\
$m_\mu/m_e$ (experiment) & 206.77 & \tagBL{} \\
\textbf{Deviation} & \textbf{0.6\%} & \\
\bottomrule
\end{tabular}
\end{center}

\begin{tcolorbox}[breakable, enhanced, colback=red!5, colframe=red!50!black,
    title=\textbf{Critical Warning: Unexplained $1/\alpha$ Enhancement}, width=\linewidth]
The factor $1/\alpha \approx 137$ is \textbf{not derived}. In standard physics,
electromagnetic corrections typically produce factors of $\alpha$, $\alpha/\pi$,
or $\ln(1/\alpha)$—\emph{not} $1/\alpha$.

An inverse-coupling enhancement of this magnitude requires a strong mechanism
(e.g., gauge kinetic ``stiffness'' $\sim 1/e^2$, resonant amplification, or
dual structure). Until such a mechanism is identified, this relation remains
a \textbf{phenomenological regularity}, not a derivation.
\end{tcolorbox}

\paragraph{Bridge note (Attempt 3B).}
In Framework v2.0 the large factor $\alpha^{-1}$ is not an independently derived
electromagnetic amplification. It enters through the framework definition of $\alpha$,
which itself encodes the scale separation captured by $m_p/m_e$. The upshot: asking
``where does $1/\alpha$ come from?'' shifts from an open EM question to a closed
repackaging of the proton--electron hierarchy. Attempt~3B (\S\ref{sec:attempt3B})
establishes this systematically.

\paragraph{Derived muon mass.}
Using the candidate electron mass from Eq.~\eqref{eq:electron_candidate}:
\begin{equation}
    m_\mu^{(\mathrm{EDC})} = 0.508 \times 205.55 = 104.4~\text{MeV}
\end{equation}
Experimental value: $m_\mu = 105.66$ MeV \tagBL{} — deviation 1.2\%.

% ==============================================================================
\subsection{Tau Mass and the Koide Constraint}
\label{sec:tau_koide}

\begin{tcolorbox}[breakable, enhanced, colback=orange!5, colframe=orange!50!black,
    title=\textbf{Warning: Not an Independent Prediction}, width=\linewidth]
The tau mass presented here is \textbf{not independently predicted}.
It is determined by imposing the Koide constraint after selecting
$m_e$ and $m_\mu$. This is documented for completeness, not as a
claimed derivation.
\end{tcolorbox}

\paragraph{The Koide relation.}
The empirical Koide formula states:
\begin{equation}
    Q \equiv \frac{m_e + m_\mu + m_\tau}{(\sqrt{m_e} + \sqrt{m_\mu} + \sqrt{m_\tau})^2}
    = \frac{2}{3}
    \label{eq:koide_ch4}
\end{equation}
This holds experimentally to $\sim 0.001\%$ accuracy \tagBL{}.

\paragraph{Tentative $\mathbb{Z}_6$ identification.}
We record the suggestive (but not derived) identification:
\begin{equation}
    Q = \frac{2}{3} \;\;\leftrightarrow\;\; \frac{|Z_2|}{|Z_3|}
    \quad \text{within } \mathbb{Z}_6 \simeq \mathbb{Z}_2 \times \mathbb{Z}_3
    \label{eq:koide_z6}
\end{equation}
This is a structural cue, not yet a derived energetic necessity.

\paragraph{Tau mass from Koide.}
Solving Eq.~\eqref{eq:koide_ch4} with the candidate $(m_e, m_\mu)$ values
yields the quadratic:
\begin{equation}
    \sqrt{m_\tau} = 2A - \sqrt{4A^2 - 3B + 2A^2}
\end{equation}
where $A = \sqrt{m_e} + \sqrt{m_\mu}$ and $B = m_e + m_\mu$.

\begin{center}
\begin{tabular}{lcc}
\toprule
\textbf{Quantity} & \textbf{Value} & \textbf{Source} \\
\midrule
$m_\tau$ (Koide solution) & 1763 MeV & Constraint, not prediction \\
$m_\tau$ (experiment) & 1776.9 MeV & \tagBL{} \\
\textbf{Deviation} & \textbf{0.8\%} & \\
\bottomrule
\end{tabular}
\end{center}

% ==============================================================================
\subsection{Summary of Candidate Relations}
\label{sec:lepton_summary}

\begin{center}
\begin{tabular}{lcccl}
\toprule
\textbf{Quantity} & \textbf{Formula} & \textbf{EDC} & \textbf{Exp.} & \textbf{Status} \\
\midrule
$m_e$ & $\pi\sqrt{\alpha\sigma\Delta\hbar c}$ & 0.508 MeV & 0.511 MeV & [P] \\
$m_\mu/m_e$ & $(3/2)/\alpha$ & 205.55 & 206.77 & [P] \\
$m_\mu$ & $m_e \times (3/2)/\alpha$ & 104.4 MeV & 105.66 MeV & [P] \\
$m_\tau$ & Koide($m_e, m_\mu$) & 1763 MeV & 1776.9 MeV & [P], not indep. \\
\bottomrule
\end{tabular}
\end{center}

% ==============================================================================
\subsection{Open Problems}
\label{sec:lepton_open}

\begin{enumerate}
    \item \textbf{Derive $\pi$:} Find an explicit integral (e.g., WKB phase,
          defect geometry, mode normalization) that produces the prefactor $\pi$
          in the electron mass formula. \textbf{[See OPR-09]}

    \item \textbf{Derive $(3/2)$:} Show that the ratio $|Z_3|/|Z_2| = 3/2$
          in the muon/electron formula emerges from $\mathbb{Z}_6$ mode spectrum
          or oscillator energetics. \textbf{[See OPR-10]}

    \item \textbf{Derive $Q = 2/3$:} Show that the Koide constraint emerges
          from $\mathbb{Z}_6$ energetics (e.g., mode degeneracy, minimal energy
          configuration) rather than being imposed as input. \textbf{[See OPR-11]}

    \item \textbf{Independent $m_\tau$:} Find a formula for $m_\tau$ that does
          not rely on the Koide constraint as input. \textbf{[Linked to OPR-11]}

    \item \textbf{Why three generations?:} Derive the number of lepton
          generations from $\mathbb{Z}_6$ structure (tentatively: $|Z_3| = 3$).
          \textbf{[See OPR-03]}
\end{enumerate}

\begin{tcolorbox}[edcCanonical, title=\textbf{Epistemic Summary}]
All relations in this section are tagged \textbf{[P]}: they represent a
compact hypothesis with strong numerical alignment ($<1\%$ for all masses)
but without first-principles derivation of:
\begin{itemize}[nosep]
    \item the $\pi$ prefactor,
    \item the $1/\alpha$ enhancement, and
    \item the dynamical reason for $Q = 2/3$.
\end{itemize}
These three items define the scope of future work aimed at promoting
the status of these relations beyond [P].
\end{tcolorbox}

% ==============================================================================
\subsection{Attempt 2: Derivation Attempts and Failure Modes}
\label{sec:lepton_attempt2}

We document here a systematic attempt to derive the three key factors
($\pi$, $1/\alpha$, $Q = 2/3$) from first principles. All three attempts
failed to produce a rigorous derivation.

\begin{center}
\begin{tabular}{p{2.2cm}p{2.5cm}p{4.5cm}p{4cm}}
\toprule
\textbf{Target} & \textbf{Outcome} & \textbf{Why it failed} & \textbf{Next step} \\
\midrule
Derive $\pi$ & Motivated, not derived &
    WKB quantization suggests $\pi$ from half-integer ground state,
    but explicit potential $V(z)$ and integral not computed &
    Obtain thick-brane $V(z)$ and evaluate integral \\
\addlinespace
Derive $1/\alpha$ & Resolved (reframed) &
    Systematic audit in Attempt~3B shows that $\alpha^{-1}$ in the candidate
    muon formula is a consequence of how $\alpha$ is defined in terms of
    geometric quantities ($m_p/m_e$), not an independent EM mechanism &
    Keep formula as [P]; pursue $(3/2)$ factor derivation \\
\addlinespace
Derive $Q = 2/3$ & Structure exists, ratio not derived &
    $\mathbb{Z}_6$ parametrization gives $Q = 2/3$ automatically,
    but $D/A = \sqrt{2}$ ratio is observed, not derived &
    Show energetic minimum forces $D = \sqrt{2}\,A$ \\
\bottomrule
\end{tabular}
\end{center}

\paragraph{Critical assessment.}
In Attempt~1, the factor $1/\alpha$ was numerically observed to produce
a 0.6\% match. In Attempt~2, no action-level argument produced a clean
$1/\alpha$ enhancement without ad-hoc $4\pi$ compensation. This remains
the critical gap: typical electromagnetic corrections produce $\alpha$,
$\alpha/\pi$, or $\ln(1/\alpha)$---\emph{not} $1/\alpha$.

\paragraph{Open problems (status: open).}
\begin{itemize}[nosep]
    \item \textbf{$\pi$ derivation:} Explicit integral from defect geometry
          or WKB phase calculation with thick-brane potential.
    \item \textbf{$1/\alpha$ mechanism:} Action-level origin for
          inverse-coupling enhancement (gauge stiffness, resonance, duality).
    \item \textbf{$Q = 2/3$ energetics:} Prove that $\mathbb{Z}_6$ energy
          minimum requires $D/A = \sqrt{2}$, not just cardinality matching.
    \item \textbf{Independent $m_\tau$:} Formula that predicts $m_\tau$
          without using Koide constraint as input.
\end{itemize}

\begin{tcolorbox}[breakable, enhanced, colback=gray!5, colframe=gray!50!black,
    title=\textbf{Methodological Note}, width=\linewidth]
If the $1/\alpha$ mechanism cannot be derived, the candidate ratio
$m_\mu/m_e = (3/2)/\alpha$ should be either:
\begin{enumerate}[nosep]
    \item demoted to ``numerical curiosity'' status, or
    \item abandoned in favor of an alternative approach.
\end{enumerate}
This is the correct scientific response to a failed derivation attempt.
\end{tcolorbox}

% ==============================================================================
\subsection{Attempt 3B: Audit of EM-in-5D Mechanisms for $1/\alpha$}
\label{sec:attempt3B}

A recurring numerical candidate in Attempt~1 was the appearance of an inverse
fine-structure factor, prompting the question: \emph{Can the factor $1/\alpha$
in the muon ratio arise from the electromagnetic sector of EDC---particularly
from 5D electrostatics, where magnetism is not fundamental?}

We tested six physically motivated options:
\begin{enumerate}[nosep]
    \item[(O1)] Full 5D gauge field $A_M$ (Kaluza--Klein reduction) --- \textcolor{red!80!black}{\textbf{RED}}
    \item[(O2)] ``E-only'' static reduction ($A_0 \neq 0$, $\vec{A} = 0$) --- \textcolor{red!80!black}{\textbf{RED}}
    \item[(O3)] Scalar potential model $\Phi$ --- \textcolor{red!80!black}{\textbf{RED}}
    \item[(O4)] Polarization/susceptibility model with $\chi \propto e^2$ --- \textcolor{orange!80!black}{\textbf{YELLOW}}
    \item[(O5)] Kinematic emergence of $\vec{B}$ from motion --- \textcolor{red!80!black}{\textbf{RED}}
    \item[(O6)] Gauge stiffness vs.\ source cancellation --- \textcolor{red!80!black}{\textbf{RED}}
\end{enumerate}
\noindent\textit{Legend:} \textcolor{red!80!black}{\textbf{RED}} = fails ($\alpha$ not inverted);
\textcolor{orange!80!black}{\textbf{YELLOW}} = plausible if $\chi \propto e^2$ derived.

\paragraph{Negative result (robust).}
Across options (O1)--(O3), (O5), and (O6), the coupling dependence follows
the standard pattern: self-energies scale as $e^2 \propto \alpha$, \textbf{not}
$1/\alpha$. Option~(O4) remains open but requires proving $\chi \propto e^2$
from brane physics---an assumption not yet derived.

\begin{tcolorbox}[breakable, enhanced, colback=yellow!6!white, colframe=yellow!45!black,
    title=\textbf{Critical conclusion for the muon ratio}, width=\linewidth]
Attempt~3B provides \textbf{no robust (derivation-level) electromagnetic mechanism}
that generates a clean $1/\alpha$ factor.

However, the audit revealed a deeper point: in Framework v2.0, $\alpha$ is
\emph{defined} as
\begin{equation}
    \alpha \;=\; \frac{4\pi + 5/6}{6\pi^5}
    \quad\Longrightarrow\quad
    \alpha^{-1} \;=\; \frac{m_p/m_e}{4\pi + 5/6}
\end{equation}
The factor $\alpha^{-1}$ in the muon formula is therefore \textbf{not} an independent
EM mechanism---it is a repackaging of the proton/electron mass hierarchy through
the geometric definition of $\alpha$.

\textbf{Verdict:} The candidate ratio $m_\mu/m_e = (3/2)(1 + \alpha^{-1})$ remains
\textbf{[P]}. The $1/\alpha$ question is \emph{resolved by reframing}, not by
finding an EM mechanism. Future effort should focus on deriving the $(3/2)$ factor
from oscillator physics or $\mathbb{Z}_6$ structure.
\end{tcolorbox}

\paragraph{Open problems (status: open).}
\begin{enumerate}[nosep]
    \item Derive $\chi \propto e^2$ for brane polarizability from the EDC action
    \item Derive the $(3/2)$ factor from oscillator spectrum in the $\xi$-dimension
    \item Prove that muon wavefunction extension samples baryon configuration space
\end{enumerate}

\paragraph{Conclusion.}
Therefore, further progress on lepton masses must target an independent dynamical
mechanism (e.g., thick-brane spectrum or susceptibility route), not EM inversion.
