\chapter{Logical Structure of EDC}
\label{app:logical_structure}

This appendix provides a rigorous classification of statements according to their epistemic status.
We adopt the \textbf{canonical epistemic scheme} defined in the Preface (see \S\ref{sec:epistemic_standard}).

%═══════════════════════════════════════════════════════════════════════════════
\section{EDC Epistemic Standard (Formal)}

% NOTE: The macros below are defined once in the main preamble via:
%   % ============================================================
% EDC Epistemic Standard (Canonical)
% Insert once (recommended: Book preamble or Chapter 3)
% ============================================================

% Evidence Status (primary)
\newcommand{\EDCstatus}[1]{\textsc{#1}}
\newcommand{\EDCderived}{\EDCstatus{Derived}}
\newcommand{\EDCidentified}{\EDCstatus{Identified}}
\newcommand{\EDCcalibrated}{\EDCstatus{Calibrated}}
\newcommand{\EDCproposed}{\EDCstatus{Proposed}}

% Role Tag (secondary; optional)
\newcommand{\EDCrole}[1]{\textsf{#1}}
\newcommand{\EDCpostulate}{\EDCrole{Postulate}}
\newcommand{\EDCmath}{\EDCrole{Mathematics}}
\newcommand{\EDCrecovered}{\EDCrole{Recovered}}
\newcommand{\EDCprediction}{\EDCrole{Prediction}}
\newcommand{\EDCconjecture}{\EDCrole{Conjecture}}
\newcommand{\EDCplaceholder}{\EDCrole{Placeholder}}

% Shorthand for tables (optional; avoid C)
\newcommand{\EDCD}{\EDCderived}
\newcommand{\EDCI}{\EDCidentified}
\newcommand{\EDCCal}{\EDCcalibrated}
\newcommand{\EDCP}{\EDCproposed}

% Canonical legend text (reuse verbatim)
\newcommand{\EDCLegendText}{
\noindent\textbf{Epistemic status (canonical).}
Every non-trivial statement carries exactly one \emph{Evidence Status}:
\begin{itemize}
  \item \EDCderived: derived explicitly from stated postulates and established mathematics, with regime stated.
  \item \EDCidentified: motivated mapping between EDC parameters and observed quantities (not unique).
  \item \EDCcalibrated: parameter fixed by observation (declared input).
  \item \EDCproposed: unproven assumption, conjecture, interpretive claim, or placeholder.
\end{itemize}
Optional \emph{Role Tags} may be appended (e.g., \EDCpostulate, \EDCprediction, \EDCrecovered, \EDCconjecture).
}

% ============================================================
% Appendix C patch template
% Replace Appendix C legend with a reference to §3.5
% ============================================================

% Example replacement paragraph for Appendix C introduction:
% "This appendix uses the canonical epistemic classification defined in §3.5.
%  No new categories are introduced here; where older drafts used 'C = Conjecture',
%  such items are now marked as \EDCproposed\ (\EDCconjecture)."

% Here we print the standard (no re-definition, no re-input):
\EDCEpistemicStandard

%═══════════════════════════════════════════════════════════════════════════════
\section{Classification Codes Used in This Appendix}
%═══════════════════════════════════════════════════════════════════════════════

In the tables below, the \textbf{Type} column uses the following codes:

\begin{itemize}
    \item \textbf{P}  = \textbf{Proposed} (conjecture/interpretation/ansatz; not yet derived)
    \item \textbf{M}  = \textbf{Mathematics} (established theorem/identity; not an EDC claim)
    \item \textbf{D}  = \textbf{Derived} (explicit derivation exists elsewhere in the book)
    \item \textbf{I}  = \textbf{Identified} (motivated mapping to observed quantities; not unique)
    \item \textbf{Cal} = \textbf{Calibrated} (parameter fixed by observation; declared input)
    \item \textbf{BL} = \textbf{Baseline} (external reference values/datasets used as declared inputs or benchmarks; not an EDC claim)
\end{itemize}

\noindent\textbf{Note.} Older drafts used \textbf{C} for \emph{Conjecture} and \textbf{R} for \emph{Recovered Result}.
In v17.49, conjectures are classified as \textbf{P} (Proposed), and recovered results are classified as \textbf{D} (Derived), with the optional role tag \emph{Recovered} stated in the \textbf{Notes} column.

%═══════════════════════════════════════════════════════════════════════════════
\section{Foundational Postulates}
%═══════════════════════════════════════════════════════════════════════════════

\begin{center}
\begin{tabular}{|c|p{9cm}|c|p{4cm}|}
\hline
\textbf{ID} & \textbf{Statement} & \textbf{Type} & \textbf{Notes} \\
\hline
P1 & Existence of a 5D Lorentzian Bulk manifold $\mathcal{M}_5$ with metric $G_{AB}$ & P & \EDCpostulate \\
P2 & Existence of a 4D membrane $\Sigma$ embedded in $\mathcal{M}_5$ & P & \EDCpostulate \\
P3 & Compact extra dimension $\xi$ with thickness scale $R_\xi$ & P & \EDCpostulate \\
P4 & Bulk energetic fluid (Plenum) with density $\rho_{\text{Plenum}}$ & P & \EDCpostulate \\
P5 & Membrane tension $\sigma$ (effective surface energy density) & P & \EDCpostulate \\
P6 & ``Scan'' mechanism: time as sequential sampling of $\Sigma$ through Bulk & P & \EDCpostulate \\
\hline
\end{tabular}
\end{center}

%═══════════════════════════════════════════════════════════════════════════════
\section{Derived Field Equations and Structures}
%═══════════════════════════════════════════════════════════════════════════════

\begin{center}
\begin{tabular}{|c|p{9cm}|c|p{4cm}|}
\hline
\textbf{ID} & \textbf{Statement} & \textbf{Type} & \textbf{Notes} \\
\hline
D1 & Maxwell equations on $\Sigma$ arise from linear Bulk oscillations (U(1)) & D & Derived in Theory Core \\
D2 & Yang--Mills equations (SU(3)) arise from nonlinear/vortical regimes & D & Derived in Theory Core \\
D3 & Schrödinger equation arises as diffusion on $\Sigma$ from Bulk viscosity & D & Derived in Ch.~6 \\
D4 & GR emerges as an effective river/flow model of the membrane geometry & D & Derived in Ch.~8--9 \\
\hline
\end{tabular}
\end{center}

%═══════════════════════════════════════════════════════════════════════════════
\section{Identifications and External Inputs}
%═══════════════════════════════════════════════════════════════════════════════

\begin{center}
\begin{tabular}{|c|p{9cm}|c|p{4cm}|}
\hline
\textbf{ID} & \textbf{Statement} & \textbf{Type} & \textbf{Notes} \\
\hline
I1 & $\hbar = \sigma_{\text{eff}} r_e^3/c$ & I & Mapping; not unique \\
I2 & $\alpha = m_e c^2/(\sigma_{\text{eff}} r_e^2)$ & I & Mapping; not unique \\
BL1 & $m_e, m_p, m_Z, \ldots$ numerical values taken from PDG/CODATA & BL & Benchmarking / declared inputs \\
BL2 & $G, c, \varepsilon_0, \ldots$ numerical values taken from CODATA/NIST & BL & Benchmarking / declared inputs \\
\hline
\end{tabular}
\end{center}

%═══════════════════════════════════════════════════════════════════════════════
\section{Recovered Physics}
%═══════════════════════════════════════════════════════════════════════════════

\noindent This section lists results that are already known in mainstream physics and are \textbf{recovered} within EDC.
They are classified as \textbf{D = Derived}, with role tag \emph{Recovered} in the notes.

\begin{center}
\begin{tabular}{|c|p{9cm}|c|p{4cm}|}
\hline
\textbf{ID} & \textbf{Statement} & \textbf{Type} & \textbf{Notes} \\
\hline
R1 & Maxwell equations (standard form) & D & \EDCrecovered \\
R2 & Yang--Mills equations (standard form) & D & \EDCrecovered \\
R3 & Schrödinger equation (standard form) & D & \EDCrecovered \\
R4 & GR weak-field limit and classical tests (as effective theory) & D & \EDCrecovered \\
\hline
\end{tabular}
\end{center}

%═══════════════════════════════════════════════════════════════════════════════
\section{Open Conjectures and Program Items}
%═══════════════════════════════════════════════════════════════════════════════

\noindent This section lists program items that are \textbf{not yet derived} in the current volume.
They are explicitly classified as \textbf{P = Proposed} (Role: Conjecture/Testable), even if they show numerical agreement.

\begin{center}
\begin{tabular}{|c|p{9.3cm}|c|p{4.0cm}|}
\hline
\textbf{ID} & \textbf{Statement} & \textbf{Type} & \textbf{Notes} \\
\hline
K1 & $\dfrac{m_p}{m_e} \overset{?}{=} 6\pi^5$ & P & Empirical target; not derived here\footnotemark \\
K2 & $SU(2)_L$ from quark doublet structure & P & \EDCconjecture; requires explicit construction \\
K3 & Three generations from $n_{\text{max}} = 3$ vibrational modes & P & \EDCconjecture; derivation pending \\
K4 & Dark matter = membrane tension wrinkles & P & Testable conjecture; needs quantitative model \\
K5 & Hubble tension from viscous drag & P & Testable conjecture; needs fit-independent prediction \\
K6 & Singularities replaced by Planck cores & P & \EDCconjecture; requires curvature bounds proof \\
\hline
\end{tabular}
\end{center}

\footnotetext{Benchmark used: PDG value for $m_p/m_e$; comparison is purely numerical and does not constitute an EDC derivation.}

%═══════════════════════════════════════════════════════════════════════════════
\section{Dependency Graph}
%═══════════════════════════════════════════════════════════════════════════════

\noindent The dependency structure of the theory can be represented as a directed acyclic graph (DAG):
\begin{itemize}
    \item Postulates (P1--P6) $\rightarrow$ Action principle $\rightarrow$ Field equations (D1--D4)
    \item Field equations + identifications (I1--I2) $\rightarrow$ numerical predictions and verifications
    \item Baselines (BL1--BL2) appear only as declared inputs or benchmarks (no fitting unless explicitly marked Cal)
\end{itemize}