% EDC_EPISTEMIC_STANDARD_FORMAL.tex
% Printable standard (no macro definitions).

\noindent This section defines the epistemic status codes and the labeling rules used throughout the book
(and in the companion Python verification toolkit). Each nontrivial statement is classified by status
and may additionally carry a role tag (e.g., Recovered, Conjecture, Prediction).

\vspace{0.75em}

\begin{center}
\begin{tabular}{|c|l|p{10.0cm}|}
\hline
\textbf{Code} & \textbf{Name} & \textbf{Definition / criterion} \\
\hline
M  & Mathematics & Established theorem/identity. Not an EDC claim. \\
BL & Baseline & External reference value/dataset used for comparison/benchmarking only (no fitting). \\
Cal & Calibrated & A parameter fixed by observation and declared explicitly as an input. \\
I  & Identified & Motivated mapping from EDC quantities to observables; mapping is not unique. \\
D  & Derived & Explicit derivation is provided in this volume (with a cross-reference). \\
P  & Proposed & Postulate/conjecture/testable idea not derived here (even if numerically accurate). \\
\hline
\end{tabular}
\end{center}

\vspace{0.75em}

\noindent\textbf{Role tags (optional).}
A statement may additionally carry a role tag to clarify intent without changing its status:
\begin{itemize}
  \item \textbf{Recovered}: known result recovered within EDC (status usually D).
  \item \textbf{Prediction}: testable consequence that was not used as input (status usually D or I).
  \item \textbf{Conjecture}: a plausible but unproven mechanism (status P).
  \item \textbf{Placeholder}: section marker / planned derivation (status P).
\end{itemize}

\vspace{0.5em}

\noindent\textbf{Labeling rules.}
\begin{itemize}
  \item Every table row that asserts a nontrivial claim must include a \textbf{Code} (M/BL/Cal/I/D/P).
  \item \textbf{D (Derived)} requires an explicit derivation somewhere in the book \emph{and} a cross-reference.
  \item \textbf{I (Identified)} must state what is being identified and why the mapping is reasonable (and acknowledge non-uniqueness).
  \item \textbf{BL (Baseline)} must name the external source (e.g., PDG/CODATA/NIST) and clarify that it is used for comparison only.
  \item \textbf{Cal (Calibrated)} must state which parameter is fixed and where the numerical value comes from.
  \item \textbf{P (Proposed)} must be explicitly flagged as not yet derived in this volume.
\end{itemize}

