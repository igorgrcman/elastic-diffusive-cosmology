%===============================================================================
% EDC_EPISTEMIC_STANDARD.tex
%===============================================================================
% Purpose:
%   - Define canonical epistemic labels/macros used across the book and code.
%   - Provide ONE printable block via \EDCEpistemicStandard (prints the standard).
%
% Usage:
%   1) In main preamble (before \begin{document}):
%        % ============================================================
% EDC Epistemic Standard (Canonical)
% Insert once (recommended: Book preamble or Chapter 3)
% ============================================================

% Evidence Status (primary)
\newcommand{\EDCstatus}[1]{\textsc{#1}}
\newcommand{\EDCderived}{\EDCstatus{Derived}}
\newcommand{\EDCidentified}{\EDCstatus{Identified}}
\newcommand{\EDCcalibrated}{\EDCstatus{Calibrated}}
\newcommand{\EDCproposed}{\EDCstatus{Proposed}}

% Role Tag (secondary; optional)
\newcommand{\EDCrole}[1]{\textsf{#1}}
\newcommand{\EDCpostulate}{\EDCrole{Postulate}}
\newcommand{\EDCmath}{\EDCrole{Mathematics}}
\newcommand{\EDCrecovered}{\EDCrole{Recovered}}
\newcommand{\EDCprediction}{\EDCrole{Prediction}}
\newcommand{\EDCconjecture}{\EDCrole{Conjecture}}
\newcommand{\EDCplaceholder}{\EDCrole{Placeholder}}

% Shorthand for tables (optional; avoid C)
\newcommand{\EDCD}{\EDCderived}
\newcommand{\EDCI}{\EDCidentified}
\newcommand{\EDCCal}{\EDCcalibrated}
\newcommand{\EDCP}{\EDCproposed}

% Canonical legend text (reuse verbatim)
\newcommand{\EDCLegendText}{
\noindent\textbf{Epistemic status (canonical).}
Every non-trivial statement carries exactly one \emph{Evidence Status}:
\begin{itemize}
  \item \EDCderived: derived explicitly from stated postulates and established mathematics, with regime stated.
  \item \EDCidentified: motivated mapping between EDC parameters and observed quantities (not unique).
  \item \EDCcalibrated: parameter fixed by observation (declared input).
  \item \EDCproposed: unproven assumption, conjecture, interpretive claim, or placeholder.
\end{itemize}
Optional \emph{Role Tags} may be appended (e.g., \EDCpostulate, \EDCprediction, \EDCrecovered, \EDCconjecture).
}

% ============================================================
% Appendix C patch template
% Replace Appendix C legend with a reference to §3.5
% ============================================================

% Example replacement paragraph for Appendix C introduction:
% "This appendix uses the canonical epistemic classification defined in §3.5.
%  No new categories are introduced here; where older drafts used 'C = Conjecture',
%  such items are now marked as \EDCproposed\ (\EDCconjecture)."

%   2) Wherever you want the standard to appear:
%        \EDCEpistemicStandard
%
% IMPORTANT:
%   - This file should be \input exactly once (typically in main.tex preamble).
%   - The printable standard is produced by \EDCEpistemicStandard.
%===============================================================================

% -----------------------------------------------------------------------------
% 1) Robust macro definitions (safe if the file is accidentally input twice)
% -----------------------------------------------------------------------------
\providecommand{\EDCstatus}[1]{\textsc{#1}}

% Evidence Status macros (canonical)
\providecommand{\EDCderived}{\EDCstatus{Derived}}        % D
\providecommand{\EDCidentified}{\EDCstatus{Identified}}  % I
\providecommand{\EDCcalibrated}{\EDCstatus{Calibrated}}  % Cal
\providecommand{\EDCproposed}{\EDCstatus{Proposed}}      % P
\providecommand{\EDCbaseline}{\EDCstatus{Baseline}}      % BL
\providecommand{\EDCmath}{\EDCstatus{Mathematics}}       % M

% Optional Role Tag macros (non-exclusive; can be combined)
\providecommand{\EDCrole}[1]{\textit{#1}}
\providecommand{\EDCpostulate}{\EDCrole{Postulate}}
\providecommand{\EDCprediction}{\EDCrole{Prediction}}
\providecommand{\EDCrecovered}{\EDCrole{Recovered}}
\providecommand{\EDCconjecture}{\EDCrole{Conjecture}}
\providecommand{\EDCansatz}{\EDCrole{Ansatz}}
\providecommand{\EDCplaceholder}{\EDCrole{Placeholder}}

% Short code tokens (useful in tables)
\providecommand{\EDCD}{\EDCstatus{D}}
\providecommand{\EDCI}{\EDCstatus{I}}
\providecommand{\EDCCal}{\EDCstatus{Cal}}
\providecommand{\EDCP}{\EDCstatus{P}}
\providecommand{\EDCBL}{\EDCstatus{BL}}
\providecommand{\EDCM}{\EDCstatus{M}}

% -----------------------------------------------------------------------------
% 2) Canonical legend text (reusable verbatim)
% -----------------------------------------------------------------------------
\providecommand{\EDCLegendText}{%
\noindent\textbf{Epistemic status (canonical).}\par
\noindent Every non-trivial statement in this book carries exactly one \emph{Evidence Status}:
\begin{itemize}
  \item \EDCderived\ (\EDCD): derived explicitly from stated postulates and established mathematics, with regime stated.
  \item \EDCidentified\ (\EDCI): motivated mapping between EDC parameters and observed quantities (not unique).
  \item \EDCcalibrated\ (\EDCCal): parameter fixed by observation (declared input).
  \item \EDCproposed\ (\EDCP): unproven assumption, conjecture, interpretive claim, or \EDCansatz.
\end{itemize}
\noindent Two auxiliary codes are allowed for transparency:
\begin{itemize}
  \item \EDCbaseline\ (\EDCBL): external reference values/datasets used as declared inputs or benchmarks (CODATA/PDG/NIST, etc.); not an EDC claim.
  \item \EDCmath\ (\EDCM): pure mathematics (theorem/identity); not an EDC claim.
\end{itemize}
\noindent Optional \emph{Role Tags} may be appended to any status (e.g., \EDCpostulate, \EDCprediction, \EDCrecovered, \EDCconjecture, \EDCansatz, \EDCplaceholder).\par
}

% -----------------------------------------------------------------------------
% 3) Printable block (call this where you want the standard to appear)
% -----------------------------------------------------------------------------
\providecommand{\EDCEpistemicStandard}{%
\EDCLegendText

\vspace{0.6em}

\noindent\textbf{Labeling rule.}
\begin{enumerate}
  \item Assign exactly one Evidence Status (\EDCD, \EDCI, \EDCCal, \EDCP; optionally \EDCBL/\EDCM).
  \item If helpful, append one or more Role Tags (e.g., \EDCprediction\ or \EDCansatz).
  \item State the regime of validity (assumptions, approximations, and parameter domains) for every \EDCderived\ claim.
\end{enumerate}

\vspace{0.6em}

\noindent\textbf{Recommended table header.}\par
\noindent\begin{tabular}{|l|l|l|}
\hline
\textbf{Statement / Quantity} & \textbf{Status} & \textbf{Notes (regime / inputs)} \\
\hline
\end{tabular}

\vspace{0.6em}

\noindent\textbf{Examples.}
\begin{itemize}
  \item ``$\hbar = \sigma_{\text{eff}} r_e^3/c$'' \quad $\to$ \EDCI\ (mapping), unless $\sigma_{\text{eff}}$ and $r_e$ are independently fixed.
  \item ``$m_Z = \frac{19}{2}\,\frac{m_e}{\alpha^2}$'' \quad $\to$ \EDCD\ for the factor $\frac{19}{2}$, with \EDCBL\ inputs ($m_e,\alpha$).
  \item ``$m_p/m_e \overset{?}{=} 6\pi^5$'' \quad $\to$ \EDCP\ (\EDCconjecture), even if numerically close to PDG.
\end{itemize}
}