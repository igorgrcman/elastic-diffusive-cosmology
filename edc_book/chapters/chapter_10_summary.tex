\chapter{Summary and Open Questions}
\label{ch:summary}

\begin{center}
\textit{What We Have Derived, What Remains Open, and How to Test EDC}
\end{center}

\vspace{1em}

% ═══════════════════════════════════════════════════════════════════════════════
% SECTION 9.1: WHAT WE HAVE DERIVED
% ═══════════════════════════════════════════════════════════════════════════════

\section{What We Have Derived}
\label{sec:derived}

Let us take stock of what EDC achieves in the quark sector:

\begin{center}
\begin{tabular}{|l|l|}
\hline
\textbf{Standard Model Feature} & \textbf{EDC Derivation} \\
\hline
$\mathbb{C}^3$ internal space & Three orthogonal vortex planes in internal space \\
$SU(3)$ symmetry & Norm-preserving rotations in $\mathbb{C}^3$ \\
8 gluons & Dimension of $\mathfrak{su}(3)$ algebra \\
Confinement & Vortex strings have positive tension \\
Glueballs & Closed vortex loops (predicted) \\
Fractional charges & $\mathbb{Z}_3$ topological locking within baryons \\
Integer observable charges & Confinement + flux conservation \\
$|Q_p| = |Q_e|$ & Gauss's theorem for topological flux \\
Proton vs. neutron stability & EM stabilization + quark mass difference \\
\hline
\multicolumn{2}{|c|}{\textbf{Mass Spectrum Ansätze (Section 4.4)}} \\
\hline
$m_p/m_e \approx 1836$ & $(4\pi + \kappa_{3q})/\alpha$ with $\kappa_{3q}=5/6$ (0.005\% match) \\
$m_\mu/m_e \approx 207$ & $(3/2)/\alpha$ (0.6\% match) \\
$m_{\pi^\pm}/m_e \approx 273$ & $2/\alpha$ (0.3\% match) \\
$m_t/m_e \approx 338000$ & $18/\alpha^2$ (0.02\% match) \\
$m_\tau/m_\mu \approx 16.8$ & $17 - 1/6$ (0.1\% match) \\
\hline
Speed of light $c$ & Membrane scanning velocity (modeling choice) \\
\hline
\end{tabular}
\end{center}

\newpage

% ═══════════════════════════════════════════════════════════════════════════════
% SECTION 9.2: WHAT REMAINS OPEN
% ═══════════════════════════════════════════════════════════════════════════════

\section{What Remains Open}
\label{sec:open}

\begin{enumerate}
    \item \textbf{Geometric origin of mass factors}: Why specifically $3/2$ (muon), $2$ (pion), $4\pi + 5/6$ (proton), and $18$ (top)? A first-principles derivation is needed.
    \item \textbf{The $\alpha^2$ mystery}: Why does the top quark have $n=2$ (mass $\propto 1/\alpha^2$) while other particles have $n=1$?
    \item \textbf{Koide formula}: Deriving $Q = 2/3$ for charged leptons from EDC geometry
    \item \textbf{W/Z/Higgs masses}: Extending the mass spectrum pattern to electroweak bosons
    \item \textbf{Full quark spectrum}: Calculating all six quark masses from vortex dynamics
    \item \textbf{Three generations}: Proving (not conjecturing) why $n_{\text{max}} = 3$
    \item \textbf{Electroweak unification}: Deriving $SU(2)_L \times U(1)_Y$ from geometry
    \item \textbf{CKM matrix}: Explaining quark mixing angles
    \item \textbf{Neutrino masses}: Mechanism for small but non-zero masses (possibly $n < 0$?)
    \item \textbf{Planck length derivation}: Deriving $\ell_P$ from EDC without importing $G$
\end{enumerate}

\newpage

% ═══════════════════════════════════════════════════════════════════════════════
% SECTION 9.3: FALSIFIABILITY
% ═══════════════════════════════════════════════════════════════════════════════

\section{Falsifiability}
\label{sec:falsifiability_summary}

A theory that cannot be tested is not physics. EDC makes specific claims that can be tested:

\begin{enumerate}
    \item \textbf{Mass spectrum ansätze}: The geometric factors in the mass formulas must be derivable from first principles. If $\kappa_{3q}$, $3/2$, $2$, and $18$ cannot be computed from the vortex geometry, the ansätze remain numerology:
    \begin{center}
    \begin{tabular}{|l|c|c|c|}
    \hline
    Ratio & Ansatz & Predicted & Measured \\
    \hline
    $m_p/m_e$ & $(4\pi + 5/6)/\alpha$ & 1836.24 & 1836.15 \\
    $m_\mu/m_e$ & $(3/2)/\alpha$ & 205.55 & 206.77 \\
    $m_{\pi^\pm}/m_e$ & $2/\alpha$ & 274.07 & 273.13 \\
    $m_t/m_e$ & $18/\alpha^2$ & 338,020 & 338,083 \\
    $m_\tau/m_\mu$ & $17 - 1/6$ & 16.833 & 16.817 \\
    \hline
    \end{tabular}
    \end{center}
    Any precision measurement significantly deviating from these values would falsify the specific ansatz (though not necessarily the entire EDC framework).
    
    \item \textbf{Glueballs}: EDC predicts stable closed vortex loops. Their masses and quantum numbers should match lattice QCD predictions.
    
    \item \textbf{Excited hadrons}: If generations are vibrational modes, there should be a spectrum of excited states with predictable mass ratios.
    
    \item \textbf{Confinement scale}: The string tension $\sigma$ should be calculable from Bulk parameters and match the observed value:
    \begin{equation}
    \sigma \approx (440 \text{ MeV})^2 \approx 0.2 \text{ GeV}^2 \approx 1 \text{ GeV/fm}
    \end{equation}
\end{enumerate}

The theory progresses from ansatz to derivation as each geometric factor is computed from first principles.

\newpage

% ═══════════════════════════════════════════════════════════════════════════════
% NOTATION SUMMARY
% ═══════════════════════════════════════════════════════════════════════════════

\section{Notation Summary}
\label{sec:notation}

\begin{center}
\begin{tabular}{|c|l|}
\hline
\textbf{Symbol} & \textbf{Meaning} \\
\hline
$\mathcal{M}_5$ & 5D Bulk manifold \\
$\Sigma$ & 3D Membrane (our universe) \\
$w$ & Bulk Time coordinate \\
$\xi$ & Compact internal dimension \\
$R_\xi$ & Membrane thickness (Weak scale, $\sim 10^{-18}$ m) \\
$r_e$ & Topological knot radius (EM scale, $\sim 10^{-15}$ m) \\
$\ell_P$ & Planck length (Intrinsic scale, $\sim 10^{-35}$ m) \\
$\vec{\Phi}$ & Matter field $\in \mathbb{C}^3$ \\
$\phi_i$ & Vortex component ($i = 1, 2, 3$) \\
$f(r)$ & Vortex amplitude profile \\
$n$ & Winding number \\
$U(3)$, $SU(3)$ & Unitary, special unitary groups \\
$\lambda_a$ & Gell-Mann matrices ($a = 1, \ldots, 8$) \\
$f_{abc}$ & Structure constants of $SU(3)$ \\
$\sigma$, $\sigma_{eff}$ & Surface tension (QCD string / membrane) \\
$Q$ & Electric charge \\
$J^A$ & Topological current \\
$v_{\text{scan}}$ & Membrane scanning velocity ($= c$) \\
$T_{\text{int}}$ & Internal tangent space \\
$G_{AB}$ & Bulk metric tensor \\
$g_{\mu\nu}$ & Induced membrane metric \\
\hline
\end{tabular}
\end{center}

\vspace{2em}

\begin{center}
\rule{0.5\textwidth}{0.4pt}

\textit{This book provides the mathematical skeleton of EDC's matter sector. The flesh --- detailed calculations, numerical predictions, experimental tests --- awaits further development. But the skeleton is sound: topology determines physics.}
\end{center}

