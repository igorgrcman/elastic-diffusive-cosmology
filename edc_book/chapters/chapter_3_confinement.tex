\chapter{Confinement and the Strong Force}
\label{ch:confinement}

\begin{center}
\textit{The Geometric Origin of Quark Confinement and Electric Charge}
\end{center}

\vspace{1em}

\noindent In the previous chapter, we derived the $SU(3)$ color symmetry from the geometry of vortices in three-dimensional internal space. We now turn to two of the most profound features of particle physics: \textbf{confinement} (why quarks are never observed in isolation) and \textbf{charge quantization} (why electric charges come in discrete units).

In the Standard Model, confinement is demonstrated by lattice QCD simulations but not analytically proven. In EDC, confinement has a simple geometric origin. Similarly, charge quantization is a postulate in the Standard Model; in EDC, it emerges from topology.

% ═══════════════════════════════════════════════════════════════════════════════
% SECTION 3.1: CONFINEMENT, GLUEBALLS, AND STABILITY
% ═══════════════════════════════════════════════════════════════════════════════

\section{Confinement, Glueballs, and Stability}
\label{sec:confinement}

One of the most mysterious features of the strong force is \textbf{confinement}: quarks are never observed in isolation. In the Standard Model, confinement is demonstrated by lattice QCD simulations but not analytically proven. In EDC, confinement has a simple geometric origin.

\subsection{Quarks as String Endpoints}
\label{subsec:quarks_strings}

In the Standard Model, quarks are point particles that happen to be confined. In EDC, confinement has a geometric origin: \textbf{quarks are not particles but endpoints of vortex strings}.

A vortex filament extends through the Bulk. Where it intersects the membrane, we observe a ``quark.'' The filament cannot end in empty space (this would be a topological discontinuity --- like a magnetic monopole, which has never been observed), so it must either:
\begin{enumerate}
    \item Connect to an anti-vortex (antiquark) $\to$ \textbf{Meson} ($q\bar{q}$)
    \item Join with two other vortices in a Y-junction $\to$ \textbf{Baryon} ($qqq$)
    \item Form a closed loop with no endpoints $\to$ \textbf{Glueball}
\end{enumerate}

\subsection{The String Energy}
\label{subsec:string_energy}

Consider a vortex filament of length $L$ connecting two quarks. The energy stored in the filament is:
\begin{equation}
\boxed{E_{\text{string}} = \sigma \cdot L}
\label{eq:string_energy}
\end{equation}

where $\sigma$ is the \textbf{string tension} (energy per unit length).\footnote{In this chapter, $\sigma$ denotes the QCD string tension with units J/m (energy per length), distinct from the membrane surface tension $\sigma_{eff}$ (units J/m$^2$, energy per area) discussed in Chapter 7. The QCD string tension $\sigma \approx 1$ GeV/fm $\approx 0.2$ GeV$^2$ characterizes confinement, while $\sigma_{eff} \approx 10^{18}$ J/m$^2$ characterizes the membrane's elastic response to topological knots.}

The string tension is determined by the vortex profile. For a vortex along the $z$-axis:
\begin{equation}
\sigma = \int_{\text{cross-section}} d^2x_\perp \left[ \frac{1}{2} |\nabla_\perp \vec{\Phi}|^2 + V(|\vec{\Phi}|) \right]
\label{eq:string_tension}
\end{equation}

This integral over the vortex cross-section gives the energy per unit length --- exactly like the tension in a stretched rubber band.

\subsection{Positivity of String Tension}
\label{subsec:positivity}

\vspace{0.5em}
\noindent\textbf{Lemma 3.1.}
\textit{$\sigma > 0$ for any non-trivial vortex configuration.}

\vspace{0.5em}
\noindent\textit{Proof.}
Both terms in Eq.~(\ref{eq:string_tension}) are non-negative:
\begin{itemize}
    \item $|\nabla_\perp \vec{\Phi}|^2 \geq 0$ (squared magnitude)
    \item $V(|\vec{\Phi}|) \geq 0$ with $V = 0$ only at $|\vec{\Phi}| = v$
\end{itemize}

For a non-trivial vortex, $|\vec{\Phi}| = 0$ at the core and $|\vec{\Phi}| = v$ asymptotically. Therefore $V > 0$ in some region (the core), and the gradient is non-zero in the transition region.

Hence $\sigma > 0$. \hfill $\square$

\vspace{0.5em}
This simple lemma has profound consequences: the string \textit{always} wants to shrink. It costs energy to stretch the string.

\subsection{Proof of Confinement}
\label{subsec:proof_confinement}

\vspace{0.5em}
\noindent\textbf{Theorem 3.1 (Confinement).}
\textit{Isolated quarks cannot exist as asymptotic states.}

\vspace{0.5em}
\noindent\textit{Proof.}
Suppose a single quark exists at position $\vec{x}_0$. This is the endpoint of a vortex filament. The filament must extend somewhere --- either to infinity or to another endpoint.

\textbf{Case 1: Filament extends to infinity.}
\begin{equation}
E = \sigma \cdot L \to \sigma \cdot \infty = \infty
\end{equation}
This state has infinite energy and cannot be realized in any physical system with finite total energy.

\textbf{Case 2: Filament connects to an antiquark at position $\vec{x}_1$.}
\begin{equation}
E = \sigma \cdot |\vec{x}_1 - \vec{x}_0| < \infty
\end{equation}
This is a meson --- finite energy, physically realizable.

\textbf{Case 3: Three filaments meet at a junction.}
\begin{equation}
E = \sigma \cdot (L_1 + L_2 + L_3) < \infty
\end{equation}
This is a baryon --- finite energy, physically realizable.

Therefore, the only finite-energy configurations are color-neutral hadrons. \hfill $\square$

\vspace{0.5em}
This is \textbf{confinement from geometry}. No lattice QCD, no complex dynamics --- just the simple fact that strings have positive tension.

\subsection{Glueballs: Closed Vortex Loops}
\label{subsec:glueballs}

A corollary of the confinement theorem is the existence of \textbf{closed vortex loops} without quark endpoints.

A closed loop of circumference $L_{\text{circ}}$ has finite energy:
\begin{equation}
E_{\text{glueball}} = \sigma \cdot L_{\text{circ}} < \infty
\end{equation}

Such toroidal configurations are topologically stable knots of the pure gauge field --- no quarks, just gluon flux.

\vspace{0.5em}
\noindent\textbf{Prediction.} EDC predicts the existence of \textbf{glueballs} --- massive, color-neutral particles with no quark content. These correspond to the scalar meson candidates observed in QCD, such as $f_0(1500)$ and $f_0(1710)$.

\vspace{0.5em}
The difficulty in experimentally identifying glueballs arises because they mix with quark-antiquark mesons of the same quantum numbers ($J^{PC} = 0^{++}$), obscuring their pure-glue nature. EDC provides a clear geometric picture: glueballs are closed loops, mesons are open strings with quark endpoints.

\subsection{Hadronization}
\label{subsec:hadronization}

What happens when we try to separate quarks (e.g., in a high-energy collision at the LHC)?

As the separation $L$ increases, the string energy $E = \sigma L$ increases. The string is being stretched, storing more and more energy. When $E$ exceeds the rest mass energy of a quark-antiquark pair:
\begin{equation}
E > 2 m_q c^2
\end{equation}
it becomes energetically favorable to \textbf{break the string} by creating a new pair:
\begin{equation}
\text{String} \to \text{String}_1 + \text{String}_2 + (q\bar{q})_{\text{new}}
\end{equation}

This is \textbf{hadronization}: the process by which free quarks immediately become bound into hadrons. It is not a mysterious ``strong force'' --- it is the elastic response of the vortex string to stretching. Pull too hard, and the string snaps, but each fragment immediately re-forms into a hadron.

\subsection{Proton vs. Neutron Stability}
\label{subsec:proton_neutron}

Both the proton ($uud$) and neutron ($udd$) satisfy the geometric requirement of ``three-vortex triangulation'' (forming a closed volume). However, they are not equally stable.

The free neutron decays with half-life $\tau_n \approx 880$ s:
\begin{equation}
n \to p + e^- + \bar{\nu}_e
\end{equation}

The free proton is stable (half-life $> 10^{34}$ years).

\textbf{EDC explanation --- Secondary Stability Factors:}

\begin{enumerate}
    \item \textbf{Electromagnetic stabilization:} The proton has net charge $+1e$. The resulting electromagnetic field creates a self-energy contribution that reinforces the topological knot. The neutron is neutral ($Q=0$), lacking this electromagnetic reinforcement.
    
    \item \textbf{Quark mass difference:} In EDC, quark mass depends on the winding geometry in the $\xi$ dimension:
    \begin{itemize}
        \item Winding $+2/3$ (u-quark): ``Gentler'' twist $\to$ lower energy $\to$ lower mass
        \item Winding $-1/3$ (d-quark): ``Sharper'' twist $\to$ higher energy $\to$ higher mass
    \end{itemize}
    Experimentally: $m_d - m_u \approx 2.3$ MeV.
    
    \item \textbf{Energy balance:} The neutron ($udd$) contains two heavier d-quarks; the proton ($uud$) contains only one. Mass difference: $m_n - m_p \approx 1.3$ MeV. This difference enables the decay $n \to p$ (exothermic process).
\end{enumerate}

\textbf{Geometric interpretation}: The neutron's knot is ``tenser'' than the proton's because it contains two sharply-twisted vortices (d-quarks) instead of one.

\subsection{Mass from Vortex Energy}
\label{subsec:mass_vortex}

Unlike Kaluza-Klein theories (where mass comes from momentum in compact dimensions), EDC derives mass from the \textbf{total energy of the vortex configuration}.

For a static configuration, the mass is:
\begin{equation}
\boxed{M = \frac{1}{c^2} \int_{\mathcal{V}} d^4X \sqrt{|G|} \left[ \frac{1}{2} G^{AB} \partial_A \vec{\Phi}^\dagger \partial_B \vec{\Phi} + V(|\vec{\Phi}|) \right]}
\label{eq:mass_integral}
\end{equation}

where the integral is over the 4D Bulk volume (three spatial dimensions plus $\xi$).

\textbf{Key insight}: Different vortex profiles $f(r)$ yield different masses. This opens the door to explaining the mass hierarchy:
\begin{itemize}
    \item \textbf{Light quarks} ($u$, $d$): Broad, diffuse vortex profiles
    \item \textbf{Heavy quarks} ($c$, $b$, $t$): Concentrated, tight vortex profiles
\end{itemize}

The detailed calculation of quark masses from vortex profiles remains an open problem.

\newpage

% ═══════════════════════════════════════════════════════════════════════════════
% SECTION 3.2: ELECTRIC CHARGE FROM TOPOLOGICAL FLUX
% ═══════════════════════════════════════════════════════════════════════════════

\section{Electric Charge from Topological Flux}
\label{sec:electric_charge}

We have explained color. But what about electric charge? Why do quarks have the peculiar values $+2/3$ and $-1/3$, and why is the electron's charge exactly equal and opposite to the proton's?

\subsection{Charge as Winding Number}
\label{subsec:charge_winding}

Electric charge is associated with the $U(1)$ factor in $U(3) = SU(3) \times U(1)$. Geometrically, this $U(1)$ corresponds to \textbf{rotation in the compact $\xi$ dimension}.

For a field configuration with $\xi$-dependence:
\begin{equation}
\Phi(\xi) \sim e^{in\xi/R_\xi}
\end{equation}
the \textbf{winding number} $n$ counts how many times the phase wraps around as $\xi$ traverses its period.

The electric charge is proportional to this winding:
\begin{equation}
\boxed{Q = \frac{e}{2\pi} \oint d\xi \, \partial_\xi(\arg \Phi) = e \cdot n}
\label{eq:charge_winding}
\end{equation}

For a \textbf{lepton} (electron), which is a simple vortex with $n = -1$:
\begin{equation}
Q_e = -e
\end{equation}

This is the geometric origin of charge quantization: charge is quantized because winding number is quantized.

\subsection{The Fractional Charge Puzzle}
\label{subsec:fractional_puzzle}

Here we encounter a subtlety. If quarks had fractional winding (e.g., $n = 1/3$), the field would not be single-valued:
\begin{equation}
e^{i(1/3)\xi} \xrightarrow{\xi \to \xi + 2\pi} e^{i(1/3)(\xi + 2\pi)} = e^{i\xi/3} \cdot e^{2\pi i/3} \neq e^{i\xi/3}
\end{equation}

This is mathematically inconsistent --- the field must return to itself after one period. How can quarks have fractional charges?

\subsection{Resolution: The $\mathbb{Z}_3$ Topological Locking}
\label{subsec:z3_locking}

The resolution lies in the \textbf{composite nature of baryons}.

\vspace{0.5em}
\noindent\textbf{Postulate 3.1 ($\mathbb{Z}_3$ Topological Locking).}
\textit{Within a baryon, the three quarks share the single $\xi$ coordinate through a topological locking mechanism. Each quark wavefunction is constrained to a $120°$ ($2\pi/3$) sector of the $\xi$-circle.}

\begin{tcolorbox}[colback=yellow!10,colframe=orange!80!black,title=Important Note on Status]
Postulate 3.1 is \textbf{additional} to the three foundational postulates of EDC (Bulk, Causality, Membrane). It is required to explain fractional quark charges but is \textbf{not derived} from geometry alone. A complete theory would derive this locking mechanism from dynamics.
\end{tcolorbox}

\vspace{0.5em}
\textbf{Physical interpretation}: Imagine a round table (the full $\xi$-circle of $2\pi$). Three people (three quarks) sit at the table, each ``controlling'' $120°$ in front of them. The table has not changed --- it is merely partitioned.

Consider a baryon with total winding $n_{\text{total}} = 1$. This winding is \textbf{distributed} among three quarks:
\begin{equation}
n_1 + n_2 + n_3 = 1
\end{equation}

If the distribution respects $\mathbb{Z}_3$ symmetry, the possibilities include:
\begin{itemize}
    \item $(+2/3, +2/3, -1/3)$: Two quarks contribute $+2/3$, one contributes $-1/3$ $\to$ \textbf{Proton} ($uud$)
    \item $(+2/3, -1/3, -1/3)$: One quark contributes $+2/3$, two contribute $-1/3$ $\to$ \textbf{Neutron} ($udd$)
\end{itemize}

\textbf{Key point}: This is not a change in the topology of the manifold, but a \textbf{partitioning of the domain} within the baryon. The mechanism arises from energy minimization --- the three vortices ``negotiate'' the division of $\xi$-space to minimize total energy.

\subsection{Charge Quantization}
\label{subsec:charge_quantization}

\vspace{0.5em}
\noindent\textbf{Theorem 3.2 (Charge Quantization).}
\textit{All observable (color-neutral) configurations have integer electric charge.}

\vspace{0.5em}
\noindent\textit{Proof.}
A color-neutral configuration must have quarks combining to form an $SU(3)$ singlet.

\textbf{For baryons}: This requires exactly three quarks (one of each color). The total winding around $\xi$ is:
\begin{equation}
n_{\text{total}} = n_1 + n_2 + n_3 \in \mathbb{Z}
\end{equation}
because the complete $\xi$-circle is covered by the three sectors.

\textbf{For mesons}: This requires a quark-antiquark pair. The antiquark has opposite winding:
\begin{equation}
n_{\text{total}} = n_q + n_{\bar{q}} \in \mathbb{Z}
\end{equation}

In either case:
\begin{equation}
Q_{\text{total}} = e \cdot n_{\text{total}} \in e\mathbb{Z}
\end{equation}
\hfill $\square$

\vspace{0.5em}
Fractional charges exist, but they are always \textit{confined} inside hadrons. Only integer charges can be isolated --- exactly as observed.

\subsection{Clarification: Color vs. Flavor}
\label{subsec:color_flavor}

It is crucial to distinguish between \textbf{color} and \textbf{flavor}:

\begin{itemize}
    \item \textbf{Color} (Section~\ref{sec:su3_derivation}): The $SU(3)$ quantum number associated with vortex orientation in internal space. All quarks carry color; color determines the strong interaction.
    
    \item \textbf{Flavor}: The quantum number distinguishing up-type quarks ($u$, $c$, $t$) from down-type quarks ($d$, $s$, $b$). Flavor determines the electric charge.
\end{itemize}

The electric charge assignments are:
\begin{align}
Q(u) = Q(c) = Q(t) &= +\frac{2}{3}e \\
Q(d) = Q(s) = Q(b) &= -\frac{1}{3}e
\end{align}

\textbf{These charges are the same for all three colors.} A red up-quark, a green up-quark, and a blue up-quark all have $Q = +2/3$.

\newpage

