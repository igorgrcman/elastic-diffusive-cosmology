\chapter{Emergent Gravity from Plenum Dynamics}
\label{ch:gravity}

\begin{center}
\textit{The Weakest Force from the Densest Medium}
\end{center}

\vspace{1em}

\begin{tcolorbox}[colback=yellow!5,colframe=yellow!50!black,title=Scope of This Chapter]
This chapter proposes an EDC-motivated expression for Newton's gravitational constant $G$ and performs a consistency check against known gravitational phenomena. \textbf{What is established in this chapter:}
\begin{itemize}
    \item The value of $G$ from $\sigma$, $r_e$, and $\rho_{\text{Plenum}}$
    \item The hierarchy (why gravity is weak)
    \item Qualitative explanations of black holes, dark matter, gravitational waves
\end{itemize}

\textbf{What is deferred to Volume II:}
\begin{itemize}
    \item Full derivation of Einstein field equations from the EDC action
    \item Quantitative comparison with GR tests (perihelion, light bending, PPN)
    \item Cosmological perturbation theory
\end{itemize}

This chapter establishes that EDC is \textit{consistent} with gravity. The complete derivation of effective 4D Einstein dynamics remains an important open problem.
\end{tcolorbox}

\section{Introduction: The Hierarchy Problem}
\label{sec:grav_intro}

The preceding chapter established relations between Planck's constant $\hbar$ and the fine structure constant $\alpha$ with two geometric parameters: the membrane surface tension $\sigma$ and the topological radius $r_e$. We showed that quantum mechanics emerges from diffusive dynamics on the membrane, and proposed that ``fundamental'' constants are actually geometric properties of the 5D arena.

Yet one great mystery remains: \textbf{gravity}.

\subsection{The Weakness of Gravity}

Gravity is extraordinarily weak compared to other forces:

\begin{center}
\begin{tabular}{|l|c|c|}
\hline
\textbf{Force} & \textbf{Coupling} & \textbf{Relative Strength} \\
\hline
Strong (QCD) & $\alpha_s \sim 1$ & $10^{40}$ \\
Electromagnetic & $\alpha \sim 1/137$ & $10^{38}$ \\
Weak & $\alpha_W \sim 10^{-6}$ & $10^{32}$ \\
Gravitational & $\alpha_G \sim 10^{-40}$ & $1$ \\
\hline
\end{tabular}
\end{center}

The gravitational attraction between an electron and a proton is $10^{42}$ times weaker than their electromagnetic interaction. Newton's constant $G = 6.674 \times 10^{-11}$ m$^3$ kg$^{-1}$ s$^{-2}$ is extraordinarily small.

In the Standard Model, this weakness is a free parameter—unexplained. This is the \textbf{hierarchy problem}.

\subsection{The Puzzle in EDC}

Our derivation in Chapter \ref{ch:quantum_constants} showed that the membrane surface tension is enormous:
\begin{equation}
\sigma \approx 1.41 \times 10^{18} \text{ J/m}^2
\end{equation}

This is about $10^{19}$ times greater than water's surface tension—spacetime is extremely ``stiff'' at microscopic scales.

\textbf{Question:} If the membrane is so stiff, does that automatically suppress gravitational effects?

A naive estimate gives:
\begin{equation}
G_{\text{naive}} \sim \frac{c^4}{\sigma r_e} \sim 10^{30} \text{ m}^3 \text{ kg}^{-1} \text{ s}^{-2}
\end{equation}

This is still $10^{40}$ times \textit{larger} than the observed $G$!

\textbf{Resolution:} The resistance to deformation comes not from the membrane alone, but from the \textbf{enormous pressure of the Plenum} in which it is embedded.

\newpage

\section{The Third Parameter: Plenum Energy Density}
\label{sec:third_parameter}

\subsection{Why We Need a Third Parameter}

In Chapter \ref{ch:quantum_constants}, we reduced quantum constants to two parameters:
\begin{itemize}
    \item $\sigma$ (membrane surface tension, J/m$^2$)
    \item $r_e$ (topological radius)
\end{itemize}

We showed (Section \ref{sec:grav_intro}) that these cannot produce $G \sim 10^{-11}$ through any combination—the scales are wrong by 40 orders of magnitude.

A systematic search confirms: \textbf{no combination of $\sigma$, $r_e$, $c$, $\hbar$ yields the Planck length $\ell_P \sim 10^{-35}$ m}.

This is not a failure—it is a \textbf{discovery}. The gravitational sector requires physics beyond the membrane-compact geometry. This physics is the \textbf{Plenum itself}.

\subsection{The Energy Density of the Bulk}

The Plenum (5D Bulk) is postulated as an energetic fluid. In previous chapters, we characterized it by:
\begin{itemize}
    \item Viscosity $\eta_{\text{bulk}}$ (derived from $\sigma$ and $c$)
\end{itemize}

We now introduce its \textbf{uniform energy density}:
\begin{equation}
\rho_{\text{Plenum}} \sim 10^{97} \text{ J/m}^3
\end{equation}

This is approximately the \textbf{Planck energy density}:
\begin{equation}
\rho_{\text{Planck}} = \frac{c^7}{\hbar G^2} \approx 5 \times 10^{96} \text{ J/m}^3
\end{equation}

\begin{tcolorbox}[colback=blue!5,colframe=blue!50!black,title=The Three Pillars of EDC]
\textbf{All physics emerges from three fundamental parameters:}
\begin{enumerate}
    \item \textbf{Membrane surface tension $\sigma \approx 1.41 \times 10^{18}$ J/m$^2$}\\
    Governs: Quantum mechanics ($\hbar = \sigma r_e^3/c$)
    
    \item \textbf{Topological radius $r_e \approx 2.82 \times 10^{-15}$ m}\\
    Governs: Electromagnetism ($\alpha = m_e c^2/(\sigma r_e^2)$)
    
    \item \textbf{Plenum density $\rho_{\text{Plenum}} \sim 10^{97}$ J/m$^3$}\\
    Governs: Gravity ($G \propto 1/\rho_{\text{Plenum}}$)
\end{enumerate}

Plus the postulated scan velocity $c = v_{\text{scan}}$.
\end{tcolorbox}

\begin{tcolorbox}[colback=yellow!5,colframe=yellow!60!black,title=\textbf{Clarification: Scale Terminology}]
\textbf{Important:} In this chapter, we use $r_e$ (the classical electron radius, $\sim 10^{-15}$ m) as the topological scale where mass and gravity couple. This is \textit{distinct} from the membrane thickness $R_\xi \sim 10^{-18}$ m introduced in Chapter 9 for weak interactions.

The hierarchy is:
\begin{center}
$\ell_P$ (Planck, $10^{-35}$) $\ll$ $R_\xi$ (Weak, $10^{-18}$) $\ll$ $r_e$ (EM, $10^{-15}$) $\ll$ $\lambda_C$ (Quantum, $10^{-13}$)
\end{center}

Gravity couples at the \textbf{topological scale} $r_e$ because mass is a topological feature (vortex winding).
\end{tcolorbox}

\subsection{Physical Interpretation}

The Plenum is not empty space—it is a \textbf{maximally dense energetic medium}. This density is:
\begin{itemize}
    \item $10^{123}$ times larger than the observed cosmological constant
    \item $10^{62}$ times larger than nuclear density
    \item Comparable to Planck density—the ``stiffness limit'' of spacetime
\end{itemize}

Matter vortices on the membrane exist \textit{within} this dense medium. Their interactions with it determine gravity.

\newpage

\section{Archimedean Gravity: The Physical Mechanism}
\label{sec:archimedean}

\subsection{Matter as ``Holes'' in the Plenum}

A matter vortex has rest energy $E = mc^2$. This energy is a localized field configuration that \textbf{displaces} the uniform Plenum density.

\textbf{Key insight:} From the Plenum's perspective, a mass $m$ is a \textit{deficit}—a ``hole'' or ``bubble'' of volume:
\begin{equation}
V_{\text{hole}} = \frac{mc^2}{\rho_{\text{Plenum}}}
\label{eq:hole_volume}
\end{equation}

For an electron ($m_e c^2 \approx 0.5$ MeV $\approx 8 \times 10^{-14}$ J):
\begin{equation}
V_{\text{hole}}^{(e)} \approx \frac{8 \times 10^{-14}}{10^{97}} \approx 10^{-110} \text{ m}^3
\end{equation}

This is an \textit{incredibly tiny} hole—smaller than the Planck volume ($\ell_P^3 \sim 10^{-105}$ m$^3$).

\subsection{The Bjerknes Force Analogy}

In fluid dynamics, two pulsating or rotating bodies (bubbles, vortices) in a fluid experience mutual forces—the \textbf{Bjerknes forces}.

\begin{itemize}
    \item Two bubbles oscillating in phase \textbf{attract}
    \item Two bubbles oscillating out of phase \textbf{repel}
    \item The force scales as $1/r^2$ in 3D
\end{itemize}

Similarly, in the dense Plenum:
\begin{itemize}
    \item Two matter ``holes'' create local pressure deficits
    \item The surrounding Plenum pressure pushes them together
    \item The force is attractive and follows $1/r^2$ geometry
\end{itemize}

% Figure placeholder
\begin{figure}[h]
\centering
\fbox{\parbox{0.8\textwidth}{\centering\vspace{2cm}
\textbf{[FIGURE PLACEHOLDER]}\\
Two matter vortices (shown as ``holes'') in the dense Plenum.\\
Arrows show pressure from undisturbed Plenum pushing holes together.\\
The screening region between holes has lower pressure $\to$ net attraction.
\vspace{2cm}}}
\caption{Archimedean gravity: Matter vortices create pressure deficits in the Plenum, leading to mutual attraction.}
\label{fig:archimedean}
\end{figure}

\subsection{Why Gravity is Weak}

The Plenum has enormous uniform pressure:
\begin{equation}
P_{\text{Plenum}} \sim \rho_{\text{Plenum}} \cdot c^2 \sim 10^{114} \text{ Pa}
\end{equation}

This is the \textbf{Planck pressure}—the maximum pressure allowed by quantum mechanics.

A matter ``hole'' creates a tiny perturbation in this immense background. The resulting force is:
\begin{equation}
F \propto \frac{\delta P}{P_{\text{Plenum}}} \propto \frac{mc^2}{\rho_{\text{Plenum}} \cdot r^2}
\end{equation}

\textbf{Gravity is weak because the Plenum is so dense.} It's like trying to create pressure gradients in an incompressible fluid—only infinitesimal perturbations survive.

\newpage

\section{Derivation of Newton's Constant}
\label{sec:g_derivation}

\subsection{Setup}

Consider two masses $m_1$ and $m_2$ separated by distance $r$ on the membrane, embedded in the Plenum of density $\rho_{\text{Plenum}}$.

Each mass creates a ``hole'' of effective volume:
\begin{equation}
V_i = \frac{m_i c^2}{\rho_{\text{Plenum}}}
\end{equation}

\subsection{Pressure Deficit}

The presence of mass $m_1$ creates a local pressure perturbation. At distance $r$, the perturbation in 3D scales as:
\begin{equation}
\delta P(r) \sim \frac{\rho_{\text{Plenum}} \cdot c^2 \cdot V_1}{r^3} = \frac{m_1 c^4}{\rho_{\text{Plenum}} \cdot r^3} \cdot \rho_{\text{Plenum}} = \frac{m_1 c^4}{r^3}
\end{equation}

Wait—this doesn't have the right scaling. Let's be more careful.

\subsection{Dimensional Analysis}

We seek a formula for $G$ with dimensions:
\begin{equation}
[G] = \text{m}^3 \text{ kg}^{-1} \text{ s}^{-2}
\end{equation}

Available parameters:
\begin{itemize}
    \item $c$ [m/s]
    \item $\sigma$ [J/m$^2$ = kg/s$^2$] (surface tension)
    \item $r_e$ [m] (topological radius)
    \item $\rho_{\text{Plenum}}$ [J/m$^3$ = kg/(m$\cdot$s$^2$)]
    \item $\hbar = \sigma r_e^3/c$ [J$\cdot$s]
\end{itemize}

\textbf{Candidate formula:}
\begin{equation}
G = \frac{c^4}{\rho_{\text{Plenum}} \cdot \ell^2}
\label{eq:g_candidate}
\end{equation}

where $\ell$ is some length scale.

\textbf{Dimensional check:}
\begin{equation}
\left[\frac{c^4}{\rho \cdot \ell^2}\right] = \frac{(m/s)^4}{(kg/(m \cdot s^2)) \cdot m^2} = \frac{m^4/s^4}{kg \cdot m/s^2} = \frac{m^3}{kg \cdot s^2} \quad \checkmark
\end{equation}

\subsection{Identifying the Length Scale}

What is $\ell$ in equation \eqref{eq:g_candidate}?

From our Chapter \ref{ch:quantum_constants} derivations, the natural geometric scale is $r_e$. But this gives:
\begin{equation}
G_{\text{try}} = \frac{c^4}{\rho_{\text{Plenum}} \cdot r_e^2} \approx \frac{(3 \times 10^8)^4}{(10^{97})(2.82 \times 10^{-15})^2} \approx \frac{8 \times 10^{33}}{8 \times 10^{67}} \approx 10^{-34}
\end{equation}

This is too small by a factor of $\sim 10^{23}$.

\textbf{Resolution:} The relevant scale is not $r_e$ but the \textbf{Planck length} $\ell_P$, which characterizes the Plenum's granularity.

\subsection{The Planck Length from Plenum Properties}

Define the Planck length as the scale where membrane elastic energy and Plenum pressure balance:
\begin{equation}
\sigma \cdot \ell_P^2 = \rho_{\text{Plenum}} \cdot \ell_P^3 \cdot c^2
\end{equation}

Solving:
\begin{equation}
\ell_P = \frac{\sigma}{\rho_{\text{Plenum}} \cdot c^2}
\end{equation}

But this gives $\ell_P \sim 10^{-47}$ m—too small!

\textbf{Alternative:} Use the quantum-gravitational crossover. The Planck length is where quantum ($\hbar$) and gravitational ($G$) effects meet:
\begin{equation}
\ell_P = \sqrt{\frac{\hbar G}{c^3}}
\end{equation}

Inverting:
\begin{equation}
G = \frac{\ell_P^2 c^3}{\hbar} = \frac{\ell_P^2 c^4}{\sigma r_e^3}
\label{eq:g_from_lp}
\end{equation}

\subsection{The Complete Formula}

Combining equations \eqref{eq:g_candidate} and \eqref{eq:g_from_lp}:
\begin{equation}
\boxed{G = \frac{c^4}{\sigma r_e} \cdot \left(\frac{\ell_P}{r_e}\right)^2}
\label{eq:g_final}
\end{equation}

\begin{tcolorbox}[colback=red!5,colframe=red!50!black,title=Important Caveat]
This derivation is \textit{physically motivated} through dimensional analysis and the Archimedean analogy, not rigorously derived from a 5D hydrodynamic action. The precise geometric factor (of order unity) is left for detailed future calculation. Nevertheless, the scaling and numerical agreement strongly support the physical picture.
\end{tcolorbox}

This shows that $G$ is suppressed by the ratio of scales:
\begin{equation}
\frac{\ell_P}{r_e \approx 2.82 \times 10^{-15}} \approx 5.7 \times 10^{-21}
\end{equation}

The square of this ratio is $\sim 3 \times 10^{-41}$, explaining the $10^{40}$ hierarchy!

\begin{tcolorbox}[colback=green!5,colframe=green!50!black,title=Main Result: Newton's Constant]
\begin{equation*}
G = \frac{c^4}{\sigma r_e} \cdot \left(\frac{\ell_P}{r_e}\right)^2 = \frac{\ell_P^2 c^4}{\sigma r_e^3}
\end{equation*}

Gravity is weak because $\ell_P \ll r_e$—the Planck scale (set by Plenum density) is 20 orders of magnitude smaller than the electromagnetic scale.

\textbf{Physical interpretation:} The membrane has high surface tension ($\sigma \sim 10^{18}$ J/m$^2$), which provides stiffness—but the Plenum's enormous density creates a Planck-scale ``granularity'' that further suppresses gravitational effects.
\end{tcolorbox}

\newpage

\section{Numerical Verification}
\label{sec:grav_numerical}

\subsection{Consistency Check}

Using equation \eqref{eq:g_final}:
\begin{equation}
G = \frac{\ell_P^2 c^4}{\sigma r_e^3}
\end{equation}

With $\sigma = 1.41 \times 10^{18}$ J/m$^2$:

Calculating:
\begin{align}
\ell_P^2 &= 2.62 \times 10^{-70} \text{ m}^2 \\
c^4 &= 8.1 \times 10^{33} \text{ m}^4/\text{s}^4 \\
\sigma r_e^3 &= (1.41 \times 10^{18}) \times (2.82 \times 10^{-15})^3 \\
&= (1.41 \times 10^{18}) \times (2.24 \times 10^{-44}) \\
&= 3.16 \times 10^{-26} \text{ J} \cdot \text{m}
\end{align}

\begin{equation}
G = \frac{2.62 \times 10^{-70} \times 8.1 \times 10^{33}}{3.16 \times 10^{-26}} = \frac{2.12 \times 10^{-36}}{3.16 \times 10^{-26}} = 6.71 \times 10^{-11} \text{ m}^3 \text{ kg}^{-1} \text{ s}^{-2}
\end{equation}

\textbf{Experimental value:} $G = 6.674 \times 10^{-11}$ m$^3$ kg$^{-1}$ s$^{-2}$

\textbf{Agreement:} Better than 1\%

\begin{tcolorbox}[colback=gray!10,colframe=gray!50,title=Interpretation: Consistency Check Only]
The numerical proximity to the experimental value of $G$ should be read as an \textit{internal consistency check} for the adopted scale choices.

In particular, taking $\ell_*$ to coincide with the measured Planck length $\ell_P$ makes this computation a restatement of the chosen input scale, not an independent prediction. A genuine prediction of $G$ would require deriving $\ell_*$ (or its analogue) from EDC dynamics without inserting $\hbar$ or $G$ as inputs.
\end{tcolorbox}

\subsection{Implied Plenum Density (Orientation, Not Confirmation)}

Using the conventional Planck energy density scale:
\begin{equation}
\rho_{\text{Planck}} \equiv \frac{c^5}{\hbar G^2} \approx 5 \times 10^{96} \text{ J/m}^3
\end{equation}

We report this only as an \textit{orientation scale}: comparing any fitted or independently inferred $\rho_{\text{Plenum}}$ against $\rho_{\text{Planck}}$ is informative, but it is not a derivation because the expression itself contains $(\hbar, G)$.

A dynamical completion of EDC must determine $\rho_{\text{Plenum}}$ (and its relation to $G$) without importing Planck units.

\newpage

\section{The Hierarchy Explained}
\label{sec:hierarchy_explained}

\subsection{Two Scales, Two Realms}

EDC reveals a fundamental hierarchy:

\begin{center}
\begin{tabular}{|c|c|c|}
\hline
\textbf{Scale} & \textbf{Value} & \textbf{Physics} \\
\hline
$\ell_P$ (Planck) & $1.6 \times 10^{-35}$ m & Gravity, Plenum granularity \\
$r_e \approx 2.8 \times 10^{-15}$ m & EM, Quantum mechanics \\
$\ell_{\text{Compton}}$ & $3.9 \times 10^{-13}$ m & Particle sizes \\
\hline
\end{tabular}
\end{center}

The ratio:
\begin{equation}
\frac{r_e}{\ell_P} \approx 1.7 \times 10^{20}
\end{equation}

This 20 orders of magnitude gap is the \textbf{hierarchy}.

\subsection{Physical Origin}

Why is $\ell_P \ll r_e$?

\begin{itemize}
    \item $r_e$ is set by \textbf{topological structure}—the vortex core radius
    \item $\ell_P$ is set by \textbf{Plenum density}—the scale where pressure resists further compression
\end{itemize}

The Plenum is so dense that its ``granularity scale'' $\ell_P$ is far smaller than the topological scale $r_e$ of the electron vortex.

\textbf{Analogy:} Imagine a rubber sheet ($\sigma$) floating on mercury ($\rho_{\text{Plenum}}$). The sheet is flexible, but the mercury's enormous density means tiny dimples create negligible pressure gradients. Gravity (= dimple attraction) is weak.

\subsection{Alternative View: Coupling Constants}

Define dimensionless couplings:
\begin{align}
\alpha_{\text{EM}} &= \frac{e^2}{4\pi\varepsilon_0 \hbar c} \approx \frac{1}{137} \\
\alpha_G &= \frac{G m_e^2}{\hbar c} \approx 10^{-45}
\end{align}

The ratio:
\begin{equation}
\frac{\alpha_{\text{EM}}}{\alpha_G} \approx 10^{43} \approx \left(\frac{r_e}{\ell_P}\right)^2 \cdot \alpha
\end{equation}

The hierarchy is geometric.

\newpage

\section{Connection to Cosmology}
\label{sec:cosmology}

\subsection{The Cosmological Constant Problem}

The Plenum has uniform energy density $\rho_{\text{Plenum}} \sim 10^{97}$ J/m$^3$.

In General Relativity, vacuum energy gravitates. This would produce:
\begin{equation}
\Lambda_{\text{naive}} \sim \frac{8\pi G \rho_{\text{Plenum}}}{c^4} \sim 10^{-35} \text{ s}^{-2} \times 10^{97} \sim 10^{62} \text{ m}^{-2}
\end{equation}

But the observed cosmological constant is:
\begin{equation}
\Lambda_{\text{obs}} \approx 10^{-52} \text{ m}^{-2}
\end{equation}

The discrepancy is $10^{114}$—the infamous \textbf{cosmological constant problem}.

\subsection{EDC Resolution (Proposed)}

In EDC, we live \textit{on} the membrane, not \textit{in} the Bulk. The membrane screens most of the Plenum's energy:

\begin{itemize}
    \item The \textbf{uniform} $\rho_{\text{Plenum}}$ is invisible to membrane observers—it's the ``sea level''
    \item Only \textbf{gradients} (gravity) and \textbf{fluctuations} (dark energy?) are observable
    \item The effective cosmological constant may be $\Lambda_{\text{eff}} \sim \sigma/\ell_{\text{cosmo}}^2$ for some cosmological scale
\end{itemize}

This remains speculative and requires further development.

\subsection{Dark Energy}

The membrane surface tension $\sigma \approx 1.41 \times 10^{18}$ J/m$^2$ could contribute to cosmic acceleration:
\begin{equation}
\rho_{\text{DE}} \sim \frac{\sigma r_e \approx 2.8 \times 10^{-15})}{(3 \times 10^8)^2 \cdot (10^{26})^2} \sim 10^{-36} \text{ kg/m}^3
\end{equation}

This is of the right order for dark energy ($\sim 10^{-27}$ kg/m$^3$), though the match is not exact.

\newpage

\section{Discussion and Open Questions}
\label{sec:grav_discussion}

\subsection{What We Have Achieved}

\begin{enumerate}
    \item \textbf{Identified the third parameter:} Plenum energy density $\rho_{\text{Plenum}}$
    
    \item \textbf{Explained the hierarchy:} Gravity is weak because $\ell_P \ll r_e$
    
    \item \textbf{Proposed expression for Newton's constant (consistency check):} $G = \ell_P^2 c^4 / (\sigma r_e^3)$  (agreement checked numerically; not a standalone derivation claim)
    
    \item \textbf{Physical mechanism:} Archimedean/Bjerknes attraction of pressure deficits
\end{enumerate}

\subsection{Remaining Challenges}

\begin{enumerate}
    \item \textbf{Rigorous derivation:} The Bjerknes analogy is intuitive but not mathematically complete. A full derivation from the 5D action is needed.
    
    \item \textbf{General Relativity:} We derived Newtonian gravity. The relativistic generalization (Einstein field equations, curved spacetime) remains to be shown.
    
    \item \textbf{Black holes:} How do black hole solutions emerge? Is there a maximum mass before the ``hole'' in the Plenum closes?
    
    \item \textbf{Gravitational waves:} Are they membrane ripples, Plenum waves, or both?
    
    \item \textbf{Cosmological constant:} The screening mechanism needs rigorous formulation.
    
    \item \textbf{Why is $\rho_{\text{Plenum}} = \rho_{\text{Planck}}$?} This identification is consistent but not derived. Is Planck density the maximum possible, or is it set by deeper physics?
    
    \item \textbf{Quantum gravity and singularities:} Does the finite Planck-scale granularity of the Plenum resolve classical singularities? The minimum scale $\ell_P$ suggests a natural UV cutoff that may regularize black hole and Big Bang singularities.
\end{enumerate}

\subsection{Unification Summary}

EDC now provides a complete framework for fundamental constants:

\begin{center}
\begin{tabular}{|c|c|c|}
\hline
\textbf{Constant} & \textbf{Formula} & \textbf{From Parameters} \\
\hline
$c$ & $v_{\text{scan}}$ & Postulate \\
$\hbar$ & $\sigma r_e^3 / c$ & $\sigma$, $r_e$ \\
$\alpha$ & $m_e c^2 / (\sigma r_e^2)$ & $\sigma$, $r_e$ \\
$G$ & $\ell_P^2 c^4 / (\sigma r_e^3)$ & $\sigma$, $r_e$, $\rho_{\text{Plenum}}$ \\
\hline
\end{tabular}
\end{center}

Three parameters ($\sigma$, $r_e$, $\rho_{\text{Plenum}}$) plus one postulate ($c$) generate all fundamental constants. Note that $\sigma$ now has units J/m$^2$ (true surface tension).

\textbf{With gravity now emergent, EDC achieves geometric unification of quantum mechanics, electromagnetism, and gravitation from a single physical picture: an elastic membrane in a dense, viscous 5D fluid.}


%═══════════════════════════════════════════════════════════════════════════════
\section{The Limit of Curvature: Why Singularities Cannot Exist}
\label{sec:singularities}

\subsection{The Classical Singularity Problem}

General Relativity predicts that sufficiently massive objects collapse to a \textbf{singularity}---a point of infinite density and curvature. This occurs at the center of black holes and at the Big Bang.

Mathematically, the Schwarzschild solution diverges at $r = 0$:
\begin{equation}
R_{\mu\nu\rho\sigma}R^{\mu\nu\rho\sigma} \to \infty \quad \text{as} \quad r \to 0
\end{equation}

This is widely regarded as a \textit{failure} of the theory---a sign that GR is incomplete.

\subsection{The EDC Resolution: Maximum Curvature}

In EDC, the membrane has a finite tensile strength:
\begin{equation}
\sigma = 1.41 \times 10^{18} \text{ J/m}^2
\end{equation}

As shown in Chapter 11, this is approximately equal to the \textbf{Schwinger limit}---the field strength at which the vacuum spontaneously produces particle pairs.

\textbf{Key insight:} The membrane \textit{cannot} be curved beyond the point where its tension equals the Schwinger limit. Beyond this, the membrane ``tears''---converting stored elastic energy into particle creation.

\subsection{Black Holes as Plenum Bubbles}

In EDC, a black hole is not a singularity but a \textbf{bubble of pure Plenum} bounded by membrane at its breaking point:

\begin{enumerate}
    \item As matter collapses, membrane curvature increases
    \item At $r \sim \ell_P$, curvature reaches the Schwinger limit
    \item The membrane ``phase transitions''---releasing energy back into the Plenum
    \item The result is a Planck-density core, \textit{not} a singularity
\end{enumerate}

\begin{tcolorbox}[colback=yellow!10,colframe=orange!80!black,title=The Elastic Horizon]
The event horizon is not merely a ``point of no return.'' It is a \textbf{zone of phase transition} where the membrane tension reaches its maximum value $\sigma_{max} \approx E_S$.

Inside this zone, spacetime as we know it ceases to exist---replaced by pure Plenum at Planck density.
\end{tcolorbox}

\subsection{Testable Prediction: Gravitational Wave Echoes}

If black holes have structure at the Planck scale (rather than smooth horizons), gravitational waves from mergers should exhibit \textbf{echoes}---delayed secondary signals reflected from the Planck core.

The echo delay time is approximately:
\begin{equation}
\Delta t \sim \frac{r_S}{c} \ln\left(\frac{r_S}{\ell_P}\right)
\end{equation}

For a 10 solar mass black hole:
\begin{equation}
\Delta t \sim 10 \text{ ms}
\end{equation}

This is within LIGO/Virgo sensitivity. Several groups have claimed tentative detections of such echoes, though the evidence remains controversial.


%═══════════════════════════════════════════════════════════════════════════════
\section{Galactic Rotation: A Hydrodynamic Approach}
\label{sec:dark_matter}

\subsection{The Dark Matter Problem}

Galaxies rotate too fast. Stars at large radii orbit at velocities inconsistent with visible matter:
\begin{equation}
v_{\text{observed}} \gg v_{\text{Keplerian}} = \sqrt{\frac{GM_{\text{visible}}}{r}}
\end{equation}

The standard solution posits invisible ``dark matter'' halos. Despite decades of searching, no dark matter particle has been detected.

\subsection{The EDC Alternative: Plenum Entrainment}

In EDC, space is not empty---it is filled with Plenum at Planck density. A rotating galaxy does not merely curve spacetime; it \textbf{drags the Plenum} into rotation.

Consider the analogy of a spinning disk in honey:
\begin{itemize}
    \item The disk (galaxy) rotates with angular velocity $\Omega$
    \item Viscous coupling transfers angular momentum to the fluid (Plenum)
    \item The fluid rotates faster than it would in the absence of viscosity
    \item Objects embedded in the fluid (stars) are carried along
\end{itemize}

\subsection{Qualitative Prediction}

If galaxies entrain the Plenum:
\begin{enumerate}
    \item Rotation curves should flatten at large $r$ (observed!)
    \item The effect should scale with galaxy mass (observed!)
    \item No particle detection is expected (consistent with null results!)
    \item The effect should be stronger in denser regions (testable)
\end{enumerate}

\begin{tcolorbox}[colback=blue!5,colframe=blue!50!black,title=Future Work: Dark Matter as Plenum Dynamics]
A full calculation requires solving the Navier-Stokes equations for a viscous fluid (Plenum) coupled to rotating sources (galaxies).

\textbf{Preliminary hypothesis:} ``Dark matter'' is not matter at all---it is the kinematic effect of Plenum entrainment by rotating galactic structures.

Quantitative development is reserved for Volume II.
\end{tcolorbox}


%═══════════════════════════════════════════════════════════════════════════════
\section{Two Types of Gravitational Waves}
\label{sec:gw_types}

LIGO has detected gravitational waves from merging black holes and neutron stars. In EDC, these waves have a natural interpretation---but with an additional prediction.

\subsection{Surface Waves (LIGO Detections)}

The gravitational waves detected by LIGO are \textbf{transverse} oscillations of the membrane---ripples on the surface of spacetime. These are analogous to water waves on a pond.

Properties:
\begin{itemize}
    \item Speed: $c$ (the membrane scanning velocity)
    \item Polarization: Two transverse modes ($+$ and $\times$)
    \item Source: Accelerating masses (membrane deformations)
\end{itemize}

These are the standard gravitational waves predicted by GR.

\subsection{Longitudinal Waves: Sound of the Plenum}

In addition to surface waves, EDC predicts \textbf{longitudinal} oscillations---compressions and rarefactions of the Plenum itself. These are analogous to sound waves in air.

Properties:
\begin{itemize}
    \item Speed: Potentially different from $c$ (depends on Plenum compressibility)
    \item Polarization: One longitudinal mode
    \item Source: Explosive events (supernovae, Big Bang)
\end{itemize}

\begin{tcolorbox}[colback=yellow!10,colframe=orange!80!black,title=Primordial Gravitational Waves]
The longitudinal Plenum waves from the Big Bang may be what cosmologists seek as the ``primordial gravitational wave background'' in the CMB B-mode polarization.

If these waves travel at a different speed than transverse waves, they would arrive at different times from the same cosmic event---a smoking gun for EDC.
\end{tcolorbox}

\vspace{2em}
\begin{center}
\rule{0.5\textwidth}{0.4pt}

\textit{``The universe is not heavy---it floats on the densest sea.''}
\end{center}
