% ==============================================================================
% Section 1: Scope and Empirical Baselines
% ==============================================================================

\begin{tcolorbox}[edcGuardrail, title={Scope Guardrail}]
This paper develops a \textbf{structural/ontological framework} for weak decays.
It does \textbf{not}:
\begin{itemize}[nosep]
  \item Derive particle masses from first principles
  \item Derive lifetimes or decay rates numerically
  \item Claim to replace Standard Model calculations
  \item Tune parameters to fit experimental data
\end{itemize}
All empirical values are treated as baselines \tagBL{}. The framework provides
mechanistic interpretation, not numerical prediction (yet).
\end{tcolorbox}

\subsection{Empirical Baselines}

The following experimental values define the phenomena we seek to interpret
mechanistically. All values are from PDG 2024~\cite{pdg:2024} unless otherwise noted.

\begin{table}[ht]
\centering
\caption{Empirical baselines for weak-sector phenomena \tagBL{}}
\label{tab:baselines}
\begin{tabular}{llll}
\toprule
\textbf{Quantity} & \textbf{Value} & \textbf{Source} & \textbf{Status} \\
\midrule
Neutron lifetime $\tau_n$ & $879.4 \pm 0.6$ s & PDG 2024 & \tagBL{} \\
Muon lifetime $\tau_\mu$ & $2.1969811 \times 10^{-6}$ s & PDG 2024 & \tagBL{} \\
Tau lifetime $\tau_\tau$ & $2.903 \times 10^{-13}$ s & PDG 2024 & \tagBL{} \\
Pion lifetime $\tau_{\pi^\pm}$ & $2.6033 \times 10^{-8}$ s & PDG 2024 & \tagBL{} \\
\midrule
Electron mass $m_e$ & $0.51100$ MeV & CODATA 2022 & \tagBL{} \\
Muon mass $m_\mu$ & $105.658$ MeV & PDG 2024 & \tagBL{} \\
Tau mass $m_\tau$ & $1776.86$ MeV & PDG 2024 & \tagBL{} \\
Pion mass $m_{\pi^\pm}$ & $139.570$ MeV & PDG 2024 & \tagBL{} \\
Neutron mass $m_n$ & $939.565$ MeV & CODATA 2022 & \tagBL{} \\
Proton mass $m_p$ & $938.272$ MeV & CODATA 2022 & \tagBL{} \\
\midrule
$n$--$p$ mass difference & $1.293$ MeV & CODATA 2022 & \tagBL{} \\
Fermi constant $G_F$ & $1.1663788 \times 10^{-5}$ GeV$^{-2}$ & PDG 2024 & \tagBL{} \\
\midrule
$\mathrm{BR}(\pi^+ \to \mu^+\nu_\mu)$ & $99.98770\%$ & PDG 2024 & \tagBL{} \\
$\mathrm{BR}(\pi^+ \to e^+\nu_e)$ & $1.230 \times 10^{-4}$ & PDG 2024 & \tagBL{} \\
\bottomrule
\end{tabular}
\end{table}

\subsection{What This Framework Provides}

The EDC Weak Sector framework offers:

\begin{enumerate}[nosep]
  \item \textbf{Mechanistic interpretation}: Why weak decays occur as bulk$\to$brane
        energy transfer, not ``particle transmutation''
  \item \textbf{Ontological classification}: What each particle ``is'' in 5D terms
        (bulk-core vs.\ brane-dominant vs.\ edge mode)
  \item \textbf{Unified pipeline}: A single Absorption$\to$Dissipation$\to$Release
        sequence applicable to all weak processes
  \item \textbf{Structural constraints}: Which outputs are kinematically and
        topologically allowed
  \item \textbf{Falsifiability handles}: Explicit conditions under which the
        framework would fail
\end{enumerate}

\subsection{What Remains Open}

The following are explicitly flagged as open problems requiring future work:

\begin{itemize}[nosep]
  \item Numerical derivation of $\tau_n$, $\tau_\mu$, $\tau_\tau$ from 5D parameters
  \item First-principles calculation of lepton mass ratios
  \item Derivation of $G_F$ magnitude (structural pathway exists; numerical closure pending)
  \item Helicity suppression factor from explicit boundary-condition computation
  \item Quark confinement mechanism in 5D (for pion micro-ontology)
\end{itemize}

These open problems do not invalidate the structural framework; they define the
research program's completion targets.
