% ==============================================================================
% Section 3: Unified Decay Pipeline
% ==============================================================================

All weak decays in the EDC framework follow a universal three-stage pipeline.
This section defines each stage and provides the canonical diagram.

\subsection{The Three Stages}

\begin{definition}[Absorption--Dissipation--Release Pipeline \tagDef{}]
\label{def:pipeline}
A weak decay proceeds through three sequential stages:

\textbf{Stage 1: Absorption.}
Energy from the decaying configuration enters the brane layer. For bulk-core
particles (neutron), this occurs when the junction configuration relaxes; for
brane-dominant particles (muon, tau), the initial state already resides in the
brane layer.

\textbf{Stage 2: Dissipation.}
Energy redistributes within the brane layer according to available modes and
conservation laws. The \emph{ledger} (energy/momentum/charge bookkeeping) must
close: total energy in = total energy out.

\textbf{Stage 3: Release.}
The frozen projection operator $\mathcal{P}_{\mathrm{frozen}}$ determines which
output configurations appear to an observer. Only modes satisfying kinematic,
mode-matching, and chirality constraints pass through.
\end{definition}

\subsection{Pipeline Diagram}

Figure~\ref{fig:pipeline} illustrates the unified pipeline with projection
operator hooks.

\begin{figure}[ht]
\centering
% ==============================================================================
% Figure: Unified Weak Decay Pipeline
% Absorption → Dissipation → Release with projection operator
% ==============================================================================

\begin{tikzpicture}[
  scale=0.9,
  stage/.style={
    rectangle, rounded corners=8pt, minimum width=3.0cm, minimum height=1.2cm,
    draw=black, thick, font=\small\bfseries
  },
  operator/.style={
    rectangle, rounded corners=4pt, minimum width=2.0cm, minimum height=0.7cm,
    draw=blue!70!black, thick, fill=blue!10, font=\footnotesize
  },
  arrow/.style={-{Stealth[length=8pt]}, thick, black},
  label/.style={font=\footnotesize\itshape, text=gray!70!black}
]

% === Stage boxes ===
\node[stage, fill=red!15] (abs) at (0,0) {Absorption};
\node[stage, fill=yellow!20] (dis) at (4.5,0) {Dissipation};
\node[stage, fill=green!15] (rel) at (9,0) {Release};

% === Main flow arrows ===
\draw[arrow] (abs) -- (dis);
\draw[arrow] (dis) -- (rel);

% === Projection operator box ===
\node[operator] (pfrozen) at (6.75, -2.0) {$\mathcal{P}_{\mathrm{frozen}}$};

% === Sub-operators ===
\node[operator, minimum width=1.4cm, font=\tiny] (penergy) at (5.0, -3.3) {$\mathcal{P}_{\mathrm{energy}}$};
\node[operator, minimum width=1.4cm, font=\tiny] (pmode) at (6.75, -3.3) {$\mathcal{P}_{\mathrm{mode}}$};
\node[operator, minimum width=1.4cm, font=\tiny] (pchir) at (8.5, -3.3) {$\mathcal{P}_{\mathrm{chir}}$};

% === Operator decomposition ===
\draw[-{Stealth[length=4pt]}, thick, blue!70!black] (pfrozen) -- (penergy);
\draw[-{Stealth[length=4pt]}, thick, blue!70!black] (pfrozen) -- (pmode);
\draw[-{Stealth[length=4pt]}, thick, blue!70!black] (pfrozen) -- (pchir);

% === Hook from Release to P_frozen ===
\draw[-{Stealth[length=5pt]}, thick, dashed, blue!70!black]
  (rel.south) -- ++(0,-0.4) -| (pfrozen.north);

% === Input/output labels ===
\node[label, anchor=east, text width=2.2cm, align=right] at (-1.0, 0)
  {Bulk trigger\\(decay event)};
\node[label, anchor=west, text width=2.2cm, align=left] at (10.2, 0)
  {3D outputs\\$(e^-, \bar\nu, p, \ldots)$};

% === Stage descriptions ===
\node[label, anchor=north, text width=2.8cm, align=center] at (0, -0.9)
  {Energy enters\\brane layer};
\node[label, anchor=north, text width=2.8cm, align=center] at (4.5, -0.9)
  {Redistribution\\(ledger closure)};
\node[label, anchor=north, text width=2.8cm, align=center] at (9, -0.9)
  {Observer-facing\\projection};

% === Operator labels ===
\node[label, anchor=north, text width=1.3cm, align=center] at (5.0, -3.9) {Kinematic};
\node[label, anchor=north, text width=1.3cm, align=center] at (6.75, -3.9) {Mode\\overlap};
\node[label, anchor=north, text width=1.3cm, align=center] at (8.5, -3.9) {Chirality\\(V$-$A)};

% === Layer indicators ===
\fill[gray!10] (-1.5, 1.2) rectangle (10.8, 1.7);
\node[font=\scriptsize, gray] at (4.65, 1.45) {Bulk (5D)};

\fill[blue!8] (-1.5, 0.7) rectangle (10.8, 1.2);
\draw[thick, blue!40] (-1.5, 0.7) -- (10.8, 0.7);
\draw[thick, blue!40] (-1.5, 1.2) -- (10.8, 1.2);
\node[font=\scriptsize, blue!60!black] at (4.65, 0.95) {Brane Layer};

\fill[green!8] (-1.5, 0.3) rectangle (10.8, 0.7);
\node[font=\scriptsize, green!50!black] at (4.65, 0.5) {Observer (3D)};

\end{tikzpicture}

\caption{Unified weak decay pipeline: Absorption $\to$ Dissipation $\to$
Release. The frozen projection operator $\mathcal{P}_{\mathrm{frozen}}$
filters outputs via energy, mode-matching, and chirality constraints.}
\label{fig:pipeline}
\end{figure}

\subsection{Energy Ledger}

\begin{postulate}[Ledger Closure \tagP{}]
\label{post:ledger}
For any weak decay, the energy ledger must close:
\begin{equation}
  E_{\mathrm{initial}} = E_{\mathrm{outputs}} + E_{\mathrm{bulk\,leakage}}
  \label{eq:ledger}
\end{equation}
where $E_{\mathrm{bulk\,leakage}}$ is the (suppressed) energy that escapes
into bulk modes rather than appearing as 3D outputs.
\end{postulate}

For most decays, bulk leakage is exponentially suppressed and can be neglected
at leading order. The ledger then reduces to:
\begin{equation}
  E_{\mathrm{initial}} \approx \sum_i E_i^{\mathrm{(output)}}
\end{equation}

\subsection{Why ``Decay'' is Redistribution}

In the Standard Model language, $n \to p + e^- + \bar{\nu}_e$ suggests the
neutron ``becomes'' a proton plus leptons. In EDC, the mechanistic picture is:

\begin{enumerate}[nosep]
  \item The neutron is a bulk-core junction configuration with stored binding energy
  \item Junction relaxation releases energy into the brane layer (Absorption)
  \item Energy redistributes into available brane modes (Dissipation)
  \item The proton, electron, and antineutrino are the projected outputs (Release)
\end{enumerate}

No particle is ``created from nothing''; energy moves between configurations.
The proton was already implicit in the neutron's junction structure; the leptons
are brane modes excited by the released energy.

\subsection{Universality of the Pipeline}

The same pipeline applies to all weak decays studied in this paper:

\begin{table}[ht]
\centering
\caption{Pipeline instantiation for different decays}
\label{tab:pipeline_instances}
\begin{tabular}{llll}
\toprule
\textbf{Decay} & \textbf{Absorption} & \textbf{Dissipation} & \textbf{Release} \\
\midrule
$n \to p e^- \bar{\nu}$ & Junction relaxation & Brane redistribution & $\mathcal{P}_{\mathrm{frozen}}$ \\
$\mu^- \to e^- \nu_\mu \bar{\nu}_e$ & Mode de-excitation & Brane redistribution & $\mathcal{P}_{\mathrm{frozen}}$ \\
$\tau^- \to \mu^- \nu_\tau \bar{\nu}_\mu$ & Higher-mode decay & Brane redistribution & $\mathcal{P}_{\mathrm{frozen}}$ \\
$\pi^+ \to \mu^+ \nu_\mu$ & Composite dissociation & Brane redistribution & $\mathcal{P}_{\mathrm{frozen}}$ \\
\bottomrule
\end{tabular}
\end{table}

The \emph{trigger} differs (junction relaxation vs.\ mode de-excitation vs.\
composite dissociation), but the subsequent Dissipation and Release stages
follow identical logic.
