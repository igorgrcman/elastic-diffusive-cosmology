% ==============================================================================
% Section 2: The Mechanistic Dimension Principle
% ==============================================================================

Before discussing specific particles, we establish the conceptual foundation
that distinguishes EDC from conventional extra-dimensional approaches.

\subsection{Extra Dimensions as Mechanism}

\begin{definition}[Mechanistic Dimension Principle \tagDef{}]
\label{def:mechanistic}
In EDC, the fifth dimension $y$ is not a spatial direction one ``travels through''
but a \textbf{mechanistic degree of freedom} that:
\begin{enumerate}[nosep]
  \item Encodes \textbf{binding state}: how strongly a configuration is
        localized to the brane
  \item Determines \textbf{interaction channel availability}: which 3D
        final states are accessible
  \item Governs \textbf{stability}: metastable configurations occupy potential
        wells in $y$; unstable configurations sit at saddle points with decay channels
\end{enumerate}
\end{definition}

\textbf{Physical interpretation.}
The fifth dimension is not ``where particles hide'' but ``how tightly they are
bound to 3D observability.'' A bulk-extended configuration has weaker brane
coupling; a brane-localized configuration has stronger observer-facing presence.
Decay is not ``escape into extra dimensions'' but \emph{redistribution of
binding energy} followed by \emph{projection onto allowed 3D outputs}.

\subsection{The Brane as Interface}

\begin{postulate}[Brane as Mechanistic Interface \tagP{}]
\label{post:brane}
The 3D observable universe is a thick brane (finite width in $y$) embedded in
a 5D bulk. The brane acts as:
\begin{itemize}[nosep]
  \item \textbf{Energy reservoir}: Stores localized field configurations
  \item \textbf{Interaction locus}: Where bulk dynamics become observer-facing
  \item \textbf{Projection surface}: Filters which modes can propagate as 3D particles
\end{itemize}
\end{postulate}

\subsection{Language Precision}

We adopt precise vocabulary to avoid misleading implications:

\begin{table}[ht]
\centering
\caption{Canonical vocabulary for EDC weak-sector descriptions}
\label{tab:vocabulary}
\begin{tabular}{ll}
\toprule
\textbf{Use} & \textbf{Avoid} \\
\midrule
Energy redistribution & Particle creation/annihilation \\
Mode projection & Particle transmutation \\
Bulk$\to$brane transfer & Escape into extra dimensions \\
Suppressed (quantifiable) & Forbidden/impossible \\
Brane-localized & Trapped on brane \\
Ledger-closed & Energy conserved (ambiguous scope) \\
\bottomrule
\end{tabular}
\end{table}

\textbf{Key distinction}: We say ``suppressed,'' not ``cannot escape.''
Suppression is quantifiable (exponential in gap/energy ratio); absolute
prohibition would require infinite barrier height. The neutrino's weak
interaction arises from \emph{suppressed leakage} at the bulk--brane interface,
not from ``escape into extra dimensions.''

\subsection{Ontological Categories}

Particles in the EDC framework fall into four ontological categories based on
their 5D profiles:

\begin{definition}[Particle Ontologies \tagDef{}]
\label{def:ontology}
\begin{enumerate}[nosep]
  \item \textbf{Bulk-core junction} (neutron, proton): Extended configuration
        with significant bulk component; junction geometry determines stability
  \item \textbf{Brane-dominant excitation} (electron, muon, tau): Localized
        primarily at $y \approx 0$; different modes indexed by $n_\ell$
  \item \textbf{Edge mode} (neutrino): Interfacial excitation at bulk--brane
        boundary; propagates along interface, not into bulk or deep brane
  \item \textbf{Composite} (pion): Bound state of more fundamental configurations;
        brane-dominant but with internal structure
\end{enumerate}
\end{definition}

These categories are \emph{postulates} \tagP{}, not derivations. Their
justification comes from consistency: if the assigned ontology correctly
predicts decay channels, stability, and selection rules, it is validated;
if not, it fails.
