% ==============================================================================
% Section 8: Open Problems
% ==============================================================================

This section consolidates all open problems identified throughout the paper.
Each problem is actionable: it has a well-defined target and success criterion.

\subsection{Numerical Closures}

\begin{enumerate}
  \item \textbf{Neutron lifetime} \tagOpen{}\\
        \textit{Target}: Derive $\tau_n = 879.4$ s from junction barrier height.\\
        \textit{Approach}: Compute WKB tunneling rate through 5D potential barrier.\\
        \textit{Success criterion}: Agreement within experimental uncertainty.

  \item \textbf{Muon lifetime} \tagOpen{}\\
        \textit{Target}: Derive $\tau_\mu = 2.197 \times 10^{-6}$ s from mode spacing.\\
        \textit{Approach}: Compute mode transition rate in thick-brane spectrum.\\
        \textit{Success criterion}: Agreement with SM calculation.

  \item \textbf{Tau lifetime} \tagOpen{}\\
        \textit{Target}: Derive $\tau_\tau = 2.903 \times 10^{-13}$ s.\\
        \textit{Approach}: Same as muon, with higher mode index.\\
        \textit{Success criterion}: Agreement with SM calculation.

  \item \textbf{Fermi constant magnitude} \tagOpen{}\\
        \textit{Target}: Derive $G_F = 1.166 \times 10^{-5}$ GeV$^{-2}$.\\
        \textit{Approach}: Compute overlap integrals and mediator mass from 5D dynamics.\\
        \textit{Success criterion}: Correct order of magnitude without fitting.
\end{enumerate}

\subsection{Mass Spectrum}

\begin{enumerate}[resume]
  \item \textbf{Electron mass} \tagOpen{}\\
        \textit{Target}: Derive $m_e = 0.511$ MeV from brane ground-mode energy.\\
        \textit{Approach}: Solve 5D bound-state equation with brane potential.\\
        \textit{Success criterion}: Correct value from first principles.

  \item \textbf{Lepton mass ratios} \tagOpen{}\\
        \textit{Target}: Derive $m_\mu/m_e \approx 207$, $m_\tau/m_\mu \approx 17$.\\
        \textit{Approach}: Compute mode spectrum eigenvalues.\\
        \textit{Success criterion}: Ratios match without fitting.

  \item \textbf{Pion mass} \tagOpen{}\\
        \textit{Target}: Derive $m_\pi = 139.6$ MeV from composite binding.\\
        \textit{Approach}: Solve bound-state problem for junction-pair candidate.\\
        \textit{Success criterion}: Correct value from 5D binding energy.

  \item \textbf{Neutrino masses} \tagOpen{}\\
        \textit{Target}: Derive $m_\nu \lesssim 1$ eV from edge-mode spectrum.\\
        \textit{Approach}: Compute interface mode energies.\\
        \textit{Success criterion}: Consistent with oscillation data.
\end{enumerate}

\subsection{Structural Completions}

\begin{enumerate}[resume]
  \item \textbf{Helicity suppression factor} \tagOpen{}\\
        \textit{Target}: Derive $(m_e/m_\mu)^2$ from explicit $\mathcal{P}_{\mathrm{chir}}$ computation.\\
        \textit{Approach}: Evaluate chirality projection for different mass leptons.\\
        \textit{Success criterion}: Recover SM helicity suppression formula.

  \item \textbf{V$-$A structure} \tagOpen{}\\
        \textit{Target}: Derive V$-$A from interface geometry.\\
        \textit{Approach}: Analyze spinor boundary conditions at brane interface.\\
        \textit{Success criterion}: Only left-handed coupling emerges naturally.

  \item \textbf{Flavor mixing} \tagOpen{}\\
        \textit{Target}: Explain CKM/PMNS matrices geometrically.\\
        \textit{Approach}: Mode overlap between different fermion generations.\\
        \textit{Success criterion}: Reproduce observed mixing angles.
\end{enumerate}

\subsection{Extended Scope}

\begin{enumerate}[resume]
  \item \textbf{Quark confinement} \tagOpen{}\\
        \textit{Target}: Explain why quarks don't appear as free particles.\\
        \textit{Approach}: 5D flux tube or junction topology.\\
        \textit{Success criterion}: Confinement as topological necessity.

  \item \textbf{CP violation} \tagOpen{}\\
        \textit{Target}: Explain matter-antimatter asymmetry sources.\\
        \textit{Approach}: Complex phases in 5D couplings or boundary conditions.\\
        \textit{Success criterion}: Reproduce observed CP violation.

  \item \textbf{Photon ontology} \tagOpen{}\\
        \textit{Target}: Classify photon in 5D ontology (for $\pi^0 \to \gamma\gamma$).\\
        \textit{Approach}: Identify photon as gauge mode or brane fluctuation.\\
        \textit{Success criterion}: Consistent with QED.
\end{enumerate}

\subsection{Priority Ranking}

For practical progress, we recommend the following priority order:

\begin{table}[ht]
\centering
\caption{Open problem priority ranking}
\label{tab:priority}
\begin{tabular}{lll}
\toprule
\textbf{Priority} & \textbf{Problem} & \textbf{Rationale} \\
\midrule
1 (Highest) & $G_F$ magnitude & Central to weak interaction strength \\
2 & Neutron lifetime & Cleanest single-parameter test \\
3 & Helicity suppression & Direct test of $\mathcal{P}_{\mathrm{chir}}$ \\
4 & Electron mass & Foundation for lepton spectrum \\
5 & Muon/tau lifetimes & Validates mode-decay picture \\
\bottomrule
\end{tabular}
\end{table}

\subsection{Conclusion}

The EDC Weak Sector framework provides a coherent mechanistic interpretation
of weak-interaction phenomenology. The unified Absorption$\to$Dissipation$\to$Release
pipeline, the ontological classification of particles, and the structural pathway
to $G_F$ form a self-consistent picture.

What remains is numerical closure: computing the quantities marked \tagOpen{}
from first principles. These are well-posed mathematical problems, not conceptual
gaps. The framework's value lies in providing the \emph{structure} within which
such calculations can be performed.

The explicit falsifiability criteria (\S\ref{sec:falsify}) ensure that the
framework makes testable predictions. If these predictions fail, the framework
must be revised. This is the hallmark of a scientific theory: it can be wrong.
