% ==============================================================================
% Section 6: Structural Pathway to G_F
% ==============================================================================

This section presents the structural pathway from 5D mediator exchange to the
effective four-Fermi coupling. We derive the \emph{form} $G_{\mathrm{EDC}} \sim
g_{\mathrm{eff}}^2/m_\phi^2$ from geometry; numerical evaluation remains open.

\subsection{The Effective Lagrangian}

In the thick-brane framework, a 5D scalar mediator $\phi(x^\mu, y)$ with bulk
mass $m_\phi$ propagates between brane-localized fermions. Integrating out the
mediator yields an effective 4D interaction.

\begin{definition}[Effective Four-Fermi Structure \tagDef{}]
\label{def:leff}
At energies $E \ll m_\phi$, the effective interaction takes the form:
\begin{equation}
  \mathcal{L}_{\mathrm{eff}} \sim \frac{g_{\mathrm{eff}}^2}{m_\phi^2}
  \left( \bar{\psi}_1 \Gamma \psi_2 \right)
  \left( \bar{\psi}_3 \Gamma \psi_4 \right)
  \label{eq:leff}
\end{equation}
where $\Gamma$ encodes the Lorentz structure (V$-$A for weak interactions)
and $g_{\mathrm{eff}}$ is an effective coupling absorbing overlap integrals.
\end{definition}

\textbf{Physical interpretation.}
Equation~\eqref{eq:leff} is not a fundamental ``weak vertex''; it is the
low-energy residue of a 5D bulk$\to$brane transfer process. The four-Fermi
form emerges because the mediator is too heavy to propagate as a real particle
at weak-decay energies. What appears as a ``contact interaction'' is actually
mediated exchange with the propagator contracted to a point.

\subsection{The EDC Structural Analog}

\begin{definition}[EDC Structural Coupling \tagDef{}]
\label{def:gedc}
We define the EDC structural analog of the Fermi constant:
\begin{equation}
  \boxed{
    G_{\mathrm{EDC}} \sim \frac{g_{\mathrm{eff}}^2}{m_\phi^2}
  }
  \label{eq:gedc}
\end{equation}
where:
\begin{itemize}[nosep]
  \item $g_{\mathrm{eff}}$ = effective coupling (absorbs overlap and boundary factors)
  \item $m_\phi$ = mediator mass scale (determines interaction range)
\end{itemize}
\end{definition}

\subsection{Decomposition of $g_{\mathrm{eff}}$}

The effective coupling receives contributions from three sources:

\begin{equation}
  g_{\mathrm{eff}} = g_5 \times \mathcal{O}_{\mathrm{overlap}} \times \mathcal{O}_{\mathrm{BC}}
  \label{eq:geff}
\end{equation}

\begin{enumerate}
  \item \textbf{Bulk coupling $g_5$} \tagP{}: The fundamental 5D interaction strength,
        a free parameter of the theory.

  \item \textbf{Overlap factor $\mathcal{O}_{\mathrm{overlap}}$} \tagDc{}:
        \begin{equation}
          \mathcal{O}_{\mathrm{overlap}} = \int_0^{L_y} dy \,
          \psi_1^*(y) \psi_2(y) \phi(y)
        \end{equation}
        Measures wavefunction overlap in the fifth dimension. Suppressed when
        initial and final states have different $y$-profiles.

  \item \textbf{Boundary factor $\mathcal{O}_{\mathrm{BC}}$} \tagP{}/\tagDc{}:
        Encodes boundary conditions at $y = 0$ (brane) and $y = L_y$ (bulk cutoff).
        Dirichlet, Neumann, or mixed conditions affect the mode spectrum and couplings.
\end{enumerate}

\subsection{Why This Is Not a Fit}

\begin{tcolorbox}[edcGuardrail, title={No-Fit Declaration}]
We do \textbf{not} claim:
\begin{itemize}[nosep]
  \item That $G_{\mathrm{EDC}} = G_F$ (numerical equality)
  \item That we have derived $g_5$ or $m_\phi$ from first principles
  \item That the overlap integrals have been computed
\end{itemize}
We \textbf{do} claim:
\begin{itemize}[nosep]
  \item The \emph{structural form} $G \sim g^2/m^2$ emerges from mediator exchange
  \item Geometric suppression (overlap factors) naturally appears
  \item The pathway is complete in principle; numerical closure is pending
\end{itemize}
\end{tcolorbox}

\subsection{Comparison with Standard Model}

The Standard Model derives $G_F$ from $W$-boson exchange:
\begin{equation}
  G_F = \frac{g_W^2}{4\sqrt{2} m_W^2} \quad \tagBL{}
\end{equation}

The EDC structural analog has the same form but different interpretation:
\begin{itemize}[nosep]
  \item SM: $W$ boson is a fundamental gauge boson; $g_W$ is the weak coupling
  \item EDC: $\phi$ is a bulk mediator; $g_{\mathrm{eff}}$ includes geometric factors
\end{itemize}

This is a \textbf{structural comparison}, not an equivalence claim. Both give
$G \sim g^2/m^2$; the underlying physics differs.

\subsection{Closure Targets}

To complete the $G_F$ derivation, the following must be computed \tagOpen{}:

\begin{enumerate}[nosep]
  \item \textbf{Mediator mass $m_\phi$}: From 5D bulk dynamics or Kaluza-Klein spectrum
  \item \textbf{Bulk coupling $g_5$}: From fundamental 5D action (or identified with known scale)
  \item \textbf{Overlap integrals}: Numerical integration over fermion/mediator profiles
  \item \textbf{Boundary conditions}: Determine which BC choice matches observed physics
\end{enumerate}

These are well-posed mathematical problems, not conceptual gaps. The structural
pathway is established; execution is pending.
