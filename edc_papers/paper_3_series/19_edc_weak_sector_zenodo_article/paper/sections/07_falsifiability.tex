% ==============================================================================
% Section 7: Falsifiability and Epistemic Boundaries
% ==============================================================================

A scientific framework must specify conditions under which it would be falsified.
This section provides explicit falsifiability handles and epistemic boundaries.

\subsection{Falsifiability Criteria}

The EDC Weak Sector framework makes structural predictions that can be tested:

\begin{enumerate}
  \item \textbf{Ontology test}: If a particle's assigned ontology (bulk-core vs.\
        brane-dominant vs.\ edge mode) fails to predict its decay channels correctly,
        the ontology assignment is wrong.

  \item \textbf{Pipeline test}: If any weak decay requires a mechanism fundamentally
        different from Absorption$\to$Dissipation$\to$Release, the unified pipeline fails.

  \item \textbf{Selection rule test}: If a decay channel forbidden by
        $\mathcal{P}_{\mathrm{frozen}}$ is observed, the projection operator
        formalism is incorrect.

  \item \textbf{Chirality test}: If the chirality structure (V$-$A) cannot be
        explained as interface geometry, the $\mathcal{P}_{\mathrm{chir}}$
        mechanism is wrong.

  \item \textbf{Ledger test}: If energy/momentum/charge bookkeeping fails to
        close for any decay, ledger closure is violated.

  \item \textbf{Universality test}: If different particles require fundamentally
        different projection operators (not just different parameter values),
        the universality of $\mathcal{P}_{\mathrm{frozen}}$ fails.
\end{enumerate}

\begin{tcolorbox}[edcGuardrail, title={No Immunity to Falsification}]
This framework does \textbf{not} claim immunity from experimental refutation.
If future observations contradict structural predictions, the framework must
be revised or abandoned. The epistemic tags (\tagP{}, \tagOpen{}) explicitly
mark which elements are hypotheses subject to testing.
\end{tcolorbox}

\subsection{What Would Constitute Failure}

Specific failure modes:

\begin{itemize}
  \item \textbf{Neutron}: If $n \to p + \mu^- + \bar{\nu}_\mu$ is observed
        (kinematically forbidden), energy accounting is wrong.

  \item \textbf{Muon}: If $\mu \to e + \gamma$ (lepton flavor violation) is
        observed at rates inconsistent with SM radiative corrections,
        the brane-mode isolation assumption fails.

  \item \textbf{Pion}: If helicity suppression factor differs significantly
        from $(m_e/m_\mu)^2$, the $\mathcal{P}_{\mathrm{chir}}$ mechanism
        is incorrectly formulated.

  \item \textbf{Neutrino}: If neutrinos are found to have significant coupling
        to brane-bulk modes (not just interface), the edge-mode ontology fails.
\end{itemize}

\subsection{Epistemic Status Summary}

\begin{table}[ht]
\centering
\caption{Epistemic status of framework elements}
\label{tab:epistemic}
\begin{tabular}{lll}
\toprule
\textbf{Element} & \textbf{Status} & \textbf{Falsifiable by} \\
\midrule
Empirical baselines & \tagBL{} & N/A (input data) \\
Mechanistic dimension principle & \tagDef{} & Alternative interpretation \\
Particle ontologies & \tagP{} & Wrong decay channel predictions \\
Pipeline structure & \tagDef{}/\tagP{} & Anomalous decay mechanisms \\
Projection operators & \tagDef{} & Selection rule violations \\
$G_F$ structural form & \tagDef{}/\tagDc{} & Alternative derivation \\
Numerical values & \tagOpen{} & Computation (pending) \\
\bottomrule
\end{tabular}
\end{table}

\subsection{Relation to Standard Model}

The EDC framework does not ``replace'' the Standard Model. Instead:

\begin{itemize}[nosep]
  \item SM provides the \textbf{successful phenomenology} (decay rates, branching
        ratios, cross sections) that any framework must reproduce.
  \item EDC provides a \textbf{mechanistic interpretation} of why the SM
        phenomenology takes its observed form.
  \item Where EDC makes quantitative predictions, they must match SM/experiment;
        discrepancies would falsify EDC.
  \item Where EDC offers only structural insight, it complements rather than
        contradicts SM.
\end{itemize}

\subsection{Honest Uncertainty}

We explicitly acknowledge:

\begin{enumerate}[nosep]
  \item The ontological assignments (\S\ref{sec:mechanism}) are postulates, not derivations.
  \item The projection operator decomposition (\S\ref{sec:operators}) is a definition
        whose physical realization requires further work.
  \item The $G_F$ pathway (\S\ref{sec:gf}) is structurally complete but numerically open.
  \item Lepton mass hierarchy remains unexplained at the numerical level.
  \item Quark confinement in 5D is not addressed.
\end{enumerate}

These are not weaknesses to hide but research directions to pursue. A framework
that claims no open problems is either complete (rare) or dishonest (common).
