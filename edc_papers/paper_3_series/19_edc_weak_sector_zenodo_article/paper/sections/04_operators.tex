% ==============================================================================
% Section 4: Projection Operators
% ==============================================================================

The Release stage of the pipeline is governed by the frozen projection operator,
which determines what an observer can detect. This section defines the operator
and its components.

\subsection{The Frozen Projection Operator}

\begin{definition}[Frozen Projection Operator \tagDef{}]
\label{def:pfrozen}
The frozen projection operator decomposes as:
\begin{equation}
  \mathcal{P}_{\mathrm{frozen}} = \mathcal{P}_{\mathrm{energy}} \circ
  \mathcal{P}_{\mathrm{mode}} \circ \mathcal{P}_{\mathrm{chir}}
  \label{eq:pfrozen}
\end{equation}
Each component enforces a distinct selection criterion:
\begin{itemize}[nosep]
  \item $\mathcal{P}_{\mathrm{energy}}$: Kinematic accessibility (energy/momentum conservation)
  \item $\mathcal{P}_{\mathrm{mode}}$: Mode-matching (overlap between initial and final configurations)
  \item $\mathcal{P}_{\mathrm{chir}}$: Chirality filter (V$-$A structure for weak outputs)
\end{itemize}
\end{definition}

\textbf{Physical interpretation.}
The name ``frozen'' indicates that this operator acts at the brane interface
where bulk dynamics ``freeze out'' into observable 3D physics. It is not a
dynamical operator but a \emph{projection onto allowed final states}.

\subsection{Energy Projection}

\begin{definition}[$\mathcal{P}_{\mathrm{energy}}$ \tagDef{}]
\label{def:penergy}
The energy projection operator selects output channels that are kinematically
accessible:
\begin{equation}
  \mathcal{P}_{\mathrm{energy}}: \quad
  \text{channel } X \text{ allowed} \iff E_{\mathrm{available}} \geq \sum_{i \in X} m_i
\end{equation}
\end{definition}

\textbf{Example (neutron decay).}
Available energy: $\Delta m_{np} = 1.293$ MeV. Possible channels:
\begin{itemize}[nosep]
  \item $p + e^- + \bar{\nu}_e$: requires $m_e \approx 0.511$ MeV $\checkmark$ (allowed)
  \item $p + \mu^- + \bar{\nu}_\mu$: requires $m_\mu \approx 105.7$ MeV $\times$ (forbidden)
\end{itemize}

\textbf{Example (pion decay).}
Available energy: $m_\pi \approx 140$ MeV. Both $\mu$ and $e$ channels are
kinematically allowed, but their rates differ due to $\mathcal{P}_{\mathrm{chir}}$.

\subsection{Mode Projection}

\begin{definition}[$\mathcal{P}_{\mathrm{mode}}$ \tagDef{}]
\label{def:pmode}
The mode projection operator selects output channels based on wavefunction overlap:
\begin{equation}
  \mathcal{P}_{\mathrm{mode}}: \quad
  \Gamma_X \propto \left| \langle \psi_{\mathrm{final}}^X | \psi_{\mathrm{initial}} \rangle \right|^2
\end{equation}
Channels with poor overlap are suppressed even if kinematically allowed.
\end{definition}

\textbf{Physical interpretation.}
A bulk-core configuration (neutron) has good overlap with other bulk-core
configurations (proton) and edge modes (neutrino), but poor overlap with
purely brane-localized high-mode excitations.

\subsection{Chirality Projection}

\begin{definition}[$\mathcal{P}_{\mathrm{chir}}$ \tagDef{}]
\label{def:pchir}
The chirality projection operator enforces the V$-$A structure of weak interactions:
\begin{equation}
  \mathcal{P}_{\mathrm{chir}}: \quad
  \text{only left-handed fermions and right-handed antifermions coupled}
\end{equation}
\end{definition}

\textbf{Physical interpretation.}
In EDC, the V$-$A structure emerges from the geometry of the brane interface
\tagP{}. The interface has an ``inside'' (bulk) and ``outside'' (observer),
creating a natural handedness. Modes propagating along the interface inherit
this chirality.

\subsection{Helicity Suppression from Chirality}

For pseudoscalar decays like $\pi^+ \to \ell^+ \nu_\ell$, the chirality
projection leads to helicity suppression:

\begin{equation}
  \frac{\Gamma(\pi \to e\nu)}{\Gamma(\pi \to \mu\nu)} \approx
  \left( \frac{m_e}{m_\mu} \right)^2 \approx 2.3 \times 10^{-5}
  \label{eq:helicity}
\end{equation}

This ratio is a \tagBL{} from SM/PDG. In EDC, it is interpreted as:
the chirality projection $\mathcal{P}_{\mathrm{chir}}$ penalizes configurations
where the lepton's spin must flip to match the pion's zero spin \tagP{}/\tagOpen{}.
Heavier leptons (larger $m_\ell$) can more easily accommodate the required
helicity mismatch.

\subsection{Operator Composition}

The full frozen projection operator acts sequentially:

\begin{enumerate}[nosep]
  \item $\mathcal{P}_{\mathrm{energy}}$ eliminates kinematically forbidden channels
  \item $\mathcal{P}_{\mathrm{mode}}$ weights surviving channels by overlap
  \item $\mathcal{P}_{\mathrm{chir}}$ applies chirality selection (V$-$A)
\end{enumerate}

The final decay rate for channel $X$ is:
\begin{equation}
  \Gamma_X \propto \mathcal{P}_{\mathrm{frozen}}[\text{channel } X]
  = \mathcal{P}_{\mathrm{chir}} \circ \mathcal{P}_{\mathrm{mode}} \circ
  \mathcal{P}_{\mathrm{energy}}[\text{channel } X]
\end{equation}

Numerical computation of these rates remains \tagOpen{}.
