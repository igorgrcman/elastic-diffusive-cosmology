% ==============================================================================
% Section 5: Case Studies
% ==============================================================================

We now apply the unified pipeline to specific particles. Each subsection
identifies the ontology, describes the decay mechanism, and notes open problems.

% ------------------------------------------------------------------------------
\subsection{Neutron: Bulk-Core Junction}
\label{subsec:neutron}
% ------------------------------------------------------------------------------

\begin{postulate}[Neutron Ontology \tagP{}]
\label{post:neutron}
The neutron is a \textbf{bulk-core junction}: a 5D configuration with significant
extension into the bulk ($y > 0$), connected to the brane via a junction
structure. The proton is a stable junction configuration (Steiner $120^\circ$
geometry \tagP{}); the neutron is metastable with a decay channel.
\end{postulate}

\textbf{Decay mechanism.}
\begin{enumerate}[nosep]
  \item \textbf{Trigger}: Junction configuration becomes unstable (saddle point in potential)
  \item \textbf{Absorption}: Junction relaxation releases $\Delta m_{np} = 1.293$ MeV into brane layer
  \item \textbf{Dissipation}: Energy redistributes among brane modes
  \item \textbf{Release}: $\mathcal{P}_{\mathrm{frozen}}$ selects $p + e^- + \bar{\nu}_e$
\end{enumerate}

\textbf{Why only electron channel?}
$\mathcal{P}_{\mathrm{energy}}$ forbids $\mu$ channel: $m_\mu = 105.7$ MeV $> \Delta m_{np} = 1.293$ MeV.

\textbf{Ledger:}
\begin{center}
\begin{tabular}{ll}
\toprule
\textbf{In} & \textbf{Out} \\
\midrule
$m_n = 939.565$ MeV & $m_p = 938.272$ MeV \\
& $E_e + E_{\bar{\nu}}$ (kinetic) \\
& $m_e = 0.511$ MeV (rest mass) \\
\midrule
Total: 939.565 MeV & Total: 939.565 MeV $\checkmark$ \\
\bottomrule
\end{tabular}
\end{center}

\textbf{Open:} Derive $\tau_n = 879.4$ s from junction barrier height \tagOpen{}.

% ------------------------------------------------------------------------------
\subsection{Muon: Brane-Dominant Fundamental}
\label{subsec:muon}
% ------------------------------------------------------------------------------

\begin{postulate}[Muon Ontology \tagP{}]
\label{post:muon}
The muon is a \textbf{brane-dominant excitation}: a fundamental mode localized
at $y \approx 0$ with mode index $n_\mu$. It is not composed of smaller parts
but is an excited state of the brane field, heavier than the electron ground mode.
\end{postulate}

\textbf{Decay mechanism.}
\begin{enumerate}[nosep]
  \item \textbf{Trigger}: Mode de-excitation ($n_\mu \to n_e$)
  \item \textbf{Absorption}: Mode energy difference enters available pool
  \item \textbf{Dissipation}: Energy redistributes into $e^-$, $\nu_\mu$, $\bar{\nu}_e$
  \item \textbf{Release}: $\mathcal{P}_{\mathrm{frozen}}$ projects three-body final state
\end{enumerate}

\textbf{Why no hadronic decays?}
$\mathcal{P}_{\mathrm{mode}}$ suppresses hadron channels: brane-dominant muon
has poor overlap with bulk-core hadron configurations.

\textbf{Three-body kinematics.}
The Michel spectrum (electron energy distribution) is a \tagBL{} from SM.
EDC interprets this as the natural phase-space distribution when three brane
modes share the available energy.

\textbf{Open:} Derive $\tau_\mu = 2.197 \times 10^{-6}$ s from mode spacing \tagOpen{}.

% ------------------------------------------------------------------------------
\subsection{Tau: Brane-Dominant Higher Mode}
\label{subsec:tau}
% ------------------------------------------------------------------------------

\begin{postulate}[Tau Ontology \tagP{}]
\label{post:tau}
The tau is a brane-dominant excitation with mode index $n_\tau > n_\mu > n_e$.
Higher mode index corresponds to larger mass and shorter lifetime.
\end{postulate}

\textbf{Decay mechanism.}
Same pipeline as muon, but with additional channels available due to larger mass.

\textbf{Mode hierarchy:}
\begin{equation}
  n_e < n_\mu < n_\tau \quad \Leftrightarrow \quad m_e < m_\mu < m_\tau
\end{equation}

\textbf{Why more decay channels?}
$\mathcal{P}_{\mathrm{energy}}$ allows hadronic channels ($m_\tau > m_\pi$).
The tau can decay to pions, kaons, and other hadrons, unlike the lighter muon.

\textbf{Open:} Derive mode spectrum $\{n_\ell\}$ and corresponding masses \tagOpen{}.

% ------------------------------------------------------------------------------
\subsection{Pion: Brane-Dominant Composite}
\label{subsec:pion}
% ------------------------------------------------------------------------------

\begin{postulate}[Pion Ontology \tagP{}]
\label{post:pion}
The charged pion is a \textbf{brane-dominant composite}: a bound state residing
primarily in the brane layer, with internal structure (candidate: junction-pair
\tagOpen{}). It is not a fundamental mode like the leptons.
\end{postulate}

\textbf{Decay mechanism.}
\begin{enumerate}[nosep]
  \item \textbf{Trigger}: Composite dissociation (internal binding releases)
  \item \textbf{Absorption}: Pion rest mass enters brane pool
  \item \textbf{Dissipation}: Energy redistributes into lepton + neutrino
  \item \textbf{Release}: $\mathcal{P}_{\mathrm{frozen}}$ with $\mathcal{P}_{\mathrm{chir}}$ helicity suppression
\end{enumerate}

\textbf{Helicity suppression.}
Both $\mu$ and $e$ channels are kinematically allowed, but:
\begin{equation}
  \frac{\Gamma(\pi \to e\nu)}{\Gamma(\pi \to \mu\nu)} \approx
  \left( \frac{m_e}{m_\mu} \right)^2 \approx 2.3 \times 10^{-5} \quad \tagBL{}
\end{equation}
EDC interprets this via $\mathcal{P}_{\mathrm{chir}}$: the electron's small mass
makes helicity flip costly \tagP{}/\tagOpen{}.

\textbf{Open:} Derive $m_\pi$ from 5D binding; derive helicity factor from
$\mathcal{P}_{\mathrm{chir}}$ explicitly \tagOpen{}.

% ------------------------------------------------------------------------------
\subsection{Electron: Brane Defect}
\label{subsec:electron}
% ------------------------------------------------------------------------------

\begin{postulate}[Electron Ontology \tagP{}]
\label{post:electron}
The electron is the \textbf{ground-mode brane defect}: the lowest-energy
brane-localized excitation, directly facing the observer. It is stable because
there is no lower brane mode to decay into.
\end{postulate}

\textbf{Why stable?}
$\mathcal{P}_{\mathrm{energy}}$ and $\mathcal{P}_{\mathrm{mode}}$ together
forbid any decay: there is no lighter charged lepton, and charge conservation
prevents decay to neutrinos alone.

\textbf{Role in other decays.}
The electron appears as an \emph{output} in neutron, muon, and tau decays
because it is the lightest available charged brane mode. It is always selected
when $\mathcal{P}_{\mathrm{energy}}$ allows and heavier modes are suppressed.

\textbf{Open:} Derive $m_e = 0.511$ MeV from brane ground-mode energy \tagOpen{}.

% ------------------------------------------------------------------------------
\subsection{Neutrino: Edge Mode}
\label{subsec:neutrino}
% ------------------------------------------------------------------------------

\begin{postulate}[Neutrino Ontology \tagP{}]
\label{post:neutrino}
The neutrino is an \textbf{edge mode}: an interfacial excitation propagating
along the bulk--brane boundary. It is neither bulk-extended (like hadrons) nor
brane-localized (like charged leptons), but exists \emph{at the interface}.
\end{postulate}

\textbf{Why weakly interacting?}
As an edge mode, the neutrino couples only via interface dynamics. Its coupling
to brane-localized particles is suppressed by the geometric mismatch: interface
modes have poor overlap with modes localized away from the boundary.

\textbf{Chirality.}
The interface has an inherent orientation (bulk vs.\ brane side), which translates
to chirality selection. Only left-handed neutrinos (right-handed antineutrinos)
couple at leading order---the V$-$A structure emerges geometrically \tagP{}.

\textbf{Role in decays.}
The neutrino appears in all weak decays as the ``missing energy'' carrier.
It is the natural repository for energy/momentum that must leave the brane
system without being detected as a charged particle.

\textbf{Open:} Derive neutrino masses from edge-mode spectrum; explain flavor
mixing geometrically \tagOpen{}.
