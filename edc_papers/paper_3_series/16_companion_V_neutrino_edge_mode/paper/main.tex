% ==============================================================================
% Companion V: Neutrino as Boundary/Edge Mode
% Version: 0.1 (Initial Draft)
% Date: 2026-01-20
% DOI: 10.5281/zenodo.18321383
% ==============================================================================

\documentclass[11pt,a4paper]{article}

% ─────────────────────────────────────────────────────────────────────────────
% Required packages BEFORE shared style
% ─────────────────────────────────────────────────────────────────────────────
\usepackage{amsmath,amssymb}
\usepackage{enumitem}
\usepackage{tikz}
\usetikzlibrary{positioning,arrows.meta,shapes.geometric,calc,decorations.pathmorphing}
\usepackage{tcolorbox}
\tcbuselibrary{skins,breakable}

% ─────────────────────────────────────────────────────────────────────────────
% Shared EDC style
% ─────────────────────────────────────────────────────────────────────────────
% edc_style.tex — Canonical EDC Paper Style for Paper 3 Series
% Version 1.0 — 2026-01-20
%
% USAGE: Include in preamble AFTER loading packages but BEFORE \begin{document}
%   % edc_style.tex — Canonical EDC Paper Style for Paper 3 Series
% Version 1.0 — 2026-01-20
%
% USAGE: Include in preamble AFTER loading packages but BEFORE \begin{document}
%   % edc_style.tex — Canonical EDC Paper Style for Paper 3 Series
% Version 1.0 — 2026-01-20
%
% USAGE: Include in preamble AFTER loading packages but BEFORE \begin{document}
%   \input{../_shared/style/edc_style}
%   \input{../_shared/style/tikz_style_edc}  % if using TikZ figures
%
% REQUIRED PACKAGES (load these in main document before \input):
%   fontspec, amsmath, amssymb, amsthm, mathtools, geometry
%   hyperref, enumitem, booktabs, array, xcolor, tcolorbox
%
% ============================================================

% ============================================================
%  EPISTEMIC TAG COLORS
% ============================================================
\definecolor{tagDer}{RGB}{0,128,0}      % Green - Derived
\definecolor{tagDc}{RGB}{0,0,200}       % Blue - Deduced/Constrained
\definecolor{tagCal}{RGB}{200,0,0}      % Red - Calibrated
\definecolor{tagP}{RGB}{128,0,128}      % Purple - Postulated
\definecolor{tagBL}{RGB}{128,128,128}   % Gray - Baseline
\definecolor{tagI}{RGB}{255,140,0}      % Orange - Identified
\definecolor{tagOpen}{RGB}{200,100,0}   % Dark orange - Open

% ============================================================
%  EPISTEMIC TAG COMMANDS
% ============================================================
% Use these to mark claims with their epistemic status
\newcommand{\tagDer}{\textcolor{tagDer}{\textbf{[Der]}}}    % Derived from axioms
\newcommand{\tagDc}{\textcolor{tagDc}{\textbf{[Dc]}}}       % Deduced/Constrained
\newcommand{\tagCal}{\textcolor{tagCal}{\textbf{[Cal]}}}    % Calibrated (fitted)
\newcommand{\tagP}{\textcolor{tagP}{\textbf{[P]}}}          % Postulated
\newcommand{\tagBL}{\textcolor{tagBL}{\textbf{[BL]}}}       % Baseline (external fact)
\newcommand{\tagI}{\textcolor{tagI}{\textbf{[I]}}}          % Identified (pattern match)
\newcommand{\tagOpen}{\textcolor{tagOpen}{\textbf{[OPEN]}}} % Open problem
\newcommand{\tagDef}{\textcolor{tagDc}{\textbf{[Def]}}}     % Definition

% ============================================================
%  THEOREM ENVIRONMENTS
% ============================================================
\newtheorem{postulate}{Postulate}
\newtheorem{definition}{Definition}[section]
\newtheorem{theorem}{Theorem}[section]
\newtheorem{lemma}[theorem]{Lemma}
\newtheorem{corollary}[theorem]{Corollary}
\newtheorem{proposition}[theorem]{Proposition}
\newtheorem{remark}{Remark}[section]

% ============================================================
%  COMMON EDC SYMBOLS
% ============================================================
% Symmetry groups
\newcommand{\Ztwo}{\mathbb{Z}_2}
\newcommand{\Zthree}{\mathbb{Z}_3}
\newcommand{\Ztri}{\mathbb{Z}_3}    % alias
\newcommand{\Zsix}{\mathbb{Z}_6}

% Geometric objects
\newcommand{\Sthree}{S^3}           % 3-sphere
\newcommand{\Stwo}{S^2}             % 2-sphere
\newcommand{\Bthree}{B^3}           % 3-ball
\newcommand{\Mfive}{\mathcal{M}_5}  % 5D manifold
\newcommand{\Bfour}{\mathcal{B}_4}  % 4D brane

% Physical quantities
\newcommand{\tension}{\tau}         % string/flux-tube tension (E/L)
\newcommand{\re}{r_e}               % electron radius

% Operators
\newcommand{\Pfrozen}{\mathcal{P}_{\mathrm{frozen}}}  % Frozen projection operator
\newcommand{\Ebrane}{\mathcal{E}_{\mathrm{brane}}}    % Brane energy store

% Bulk-brane exchange current (canonical notation from Framework v2.0)
\newcommand{\Jbb}[1]{J^{#1}_{\mathrm{bulk}\to\mathrm{brane}}}

% ============================================================
%  TCOLORBOX STYLES FOR EDC PAPERS
% ============================================================
% Cornerstone box (blue) — key claims/foundations
\tcbset{
    edcCornerstone/.style={
        colback=blue!5,
        colframe=blue!40!black,
        fonttitle=\bfseries
    }
}

% Guardrail box (gray) — epistemic warnings/constraints
\tcbset{
    edcGuardrail/.style={
        colback=gray!5!white,
        colframe=gray!60!black,
        fonttitle=\bfseries
    }
}

% PPN box (blue, lighter) — Physical Process Narrative
\tcbset{
    edcPPN/.style={
        colback=blue!5,
        colframe=blue!50!black,
        fonttitle=\bfseries
    }
}

% Canonical box (yellow/orange) — canonical definitions/glossary
\tcbset{
    edcCanonical/.style={
        colback=yellow!5,
        colframe=orange!60!black,
        fonttitle=\bfseries
    }
}

% Conceptual box (yellow/orange, lighter) — conceptual pictures
\tcbset{
    edcConcept/.style={
        colback=yellow!5,
        colframe=orange!50!black,
        fonttitle=\bfseries
    }
}

% Pathway box (purple) — energy pathways, mechanisms
\tcbset{
    edcPathway/.style={
        colback=purple!5,
        colframe=purple!40!black,
        fonttitle=\bfseries
    }
}

% Model box (green) — mechanical analogies, heuristics
\tcbset{
    edcModel/.style={
        colback=green!5,
        colframe=green!40!black,
        fonttitle=\bfseries
    }
}

% Warning box (red) — non-overclaim, limitations
\tcbset{
    edcWarning/.style={
        colback=red!5,
        colframe=red!40!black,
        fonttitle=\bfseries
    }
}

% Framework quote box (gray) — verbatim from Framework v2.0
\tcbset{
    edcFramework/.style={
        colback=gray!5!white,
        colframe=gray!60!black,
        fonttitle=\small
    }
}

% Mechanism box (teal) — mechanistic dimension principle narrative
\tcbset{
    edcMechanism/.style={
        colback=teal!5,
        colframe=teal!50!black,
        fonttitle=\bfseries,
        title={Mechanistic Dimension Note (Canon)}
    }
}

% ============================================================
%  MECHANISTIC DIMENSION HELPER MACRO
% ============================================================
% Usage: \edcMechanismNote{bulk cause}{brane process}{3D output}
%
% Example:
%   \edcMechanismNote{Junction relaxes toward Steiner minimum}%
%                    {Energy pumps into brane-layer modes, redistributes}%
%                    {Electron, antineutrino, proton emerge on 3D side}
%
\newcommand{\edcMechanismNote}[3]{%
\begin{tcolorbox}[edcMechanism]
\begin{itemize}[nosep,leftmargin=*]
    \item \textbf{5D cause (bulk):} #1
    \item \textbf{Brane-layer process:} #2
    \item \textbf{3D observation (output):} #3
\end{itemize}
\vspace{0.3em}
\footnotesize\textit{Ledger closure must hold: bulk + brane + 3D outputs conserve energy/quantum numbers.}
\end{tcolorbox}
}

% ============================================================
%  RELATED DOCUMENTS MACRO
% ============================================================
% Usage: \edcRelatedDocs{main paper title}{main DOI}{companion list}
%
% Example:
%     A: \emph{Effective Lagrangian} (\href{...}{DOI}) $\cdot$
%     B: \emph{WKB Prefactor} (\href{...}{DOI})
%   }

% NOTE: \edcRelatedDocs macro deprecated (DOI registry consolidated)
% Use consolidated Zenodo article as primary reference instead.

% ============================================================
%  DOI REGISTRY DEPRECATED
% ============================================================
% Previous individual DOIs have been deprecated.
% All EDC Weak Sector content is now consolidated into a single
% Zenodo article. See paper_3_series/19_edc_weak_sector_zenodo_article/

% ============================================================
%  PHYSICAL NARRATION RULE REMINDER
% ============================================================
% Every key equation MUST be accompanied by a physical narrative stating:
%   1. 5D cause: What changes in the bulk-core configuration?
%   2. Brane response: How does the brane absorb/redistribute energy?
%   3. 3D observable output: What do observers detect on the 3D side?
%
% This rule eliminates "numerology smell" by ensuring every formula
% has a mechanistic interpretation.

% ============================================================
%  END OF STYLE FILE
% ============================================================

%   % tikz_style_edc.tex — Reusable TikZ styles for EDC papers
% Version 1.0 — 2026-01-20
% Include via: \input{tikz_style_edc}

% ============================================================
% REQUIRED LIBRARIES (must be loaded in main document)
% ============================================================
% \usetikzlibrary{calc,angles,quotes,decorations.markings,decorations.pathmorphing,positioning}

% ============================================================
% POSITIONING DEFAULTS
% ============================================================
\tikzset{
    % Default node distances for horizontal/vertical layouts
    edc node distance/.style={node distance=1.6cm and 2.0cm},
    % Compact variant for dense diagrams
    edc compact/.style={node distance=1.2cm and 1.5cm},
    % Spread variant for clarity
    edc spread/.style={node distance=2.0cm and 2.5cm},
}

% ============================================================
% COLOR PALETTE (consistent with epistemic tags)
% ============================================================
\definecolor{edcBulk}{RGB}{220,50,50}        % Red tones for bulk/5D
\definecolor{edcBrane}{RGB}{50,150,50}       % Green tones for brane-layer
\definecolor{edcOutput}{RGB}{50,100,200}     % Blue tones for 3D outputs
\definecolor{edcNeutral}{RGB}{100,100,100}   % Gray for neutral/annotations

% ============================================================
% BOX STYLES
% ============================================================
\tikzset{
    % Generic EDC box (base style)
    edc box/.style={
        rectangle,
        draw,
        rounded corners=3pt,
        minimum width=2.2cm,
        minimum height=0.8cm,
        align=center,
        font=\small,
        inner sep=4pt,
    },
    % Bulk-core box (red family)
    bulk box/.style={
        edc box,
        fill=red!10,
        draw=edcBulk!70!black,
        text=black,
    },
    % Brane-layer box (green family)
    brane box/.style={
        edc box,
        fill=green!10,
        draw=edcBrane!70!black,
        text=black,
    },
    % 3D output box (blue family)
    output box/.style={
        edc box,
        fill=blue!10,
        draw=edcOutput!70!black,
        text=black,
    },
    % Neutral/process box
    process box/.style={
        edc box,
        fill=gray!10,
        draw=gray!60!black,
        text=black,
    },
    % Label-only box (no background)
    label box/.style={
        rectangle,
        rounded corners=2pt,
        draw=gray!40,
        fill=white,
        inner sep=2pt,
        font=\scriptsize,
    },
}

% ============================================================
% ARROW STYLES
% ============================================================
\tikzset{
    % Standard thick arrow
    edc arrow/.style={
        ->,
        >=stealth,
        thick,
    },
    % Emphasized arrow (for main flow)
    edc flow/.style={
        ->,
        >=stealth,
        very thick,
        line width=1.2pt,
    },
    % Dashed arrow (for optional/weak connections)
    edc dashed/.style={
        ->,
        >=stealth,
        thick,
        dashed,
    },
    % Double arrow (for bidirectional)
    edc bidir/.style={
        <->,
        >=stealth,
        thick,
    },
}

% ============================================================
% REGION STYLES (for background fills)
% ============================================================
\tikzset{
    % Bulk region (5D)
    bulk region/.style={
        fill=blue!8,
    },
    % Brane layer region
    brane region/.style={
        fill=yellow!25,
    },
    % Observer/3D region
    observer region/.style={
        fill=green!8,
    },
}

% ============================================================
% LABEL STYLES
% ============================================================
\tikzset{
    % Phase label (below nodes)
    phase label/.style={
        font=\scriptsize\itshape,
        text=black!70,
    },
    % Equation label (for inline math)
    eq label/.style={
        font=\scriptsize,
        fill=white,
        inner sep=1pt,
    },
    % Section annotation
    section label/.style={
        font=\footnotesize\bfseries,
        text=black,
    },
}

% ============================================================
% JUNCTION/PARTICLE STYLES
% ============================================================
\tikzset{
    % Y-junction point
    junction point/.style={
        circle,
        fill=red!60!black,
        minimum size=4pt,
        inner sep=0pt,
    },
    % Flux tube arm
    flux arm/.style={
        thick,
        blue!60!black,
    },
    % Particle dot (electron, etc.)
    particle/.style={
        circle,
        fill=black,
        minimum size=5pt,
        inner sep=0pt,
    },
    % Neutrino (smaller, gray)
    neutrino/.style={
        circle,
        fill=gray,
        minimum size=4pt,
        inner sep=0pt,
    },
}

% ============================================================
% SPRING DECORATION (for mechanical models)
% ============================================================
\tikzset{
    spring/.style={
        thick,
        decorate,
        decoration={
            coil,
            aspect=0.5,
            segment length=2mm,
            amplitude=2mm,
        },
    },
    % Wave decoration (for field modes)
    wave field/.style={
        thick,
        decorate,
        decoration={
            snake,
            amplitude=2pt,
            segment length=8pt,
        },
    },
}

% ============================================================
% BOUNDARY STYLES
% ============================================================
\tikzset{
    % Bulk-facing boundary (dashed red)
    bulk boundary/.style={
        very thick,
        red!70!black,
        dashed,
    },
    % Observer-facing boundary (solid green)
    observer boundary/.style={
        thick,
        green!50!black,
    },
    % Brane edge (orange)
    brane edge/.style={
        thick,
        orange!70!black,
    },
}

% ============================================================
% CONVENIENCE COMMANDS
% ============================================================
% Arrow label (above)
\newcommand{\arrlabel}[1]{\scriptsize #1}
% Arrow label (below)
\newcommand{\arrlabelb}[1]{\scriptsize #1}

% ============================================================
% END OF STYLE FILE
% ============================================================
  % if using TikZ figures
%
% REQUIRED PACKAGES (load these in main document before \input):
%   fontspec, amsmath, amssymb, amsthm, mathtools, geometry
%   hyperref, enumitem, booktabs, array, xcolor, tcolorbox
%
% ============================================================

% ============================================================
%  EPISTEMIC TAG COLORS
% ============================================================
\definecolor{tagDer}{RGB}{0,128,0}      % Green - Derived
\definecolor{tagDc}{RGB}{0,0,200}       % Blue - Deduced/Constrained
\definecolor{tagCal}{RGB}{200,0,0}      % Red - Calibrated
\definecolor{tagP}{RGB}{128,0,128}      % Purple - Postulated
\definecolor{tagBL}{RGB}{128,128,128}   % Gray - Baseline
\definecolor{tagI}{RGB}{255,140,0}      % Orange - Identified
\definecolor{tagOpen}{RGB}{200,100,0}   % Dark orange - Open

% ============================================================
%  EPISTEMIC TAG COMMANDS
% ============================================================
% Use these to mark claims with their epistemic status
\newcommand{\tagDer}{\textcolor{tagDer}{\textbf{[Der]}}}    % Derived from axioms
\newcommand{\tagDc}{\textcolor{tagDc}{\textbf{[Dc]}}}       % Deduced/Constrained
\newcommand{\tagCal}{\textcolor{tagCal}{\textbf{[Cal]}}}    % Calibrated (fitted)
\newcommand{\tagP}{\textcolor{tagP}{\textbf{[P]}}}          % Postulated
\newcommand{\tagBL}{\textcolor{tagBL}{\textbf{[BL]}}}       % Baseline (external fact)
\newcommand{\tagI}{\textcolor{tagI}{\textbf{[I]}}}          % Identified (pattern match)
\newcommand{\tagOpen}{\textcolor{tagOpen}{\textbf{[OPEN]}}} % Open problem
\newcommand{\tagDef}{\textcolor{tagDc}{\textbf{[Def]}}}     % Definition

% ============================================================
%  THEOREM ENVIRONMENTS
% ============================================================
\newtheorem{postulate}{Postulate}
\newtheorem{definition}{Definition}[section]
\newtheorem{theorem}{Theorem}[section]
\newtheorem{lemma}[theorem]{Lemma}
\newtheorem{corollary}[theorem]{Corollary}
\newtheorem{proposition}[theorem]{Proposition}
\newtheorem{remark}{Remark}[section]

% ============================================================
%  COMMON EDC SYMBOLS
% ============================================================
% Symmetry groups
\newcommand{\Ztwo}{\mathbb{Z}_2}
\newcommand{\Zthree}{\mathbb{Z}_3}
\newcommand{\Ztri}{\mathbb{Z}_3}    % alias
\newcommand{\Zsix}{\mathbb{Z}_6}

% Geometric objects
\newcommand{\Sthree}{S^3}           % 3-sphere
\newcommand{\Stwo}{S^2}             % 2-sphere
\newcommand{\Bthree}{B^3}           % 3-ball
\newcommand{\Mfive}{\mathcal{M}_5}  % 5D manifold
\newcommand{\Bfour}{\mathcal{B}_4}  % 4D brane

% Physical quantities
\newcommand{\tension}{\tau}         % string/flux-tube tension (E/L)
\newcommand{\re}{r_e}               % electron radius

% Operators
\newcommand{\Pfrozen}{\mathcal{P}_{\mathrm{frozen}}}  % Frozen projection operator
\newcommand{\Ebrane}{\mathcal{E}_{\mathrm{brane}}}    % Brane energy store

% Bulk-brane exchange current (canonical notation from Framework v2.0)
\newcommand{\Jbb}[1]{J^{#1}_{\mathrm{bulk}\to\mathrm{brane}}}

% ============================================================
%  TCOLORBOX STYLES FOR EDC PAPERS
% ============================================================
% Cornerstone box (blue) — key claims/foundations
\tcbset{
    edcCornerstone/.style={
        colback=blue!5,
        colframe=blue!40!black,
        fonttitle=\bfseries
    }
}

% Guardrail box (gray) — epistemic warnings/constraints
\tcbset{
    edcGuardrail/.style={
        colback=gray!5!white,
        colframe=gray!60!black,
        fonttitle=\bfseries
    }
}

% PPN box (blue, lighter) — Physical Process Narrative
\tcbset{
    edcPPN/.style={
        colback=blue!5,
        colframe=blue!50!black,
        fonttitle=\bfseries
    }
}

% Canonical box (yellow/orange) — canonical definitions/glossary
\tcbset{
    edcCanonical/.style={
        colback=yellow!5,
        colframe=orange!60!black,
        fonttitle=\bfseries
    }
}

% Conceptual box (yellow/orange, lighter) — conceptual pictures
\tcbset{
    edcConcept/.style={
        colback=yellow!5,
        colframe=orange!50!black,
        fonttitle=\bfseries
    }
}

% Pathway box (purple) — energy pathways, mechanisms
\tcbset{
    edcPathway/.style={
        colback=purple!5,
        colframe=purple!40!black,
        fonttitle=\bfseries
    }
}

% Model box (green) — mechanical analogies, heuristics
\tcbset{
    edcModel/.style={
        colback=green!5,
        colframe=green!40!black,
        fonttitle=\bfseries
    }
}

% Warning box (red) — non-overclaim, limitations
\tcbset{
    edcWarning/.style={
        colback=red!5,
        colframe=red!40!black,
        fonttitle=\bfseries
    }
}

% Framework quote box (gray) — verbatim from Framework v2.0
\tcbset{
    edcFramework/.style={
        colback=gray!5!white,
        colframe=gray!60!black,
        fonttitle=\small
    }
}

% Mechanism box (teal) — mechanistic dimension principle narrative
\tcbset{
    edcMechanism/.style={
        colback=teal!5,
        colframe=teal!50!black,
        fonttitle=\bfseries,
        title={Mechanistic Dimension Note (Canon)}
    }
}

% ============================================================
%  MECHANISTIC DIMENSION HELPER MACRO
% ============================================================
% Usage: \edcMechanismNote{bulk cause}{brane process}{3D output}
%
% Example:
%   \edcMechanismNote{Junction relaxes toward Steiner minimum}%
%                    {Energy pumps into brane-layer modes, redistributes}%
%                    {Electron, antineutrino, proton emerge on 3D side}
%
\newcommand{\edcMechanismNote}[3]{%
\begin{tcolorbox}[edcMechanism]
\begin{itemize}[nosep,leftmargin=*]
    \item \textbf{5D cause (bulk):} #1
    \item \textbf{Brane-layer process:} #2
    \item \textbf{3D observation (output):} #3
\end{itemize}
\vspace{0.3em}
\footnotesize\textit{Ledger closure must hold: bulk + brane + 3D outputs conserve energy/quantum numbers.}
\end{tcolorbox}
}

% ============================================================
%  RELATED DOCUMENTS MACRO
% ============================================================
% Usage: \edcRelatedDocs{main paper title}{main DOI}{companion list}
%
% Example:
%     A: \emph{Effective Lagrangian} (\href{...}{DOI}) $\cdot$
%     B: \emph{WKB Prefactor} (\href{...}{DOI})
%   }

% NOTE: \edcRelatedDocs macro deprecated (DOI registry consolidated)
% Use consolidated Zenodo article as primary reference instead.

% ============================================================
%  DOI REGISTRY DEPRECATED
% ============================================================
% Previous individual DOIs have been deprecated.
% All EDC Weak Sector content is now consolidated into a single
% Zenodo article. See paper_3_series/19_edc_weak_sector_zenodo_article/

% ============================================================
%  PHYSICAL NARRATION RULE REMINDER
% ============================================================
% Every key equation MUST be accompanied by a physical narrative stating:
%   1. 5D cause: What changes in the bulk-core configuration?
%   2. Brane response: How does the brane absorb/redistribute energy?
%   3. 3D observable output: What do observers detect on the 3D side?
%
% This rule eliminates "numerology smell" by ensuring every formula
% has a mechanistic interpretation.

% ============================================================
%  END OF STYLE FILE
% ============================================================

%   % tikz_style_edc.tex — Reusable TikZ styles for EDC papers
% Version 1.0 — 2026-01-20
% Include via: % tikz_style_edc.tex — Reusable TikZ styles for EDC papers
% Version 1.0 — 2026-01-20
% Include via: \input{tikz_style_edc}

% ============================================================
% REQUIRED LIBRARIES (must be loaded in main document)
% ============================================================
% \usetikzlibrary{calc,angles,quotes,decorations.markings,decorations.pathmorphing,positioning}

% ============================================================
% POSITIONING DEFAULTS
% ============================================================
\tikzset{
    % Default node distances for horizontal/vertical layouts
    edc node distance/.style={node distance=1.6cm and 2.0cm},
    % Compact variant for dense diagrams
    edc compact/.style={node distance=1.2cm and 1.5cm},
    % Spread variant for clarity
    edc spread/.style={node distance=2.0cm and 2.5cm},
}

% ============================================================
% COLOR PALETTE (consistent with epistemic tags)
% ============================================================
\definecolor{edcBulk}{RGB}{220,50,50}        % Red tones for bulk/5D
\definecolor{edcBrane}{RGB}{50,150,50}       % Green tones for brane-layer
\definecolor{edcOutput}{RGB}{50,100,200}     % Blue tones for 3D outputs
\definecolor{edcNeutral}{RGB}{100,100,100}   % Gray for neutral/annotations

% ============================================================
% BOX STYLES
% ============================================================
\tikzset{
    % Generic EDC box (base style)
    edc box/.style={
        rectangle,
        draw,
        rounded corners=3pt,
        minimum width=2.2cm,
        minimum height=0.8cm,
        align=center,
        font=\small,
        inner sep=4pt,
    },
    % Bulk-core box (red family)
    bulk box/.style={
        edc box,
        fill=red!10,
        draw=edcBulk!70!black,
        text=black,
    },
    % Brane-layer box (green family)
    brane box/.style={
        edc box,
        fill=green!10,
        draw=edcBrane!70!black,
        text=black,
    },
    % 3D output box (blue family)
    output box/.style={
        edc box,
        fill=blue!10,
        draw=edcOutput!70!black,
        text=black,
    },
    % Neutral/process box
    process box/.style={
        edc box,
        fill=gray!10,
        draw=gray!60!black,
        text=black,
    },
    % Label-only box (no background)
    label box/.style={
        rectangle,
        rounded corners=2pt,
        draw=gray!40,
        fill=white,
        inner sep=2pt,
        font=\scriptsize,
    },
}

% ============================================================
% ARROW STYLES
% ============================================================
\tikzset{
    % Standard thick arrow
    edc arrow/.style={
        ->,
        >=stealth,
        thick,
    },
    % Emphasized arrow (for main flow)
    edc flow/.style={
        ->,
        >=stealth,
        very thick,
        line width=1.2pt,
    },
    % Dashed arrow (for optional/weak connections)
    edc dashed/.style={
        ->,
        >=stealth,
        thick,
        dashed,
    },
    % Double arrow (for bidirectional)
    edc bidir/.style={
        <->,
        >=stealth,
        thick,
    },
}

% ============================================================
% REGION STYLES (for background fills)
% ============================================================
\tikzset{
    % Bulk region (5D)
    bulk region/.style={
        fill=blue!8,
    },
    % Brane layer region
    brane region/.style={
        fill=yellow!25,
    },
    % Observer/3D region
    observer region/.style={
        fill=green!8,
    },
}

% ============================================================
% LABEL STYLES
% ============================================================
\tikzset{
    % Phase label (below nodes)
    phase label/.style={
        font=\scriptsize\itshape,
        text=black!70,
    },
    % Equation label (for inline math)
    eq label/.style={
        font=\scriptsize,
        fill=white,
        inner sep=1pt,
    },
    % Section annotation
    section label/.style={
        font=\footnotesize\bfseries,
        text=black,
    },
}

% ============================================================
% JUNCTION/PARTICLE STYLES
% ============================================================
\tikzset{
    % Y-junction point
    junction point/.style={
        circle,
        fill=red!60!black,
        minimum size=4pt,
        inner sep=0pt,
    },
    % Flux tube arm
    flux arm/.style={
        thick,
        blue!60!black,
    },
    % Particle dot (electron, etc.)
    particle/.style={
        circle,
        fill=black,
        minimum size=5pt,
        inner sep=0pt,
    },
    % Neutrino (smaller, gray)
    neutrino/.style={
        circle,
        fill=gray,
        minimum size=4pt,
        inner sep=0pt,
    },
}

% ============================================================
% SPRING DECORATION (for mechanical models)
% ============================================================
\tikzset{
    spring/.style={
        thick,
        decorate,
        decoration={
            coil,
            aspect=0.5,
            segment length=2mm,
            amplitude=2mm,
        },
    },
    % Wave decoration (for field modes)
    wave field/.style={
        thick,
        decorate,
        decoration={
            snake,
            amplitude=2pt,
            segment length=8pt,
        },
    },
}

% ============================================================
% BOUNDARY STYLES
% ============================================================
\tikzset{
    % Bulk-facing boundary (dashed red)
    bulk boundary/.style={
        very thick,
        red!70!black,
        dashed,
    },
    % Observer-facing boundary (solid green)
    observer boundary/.style={
        thick,
        green!50!black,
    },
    % Brane edge (orange)
    brane edge/.style={
        thick,
        orange!70!black,
    },
}

% ============================================================
% CONVENIENCE COMMANDS
% ============================================================
% Arrow label (above)
\newcommand{\arrlabel}[1]{\scriptsize #1}
% Arrow label (below)
\newcommand{\arrlabelb}[1]{\scriptsize #1}

% ============================================================
% END OF STYLE FILE
% ============================================================


% ============================================================
% REQUIRED LIBRARIES (must be loaded in main document)
% ============================================================
% \usetikzlibrary{calc,angles,quotes,decorations.markings,decorations.pathmorphing,positioning}

% ============================================================
% POSITIONING DEFAULTS
% ============================================================
\tikzset{
    % Default node distances for horizontal/vertical layouts
    edc node distance/.style={node distance=1.6cm and 2.0cm},
    % Compact variant for dense diagrams
    edc compact/.style={node distance=1.2cm and 1.5cm},
    % Spread variant for clarity
    edc spread/.style={node distance=2.0cm and 2.5cm},
}

% ============================================================
% COLOR PALETTE (consistent with epistemic tags)
% ============================================================
\definecolor{edcBulk}{RGB}{220,50,50}        % Red tones for bulk/5D
\definecolor{edcBrane}{RGB}{50,150,50}       % Green tones for brane-layer
\definecolor{edcOutput}{RGB}{50,100,200}     % Blue tones for 3D outputs
\definecolor{edcNeutral}{RGB}{100,100,100}   % Gray for neutral/annotations

% ============================================================
% BOX STYLES
% ============================================================
\tikzset{
    % Generic EDC box (base style)
    edc box/.style={
        rectangle,
        draw,
        rounded corners=3pt,
        minimum width=2.2cm,
        minimum height=0.8cm,
        align=center,
        font=\small,
        inner sep=4pt,
    },
    % Bulk-core box (red family)
    bulk box/.style={
        edc box,
        fill=red!10,
        draw=edcBulk!70!black,
        text=black,
    },
    % Brane-layer box (green family)
    brane box/.style={
        edc box,
        fill=green!10,
        draw=edcBrane!70!black,
        text=black,
    },
    % 3D output box (blue family)
    output box/.style={
        edc box,
        fill=blue!10,
        draw=edcOutput!70!black,
        text=black,
    },
    % Neutral/process box
    process box/.style={
        edc box,
        fill=gray!10,
        draw=gray!60!black,
        text=black,
    },
    % Label-only box (no background)
    label box/.style={
        rectangle,
        rounded corners=2pt,
        draw=gray!40,
        fill=white,
        inner sep=2pt,
        font=\scriptsize,
    },
}

% ============================================================
% ARROW STYLES
% ============================================================
\tikzset{
    % Standard thick arrow
    edc arrow/.style={
        ->,
        >=stealth,
        thick,
    },
    % Emphasized arrow (for main flow)
    edc flow/.style={
        ->,
        >=stealth,
        very thick,
        line width=1.2pt,
    },
    % Dashed arrow (for optional/weak connections)
    edc dashed/.style={
        ->,
        >=stealth,
        thick,
        dashed,
    },
    % Double arrow (for bidirectional)
    edc bidir/.style={
        <->,
        >=stealth,
        thick,
    },
}

% ============================================================
% REGION STYLES (for background fills)
% ============================================================
\tikzset{
    % Bulk region (5D)
    bulk region/.style={
        fill=blue!8,
    },
    % Brane layer region
    brane region/.style={
        fill=yellow!25,
    },
    % Observer/3D region
    observer region/.style={
        fill=green!8,
    },
}

% ============================================================
% LABEL STYLES
% ============================================================
\tikzset{
    % Phase label (below nodes)
    phase label/.style={
        font=\scriptsize\itshape,
        text=black!70,
    },
    % Equation label (for inline math)
    eq label/.style={
        font=\scriptsize,
        fill=white,
        inner sep=1pt,
    },
    % Section annotation
    section label/.style={
        font=\footnotesize\bfseries,
        text=black,
    },
}

% ============================================================
% JUNCTION/PARTICLE STYLES
% ============================================================
\tikzset{
    % Y-junction point
    junction point/.style={
        circle,
        fill=red!60!black,
        minimum size=4pt,
        inner sep=0pt,
    },
    % Flux tube arm
    flux arm/.style={
        thick,
        blue!60!black,
    },
    % Particle dot (electron, etc.)
    particle/.style={
        circle,
        fill=black,
        minimum size=5pt,
        inner sep=0pt,
    },
    % Neutrino (smaller, gray)
    neutrino/.style={
        circle,
        fill=gray,
        minimum size=4pt,
        inner sep=0pt,
    },
}

% ============================================================
% SPRING DECORATION (for mechanical models)
% ============================================================
\tikzset{
    spring/.style={
        thick,
        decorate,
        decoration={
            coil,
            aspect=0.5,
            segment length=2mm,
            amplitude=2mm,
        },
    },
    % Wave decoration (for field modes)
    wave field/.style={
        thick,
        decorate,
        decoration={
            snake,
            amplitude=2pt,
            segment length=8pt,
        },
    },
}

% ============================================================
% BOUNDARY STYLES
% ============================================================
\tikzset{
    % Bulk-facing boundary (dashed red)
    bulk boundary/.style={
        very thick,
        red!70!black,
        dashed,
    },
    % Observer-facing boundary (solid green)
    observer boundary/.style={
        thick,
        green!50!black,
    },
    % Brane edge (orange)
    brane edge/.style={
        thick,
        orange!70!black,
    },
}

% ============================================================
% CONVENIENCE COMMANDS
% ============================================================
% Arrow label (above)
\newcommand{\arrlabel}[1]{\scriptsize #1}
% Arrow label (below)
\newcommand{\arrlabelb}[1]{\scriptsize #1}

% ============================================================
% END OF STYLE FILE
% ============================================================
  % if using TikZ figures
%
% REQUIRED PACKAGES (load these in main document before \input):
%   fontspec, amsmath, amssymb, amsthm, mathtools, geometry
%   hyperref, enumitem, booktabs, array, xcolor, tcolorbox
%
% ============================================================

% ============================================================
%  EPISTEMIC TAG COLORS
% ============================================================
\definecolor{tagDer}{RGB}{0,128,0}      % Green - Derived
\definecolor{tagDc}{RGB}{0,0,200}       % Blue - Deduced/Constrained
\definecolor{tagCal}{RGB}{200,0,0}      % Red - Calibrated
\definecolor{tagP}{RGB}{128,0,128}      % Purple - Postulated
\definecolor{tagBL}{RGB}{128,128,128}   % Gray - Baseline
\definecolor{tagI}{RGB}{255,140,0}      % Orange - Identified
\definecolor{tagOpen}{RGB}{200,100,0}   % Dark orange - Open

% ============================================================
%  EPISTEMIC TAG COMMANDS
% ============================================================
% Use these to mark claims with their epistemic status
\newcommand{\tagDer}{\textcolor{tagDer}{\textbf{[Der]}}}    % Derived from axioms
\newcommand{\tagDc}{\textcolor{tagDc}{\textbf{[Dc]}}}       % Deduced/Constrained
\newcommand{\tagCal}{\textcolor{tagCal}{\textbf{[Cal]}}}    % Calibrated (fitted)
\newcommand{\tagP}{\textcolor{tagP}{\textbf{[P]}}}          % Postulated
\newcommand{\tagBL}{\textcolor{tagBL}{\textbf{[BL]}}}       % Baseline (external fact)
\newcommand{\tagI}{\textcolor{tagI}{\textbf{[I]}}}          % Identified (pattern match)
\newcommand{\tagOpen}{\textcolor{tagOpen}{\textbf{[OPEN]}}} % Open problem
\newcommand{\tagDef}{\textcolor{tagDc}{\textbf{[Def]}}}     % Definition

% ============================================================
%  THEOREM ENVIRONMENTS
% ============================================================
\newtheorem{postulate}{Postulate}
\newtheorem{definition}{Definition}[section]
\newtheorem{theorem}{Theorem}[section]
\newtheorem{lemma}[theorem]{Lemma}
\newtheorem{corollary}[theorem]{Corollary}
\newtheorem{proposition}[theorem]{Proposition}
\newtheorem{remark}{Remark}[section]

% ============================================================
%  COMMON EDC SYMBOLS
% ============================================================
% Symmetry groups
\newcommand{\Ztwo}{\mathbb{Z}_2}
\newcommand{\Zthree}{\mathbb{Z}_3}
\newcommand{\Ztri}{\mathbb{Z}_3}    % alias
\newcommand{\Zsix}{\mathbb{Z}_6}

% Geometric objects
\newcommand{\Sthree}{S^3}           % 3-sphere
\newcommand{\Stwo}{S^2}             % 2-sphere
\newcommand{\Bthree}{B^3}           % 3-ball
\newcommand{\Mfive}{\mathcal{M}_5}  % 5D manifold
\newcommand{\Bfour}{\mathcal{B}_4}  % 4D brane

% Physical quantities
\newcommand{\tension}{\tau}         % string/flux-tube tension (E/L)
\newcommand{\re}{r_e}               % electron radius

% Operators
\newcommand{\Pfrozen}{\mathcal{P}_{\mathrm{frozen}}}  % Frozen projection operator
\newcommand{\Ebrane}{\mathcal{E}_{\mathrm{brane}}}    % Brane energy store

% Bulk-brane exchange current (canonical notation from Framework v2.0)
\newcommand{\Jbb}[1]{J^{#1}_{\mathrm{bulk}\to\mathrm{brane}}}

% ============================================================
%  TCOLORBOX STYLES FOR EDC PAPERS
% ============================================================
% Cornerstone box (blue) — key claims/foundations
\tcbset{
    edcCornerstone/.style={
        colback=blue!5,
        colframe=blue!40!black,
        fonttitle=\bfseries
    }
}

% Guardrail box (gray) — epistemic warnings/constraints
\tcbset{
    edcGuardrail/.style={
        colback=gray!5!white,
        colframe=gray!60!black,
        fonttitle=\bfseries
    }
}

% PPN box (blue, lighter) — Physical Process Narrative
\tcbset{
    edcPPN/.style={
        colback=blue!5,
        colframe=blue!50!black,
        fonttitle=\bfseries
    }
}

% Canonical box (yellow/orange) — canonical definitions/glossary
\tcbset{
    edcCanonical/.style={
        colback=yellow!5,
        colframe=orange!60!black,
        fonttitle=\bfseries
    }
}

% Conceptual box (yellow/orange, lighter) — conceptual pictures
\tcbset{
    edcConcept/.style={
        colback=yellow!5,
        colframe=orange!50!black,
        fonttitle=\bfseries
    }
}

% Pathway box (purple) — energy pathways, mechanisms
\tcbset{
    edcPathway/.style={
        colback=purple!5,
        colframe=purple!40!black,
        fonttitle=\bfseries
    }
}

% Model box (green) — mechanical analogies, heuristics
\tcbset{
    edcModel/.style={
        colback=green!5,
        colframe=green!40!black,
        fonttitle=\bfseries
    }
}

% Warning box (red) — non-overclaim, limitations
\tcbset{
    edcWarning/.style={
        colback=red!5,
        colframe=red!40!black,
        fonttitle=\bfseries
    }
}

% Framework quote box (gray) — verbatim from Framework v2.0
\tcbset{
    edcFramework/.style={
        colback=gray!5!white,
        colframe=gray!60!black,
        fonttitle=\small
    }
}

% Mechanism box (teal) — mechanistic dimension principle narrative
\tcbset{
    edcMechanism/.style={
        colback=teal!5,
        colframe=teal!50!black,
        fonttitle=\bfseries,
        title={Mechanistic Dimension Note (Canon)}
    }
}

% ============================================================
%  MECHANISTIC DIMENSION HELPER MACRO
% ============================================================
% Usage: \edcMechanismNote{bulk cause}{brane process}{3D output}
%
% Example:
%   \edcMechanismNote{Junction relaxes toward Steiner minimum}%
%                    {Energy pumps into brane-layer modes, redistributes}%
%                    {Electron, antineutrino, proton emerge on 3D side}
%
\newcommand{\edcMechanismNote}[3]{%
\begin{tcolorbox}[edcMechanism]
\begin{itemize}[nosep,leftmargin=*]
    \item \textbf{5D cause (bulk):} #1
    \item \textbf{Brane-layer process:} #2
    \item \textbf{3D observation (output):} #3
\end{itemize}
\vspace{0.3em}
\footnotesize\textit{Ledger closure must hold: bulk + brane + 3D outputs conserve energy/quantum numbers.}
\end{tcolorbox}
}

% ============================================================
%  RELATED DOCUMENTS MACRO
% ============================================================
% Usage: \edcRelatedDocs{main paper title}{main DOI}{companion list}
%
% Example:
%     A: \emph{Effective Lagrangian} (\href{...}{DOI}) $\cdot$
%     B: \emph{WKB Prefactor} (\href{...}{DOI})
%   }

% NOTE: \edcRelatedDocs macro deprecated (DOI registry consolidated)
% Use consolidated Zenodo article as primary reference instead.

% ============================================================
%  DOI REGISTRY DEPRECATED
% ============================================================
% Previous individual DOIs have been deprecated.
% All EDC Weak Sector content is now consolidated into a single
% Zenodo article. See paper_3_series/19_edc_weak_sector_zenodo_article/

% ============================================================
%  PHYSICAL NARRATION RULE REMINDER
% ============================================================
% Every key equation MUST be accompanied by a physical narrative stating:
%   1. 5D cause: What changes in the bulk-core configuration?
%   2. Brane response: How does the brane absorb/redistribute energy?
%   3. 3D observable output: What do observers detect on the 3D side?
%
% This rule eliminates "numerology smell" by ensuring every formula
% has a mechanistic interpretation.

% ============================================================
%  END OF STYLE FILE
% ============================================================

% tikz_style_edc.tex — Reusable TikZ styles for EDC papers
% Version 1.0 — 2026-01-20
% Include via: % tikz_style_edc.tex — Reusable TikZ styles for EDC papers
% Version 1.0 — 2026-01-20
% Include via: % tikz_style_edc.tex — Reusable TikZ styles for EDC papers
% Version 1.0 — 2026-01-20
% Include via: \input{tikz_style_edc}

% ============================================================
% REQUIRED LIBRARIES (must be loaded in main document)
% ============================================================
% \usetikzlibrary{calc,angles,quotes,decorations.markings,decorations.pathmorphing,positioning}

% ============================================================
% POSITIONING DEFAULTS
% ============================================================
\tikzset{
    % Default node distances for horizontal/vertical layouts
    edc node distance/.style={node distance=1.6cm and 2.0cm},
    % Compact variant for dense diagrams
    edc compact/.style={node distance=1.2cm and 1.5cm},
    % Spread variant for clarity
    edc spread/.style={node distance=2.0cm and 2.5cm},
}

% ============================================================
% COLOR PALETTE (consistent with epistemic tags)
% ============================================================
\definecolor{edcBulk}{RGB}{220,50,50}        % Red tones for bulk/5D
\definecolor{edcBrane}{RGB}{50,150,50}       % Green tones for brane-layer
\definecolor{edcOutput}{RGB}{50,100,200}     % Blue tones for 3D outputs
\definecolor{edcNeutral}{RGB}{100,100,100}   % Gray for neutral/annotations

% ============================================================
% BOX STYLES
% ============================================================
\tikzset{
    % Generic EDC box (base style)
    edc box/.style={
        rectangle,
        draw,
        rounded corners=3pt,
        minimum width=2.2cm,
        minimum height=0.8cm,
        align=center,
        font=\small,
        inner sep=4pt,
    },
    % Bulk-core box (red family)
    bulk box/.style={
        edc box,
        fill=red!10,
        draw=edcBulk!70!black,
        text=black,
    },
    % Brane-layer box (green family)
    brane box/.style={
        edc box,
        fill=green!10,
        draw=edcBrane!70!black,
        text=black,
    },
    % 3D output box (blue family)
    output box/.style={
        edc box,
        fill=blue!10,
        draw=edcOutput!70!black,
        text=black,
    },
    % Neutral/process box
    process box/.style={
        edc box,
        fill=gray!10,
        draw=gray!60!black,
        text=black,
    },
    % Label-only box (no background)
    label box/.style={
        rectangle,
        rounded corners=2pt,
        draw=gray!40,
        fill=white,
        inner sep=2pt,
        font=\scriptsize,
    },
}

% ============================================================
% ARROW STYLES
% ============================================================
\tikzset{
    % Standard thick arrow
    edc arrow/.style={
        ->,
        >=stealth,
        thick,
    },
    % Emphasized arrow (for main flow)
    edc flow/.style={
        ->,
        >=stealth,
        very thick,
        line width=1.2pt,
    },
    % Dashed arrow (for optional/weak connections)
    edc dashed/.style={
        ->,
        >=stealth,
        thick,
        dashed,
    },
    % Double arrow (for bidirectional)
    edc bidir/.style={
        <->,
        >=stealth,
        thick,
    },
}

% ============================================================
% REGION STYLES (for background fills)
% ============================================================
\tikzset{
    % Bulk region (5D)
    bulk region/.style={
        fill=blue!8,
    },
    % Brane layer region
    brane region/.style={
        fill=yellow!25,
    },
    % Observer/3D region
    observer region/.style={
        fill=green!8,
    },
}

% ============================================================
% LABEL STYLES
% ============================================================
\tikzset{
    % Phase label (below nodes)
    phase label/.style={
        font=\scriptsize\itshape,
        text=black!70,
    },
    % Equation label (for inline math)
    eq label/.style={
        font=\scriptsize,
        fill=white,
        inner sep=1pt,
    },
    % Section annotation
    section label/.style={
        font=\footnotesize\bfseries,
        text=black,
    },
}

% ============================================================
% JUNCTION/PARTICLE STYLES
% ============================================================
\tikzset{
    % Y-junction point
    junction point/.style={
        circle,
        fill=red!60!black,
        minimum size=4pt,
        inner sep=0pt,
    },
    % Flux tube arm
    flux arm/.style={
        thick,
        blue!60!black,
    },
    % Particle dot (electron, etc.)
    particle/.style={
        circle,
        fill=black,
        minimum size=5pt,
        inner sep=0pt,
    },
    % Neutrino (smaller, gray)
    neutrino/.style={
        circle,
        fill=gray,
        minimum size=4pt,
        inner sep=0pt,
    },
}

% ============================================================
% SPRING DECORATION (for mechanical models)
% ============================================================
\tikzset{
    spring/.style={
        thick,
        decorate,
        decoration={
            coil,
            aspect=0.5,
            segment length=2mm,
            amplitude=2mm,
        },
    },
    % Wave decoration (for field modes)
    wave field/.style={
        thick,
        decorate,
        decoration={
            snake,
            amplitude=2pt,
            segment length=8pt,
        },
    },
}

% ============================================================
% BOUNDARY STYLES
% ============================================================
\tikzset{
    % Bulk-facing boundary (dashed red)
    bulk boundary/.style={
        very thick,
        red!70!black,
        dashed,
    },
    % Observer-facing boundary (solid green)
    observer boundary/.style={
        thick,
        green!50!black,
    },
    % Brane edge (orange)
    brane edge/.style={
        thick,
        orange!70!black,
    },
}

% ============================================================
% CONVENIENCE COMMANDS
% ============================================================
% Arrow label (above)
\newcommand{\arrlabel}[1]{\scriptsize #1}
% Arrow label (below)
\newcommand{\arrlabelb}[1]{\scriptsize #1}

% ============================================================
% END OF STYLE FILE
% ============================================================


% ============================================================
% REQUIRED LIBRARIES (must be loaded in main document)
% ============================================================
% \usetikzlibrary{calc,angles,quotes,decorations.markings,decorations.pathmorphing,positioning}

% ============================================================
% POSITIONING DEFAULTS
% ============================================================
\tikzset{
    % Default node distances for horizontal/vertical layouts
    edc node distance/.style={node distance=1.6cm and 2.0cm},
    % Compact variant for dense diagrams
    edc compact/.style={node distance=1.2cm and 1.5cm},
    % Spread variant for clarity
    edc spread/.style={node distance=2.0cm and 2.5cm},
}

% ============================================================
% COLOR PALETTE (consistent with epistemic tags)
% ============================================================
\definecolor{edcBulk}{RGB}{220,50,50}        % Red tones for bulk/5D
\definecolor{edcBrane}{RGB}{50,150,50}       % Green tones for brane-layer
\definecolor{edcOutput}{RGB}{50,100,200}     % Blue tones for 3D outputs
\definecolor{edcNeutral}{RGB}{100,100,100}   % Gray for neutral/annotations

% ============================================================
% BOX STYLES
% ============================================================
\tikzset{
    % Generic EDC box (base style)
    edc box/.style={
        rectangle,
        draw,
        rounded corners=3pt,
        minimum width=2.2cm,
        minimum height=0.8cm,
        align=center,
        font=\small,
        inner sep=4pt,
    },
    % Bulk-core box (red family)
    bulk box/.style={
        edc box,
        fill=red!10,
        draw=edcBulk!70!black,
        text=black,
    },
    % Brane-layer box (green family)
    brane box/.style={
        edc box,
        fill=green!10,
        draw=edcBrane!70!black,
        text=black,
    },
    % 3D output box (blue family)
    output box/.style={
        edc box,
        fill=blue!10,
        draw=edcOutput!70!black,
        text=black,
    },
    % Neutral/process box
    process box/.style={
        edc box,
        fill=gray!10,
        draw=gray!60!black,
        text=black,
    },
    % Label-only box (no background)
    label box/.style={
        rectangle,
        rounded corners=2pt,
        draw=gray!40,
        fill=white,
        inner sep=2pt,
        font=\scriptsize,
    },
}

% ============================================================
% ARROW STYLES
% ============================================================
\tikzset{
    % Standard thick arrow
    edc arrow/.style={
        ->,
        >=stealth,
        thick,
    },
    % Emphasized arrow (for main flow)
    edc flow/.style={
        ->,
        >=stealth,
        very thick,
        line width=1.2pt,
    },
    % Dashed arrow (for optional/weak connections)
    edc dashed/.style={
        ->,
        >=stealth,
        thick,
        dashed,
    },
    % Double arrow (for bidirectional)
    edc bidir/.style={
        <->,
        >=stealth,
        thick,
    },
}

% ============================================================
% REGION STYLES (for background fills)
% ============================================================
\tikzset{
    % Bulk region (5D)
    bulk region/.style={
        fill=blue!8,
    },
    % Brane layer region
    brane region/.style={
        fill=yellow!25,
    },
    % Observer/3D region
    observer region/.style={
        fill=green!8,
    },
}

% ============================================================
% LABEL STYLES
% ============================================================
\tikzset{
    % Phase label (below nodes)
    phase label/.style={
        font=\scriptsize\itshape,
        text=black!70,
    },
    % Equation label (for inline math)
    eq label/.style={
        font=\scriptsize,
        fill=white,
        inner sep=1pt,
    },
    % Section annotation
    section label/.style={
        font=\footnotesize\bfseries,
        text=black,
    },
}

% ============================================================
% JUNCTION/PARTICLE STYLES
% ============================================================
\tikzset{
    % Y-junction point
    junction point/.style={
        circle,
        fill=red!60!black,
        minimum size=4pt,
        inner sep=0pt,
    },
    % Flux tube arm
    flux arm/.style={
        thick,
        blue!60!black,
    },
    % Particle dot (electron, etc.)
    particle/.style={
        circle,
        fill=black,
        minimum size=5pt,
        inner sep=0pt,
    },
    % Neutrino (smaller, gray)
    neutrino/.style={
        circle,
        fill=gray,
        minimum size=4pt,
        inner sep=0pt,
    },
}

% ============================================================
% SPRING DECORATION (for mechanical models)
% ============================================================
\tikzset{
    spring/.style={
        thick,
        decorate,
        decoration={
            coil,
            aspect=0.5,
            segment length=2mm,
            amplitude=2mm,
        },
    },
    % Wave decoration (for field modes)
    wave field/.style={
        thick,
        decorate,
        decoration={
            snake,
            amplitude=2pt,
            segment length=8pt,
        },
    },
}

% ============================================================
% BOUNDARY STYLES
% ============================================================
\tikzset{
    % Bulk-facing boundary (dashed red)
    bulk boundary/.style={
        very thick,
        red!70!black,
        dashed,
    },
    % Observer-facing boundary (solid green)
    observer boundary/.style={
        thick,
        green!50!black,
    },
    % Brane edge (orange)
    brane edge/.style={
        thick,
        orange!70!black,
    },
}

% ============================================================
% CONVENIENCE COMMANDS
% ============================================================
% Arrow label (above)
\newcommand{\arrlabel}[1]{\scriptsize #1}
% Arrow label (below)
\newcommand{\arrlabelb}[1]{\scriptsize #1}

% ============================================================
% END OF STYLE FILE
% ============================================================


% ============================================================
% REQUIRED LIBRARIES (must be loaded in main document)
% ============================================================
% \usetikzlibrary{calc,angles,quotes,decorations.markings,decorations.pathmorphing,positioning}

% ============================================================
% POSITIONING DEFAULTS
% ============================================================
\tikzset{
    % Default node distances for horizontal/vertical layouts
    edc node distance/.style={node distance=1.6cm and 2.0cm},
    % Compact variant for dense diagrams
    edc compact/.style={node distance=1.2cm and 1.5cm},
    % Spread variant for clarity
    edc spread/.style={node distance=2.0cm and 2.5cm},
}

% ============================================================
% COLOR PALETTE (consistent with epistemic tags)
% ============================================================
\definecolor{edcBulk}{RGB}{220,50,50}        % Red tones for bulk/5D
\definecolor{edcBrane}{RGB}{50,150,50}       % Green tones for brane-layer
\definecolor{edcOutput}{RGB}{50,100,200}     % Blue tones for 3D outputs
\definecolor{edcNeutral}{RGB}{100,100,100}   % Gray for neutral/annotations

% ============================================================
% BOX STYLES
% ============================================================
\tikzset{
    % Generic EDC box (base style)
    edc box/.style={
        rectangle,
        draw,
        rounded corners=3pt,
        minimum width=2.2cm,
        minimum height=0.8cm,
        align=center,
        font=\small,
        inner sep=4pt,
    },
    % Bulk-core box (red family)
    bulk box/.style={
        edc box,
        fill=red!10,
        draw=edcBulk!70!black,
        text=black,
    },
    % Brane-layer box (green family)
    brane box/.style={
        edc box,
        fill=green!10,
        draw=edcBrane!70!black,
        text=black,
    },
    % 3D output box (blue family)
    output box/.style={
        edc box,
        fill=blue!10,
        draw=edcOutput!70!black,
        text=black,
    },
    % Neutral/process box
    process box/.style={
        edc box,
        fill=gray!10,
        draw=gray!60!black,
        text=black,
    },
    % Label-only box (no background)
    label box/.style={
        rectangle,
        rounded corners=2pt,
        draw=gray!40,
        fill=white,
        inner sep=2pt,
        font=\scriptsize,
    },
}

% ============================================================
% ARROW STYLES
% ============================================================
\tikzset{
    % Standard thick arrow
    edc arrow/.style={
        ->,
        >=stealth,
        thick,
    },
    % Emphasized arrow (for main flow)
    edc flow/.style={
        ->,
        >=stealth,
        very thick,
        line width=1.2pt,
    },
    % Dashed arrow (for optional/weak connections)
    edc dashed/.style={
        ->,
        >=stealth,
        thick,
        dashed,
    },
    % Double arrow (for bidirectional)
    edc bidir/.style={
        <->,
        >=stealth,
        thick,
    },
}

% ============================================================
% REGION STYLES (for background fills)
% ============================================================
\tikzset{
    % Bulk region (5D)
    bulk region/.style={
        fill=blue!8,
    },
    % Brane layer region
    brane region/.style={
        fill=yellow!25,
    },
    % Observer/3D region
    observer region/.style={
        fill=green!8,
    },
}

% ============================================================
% LABEL STYLES
% ============================================================
\tikzset{
    % Phase label (below nodes)
    phase label/.style={
        font=\scriptsize\itshape,
        text=black!70,
    },
    % Equation label (for inline math)
    eq label/.style={
        font=\scriptsize,
        fill=white,
        inner sep=1pt,
    },
    % Section annotation
    section label/.style={
        font=\footnotesize\bfseries,
        text=black,
    },
}

% ============================================================
% JUNCTION/PARTICLE STYLES
% ============================================================
\tikzset{
    % Y-junction point
    junction point/.style={
        circle,
        fill=red!60!black,
        minimum size=4pt,
        inner sep=0pt,
    },
    % Flux tube arm
    flux arm/.style={
        thick,
        blue!60!black,
    },
    % Particle dot (electron, etc.)
    particle/.style={
        circle,
        fill=black,
        minimum size=5pt,
        inner sep=0pt,
    },
    % Neutrino (smaller, gray)
    neutrino/.style={
        circle,
        fill=gray,
        minimum size=4pt,
        inner sep=0pt,
    },
}

% ============================================================
% SPRING DECORATION (for mechanical models)
% ============================================================
\tikzset{
    spring/.style={
        thick,
        decorate,
        decoration={
            coil,
            aspect=0.5,
            segment length=2mm,
            amplitude=2mm,
        },
    },
    % Wave decoration (for field modes)
    wave field/.style={
        thick,
        decorate,
        decoration={
            snake,
            amplitude=2pt,
            segment length=8pt,
        },
    },
}

% ============================================================
% BOUNDARY STYLES
% ============================================================
\tikzset{
    % Bulk-facing boundary (dashed red)
    bulk boundary/.style={
        very thick,
        red!70!black,
        dashed,
    },
    % Observer-facing boundary (solid green)
    observer boundary/.style={
        thick,
        green!50!black,
    },
    % Brane edge (orange)
    brane edge/.style={
        thick,
        orange!70!black,
    },
}

% ============================================================
% CONVENIENCE COMMANDS
% ============================================================
% Arrow label (above)
\newcommand{\arrlabel}[1]{\scriptsize #1}
% Arrow label (below)
\newcommand{\arrlabelb}[1]{\scriptsize #1}

% ============================================================
% END OF STYLE FILE
% ============================================================


% ─────────────────────────────────────────────────────────────────────────────
% Additional packages
% ─────────────────────────────────────────────────────────────────────────────
\usepackage{booktabs}
\usepackage{array}
\usepackage{hyperref}

% ─────────────────────────────────────────────────────────────────────────────
% Document metadata
% ─────────────────────────────────────────────────────────────────────────────
\title{%
  \textbf{Companion V: Neutrino as Boundary/Edge Mode}\\[0.5em]
  \large Ledger Closure Without Bulk Escape in the\\
  EDC Weak Program
}
\author{%
  Igor Grčman\\
  \small Elastic Diffusive Cosmology Collaboration
}
\date{January 2026 \quad v0.1}

% ─────────────────────────────────────────────────────────────────────────────
\begin{document}
% ─────────────────────────────────────────────────────────────────────────────

\maketitle

\begin{abstract}
This companion document establishes the neutrino's ontological status within
the EDC framework: a boundary/edge excitation localized at the interface
between bulk and brane layers. Unlike bulk-core particles (proton) or
observer-facing defects (electron), the neutrino resides in the
\emph{interfacial zone} and carries ledger information (spin, helicity, lepton
number) without requiring ``escape into the bulk.'' We replace the informal
notion of ``weakly interacting because it enters the bulk'' with the precise
statement: \textbf{leakage suppressed by boundary conditions} \tagP{}. The
neutrino's role as ledger-closure partner in weak decays is systematically
treated.
\end{abstract}

% ==============================================================================
\section{Introduction and Motivation}
\label{sec:intro}
% ==============================================================================

In the EDC Weak Program, every decay process closes a \emph{ledger}: energy,
momentum, charge, and quantum numbers must balance across bulk, brane, and
3D outputs. The neutrino plays a unique role: it carries away ``missing''
quantities without being detected as a charged particle.

The Standard Model treats neutrinos as fundamental fermions with extremely
small cross-sections. EDC reinterprets this:

\begin{tcolorbox}[edcCornerstone, title={Core Reinterpretation [P]}]
The neutrino is not ``weakly interacting because it escapes into the bulk.''
Instead, it is a \textbf{boundary/edge mode}---an excitation localized at the
bulk-brane interface whose coupling to observer-facing states is
\textbf{suppressed by boundary conditions}.
\end{tcolorbox}

This reinterpretation has advantages:
\begin{enumerate}[nosep]
  \item No need for ``leakage into 5D'' (which would violate 4D energy
        conservation from the observer perspective)
  \item Explains helicity/chirality structure via boundary conditions
  \item Provides a natural ledger-closure mechanism
\end{enumerate}

\begin{tcolorbox}[edcGuardrail, title={Scope Guardrail}]
\begin{itemize}[nosep]
  \item We do \textbf{not} derive neutrino masses (mass origin \tagOpen{})
  \item We do \textbf{not} address neutrino oscillations in this document
        (\tagOpen{})
  \item We \textbf{do} explain the neutrino's structural role in ledger
        closure and why it appears ``invisible'' to charged-current
        interactions after emission
\end{itemize}
\end{tcolorbox}

% ==============================================================================
\section{Neutrino Ontology: Edge Mode Definition}
\label{sec:ontology}
% ==============================================================================

\subsection{The Interfacial Zone}

Recall from Companion L and Framework v2.0 that the thick brane has three
conceptual layers. The neutrino resides in a fourth structural element: the
\emph{interface} between bulk and brane.

\begin{definition}[Interfacial Zone {\normalfont [Def]}]
\label{def:interface}
The \textbf{interfacial zone} is the transitional region between the bulk
(5D) and the internal brane layer. It is characterized by:
\begin{enumerate}[nosep]
  \item Partial localization: modes here are neither fully bulk nor fully
        brane-bound
  \item Boundary-condition sensitivity: excitations must satisfy matching
        conditions between bulk and brane
  \item Suppressed observer coupling: direct interaction with observer-facing
        layer is exponentially suppressed
\end{enumerate}
\end{definition}

\begin{figure}[ht]
\centering
% fig_neutrino_localization.tex — Neutrino localization in interfacial zone
% Uses styles from tikz_style_edc.tex

\begin{tikzpicture}[scale=0.85, every node/.style={transform shape}]

  % Background regions
  \fill[bulk region] (-4,-2) rectangle (-1.5,2);
  \fill[brane region] (-1.5,-2) rectangle (1.5,2);
  \fill[observer region] (1.5,-2) rectangle (4,2);

  % Interfacial zone (highlighted)
  \fill[yellow!40] (-1.5,-2) rectangle (-0.5,2);

  % Region labels
  \node[section label] at (-2.75,2.4) {\textbf{5D Bulk}};
  \node[section label] at (-1,2.4) {\footnotesize\textbf{Interface}};
  \node[section label] at (0.5,2.4) {\textbf{Brane}};
  \node[section label] at (2.75,2.4) {\textbf{3D Observer}};

  % Boundaries
  \draw[bulk boundary] (-1.5,-2) -- (-1.5,2);
  \draw[brane edge] (-0.5,-2) -- (-0.5,2);
  \draw[observer boundary] (1.5,-2) -- (1.5,2);

  % Proton junction (extends into bulk)
  \node[junction point] (pj) at (-2,-0.5) {};
  \draw[flux arm] (pj) -- ++(120:0.5);
  \draw[flux arm] (pj) -- ++(240:0.5);
  \draw[flux arm] (pj) -- ++(0:0.5);
  \node[font=\scriptsize, below=0.2cm of pj] {proton};

  % Neutrino (in interfacial zone)
  \node[neutrino, fill=purple!70] (nu) at (-1,0.8) {};
  \node[font=\scriptsize, right=0.1cm of nu] {$\bar\nu_e$};

  % Electron (observer-facing)
  \node[particle, fill=blue!70!black] (elec) at (1,0.8) {};
  \node[font=\scriptsize, right=0.1cm of elec] {$e^-$};

  % y-axis indicator
  \draw[->, thick] (-3.5,-1.8) -- (-3.5,-0.8);
  \node[font=\scriptsize] at (-3.5,-1.3) [left] {$y$};

  % Suppressed coupling arrows
  \draw[edc dashed, purple!60!black] (nu) -- ++(-0.6,0);
  \draw[edc dashed, purple!60!black] (nu) -- ++(0.8,0);
  \node[font=\tiny, text width=1.5cm] at (-0.2,-0.5) {suppressed\\coupling};

  % Key
  \node[label box, text width=3.2cm] at (2.75,-1) {
    \footnotesize\textbf{Edge mode:} $\nu$ at interface,\\
    suppressed leakage to bulk\\
    and observer-facing layers
  };

\end{tikzpicture}

\caption{Neutrino localization in the interfacial zone. Unlike the electron
(observer-facing) or proton junction (bulk-extending), the neutrino $\nu$
resides at the bulk-brane interface, with suppressed coupling to both sides.}
\label{fig:neutrino_loc}
\end{figure}

\subsection{Edge Mode Characterization}

\begin{postulate}[Neutrino as Edge Mode {\normalfont [P]/[Def]}]
\label{post:neutrino}
The neutrino is an \textbf{edge mode}---a boundary excitation that:
\begin{enumerate}[nosep]
  \item Carries conserved quantum numbers (lepton number, spin, helicity)
  \item Is localized at the bulk-brane interface (not inside bulk, not on
        observer-facing layer)
  \item Has \textbf{suppressed leakage} into both bulk and observer-facing
        regions due to boundary conditions
\end{enumerate}
\end{postulate}

\edcMechanismNote{Bulk-brane energy exchange creates interface excitation}%
                 {Boundary conditions localize mode at interface; chirality filter selects helicity}%
                 {Neutrino carries ledger balance (spin, momentum, lepton number) away from decay vertex}

\textbf{Key distinction from ``bulk escape.''}
The neutrino does not ``escape into the 5th dimension.'' It remains
interface-localized. Its weak interaction with 3D matter arises from
\emph{suppressed coupling} across the interface, not from being ``elsewhere.''

% ==============================================================================
\section{Ledger Role: What the Neutrino Carries}
\label{sec:ledger}
% ==============================================================================

In every weak decay, the neutrino (or antineutrino) closes the conservation
ledger:

\begin{table}[ht]
\centering
\caption{Neutrino ledger contributions in representative decays}
\label{tab:ledger}
\begin{tabular}{lcccc}
\toprule
\textbf{Decay} & \textbf{Energy} & \textbf{Momentum} & \textbf{Spin} & \textbf{Lepton \#} \\
\midrule
$n \to p + e^- + \bar\nu_e$ & $E_\nu$ & $\vec{p}_\nu$ & $+1/2$ (RH) & $-1$ \\
$\mu^- \to e^- + \bar\nu_e + \nu_\mu$ & $E_{\nu_\mu}$, $E_{\bar\nu_e}$ & balanced & mixed & $0$ (net) \\
$\pi^- \to \mu^- + \bar\nu_\mu$ & $E_{\bar\nu_\mu}$ & $\vec{p}_{\bar\nu}$ & $+1/2$ (RH) & $-1$ \\
\bottomrule
\end{tabular}
\end{table}

\subsection{Chirality and Helicity}

The boundary conditions at the bulk-brane interface impose a \textbf{chirality
filter}:

\begin{definition}[Chirality Filter {\normalfont [P]/[Dc]}]
\label{def:chiral}
The brane boundary conditions select:
\begin{itemize}[nosep]
  \item \textbf{Left-handed} charged leptons ($e^-_L$, $\mu^-_L$, $\tau^-_L$)
  \item \textbf{Right-handed} antineutrinos ($\bar\nu_R$)
  \item \textbf{Left-handed} neutrinos ($\nu_L$)
\end{itemize}
In the massless limit, helicity equals chirality. For massive particles,
the boundary condition selects chirality; helicity follows approximately.
\end{definition}

This is consistent with the observed V$-$A structure of weak interactions
\tagBL{}.

\begin{figure}[ht]
\centering
% fig_chirality_filter.tex — Chirality filter as boundary-condition projection
% Uses styles from tikz_style_edc.tex

\begin{tikzpicture}[edc node distance, scale=0.85, every node/.style={transform shape}]

  % Input modes
  \node[bulk box, text width=2.4cm] (input) at (0,0) {
    \textbf{Brane Modes}\\[2pt]
    \footnotesize $\psi_L$, $\psi_R$\\
    (all chiralities)
  };

  % BC filter
  \node[process box, text width=2.8cm, right=2cm of input] (filter) {
    \textbf{Chirality Filter}\\[2pt]
    \footnotesize $\mathcal{P}_{\mathrm{chir}}$\\
    (boundary conditions)
  };

  % Allowed outputs (split)
  \node[output box, text width=2.2cm, right=2cm of filter, yshift=0.8cm] (lept) {
    \textbf{Charged lepton}\\
    \footnotesize $\ell^-_L$ (left-handed)
  };

  \node[output box, text width=2.2cm, right=2cm of filter, yshift=-0.8cm] (nu) {
    \textbf{Antineutrino}\\
    \footnotesize $\bar\nu_R$ (right-handed)
  };

  % Suppressed modes
  \node[label box, text width=2.0cm, below=1.5cm of filter] (forbid) {
    \footnotesize $\ell^-_R$, $\bar\nu_L$\\
    \textbf{suppressed}\\
    by BC
  };

  % Arrows
  \draw[edc flow] (input) -- node[above, font=\scriptsize] {all modes} (filter);
  \draw[edc arrow] (filter) -- node[above, font=\scriptsize, sloped] {select L} (lept);
  \draw[edc arrow] (filter) -- node[below, font=\scriptsize, sloped] {select R} (nu);
  \draw[edc dashed, red!60!black] (filter) -- (forbid);

  % V-A output annotation
  \node[label box, text width=3.5cm, right=0.5cm of nu, yshift=-0.3cm] (va) {
    \footnotesize\textbf{Result:} V$-$A structure\\
    $\bar\psi_\ell \gamma^\mu (1-\gamma^5) \psi_\nu$
  };
  \draw[gray, dashed] (nu.east) -- (va.west);

\end{tikzpicture}

\caption{Chirality filter as boundary-condition projection. The brane BC
selects left-handed charged leptons and right-handed antineutrinos as allowed
outputs.}
\label{fig:chiral}
\end{figure}

\subsection{Why ``Suppressed Leakage'' Instead of ``Cannot Escape''}

\begin{tcolorbox}[edcGuardrail, title={Language Precision}]
\textbf{Avoid:} ``The neutrino cannot interact because it escapes into the
bulk.''

\textbf{Use:} ``The neutrino's coupling to observer-facing states is
\textbf{suppressed by boundary conditions} at the interface.'' \tagP{}

This avoids implying energy loss to extra dimensions (which would violate
observed 4D conservation).
\end{tcolorbox}

The suppression mechanism:
\begin{enumerate}[nosep]
  \item Neutrino is interface-localized (edge mode)
  \item Observer-facing layer is separated by internal brane layer
  \item Coupling across layers is exponentially suppressed by wavefunction
        overlap \tagP{}
  \item Result: neutrino appears ``invisible'' after emission
\end{enumerate}

% ==============================================================================
\section{Connection to Weak Companions}
\label{sec:connections}
% ==============================================================================

The neutrino edge-mode picture is consistent with all Weak Program companions:

\begin{itemize}[nosep]
  \item \textbf{Companion N} (Neutron): $\bar\nu_e$ carries away lepton number
        $-1$, momentum, and part of $Q_\beta$
  \item \textbf{Companion M} (Muon): two neutrinos ($\nu_\mu$, $\bar\nu_e$)
        share the energy spectrum; interference requires both to be edge modes
  \item \textbf{Companion P} (Pion): $\bar\nu_\mu$ enables helicity suppression
        by carrying opposite helicity to $\mu^-$
  \item \textbf{Companion L} (Electron): neutrino completes the ledger,
        allowing electron selection
\end{itemize}

% ==============================================================================
\section{Falsifiability and Open Questions}
\label{sec:falsify}
% ==============================================================================

\begin{tcolorbox}[edcWarning, title={Falsifiability Handles}]
The neutrino-as-edge-mode hypothesis would be \textbf{challenged} if:
\begin{enumerate}[nosep]
  \item Neutrino were detected with wrong-sign helicity at appreciable rate
        (would require BC modification)
  \item Ledger closure failed: missing energy not accountable to neutrino
        spectrum
  \item Neutrino showed bulk-like propagation (different dispersion relation
        at high energy)
  \item Sterile neutrino mixing violated interface localization picture
\end{enumerate}
Current experimental data are consistent with EDC edge-mode predictions
\tagBL{}.
\end{tcolorbox}

\begin{table}[ht]
\centering
\caption{Open questions and observable handles}
\label{tab:open}
\begin{tabular}{p{5.5cm}p{6cm}}
\toprule
\textbf{Open Question} & \textbf{Observable Handle} \\
\midrule
Origin of neutrino masses & Oscillation parameters, cosmological bounds
\tagOpen{} \\
Why three flavors & Mode spectrum from interface geometry \tagOpen{} \\
Dirac vs Majorana nature & Neutrinoless double-beta decay \tagBL{} \\
Sterile neutrino coupling & Short-baseline oscillation anomalies \tagBL{} \\
\bottomrule
\end{tabular}
\end{table}

% ==============================================================================
\section{The Chiral Filter: Minimal Formalization}
\label{sec:chiral_filter}
% ==============================================================================

\subsection{Boundary Condition Operator}

We define a projection operator $\mathcal{P}_{\mathrm{chir}}$ that encodes the
chirality selection:

\begin{definition}[Chiral Projection Operator {\normalfont [P]/[Def]}]
\label{def:pchir}
The operator $\mathcal{P}_{\mathrm{chir}}$ acts on fermion modes at the brane
boundary:
\begin{align}
  \mathcal{P}_{\mathrm{chir}} \psi_{\ell^-} &= \psi_{L} \quad
    \text{(left-handed charged lepton)} \\
  \mathcal{P}_{\mathrm{chir}} \psi_{\bar\nu} &= \psi_{R} \quad
    \text{(right-handed antineutrino)}
\end{align}
where $L/R$ refer to chirality eigenstates.
\end{definition}

\textbf{V$-$A consistency.}
The operator $\mathcal{P}_{\mathrm{chir}}$ is postulated \tagP{} to arise from
the boundary conditions at the bulk-brane interface. It produces V$-$A
structure as an \emph{output}, not an input assumption:
\begin{equation}
  \mathcal{J}^\mu_{\mathrm{weak}} \propto \bar\psi_{\ell,L} \gamma^\mu \psi_{\nu,L}
  = \bar\psi_\ell \gamma^\mu (1 - \gamma^5) \psi_\nu / 2
\end{equation}
which is the standard V$-$A current \tagBL{}.

\subsection{What Would Falsify This}

\begin{tcolorbox}[edcWarning, title={Chiral Filter Falsifiability}]
The chiral filter hypothesis would be \textbf{falsified} if:
\begin{enumerate}[nosep]
  \item Right-handed $W$ bosons discovered with SM-like coupling
  \item Parity violation in weak decays showed energy-dependent deviation
  \item Charged-current interactions showed V$+$A component at any scale
\end{enumerate}
None of these have been observed \tagBL{}.
\end{tcolorbox}

% ==============================================================================
\section{Summary}
\label{sec:summary}
% ==============================================================================

\begin{tcolorbox}[edcCornerstone, title={Companion V Summary}]
\begin{enumerate}[nosep]
  \item The neutrino is an \textbf{edge mode} at the bulk-brane interface
        \tagP{}/\tagDef{}
  \item It carries \textbf{ledger information} (energy, momentum, spin,
        lepton number) \tagDc{}
  \item Its weak interaction arises from \textbf{suppressed leakage}, not
        ``bulk escape'' \tagP{}
  \item The \textbf{chirality filter} $\mathcal{P}_{\mathrm{chir}}$ selects
        L-handed leptons, R-handed antineutrinos \tagP{}
  \item This reproduces V$-$A structure as an \textbf{output} of boundary
        conditions \tagDc{}
  \item Mass origin and flavor structure remain \tagOpen{}
\end{enumerate}
\end{tcolorbox}

% ==============================================================================
% Related Documents
% ==============================================================================
\vspace{1em}
\hrule
\vspace{0.5em}
\footnotesize
\textbf{Related Documents:}\\
Framework v2.0 (DOI: \href{https://doi.org/10.5281/zenodo.18299085}{10.5281/zenodo.18299085}) $\cdot$
Paper 3 NJSR (DOI: \href{https://doi.org/10.5281/zenodo.18262721}{10.5281/zenodo.18262721})\\
Companion N (DOI: \href{https://doi.org/10.5281/zenodo.18315110}{10.5281/zenodo.18315110}) $\cdot$
Companion L (Electron) — this series\\
Weak Program Overview (DOI: \href{https://doi.org/10.5281/zenodo.18319921}{10.5281/zenodo.18319921})
\normalsize

% ==============================================================================
\end{document}
% ==============================================================================
