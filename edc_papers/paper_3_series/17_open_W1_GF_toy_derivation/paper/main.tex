% ==============================================================================
% OPEN-W1: Toy Derivation of G_F Scaling from Thick-Brane Mediator
% Version: 0.1 (Derivation Skeleton)
% Date: 2026-01-20
% Status: OPEN (structural pathway, not numerical derivation)
% DOI: 10.5281/zenodo.18321396
% ==============================================================================

\documentclass[11pt,a4paper]{article}

% ─────────────────────────────────────────────────────────────────────────────
% Required packages BEFORE shared style
% ─────────────────────────────────────────────────────────────────────────────
\usepackage{amsmath,amssymb}
\usepackage{enumitem}
\usepackage{tikz}
\usetikzlibrary{positioning,arrows.meta,shapes.geometric,calc}
\usepackage{tcolorbox}
\tcbuselibrary{skins,breakable}

% ─────────────────────────────────────────────────────────────────────────────
% Shared EDC style
% ─────────────────────────────────────────────────────────────────────────────
% edc_style.tex — Canonical EDC Paper Style for Paper 3 Series
% Version 1.0 — 2026-01-20
%
% USAGE: Include in preamble AFTER loading packages but BEFORE \begin{document}
%   % edc_style.tex — Canonical EDC Paper Style for Paper 3 Series
% Version 1.0 — 2026-01-20
%
% USAGE: Include in preamble AFTER loading packages but BEFORE \begin{document}
%   % edc_style.tex — Canonical EDC Paper Style for Paper 3 Series
% Version 1.0 — 2026-01-20
%
% USAGE: Include in preamble AFTER loading packages but BEFORE \begin{document}
%   \input{../_shared/style/edc_style}
%   \input{../_shared/style/tikz_style_edc}  % if using TikZ figures
%
% REQUIRED PACKAGES (load these in main document before \input):
%   fontspec, amsmath, amssymb, amsthm, mathtools, geometry
%   hyperref, enumitem, booktabs, array, xcolor, tcolorbox
%
% ============================================================

% ============================================================
%  EPISTEMIC TAG COLORS
% ============================================================
\definecolor{tagDer}{RGB}{0,128,0}      % Green - Derived
\definecolor{tagDc}{RGB}{0,0,200}       % Blue - Deduced/Constrained
\definecolor{tagCal}{RGB}{200,0,0}      % Red - Calibrated
\definecolor{tagP}{RGB}{128,0,128}      % Purple - Postulated
\definecolor{tagBL}{RGB}{128,128,128}   % Gray - Baseline
\definecolor{tagI}{RGB}{255,140,0}      % Orange - Identified
\definecolor{tagOpen}{RGB}{200,100,0}   % Dark orange - Open

% ============================================================
%  EPISTEMIC TAG COMMANDS
% ============================================================
% Use these to mark claims with their epistemic status
\newcommand{\tagDer}{\textcolor{tagDer}{\textbf{[Der]}}}    % Derived from axioms
\newcommand{\tagDc}{\textcolor{tagDc}{\textbf{[Dc]}}}       % Deduced/Constrained
\newcommand{\tagCal}{\textcolor{tagCal}{\textbf{[Cal]}}}    % Calibrated (fitted)
\newcommand{\tagP}{\textcolor{tagP}{\textbf{[P]}}}          % Postulated
\newcommand{\tagBL}{\textcolor{tagBL}{\textbf{[BL]}}}       % Baseline (external fact)
\newcommand{\tagI}{\textcolor{tagI}{\textbf{[I]}}}          % Identified (pattern match)
\newcommand{\tagOpen}{\textcolor{tagOpen}{\textbf{[OPEN]}}} % Open problem
\newcommand{\tagDef}{\textcolor{tagDc}{\textbf{[Def]}}}     % Definition

% ============================================================
%  THEOREM ENVIRONMENTS
% ============================================================
\newtheorem{postulate}{Postulate}
\newtheorem{definition}{Definition}[section]
\newtheorem{theorem}{Theorem}[section]
\newtheorem{lemma}[theorem]{Lemma}
\newtheorem{corollary}[theorem]{Corollary}
\newtheorem{proposition}[theorem]{Proposition}
\newtheorem{remark}{Remark}[section]

% ============================================================
%  COMMON EDC SYMBOLS
% ============================================================
% Symmetry groups
\newcommand{\Ztwo}{\mathbb{Z}_2}
\newcommand{\Zthree}{\mathbb{Z}_3}
\newcommand{\Ztri}{\mathbb{Z}_3}    % alias
\newcommand{\Zsix}{\mathbb{Z}_6}

% Geometric objects
\newcommand{\Sthree}{S^3}           % 3-sphere
\newcommand{\Stwo}{S^2}             % 2-sphere
\newcommand{\Bthree}{B^3}           % 3-ball
\newcommand{\Mfive}{\mathcal{M}_5}  % 5D manifold
\newcommand{\Bfour}{\mathcal{B}_4}  % 4D brane

% Physical quantities
\newcommand{\tension}{\tau}         % string/flux-tube tension (E/L)
\newcommand{\re}{r_e}               % electron radius

% Operators
\newcommand{\Pfrozen}{\mathcal{P}_{\mathrm{frozen}}}  % Frozen projection operator
\newcommand{\Ebrane}{\mathcal{E}_{\mathrm{brane}}}    % Brane energy store

% Bulk-brane exchange current (canonical notation from Framework v2.0)
\newcommand{\Jbb}[1]{J^{#1}_{\mathrm{bulk}\to\mathrm{brane}}}

% ============================================================
%  TCOLORBOX STYLES FOR EDC PAPERS
% ============================================================
% Cornerstone box (blue) — key claims/foundations
\tcbset{
    edcCornerstone/.style={
        colback=blue!5,
        colframe=blue!40!black,
        fonttitle=\bfseries
    }
}

% Guardrail box (gray) — epistemic warnings/constraints
\tcbset{
    edcGuardrail/.style={
        colback=gray!5!white,
        colframe=gray!60!black,
        fonttitle=\bfseries
    }
}

% PPN box (blue, lighter) — Physical Process Narrative
\tcbset{
    edcPPN/.style={
        colback=blue!5,
        colframe=blue!50!black,
        fonttitle=\bfseries
    }
}

% Canonical box (yellow/orange) — canonical definitions/glossary
\tcbset{
    edcCanonical/.style={
        colback=yellow!5,
        colframe=orange!60!black,
        fonttitle=\bfseries
    }
}

% Conceptual box (yellow/orange, lighter) — conceptual pictures
\tcbset{
    edcConcept/.style={
        colback=yellow!5,
        colframe=orange!50!black,
        fonttitle=\bfseries
    }
}

% Pathway box (purple) — energy pathways, mechanisms
\tcbset{
    edcPathway/.style={
        colback=purple!5,
        colframe=purple!40!black,
        fonttitle=\bfseries
    }
}

% Model box (green) — mechanical analogies, heuristics
\tcbset{
    edcModel/.style={
        colback=green!5,
        colframe=green!40!black,
        fonttitle=\bfseries
    }
}

% Warning box (red) — non-overclaim, limitations
\tcbset{
    edcWarning/.style={
        colback=red!5,
        colframe=red!40!black,
        fonttitle=\bfseries
    }
}

% Framework quote box (gray) — verbatim from Framework v2.0
\tcbset{
    edcFramework/.style={
        colback=gray!5!white,
        colframe=gray!60!black,
        fonttitle=\small
    }
}

% Mechanism box (teal) — mechanistic dimension principle narrative
\tcbset{
    edcMechanism/.style={
        colback=teal!5,
        colframe=teal!50!black,
        fonttitle=\bfseries,
        title={Mechanistic Dimension Note (Canon)}
    }
}

% ============================================================
%  MECHANISTIC DIMENSION HELPER MACRO
% ============================================================
% Usage: \edcMechanismNote{bulk cause}{brane process}{3D output}
%
% Example:
%   \edcMechanismNote{Junction relaxes toward Steiner minimum}%
%                    {Energy pumps into brane-layer modes, redistributes}%
%                    {Electron, antineutrino, proton emerge on 3D side}
%
\newcommand{\edcMechanismNote}[3]{%
\begin{tcolorbox}[edcMechanism]
\begin{itemize}[nosep,leftmargin=*]
    \item \textbf{5D cause (bulk):} #1
    \item \textbf{Brane-layer process:} #2
    \item \textbf{3D observation (output):} #3
\end{itemize}
\vspace{0.3em}
\footnotesize\textit{Ledger closure must hold: bulk + brane + 3D outputs conserve energy/quantum numbers.}
\end{tcolorbox}
}

% ============================================================
%  RELATED DOCUMENTS MACRO
% ============================================================
% Usage: \edcRelatedDocs{main paper title}{main DOI}{companion list}
%
% Example:
%     A: \emph{Effective Lagrangian} (\href{...}{DOI}) $\cdot$
%     B: \emph{WKB Prefactor} (\href{...}{DOI})
%   }

% NOTE: \edcRelatedDocs macro deprecated (DOI registry consolidated)
% Use consolidated Zenodo article as primary reference instead.

% ============================================================
%  DOI REGISTRY DEPRECATED
% ============================================================
% Previous individual DOIs have been deprecated.
% All EDC Weak Sector content is now consolidated into a single
% Zenodo article. See paper_3_series/19_edc_weak_sector_zenodo_article/

% ============================================================
%  PHYSICAL NARRATION RULE REMINDER
% ============================================================
% Every key equation MUST be accompanied by a physical narrative stating:
%   1. 5D cause: What changes in the bulk-core configuration?
%   2. Brane response: How does the brane absorb/redistribute energy?
%   3. 3D observable output: What do observers detect on the 3D side?
%
% This rule eliminates "numerology smell" by ensuring every formula
% has a mechanistic interpretation.

% ============================================================
%  END OF STYLE FILE
% ============================================================

%   % tikz_style_edc.tex — Reusable TikZ styles for EDC papers
% Version 1.0 — 2026-01-20
% Include via: \input{tikz_style_edc}

% ============================================================
% REQUIRED LIBRARIES (must be loaded in main document)
% ============================================================
% \usetikzlibrary{calc,angles,quotes,decorations.markings,decorations.pathmorphing,positioning}

% ============================================================
% POSITIONING DEFAULTS
% ============================================================
\tikzset{
    % Default node distances for horizontal/vertical layouts
    edc node distance/.style={node distance=1.6cm and 2.0cm},
    % Compact variant for dense diagrams
    edc compact/.style={node distance=1.2cm and 1.5cm},
    % Spread variant for clarity
    edc spread/.style={node distance=2.0cm and 2.5cm},
}

% ============================================================
% COLOR PALETTE (consistent with epistemic tags)
% ============================================================
\definecolor{edcBulk}{RGB}{220,50,50}        % Red tones for bulk/5D
\definecolor{edcBrane}{RGB}{50,150,50}       % Green tones for brane-layer
\definecolor{edcOutput}{RGB}{50,100,200}     % Blue tones for 3D outputs
\definecolor{edcNeutral}{RGB}{100,100,100}   % Gray for neutral/annotations

% ============================================================
% BOX STYLES
% ============================================================
\tikzset{
    % Generic EDC box (base style)
    edc box/.style={
        rectangle,
        draw,
        rounded corners=3pt,
        minimum width=2.2cm,
        minimum height=0.8cm,
        align=center,
        font=\small,
        inner sep=4pt,
    },
    % Bulk-core box (red family)
    bulk box/.style={
        edc box,
        fill=red!10,
        draw=edcBulk!70!black,
        text=black,
    },
    % Brane-layer box (green family)
    brane box/.style={
        edc box,
        fill=green!10,
        draw=edcBrane!70!black,
        text=black,
    },
    % 3D output box (blue family)
    output box/.style={
        edc box,
        fill=blue!10,
        draw=edcOutput!70!black,
        text=black,
    },
    % Neutral/process box
    process box/.style={
        edc box,
        fill=gray!10,
        draw=gray!60!black,
        text=black,
    },
    % Label-only box (no background)
    label box/.style={
        rectangle,
        rounded corners=2pt,
        draw=gray!40,
        fill=white,
        inner sep=2pt,
        font=\scriptsize,
    },
}

% ============================================================
% ARROW STYLES
% ============================================================
\tikzset{
    % Standard thick arrow
    edc arrow/.style={
        ->,
        >=stealth,
        thick,
    },
    % Emphasized arrow (for main flow)
    edc flow/.style={
        ->,
        >=stealth,
        very thick,
        line width=1.2pt,
    },
    % Dashed arrow (for optional/weak connections)
    edc dashed/.style={
        ->,
        >=stealth,
        thick,
        dashed,
    },
    % Double arrow (for bidirectional)
    edc bidir/.style={
        <->,
        >=stealth,
        thick,
    },
}

% ============================================================
% REGION STYLES (for background fills)
% ============================================================
\tikzset{
    % Bulk region (5D)
    bulk region/.style={
        fill=blue!8,
    },
    % Brane layer region
    brane region/.style={
        fill=yellow!25,
    },
    % Observer/3D region
    observer region/.style={
        fill=green!8,
    },
}

% ============================================================
% LABEL STYLES
% ============================================================
\tikzset{
    % Phase label (below nodes)
    phase label/.style={
        font=\scriptsize\itshape,
        text=black!70,
    },
    % Equation label (for inline math)
    eq label/.style={
        font=\scriptsize,
        fill=white,
        inner sep=1pt,
    },
    % Section annotation
    section label/.style={
        font=\footnotesize\bfseries,
        text=black,
    },
}

% ============================================================
% JUNCTION/PARTICLE STYLES
% ============================================================
\tikzset{
    % Y-junction point
    junction point/.style={
        circle,
        fill=red!60!black,
        minimum size=4pt,
        inner sep=0pt,
    },
    % Flux tube arm
    flux arm/.style={
        thick,
        blue!60!black,
    },
    % Particle dot (electron, etc.)
    particle/.style={
        circle,
        fill=black,
        minimum size=5pt,
        inner sep=0pt,
    },
    % Neutrino (smaller, gray)
    neutrino/.style={
        circle,
        fill=gray,
        minimum size=4pt,
        inner sep=0pt,
    },
}

% ============================================================
% SPRING DECORATION (for mechanical models)
% ============================================================
\tikzset{
    spring/.style={
        thick,
        decorate,
        decoration={
            coil,
            aspect=0.5,
            segment length=2mm,
            amplitude=2mm,
        },
    },
    % Wave decoration (for field modes)
    wave field/.style={
        thick,
        decorate,
        decoration={
            snake,
            amplitude=2pt,
            segment length=8pt,
        },
    },
}

% ============================================================
% BOUNDARY STYLES
% ============================================================
\tikzset{
    % Bulk-facing boundary (dashed red)
    bulk boundary/.style={
        very thick,
        red!70!black,
        dashed,
    },
    % Observer-facing boundary (solid green)
    observer boundary/.style={
        thick,
        green!50!black,
    },
    % Brane edge (orange)
    brane edge/.style={
        thick,
        orange!70!black,
    },
}

% ============================================================
% CONVENIENCE COMMANDS
% ============================================================
% Arrow label (above)
\newcommand{\arrlabel}[1]{\scriptsize #1}
% Arrow label (below)
\newcommand{\arrlabelb}[1]{\scriptsize #1}

% ============================================================
% END OF STYLE FILE
% ============================================================
  % if using TikZ figures
%
% REQUIRED PACKAGES (load these in main document before \input):
%   fontspec, amsmath, amssymb, amsthm, mathtools, geometry
%   hyperref, enumitem, booktabs, array, xcolor, tcolorbox
%
% ============================================================

% ============================================================
%  EPISTEMIC TAG COLORS
% ============================================================
\definecolor{tagDer}{RGB}{0,128,0}      % Green - Derived
\definecolor{tagDc}{RGB}{0,0,200}       % Blue - Deduced/Constrained
\definecolor{tagCal}{RGB}{200,0,0}      % Red - Calibrated
\definecolor{tagP}{RGB}{128,0,128}      % Purple - Postulated
\definecolor{tagBL}{RGB}{128,128,128}   % Gray - Baseline
\definecolor{tagI}{RGB}{255,140,0}      % Orange - Identified
\definecolor{tagOpen}{RGB}{200,100,0}   % Dark orange - Open

% ============================================================
%  EPISTEMIC TAG COMMANDS
% ============================================================
% Use these to mark claims with their epistemic status
\newcommand{\tagDer}{\textcolor{tagDer}{\textbf{[Der]}}}    % Derived from axioms
\newcommand{\tagDc}{\textcolor{tagDc}{\textbf{[Dc]}}}       % Deduced/Constrained
\newcommand{\tagCal}{\textcolor{tagCal}{\textbf{[Cal]}}}    % Calibrated (fitted)
\newcommand{\tagP}{\textcolor{tagP}{\textbf{[P]}}}          % Postulated
\newcommand{\tagBL}{\textcolor{tagBL}{\textbf{[BL]}}}       % Baseline (external fact)
\newcommand{\tagI}{\textcolor{tagI}{\textbf{[I]}}}          % Identified (pattern match)
\newcommand{\tagOpen}{\textcolor{tagOpen}{\textbf{[OPEN]}}} % Open problem
\newcommand{\tagDef}{\textcolor{tagDc}{\textbf{[Def]}}}     % Definition

% ============================================================
%  THEOREM ENVIRONMENTS
% ============================================================
\newtheorem{postulate}{Postulate}
\newtheorem{definition}{Definition}[section]
\newtheorem{theorem}{Theorem}[section]
\newtheorem{lemma}[theorem]{Lemma}
\newtheorem{corollary}[theorem]{Corollary}
\newtheorem{proposition}[theorem]{Proposition}
\newtheorem{remark}{Remark}[section]

% ============================================================
%  COMMON EDC SYMBOLS
% ============================================================
% Symmetry groups
\newcommand{\Ztwo}{\mathbb{Z}_2}
\newcommand{\Zthree}{\mathbb{Z}_3}
\newcommand{\Ztri}{\mathbb{Z}_3}    % alias
\newcommand{\Zsix}{\mathbb{Z}_6}

% Geometric objects
\newcommand{\Sthree}{S^3}           % 3-sphere
\newcommand{\Stwo}{S^2}             % 2-sphere
\newcommand{\Bthree}{B^3}           % 3-ball
\newcommand{\Mfive}{\mathcal{M}_5}  % 5D manifold
\newcommand{\Bfour}{\mathcal{B}_4}  % 4D brane

% Physical quantities
\newcommand{\tension}{\tau}         % string/flux-tube tension (E/L)
\newcommand{\re}{r_e}               % electron radius

% Operators
\newcommand{\Pfrozen}{\mathcal{P}_{\mathrm{frozen}}}  % Frozen projection operator
\newcommand{\Ebrane}{\mathcal{E}_{\mathrm{brane}}}    % Brane energy store

% Bulk-brane exchange current (canonical notation from Framework v2.0)
\newcommand{\Jbb}[1]{J^{#1}_{\mathrm{bulk}\to\mathrm{brane}}}

% ============================================================
%  TCOLORBOX STYLES FOR EDC PAPERS
% ============================================================
% Cornerstone box (blue) — key claims/foundations
\tcbset{
    edcCornerstone/.style={
        colback=blue!5,
        colframe=blue!40!black,
        fonttitle=\bfseries
    }
}

% Guardrail box (gray) — epistemic warnings/constraints
\tcbset{
    edcGuardrail/.style={
        colback=gray!5!white,
        colframe=gray!60!black,
        fonttitle=\bfseries
    }
}

% PPN box (blue, lighter) — Physical Process Narrative
\tcbset{
    edcPPN/.style={
        colback=blue!5,
        colframe=blue!50!black,
        fonttitle=\bfseries
    }
}

% Canonical box (yellow/orange) — canonical definitions/glossary
\tcbset{
    edcCanonical/.style={
        colback=yellow!5,
        colframe=orange!60!black,
        fonttitle=\bfseries
    }
}

% Conceptual box (yellow/orange, lighter) — conceptual pictures
\tcbset{
    edcConcept/.style={
        colback=yellow!5,
        colframe=orange!50!black,
        fonttitle=\bfseries
    }
}

% Pathway box (purple) — energy pathways, mechanisms
\tcbset{
    edcPathway/.style={
        colback=purple!5,
        colframe=purple!40!black,
        fonttitle=\bfseries
    }
}

% Model box (green) — mechanical analogies, heuristics
\tcbset{
    edcModel/.style={
        colback=green!5,
        colframe=green!40!black,
        fonttitle=\bfseries
    }
}

% Warning box (red) — non-overclaim, limitations
\tcbset{
    edcWarning/.style={
        colback=red!5,
        colframe=red!40!black,
        fonttitle=\bfseries
    }
}

% Framework quote box (gray) — verbatim from Framework v2.0
\tcbset{
    edcFramework/.style={
        colback=gray!5!white,
        colframe=gray!60!black,
        fonttitle=\small
    }
}

% Mechanism box (teal) — mechanistic dimension principle narrative
\tcbset{
    edcMechanism/.style={
        colback=teal!5,
        colframe=teal!50!black,
        fonttitle=\bfseries,
        title={Mechanistic Dimension Note (Canon)}
    }
}

% ============================================================
%  MECHANISTIC DIMENSION HELPER MACRO
% ============================================================
% Usage: \edcMechanismNote{bulk cause}{brane process}{3D output}
%
% Example:
%   \edcMechanismNote{Junction relaxes toward Steiner minimum}%
%                    {Energy pumps into brane-layer modes, redistributes}%
%                    {Electron, antineutrino, proton emerge on 3D side}
%
\newcommand{\edcMechanismNote}[3]{%
\begin{tcolorbox}[edcMechanism]
\begin{itemize}[nosep,leftmargin=*]
    \item \textbf{5D cause (bulk):} #1
    \item \textbf{Brane-layer process:} #2
    \item \textbf{3D observation (output):} #3
\end{itemize}
\vspace{0.3em}
\footnotesize\textit{Ledger closure must hold: bulk + brane + 3D outputs conserve energy/quantum numbers.}
\end{tcolorbox}
}

% ============================================================
%  RELATED DOCUMENTS MACRO
% ============================================================
% Usage: \edcRelatedDocs{main paper title}{main DOI}{companion list}
%
% Example:
%     A: \emph{Effective Lagrangian} (\href{...}{DOI}) $\cdot$
%     B: \emph{WKB Prefactor} (\href{...}{DOI})
%   }

% NOTE: \edcRelatedDocs macro deprecated (DOI registry consolidated)
% Use consolidated Zenodo article as primary reference instead.

% ============================================================
%  DOI REGISTRY DEPRECATED
% ============================================================
% Previous individual DOIs have been deprecated.
% All EDC Weak Sector content is now consolidated into a single
% Zenodo article. See paper_3_series/19_edc_weak_sector_zenodo_article/

% ============================================================
%  PHYSICAL NARRATION RULE REMINDER
% ============================================================
% Every key equation MUST be accompanied by a physical narrative stating:
%   1. 5D cause: What changes in the bulk-core configuration?
%   2. Brane response: How does the brane absorb/redistribute energy?
%   3. 3D observable output: What do observers detect on the 3D side?
%
% This rule eliminates "numerology smell" by ensuring every formula
% has a mechanistic interpretation.

% ============================================================
%  END OF STYLE FILE
% ============================================================

%   % tikz_style_edc.tex — Reusable TikZ styles for EDC papers
% Version 1.0 — 2026-01-20
% Include via: % tikz_style_edc.tex — Reusable TikZ styles for EDC papers
% Version 1.0 — 2026-01-20
% Include via: \input{tikz_style_edc}

% ============================================================
% REQUIRED LIBRARIES (must be loaded in main document)
% ============================================================
% \usetikzlibrary{calc,angles,quotes,decorations.markings,decorations.pathmorphing,positioning}

% ============================================================
% POSITIONING DEFAULTS
% ============================================================
\tikzset{
    % Default node distances for horizontal/vertical layouts
    edc node distance/.style={node distance=1.6cm and 2.0cm},
    % Compact variant for dense diagrams
    edc compact/.style={node distance=1.2cm and 1.5cm},
    % Spread variant for clarity
    edc spread/.style={node distance=2.0cm and 2.5cm},
}

% ============================================================
% COLOR PALETTE (consistent with epistemic tags)
% ============================================================
\definecolor{edcBulk}{RGB}{220,50,50}        % Red tones for bulk/5D
\definecolor{edcBrane}{RGB}{50,150,50}       % Green tones for brane-layer
\definecolor{edcOutput}{RGB}{50,100,200}     % Blue tones for 3D outputs
\definecolor{edcNeutral}{RGB}{100,100,100}   % Gray for neutral/annotations

% ============================================================
% BOX STYLES
% ============================================================
\tikzset{
    % Generic EDC box (base style)
    edc box/.style={
        rectangle,
        draw,
        rounded corners=3pt,
        minimum width=2.2cm,
        minimum height=0.8cm,
        align=center,
        font=\small,
        inner sep=4pt,
    },
    % Bulk-core box (red family)
    bulk box/.style={
        edc box,
        fill=red!10,
        draw=edcBulk!70!black,
        text=black,
    },
    % Brane-layer box (green family)
    brane box/.style={
        edc box,
        fill=green!10,
        draw=edcBrane!70!black,
        text=black,
    },
    % 3D output box (blue family)
    output box/.style={
        edc box,
        fill=blue!10,
        draw=edcOutput!70!black,
        text=black,
    },
    % Neutral/process box
    process box/.style={
        edc box,
        fill=gray!10,
        draw=gray!60!black,
        text=black,
    },
    % Label-only box (no background)
    label box/.style={
        rectangle,
        rounded corners=2pt,
        draw=gray!40,
        fill=white,
        inner sep=2pt,
        font=\scriptsize,
    },
}

% ============================================================
% ARROW STYLES
% ============================================================
\tikzset{
    % Standard thick arrow
    edc arrow/.style={
        ->,
        >=stealth,
        thick,
    },
    % Emphasized arrow (for main flow)
    edc flow/.style={
        ->,
        >=stealth,
        very thick,
        line width=1.2pt,
    },
    % Dashed arrow (for optional/weak connections)
    edc dashed/.style={
        ->,
        >=stealth,
        thick,
        dashed,
    },
    % Double arrow (for bidirectional)
    edc bidir/.style={
        <->,
        >=stealth,
        thick,
    },
}

% ============================================================
% REGION STYLES (for background fills)
% ============================================================
\tikzset{
    % Bulk region (5D)
    bulk region/.style={
        fill=blue!8,
    },
    % Brane layer region
    brane region/.style={
        fill=yellow!25,
    },
    % Observer/3D region
    observer region/.style={
        fill=green!8,
    },
}

% ============================================================
% LABEL STYLES
% ============================================================
\tikzset{
    % Phase label (below nodes)
    phase label/.style={
        font=\scriptsize\itshape,
        text=black!70,
    },
    % Equation label (for inline math)
    eq label/.style={
        font=\scriptsize,
        fill=white,
        inner sep=1pt,
    },
    % Section annotation
    section label/.style={
        font=\footnotesize\bfseries,
        text=black,
    },
}

% ============================================================
% JUNCTION/PARTICLE STYLES
% ============================================================
\tikzset{
    % Y-junction point
    junction point/.style={
        circle,
        fill=red!60!black,
        minimum size=4pt,
        inner sep=0pt,
    },
    % Flux tube arm
    flux arm/.style={
        thick,
        blue!60!black,
    },
    % Particle dot (electron, etc.)
    particle/.style={
        circle,
        fill=black,
        minimum size=5pt,
        inner sep=0pt,
    },
    % Neutrino (smaller, gray)
    neutrino/.style={
        circle,
        fill=gray,
        minimum size=4pt,
        inner sep=0pt,
    },
}

% ============================================================
% SPRING DECORATION (for mechanical models)
% ============================================================
\tikzset{
    spring/.style={
        thick,
        decorate,
        decoration={
            coil,
            aspect=0.5,
            segment length=2mm,
            amplitude=2mm,
        },
    },
    % Wave decoration (for field modes)
    wave field/.style={
        thick,
        decorate,
        decoration={
            snake,
            amplitude=2pt,
            segment length=8pt,
        },
    },
}

% ============================================================
% BOUNDARY STYLES
% ============================================================
\tikzset{
    % Bulk-facing boundary (dashed red)
    bulk boundary/.style={
        very thick,
        red!70!black,
        dashed,
    },
    % Observer-facing boundary (solid green)
    observer boundary/.style={
        thick,
        green!50!black,
    },
    % Brane edge (orange)
    brane edge/.style={
        thick,
        orange!70!black,
    },
}

% ============================================================
% CONVENIENCE COMMANDS
% ============================================================
% Arrow label (above)
\newcommand{\arrlabel}[1]{\scriptsize #1}
% Arrow label (below)
\newcommand{\arrlabelb}[1]{\scriptsize #1}

% ============================================================
% END OF STYLE FILE
% ============================================================


% ============================================================
% REQUIRED LIBRARIES (must be loaded in main document)
% ============================================================
% \usetikzlibrary{calc,angles,quotes,decorations.markings,decorations.pathmorphing,positioning}

% ============================================================
% POSITIONING DEFAULTS
% ============================================================
\tikzset{
    % Default node distances for horizontal/vertical layouts
    edc node distance/.style={node distance=1.6cm and 2.0cm},
    % Compact variant for dense diagrams
    edc compact/.style={node distance=1.2cm and 1.5cm},
    % Spread variant for clarity
    edc spread/.style={node distance=2.0cm and 2.5cm},
}

% ============================================================
% COLOR PALETTE (consistent with epistemic tags)
% ============================================================
\definecolor{edcBulk}{RGB}{220,50,50}        % Red tones for bulk/5D
\definecolor{edcBrane}{RGB}{50,150,50}       % Green tones for brane-layer
\definecolor{edcOutput}{RGB}{50,100,200}     % Blue tones for 3D outputs
\definecolor{edcNeutral}{RGB}{100,100,100}   % Gray for neutral/annotations

% ============================================================
% BOX STYLES
% ============================================================
\tikzset{
    % Generic EDC box (base style)
    edc box/.style={
        rectangle,
        draw,
        rounded corners=3pt,
        minimum width=2.2cm,
        minimum height=0.8cm,
        align=center,
        font=\small,
        inner sep=4pt,
    },
    % Bulk-core box (red family)
    bulk box/.style={
        edc box,
        fill=red!10,
        draw=edcBulk!70!black,
        text=black,
    },
    % Brane-layer box (green family)
    brane box/.style={
        edc box,
        fill=green!10,
        draw=edcBrane!70!black,
        text=black,
    },
    % 3D output box (blue family)
    output box/.style={
        edc box,
        fill=blue!10,
        draw=edcOutput!70!black,
        text=black,
    },
    % Neutral/process box
    process box/.style={
        edc box,
        fill=gray!10,
        draw=gray!60!black,
        text=black,
    },
    % Label-only box (no background)
    label box/.style={
        rectangle,
        rounded corners=2pt,
        draw=gray!40,
        fill=white,
        inner sep=2pt,
        font=\scriptsize,
    },
}

% ============================================================
% ARROW STYLES
% ============================================================
\tikzset{
    % Standard thick arrow
    edc arrow/.style={
        ->,
        >=stealth,
        thick,
    },
    % Emphasized arrow (for main flow)
    edc flow/.style={
        ->,
        >=stealth,
        very thick,
        line width=1.2pt,
    },
    % Dashed arrow (for optional/weak connections)
    edc dashed/.style={
        ->,
        >=stealth,
        thick,
        dashed,
    },
    % Double arrow (for bidirectional)
    edc bidir/.style={
        <->,
        >=stealth,
        thick,
    },
}

% ============================================================
% REGION STYLES (for background fills)
% ============================================================
\tikzset{
    % Bulk region (5D)
    bulk region/.style={
        fill=blue!8,
    },
    % Brane layer region
    brane region/.style={
        fill=yellow!25,
    },
    % Observer/3D region
    observer region/.style={
        fill=green!8,
    },
}

% ============================================================
% LABEL STYLES
% ============================================================
\tikzset{
    % Phase label (below nodes)
    phase label/.style={
        font=\scriptsize\itshape,
        text=black!70,
    },
    % Equation label (for inline math)
    eq label/.style={
        font=\scriptsize,
        fill=white,
        inner sep=1pt,
    },
    % Section annotation
    section label/.style={
        font=\footnotesize\bfseries,
        text=black,
    },
}

% ============================================================
% JUNCTION/PARTICLE STYLES
% ============================================================
\tikzset{
    % Y-junction point
    junction point/.style={
        circle,
        fill=red!60!black,
        minimum size=4pt,
        inner sep=0pt,
    },
    % Flux tube arm
    flux arm/.style={
        thick,
        blue!60!black,
    },
    % Particle dot (electron, etc.)
    particle/.style={
        circle,
        fill=black,
        minimum size=5pt,
        inner sep=0pt,
    },
    % Neutrino (smaller, gray)
    neutrino/.style={
        circle,
        fill=gray,
        minimum size=4pt,
        inner sep=0pt,
    },
}

% ============================================================
% SPRING DECORATION (for mechanical models)
% ============================================================
\tikzset{
    spring/.style={
        thick,
        decorate,
        decoration={
            coil,
            aspect=0.5,
            segment length=2mm,
            amplitude=2mm,
        },
    },
    % Wave decoration (for field modes)
    wave field/.style={
        thick,
        decorate,
        decoration={
            snake,
            amplitude=2pt,
            segment length=8pt,
        },
    },
}

% ============================================================
% BOUNDARY STYLES
% ============================================================
\tikzset{
    % Bulk-facing boundary (dashed red)
    bulk boundary/.style={
        very thick,
        red!70!black,
        dashed,
    },
    % Observer-facing boundary (solid green)
    observer boundary/.style={
        thick,
        green!50!black,
    },
    % Brane edge (orange)
    brane edge/.style={
        thick,
        orange!70!black,
    },
}

% ============================================================
% CONVENIENCE COMMANDS
% ============================================================
% Arrow label (above)
\newcommand{\arrlabel}[1]{\scriptsize #1}
% Arrow label (below)
\newcommand{\arrlabelb}[1]{\scriptsize #1}

% ============================================================
% END OF STYLE FILE
% ============================================================
  % if using TikZ figures
%
% REQUIRED PACKAGES (load these in main document before \input):
%   fontspec, amsmath, amssymb, amsthm, mathtools, geometry
%   hyperref, enumitem, booktabs, array, xcolor, tcolorbox
%
% ============================================================

% ============================================================
%  EPISTEMIC TAG COLORS
% ============================================================
\definecolor{tagDer}{RGB}{0,128,0}      % Green - Derived
\definecolor{tagDc}{RGB}{0,0,200}       % Blue - Deduced/Constrained
\definecolor{tagCal}{RGB}{200,0,0}      % Red - Calibrated
\definecolor{tagP}{RGB}{128,0,128}      % Purple - Postulated
\definecolor{tagBL}{RGB}{128,128,128}   % Gray - Baseline
\definecolor{tagI}{RGB}{255,140,0}      % Orange - Identified
\definecolor{tagOpen}{RGB}{200,100,0}   % Dark orange - Open

% ============================================================
%  EPISTEMIC TAG COMMANDS
% ============================================================
% Use these to mark claims with their epistemic status
\newcommand{\tagDer}{\textcolor{tagDer}{\textbf{[Der]}}}    % Derived from axioms
\newcommand{\tagDc}{\textcolor{tagDc}{\textbf{[Dc]}}}       % Deduced/Constrained
\newcommand{\tagCal}{\textcolor{tagCal}{\textbf{[Cal]}}}    % Calibrated (fitted)
\newcommand{\tagP}{\textcolor{tagP}{\textbf{[P]}}}          % Postulated
\newcommand{\tagBL}{\textcolor{tagBL}{\textbf{[BL]}}}       % Baseline (external fact)
\newcommand{\tagI}{\textcolor{tagI}{\textbf{[I]}}}          % Identified (pattern match)
\newcommand{\tagOpen}{\textcolor{tagOpen}{\textbf{[OPEN]}}} % Open problem
\newcommand{\tagDef}{\textcolor{tagDc}{\textbf{[Def]}}}     % Definition

% ============================================================
%  THEOREM ENVIRONMENTS
% ============================================================
\newtheorem{postulate}{Postulate}
\newtheorem{definition}{Definition}[section]
\newtheorem{theorem}{Theorem}[section]
\newtheorem{lemma}[theorem]{Lemma}
\newtheorem{corollary}[theorem]{Corollary}
\newtheorem{proposition}[theorem]{Proposition}
\newtheorem{remark}{Remark}[section]

% ============================================================
%  COMMON EDC SYMBOLS
% ============================================================
% Symmetry groups
\newcommand{\Ztwo}{\mathbb{Z}_2}
\newcommand{\Zthree}{\mathbb{Z}_3}
\newcommand{\Ztri}{\mathbb{Z}_3}    % alias
\newcommand{\Zsix}{\mathbb{Z}_6}

% Geometric objects
\newcommand{\Sthree}{S^3}           % 3-sphere
\newcommand{\Stwo}{S^2}             % 2-sphere
\newcommand{\Bthree}{B^3}           % 3-ball
\newcommand{\Mfive}{\mathcal{M}_5}  % 5D manifold
\newcommand{\Bfour}{\mathcal{B}_4}  % 4D brane

% Physical quantities
\newcommand{\tension}{\tau}         % string/flux-tube tension (E/L)
\newcommand{\re}{r_e}               % electron radius

% Operators
\newcommand{\Pfrozen}{\mathcal{P}_{\mathrm{frozen}}}  % Frozen projection operator
\newcommand{\Ebrane}{\mathcal{E}_{\mathrm{brane}}}    % Brane energy store

% Bulk-brane exchange current (canonical notation from Framework v2.0)
\newcommand{\Jbb}[1]{J^{#1}_{\mathrm{bulk}\to\mathrm{brane}}}

% ============================================================
%  TCOLORBOX STYLES FOR EDC PAPERS
% ============================================================
% Cornerstone box (blue) — key claims/foundations
\tcbset{
    edcCornerstone/.style={
        colback=blue!5,
        colframe=blue!40!black,
        fonttitle=\bfseries
    }
}

% Guardrail box (gray) — epistemic warnings/constraints
\tcbset{
    edcGuardrail/.style={
        colback=gray!5!white,
        colframe=gray!60!black,
        fonttitle=\bfseries
    }
}

% PPN box (blue, lighter) — Physical Process Narrative
\tcbset{
    edcPPN/.style={
        colback=blue!5,
        colframe=blue!50!black,
        fonttitle=\bfseries
    }
}

% Canonical box (yellow/orange) — canonical definitions/glossary
\tcbset{
    edcCanonical/.style={
        colback=yellow!5,
        colframe=orange!60!black,
        fonttitle=\bfseries
    }
}

% Conceptual box (yellow/orange, lighter) — conceptual pictures
\tcbset{
    edcConcept/.style={
        colback=yellow!5,
        colframe=orange!50!black,
        fonttitle=\bfseries
    }
}

% Pathway box (purple) — energy pathways, mechanisms
\tcbset{
    edcPathway/.style={
        colback=purple!5,
        colframe=purple!40!black,
        fonttitle=\bfseries
    }
}

% Model box (green) — mechanical analogies, heuristics
\tcbset{
    edcModel/.style={
        colback=green!5,
        colframe=green!40!black,
        fonttitle=\bfseries
    }
}

% Warning box (red) — non-overclaim, limitations
\tcbset{
    edcWarning/.style={
        colback=red!5,
        colframe=red!40!black,
        fonttitle=\bfseries
    }
}

% Framework quote box (gray) — verbatim from Framework v2.0
\tcbset{
    edcFramework/.style={
        colback=gray!5!white,
        colframe=gray!60!black,
        fonttitle=\small
    }
}

% Mechanism box (teal) — mechanistic dimension principle narrative
\tcbset{
    edcMechanism/.style={
        colback=teal!5,
        colframe=teal!50!black,
        fonttitle=\bfseries,
        title={Mechanistic Dimension Note (Canon)}
    }
}

% ============================================================
%  MECHANISTIC DIMENSION HELPER MACRO
% ============================================================
% Usage: \edcMechanismNote{bulk cause}{brane process}{3D output}
%
% Example:
%   \edcMechanismNote{Junction relaxes toward Steiner minimum}%
%                    {Energy pumps into brane-layer modes, redistributes}%
%                    {Electron, antineutrino, proton emerge on 3D side}
%
\newcommand{\edcMechanismNote}[3]{%
\begin{tcolorbox}[edcMechanism]
\begin{itemize}[nosep,leftmargin=*]
    \item \textbf{5D cause (bulk):} #1
    \item \textbf{Brane-layer process:} #2
    \item \textbf{3D observation (output):} #3
\end{itemize}
\vspace{0.3em}
\footnotesize\textit{Ledger closure must hold: bulk + brane + 3D outputs conserve energy/quantum numbers.}
\end{tcolorbox}
}

% ============================================================
%  RELATED DOCUMENTS MACRO
% ============================================================
% Usage: \edcRelatedDocs{main paper title}{main DOI}{companion list}
%
% Example:
%     A: \emph{Effective Lagrangian} (\href{...}{DOI}) $\cdot$
%     B: \emph{WKB Prefactor} (\href{...}{DOI})
%   }

% NOTE: \edcRelatedDocs macro deprecated (DOI registry consolidated)
% Use consolidated Zenodo article as primary reference instead.

% ============================================================
%  DOI REGISTRY DEPRECATED
% ============================================================
% Previous individual DOIs have been deprecated.
% All EDC Weak Sector content is now consolidated into a single
% Zenodo article. See paper_3_series/19_edc_weak_sector_zenodo_article/

% ============================================================
%  PHYSICAL NARRATION RULE REMINDER
% ============================================================
% Every key equation MUST be accompanied by a physical narrative stating:
%   1. 5D cause: What changes in the bulk-core configuration?
%   2. Brane response: How does the brane absorb/redistribute energy?
%   3. 3D observable output: What do observers detect on the 3D side?
%
% This rule eliminates "numerology smell" by ensuring every formula
% has a mechanistic interpretation.

% ============================================================
%  END OF STYLE FILE
% ============================================================

% tikz_style_edc.tex — Reusable TikZ styles for EDC papers
% Version 1.0 — 2026-01-20
% Include via: % tikz_style_edc.tex — Reusable TikZ styles for EDC papers
% Version 1.0 — 2026-01-20
% Include via: % tikz_style_edc.tex — Reusable TikZ styles for EDC papers
% Version 1.0 — 2026-01-20
% Include via: \input{tikz_style_edc}

% ============================================================
% REQUIRED LIBRARIES (must be loaded in main document)
% ============================================================
% \usetikzlibrary{calc,angles,quotes,decorations.markings,decorations.pathmorphing,positioning}

% ============================================================
% POSITIONING DEFAULTS
% ============================================================
\tikzset{
    % Default node distances for horizontal/vertical layouts
    edc node distance/.style={node distance=1.6cm and 2.0cm},
    % Compact variant for dense diagrams
    edc compact/.style={node distance=1.2cm and 1.5cm},
    % Spread variant for clarity
    edc spread/.style={node distance=2.0cm and 2.5cm},
}

% ============================================================
% COLOR PALETTE (consistent with epistemic tags)
% ============================================================
\definecolor{edcBulk}{RGB}{220,50,50}        % Red tones for bulk/5D
\definecolor{edcBrane}{RGB}{50,150,50}       % Green tones for brane-layer
\definecolor{edcOutput}{RGB}{50,100,200}     % Blue tones for 3D outputs
\definecolor{edcNeutral}{RGB}{100,100,100}   % Gray for neutral/annotations

% ============================================================
% BOX STYLES
% ============================================================
\tikzset{
    % Generic EDC box (base style)
    edc box/.style={
        rectangle,
        draw,
        rounded corners=3pt,
        minimum width=2.2cm,
        minimum height=0.8cm,
        align=center,
        font=\small,
        inner sep=4pt,
    },
    % Bulk-core box (red family)
    bulk box/.style={
        edc box,
        fill=red!10,
        draw=edcBulk!70!black,
        text=black,
    },
    % Brane-layer box (green family)
    brane box/.style={
        edc box,
        fill=green!10,
        draw=edcBrane!70!black,
        text=black,
    },
    % 3D output box (blue family)
    output box/.style={
        edc box,
        fill=blue!10,
        draw=edcOutput!70!black,
        text=black,
    },
    % Neutral/process box
    process box/.style={
        edc box,
        fill=gray!10,
        draw=gray!60!black,
        text=black,
    },
    % Label-only box (no background)
    label box/.style={
        rectangle,
        rounded corners=2pt,
        draw=gray!40,
        fill=white,
        inner sep=2pt,
        font=\scriptsize,
    },
}

% ============================================================
% ARROW STYLES
% ============================================================
\tikzset{
    % Standard thick arrow
    edc arrow/.style={
        ->,
        >=stealth,
        thick,
    },
    % Emphasized arrow (for main flow)
    edc flow/.style={
        ->,
        >=stealth,
        very thick,
        line width=1.2pt,
    },
    % Dashed arrow (for optional/weak connections)
    edc dashed/.style={
        ->,
        >=stealth,
        thick,
        dashed,
    },
    % Double arrow (for bidirectional)
    edc bidir/.style={
        <->,
        >=stealth,
        thick,
    },
}

% ============================================================
% REGION STYLES (for background fills)
% ============================================================
\tikzset{
    % Bulk region (5D)
    bulk region/.style={
        fill=blue!8,
    },
    % Brane layer region
    brane region/.style={
        fill=yellow!25,
    },
    % Observer/3D region
    observer region/.style={
        fill=green!8,
    },
}

% ============================================================
% LABEL STYLES
% ============================================================
\tikzset{
    % Phase label (below nodes)
    phase label/.style={
        font=\scriptsize\itshape,
        text=black!70,
    },
    % Equation label (for inline math)
    eq label/.style={
        font=\scriptsize,
        fill=white,
        inner sep=1pt,
    },
    % Section annotation
    section label/.style={
        font=\footnotesize\bfseries,
        text=black,
    },
}

% ============================================================
% JUNCTION/PARTICLE STYLES
% ============================================================
\tikzset{
    % Y-junction point
    junction point/.style={
        circle,
        fill=red!60!black,
        minimum size=4pt,
        inner sep=0pt,
    },
    % Flux tube arm
    flux arm/.style={
        thick,
        blue!60!black,
    },
    % Particle dot (electron, etc.)
    particle/.style={
        circle,
        fill=black,
        minimum size=5pt,
        inner sep=0pt,
    },
    % Neutrino (smaller, gray)
    neutrino/.style={
        circle,
        fill=gray,
        minimum size=4pt,
        inner sep=0pt,
    },
}

% ============================================================
% SPRING DECORATION (for mechanical models)
% ============================================================
\tikzset{
    spring/.style={
        thick,
        decorate,
        decoration={
            coil,
            aspect=0.5,
            segment length=2mm,
            amplitude=2mm,
        },
    },
    % Wave decoration (for field modes)
    wave field/.style={
        thick,
        decorate,
        decoration={
            snake,
            amplitude=2pt,
            segment length=8pt,
        },
    },
}

% ============================================================
% BOUNDARY STYLES
% ============================================================
\tikzset{
    % Bulk-facing boundary (dashed red)
    bulk boundary/.style={
        very thick,
        red!70!black,
        dashed,
    },
    % Observer-facing boundary (solid green)
    observer boundary/.style={
        thick,
        green!50!black,
    },
    % Brane edge (orange)
    brane edge/.style={
        thick,
        orange!70!black,
    },
}

% ============================================================
% CONVENIENCE COMMANDS
% ============================================================
% Arrow label (above)
\newcommand{\arrlabel}[1]{\scriptsize #1}
% Arrow label (below)
\newcommand{\arrlabelb}[1]{\scriptsize #1}

% ============================================================
% END OF STYLE FILE
% ============================================================


% ============================================================
% REQUIRED LIBRARIES (must be loaded in main document)
% ============================================================
% \usetikzlibrary{calc,angles,quotes,decorations.markings,decorations.pathmorphing,positioning}

% ============================================================
% POSITIONING DEFAULTS
% ============================================================
\tikzset{
    % Default node distances for horizontal/vertical layouts
    edc node distance/.style={node distance=1.6cm and 2.0cm},
    % Compact variant for dense diagrams
    edc compact/.style={node distance=1.2cm and 1.5cm},
    % Spread variant for clarity
    edc spread/.style={node distance=2.0cm and 2.5cm},
}

% ============================================================
% COLOR PALETTE (consistent with epistemic tags)
% ============================================================
\definecolor{edcBulk}{RGB}{220,50,50}        % Red tones for bulk/5D
\definecolor{edcBrane}{RGB}{50,150,50}       % Green tones for brane-layer
\definecolor{edcOutput}{RGB}{50,100,200}     % Blue tones for 3D outputs
\definecolor{edcNeutral}{RGB}{100,100,100}   % Gray for neutral/annotations

% ============================================================
% BOX STYLES
% ============================================================
\tikzset{
    % Generic EDC box (base style)
    edc box/.style={
        rectangle,
        draw,
        rounded corners=3pt,
        minimum width=2.2cm,
        minimum height=0.8cm,
        align=center,
        font=\small,
        inner sep=4pt,
    },
    % Bulk-core box (red family)
    bulk box/.style={
        edc box,
        fill=red!10,
        draw=edcBulk!70!black,
        text=black,
    },
    % Brane-layer box (green family)
    brane box/.style={
        edc box,
        fill=green!10,
        draw=edcBrane!70!black,
        text=black,
    },
    % 3D output box (blue family)
    output box/.style={
        edc box,
        fill=blue!10,
        draw=edcOutput!70!black,
        text=black,
    },
    % Neutral/process box
    process box/.style={
        edc box,
        fill=gray!10,
        draw=gray!60!black,
        text=black,
    },
    % Label-only box (no background)
    label box/.style={
        rectangle,
        rounded corners=2pt,
        draw=gray!40,
        fill=white,
        inner sep=2pt,
        font=\scriptsize,
    },
}

% ============================================================
% ARROW STYLES
% ============================================================
\tikzset{
    % Standard thick arrow
    edc arrow/.style={
        ->,
        >=stealth,
        thick,
    },
    % Emphasized arrow (for main flow)
    edc flow/.style={
        ->,
        >=stealth,
        very thick,
        line width=1.2pt,
    },
    % Dashed arrow (for optional/weak connections)
    edc dashed/.style={
        ->,
        >=stealth,
        thick,
        dashed,
    },
    % Double arrow (for bidirectional)
    edc bidir/.style={
        <->,
        >=stealth,
        thick,
    },
}

% ============================================================
% REGION STYLES (for background fills)
% ============================================================
\tikzset{
    % Bulk region (5D)
    bulk region/.style={
        fill=blue!8,
    },
    % Brane layer region
    brane region/.style={
        fill=yellow!25,
    },
    % Observer/3D region
    observer region/.style={
        fill=green!8,
    },
}

% ============================================================
% LABEL STYLES
% ============================================================
\tikzset{
    % Phase label (below nodes)
    phase label/.style={
        font=\scriptsize\itshape,
        text=black!70,
    },
    % Equation label (for inline math)
    eq label/.style={
        font=\scriptsize,
        fill=white,
        inner sep=1pt,
    },
    % Section annotation
    section label/.style={
        font=\footnotesize\bfseries,
        text=black,
    },
}

% ============================================================
% JUNCTION/PARTICLE STYLES
% ============================================================
\tikzset{
    % Y-junction point
    junction point/.style={
        circle,
        fill=red!60!black,
        minimum size=4pt,
        inner sep=0pt,
    },
    % Flux tube arm
    flux arm/.style={
        thick,
        blue!60!black,
    },
    % Particle dot (electron, etc.)
    particle/.style={
        circle,
        fill=black,
        minimum size=5pt,
        inner sep=0pt,
    },
    % Neutrino (smaller, gray)
    neutrino/.style={
        circle,
        fill=gray,
        minimum size=4pt,
        inner sep=0pt,
    },
}

% ============================================================
% SPRING DECORATION (for mechanical models)
% ============================================================
\tikzset{
    spring/.style={
        thick,
        decorate,
        decoration={
            coil,
            aspect=0.5,
            segment length=2mm,
            amplitude=2mm,
        },
    },
    % Wave decoration (for field modes)
    wave field/.style={
        thick,
        decorate,
        decoration={
            snake,
            amplitude=2pt,
            segment length=8pt,
        },
    },
}

% ============================================================
% BOUNDARY STYLES
% ============================================================
\tikzset{
    % Bulk-facing boundary (dashed red)
    bulk boundary/.style={
        very thick,
        red!70!black,
        dashed,
    },
    % Observer-facing boundary (solid green)
    observer boundary/.style={
        thick,
        green!50!black,
    },
    % Brane edge (orange)
    brane edge/.style={
        thick,
        orange!70!black,
    },
}

% ============================================================
% CONVENIENCE COMMANDS
% ============================================================
% Arrow label (above)
\newcommand{\arrlabel}[1]{\scriptsize #1}
% Arrow label (below)
\newcommand{\arrlabelb}[1]{\scriptsize #1}

% ============================================================
% END OF STYLE FILE
% ============================================================


% ============================================================
% REQUIRED LIBRARIES (must be loaded in main document)
% ============================================================
% \usetikzlibrary{calc,angles,quotes,decorations.markings,decorations.pathmorphing,positioning}

% ============================================================
% POSITIONING DEFAULTS
% ============================================================
\tikzset{
    % Default node distances for horizontal/vertical layouts
    edc node distance/.style={node distance=1.6cm and 2.0cm},
    % Compact variant for dense diagrams
    edc compact/.style={node distance=1.2cm and 1.5cm},
    % Spread variant for clarity
    edc spread/.style={node distance=2.0cm and 2.5cm},
}

% ============================================================
% COLOR PALETTE (consistent with epistemic tags)
% ============================================================
\definecolor{edcBulk}{RGB}{220,50,50}        % Red tones for bulk/5D
\definecolor{edcBrane}{RGB}{50,150,50}       % Green tones for brane-layer
\definecolor{edcOutput}{RGB}{50,100,200}     % Blue tones for 3D outputs
\definecolor{edcNeutral}{RGB}{100,100,100}   % Gray for neutral/annotations

% ============================================================
% BOX STYLES
% ============================================================
\tikzset{
    % Generic EDC box (base style)
    edc box/.style={
        rectangle,
        draw,
        rounded corners=3pt,
        minimum width=2.2cm,
        minimum height=0.8cm,
        align=center,
        font=\small,
        inner sep=4pt,
    },
    % Bulk-core box (red family)
    bulk box/.style={
        edc box,
        fill=red!10,
        draw=edcBulk!70!black,
        text=black,
    },
    % Brane-layer box (green family)
    brane box/.style={
        edc box,
        fill=green!10,
        draw=edcBrane!70!black,
        text=black,
    },
    % 3D output box (blue family)
    output box/.style={
        edc box,
        fill=blue!10,
        draw=edcOutput!70!black,
        text=black,
    },
    % Neutral/process box
    process box/.style={
        edc box,
        fill=gray!10,
        draw=gray!60!black,
        text=black,
    },
    % Label-only box (no background)
    label box/.style={
        rectangle,
        rounded corners=2pt,
        draw=gray!40,
        fill=white,
        inner sep=2pt,
        font=\scriptsize,
    },
}

% ============================================================
% ARROW STYLES
% ============================================================
\tikzset{
    % Standard thick arrow
    edc arrow/.style={
        ->,
        >=stealth,
        thick,
    },
    % Emphasized arrow (for main flow)
    edc flow/.style={
        ->,
        >=stealth,
        very thick,
        line width=1.2pt,
    },
    % Dashed arrow (for optional/weak connections)
    edc dashed/.style={
        ->,
        >=stealth,
        thick,
        dashed,
    },
    % Double arrow (for bidirectional)
    edc bidir/.style={
        <->,
        >=stealth,
        thick,
    },
}

% ============================================================
% REGION STYLES (for background fills)
% ============================================================
\tikzset{
    % Bulk region (5D)
    bulk region/.style={
        fill=blue!8,
    },
    % Brane layer region
    brane region/.style={
        fill=yellow!25,
    },
    % Observer/3D region
    observer region/.style={
        fill=green!8,
    },
}

% ============================================================
% LABEL STYLES
% ============================================================
\tikzset{
    % Phase label (below nodes)
    phase label/.style={
        font=\scriptsize\itshape,
        text=black!70,
    },
    % Equation label (for inline math)
    eq label/.style={
        font=\scriptsize,
        fill=white,
        inner sep=1pt,
    },
    % Section annotation
    section label/.style={
        font=\footnotesize\bfseries,
        text=black,
    },
}

% ============================================================
% JUNCTION/PARTICLE STYLES
% ============================================================
\tikzset{
    % Y-junction point
    junction point/.style={
        circle,
        fill=red!60!black,
        minimum size=4pt,
        inner sep=0pt,
    },
    % Flux tube arm
    flux arm/.style={
        thick,
        blue!60!black,
    },
    % Particle dot (electron, etc.)
    particle/.style={
        circle,
        fill=black,
        minimum size=5pt,
        inner sep=0pt,
    },
    % Neutrino (smaller, gray)
    neutrino/.style={
        circle,
        fill=gray,
        minimum size=4pt,
        inner sep=0pt,
    },
}

% ============================================================
% SPRING DECORATION (for mechanical models)
% ============================================================
\tikzset{
    spring/.style={
        thick,
        decorate,
        decoration={
            coil,
            aspect=0.5,
            segment length=2mm,
            amplitude=2mm,
        },
    },
    % Wave decoration (for field modes)
    wave field/.style={
        thick,
        decorate,
        decoration={
            snake,
            amplitude=2pt,
            segment length=8pt,
        },
    },
}

% ============================================================
% BOUNDARY STYLES
% ============================================================
\tikzset{
    % Bulk-facing boundary (dashed red)
    bulk boundary/.style={
        very thick,
        red!70!black,
        dashed,
    },
    % Observer-facing boundary (solid green)
    observer boundary/.style={
        thick,
        green!50!black,
    },
    % Brane edge (orange)
    brane edge/.style={
        thick,
        orange!70!black,
    },
}

% ============================================================
% CONVENIENCE COMMANDS
% ============================================================
% Arrow label (above)
\newcommand{\arrlabel}[1]{\scriptsize #1}
% Arrow label (below)
\newcommand{\arrlabelb}[1]{\scriptsize #1}

% ============================================================
% END OF STYLE FILE
% ============================================================


% ─────────────────────────────────────────────────────────────────────────────
% Additional packages
% ─────────────────────────────────────────────────────────────────────────────
\usepackage{booktabs}
\usepackage{array}
\usepackage{hyperref}

% ─────────────────────────────────────────────────────────────────────────────
% Document metadata
% ─────────────────────────────────────────────────────────────────────────────
\title{%
  \textbf{OPEN-W1: Toy Derivation of $G_F$ Scaling}\\[0.5em]
  \large From Thick-Brane Mediator Integration\\
  (Derivation Skeleton --- No Numerical Fit)
}
\author{%
  Igor Grčman\\
  \small Elastic Diffusive Cosmology Collaboration
}
\date{January 2026 \quad v0.1}

% ─────────────────────────────────────────────────────────────────────────────
\begin{document}
% ─────────────────────────────────────────────────────────────────────────────

\maketitle

\begin{abstract}
This document provides a \textbf{structural derivation pathway}---not a
numerical derivation---for an effective Fermi-like coefficient $G_{\mathrm{EDC}}$
emerging from thick-brane microphysics. We show how integrating out a
brane-layer mediator field at tree level produces a contact interaction with
coefficient $G_{\mathrm{EDC}} \sim g_{\mathrm{eff}}^2 / m_\phi^2$, where the
effective coupling $g_{\mathrm{eff}}$ encodes overlap integrals and
boundary-condition operators. The chirality filter from Companion V enters
the boundary-condition factor, connecting this derivation to V$-$A structure.
\textbf{No parameter is tuned to match the PDG value of $G_F$}; all numerical
gaps are explicitly marked \tagOpen{}.
\end{abstract}

% ==============================================================================
\section{Scope and Guardrails}
\label{sec:scope}
% ==============================================================================

\begin{tcolorbox}[edcGuardrail, title={Non-Negotiable Guardrails}]
\begin{enumerate}[nosep]
  \item This is a \textbf{structural pathway}, NOT a numerical derivation.
  \item We do \textbf{not} fit any parameter to $G_F$, $\tau_\mu$, $\tau_n$,
        or any weak observable. Those are \tagBL{} benchmarks only.
  \item The following remain explicitly \tagOpen{}:
        \begin{itemize}[nosep]
          \item Brane profile normalization (mode functions)
          \item Mediator gap $m_\phi$ from geometry
          \item Overlap integrals with explicit profiles
        \end{itemize}
  \item Any numerical agreement obtained by selecting normalizations is
        \textbf{calibration}, not derivation.
\end{enumerate}
\end{tcolorbox}

\textbf{What this document does.}
We demonstrate the \emph{mechanism} by which a weak-scale contact interaction
emerges from 5D brane microphysics: bulk-to-brane coupling $\to$ layer
mediator $\to$ integrate out $\to$ effective 4D operator. This establishes
``weakness'' as \emph{geometric suppression}, not a fundamental 4D vertex.

% ==============================================================================
\section{Toy Geometry Setup}
\label{sec:geometry}
% ==============================================================================

\begin{definition}[Thick-Brane Coordinates {\normalfont [Def]}]
\label{def:coords}
We adopt a minimal coordinate system:
\begin{itemize}[nosep]
  \item $x^\mu$ ($\mu = 0,1,2,3$): 4D spacetime coordinates (observer-facing)
  \item $y \in [-\delta/2, +\delta/2]$: transverse coordinate across brane
        thickness $\delta$
  \item $\xi$: compact/throat direction (enters only as a scale placeholder
        for mediator mass)
\end{itemize}
\end{definition}

\textbf{Boundary identification.}
\begin{itemize}[nosep]
  \item $y = -\delta/2$: \textbf{bulk-facing boundary} (where bulk pumping
        couples in)
  \item $y = +\delta/2$: \textbf{observer-facing boundary} (where 3D outputs
        emerge)
\end{itemize}

\edcMechanismNote{Bulk energy flux enters at $y = -\delta/2$ via junction relaxation}%
                 {Mediator field $\phi$ propagates through brane layer}%
                 {Effective interaction projects to observer-facing layer at $y = +\delta/2$}

% ==============================================================================
\section{Toy Lagrangian}
\label{sec:lagrangian}
% ==============================================================================

\subsection{Layer Mediator Field}

\begin{definition}[Mediator Sector {\normalfont [Def]/[P]}]
\label{def:mediator}
We introduce a scalar layer-mediator field $\phi(x,y)$ with mass gap $m_\phi$:
\begin{equation}
  \mathcal{L}_\phi = \frac{1}{2}(\partial_\mu \phi)^2
                   + \frac{1}{2}(\partial_y \phi)^2
                   - \frac{1}{2} m_\phi^2 \phi^2
  \label{eq:L_phi}
\end{equation}
The gap $m_\phi$ is postulated \tagP{} to arise from the $\xi$-sector geometry
(KK mass or throat scale). Its explicit derivation remains \tagOpen{}.
\end{definition}

\textbf{Physical interpretation.}
The field $\phi$ represents the lowest brane-layer mode that can mediate
energy transfer from the bulk-facing to observer-facing boundaries. In a full
treatment, $\phi$ would be the zero-mode or first KK mode of a higher-dimensional
gauge/mediator sector.

\subsection{Bulk-Brane Coupling}

\begin{definition}[Interaction Term {\normalfont [Def]/[P]}]
\label{def:interaction}
The bulk-core dynamics couple to the mediator at the bulk-facing boundary:
\begin{equation}
  \mathcal{L}_{\mathrm{int}} = g_5 \, J(x) \, \phi(x, y = -\delta/2)
  \label{eq:L_int}
\end{equation}
where $J(x)$ is a source current from the bulk trigger (e.g., junction
relaxation) and $g_5$ is the fundamental 5D coupling scale \tagP{}.
\end{definition}

\textbf{Physical interpretation.}
The junction relaxation (neutron $\to$ proton transition) ``pumps'' energy
into the brane layer through the source $J(x)$. The coupling $g_5$ encodes
how efficiently this pump excites the mediator mode $\phi$.

% ==============================================================================
\section{Integrate Out \texorpdfstring{$\phi$}{phi} at Tree Level}
\label{sec:integrate}
% ==============================================================================

\subsection{Tree-Level Elimination}

At tree level (Gaussian integration), the mediator field satisfies:
\begin{equation}
  (\Box + m_\phi^2) \phi(x,y) = g_5 \, J(x) \, \delta(y + \delta/2)
  \label{eq:eom}
\end{equation}
Solving in momentum space and evaluating the propagator:
\begin{equation}
  \phi(x, y) \approx \frac{g_5}{m_\phi^2} \, J(x) \, G(y, -\delta/2)
  \label{eq:phi_solution}
\end{equation}
where $G(y, y')$ is the Green's function across the brane thickness, normalized
so that $G(-\delta/2, -\delta/2) = \mathcal{O}(1)$.

\subsection{Effective Contact Interaction}

Substituting back into the action and integrating out $\phi$:
\begin{equation}
  \boxed{
    \mathcal{L}_{\mathrm{eff}} = -\frac{g_5^2}{2 m_\phi^2} \,
    \mathcal{O}_{\mathrm{overlap}} \, J(x) J(x)
  }
  \label{eq:L_eff}
\end{equation}
where $\mathcal{O}_{\mathrm{overlap}}$ is a dimensionless factor encoding
the Green's function normalization and wavefunction overlaps.

\begin{tcolorbox}[edcCornerstone, title={Key Structural Result [Dc]}]
\textbf{Physical interpretation (canonical):}
Equation~\eqref{eq:L_eff} is not a fundamental ``weak vertex''; it is the
low-energy residue of a 5D bulk$\to$brane transfer process. The source $J(x)$
represents bulk-facing pumping into the brane layer via the mediator $\phi$
at $y = -\delta/2$; integrating out $\phi$ compresses that transfer into an
effective local contact $JJ$ term. The apparent smallness of the coupling is
therefore a \textbf{geometric suppression}---set by the mediator gap $m_\phi$,
mode-profile overlap $\mathcal{O}_{\mathrm{overlap}}$, and boundary/projection
factor $\mathcal{O}_{\mathrm{BC}} = \mathcal{O}(\mathcal{P}_{\mathrm{frozen}},
\mathcal{P}_{\mathrm{chir}})$---rather than a tunable parameter.

\smallskip
\noindent\textit{(See Companion V for $\mathcal{P}_{\mathrm{chir}}$ as the
origin of V$-$A selection in outputs.)}
\end{tcolorbox}

\subsection{Identification with Fermi-Like Coefficient}

Comparing with the standard four-fermion form:
\begin{equation}
  \mathcal{L}_{\mathrm{Fermi}} = -\frac{G_F}{\sqrt{2}} \, J^\mu J_\mu
  \label{eq:L_fermi}
\end{equation}
we identify the \textbf{EDC analog}:
\begin{equation}
  \boxed{
    G_{\mathrm{EDC}} \sim \frac{g_{\mathrm{eff}}^2}{m_\phi^2}
    \quad \text{where} \quad
    g_{\mathrm{eff}} \equiv g_5 \sqrt{\mathcal{O}_{\mathrm{overlap}}}
  }
  \label{eq:G_EDC}
\end{equation}

% ==============================================================================
\section{Where \texorpdfstring{$g_{\mathrm{eff}}$}{g\_eff} Comes From: Overlap and BC Operators}
\label{sec:geff}
% ==============================================================================

\subsection{Overlap Integral}

\begin{definition}[Overlap Factor {\normalfont [Def]/[OPEN]}]
\label{def:overlap}
The overlap factor encodes wavefunction overlaps across the brane:
\begin{equation}
  \mathcal{O}_{\mathrm{overlap}} \sim
  \int d\xi \int_{-\delta/2}^{+\delta/2} dy \;
  f_a(\xi, y) \, f_b(\xi, y) \, f_\phi(\xi, y)
  \label{eq:overlap}
\end{equation}
where $f_a, f_b$ are fermion mode profiles and $f_\phi$ is the mediator
profile. \tagOpen{}: Explicit profiles require solving the brane-localization
BVP.
\end{definition}

\subsection{Boundary-Condition Operator}

\begin{definition}[BC Factor {\normalfont [Def]/[P]}]
\label{def:bc}
The boundary conditions impose a projection factor:
\begin{equation}
  \mathcal{O}_{\mathrm{BC}} = \mathcal{O}(\mathcal{P}_{\mathrm{frozen}},
  \mathcal{P}_{\mathrm{chir}})
  \label{eq:O_BC}
\end{equation}
where $\mathcal{P}_{\mathrm{frozen}}$ is the frozen projection operator and
$\mathcal{P}_{\mathrm{chir}}$ is the chirality filter from Companion V.
\end{definition}

\textbf{Connection to Companion V.}
The chirality filter $\mathcal{P}_{\mathrm{chir}}$ (see Companion V,
Definition~3) selects left-handed charged leptons and right-handed
antineutrinos. This is where \textbf{V$-$A structure enters} the effective
interaction: not as an input assumption, but as an output of boundary
conditions.

\subsection{Full Effective Coupling}

Combining all factors:
\begin{equation}
  \boxed{
    g_{\mathrm{eff}} \equiv g_5 \times \mathcal{O}_{\mathrm{overlap}}
    \times \mathcal{O}_{\mathrm{BC}}
  }
  \label{eq:geff_full}
\end{equation}

\begin{tcolorbox}[edcGuardrail, title={Epistemic Status of Each Factor}]
\begin{tabular}{lll}
\textbf{Factor} & \textbf{Status} & \textbf{What closes it} \\
\hline
$g_5$ & \tagOpen{} & 5D action normalization \\
$\mathcal{O}_{\mathrm{overlap}}$ & \tagOpen{} & Brane profile BVP solution \\
$\mathcal{O}_{\mathrm{BC}}$ & \tagP{}/\tagDc{} & Chirality filter (V), frozen projection \\
$m_\phi$ & \tagOpen{} & $\xi$-sector KK reduction \\
\end{tabular}
\end{tcolorbox}

% ==============================================================================
\section{Dimensional Analysis}
\label{sec:dimensions}
% ==============================================================================

\subsection{Unit Conventions}

We work in natural units: $\hbar = c = 1$. Energy has dimension $[E] = 1$,
length has $[L] = -1$.

\subsection{Dimensional Check}

\begin{table}[ht]
\centering
\caption{Dimensions of quantities in the toy derivation}
\label{tab:dimensions}
\begin{tabular}{lcc}
\toprule
\textbf{Quantity} & \textbf{Symbol} & \textbf{Dimension} \\
\midrule
5D coupling & $g_5$ & $[E]^{-1/2}$ \\
Mediator mass & $m_\phi$ & $[E]^{+1}$ \\
Brane thickness & $\delta$ & $[E]^{-1}$ \\
Overlap factor & $\mathcal{O}_{\mathrm{overlap}}$ & dimensionless \\
BC factor & $\mathcal{O}_{\mathrm{BC}}$ & dimensionless \\
Effective coupling & $g_{\mathrm{eff}}$ & $[E]^{-1/2}$ \\
Fermi constant & $G_{\mathrm{EDC}}$ & $[E]^{-2}$ \\
\bottomrule
\end{tabular}
\end{table}

\textbf{Consistency check.}
\begin{equation}
  [G_{\mathrm{EDC}}] = \frac{[g_{\mathrm{eff}}]^2}{[m_\phi]^2}
  = \frac{[E]^{-1}}{[E]^{2}} = [E]^{-2} \quad \checkmark
  \label{eq:dim_check}
\end{equation}
This matches the required dimension of a four-fermion coupling constant.

% ==============================================================================
\section{How This Would Close to Real \texorpdfstring{$G_F$}{GF}}
\label{sec:closure}
% ==============================================================================

\subsection{Closure Checklist}

A complete derivation of $G_F$ (not just its structure) requires:

\begin{enumerate}
  \item \textbf{Derive $m_\phi$ from $\xi$-sector geometry} \tagOpen{}
  \begin{itemize}[nosep]
    \item Perform KK reduction along $\xi$
    \item Identify first massive mode as the mediator
    \item Express $m_\phi$ in terms of $R_\xi$ or throat scale
  \end{itemize}

  \item \textbf{Derive mode profiles and normalizations} \tagOpen{}
  \begin{itemize}[nosep]
    \item Solve the thick-brane BVP for fermion localization
    \item Normalize the 5D action to canonical 4D kinetic terms
    \item Extract $g_5$ from the normalized action
  \end{itemize}

  \item \textbf{Compute overlap integrals with BC operators} \tagOpen{}
  \begin{itemize}[nosep]
    \item Evaluate $\mathcal{O}_{\mathrm{overlap}}$ with explicit profiles
    \item Include $\mathcal{P}_{\mathrm{chir}}$ selection effects
    \item Obtain numerical value of $g_{\mathrm{eff}}$
  \end{itemize}
\end{enumerate}

\subsection{Gap Closure Table}

\begin{table}[ht]
\centering
\caption{What closes each gap}
\label{tab:gaps}
\begin{tabular}{p{3cm}p{4cm}p{4.5cm}}
\toprule
\textbf{Gap} & \textbf{Needed Input} & \textbf{Candidate Source} \\
\midrule
Mediator mass $m_\phi$ & $\xi$-geometry, throat scale & Framework v2.0 + KK analysis \\
5D coupling $g_5$ & 5D action normalization & Companion H thick-brane action \\
Overlap $\mathcal{O}_{\mathrm{overlap}}$ & Mode profiles, BVP & Future Companion: Brane Profiles \\
BC factor $\mathcal{O}_{\mathrm{BC}}$ & Chirality filter & Companion V (already defined) \\
\bottomrule
\end{tabular}
\end{table}

% ==============================================================================
\section{Benchmarks (PDG)}
\label{sec:benchmarks}
% ==============================================================================

For reference only, the PDG value is:
\begin{equation}
  G_F = 1.1663788(6) \times 10^{-5} \; \mathrm{GeV}^{-2}
  \quad \tagBL{}
  \label{eq:GF_PDG}
\end{equation}

\begin{tcolorbox}[edcWarning, title={No-Fit Statement}]
\textbf{No parameter in this document is tuned to match the PDG value.}
The value above is a benchmark only. Any future numerical derivation must
obtain $G_F$ from geometric quantities ($\delta$, $R_\xi$, $\sigma$, etc.)
without fitting to weak observables.
\end{tcolorbox}

% ==============================================================================
\section{Summary}
\label{sec:summary}
% ==============================================================================

\begin{tcolorbox}[edcCornerstone, title={OPEN-W1 Summary}]
\begin{enumerate}[nosep]
  \item Integrating out a brane-layer mediator produces a contact interaction
        with coefficient $G_{\mathrm{EDC}} \sim g_{\mathrm{eff}}^2 / m_\phi^2$
        \tagDc{}
  \item The effective coupling is
        $g_{\mathrm{eff}} = g_5 \times \mathcal{O}_{\mathrm{overlap}} \times
        \mathcal{O}_{\mathrm{BC}}$ \tagDef{}
  \item V$-$A structure enters via $\mathcal{P}_{\mathrm{chir}}$ from
        Companion V \tagP{}/\tagDc{}
  \item Dimensional analysis confirms $[G_{\mathrm{EDC}}] = [E]^{-2}$ \tagDc{}
  \item Numerical closure requires: $m_\phi$ from geometry, profiles from
        BVP, overlaps computed \tagOpen{}
\end{enumerate}
\end{tcolorbox}

% ==============================================================================
% Related Documents
% ==============================================================================
\vspace{1em}
\hrule
\vspace{0.5em}
\footnotesize
\textbf{Related Documents:}\\
Framework v2.0 (DOI: \href{https://doi.org/10.5281/zenodo.18299085}{10.5281/zenodo.18299085}) $\cdot$
Companion H (DOI: \href{https://doi.org/10.5281/zenodo.18307539}{10.5281/zenodo.18307539})\\
Companion V (Neutrino/Chirality) — this series $\cdot$
Weak Program Overview (DOI: \href{https://doi.org/10.5281/zenodo.18319921}{10.5281/zenodo.18319921})
\normalsize

% ==============================================================================
\end{document}
% ==============================================================================
