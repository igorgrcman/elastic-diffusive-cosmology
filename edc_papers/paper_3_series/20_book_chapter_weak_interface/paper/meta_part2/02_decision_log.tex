%% 02_decision_log.tex — Decision Log for Part II
%% Created: 2026-01-22
%% Purpose: Document key architectural and methodological decisions

\section*{Decision Log}
\label{sec:meta_decision_log}

This log records key decisions made during Part~II development, including alternatives considered and rationale for choices.

\vspace{1em}

%% ========== DECISION 1 ==========
\decisionEntry{DEC-001}{2026-01-20}{%
  Use GREEN-A/YELLOW-B/RED-C naming for derivation levels%
}{%
  Previous stoplight was ambiguous (GREEN could mean ``derived'' or ``correct'').
  Three-level naming clarifies: A = electroweak consistency, B = geometric mechanism, C = first-principles.%
}{%
  (a) Keep simple GREEN/YELLOW/RED — rejected as insufficiently granular;
  (b) Use numbered levels 1/2/3 — rejected as non-intuitive%
}{aba4822}

%% ========== DECISION 2 ==========
\decisionEntry{DEC-002}{2026-01-21}{%
  Acknowledge G$_F$ v-circularity explicitly rather than hide it%
}{%
  Reviewer will attack ``exact G$_F$ agreement'' as trivial if $v = (\sqrt{2}G_F)^{-1/2}$ is input.
  Honest acknowledgment preempts attack and redirects to true prediction: $\sin^2\theta_W = 1/4$.%
}{%
  (a) Claim G$_F$ as independent prediction — rejected as intellectually dishonest;
  (b) Remove G$_F$ discussion entirely — rejected as loses valid consistency check%
}{b4ff06a}

%% ========== DECISION 3 ==========
\decisionEntry{DEC-003}{2026-01-22}{%
  Ch7 CKM: compute $Z_3$ DFT baseline first, then falsify, then overlap model%
}{%
  ``Null hypothesis'' approach: show simplest geometric model fails with quantified breaking,
  then propose mechanism that succeeds. More convincing than jumping to working model.%
}{%
  (a) Skip baseline, go directly to overlap — rejected as loses pedagogical value;
  (b) Only show baseline failure — rejected as leaves no constructive path forward%
}{a2e9a6e}

%% ========== DECISION 4 ==========
\decisionEntry{DEC-004}{2026-01-22}{%
  Use exponential overlap ansatz $f(z) \propto \exp(-|z-z_0|/\kappa)$ for CKM%
}{%
  Exponential profile is (a) analytically tractable, (b) physically motivated by localization,
  (c) gives $\lambda^{|i-j|}$ scaling directly from overlap integrals.%
}{%
  (a) Gaussian profiles — more complex, same qualitative behavior;
  (b) Numeric BVP solution — RED-C level, not yet available%
}{3b1aa94}

%% ========== DECISION 5 ==========
\decisionEntry{DEC-005}{2026-01-22}{%
  CKM vs PMNS asymmetry explained via localization width $\kappa$%
}{%
  Quarks have color coupling $\to$ tight localization $\to$ small $\kappa$ $\to$ suppressed off-diagonal.
  Neutrinos are color-neutral edge modes $\to$ broad $\kappa$ $\to$ large PMNS angles.
  Same $Z_3$ structure, different localization $\to$ different mixing.%
}{%
  (a) Separate mechanisms for CKM and PMNS — rejected as loses unification;
  (b) Claim numerical prediction — rejected as $\kappa$ ratio not computed%
}{3b1aa94}

%% ========== DECISION 6 ==========
\decisionEntry{DEC-006}{2026-01-22}{%
  Create meta-documentation appendix with optional compile switch%
}{%
  Research narrative valuable for reproducibility but adds 20+ pages.
  Switch allows: (a) full build for archive, (b) compact build for submission.%
}{%
  (a) Always include meta — rejected as too long for journals;
  (b) Separate document — rejected as loses traceability to main text%
}{pending}

\vspace{2em}
\hrule
\vspace{0.5em}
\textbf{Log Statistics:} 6 decisions documented

\clearpage
