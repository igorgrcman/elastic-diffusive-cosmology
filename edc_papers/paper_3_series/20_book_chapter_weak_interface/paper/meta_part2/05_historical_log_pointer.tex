%% 05_historical_log_pointer.tex — Pointer to Extended Logs
%% Created: 2026-01-22
%% Purpose: Direct readers to Markdown logs for full detail

\section*{Extended Historical Logs}
\label{sec:meta_historical_pointer}

The LaTeX meta-documentation provides a structured summary suitable for the PDF.
For complete, machine-parseable research records, consult the Markdown files in:

\begin{center}
\texttt{paper/meta\_part2\_md/}
\end{center}

\subsection*{Available Logs}

\begin{description}
  \item[\texttt{CLAIM\_LEDGER.md}] \hfill \\
    Complete claim database with YAML-style headers for each claim.
    Includes: ID, status, chapter, equation references, evidence paths, git commits.
    Suitable for automated parsing and CI validation.

  \item[\texttt{DECISION\_LOG.md}] \hfill \\
    Detailed decision records with full context.
    Includes: problem statement, alternatives matrix, trade-off analysis, final rationale.
    Longer than LaTeX version; preserves discussion threads.

  \item[\texttt{RESEARCH\_TIMELINE.md}] \hfill \\
    Day-by-day development log.
    Includes: what was attempted, what failed, what succeeded.
    Preserves the ``A$\to$B'' narrative of how conclusions emerged.

  \item[\texttt{EVIDENCE\_MAP.md}] \hfill \\
    Extended cross-reference database.
    Includes: file paths, line numbers, git blame, external URLs.
    Links every claim to its primary evidence.
\end{description}

\subsection*{Why Markdown?}

\begin{itemize}
  \item \textbf{Version control friendly}: Clean diffs, easy merges
  \item \textbf{Machine parseable}: YAML headers enable automation
  \item \textbf{Human readable}: No compilation required
  \item \textbf{Extensible}: New entries without recompiling PDF
\end{itemize}

\vspace{1em}

\begin{center}
\fbox{\parbox{0.8\textwidth}{%
  \centering
  \textbf{Reproducibility Note}\\[0.5em]
  All research documented here can be reproduced from the git history.\\
  Each decision log entry includes a commit hash for traceability.\\
  Build with \texttt{latexmk -xelatex EDC\_Part\_II\_Weak\_Sector.tex}
}}
\end{center}

\clearpage
