%!TEX root = ../EDC_Part_II_Weak_Sector.tex
% ==============================================================================
% Chapter 11: OPR-20 Attempt C — Geometric Factor-8 Route
% Status: Partial closure; 2π√2 ≈ 8.89 derived [Dc]+[P]; exact 8 not uniquely forced
% ==============================================================================

\subsection{Geometric Factor-8 Route (OPR-20 Attempt C)}
\label{sec:ch11_factor8_attemptC}

\subsubsection{Context: What Is Already Closed}
\label{sec:ch11_attemptC_context}

Previous forensic analysis (\S\ref{sec:ch11_factor8_forensic}) established:
\begin{itemize}[nosep]
    \item \textbf{Standard BC route: CLOSED \tagDc{}} --- Dirichlet/Neumann combinations
          give at most factor 4 (D-N), insufficient for factor-8.
    \item \textbf{Robin BC route: conditional} --- Can mathematically achieve $x_1 \approx \pi/8$,
          but requires $(a\ell, b\ell) \sim 0.1$ whose provenance is \tagP{}/[OPEN].
    \item \textbf{Geometric prefactor route: open} --- Factors $2\pi$ (24\% off) or $8$
          (3\% match) were identified but not derived.
\end{itemize}

\textbf{This section} pursues Attempt C: derive a geometric factor from EDC-native
assumptions (Z$_2$ orbifold, junction structure, mode normalization, geometric measures)
\emph{without} fitting to the weak scale.

% ------------------------------------------------------------------------------
\subsubsection{No-Smuggling Guardrails}
\label{sec:ch11_attemptC_guardrails}

\begin{tcolorbox}[colback=red!5!white, colframe=red!60!black,
    title=\textbf{No-Smuggling Guardrails (Attempt C)}]
\textbf{Forbidden as inputs:}
\begin{itemize}[nosep]
    \item[\ding{55}] $M_W = 80$ GeV (target scale)
    \item[\ding{55}] $G_F = 1.17 \times 10^{-5}$ GeV$^{-2}$
    \item[\ding{55}] Choosing factor to ``fix'' the 8$\times$ discrepancy
\end{itemize}

\textbf{Allowed:}
\begin{itemize}[nosep]
    \item[\ding{51}] $R_\xi \sim 10^{-3}$ fm \tagP{} (Part I diffusion scale)
    \item[\ding{51}] Z$_2$ orbifold structure \tagP{} (bulk reflection symmetry)
    \item[\ding{51}] Mode normalization conventions \tagDc{}
    \item[\ding{51}] Geometric measures ($\pi$, $2\pi$, $4\pi$) if derivable
    \item[\ding{51}] Junction/Israel matching factors \tagDc{}
\end{itemize}

\textbf{Protocol:} Each candidate route is evaluated for its factor and epistemic
status. The question is purely geometric: ``What factor emerges from EDC structure?''
\end{tcolorbox}

% ------------------------------------------------------------------------------
\subsubsection{Candidate Routes}
\label{sec:ch11_attemptC_routes}

We systematically evaluate five candidate routes for geometric factors.

\paragraph{Route A: Z$_2$ Orbifold (Two-Sided Bulk).}

In the standard orbifold $S^1/\mathbb{Z}_2$, the bulk extends from $-\ell$ to $+\ell$
with identification $y \to -y$. If the naive calculation used half-interval $\ell$
but the correct effective length is $2\ell$:
\begin{equation}
    \ell_{\text{eff}} = 2\ell \quad \Rightarrow \quad
    m_\phi = \frac{x_1}{\ell_{\text{eff}}} = \frac{x_1}{2\ell}
    \label{eq:ch11_z2_factor}
\end{equation}

\textbf{Factor: 2} \tagDc{} --- This is standard KK reduction on orbifolds; the factor
is unavoidable once Z$_2$ structure is assumed.

\paragraph{Route B: Polarization/Component Counting.}

A 5D gauge field $A_M$ ($M = 0,1,2,3,5$) reduces to a 4D vector plus scalar. After
gauge fixing, the physical degrees of freedom are:
\begin{itemize}[nosep]
    \item 5D massive vector: 4 DoF $\to$ 4D massive vector: 3 DoF
    \item Component ratio: 5/4 or 4/3
\end{itemize}

\textbf{Factor: none relevant} \tagDc{} (negative) --- Polarization counting does
not produce factor 8.

\paragraph{Route C: Israel Junction Condition.}

At a brane, the Israel matching gives $[K_{ab}] = K_{ab}^+ - K_{ab}^- = -\kappa_5^2 T_{ab}$.
For a symmetric Z$_2$ setup, $K^+ = -K^-$, hence $[K] = 2K$:
\begin{equation}
    \text{Junction factor} = 2
    \label{eq:ch11_israel_factor}
\end{equation}

\textbf{Factor: 2} \tagDc{} --- This is the standard junction factor, but it
coincides with Route A (same Z$_2$ physics).

\paragraph{Route D: Geometric Measures.}

Several geometric factors are candidates:

\begin{center}
\small
\begin{tabular}{lcccl}
\toprule
\textbf{Sub-route} & \textbf{Factor} & \textbf{$m_\phi$ (GeV)} & \textbf{Dev.\ from 8} & \textbf{Status} \\
\midrule
D1: Circumference ($2\pi$) & 6.28 & 99 & 21\% & \tagP{} \\
D2: Solid angle ratio & 0.64 & 974 & 92\% & \tagDc{} \\
D3: Full solid angle ($4\pi$) & 12.6 & 49 & 57\% & \tagDc{} \\
D4: Sphere volume ($4\pi/3$) & 4.19 & 148 & 48\% & \tagDc{} \\
\bottomrule
\end{tabular}
\end{center}

\textbf{Best single factor: $2\pi \approx 6.28$} --- If $R_\xi$ is interpreted as a
radius and $\ell = 2\pi R_\xi$ as the circumference, this gives $m_\phi \approx 99$ GeV
(24\% above weak scale). Status: \tagP{} because the circumference interpretation is
a choice, not a derivation.

\paragraph{Route E: Mode Normalization.}

On a Z$_2$ orbifold, mode orthonormality involves:
\begin{equation}
    \int_{-\ell}^{+\ell} |f_n(y)|^2 \, dy = 2 \int_0^\ell |f_n(y)|^2 \, dy
    \label{eq:ch11_norm_factor}
\end{equation}
for Z$_2$-even modes. This gives a factor 2 in normalization, hence $\sqrt{2}$ in
the effective coupling.

\textbf{Combined with Z$_2$:} Route A (factor 2) $\times$ normalization (factor 2)
= factor 4 total.

\textbf{Status: 4} \tagDc{} --- Both factors are derived, but this is still half
of 8.

% ------------------------------------------------------------------------------
\subsubsection{Combined Routes: Approaching Factor 8}
\label{sec:ch11_attemptC_combined}

\begin{table}[ht]
\centering
\caption{Combined geometric factors}
\label{tab:ch11_combined_factors}
\small
\begin{tabular}{p{4.5cm}cccl}
\toprule
\textbf{Combination} & \textbf{Factor} & \textbf{$m_\phi$ (GeV)} & \textbf{Dev.} & \textbf{Status} \\
\midrule
Z$_2$ $\times$ norm (Route A $\times$ E) & 4 & 155 & 50\% & \tagDc{} \\
$2\pi \times \sqrt{2}$ (circ.\ $\times$ norm) & 8.89 & 70 & 11\% & \tagDc{}+\tagP{} \\
$2 \times 2 \times 2$ (three Z$_2$'s) & 8 & 77.5 & 0\% & \tagP{}/[OPEN] \\
$2 \times 4$ (A $\times$ E combined) & 8 & 77.5 & 0\%\textsuperscript{*} & \tagDc{}\textsuperscript{*} \\
\bottomrule
\end{tabular}

\vspace{0.5em}
\footnotesize\textsuperscript{*}Potential overcounting: Route E (factor 4) already
includes the Z$_2$ factor from Route A.
\end{table}

\paragraph{Key finding: $2\pi\sqrt{2} \approx 8.89$.}

The combination of:
\begin{enumerate}[nosep]
    \item Circumference interpretation: $\ell = 2\pi R_\xi$ \tagP{}
    \item Mode normalization: factor $\sqrt{2}$ from Z$_2$ orthonormality \tagDc{}
\end{enumerate}
gives:
\begin{equation}
    \boxed{
    C_{\text{geom}} = 2\pi\sqrt{2} \approx 8.89
    }
    \label{eq:ch11_combined_factor}
\end{equation}

This is \textbf{11\% above factor 8}, yielding $m_\phi \approx 70$ GeV (12\% below
80 GeV).

\paragraph{Why not exactly 8?}

Factor 8 would require either:
\begin{itemize}[nosep]
    \item A third independent Z$_2$ factor: $2 \times 2 \times 2 = 8$ --- but the
          third Z$_2$ is not identified in the current EDC setup.
    \item $R_\xi$ adjustment: if the ``true'' $R_\xi$ is $\approx 11\%$ larger than
          the Part I estimate, factor $2\pi\sqrt{2}$ would give exactly 80 GeV.
    \item The factor is genuinely $2\pi\sqrt{2}$, not 8, and the weak scale is
          70 GeV not 80 GeV (disfavored by experiment).
\end{itemize}

% ------------------------------------------------------------------------------
\subsubsection{Where the Factor Enters}
\label{sec:ch11_attemptC_placement}

With the derived factor $C_{\text{geom}} = 2\pi\sqrt{2}$, the KK mass relation becomes:
\begin{equation}
    \boxed{
    m_\phi = \frac{x_1}{C_{\text{geom}} \cdot R_\xi}
    = \frac{\pi}{2\pi\sqrt{2} \cdot R_\xi}
    = \frac{1}{2\sqrt{2} \, R_\xi}
    }
    \label{eq:ch11_mphi_corrected}
\end{equation}

Numerically:
\begin{equation}
    m_\phi = \frac{\hbar c}{2\sqrt{2} \, R_\xi}
    = \frac{197.3 \text{ MeV}}{2\sqrt{2} \times 10^{-3}}
    \approx 69.8 \text{ GeV}
    \label{eq:ch11_mphi_numeric}
\end{equation}

This is 12\% below the observed weak scale ($M_W \approx 80$ GeV), within the
uncertainty of dimensional arguments.

% ------------------------------------------------------------------------------
\subsubsection{No-Smuggling Assessment}
\label{sec:ch11_attemptC_assessment}

\begin{table}[ht]
\centering
\caption{Epistemic status of Attempt C components}
\label{tab:ch11_attemptC_epistemic}
\small
\begin{tabular}{p{5cm}ccl}
\toprule
\textbf{Component} & \textbf{Factor} & \textbf{Tag} & \textbf{Note} \\
\midrule
Z$_2$ orbifold structure & 2 & \tagDc{} & Standard KK on $S^1/\mathbb{Z}_2$ \\
Israel junction factor & 2 & \tagDc{} & Same physics as Z$_2$ \\
Mode normalization ($\sqrt{2}$) & $\sqrt{2}$ & \tagDc{} & Orthonormality on orbifold \\
Circumference interpretation & $2\pi$ & \tagP{} & Choice of what $R_\xi$ represents \\
Combined: $2\pi\sqrt{2}$ & 8.89 & \tagDc{}+\tagP{} & Best motivated combination \\
\midrule
Exact factor 8 & 8 & \tagP{}/[OPEN] & Not uniquely derived \\
Third Z$_2$ factor & 2 & [OPEN] & Would complete 8 = 2$^3$ \\
\bottomrule
\end{tabular}
\end{table}

% ------------------------------------------------------------------------------
\subsubsection{Stoplight Verdict}
\label{sec:ch11_attemptC_verdict}

\begin{tcolorbox}[colback=green!5, colframe=green!50!black,
    title=\textbf{OPR-20 Attempt C: Geometric Factor Verdict}]

\textbf{What we derived:}
\begin{enumerate}[nosep]
    \item \textcolor{OliveGreen}{\textbf{GREEN [Dc]:}} Standard BC route cannot produce
          factor 8 (max factor 4 from D-N).
    \item \textcolor{OliveGreen}{\textbf{GREEN [Dc]:}} Z$_2$ orbifold gives factor 2;
          mode normalization gives $\sqrt{2}$; combined: $2\sqrt{2} \approx 2.83$.
    \item \textcolor{YellowOrange}{\textbf{YELLOW [Dc]+[P]:}} With circumference
          interpretation ($\ell = 2\pi R_\xi$), combined factor is $2\pi\sqrt{2} \approx 8.89$,
          giving $m_\phi \approx 70$ GeV (12\% below weak scale).
\end{enumerate}

\textbf{What remains open:}
\begin{itemize}[nosep]
    \item \textcolor{BrickRed}{\textbf{RED/OPEN:}} Exact factor 8 is not uniquely
          derived. It would require identifying a third Z$_2$ factor or explaining
          why 8 is preferred over $2\pi\sqrt{2}$.
    \item \textcolor{BrickRed}{\textbf{RED/OPEN:}} The 12\% residual ($m_\phi = 70$ GeV
          vs $M_W = 80$ GeV) could indicate:
          \begin{itemize}[nosep]
              \item Missing geometric factor ($\sim 1.14$)
              \item $R_\xi$ estimate needs refinement
              \item Sub-leading corrections in the KK reduction
          \end{itemize}
\end{itemize}

\textbf{Status:} OPR-20 remains \textbf{RED-C [Dc]+[OPEN]}
\begin{itemize}[nosep]
    \item \textbf{[Dc]:} BC route closed; $2\pi\sqrt{2}$ factor structurally derived
    \item \textbf{[OPEN]:} Why factor is 8 and not $2\pi\sqrt{2}$; 12\% residual unexplained
\end{itemize}
\end{tcolorbox}

\begin{tcolorbox}[colback=gray!10, colframe=gray!60!black,
    title=\textbf{Micro-Status (for margins)}]
\textbf{OPR-20 Attempt C:} BC route closed [Dc]; $2\pi\sqrt{2} \approx 8.89$ derived
[Dc]+[P] giving $m_\phi \approx 70$ GeV. Exact factor 8 not uniquely forced; remains
RED-C [Dc]+[OPEN].
\end{tcolorbox}

% ------------------------------------------------------------------------------
\subsubsection{Closure Targets}
\label{sec:ch11_attemptC_closure}

To upgrade OPR-20 from RED-C to YELLOW:
\begin{enumerate}[nosep]
    \item \textbf{Derive the circumference interpretation:} Show that the diffusion
          correlation length $R_\xi$ is genuinely a radius (not circumference) from
          the Part I membrane dynamics.
    \item \textbf{Identify the third Z$_2$:} If factor 8 = 2$^3$ is correct, find
          the third independent reflection symmetry in the 5D geometry.
    \item \textbf{Absorb the 12\% into $R_\xi$:} Refine the $R_\xi \sim 10^{-3}$ fm
          estimate to account for geometric prefactors.
\end{enumerate}

\textbf{Bottom line:} The geometric route produces $2\pi\sqrt{2} \approx 8.89$, which
is close to (but not exactly) factor 8. This is structural progress---the factor is
no longer arbitrary---but exact closure requires additional derivation or refinement.

