% ==============================================================================
% Chapter 6 Subsection: PMNS Attempt 1 — Z₃ Symmetry Baseline
% Status: Negative result — DFT baseline falsified by θ₁₃
% ==============================================================================

\subsection{Attempt PMNS-1: Symmetry Baseline and Minimal Breaking}
\label{sec:ch6_pmns_attempt1}

\begin{tcolorbox}[edcGuardrail, title=\textbf{Purpose}]
This subsection computes what the PMNS matrix would be under \textbf{exact
$\mathbb{Z}_3$ symmetry}. The goal is to close the logical loop: either
$\mathbb{Z}_3$ predicts the observed mixing, or it doesn't (requiring breaking).
\end{tcolorbox}

% ------------------------------------------------------------------------------
\subsubsection{The Discrete Fourier Transform Baseline}
\label{sec:ch6_dft_baseline}

Under the identification of three neutrino flavors with $\mathbb{Z}_3$ elements
(Section~\ref{sec:ch6_three_flavors}), we assign angular positions:
\begin{equation}
    \phi_\alpha = \frac{2\pi\alpha}{3}, \qquad \alpha \in \{0, 1, 2\}
    \quad\leftrightarrow\quad (\nu_e, \nu_\mu, \nu_\tau)
    \label{eq:ch6_z3_positions}
\end{equation}

\paragraph{Hypothesis (minimal symmetric assumption).}
If the Higgs/mass mechanism is $\mathbb{Z}_3$-invariant, the mass eigenstates
are the \textbf{delocalized} Fourier modes \tagP{}:
\begin{equation}
    |\nu_i\rangle = \frac{1}{\sqrt{3}} \sum_{\alpha=0}^{2} \omega^{-\alpha i} |\nu_\alpha\rangle,
    \qquad \omega = e^{2\pi i/3}
    \label{eq:ch6_fourier_modes}
\end{equation}

This is the discrete Fourier transform (DFT) on $\mathbb{Z}_3$. The PMNS matrix
becomes \tagDc{}:
\begin{equation}
    U_{\alpha i}^{\text{DFT}} = \langle\nu_\alpha|\nu_i\rangle
    = \frac{1}{\sqrt{3}} \omega^{-\alpha i}
    \label{eq:ch6_dft_pmns}
\end{equation}

Explicitly, with $\omega = e^{2\pi i/3}$ and $\omega^* = \omega^2 = e^{-2\pi i/3}$:
\begin{equation}
    U^{\text{DFT}} = \frac{1}{\sqrt{3}}
    \begin{pmatrix}
        1 & 1 & 1 \\
        1 & \omega^* & \omega \\
        1 & \omega & \omega^*
    \end{pmatrix}
    \label{eq:ch6_dft_matrix}
\end{equation}

\paragraph{Key property.}
All elements have equal magnitude:
\begin{equation}
    |U_{\alpha i}^{\text{DFT}}|^2 = \frac{1}{3} \quad \forall\, \alpha, i
    \label{eq:ch6_democratic}
\end{equation}
This is the ``democratic'' or ``trimaximal'' pattern.

% ------------------------------------------------------------------------------
\subsubsection{Predicted Mixing Angles}
\label{sec:ch6_dft_angles}

Using the standard PMNS parametrization \tagBL{}:
\begin{align}
    \sin^2\theta_{13} &= |U_{e3}|^2 \\
    \sin^2\theta_{12} &= \frac{|U_{e2}|^2}{1 - |U_{e3}|^2} \\
    \sin^2\theta_{23} &= \frac{|U_{\mu3}|^2}{1 - |U_{e3}|^2}
\end{align}

For the DFT matrix with $|U_{\alpha i}|^2 = 1/3$:
\begin{align}
    \sin^2\theta_{13}^{\text{DFT}} &= \frac{1}{3} \approx 0.333
    \label{eq:ch6_theta13_dft} \\
    \sin^2\theta_{12}^{\text{DFT}} &= \frac{1/3}{1 - 1/3} = \frac{1}{2} = 0.5
    \label{eq:ch6_theta12_dft} \\
    \sin^2\theta_{23}^{\text{DFT}} &= \frac{1/3}{1 - 1/3} = \frac{1}{2} = 0.5
    \label{eq:ch6_theta23_dft}
\end{align}

% ------------------------------------------------------------------------------
\subsubsection{Comparison with PDG Data}
\label{sec:ch6_dft_comparison}

\begin{table}[ht]
\centering
\caption{DFT baseline vs.\ observed PMNS angles}
\label{tab:ch6_dft_comparison}
\begin{tabular}{lcccl}
\toprule
\textbf{Angle} & \textbf{DFT Prediction} & \textbf{PDG 2024} \tagBL{} & \textbf{Ratio} & \textbf{Status} \\
\midrule
$\sin^2\theta_{13}$ & 0.333 & $0.0220 \pm 0.0007$ & $\times 15$ & \textcolor{red}{\textbf{FALSIFIED}} \\
$\sin^2\theta_{12}$ & 0.500 & $0.307 \pm 0.013$ & $\times 1.6$ & \textcolor{orange}{\textbf{OFF}} \\
$\sin^2\theta_{23}$ & 0.500 & $0.546 \pm 0.021$ & $\times 0.9$ & \textcolor{green!50!black}{\textbf{OK}} \\
\bottomrule
\end{tabular}
\end{table}

\begin{tcolorbox}[colback=red!5, colframe=red!50!black,
    title=\textbf{Verdict: DFT Baseline FALSIFIED}]
The exact $\mathbb{Z}_3$ symmetric (DFT) mixing pattern predicts
$\sin^2\theta_{13} = 1/3$, which is \textbf{15 times larger} than the observed
value of $\approx 0.022$.

\textbf{Conclusion:} The observed small $\theta_{13}$ \emph{requires breaking
of the naive $\mathbb{Z}_3$ symmetry}.
\end{tcolorbox}

% ------------------------------------------------------------------------------
\subsubsection{Implications: What Breaking Is Needed?}
\label{sec:ch6_breaking}

The failure of the DFT baseline identifies the key requirement: the electron
neutrino must have \textbf{suppressed coupling} to the third mass eigenstate
($|U_{e3}|^2 \ll 1/3$).

\paragraph{Candidate breaking mechanisms.}
\begin{enumerate}[nosep]
    \item \textbf{$\mathbb{Z}_2$ breaking from $\mathbb{Z}_6$:}
          The full hexagonal symmetry is $\mathbb{Z}_6 = \mathbb{Z}_2 \times \mathbb{Z}_3$.
          The $\mathbb{Z}_2$ factor distinguishes even/odd modes and could
          selectively suppress $U_{e3}$ \tagP{}.

    \item \textbf{Localization asymmetry:}
          If $\nu_e$ is more localized than $\nu_\mu, \nu_\tau$ (different
          penetration depths $\kappa_\alpha^{-1}$), the overlap with the third
          mass eigenstate could be suppressed \tagP{}.

    \item \textbf{Higgs profile anisotropy:}
          If the Higgs/mass mechanism couples differently to different $\mathbb{Z}_3$
          sectors, the democratic mixing is broken \tagP{}.
\end{enumerate}

\paragraph{Minimal perturbation estimate.}
To reduce $\sin^2\theta_{13}$ from $1/3$ to $\sim 0.02$, we need:
\begin{equation}
    |U_{e3}|^2 \approx \frac{1}{3} \cdot \epsilon^2, \qquad
    \epsilon \approx \sqrt{\frac{0.022}{0.333}} \approx 0.26
    \label{eq:ch6_epsilon}
\end{equation}
A $\sim 25\%$ breaking of $\mathbb{Z}_3$ symmetry in the $\nu_e$--$\nu_3$
coupling would suffice.

\begin{tcolorbox}[edcGuardrail, title=\textbf{Status}]
The breaking mechanism is \textbf{postulated} \tagP{}, not derived.
Explicit computation of $\epsilon$ from EDC geometry remains (open).
\end{tcolorbox}

% ------------------------------------------------------------------------------
\subsubsection{Alternative: Tri-Bimaximal as Target}
\label{sec:ch6_tbm}

For reference, the tri-bimaximal (TBM) mixing pattern \tagBL{}:
\begin{equation}
    U^{\text{TBM}} =
    \begin{pmatrix}
        \sqrt{2/3} & 1/\sqrt{3} & 0 \\
        -1/\sqrt{6} & 1/\sqrt{3} & 1/\sqrt{2} \\
        1/\sqrt{6} & -1/\sqrt{3} & 1/\sqrt{2}
    \end{pmatrix}
    \label{eq:ch6_tbm}
\end{equation}
predicts $\theta_{13} = 0$, $\sin^2\theta_{12} = 1/3$, $\sin^2\theta_{23} = 1/2$.

TBM arises from discrete flavor symmetries like $A_4$ or $S_4$ \tagBL{}.
However, $\mathbb{Z}_6$ is abelian and \textbf{cannot contain} the non-abelian
$A_4$ as a subgroup \tagM{}. Therefore:

\begin{tcolorbox}[colback=orange!5, colframe=orange!50!black,
    title=\textbf{Structural Limitation}]
The EDC $\mathbb{Z}_6$ hexagonal symmetry alone \textbf{cannot derive}
tri-bimaximal mixing. TBM-like patterns would require additional structure
beyond $\mathbb{Z}_6$ \tagP{}.
\end{tcolorbox}

% ------------------------------------------------------------------------------
\subsubsection{Updated Stoplight: PMNS Mechanism}
\label{sec:ch6_pmns_stoplight_updated}

\begin{table}[ht]
\centering
\caption{Updated PMNS mixing audit (post-Attempt 1)}
\label{tab:ch6_pmns_stoplight_v2}
\begin{tabular}{lccl}
\toprule
\textbf{Claim} & \textbf{Status} & \textbf{Tag} & \textbf{Note} \\
\midrule
$U_{\text{PMNS}}$ exists & GREEN & \tagBL{} & Observed \\
$\mathbb{Z}_3$ DFT baseline computed & GREEN & \tagDc{} & Eq.~\eqref{eq:ch6_dft_matrix} \\
DFT predicts $\theta_{13}$ & COMPUTED & \tagDc{} & $\sin^2\theta_{13} = 1/3$ \\
DFT vs.\ PDG comparison & \textcolor{red}{FALSIFIED} & --- & Factor 15 off \\
Breaking mechanism identified & YELLOW & \tagP{} & $\sim 25\%$ anisotropy needed \\
Explicit $\epsilon$ derivation & RED & (open) & Not computed \\
$\theta_{12}, \theta_{23}$ from geometry & RED & (open) & Requires breaking model \\
CP phase $\delta$ & RED & (open) & Not addressed \\
\bottomrule
\end{tabular}
\end{table}

\paragraph{Overall verdict.}
The PMNS attempt upgrades from pure RED to \textbf{YELLOW with a computed
negative baseline}:
\begin{itemize}[nosep]
    \item We now know what $\mathbb{Z}_3$ symmetry predicts (DFT matrix) \tagDc{}
    \item We know it fails for $\theta_{13}$ by a factor of 15 \tagDc{}
    \item We know breaking is required at the $\sim 25\%$ level \tagI{}
    \item The specific breaking mechanism remains open \tagP{}
\end{itemize}

This closes the logical loop: the question ``what does $\mathbb{Z}_3$ predict
for PMNS?'' now has a concrete, falsified answer.

