%!TEX root = ../EDC_Part_II_Weak_Sector.tex
% ==============================================================================
% BVP Work Package: Thick-Brane Solver Specification (OPR-02/21)
% Status: Infrastructure definition — NOT claiming closure
% ==============================================================================

\subsection{BVP Work Package: Thick-Brane Solver Specification}
\label{sec:ch12_bvp_workpackage}

This subsection defines a \textbf{Work Package} for the thick-brane boundary value
problem (BVP) that appears in multiple OPR items. The goal is infrastructure, not
closure: define the problem precisely, establish acceptance criteria, and provide
a minimal solver skeleton for future development.

\begin{tcolorbox}[colback=yellow!5!white, colframe=yellow!60!black,
    title=\textbf{Scope Limitation}]
This work package does \textbf{not} claim to:
\begin{itemize}[nosep]
    \item Derive generation counting (OPR-02)
    \item Close CKM/PMNS from first principles
    \item Provide complete $G_F$ derivation
\end{itemize}
It \textbf{does} provide:
\begin{itemize}[nosep]
    \item Precise mathematical specification of the BVP
    \item Acceptance criteria for ``success''
    \item Failure modes and their implications
    \item Minimal numerical skeleton for testing
\end{itemize}
\end{tcolorbox}

% ------------------------------------------------------------------------------
\subsubsection{WP-BVP-0: Problem Definition}
\label{sec:bvp_definition}

\paragraph{The fermion localization equation.}
In a thick-brane scenario, fermion profiles $f(z)$ in the extra dimension satisfy
a Schr\"odinger-like equation \tagP{}:
\begin{equation}
    \boxed{
    \left[ -\frac{d^2}{dz^2} + V(z) \right] f(z) = m^2 f(z)
    }
    \label{eq:bvp_schrodinger}
\end{equation}
where:
\begin{itemize}[nosep]
    \item $z \in [0, \ell]$ is the extra-dimensional coordinate
    \item $V(z)$ is an effective potential from the brane geometry
    \item $m^2$ is the 4D mass-squared eigenvalue
    \item $f(z)$ is the fermion profile (to be normalized)
\end{itemize}

\paragraph{Potential ansatz.}
The simplest thick-brane potential is a symmetric well \tagP{}:
\begin{equation}
    V(z) = V_0 \left[ 1 - \operatorname{sech}^2\left(\frac{z - \ell/2}{w}\right) \right]
    \label{eq:bvp_potential}
\end{equation}
where $V_0$ is the barrier height and $w$ is the wall width. Alternative potentials
(square well, linear, exponential) are also valid test cases.

\paragraph{Boundary conditions.}
Three physically motivated BC choices:
\begin{enumerate}[nosep]
    \item \textbf{Dirichlet:} $f(0) = f(\ell) = 0$ (hard walls)
    \item \textbf{Neumann:} $f'(0) = f'(\ell) = 0$ (no flux)
    \item \textbf{Mixed:} $f(0) = 0$, $f'(\ell) = 0$ (or vice versa)
\end{enumerate}
The physical BC depends on brane microphysics and is currently \textbf{[OPEN]}.

% ------------------------------------------------------------------------------
\subsubsection{WP-BVP-1: Dimensionless Reduction}
\label{sec:bvp_dimensionless}

\paragraph{Rescaling.}
Define dimensionless variables \tagDc{}:
\begin{align}
    \xi &= z/\ell \in [0,1] \label{eq:bvp_xi} \\
    \tilde{V}(\xi) &= \ell^2 V(\ell\xi) \label{eq:bvp_vtilde} \\
    \tilde{m}^2 &= \ell^2 m^2 \label{eq:bvp_mtilde}
\end{align}

\paragraph{Dimensionless BVP.}
The eigenvalue equation becomes:
\begin{equation}
    \left[ -\frac{d^2}{d\xi^2} + \tilde{V}(\xi) \right] \tilde{f}(\xi) = \tilde{m}^2 \tilde{f}(\xi)
    \label{eq:bvp_dimensionless}
\end{equation}
This is pure mathematics; no physics assumptions enter the rescaling.

\paragraph{Normalization.}
The profile must satisfy:
\begin{equation}
    \int_0^1 |\tilde{f}(\xi)|^2 \, d\xi = 1
    \label{eq:bvp_normalization}
\end{equation}

% ------------------------------------------------------------------------------
\subsubsection{WP-BVP-2: Numerical Method}
\label{sec:bvp_numerics}

\paragraph{Method choice.}
For the skeleton implementation, we use finite differences with shooting \tagP{}:
\begin{enumerate}[nosep]
    \item Discretize $\xi_i = i/N$ for $i = 0, \ldots, N$
    \item Approximate $d^2f/d\xi^2 \approx (f_{i+1} - 2f_i + f_{i-1})/h^2$
    \item Solve the resulting matrix eigenvalue problem
    \item Or: use shooting method with scipy \texttt{solve\_bvp}
\end{enumerate}

\paragraph{Alternative methods.}
More sophisticated approaches (spectral, collocation, WKB) are valid but not
required for the skeleton. The goal is demonstrating that solutions exist,
not optimal numerics.

% ------------------------------------------------------------------------------
\subsubsection{WP-BVP-3: Acceptance Criteria}
\label{sec:bvp_acceptance}

\begin{tcolorbox}[colback=green!5!white, colframe=green!50!black,
    title=\textbf{Acceptance Criteria for BVP Skeleton}]
A successful BVP demonstration must show:
\begin{enumerate}
    \item \textbf{Existence:} At least one bound state exists for reasonable $\tilde{V}$
    \item \textbf{Normalization:} Profile satisfies $\int |\tilde{f}|^2 d\xi = 1$
    \item \textbf{Convergence:} Eigenvalue stable under grid refinement
          ($N = 100, 200, 400$ give consistent $\tilde{m}^2$)
    \item \textbf{Reproducibility:} Different initial conditions converge to same solution
\end{enumerate}

\textbf{What this does NOT require:}
\begin{itemize}[nosep]
    \item Matching to physical particle masses
    \item Deriving the potential from first principles
    \item Computing overlap integrals with specific CKM/PMNS values
\end{itemize}
\end{tcolorbox}

% ------------------------------------------------------------------------------
\subsubsection{WP-BVP-4: Failure Modes}
\label{sec:bvp_failure}

\begin{tcolorbox}[colback=red!5!white, colframe=red!50!black,
    title=\textbf{Failure Modes and Implications}]
\begin{description}[nosep, font=\normalfont\bfseries]
    \item[F1: No bound states]
        If $\tilde{V}$ is too shallow, no localized modes exist.
        \emph{Implication:} Potential ansatz inadequate; need deeper well or different form.

    \item[F2: Non-convergence]
        Eigenvalue changes significantly with grid refinement.
        \emph{Implication:} Numerical method unstable; need higher order or different approach.

    \item[F3: Multiple degenerate modes]
        Unexpected degeneracy in spectrum.
        \emph{Implication:} May indicate symmetry; check BC consistency.

    \item[F4: Profile not localized]
        Solution extends uniformly across domain (not peaked).
        \emph{Implication:} Potential wrong sign or parameters; check physics.
\end{description}
\end{tcolorbox}

% ------------------------------------------------------------------------------
\subsubsection{WP-BVP-5: Overlap Integral Definition}
\label{sec:bvp_overlap}

\paragraph{The overlap integral.}
Once profiles $\tilde{f}_i(\xi)$ are obtained, the overlap integral is:
\begin{equation}
    \mathcal{O}_{ij} = \int_0^1 \tilde{f}_i(\xi) \, \tilde{f}_j(\xi) \, w(\xi) \, d\xi
    \label{eq:bvp_overlap}
\end{equation}
where $w(\xi)$ is an optional weight function (often $w = 1$).

\paragraph{Four-point overlap for $G_F$.}
The Fermi constant involves a four-fermion contact term:
\begin{equation}
    I_4 = \int_0^1 |\tilde{f}_L(\xi)|^4 \, d\xi
    \label{eq:bvp_I4}
\end{equation}
This integral measures how ``localized'' the profile is. For a delta-function,
$I_4 \to \infty$; for a uniform distribution, $I_4 = 1$.

\paragraph{What the skeleton computes.}
The minimal skeleton will compute:
\begin{itemize}[nosep]
    \item One profile $\tilde{f}_0(\xi)$ (ground state)
    \item Normalization check: $\int |\tilde{f}_0|^2 = 1$
    \item $I_4$ value for the ground state
\end{itemize}

% ------------------------------------------------------------------------------
\subsubsection{Summary: BVP Work Package Status}
\label{sec:bvp_summary}

\begin{table}[ht]
\centering
\caption{BVP Work Package: components and status}
\label{tab:bvp_wp_status}
\small
\begin{tabular}{clcl}
\toprule
\textbf{WP} & \textbf{Component} & \textbf{Status} & \textbf{Notes} \\
\midrule
0 & Problem definition & \textcolor{OliveGreen}{\textbf{DONE}} & Eq.~\eqref{eq:bvp_schrodinger}, potential ansatz, BCs \\
1 & Dimensionless reduction & \textcolor{OliveGreen}{\textbf{DONE}} & Eq.~\eqref{eq:bvp_dimensionless}, pure math \\
2 & Numerical method & \textcolor{YellowOrange}{\textbf{SKELETON}} & Finite differences; see \texttt{code/} \\
3 & Acceptance criteria & \textcolor{OliveGreen}{\textbf{DEFINED}} & Existence, normalization, convergence \\
4 & Failure modes & \textcolor{OliveGreen}{\textbf{DOCUMENTED}} & F1--F4 identified \\
5 & Overlap outputs & \textcolor{YellowOrange}{\textbf{DEFINED}} & $\mathcal{O}_{ij}$, $I_4$; to be computed \\
\bottomrule
\end{tabular}
\end{table}

\begin{tcolorbox}[colback=blue!5!white, colframe=blue!50!black,
    title=\textbf{BVP Work Package: Bottom Line}]
\textbf{What is established:}
\begin{itemize}[nosep]
    \item Mathematical specification of thick-brane BVP (Eq.~\eqref{eq:bvp_schrodinger})
    \item Dimensionless formulation for numerical work
    \item Clear acceptance criteria and failure modes
    \item Overlap integral definitions for downstream use
\end{itemize}

\textbf{What remains for OPR-02/21 closure:}
\begin{itemize}[nosep]
    \item Derive potential $V(z)$ from membrane parameters $(\sigma, r_e)$
    \item Determine boundary conditions from physical consistency
    \item Solve BVP numerically with physical parameters
    \item Show spectrum matches generation counting (OPR-02)
    \item Compute overlaps for CKM/PMNS structure (OPR-21)
\end{itemize}

\medskip
\noindent\fbox{\parbox{0.92\textwidth}{\small
\textbf{Status:} BVP Work Package defined; solver skeleton exists; closure requires
physical EOM derivation + BC justification + verified profile computation.
OPR-02/21 remain RED but with concrete path forward.}}
\end{tcolorbox}

