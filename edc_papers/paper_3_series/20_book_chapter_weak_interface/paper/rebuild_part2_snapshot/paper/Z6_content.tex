% ==============================================================================
% Z₆ PROGRAM CONTENT FOR BOOK INCLUSION
% ==============================================================================
% This file provides the Chapter 3 content for the unified Part II book.
% The full derivation is available in Z6_PROGRAM_COMPLETE_DERIVATION.pdf
% ==============================================================================

\section*{Abstract}

We derive color confinement and $SU(3)$ gauge symmetry from a geometric principle:
the hexagonal crystallization of flux tubes on a thick-brane interface in 5D
spacetime. Starting from the Kepler-Hales theorem (optimal sphere packing),
we show that energy minimization forces flux tubes into a hexagonal lattice
with $\mathbb{Z}_6$ rotational symmetry. This symmetry contains $\mathbb{Z}_3$
as a subgroup, which we identify with the center of $SU(3)$.

% ==============================================================================
\section{The Genesis: Igor's Question}
\label{sec:z6_genesis}
% ==============================================================================

On January 21, 2026, the following question catalyzed this investigation:

\begin{quote}
\textit{``Can you mathematically prove (using 5D EDC mathematics) that the proton
is a topological energy minimum, and that the Steiner 120° geometry is
geometrically determined by the topology of $M_5$ and the boundary conditions
we have established?''}
\end{quote}

What follows is the answer---and far more than was originally asked.

% ==============================================================================
\section{Summary of Main Results}
\label{sec:z6_results}
% ==============================================================================

The derivation proceeds in eight logical steps:

\subsection{Step 1--3: From Steiner to $\mathbb{Z}_6$}

\begin{enumerate}
\item \textbf{Steiner Problem} [M]: The 120° angles minimize total string length
\item \textbf{$\mathbb{Z}_6$ Hypothesis} [P]: The brane interface has hexagonal symmetry
\item \textbf{Kepler-Hales} [M] $\to$ \textbf{Emergence} [Dc]: Flux tube crystallization
      forces hexagonal packing, hence $\mathbb{Z}_6$ boundary conditions
\end{enumerate}

\subsection{Step 4: Proton Stability}

\begin{theorem}[Proton as $\mathbb{Z}_3$ Fixed Point]
The proton Y-junction configuration is a local minimum of the energy functional
$\mathcal{E}[\Psi]$ within its topological sector. The Hessian is positive-definite,
ensuring metastability.
\end{theorem}

\textbf{Physical meaning:} The proton is topologically protected---it cannot decay
without changing the $\mathbb{Z}_3$ charge (which is conserved).

\subsection{Step 5: Neutron as Dislocation}

\begin{definition}[Neutron]
The neutron is a \emph{dislocation} in the $\mathbb{Z}_6$ lattice---a localized
defect with Burgers vector $|\vec{b}| = a$ (one lattice constant).
\end{definition}

\textbf{Key result:} The dislocation energy is:
\begin{equation}
E_{\text{disl}} = \frac{\mu b^2}{4\pi(1-\nu)} \ln\left(\frac{R}{r_0}\right) \approx 1.29 \text{ MeV}
\end{equation}
matching the neutron-proton mass difference $\Delta m_{np} = 1.293$ MeV [BL].

\textbf{$\beta$-decay mechanism:} The dislocation annihilates via Peierls barrier
tunneling, releasing energy as $e^- + \bar\nu_e$.

\subsection{Step 6: $\mathbb{Z}_3 \to SU(3)$ Emergence}

\begin{theorem}[Link Variable Construction]
From the $\mathbb{Z}_3$ vortex structure, we construct Wilson-type link variables:
\begin{equation}
U_\ell = \exp\left( i g \int_\ell A_\mu \, dx^\mu \right) \in SU(3)
\end{equation}
satisfying:
\begin{enumerate}[nosep]
\item $U_\ell U_\ell^\dagger = \mathbb{1}$
\item Wilson loop: $W(C) = \text{Tr}[\prod_{\ell \in C} U_\ell] \neq 0$
\item Area law: $\langle W(C) \rangle \sim e^{-\sigma A(C)}$ (confinement)
\end{enumerate}
\end{theorem}

\textbf{Confinement mechanism:} Isolated quarks have $\mathbb{Z}_3$ charge $N \not\equiv 0$,
giving $W(C) = \omega^N \neq 1$ and infinite string energy.

\subsection{Step 7: Unification Vision}

The factorization $\mathbb{Z}_6 = \mathbb{Z}_2 \times \mathbb{Z}_3$ suggests:
\begin{itemize}[nosep]
\item $\mathbb{Z}_3$: Color (strong force) --- \textbf{derived above}
\item $\mathbb{Z}_2$: Electroweak? --- \textbf{[OPEN]}
\end{itemize}

\subsection{Step 8: Mass Hierarchy}

Three fermion generations are identified with radial harmonics of the $\mathbb{Z}_3$
vortex. The Koide formula:
\begin{equation}
\frac{m_e + m_\mu + m_\tau}{(\sqrt{m_e} + \sqrt{m_\mu} + \sqrt{m_\tau})^2} = \frac{2}{3}
\end{equation}
emerges from the $\mathbb{Z}_3$ phase structure with $\theta = 2\pi/9$.

% ==============================================================================
\section{Epistemic Summary}
\label{sec:z6_epistemic}
% ==============================================================================

\begin{center}
\begin{tabular}{llc}
\toprule
\textbf{Result} & \textbf{Status} & \textbf{Tag} \\
\midrule
Steiner 120° angles & Mathematical theorem & [M] \\
Hexagonal packing optimality & Kepler-Hales theorem & [M] \\
$\mathbb{Z}_6$ boundary conditions & Derived from packing & [Dc] \\
Proton as $\mathbb{Z}_3$ fixed point & Derived & [Dc] \\
Positive Hessian (stability) & Derived & [Dc] \\
Neutron as dislocation & Identification & [I] \\
Dislocation energy $\approx 1.29$ MeV & Calibrated match & [Cal] \\
$SU(3)$ link variables & Constructed & [Dc] \\
Wilson loop area law & Derived & [Dc] \\
Three generations = harmonics & Identification & [I] \\
$\mathbb{Z}_2 \leftrightarrow$ electroweak & Open conjecture & [OPEN] \\
\bottomrule
\end{tabular}
\end{center}

% ==============================================================================
\section{Bridge to Chapter 1}
\label{sec:z6_bridge}
% ==============================================================================

While this chapter rigorously derives the stability of the baryonic sector from
$\mathbb{Z}_6$ geometry, the weak decays of lighter excitations (muon, tau, pion)
are modeled in Chapter~1 via the frozen projection operator
$\mathcal{P}_{\text{frozen}}$. The connection is:

\begin{center}
\begin{tabular}{lll}
\toprule
\textbf{Particle} & \textbf{This Chapter} & \textbf{Chapter 1} \\
\midrule
Proton & $\mathbb{Z}_3$ fixed point (stable) & Ledger benchmark \\
Neutron & Dislocation (metastable) & Junction relaxation case \\
Muon, Tau & --- & Brane-dominant modes \\
Pion & --- & Junction-pair composite \\
\bottomrule
\end{tabular}
\end{center}

The mathematical foundation here (this chapter) supports the physical mechanism
described there (Chapter~1).

% ==============================================================================
\section{Full Derivation Reference}
\label{sec:z6_full_ref}
% ==============================================================================

The complete step-by-step derivation, including all proofs, diagrams, and
intermediate calculations, is available in the standalone document:

\begin{center}
\fbox{\parbox{0.8\textwidth}{
\centering
\textbf{Z6\_PROGRAM\_COMPLETE\_DERIVATION.pdf}\\[0.5em]
\textit{Geometric Origin of Color Confinement:\\
From Hexagonal Packing to $SU(3)$ Emergence}\\[0.5em]
18 pages $\cdot$ Full proofs $\cdot$ All epistemic tags
}}
\end{center}

% End of Z6 content
