% ==============================================================================
% COVER LETTER FOR arXiv SUBMISSION
% Paper: Geometric Origin of Color Confinement
% ==============================================================================

\documentclass[11pt,a4paper]{letter}
\usepackage{geometry}
\geometry{margin=1in}
\usepackage{hyperref}

\signature{Igor Grčman\\Elastic Diffusive Cosmology Research Program}
\address{EDC Research Program\\Zagreb, Croatia\\igor.grcman@edc-research.org}

\begin{document}

\begin{letter}{arXiv Moderation Team\\Cornell University}

\opening{Dear Editors,}

I am submitting the manuscript entitled \textbf{``Geometric Origin of Color
Confinement: From Hexagonal Packing to $SU(3)$ Emergence''} for consideration
on arXiv in the High Energy Physics - Theory (hep-th) category.

\textbf{Summary of the Work:}

This paper presents a novel geometric derivation of color confinement and
$SU(3)$ gauge symmetry emergence. Starting from the established mathematical
result that hexagonal packing is optimal (Kepler-Hales theorem, proven 2005),
we show that flux tubes on a thick-brane interface in 5D spacetime naturally
crystallize into a lattice with $\mathbb{Z}_6$ rotational symmetry.

The key results are:

\begin{enumerate}
    \item \textbf{Explicit link variable construction:} We construct Wilson-type
    gauge variables $U_\ell \in SU(3)$ directly from the $\mathbb{Z}_3$ vortex
    structure of the hexagonal lattice.

    \item \textbf{Topological confinement:} We prove that the Wilson loop
    acquires a phase factor when encircling quarks, yielding confinement as
    a topological necessity rather than a dynamical phenomenon.

    \item \textbf{Proton stability:} We derive that the proton configuration
    is a local energy minimum protected by $\mathbb{Z}_3$ symmetry.

    \item \textbf{Neutron instability:} We explain the neutron-proton mass
    difference ($\Delta m \approx 1.29$ MeV) as the energy of a lattice
    dislocation.
\end{enumerate}

\textbf{Significance:}

This work provides a concrete geometric mechanism for confinement that does
not rely on perturbative QCD or lattice Monte Carlo simulations. The approach
connects disparate phenomena (proton stability, neutron decay, confinement)
through a single geometric principle.

\textbf{Epistemic Transparency:}

All assumptions are explicitly marked throughout the paper using epistemic
tags: [M] for mathematical theorems, [P] for postulates, [Dc] for derived
consequences, [I] for calibrations, and [OPEN] for unresolved questions.
This allows readers to trace every claim to its foundations.

\textbf{Suggested Category:} hep-th (primary), hep-ph (cross-list)

\textbf{Keywords:} color confinement, $SU(3)$ gauge theory, topological defects,
center vortices, hexagonal lattice, extra dimensions

Thank you for your consideration.

\closing{Sincerely,}

\end{letter}

\end{document}
