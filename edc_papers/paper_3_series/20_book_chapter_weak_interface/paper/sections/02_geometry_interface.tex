% ==============================================================================
% Section 1.3: Geometry and Interface
% ==============================================================================

\section{Geometry of the Brane Interface}
\label{sec:geometry_interface}

This section addresses a fundamental question: in a 5D bulk, how does our particular
4D universe get selected, and what makes weak-sector mechanics possible?

\subsection{The Continuum of 4D Submanifolds in 5D}

Geometrically, a 5D manifold contains a continuum of possible 4D submanifolds.
If we parameterize the fifth dimension by a coordinate $\xi$, then surfaces of
constant $\xi$ are 4D hypersurfaces. But the geometry is richer: 4D submanifolds
can have arbitrary orientations, curvatures, and embeddings.

The question is: \emph{why does physics select a particular 4D hypersurface as
``our universe''?}

In delta-function brane models, this is typically assumed rather than derived.
In EDC's thick-brane picture, the selection arises from \emph{dynamics}: the
brane is not imposed but emerges as a stable interface configuration \tagP{}.

\subsection{Mechanistic Selection: Boundary Conditions and Couplings}

Not every 4D submanifold can support the physics we observe. The EDC dynamics
selects a specific interface through \tagDc{}:

\paragraph{1. Boundary conditions.}
The brane has two faces: a bulk-facing (Plenum-facing) side at $y = -\delta/2$
and an observer-facing side at $y = +\delta/2$. Each face carries boundary
conditions that determine what modes can propagate and what couplings are allowed.

\paragraph{2. Coupling structure.}
The effective 4D couplings arise from overlap integrals of 5D mode profiles
across the brane thickness. This structure is not free: it is constrained by
the 5D dynamics and boundary conditions.

\paragraph{3. Stability requirements.}
The interface must be stable against small perturbations. An unstable interface
would not persist long enough to support the observed physics.

\subsection{The Viability Filter: What Makes a Universe Observable?}

We propose that the interface we call ``our universe'' satisfies a set of
\emph{viability conditions} \tagP{}/\tagDc{}:

\begin{tcolorbox}[guardrail, title={Viability Filter Conditions}]
\begin{enumerate}
  \item \textbf{Proton-Anchor Stability} \tagP{}:
        The proton configuration (modeled as a Y-junction in the brane layer)
        must be a stable minimum-energy topology. Without a stable anchor,
        there is no stable matter and no observers.

  \item \textbf{Ledger Closure}:
        Energy and quantum numbers must be conserved across the bulk$\to$brane$\to$3D
        pipeline. Without ledger closure, the mechanism is not self-consistent.

  \item \textbf{Suppressed Leakage} \tagP{}:
        Bulk modes must not leak freely into the 3D sector. If everything leaked,
        there would be no selection rules and no structured particle spectrum.
\end{enumerate}
\end{tcolorbox}

These conditions are not arbitrary: they are necessary for the existence of
stable matter and observable weak processes.

\subsection{Proton-Anchor Stability Principle}

\begin{tcolorbox}[mechanism, title={Proton-Anchor Stability}]
\textbf{Postulate} \tagP{}:
Our universe is stable because the proton Y-junction configuration represents a
local minimum of the 5D energy functional. The proton is not just ``the lightest
baryon''; it is the \emph{topological anchor} that stabilizes the brane-observer
interface.

\textbf{Consequence} \tagDc{}:
If the proton were unstable, baryonic matter would decay, and the conditions for
complex chemistry and observers would not persist.
\end{tcolorbox}

\paragraph{Falsifiability.}
This principle is falsifiable: if proton decay were observed with a lifetime
shorter than $\sim 10^{34}$ years \tagBL{}, the claim that the proton is a
stable anchor would require revision.

\paragraph{What this explains.}
The proton-anchor principle explains why EDC treats the neutron-to-proton
transition as a \emph{relaxation toward a stable minimum} rather than an
arbitrary decay. The proton is the endpoint because it is the stable configuration.

\subsection{Generative Closure Principle}
\label{sec:generative_closure_principle}

\begin{tcolorbox}[mechanism, title={Generative Closure}]
\textbf{Postulate} \tagP{}:
The electron sector (electron as ground-mode brane defect, neutrino as edge mode)
together with the proton anchor constitutes a \emph{closed generative substrate}.
All weak-sector outputs must be expressible in terms of these fundamental components.

\textbf{Consequence} \tagDc{}:
Weak decays do not produce arbitrary particles; they produce combinations of
$\{p, e^\pm, \nu, \bar\nu\}$ because these are the stable outputs allowed by
the interface mechanism.
\end{tcolorbox}

This principle constrains what can appear as a weak-sector output: not because
of an inserted selection rule, but because only certain modes survive the
projection through the observer-facing boundary.

\subsection{Where This Leads}

The geometry-interface picture establishes the \emph{arena} for weak-sector
dynamics. Section~\ref{sec:unified_pipeline} formalizes the \emph{mechanism}:
the unified pipeline (Absorption $\to$ Dissipation $\to$ Release) with explicit
energy flow, projection operators, and ledger closure requirements.
