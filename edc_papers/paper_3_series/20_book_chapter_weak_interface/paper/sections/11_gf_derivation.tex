% ==============================================================================
% Chapter 11: The Fermi Constant from Geometry
% Status: [Dc] numerical closure via electroweak relations + [P] mode overlap mechanism
% ==============================================================================

\section{The Fermi Constant from Geometry}
\label{sec:ch11_gf}

\begin{tcolorbox}[edcGuardrail, title=\textbf{Epistemic Status}]
This chapter consolidates the EDC treatment of the Fermi constant $G_F$:
\begin{itemize}[nosep]
    \item Structural pathway: $G_F$ emerges from integrating out a 5D mediator \tagDc{}
    \item Numerical closure: $G_F$ exact from electroweak relations + $\sin^2\theta_W = 1/4$ \tagDc{}
    \item Mode overlap: geometric suppression explains ``weakness'' \tagP{}
    \item Connection to V$-$A: chirality filter enters via boundary conditions \tagDc{}
\end{itemize}
\textbf{What is NOT claimed:} We do not derive the numerical value of $G_F$
from first principles alone. The derivation uses electroweak relations that
incorporate the measured $\alpha$ and $v$ (or equivalently $G_F$ itself).
The structural claim is that $G_F$'s \emph{smallness} has geometric origin.
\end{tcolorbox}

% ------------------------------------------------------------------------------
% BOOK-READY INTRODUCTION
% ------------------------------------------------------------------------------

\paragraph{Chapter overview.}
The Fermi constant $G_F \approx 1.17 \times 10^{-5}$ GeV$^{-2}$ sets the scale of
weak interactions. In the Standard Model, this small value comes from $W$-boson
exchange with $G_F = \sqrt{2}g^2/(8M_W^2)$. But \emph{why} is $G_F$ so small?
Why is the weak force ``weak''?

EDC offers a geometric answer: weak interactions are not fundamental gauge vertices
but \textbf{effective contact terms} arising from integrating out a brane-layer
mediator. The smallness of $G_F$ reflects:
\begin{enumerate}[nosep]
    \item The mediator mass gap $m_\phi$ (set by brane geometry)
    \item Mode overlap suppression (fermions are tightly localized)
    \item Chirality selection (only left-handed modes couple efficiently)
\end{enumerate}

This chapter presents both the \emph{structural pathway} (mechanism) and the
\emph{numerical closure} (quantitative derivation), carefully distinguishing what
is derived from what remains postulated.

% ------------------------------------------------------------------------------
% DERIVATION CHAIN BOX
% ------------------------------------------------------------------------------

\begin{tcolorbox}[colback=green!5, colframe=green!50!black,
    title=\textbf{Derivation Chain: What Is Independent vs.\ What Is Not}]
\begin{description}[style=nextline, leftmargin=1em, font=\normalfont\bfseries]
    \item[Independent EDC step \tagDer{}:]
        $\mathbb{Z}_6$ subgroup counting $\Rightarrow \sin^2\theta_W = 1/4$ (bare).
        \emph{This is the geometrically derived prediction.}

    \item[Standard physics step \tagBL{}:]
        RG running from lattice scale to $M_Z$ using known beta functions.

    \item[Derived identities \tagDc{}:]
        Electroweak coupling relations: $g^2 = 4\pi\alpha/\sin^2\theta_W$,
        $M_W = gv/2$, $G_F = g^2/(4\sqrt{2}M_W^2)$.

    \item[Circularity caveat (important):]
        The Higgs VEV $v = (\sqrt{2}G_F)^{-1/2} = 246.2$ GeV is experimentally
        determined \emph{from} $G_F$ (muon decay). Therefore:
        \textbf{$G_F$ ``exact agreement'' is a consistency closure within
        SM relations, not an independent EDC prediction.}
        The true independent prediction is $\sin^2\theta_W = 1/4$.
\end{description}
\end{tcolorbox}

% ------------------------------------------------------------------------------
% READER MAP
% ------------------------------------------------------------------------------

\begin{tcolorbox}[colback=blue!5, colframe=blue!50!black,
    title=\textbf{Reader Map: What This Chapter Establishes}]
\begin{description}[style=nextline, leftmargin=1em, font=\normalfont\bfseries]
    \item[Derived \tagDc{}:]
        $G_F = g^2/(4\sqrt{2}M_W^2)$ from electroweak relations;
        numerical value exact once $\sin^2\theta_W = 1/4$ is fixed;
        structural form $G_{\text{EDC}} \sim g_{\text{eff}}^2/m_\phi^2$;
        dimensional consistency.

    \item[Identified \tagI{}:]
        Mediator mass $\leftrightarrow$ brane thickness;
        overlap suppression $\leftrightarrow$ mode localization;
        chirality filter $\leftrightarrow$ V$-$A structure (Ch.~\ref{ch:va_structure}).

    \item[Postulated \tagP{}:]
        5D coupling $g_5$ normalization;
        explicit mode profiles;
        mediator spectrum from $\xi$-geometry;
        overlap integrals with explicit boundary conditions.

    \item[Open (not addressed):]
        First-principles $G_F$ without using $\alpha$ or $v$ as inputs;
        mediator mass from throat geometry;
        complete thick-brane BVP solution.
\end{description}
\end{tcolorbox}

% ==============================================================================
\subsection{Baseline: The Fermi Constant in Standard Model}
\label{sec:ch11_baseline}

The Fermi constant is the effective coupling strength of weak interactions \tagBL{}:
\begin{equation}
    G_F = 1.1663787(6) \times 10^{-5}~\text{GeV}^{-2}
    \label{eq:ch11_GF_value}
\end{equation}

In the Standard Model, $G_F$ arises from $W$-boson exchange \tagBL{}:
\begin{equation}
    G_F = \frac{\sqrt{2}}{8} \frac{g^2}{M_W^2} = \frac{g^2}{4\sqrt{2} M_W^2}
    \label{eq:ch11_GF_SM}
\end{equation}
where $g \approx 0.65$ is the $SU(2)_L$ gauge coupling and $M_W \approx 80.4$ GeV.

\paragraph{Why is $G_F$ small?}
In the SM, this is ``explained'' by $M_W$ being heavy. But why is $M_W \approx 80$ GeV?
That requires the Higgs mechanism with a VEV $v \approx 246$ GeV. The hierarchy
$G_F \sim 1/v^2$ is ultimately unexplained---it's an input, not an output.

% ==============================================================================
\subsection{Structural Pathway: Mediator Integration}
\label{sec:ch11_structural}

In EDC, we treat weak interactions not as fundamental gauge vertices but as
effective contact terms arising from thick-brane microphysics \tagDc{}.

\subsubsection{The Toy Setup}

Introduce a mediator field $\phi(x,y)$ localized in the brane layer with mass
gap $m_\phi$ \tagP{}:
\begin{equation}
    \mathcal{L}_\phi = \frac{1}{2}(\partial_\mu\phi)^2 + \frac{1}{2}(\partial_y\phi)^2
    - \frac{1}{2}m_\phi^2 \phi^2
    \label{eq:ch11_L_phi}
\end{equation}

The bulk-brane dynamics couple to the mediator at the bulk-facing boundary:
\begin{equation}
    \mathcal{L}_{\text{int}} = g_5 \, J(x) \, \phi(x, y = -\delta/2)
    \label{eq:ch11_L_int}
\end{equation}
where $J(x)$ is a source current from bulk pumping (e.g., junction relaxation).

\subsubsection{Tree-Level Integration}

Integrating out $\phi$ at tree level yields the effective contact interaction
\tagDc{}:
\begin{equation}
    \boxed{
    \mathcal{L}_{\text{eff}} = -\frac{g_5^2}{2m_\phi^2} \,
    \mathcal{O}_{\text{overlap}} \, J(x) J(x)
    }
    \label{eq:ch11_Leff}
\end{equation}
where $\mathcal{O}_{\text{overlap}}$ encodes wavefunction overlaps and
boundary-condition effects.

\begin{tcolorbox}[edcCornerstone, title=\textbf{Physical Interpretation (Canonical)}]
Equation~\eqref{eq:ch11_Leff} is \textbf{not} a fundamental ``weak vertex'';
it is the low-energy residue of a 5D bulk$\to$brane transfer process \tagDc{}.

The source $J(x)$ represents bulk-facing pumping into the brane layer via the
mediator $\phi$. Integrating out $\phi$ compresses that transfer into an
effective local $JJ$ term. The apparent smallness of the coupling is therefore
\textbf{geometric suppression}---set by:
\begin{itemize}[nosep]
    \item Mediator mass gap $m_\phi$ (from brane thickness/KK spectrum)
    \item Mode-profile overlap $\mathcal{O}_{\text{overlap}}$ (localization)
    \item Boundary-condition factor $\mathcal{O}_{\text{BC}}$ (chirality filter)
\end{itemize}
\end{tcolorbox}

\subsubsection{The Effective Coupling}

Define the effective coupling \tagP{}:
\begin{equation}
    g_{\text{eff}} \equiv g_5 \times \mathcal{O}_{\text{overlap}}
    \times \mathcal{O}_{\text{BC}}
    \label{eq:ch11_geff}
\end{equation}
so that:
\begin{equation}
    \boxed{
    G_{\text{EDC}} \sim \frac{g_{\text{eff}}^2}{m_\phi^2}
    }
    \label{eq:ch11_GEDC}
\end{equation}
with $[G_{\text{EDC}}] = [E]^{-2}$ as required.

% ==============================================================================
\subsection{Numerical Closure via Electroweak Relations}
\label{sec:ch11_numerical}

While the structural pathway is incomplete (open factors), EDC achieves
\emph{numerical closure} through electroweak relations once $\sin^2\theta_W$ is
fixed by geometry.

\subsubsection{The Derivation Chain}

\begin{theorem}[$G_F$ from Electroweak Unification {\normalfont \tagDc{}}]
\label{thm:ch11_GF}
From the EDC-derived $\sin^2\theta_W = 1/4$ at the lattice scale
(Chapter~\ref{ch:z6_program}), after RG running to $M_Z$:
\begin{align}
    \sin^2\theta_W(M_Z) &= 0.2314 \quad \text{(0.08\% from PDG)} \\
    g^2 &= \frac{4\pi\alpha}{\sin^2\theta_W} = 0.4246 \\
    M_W &= \frac{gv}{2} = \frac{0.6516 \times 246.2}{2} = 80.2 \text{ GeV}
\end{align}

The Fermi constant then follows:
\begin{equation}
    \boxed{
    G_F = \frac{g^2}{4\sqrt{2}M_W^2} = \frac{0.4246}{4\sqrt{2}(80.2)^2}
    = 1.166 \times 10^{-5} \text{ GeV}^{-2}
    }
    \label{eq:ch11_GF_derived}
\end{equation}

\textbf{Experimental:} $G_F^{\text{exp}} = 1.166 \times 10^{-5}$ GeV$^{-2}$
\tagBL{} --- \textbf{exact within adopted EW identities}.
\end{theorem}

\begin{remark}[Self-Consistency, Not Independent Prediction]
\label{rem:ch11_firewall}
The ``exact agreement'' for $G_F$ reflects the self-consistency of electroweak
relations, \textbf{not} an independent EDC prediction.

\textbf{The circularity caveat:} In SM conventions, the Higgs VEV is determined
from $G_F$ via $v = (\sqrt{2}G_F)^{-1/2}$. Since we use $v$ as input to compute
$M_W$, and then derive $G_F$ from $M_W$, the ``exact'' result is a consistency
identity. This is analogous to computing $G_F$ from $G_F$---not a derivation.

\textbf{The true EDC prediction} is:
\begin{equation}
    \sin^2\theta_W(\mu_{\text{lattice}}) = \frac{|\mathbb{Z}_2|}{|\mathbb{Z}_6|}
    = \frac{2}{6} = \frac{1}{4}
    \label{eq:ch11_sin2_input}
\end{equation}

After RG running, this gives $\sin^2\theta_W(M_Z) = 0.2314$, which agrees with
PDG at 0.08\%. \textbf{This} is the non-trivial, falsifiable prediction.

Everything else ($g^2$, $M_W$, $G_F$) follows from:
\begin{itemize}[nosep]
    \item Standard electroweak unification relations \tagBL{}
    \item Standard RG running from lattice scale to $M_Z$ \tagBL{}
    \item Measured values of $\alpha$ and $v$ (where $v$ depends on $G_F$) \tagBL{}
\end{itemize}

\textbf{Bottom line:} $G_F$ numerical closure is \emph{conditional} on SM
relations. The independent EDC content is $\sin^2\theta_W = 1/4$.
\end{remark}

% ==============================================================================
\subsection{Mode Overlap: Why $G_F$ Is Small}
\label{sec:ch11_overlap}

The structural pathway identifies \emph{why} weak interactions are weak:
geometric suppression from mode localization.

\subsubsection{The Overlap Integral}

The 5D Fermi coupling has dimension $[G_5] = [E]^{-3}$. To get the 4D coupling
$[G_F] = [E]^{-2}$, we integrate over the fifth dimension \tagDc{}:
\begin{equation}
    G_F = G_5 \int_0^\infty dz \, |f_L(z)|^4 = G_5 \times I_4
    \label{eq:ch11_overlap}
\end{equation}
where $f_L(z)$ is the left-handed fermion mode profile and $I_4$ has dimension
of length (inverse energy in natural units).

\subsubsection{Estimating $I_4$}

For the asymmetric mass profile $m(z) = m_0(1 - e^{-z/\lambda})$
(Chapter~\ref{ch:va_structure}), the left-handed mode is:
\begin{equation}
    f_L(z) = N_L \exp\left(-m_0\chi(z)\right), \quad
    \chi(z) = z - \lambda(1 - e^{-z/\lambda})
    \label{eq:ch11_fL}
\end{equation}

The mode is localized at $z = 0$ with effective width $\sigma_L \sim 1/m_0$.
For $m_0 \sim 200$ MeV:
\begin{equation}
    I_4 \sim \frac{1}{\sigma_L} \sim m_0 \sim 200 \text{ MeV}
    \label{eq:ch11_I4_estimate}
\end{equation}

\subsubsection{Order-of-Magnitude Check}

Combining $G_5 \sim g_5^2/M_5^2$ with $I_4$ \tagP{}:
\begin{equation}
    G_F \sim \frac{g_5^2}{M_5^2} \times I_4
    \sim \frac{(4\pi)^2}{(200 \text{ GeV})^2} \times 0.2 \text{ GeV}
    \sim 10^{-3} \text{ GeV}^{-2}
    \label{eq:ch11_GF_estimate}
\end{equation}

This is $\sim 100\times$ larger than observed! The discrepancy indicates:
\begin{itemize}[nosep]
    \item Additional suppression from wave function normalization
    \item Factors of $4\pi$ from angular integrals
    \item The SM relation $G_F = g^2/(4\sqrt{2}M_W^2)$ captures the correct physics
\end{itemize}

\begin{tcolorbox}[colback=yellow!5, colframe=orange!60!black,
    title=\textbf{Honest Assessment: Mode Overlap Status (YELLOW-B)}]
The mode overlap mechanism provides the \textbf{qualitative understanding}
of why $G_F$ is small:
\begin{itemize}[nosep]
    \item Fermions are tightly localized ($\sigma_L \ll$ brane thickness)
    \item The overlap of four mode functions is highly suppressed
    \item This is the geometric origin of weak interaction ``weakness''
\end{itemize}

However, the \textbf{quantitative precision} requires the full electroweak
machinery. The mode overlap is \tagP{}; the numerical closure is \tagDc{}.
\end{tcolorbox}

\subsubsection{What Exactly Is Missing for RED-C $\to$ GREEN-A?}

To upgrade mode overlap from qualitative (YELLOW-B) to quantitative (GREEN-A),
the following concrete calculations are required:

\begin{enumerate}[nosep]
    \item \textbf{5D gauge coupling $g_5$ from action normalization:}
          Derive $g_5$ from the canonical normalization of the 5D gauge field
          action, not from dimensional estimates. This requires specifying the
          5D gauge kinetic term and its reduction to 4D.

    \item \textbf{Mediator mass $m_\phi$ from KK reduction:}
          Perform the Kaluza-Klein reduction along the $\xi$-direction (throat
          geometry) to obtain the spectrum. Identify the lowest massive mode
          as the mediator and express $m_\phi$ in terms of geometric parameters
          ($R_\xi$, throat length, etc.).

    \item \textbf{Mode profiles $f_L(z)$ from thick-brane BVP:}
          Solve the boundary value problem for fermion localization with
          explicit boundary conditions at both brane faces. Normalize the
          solutions and compute the overlap integral $I_4 = \int |f_L|^4 dz$
          exactly, not by order-of-magnitude.

    \item \textbf{Boundary-condition factor $\mathcal{O}_{\text{BC}}$:}
          Evaluate the chirality projection and frozen-mode operators on the
          actual mode profiles to get the numerical suppression factor.
\end{enumerate}

Until these are computed, the mode overlap remains a \textbf{mechanism}, not a
\textbf{derivation}.

% ==============================================================================
\subsection{Connection to V$-$A Structure}
\label{sec:ch11_va}

The chirality filter from Chapter~\ref{ch:va_structure} enters the effective
coupling via the boundary-condition factor $\mathcal{O}_{\text{BC}}$.

\paragraph{Left-handed localization.}
The asymmetric mass profile selects chirality:
\begin{align}
    \psi_L &\propto \exp\left(-\int_0^z m(z')\,dz'\right) \quad \text{normalizable} \\
    \psi_R &\propto \exp\left(+\int_0^z m(z')\,dz'\right) \quad \text{non-normalizable}
\end{align}

Only left-handed fermions couple efficiently to the brane-layer mediator.
This is the geometric origin of V$-$A structure in weak currents.

\paragraph{Quantitative suppression.}
The right-handed mode amplitude at the brane is suppressed by \tagDc{}:
\begin{equation}
    \frac{|\psi_R(0)|}{|\psi_L(0)|} \sim e^{-m_0\lambda} \sim e^{-200 \text{ MeV} \times 1 \text{ fm}}
    \sim e^{-1} \sim 0.37
    \label{eq:ch11_chiral_suppression}
\end{equation}

For the $|f|^4$ overlap, this becomes $(0.37)^4 \approx 0.02$---roughly 50-fold
suppression of right-handed contributions, consistent with the observed
V$-$A dominance.

% ==============================================================================
\subsection{Summary: What Determines $G_F$?}
\label{sec:ch11_summary_table}

\begin{table}[ht]
\centering
\caption{What determines $G_F$ in EDC (color-coded by derivation level)}
\label{tab:ch11_summary}
\begin{tabular}{p{3.5cm}p{4.5cm}cc}
\toprule
\textbf{Factor} & \textbf{Physical Origin} & \textbf{Tag} & \textbf{Level} \\
\midrule
\multicolumn{4}{l}{\textit{GREEN-A: Electroweak consistency closure}} \\
$\sin^2\theta_W = 1/4$ & $\mathbb{Z}_6$ subgroup counting & \tagDer{} & GREEN-A \\
$g^2 = 4\pi\alpha/\sin^2\theta_W$ & Electroweak unification & \tagDc{} & GREEN-A \\
$M_W = gv/2$ & Higgs mechanism (+$v$ caveat) & \tagDc{} & GREEN-A \\
$G_F = g^2/(4\sqrt{2}M_W^2)$ & Electroweak relation & \tagDc{} & GREEN-A \\
\midrule
\multicolumn{4}{l}{\textit{YELLOW-B: Geometric suppression intuition}} \\
Mode overlap $I_4$ & Fermion localization & \tagP{} & YELLOW-B \\
Why weak is ``weak'' & Overlap suppression & \tagI{} & YELLOW-B \\
\midrule
\multicolumn{4}{l}{\textit{RED-C: Full 5D first-principles (open)}} \\
5D coupling $g_5$ & Action normalization & \tagP{} & RED-C (OPR-19) \\
Mediator mass $m_\phi$ & $\xi$-geometry KK reduction & \tagP{} & RED-C (OPR-20) \\
Mode profiles $f_L(z)$ & Thick-brane BVP & (open) & RED-C (OPR-21) \\
First-principles $G_F$ & Complete derivation & (open) & RED-C (OPR-22) \\
\bottomrule
\end{tabular}
\end{table}

% ==============================================================================
\subsection{Stoplight Verdict}
\label{sec:ch11_verdict}

\begin{table}[ht]
\centering
\caption{Chapter 11 overall verdict}
\label{tab:ch11_verdict}
\begin{tabular}{lcc}
\toprule
\textbf{Claim} & \textbf{Level} & \textbf{Tag} \\
\midrule
$\sin^2\theta_W = 1/4$ (independent prediction) & GREEN-A & \tagDer{} \\
$G_F$ via EW relations ($v$ caveat) & GREEN-A & \tagDc{} \\
Structural form $G \sim g^2/m^2$ & GREEN-A & \tagDc{} \\
Connection to V$-$A (Ch.~\ref{ch:va_structure}) & GREEN-A & \tagDc{} \\
\midrule
Mode overlap mechanism & YELLOW-B & \tagP{} \\
Why weak is ``weak'' & YELLOW-B & \tagP{}/\tagI{} \\
\midrule
First-principles $G_F$ & RED-C & (open) (OPR-22) \\
\bottomrule
\end{tabular}
\end{table}

\begin{tcolorbox}[colback=green!5, colframe=green!50!black,
    title=\textbf{Chapter 11 Summary}]
\textbf{The strongest independent claim (GREEN-A):}
\begin{quote}
EDC predicts $\sin^2\theta_W = 1/4$ (bare) from $\mathbb{Z}_6$ subgroup counting.
After standard RG running, this gives $\sin^2\theta_W(M_Z) = 0.2314$, agreeing
with PDG at \textbf{0.08\%}. This is a non-trivial, falsifiable prediction.
\end{quote}

\textbf{The conditional closure (GREEN-A with caveat):}
\begin{enumerate}[nosep]
    \item $G_F = 1.166 \times 10^{-5}$ GeV$^{-2}$ from electroweak relations,
          \textbf{but} this uses $v$ which is itself determined from $G_F$.
          This is consistency, not independent prediction.
    \item Structural form $G_{\text{EDC}} \sim g_{\text{eff}}^2/m_\phi^2$
          established \tagDc{}.
    \item V$-$A connection via chirality filter \tagDc{}.
\end{enumerate}

\textbf{The geometric intuition (YELLOW-B):}
Mode overlap suppression explains \emph{why} $G_F$ is small (qualitative).

\textbf{The open frontier (RED-C):}
First-principles derivation requires: $g_5$ from action, $m_\phi$ from KK,
profiles from BVP. Until then, quantitative mode overlap remains postulated.
\end{tcolorbox}

\textbf{Bottom line:} The true EDC prediction is $\sin^2\theta_W = 1/4$ (0.08\%
agreement after RG). The $G_F$ numerical closure is a consistency check within
SM relations, not an independent prediction. The structural pathway provides
geometric understanding of weak interaction ``weakness,'' but quantitative
first-principles derivation remains open.

% ==============================================================================
% SANITY SKELETON: Chain map + dimensional checks + attack surface
% ==============================================================================
%!TEX root = ../EDC_Part_II_Weak_Sector.tex
% ==============================================================================
% G_F Sanity Skeleton: Chain Map + Dimensional Checks + Attack Surface
% Status: Audit-ready scaffold for G_F pathway (OPR-19--22)
% ==============================================================================

\subsection{G$_F$ Sanity Skeleton: Chain, Dimensions, and Open Inputs}
\label{sec:ch11_sanity_skeleton}

This subsection consolidates the G$_F$ derivation chain into an audit-ready format:
explicit epistemic tags, dimensional checks, and a map of where circularity could hide.
The goal is \emph{not} to claim first-principles closure but to make every input
and open parameter visible.

% ------------------------------------------------------------------------------
\subsubsection{Chain Map: SM Side vs.\ EDC Side}
\label{sec:ch11_chain_map}

\begin{table}[ht]
\centering
\caption{G$_F$ Chain Map with epistemic tags and circularity notes}
\label{tab:ch11_chain_map}
\small
\begin{tabular}{p{3.8cm}p{3.5cm}cp{3.5cm}}
\toprule
\textbf{Step} & \textbf{Equation/Source} & \textbf{Tag} & \textbf{Circularity Risk} \\
\midrule
\multicolumn{4}{l}{\textit{SM-side relations (electroweak consistency)}} \\
\addlinespace
$G_F$ definition & $G_F = 1.166 \times 10^{-5}$ GeV$^{-2}$ & \tagBL{} & Reference target \\
$G_F$ from $W$ exchange & $G_F = g^2/(4\sqrt{2}M_W^2)$ & \tagBL{} & SM relation \\
$g^2$ from $\alpha$, $\theta_W$ & $g^2 = 4\pi\alpha/\sin^2\theta_W$ & \tagBL{} & Uses measured $\alpha$ \\
$M_W$ from Higgs & $M_W = gv/2$ & \tagBL{} & Uses $v = (\sqrt{2}G_F)^{-1/2}$ \\
\addlinespace
\rowcolor{yellow!20}
\textbf{Circularity:} & \multicolumn{3}{l}{$v$ depends on $G_F$ $\Rightarrow$ ``$G_F$ exact'' is consistency, not prediction} \\
\midrule
\multicolumn{4}{l}{\textit{EDC-side structural mapping}} \\
\addlinespace
$\sin^2\theta_W = 1/4$ & $\mathbb{Z}_6$ subgroup counting & \tagDer{} & \textbf{Independent prediction} \\
RG running to $M_Z$ & Standard $\beta$-functions & \tagBL{} & None (physics) \\
Structural form & $G_{\text{EDC}} \sim g_{\text{eff}}^2/m_\phi^2$ & \tagDc{} & Structure, not value \\
Effective coupling & $g_{\text{eff}} \simeq g_5 \cdot \mathcal{O}_{\text{overlap}}$ & \tagP{} & Overlap not computed \\
\midrule
\multicolumn{4}{l}{\textit{Open inputs (first-principles derivation)}} \\
\addlinespace
5D gauge coupling $g_5$ & Canonical normalization & (open) & OPR-19: not derived \\
Mediator mass $m_\phi$ & KK reduction & (open) & OPR-20: not computed \\
Mode profiles $f_L(z)$ & Thick-brane BVP & (open) & OPR-21: not solved \\
Overlap integral $I_4$ & $\int |f_L|^4 dz$ & (open) & Requires OPR-21 \\
\bottomrule
\end{tabular}
\end{table}

% ------------------------------------------------------------------------------
\subsubsection{Dimensional Consistency Check}
\label{sec:ch11_dimensions}

The Fermi constant has dimension $[G_F] = [E]^{-2}$ in natural units.
Any EDC effective operator must reproduce this scaling.

\paragraph{4-Fermi operator structure.}
The effective Lagrangian from mediator integration (Eq.~\ref{eq:ch11_Leff}) has the form:
\begin{equation}
    \mathcal{L}_{\text{eff}} \sim \frac{g_{\text{eff}}^2}{m_\phi^2}
    (\bar\psi_L \gamma^\mu \psi_L)(\bar\psi_L \gamma_\mu \psi_L)
    \label{eq:ch11_dim_check}
\end{equation}

\paragraph{Dimensional analysis.}
\begin{align}
    [g_{\text{eff}}] &= [E]^0 \quad \text{(dimensionless 4D coupling)} \\
    [m_\phi] &= [E]^1 \quad \text{(mediator mass)} \\
    [g_{\text{eff}}^2/m_\phi^2] &= [E]^{-2} \quad \checkmark
\end{align}

\paragraph{What plays the role of $M$?}
In the SM, $G_F \sim 1/v^2 \sim 1/M_W^2$. In EDC:
\begin{itemize}[nosep]
    \item The \emph{mediator mass} $m_\phi$ sets the scale (not $M_W$ directly)
    \item For consistency: $m_\phi \sim M_W \sim 80$ GeV
    \item This is \emph{identified} \tagI{}, not \emph{derived}
\end{itemize}

\begin{tcolorbox}[colback=green!5, colframe=green!50!black,
    title=\textbf{Dimensional Check: PASS}]
The EDC effective operator $\mathcal{L}_{\text{eff}} \sim g_{\text{eff}}^2/m_\phi^2$
has the correct dimension $[E]^{-2}$ for matching $G_F$.
\\[0.5em]
\textbf{Note:} This is a consistency check, not a derivation. The numerical value
requires computing $g_{\text{eff}}$ and $m_\phi$ from first principles.
\end{tcolorbox}

% ------------------------------------------------------------------------------
\subsubsection{Circularity Attack-Surface Analysis}
\label{sec:ch11_attack_surface}

\begin{tcolorbox}[colback=red!5!white, colframe=red!50!black,
    title=\textbf{No-Smuggling Guardrail: Where Circularity Could Hide}]

\textbf{Potential attack points:}
\begin{enumerate}[nosep]
    \item \textbf{$v = 246$ GeV input:} The Higgs VEV is experimentally determined
          \emph{from} $G_F$. Using $v$ to compute $M_W$, then $G_F$ from $M_W$,
          is circular. \textbf{Status:} Acknowledged in Remark~\ref{rem:ch11_firewall}.

    \item \textbf{$\alpha$ input:} The fine-structure constant is an independent
          measurement (QED, not weak). \textbf{Status:} Legitimate baseline \tagBL{}.

    \item \textbf{$m_\phi \sim M_W$ identification:} If we \emph{choose} $m_\phi = M_W$
          to match $G_F$, that is calibration, not derivation.
          \textbf{Status:} Explicitly marked \tagI{} (OPR-20).

    \item \textbf{Overlap normalization:} If $\mathcal{O}_{\text{overlap}}$ is tuned
          to give the right $G_F$, that is smuggling.
          \textbf{Status:} Not tuned; left as (open) (OPR-21).

    \item \textbf{$g_5$ choice:} If $g_5 \sim 4\pi$ is assumed without derivation,
          the ``geometric suppression'' claim is weakened.
          \textbf{Status:} Explicitly postulated \tagP{} (OPR-19).
\end{enumerate}

\textbf{What the skeleton prevents:}
\begin{itemize}[nosep]
    \item Unlabeled tuning of parameters
    \item Implicit use of $G_F$ to ``derive'' $G_F$
    \item Conflating SM consistency with EDC prediction
\end{itemize}

\textbf{What remains for true derivation:}
\begin{itemize}[nosep]
    \item Derive $g_5$ from 5D gauge action normalization (OPR-19)
    \item Compute $m_\phi$ from KK reduction of throat geometry (OPR-20)
    \item Solve fermion BVP for explicit mode profiles (OPR-21)
    \item Assemble all factors without calibration (OPR-22)
\end{itemize}
\end{tcolorbox}

% ------------------------------------------------------------------------------
\subsubsection{Minimal Closure Plan}
\label{sec:ch11_closure_plan}

\begin{tcolorbox}[colback=yellow!5!white, colframe=yellow!60!black,
    title=\textbf{G$_F$ Chain Stoplight: OPR-19--22}]

\textbf{\textcolor{OliveGreen}{GREEN} --- Already closed:}
\begin{itemize}[nosep]
    \item Dimensional consistency of effective operator \tagDc{}
    \item Structural form $G_{\text{EDC}} \sim g_{\text{eff}}^2/m_\phi^2$ \tagDc{}
    \item Numerical closure via SM relations (with $v$ caveat) \tagDc{}
    \item $\sin^2\theta_W = 1/4$ independent prediction (0.08\% after RG) \tagDer{}
\end{itemize}

\textbf{\textcolor{YellowOrange}{YELLOW} --- Structurally identified:}
\begin{itemize}[nosep]
    \item Mode overlap mechanism for ``why weak is weak'' \tagP{}
    \item Mediator integration picture \tagP{}
    \item Brane-localized gauge embedding (Ch.~\ref{sec:ch9_su2_embedding}) \tagP{}
    \item Effective coupling $g_{\text{eff}} \simeq g_2$ (up to brane terms) \tagP{}
\end{itemize}

\textbf{\textcolor{BrickRed}{RED} --- First-principles open:}
\begin{itemize}[nosep]
    \item OPR-19: $g_5$ from canonical 5D gauge action normalization
    \item OPR-20: $m_\phi$ from KK spectrum of throat geometry
    \item OPR-21: Mode profiles $f_L(z)$ from solved thick-brane BVP
    \item OPR-22: Complete $G_F$ derivation without SM circularity
\end{itemize}

\medskip
\noindent\fbox{\parbox{0.94\textwidth}{\small
\textbf{G$_F$ sanity skeleton:} Chain map + dimensions + attack-surface explicit.
The true independent prediction is $\sin^2\theta_W = 1/4$; numerical $G_F$ closure
uses SM relations (consistency, not derivation). First-principles $G_F$ requires
solving OPR-19--21 without calibration.}}
\end{tcolorbox}

% ------------------------------------------------------------------------------
\subsubsection{What Would Close Each OPR?}
\label{sec:ch11_closure_targets}

\begin{table}[ht]
\centering
\caption{OPR-19--22 closure targets}
\label{tab:ch11_opr_closure}
\small
\begin{tabular}{clll}
\toprule
\textbf{OPR} & \textbf{Item} & \textbf{Closure Requirement} & \textbf{Would Yield} \\
\midrule
19 & $g_5$ & Derive from $\int d^5x \, (-\frac{1}{4}F_{MN}F^{MN})$ & $g_5^2$ in terms of $L_5$ \\
20 & $m_\phi$ & KK reduction: $m_\phi^2 = (n\pi/L_\xi)^2 + \ldots$ & $m_\phi$ from geometry \\
21 & $f_L(z)$ & Solve $[\partial_z^2 - m(z)^2]f = \lambda f$ with BCs & Normalized profiles \\
22 & $G_F$ & Combine 19--21: $G_F = g_5^2 I_4 / m_\phi^2$ & First-principles value \\
\bottomrule
\end{tabular}
\end{table}

\paragraph{Upgrade path.}
Once OPR-19--21 are closed, OPR-22 follows automatically. The chain is:
\begin{equation}
    \boxed{
    g_5 \text{ (OPR-19)} + m_\phi \text{ (OPR-20)} + I_4 \text{ (OPR-21)}
    \quad\Rightarrow\quad
    G_F = \frac{g_5^2 \, I_4}{m_\phi^2} \text{ (OPR-22)}
    }
    \label{eq:ch11_upgrade_path}
\end{equation}

Until then, $G_F$ numerical closure relies on SM electroweak relations,
which is a \emph{consistency check}, not an independent EDC derivation.


% ==============================================================================
% CANONICAL g5 + KK SPECTRUM: Tightening OPR-19/20 closure path
% ==============================================================================
%!TEX root = ../EDC_Part_II_Weak_Sector.tex
% ==============================================================================
% Chapter 11: Canonical g_5 Normalization and KK Spectrum Tightening
% Status: Tightens OPR-19/20 closure path (no numerics yet)
% ==============================================================================

\subsection{Canonical \texorpdfstring{$g_5$}{g5} Normalization and KK Spectrum}
\label{sec:ch11_g5_kk}

This subsection tightens the $G_F$ derivation chain by:
\begin{enumerate}[nosep]
    \item Deriving the canonical normalization of the 5D gauge coupling $g_5$
    \item Establishing the KK eigenvalue structure for the mediator mass $m_\phi$
\end{enumerate}
The goal is to \emph{remove ambiguity}, not to close numerics. Both OPR-19 and OPR-20
remain RED-C, but with a mathematically concrete closure path.

% ------------------------------------------------------------------------------
\subsubsection{Canonical $g_5$ Normalization from 5D Action}
\label{sec:ch11_g5_canonical}

\paragraph{Starting point: 5D gauge action.}
Consider a gauge field $A_M$ propagating in five dimensions with action \tagP{}:
\begin{equation}
    S_{\text{5D}} = -\frac{1}{4g_5^2} \int d^4x \int_0^\ell d\xi \; F_{MN} F^{MN}
    \label{eq:ch11_5d_gauge_action}
\end{equation}
where $M,N \in \{0,1,2,3,5\}$, the extra dimension $\xi \in [0, \ell]$, and $g_5$ is the
5D gauge coupling with dimension $[g_5] = [E]^{-1/2}$ in natural units.

\paragraph{Alternative convention.}
Some authors absorb $g_5^{-2}$ into the field normalization. We use the convention
above because it makes the coupling explicit. The physics is identical.

\paragraph{KK decomposition.}
Decompose the 4D gauge field component as:
\begin{equation}
    A_\mu(x,\xi) = \sum_{n=0}^\infty A_\mu^{(n)}(x) \, \chi_n(\xi)
    \label{eq:ch11_kk_decomp}
\end{equation}
where $\{\chi_n(\xi)\}$ are orthonormal mode functions satisfying:
\begin{equation}
    \int_0^\ell d\xi \; \chi_m(\xi) \chi_n(\xi) = \delta_{mn}
    \label{eq:ch11_chi_orthonorm}
\end{equation}

\paragraph{4D effective action.}
Substituting into~\eqref{eq:ch11_5d_gauge_action} and using orthonormality:
\begin{equation}
    S_{\text{4D}} = -\frac{1}{4g_5^2} \sum_n \int d^4x \; F_{\mu\nu}^{(n)} F^{(n)\mu\nu}
    \label{eq:ch11_4d_effective}
\end{equation}
For canonical 4D kinetic terms $-\frac{1}{4} F_{\mu\nu}^{(n)} F^{(n)\mu\nu}$, we identify:
\begin{equation}
    \boxed{
    g_4^2 = g_5^2
    }
    \label{eq:ch11_g4_g5_relation}
\end{equation}
This is \emph{not} a typo: with the normalization convention~\eqref{eq:ch11_5d_gauge_action}
and orthonormal modes~\eqref{eq:ch11_chi_orthonorm}, the 4D and 5D couplings are numerically equal.

\paragraph{Where the extra dimension enters.}
The 5D nature enters through:
\begin{enumerate}[nosep]
    \item The mode spectrum $m_n$ (eigenvalues of KK equation)
    \item The overlap integrals with fermion profiles (coupling strengths)
    \item The brane kinetic terms (if present), which modify the zero-mode coupling
\end{enumerate}

\begin{tcolorbox}[colback=blue!5, colframe=blue!50!black,
    title=\textbf{Dimensional Sanity: $g_5$ and $g_4$}]
\begin{align}
    [g_5] &= [E]^{-1/2} \quad \text{(5D coupling)} \label{eq:ch11_dim_g5} \\
    [g_4] &= [E]^0 \quad \text{(dimensionless 4D coupling)} \label{eq:ch11_dim_g4} \\
    [G_F] &= [E]^{-2} \quad \text{(Fermi constant)} \label{eq:ch11_dim_GF}
\end{align}
\textbf{Consistency:} With orthonormal modes, $g_4 = g_5$ (numerically).
The mode normalization absorbs the factor of $\ell$.

\medskip
\textbf{Alternative:} If modes are normalized as $\int d\xi \, \chi^2 = \ell$, then
$g_4 = g_5/\sqrt{\ell}$. Both conventions give the same physics.
\end{tcolorbox}

\paragraph{Brane kinetic terms (optional extension).}
If brane-localized gauge kinetic terms are present \tagP{}:
\begin{equation}
    S_{\text{brane}} = -\frac{\kappa}{4} \int d^4x \; F_{\mu\nu} F^{\mu\nu} \Big|_{\xi = 0}
    \label{eq:ch11_brane_kinetic}
\end{equation}
then the effective 4D coupling is modified:
\begin{equation}
    \frac{1}{g_{\text{eff}}^2} = \frac{1}{g_5^2} + \kappa
    \quad\Rightarrow\quad
    g_{\text{eff}} \simeq g_5 \quad \text{for } \kappa \ll g_5^{-2}
    \label{eq:ch11_geff_brane}
\end{equation}
The brane kinetic term $\kappa$ is currently \textbf{[OPEN]} and not derived.

% ------------------------------------------------------------------------------
\subsubsection{KK Spectrum and Mediator Mass $m_\phi$}
\label{sec:ch11_kk_spectrum}

\paragraph{The eigenvalue problem.}
The KK mode functions $\chi_n(\xi)$ satisfy the eigenvalue equation:
\begin{equation}
    -\partial_\xi^2 \chi_n(\xi) = m_n^2 \chi_n(\xi)
    \label{eq:ch11_kk_eigenvalue}
\end{equation}
subject to boundary conditions at $\xi = 0$ and $\xi=\ell$.

\paragraph{Boundary conditions and spectrum.}
Three canonical choices yield different spectra:

\begin{center}
\begin{tabular}{llcc}
\toprule
\textbf{BC Type} & \textbf{Conditions} & \textbf{Zero Mode?} & \textbf{Spectrum $m_n$} \\
\midrule
Neumann--Neumann & $\chi'(0) = \chi'(\ell) = 0$ & Yes & $n\pi/\ell$ \\
Dirichlet--Dirichlet & $\chi(0) = \chi(\ell) = 0$ & No & $(n+1)\pi/\ell$ \\
Mixed (N--D) & $\chi'(0) = 0$, $\chi(\ell) = 0$ & No & $(n+\tfrac{1}{2})\pi/\ell$ \\
\bottomrule
\end{tabular}
\end{center}

\paragraph{The mediator mass scale.}
The first massive mode (or zero mode if present) sets the mediator mass:
\begin{equation}
    \boxed{
    m_\phi = \frac{x_1}{\ell}
    }
    \label{eq:ch11_mphi_scale}
\end{equation}
where $x_1$ is a dimensionless constant depending on boundary conditions:
\begin{itemize}[nosep]
    \item $x_1 = 0$ for N--N zero mode (massless)
    \item $x_1 = \pi$ for D--D or N--N first massive mode
    \item $x_1 = \pi/2$ for mixed BC first mode
\end{itemize}

\paragraph{Identification vs.\ derivation.}
The identification $m_\phi \sim M_W \approx 80$ GeV is currently \tagI{}:
\begin{equation}
    m_\phi \sim M_W \quad\Rightarrow\quad \ell \sim \frac{\pi}{M_W} \approx 0.04 \text{ fm}
    \label{eq:ch11_ell_estimate}
\end{equation}
This is \textbf{not} derived from EDC first principles. The derivation would require:
\begin{enumerate}[nosep]
    \item Deriving the brane layer thickness $\ell$ from membrane tension $\sigma$
    \item Specifying the boundary conditions from physical principles
    \item Computing $x_1$ for the actual BC configuration
\end{enumerate}

\begin{tcolorbox}[colback=yellow!5!white, colframe=yellow!60!black,
    title=\textbf{KK Spectrum: What Is [Dc] vs.\ [I] vs.\ [OPEN]}]
\begin{description}[nosep, font=\normalfont\bfseries]
    \item[[Dc]:] The form $m_\phi = x_1/\ell$ from KK eigenvalue equation
    \item[[I]:] The numerical value $m_\phi \sim M_W$ (calibration, not derivation)
    \item[[OPEN]:] The boundary conditions (N/D/mixed) and brane thickness $\ell$
\end{description}

\medskip
\noindent To upgrade [I] $\to$ [Dc]: derive $\ell$ from $\sigma$ and BCs from physics.
\end{tcolorbox}

% ------------------------------------------------------------------------------
\subsubsection{Chain Tightening Summary}
\label{sec:ch11_chain_tightened}

\begin{tcolorbox}[colback=green!5!white, colframe=green!50!black,
    title=\textbf{Chain Map Tightened: OPR-19/20}]
\textbf{Before (§\ref{sec:ch11_sanity_skeleton}):}
\begin{itemize}[nosep]
    \item OPR-19: ``$g_5$ postulated''
    \item OPR-20: ``$m_\phi \sim M_W$ identified''
\end{itemize}

\textbf{After (this section):}
\begin{itemize}[nosep]
    \item OPR-19: Canonical normalization from 5D action gives $g_4 = g_5$
          (with orthonormal modes). Brane kinetic terms are optional [P] extension.
          \textbf{Closure path:} compute $g_5$ from underlying 5D theory.
    \item OPR-20: KK eigenvalue problem gives $m_\phi = x_1/\ell$ where $x_1$
          depends on boundary conditions. \textbf{Closure path:} derive $\ell$ from
          membrane parameters $(\sigma, r_e)$ and BCs from physics.
\end{itemize}

\medskip
\noindent\fbox{\parbox{0.92\textwidth}{\small
\textbf{Status:} OPR-19/20 remain RED-C but with mathematically concrete closure paths.
The ``SM-help'' impression is reduced: we now have explicit derivation spines,
not just dimensional arguments.}}
\end{tcolorbox}

% ------------------------------------------------------------------------------
\subsubsection{Updated Stoplight: OPR-19/20}
\label{sec:ch11_opr19_20_stoplight}

\begin{table}[ht]
\centering
\caption{OPR-19/20 status after chain tightening}
\label{tab:ch11_opr19_20_status}
\small
\begin{tabular}{clcl}
\toprule
\textbf{OPR} & \textbf{Item} & \textbf{Status} & \textbf{Notes} \\
\midrule
19 & $g_5$ canonical normalization & RED-C & $g_4 = g_5$ [Dc]; $g_5$ value [OPEN] \\
20 & $m_\phi$ KK spectrum & RED-C & $m_\phi = x_1/\ell$ [Dc]; $\ell$, BC [OPEN] \\
\bottomrule
\end{tabular}
\end{table}

\paragraph{What ``RED-C'' means.}
The status remains RED (not derived from first principles), but the ``C'' indicates
a \emph{concrete closure path} is now defined:
\begin{itemize}[nosep]
    \item OPR-19: Need underlying 5D gauge theory to fix $g_5$ value
    \item OPR-20: Need $\ell$ from membrane physics + BCs from consistency
\end{itemize}

\paragraph{Improvement over previous state.}
Before this section, OPR-19/20 were ``open with dimensional argument.''
Now they are ``open with derivation spine.'' The mathematical structure is explicit;
only the physical inputs $(\ell, \text{BC})$ remain to be derived.



