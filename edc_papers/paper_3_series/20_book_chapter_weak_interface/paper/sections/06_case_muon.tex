% ==============================================================================
% Case Study II: Muon Decay as Brane-Dominant Mode Relaxation
% ==============================================================================

\subsection{Muon Decay: Brane-Dominant Mode Relaxation}
\label{sec:muon_story}

Unlike the neutron, the muon does not require a bulk-core junction ontology.
In the EDC Weak Program, the muon is treated as a \emph{brane-dominant excitation}:
a localized brane-layer defect/mode that stores energy primarily in the brane subsystem,
and relaxes via the same three-phase pipeline (absorption $\to$ dissipation $\to$ release),
but with the \emph{bulk trigger removed at leading order}.

\subsubsection{Ontology and What Makes the Muon a Clean Test}

\begin{edcDefinitionBox}{Muon as a brane-dominant excitation}{[Def]/[P]}
We model the muon as an excitation $\Psi_\mu$ in the brane-layer mode spectrum, with dominant
energy support in the brane (not in the bulk-core junction). The decay is the relaxation
$\Psi_\mu \to \Psi_e$ through brane-layer redistribution and frozen projection into allowed outputs.
\end{edcDefinitionBox}

This makes $\mu$-decay a clean test of the brane mechanism because:
\begin{enumerate}[nosep]
  \item[(i)] there is no ambiguity of bulk-core topology,
  \item[(ii)] the output channel is experimentally sharp, and
  \item[(iii)] the chirality structure is a strong selection signature.
\end{enumerate}

\paragraph{Baseline observable.}
Experimentally the dominant decay is \tagBL{}:
\begin{equation}
\mu^- \to e^- + \bar\nu_e + \nu_\mu,
\label{eq:mu_decay_channel_story}
\end{equation}
with an almost purely leptonic final state and lifetime $\tau_\mu \approx 2.197 \times 10^{-6}$ s \tagBL{}.

\subsubsection{Pipeline for Muon Decay}

We reuse the unified pipeline:
\begin{equation}
\label{eq:mu_pipeline}
\Psi_\mu
\;\Rightarrow\;
E_{\mathrm{brane}} \text{ (stored)}
\;\Rightarrow\;
\Gamma_{\mathrm{eff}} \text{ (redistribution)}
\;\Rightarrow\;
\mathcal{P}_{\mathrm{frozen}} \text{ (3D outputs)}.
\end{equation}

\paragraph{Absorption / charging \tagDc{}.}
For a brane-dominant excitation, the ``absorption'' stage is not pumping from bulk,
but simply the existence of stored brane energy in the excited configuration:
\begin{equation}
\label{eq:mu_brane_energy}
E_{\mathrm{brane}}(t_0) \approx m_\mu c^2 \approx 105.7~\text{MeV} \quad \text{\tagBL{}},
\end{equation}
up to small corrections (soft, recoil, residual leakage) \tagOpen{}.

\paragraph{Dissipation (mode redistribution) \tagDc{}/\tagP{}.}
We use the same phenomenological release-rate definition:
\begin{equation}
\label{eq:mu_release_power}
\Pi_{\mathrm{release}}(t) \equiv \Gamma_{\mathrm{eff}}\,E_{\mathrm{brane}}(t),
\end{equation}
where $\Gamma_{\mathrm{eff}}$ must ultimately be derived from thick-brane microphysics \tagOpen{}.

\paragraph{Release map (allowed outputs) \tagDef{}/\tagDc{}.}
The frozen projection maps brane-layer modes into allowed 3D outputs:
\begin{equation}
\label{eq:mu_release}
E_{\mathrm{brane}}
\;\xrightarrow{\;\mathcal{P}_{\mathrm{frozen}}\;}
e^- + \bar\nu_e + \nu_\mu + \text{(soft/recoil)}.
\end{equation}

\subsubsection{Chiral Filter as Boundary Projection, Not a Fundamental Vertex}

A key empirical signature is the V--A chirality structure of weak outputs \tagBL{}.
In EDC we do not postulate a fundamental 3D ``weak vertex''.
Instead we hypothesize that chirality selection arises from a boundary/projection operator:
\begin{equation}
\label{eq:Pfrozen_factorization_mu}
\mathcal{P}_{\mathrm{frozen}}
=
\mathcal{P}_{\mathrm{energy}}\circ
\mathcal{P}_{\mathrm{mode}}\circ
\mathcal{P}_{\mathrm{chir}}.
\end{equation}

\begin{tcolorbox}[mechanism, title={Chirality as Boundary Phenomenon}]
\textbf{Claim} \tagP{}/\tagOpen{}: The observed V--A structure is consistent with
a boundary projection that selects allowed helicity/chirality outputs.
The operator $\mathcal{P}_{\mathrm{chir}}$ is an operator-level mechanism whose explicit derivation
requires the thick-brane boundary conditions (see \S\ref{sec:case_neutrino}).
\end{tcolorbox}

\subsubsection{Muon Ledger Closure}

\begin{edcLedgerBox}{Muon decay bookkeeping}{[Dc]}
\begin{equation}
\label{eq:mu_ledger}
m_\mu c^2
=
E_{e^-}+E_{\bar\nu_e}+E_{\nu_\mu}
+E_{\mathrm{recoil}}+E_{\mathrm{soft}}+E_{\mathrm{bulk,res}}.
\end{equation}
The role of the neutrino sector is not optional: it carries ledger-consistent energy/momentum
in a way compatible with chirality selection.
\end{edcLedgerBox}

\subsubsection{Process Diagram: Muon Decay}

\begin{figure}[ht]
\centering
% figures/fig_muon_process_pipeline.tex
% Muon decay process pipeline diagram (brane-dominant)
\begin{tikzpicture}[scale=0.90, transform shape]

% Load styles

% ─────────────────────────────────────────────────────────────────────────────
% Background regions (no bulk needed for brane-dominant)
% ─────────────────────────────────────────────────────────────────────────────
\fill[blue!8] (-5.2,0.4) rectangle (5.5,2.0);
\fill[green!8] (-5.2,-1.0) rectangle (5.5,0.4);

% Region labels
\node[font=\scriptsize, blue!50!black] at (-4.5,1.7) {Thick brane layer};
\node[font=\scriptsize, green!50!black] at (-4.5,0.1) {3D outputs};

% ─────────────────────────────────────────────────────────────────────────────
% Brane layer nodes
% ─────────────────────────────────────────────────────────────────────────────
\node[brane box, text width=2.6cm] (mu) at (-3.2,1.2)
  {Muon $\Psi_\mu$\\{\tiny brane-dominant}};

\node[brane box, text width=2.6cm] (diss) at (0.3,1.2)
  {Dissipation\\{\tiny $\Gamma_{\mathrm{eff}} E_{\mathrm{brane}}$}};

\node[gate box, text width=2.4cm, minimum height=0.8cm] (frozen) at (3.5,1.2)
  {$\mathcal{P}_{\mathrm{frozen}}$\\{\tiny release gate}};

% ─────────────────────────────────────────────────────────────────────────────
% Output layer nodes
% ─────────────────────────────────────────────────────────────────────────────
\node[output box, text width=1.6cm] (eout) at (1.0,-0.3)
  {$e^-$};

\node[output box, text width=1.6cm] (nue) at (2.8,-0.3)
  {$\bar\nu_e$};

\node[output box, text width=1.6cm] (numu) at (4.6,-0.3)
  {$\nu_\mu$};

% ─────────────────────────────────────────────────────────────────────────────
% Arrows
% ─────────────────────────────────────────────────────────────────────────────
\draw[edc flow] (mu) -- (diss);
\draw[edc flow] (diss) -- (frozen);

% Release to outputs
\draw[edc arrow] (frozen.south) -- ++(0,-0.25) -| (eout.north);
\draw[edc arrow] (frozen.south) -- ++(0,-0.25) -| (nue.north);
\draw[edc arrow] (frozen.south) -- ++(0,-0.25) -| (numu.north);

% ─────────────────────────────────────────────────────────────────────────────
% Annotations
% ─────────────────────────────────────────────────────────────────────────────

% Chiral filter annotation
\node[rectangle, draw=purple!40, fill=purple!5, rounded corners=2pt,
      font=\tiny, align=center, text width=4.5cm] at (-0.5,0.0)
  {$\mathcal{P}_{\mathrm{frozen}} = \mathcal{P}_{\mathrm{energy}} \circ \mathcal{P}_{\mathrm{mode}} \circ \mathcal{P}_{\mathrm{chir}}$\\
   V--A selection via boundary projection};

% No bulk trigger note
\node[rectangle, draw=gray!40, fill=gray!5, rounded corners=2pt,
      font=\tiny, align=center, text width=2.8cm] at (-3.2,0.0)
  {No bulk trigger\\(clean brane test)};

\end{tikzpicture}

\caption{\textbf{Muon decay in EDC as brane-dominant relaxation.}
Stored brane energy redistributes (dissipation) and is released via frozen projection
into allowed outputs, with chirality selection implemented as a boundary operator.
Unlike neutron decay, there is no bulk trigger---the muon is a clean test of the
brane-layer mechanism.}
\label{fig:muon_process_pipeline}
\end{figure}

\subsubsection{Why Muon Is a Clean Universality Test}

The muon decay channel tests whether:
\begin{itemize}[nosep]
  \item The same $\mathcal{P}_{\mathrm{frozen}}$ operator applies to brane-dominant excitations
        (not just bulk-core junctions)
  \item The chirality filter produces V--A structure without vertex tuning
  \item Ledger closure works for purely brane-layer relaxation
\end{itemize}

If these conditions hold, the weak-sector brane interface is a \emph{universal mechanism},
not a special case of neutron physics.

\subsubsection{Falsifiability Hooks}

\begin{tcolorbox}[falsifiability]
\begin{itemize}[nosep]
  \item If the mechanism predicts a dominant 2-body channel ($\mu \to e\gamma$) inconsistent with
        observed spectrum, it fails.
  \item If $\mathcal{P}_{\mathrm{chir}}$ cannot be realized as a boundary operator
        without tuning, the proposed interpretation weakens.
  \item If ledger closure requires hidden sinks not accounted by $E_{\mathrm{other}}$,
        it fails.
  \item If the muon lifetime cannot be connected to $\Gamma_{\mathrm{eff}}$ from
        brane microphysics, the quantitative program is incomplete \tagOpen{}.
\end{itemize}
\end{tcolorbox}

