% ==============================================================================
% Chapter 7: CKM Matrix and CP Violation
% Status: [Dc] baseline computed, FALSIFIED — breaking required
% ==============================================================================

\section{CKM Matrix and CP Violation}
\label{sec:ch7_ckm}

\begin{tcolorbox}[edcGuardrail, title=\textbf{Epistemic Status}]
This chapter applies the $\mathbb{Z}_3$ symmetry analysis from Chapter~\ref{ch:neutrinos_edge}
to quark mixing (CKM matrix):
\begin{itemize}[nosep]
    \item $\mathbb{Z}_3$ DFT baseline computed \tagDc{}
    \item Baseline predicts ``democratic'' mixing: all $|V_{ij}|^2 = 1/3$
    \item Comparison with PDG: \textbf{strongly falsified} (CKM is nearly diagonal)
    \item Breaking requirement: much stronger than for PMNS \tagP{}
\end{itemize}
\textbf{What is NOT claimed:} CKM angles are not derived. CP violation phase
is not addressed. Breaking mechanism is postulated, not computed.
\end{tcolorbox}

% ------------------------------------------------------------------------------
% BOOK-READY INTRODUCTION
% ------------------------------------------------------------------------------

\paragraph{Chapter overview.}
The Cabibbo--Kobayashi--Maskawa (CKM) matrix describes quark flavor mixing in
weak interactions. Unlike the PMNS matrix (which has large mixing angles),
the CKM matrix is nearly diagonal---the dominant transitions are within
generations ($u \leftrightarrow d$, $c \leftrightarrow s$, $t \leftrightarrow b$),
with inter-generational mixing strongly suppressed.

In EDC, if quark generations also correspond to $\mathbb{Z}_3$ modes, the same
DFT baseline analysis applies. This chapter computes that baseline and shows
it is \emph{dramatically falsified}: the observed CKM hierarchy requires
\textbf{much stronger $\mathbb{Z}_3$ breaking} than the lepton sector.
This asymmetry between quark and lepton mixing is itself a puzzle that EDC
must eventually explain.

% ------------------------------------------------------------------------------
% READER MAP
% ------------------------------------------------------------------------------

\begin{tcolorbox}[colback=blue!5, colframe=blue!50!black,
    title=\textbf{Reader Map: What This Chapter Establishes}]
\begin{description}[style=nextline, leftmargin=1em, font=\normalfont\bfseries]
    \item[Derived \tagDc{}:]
        $\mathbb{Z}_3$ DFT baseline: all $|V_{ij}|^2 = 1/3$;
        comparison showing strong falsification.

    \item[Identified \tagI{}:]
        Three quark generations $\leftrightarrow$ $|\mathbb{Z}_3| = 3$
        (same as leptons).

    \item[Postulated \tagP{}:]
        $\mathbb{Z}_3$ breaking mechanism in quark sector;
        why quark breaking is stronger than lepton breaking.

    \item[Open (not addressed):]
        CKM angles from geometry;
        CP violation phase $\delta$;
        Jarlskog invariant;
        why quarks vs.\ leptons differ.
\end{description}
\end{tcolorbox}

% ==============================================================================
\subsection{The CKM Matrix: Baseline Facts}
\label{sec:ch7_baseline}

The CKM matrix connects weak eigenstates to mass eigenstates for quarks \tagBL{}:
\begin{equation}
    \begin{pmatrix} d' \\ s' \\ b' \end{pmatrix}
    = V_{\text{CKM}}
    \begin{pmatrix} d \\ s \\ b \end{pmatrix}
    \label{eq:ch7_ckm_def}
\end{equation}

\paragraph{PDG 2024 magnitudes.}
The observed CKM matrix elements (magnitudes) are \tagBL{}:
\begin{equation}
    |V_{\text{CKM}}| \approx
    \begin{pmatrix}
        0.974 & 0.225 & 0.004 \\
        0.225 & 0.973 & 0.041 \\
        0.009 & 0.040 & 0.999
    \end{pmatrix}
    \label{eq:ch7_ckm_pdg}
\end{equation}

\paragraph{Key observation.}
The CKM matrix is \textbf{nearly diagonal}:
\begin{itemize}[nosep]
    \item Diagonal elements: $|V_{ud}|, |V_{cs}|, |V_{tb}| \approx 0.97$--$1.00$
    \item First off-diagonal: $|V_{us}|, |V_{cd}| \approx 0.22$ (Cabibbo angle)
    \item Second off-diagonal: $|V_{cb}|, |V_{ts}| \approx 0.04$
    \item Corner elements: $|V_{ub}|, |V_{td}| \approx 0.004$--$0.009$
\end{itemize}

This strong hierarchy is in stark contrast to PMNS, where mixing angles are large
($\theta_{12} \approx 33°$, $\theta_{23} \approx 45°$).

% ==============================================================================
\subsection{$\mathbb{Z}_3$ DFT Baseline for CKM}
\label{sec:ch7_dft_baseline}

\subsubsection{Hypothesis: Same $\mathbb{Z}_3$ Structure as Leptons}

If quark generations also arise from $\mathbb{Z}_3$ modes in the hexagonal
lattice (Chapter~\ref{ch:three_generations}), the same DFT analysis applies.

\begin{postulate}[$\mathbb{Z}_3$ Symmetric Quark Mixing {\normalfont \tagP{}}]
\label{post:ch7_z3_quarks}
Under exact $\mathbb{Z}_3$ symmetry, the CKM matrix equals the discrete Fourier
transform:
\begin{equation}
    V_{ij}^{\text{DFT}} = \frac{1}{\sqrt{3}} \omega^{-ij},
    \qquad \omega = e^{2\pi i/3}
    \label{eq:ch7_dft_ckm}
\end{equation}
giving the ``democratic'' matrix with all $|V_{ij}|^2 = 1/3$.
\end{postulate}

\subsubsection{DFT Baseline Predictions}

The DFT matrix predicts \tagDc{}:
\begin{equation}
    |V^{\text{DFT}}| =
    \begin{pmatrix}
        1/\sqrt{3} & 1/\sqrt{3} & 1/\sqrt{3} \\
        1/\sqrt{3} & 1/\sqrt{3} & 1/\sqrt{3} \\
        1/\sqrt{3} & 1/\sqrt{3} & 1/\sqrt{3}
    \end{pmatrix}
    \approx
    \begin{pmatrix}
        0.577 & 0.577 & 0.577 \\
        0.577 & 0.577 & 0.577 \\
        0.577 & 0.577 & 0.577
    \end{pmatrix}
    \label{eq:ch7_dft_numeric}
\end{equation}

All elements equal: $|V_{ij}^{\text{DFT}}| = 1/\sqrt{3} \approx 0.577$.

% ==============================================================================
\subsection{Comparison with PDG Data}
\label{sec:ch7_comparison}

\begin{table}[ht]
\centering
\caption{CKM: DFT baseline vs.\ PDG magnitudes}
\label{tab:ch7_ckm_comparison}
\begin{tabular}{lccrl}
\toprule
\textbf{Element} & \textbf{DFT} & \textbf{PDG 2024} \tagBL{} & \textbf{Ratio} & \textbf{Status} \\
\midrule
$|V_{ud}|$ & 0.577 & 0.974 & $\times 0.59$ & \textcolor{red}{OFF} \\
$|V_{us}|$ & 0.577 & 0.225 & $\times 2.6$ & \textcolor{red}{OFF} \\
$|V_{ub}|$ & 0.577 & 0.004 & $\times 144$ & \textcolor{red}{\textbf{FALSIFIED}} \\
$|V_{cd}|$ & 0.577 & 0.225 & $\times 2.6$ & \textcolor{red}{OFF} \\
$|V_{cs}|$ & 0.577 & 0.973 & $\times 0.59$ & \textcolor{red}{OFF} \\
$|V_{cb}|$ & 0.577 & 0.041 & $\times 14$ & \textcolor{red}{\textbf{FALSIFIED}} \\
$|V_{td}|$ & 0.577 & 0.009 & $\times 64$ & \textcolor{red}{\textbf{FALSIFIED}} \\
$|V_{ts}|$ & 0.577 & 0.040 & $\times 14$ & \textcolor{red}{\textbf{FALSIFIED}} \\
$|V_{tb}|$ & 0.577 & 0.999 & $\times 0.58$ & \textcolor{red}{OFF} \\
\bottomrule
\end{tabular}
\end{table}

\begin{tcolorbox}[colback=red!5, colframe=red!50!black,
    title=\textbf{Verdict: DFT Baseline STRONGLY FALSIFIED}]
The $\mathbb{Z}_3$ symmetric (DFT) CKM matrix fails dramatically:
\begin{itemize}[nosep]
    \item Corner elements ($|V_{ub}|$, $|V_{td}|$): off by factors of 60--140
    \item Second off-diagonal ($|V_{cb}|$, $|V_{ts}|$): off by factor $\sim 14$
    \item First off-diagonal ($|V_{us}|$, $|V_{cd}|$): off by factor $\sim 2.6$
    \item Diagonal elements: off by factor $\sim 0.6$ (wrong direction)
\end{itemize}

\textbf{Conclusion:} The CKM hierarchy requires \emph{very strong breaking}
of $\mathbb{Z}_3$ symmetry---much stronger than in the lepton sector.
\end{tcolorbox}

% ==============================================================================
\subsection{Quantifying the Breaking Requirement}
\label{sec:ch7_breaking}

\subsubsection{Lepton vs.\ Quark Breaking Asymmetry}

\begin{table}[ht]
\centering
\caption{Breaking required: PMNS vs.\ CKM}
\label{tab:ch7_breaking_comparison}
\begin{tabular}{lccc}
\toprule
\textbf{Matrix} & \textbf{Worst DFT error} & \textbf{Breaking scale} & \textbf{Status} \\
\midrule
PMNS (neutrinos) & $\theta_{13}$: $\times 15$ & $\sim 25\%$ & \tagP{} \\
CKM (quarks) & $|V_{ub}|$: $\times 144$ & $\sim 99\%$ & \tagP{} \\
\bottomrule
\end{tabular}
\end{table}

The quark sector requires \textbf{near-complete} breaking of $\mathbb{Z}_3$
symmetry to achieve the observed hierarchy. This asymmetry is itself a puzzle.

\subsubsection{Candidate Breaking Mechanisms}

Three mechanisms could produce strong CKM hierarchy \tagP{}:

\begin{enumerate}
    \item \textbf{$\mathbb{Z}_2$ generation selection from $\mathbb{Z}_6$:}
          The $\mathbb{Z}_2 \subset \mathbb{Z}_6$ factor distinguishes even/odd
          modes. If up-type and down-type quarks couple differently to this
          $\mathbb{Z}_2$, inter-generational mixing is suppressed.

    \item \textbf{Different localization depths for up vs.\ down sectors:}
          If up-type quarks ($u, c, t$) have different penetration depths
          $\kappa_u^{-1}$ than down-type quarks ($d, s, b$), the overlap
          integrals for CKM become highly non-democratic.

    \item \textbf{Quark-sector potential anisotropy:}
          If the confining potential for quarks has stronger angular anisotropy
          than for leptons, the $\mathbb{Z}_3$ breaking is enhanced in the
          quark sector.
\end{enumerate}

\begin{tcolorbox}[edcGuardrail, title=\textbf{Status}]
All three mechanisms are \textbf{postulated} \tagP{}. No explicit calculation
of CKM elements from EDC geometry has been performed. The mechanisms are
identified as logical possibilities, not derived predictions.
\end{tcolorbox}

% ==============================================================================
\subsection{CP Violation}
\label{sec:ch7_cp}

\subsubsection{The Jarlskog Invariant}

CP violation in the quark sector is characterized by the Jarlskog invariant
\tagBL{}:
\begin{equation}
    J = \text{Im}(V_{us} V_{cb} V_{ub}^* V_{cs}^*) \approx 3.0 \times 10^{-5}
    \label{eq:ch7_jarlskog}
\end{equation}

\subsubsection{EDC Status}

\begin{tcolorbox}[colback=gray!5, colframe=gray!50!black,
    title=\textbf{CP Violation: Not Addressed}]
The origin of CP violation in the quark sector is \textbf{not addressed}
in this chapter. Potential EDC mechanisms include:
\begin{itemize}[nosep]
    \item Complex phases in the $\mathbb{Z}_6$ lattice structure
    \item Asymmetric boundary conditions at the bulk-brane interface
    \item CP-violating terms in the Plenum stress tensor
\end{itemize}
All of these are speculative and remain (open).
\end{tcolorbox}

% ==============================================================================
\subsection{Summary and Stoplight}
\label{sec:ch7_summary}

\subsubsection{What Chapter 7 Establishes}

\begin{tcolorbox}[colback=green!5, colframe=green!50!black,
    title=\textbf{Chapter 7 Summary}]
\begin{enumerate}[nosep]
    \item \textbf{$\mathbb{Z}_3$ DFT baseline computed} \tagDc{}:
          All $|V_{ij}|^2 = 1/3$ under exact symmetry.
    \item \textbf{Strong falsification} \tagDc{}:
          Observed CKM is nearly diagonal; corner elements off by factors
          of 60--140.
    \item \textbf{Breaking asymmetry identified} \tagI{}:
          Quark sector requires $\sim 99\%$ breaking vs.\ $\sim 25\%$ for leptons.
    \item \textbf{Mechanisms postulated} \tagP{}:
          $\mathbb{Z}_2$ selection, localization asymmetry, potential anisotropy.
\end{enumerate}
\end{tcolorbox}

\subsubsection{Stoplight Verdict}

\begin{table}[ht]
\centering
\caption{Chapter 7 overall verdict}
\label{tab:ch7_verdict}
\begin{tabular}{lcc}
\toprule
\textbf{Claim} & \textbf{Verdict} & \textbf{Tag} \\
\midrule
$\mathbb{Z}_3$ DFT baseline computed & GREEN & \tagDc{} \\
DFT vs.\ PDG comparison & \textcolor{red}{FALSIFIED} & --- \\
CKM hierarchy requires breaking & GREEN & \tagDc{} \\
Breaking mechanism & RED & \tagP{} \\
Why quarks $\neq$ leptons & RED & (open) \\
CP violation phase & RED & (open) \\
Jarlskog invariant & RED & (open) \\
\bottomrule
\end{tabular}
\end{table}

\textbf{Bottom line:} The $\mathbb{Z}_3$ baseline analysis that worked partially
for PMNS ($\sim 25\%$ breaking needed) fails dramatically for CKM ($\sim 99\%$
breaking needed). This identifies a key structural question for EDC: why is
$\mathbb{Z}_3$ nearly preserved in the lepton sector but almost completely
broken in the quark sector? The answer likely involves different localization
properties or potential shapes for quarks vs.\ leptons, but this remains (open).

