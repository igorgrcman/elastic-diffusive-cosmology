% ==============================================================================
% Chapter 7: CKM Matrix and CP Violation
% Status: [Dc] baseline computed, FALSIFIED — breaking required
% Framework v2.0 tags only
% ==============================================================================

\section{CKM Matrix and CP Violation}
\label{sec:ch7_ckm}

% ------------------------------------------------------------------------------
% EPISTEMIC STATUS BOX
% ------------------------------------------------------------------------------

\begin{tcolorbox}[edcGuardrail, title=\textbf{Epistemic Status}]
This chapter applies the $\mathbb{Z}_3$ symmetry analysis from Chapter~\ref{ch:neutrinos_edge}
to quark mixing (CKM matrix):
\begin{itemize}[nosep]
    \item $\mathbb{Z}_3$ DFT baseline computed \tagDc{}
    \item Baseline predicts ``democratic'' mixing: all $|V_{ij}|^2 = 1/3$
    \item Comparison with PDG: \textbf{strongly falsified} (CKM is nearly diagonal)
    \item Breaking requirement: much stronger than for PMNS \tagP{}
\end{itemize}
\textbf{What is NOT claimed:} CKM angles are not derived. CP violation phase
is not addressed. Breaking mechanism is postulated, not computed.
\end{tcolorbox}

% ------------------------------------------------------------------------------
% CHAIN BOX: What is independent vs. what is not
% ------------------------------------------------------------------------------

\begin{tcolorbox}[colback=green!5, colframe=green!50!black,
    title=\textbf{Derivation Chain: What Is Computed vs.\ What Fails}]
\begin{description}[style=nextline, leftmargin=1em, font=\normalfont\bfseries]
    \item[Identification \tagI{}:]
        Three quark generations $\leftrightarrow$ $|\mathbb{Z}_3| = 3$
        (same identification as for leptons in Ch6).

    \item[Computed baseline \tagDc{}:]
        $\mathbb{Z}_3$ DFT matrix: $V_{ij}^{\text{DFT}} = \omega^{-ij}/\sqrt{3}$
        with all $|V_{ij}|^2 = 1/3$.

    \item[Falsification result \tagDc{}:]
        Baseline vs.\ PDG: corner elements off by \textbf{$\times 140$};
        diagonal elements off by \textbf{$\times 0.6$} (wrong direction).

    \item[Breaking scale \tagI{}:]
        CKM requires $\varepsilon \lesssim 0.22$ (Cabibbo) to $\varepsilon \lesssim 0.004$
        ($|V_{ub}|$). This is \textbf{near-complete breaking} of $\mathbb{Z}_3$.

    \item[Open:]
        Breaking mechanism; CP phase $\delta$; Jarlskog invariant $J$;
        why quarks break more than leptons.
\end{description}
\end{tcolorbox}

% ------------------------------------------------------------------------------
% BOOK-READY INTRODUCTION
% ------------------------------------------------------------------------------

\paragraph{Chapter overview.}
The Cabibbo--Kobayashi--Maskawa (CKM) matrix describes quark flavor mixing in
weak interactions. Unlike the PMNS matrix (which has large mixing angles),
the CKM matrix is nearly diagonal---the dominant transitions are within
generations ($u \leftrightarrow d$, $c \leftrightarrow s$, $t \leftrightarrow b$),
with inter-generational mixing strongly suppressed.

In EDC, if quark generations also correspond to $\mathbb{Z}_3$ modes, the same
DFT baseline analysis applies. This chapter computes that baseline and shows
it is \emph{dramatically falsified}: the observed CKM hierarchy requires
\textbf{much stronger $\mathbb{Z}_3$ breaking} than the lepton sector.
This asymmetry between quark and lepton mixing is itself a puzzle that EDC
must eventually explain.

% ------------------------------------------------------------------------------
% READER MAP
% ------------------------------------------------------------------------------

\begin{tcolorbox}[colback=blue!5, colframe=blue!50!black,
    title=\textbf{Reader Map: What This Chapter Establishes}]
\begin{description}[style=nextline, leftmargin=1em, font=\normalfont\bfseries]
    \item[Derived \tagDc{}:]
        $\mathbb{Z}_3$ DFT baseline: all $|V_{ij}|^2 = 1/3$;
        comparison showing strong falsification.

    \item[Identified \tagI{}:]
        Three quark generations $\leftrightarrow$ $|\mathbb{Z}_3| = 3$
        (same as leptons).

    \item[Postulated \tagP{}:]
        $\mathbb{Z}_3$ breaking mechanism in quark sector;
        why quark breaking is stronger than lepton breaking.

    \item[Open (not addressed):]
        CKM angles from geometry;
        CP violation phase $\delta$;
        Jarlskog invariant;
        why quarks vs.\ leptons differ.
\end{description}
\end{tcolorbox}

% ==============================================================================
\subsection{The CKM Matrix: Baseline Facts}
\label{sec:ch7_baseline}

The CKM matrix connects weak eigenstates to mass eigenstates for quarks \tagBL{}:
\begin{equation}
    \begin{pmatrix} d' \\ s' \\ b' \end{pmatrix}
    = V_{\text{CKM}}
    \begin{pmatrix} d \\ s \\ b \end{pmatrix}
    \label{eq:ch7_ckm_def}
\end{equation}

\paragraph{PDG 2024 magnitudes.}
The observed CKM matrix elements (magnitudes) are \tagBL{}:
\begin{equation}
    |V_{\text{CKM}}| \approx
    \begin{pmatrix}
        0.974 & 0.225 & 0.004 \\
        0.225 & 0.973 & 0.041 \\
        0.009 & 0.040 & 0.999
    \end{pmatrix}
    \label{eq:ch7_ckm_pdg}
\end{equation}

\paragraph{Key observation.}
The CKM matrix is \textbf{nearly diagonal}:
\begin{itemize}[nosep]
    \item Diagonal elements: $|V_{ud}|, |V_{cs}|, |V_{tb}| \approx 0.97$--$1.00$
    \item First off-diagonal: $|V_{us}|, |V_{cd}| \approx 0.22$ (Cabibbo angle)
    \item Second off-diagonal: $|V_{cb}|, |V_{ts}| \approx 0.04$
    \item Corner elements: $|V_{ub}|, |V_{td}| \approx 0.004$--$0.009$
\end{itemize}

This strong hierarchy is in stark contrast to PMNS, where mixing angles are large
($\theta_{12} \approx 33°$, $\theta_{23} \approx 45°$).

% ==============================================================================
\subsection{Attempt 1: $\mathbb{Z}_3$ DFT Baseline for CKM}
\label{sec:ch7_dft_baseline}

This section parallels the PMNS Attempt~1 (Section~\ref{sec:ch6_pmns_attempt1}):
we compute what the CKM matrix would be under exact $\mathbb{Z}_3$ symmetry,
then compare to PDG data to quantify the required breaking.

\subsubsection{Hypothesis: Same $\mathbb{Z}_3$ Structure as Leptons}

If quark generations also arise from $\mathbb{Z}_3$ modes in the hexagonal
lattice (Chapter~\ref{ch:three_generations}), the same DFT analysis applies.

Define the discrete Fourier transform basis:
\begin{equation}
    \omega = e^{2\pi i/3}, \qquad
    U_{ij}^{\text{DFT}} = \frac{1}{\sqrt{3}} \omega^{-ij}
    \label{eq:ch7_dft_def}
\end{equation}

\paragraph{Two minimal options for CKM.}
The CKM matrix is $V = U_u^\dagger U_d$, where $U_u$ and $U_d$ transform
up-type and down-type quarks to mass eigenstates. We consider two options
\tagP{}:

\begin{description}[style=nextline, leftmargin=1em]
    \item[\textbf{Option A} (aligned sectors):]
        Both up and down sectors have the same $\mathbb{Z}_3$ DFT basis
        ($U_u = U_d = U^{\text{DFT}}$). Then:
        \begin{equation}
            V^{\text{(A)}} = (U^{\text{DFT}})^\dagger U^{\text{DFT}} = \mathbb{1}
            \label{eq:ch7_option_a}
        \end{equation}
        \textbf{Result:} CKM is the identity---\emph{no mixing}.
        This is falsified by the observed Cabibbo angle.

    \item[\textbf{Option B} (misaligned sectors):]
        Up sector is in site basis ($U_u = \mathbb{1}$), down sector is in
        DFT basis ($U_d = U^{\text{DFT}}$). Then:
        \begin{equation}
            V^{\text{(B)}} = \mathbb{1}^\dagger \cdot U^{\text{DFT}} = U^{\text{DFT}}
            \label{eq:ch7_option_b}
        \end{equation}
        \textbf{Result:} CKM equals the DFT matrix---all $|V_{ij}|^2 = 1/3$.
        This is ``democratic'' mixing.
\end{description}

\begin{tcolorbox}[colback=gray!5, colframe=gray!50!black,
    title=\textbf{Assessment of Options}]
\textbf{Option A} (identity) is \emph{more wrong} than Option B: it predicts
zero mixing, contradicting the observed Cabibbo angle $\theta_C \approx 13°$.

\textbf{Option B} (democratic) predicts $|V_{us}| = 1/\sqrt{3} \approx 0.577$,
which is $\times 2.6$ larger than PDG ($|V_{us}| \approx 0.225$).

Both options fail, but \textbf{Option B is closer} and serves as the baseline
for quantifying the required breaking. We adopt Option B for the remainder
of this chapter.
\end{tcolorbox}

\begin{postulate}[$\mathbb{Z}_3$ Symmetric Quark Mixing {\normalfont \tagP{}}]
\label{post:ch7_z3_quarks}
Under the Option B assumption (up sector in site basis, down sector in DFT basis),
the CKM matrix equals the discrete Fourier transform:
\begin{equation}
    V_{ij}^{\text{DFT}} = \frac{1}{\sqrt{3}} \omega^{-ij},
    \qquad \omega = e^{2\pi i/3}
    \label{eq:ch7_dft_ckm}
\end{equation}
giving the ``democratic'' matrix with all $|V_{ij}|^2 = 1/3$.
\end{postulate}

\subsubsection{DFT Baseline Predictions}

The DFT matrix predicts \tagDc{}:
\begin{equation}
    |V^{\text{DFT}}| =
    \begin{pmatrix}
        1/\sqrt{3} & 1/\sqrt{3} & 1/\sqrt{3} \\
        1/\sqrt{3} & 1/\sqrt{3} & 1/\sqrt{3} \\
        1/\sqrt{3} & 1/\sqrt{3} & 1/\sqrt{3}
    \end{pmatrix}
    \approx
    \begin{pmatrix}
        0.577 & 0.577 & 0.577 \\
        0.577 & 0.577 & 0.577 \\
        0.577 & 0.577 & 0.577
    \end{pmatrix}
    \label{eq:ch7_dft_numeric}
\end{equation}

All elements equal: $|V_{ij}^{\text{DFT}}| = 1/\sqrt{3} \approx 0.577$.

% ==============================================================================
\subsection{Comparison with PDG Data}
\label{sec:ch7_comparison}

\begin{table}[ht]
\centering
\caption{CKM: DFT baseline vs.\ PDG magnitudes}
\label{tab:ch7_ckm_comparison}
\begin{tabular}{lccrl}
\toprule
\textbf{Element} & \textbf{DFT} & \textbf{PDG 2024} \tagBL{} & \textbf{Ratio} & \textbf{Status} \\
\midrule
$|V_{ud}|$ & 0.577 & 0.974 & $\times 0.59$ & \textcolor{red}{OFF} \\
$|V_{us}|$ & 0.577 & 0.225 & $\times 2.6$ & \textcolor{red}{OFF} \\
$|V_{ub}|$ & 0.577 & 0.004 & $\times 144$ & \textcolor{red}{\textbf{FALSIFIED}} \\
$|V_{cd}|$ & 0.577 & 0.225 & $\times 2.6$ & \textcolor{red}{OFF} \\
$|V_{cs}|$ & 0.577 & 0.973 & $\times 0.59$ & \textcolor{red}{OFF} \\
$|V_{cb}|$ & 0.577 & 0.041 & $\times 14$ & \textcolor{red}{\textbf{FALSIFIED}} \\
$|V_{td}|$ & 0.577 & 0.009 & $\times 64$ & \textcolor{red}{\textbf{FALSIFIED}} \\
$|V_{ts}|$ & 0.577 & 0.040 & $\times 14$ & \textcolor{red}{\textbf{FALSIFIED}} \\
$|V_{tb}|$ & 0.577 & 0.999 & $\times 0.58$ & \textcolor{red}{OFF} \\
\bottomrule
\end{tabular}
\end{table}

\begin{tcolorbox}[colback=red!5, colframe=red!50!black,
    title=\textbf{Verdict: DFT Baseline STRONGLY FALSIFIED}]
The $\mathbb{Z}_3$ symmetric (DFT) CKM matrix fails dramatically:
\begin{itemize}[nosep]
    \item Corner elements ($|V_{ub}|$, $|V_{td}|$): off by factors of 60--140
    \item Second off-diagonal ($|V_{cb}|$, $|V_{ts}|$): off by factor $\sim 14$
    \item First off-diagonal ($|V_{us}|$, $|V_{cd}|$): off by factor $\sim 2.6$
    \item Diagonal elements: off by factor $\sim 0.6$ (wrong direction)
\end{itemize}

\textbf{Conclusion:} The CKM hierarchy requires \emph{very strong breaking}
of $\mathbb{Z}_3$ symmetry---much stronger than in the lepton sector.
\end{tcolorbox}

% ==============================================================================
\subsection{Quantifying the Breaking Requirement}
\label{sec:ch7_breaking}

\subsubsection{Breaking Amplitude $\varepsilon$}

We define a breaking amplitude $\varepsilon$ such that the observed off-diagonal
elements scale as:
\begin{equation}
    |V_{ij}|_{\text{obs}} \sim \varepsilon \cdot |V_{ij}|_{\text{DFT}}
    \quad\text{for } i \neq j
    \label{eq:ch7_epsilon_def}
\end{equation}

From the PDG data \tagBL{}:
\begin{itemize}[nosep]
    \item \textbf{Cabibbo angle:} $|V_{us}| \approx 0.225$ vs.\ DFT $1/\sqrt{3} \approx 0.577$
          $\Rightarrow \varepsilon_{us} \approx 0.39$
    \item \textbf{Second off-diagonal:} $|V_{cb}| \approx 0.041$ vs.\ DFT $0.577$
          $\Rightarrow \varepsilon_{cb} \approx 0.071$
    \item \textbf{Corner element:} $|V_{ub}| \approx 0.004$ vs.\ DFT $0.577$
          $\Rightarrow \varepsilon_{ub} \approx 0.007$
\end{itemize}

The Wolfenstein parametrization captures this hierarchy \tagBL{}:
\begin{equation}
    |V_{us}| \sim \lambda, \quad
    |V_{cb}| \sim \lambda^2, \quad
    |V_{ub}| \sim \lambda^3, \qquad
    \lambda \approx 0.225
    \label{eq:ch7_wolfenstein}
\end{equation}

\textbf{Interpretation:} The CKM hierarchy is \emph{not} a small perturbation
around democratic mixing. It requires \textbf{suppression factors} of $\lambda$,
$\lambda^2$, $\lambda^3$ relative to the DFT baseline \tagDc{}.

\subsubsection{Lepton vs.\ Quark Breaking Asymmetry}

\begin{table}[ht]
\centering
\caption{Breaking required: PMNS vs.\ CKM}
\label{tab:ch7_breaking_comparison}
\begin{tabular}{lcccc}
\toprule
\textbf{Matrix} & \textbf{Worst DFT error} & \textbf{$\varepsilon$ needed} & \textbf{Breaking scale} & \textbf{Status} \\
\midrule
PMNS (neutrinos) & $\theta_{13}$: $\times 15$ & $\varepsilon \sim 0.26$ & $\sim 25\%$ & \tagI{} \\
CKM (quarks) & $|V_{ub}|$: $\times 144$ & $\varepsilon \sim 0.007$ & $\sim 99\%$ & \tagI{} \\
\bottomrule
\end{tabular}
\end{table}

The quark sector requires \textbf{near-complete} breaking of $\mathbb{Z}_3$
symmetry to achieve the observed hierarchy. This asymmetry is itself a puzzle
that EDC must eventually explain.

\subsubsection{Candidate Breaking Mechanisms}

Three mechanisms could produce strong CKM hierarchy \tagP{}:

\begin{enumerate}
    \item \textbf{$\mathbb{Z}_2$ generation selection from $\mathbb{Z}_6$:}
          The $\mathbb{Z}_2 \subset \mathbb{Z}_6$ factor distinguishes even/odd
          modes. If up-type and down-type quarks couple differently to this
          $\mathbb{Z}_2$, inter-generational mixing is suppressed.

    \item \textbf{Different localization depths for up vs.\ down sectors:}
          If up-type quarks ($u, c, t$) have different penetration depths
          $\kappa_u^{-1}$ than down-type quarks ($d, s, b$), the overlap
          integrals for CKM become highly non-democratic.

    \item \textbf{Quark-sector potential anisotropy:}
          If the confining potential for quarks has stronger angular anisotropy
          than for leptons, the $\mathbb{Z}_3$ breaking is enhanced in the
          quark sector.
\end{enumerate}

\begin{tcolorbox}[edcGuardrail, title=\textbf{Status}]
All three mechanisms are \textbf{postulated} \tagP{}. No explicit calculation
of CKM elements from EDC geometry has been performed. The mechanisms are
identified as logical possibilities, not derived predictions.
\end{tcolorbox}

% ==============================================================================
\subsection{CP Violation}
\label{sec:ch7_cp}

\subsubsection{The Jarlskog Invariant}

CP violation in the quark sector is characterized by the Jarlskog invariant
\tagBL{}:
\begin{equation}
    J = \text{Im}(V_{us} V_{cb} V_{ub}^* V_{cs}^*) \approx 3.0 \times 10^{-5}
    \label{eq:ch7_jarlskog}
\end{equation}

\subsubsection{EDC Status}

\begin{tcolorbox}[colback=gray!5, colframe=gray!50!black,
    title=\textbf{CP Violation: Not Addressed}]
The origin of CP violation in the quark sector is \textbf{not addressed}
in this chapter. Potential EDC mechanisms include:
\begin{itemize}[nosep]
    \item Complex phases in the $\mathbb{Z}_6$ lattice structure
    \item Asymmetric boundary conditions at the bulk-brane interface
    \item CP-violating terms in the Plenum stress tensor
\end{itemize}
All of these are speculative and remain (open).
\end{tcolorbox}

% ==============================================================================
\subsection{Summary and Stoplight}
\label{sec:ch7_summary}

\subsubsection{What Chapter 7 Establishes}

\begin{tcolorbox}[colback=green!5, colframe=green!50!black,
    title=\textbf{Chapter 7 Summary}]
\begin{enumerate}[nosep]
    \item \textbf{$\mathbb{Z}_3$ DFT baseline computed} \tagDc{}:
          All $|V_{ij}|^2 = 1/3$ under exact symmetry.
    \item \textbf{Strong falsification} \tagDc{}:
          Observed CKM is nearly diagonal; corner elements off by factors
          of 60--140.
    \item \textbf{Breaking asymmetry identified} \tagI{}:
          Quark sector requires $\sim 99\%$ breaking vs.\ $\sim 25\%$ for leptons.
    \item \textbf{Mechanisms postulated} \tagP{}:
          $\mathbb{Z}_2$ selection, localization asymmetry, potential anisotropy.
\end{enumerate}
\end{tcolorbox}

\subsubsection{Stoplight Verdict}

\begin{table}[ht]
\centering
\caption{Chapter 7 overall verdict}
\label{tab:ch7_verdict}
\begin{tabular}{lcc}
\toprule
\textbf{Claim} & \textbf{Verdict} & \textbf{Tag} \\
\midrule
$\mathbb{Z}_3$ DFT baseline computed & GREEN & \tagDc{} \\
DFT vs.\ PDG comparison & \textcolor{red}{FALSIFIED} & --- \\
CKM hierarchy requires breaking & GREEN & \tagDc{} \\
Breaking mechanism & RED & \tagP{} \\
Why quarks $\neq$ leptons & RED & (open) \\
CP violation phase & RED & (open) \\
Jarlskog invariant & RED & (open) \\
\bottomrule
\end{tabular}
\end{table}

\textbf{Bottom line:} The $\mathbb{Z}_3$ baseline analysis that worked partially
for PMNS ($\sim 25\%$ breaking needed) fails dramatically for CKM ($\sim 99\%$
breaking needed). This identifies a key structural question for EDC: why is
$\mathbb{Z}_3$ nearly preserved in the lepton sector but almost completely
broken in the quark sector? The answer likely involves different localization
properties or potential shapes for quarks vs.\ leptons, but this remains (open).

\subsubsection{Falsifiability}

\begin{tcolorbox}[colback=orange!5, colframe=orange!50!black,
    title=\textbf{What Would Falsify EDC's Flavor Picture?}]
\begin{enumerate}[nosep]
    \item \textbf{Fourth generation discovery:}
          If a fourth quark generation is discovered with mass below the
          electroweak scale, the $|\mathbb{Z}_3| = 3$ identification fails.
          (This is consistent with Chapter~\ref{ch:three_generations}.)

    \item \textbf{No geometric breaking mechanism exists:}
          If no controlled perturbation of $\mathbb{Z}_6$ geometry can produce
          the CKM hierarchy (with $\varepsilon_{ub} \sim 0.007$), the EDC
          flavor picture is incomplete.

    \item \textbf{Lepton-quark symmetry found:}
          If future precision measurements reveal that PMNS and CKM actually
          have similar hierarchies (contrary to current data), the asymmetry
          puzzle dissolves but the explanation shifts.
\end{enumerate}
\textbf{Current status:} None of these falsifiers are triggered. The framework
is consistent with $N_g = 3$ and with the observed lepton-quark asymmetry
being a puzzle requiring explanation.
\end{tcolorbox}

