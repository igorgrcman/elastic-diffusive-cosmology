% ==============================================================================
% Chapter 7: Z2 Parity Origin for CKM Sign-Selection Mechanism
% Status: YELLOW [Dc]+[P] — Structural derivation; specific assignment postulated
% ==============================================================================

\subsection{Z\texorpdfstring{$_2$}{2} Parity Origin for the Sign-Selection Mechanism}
\label{sec:ch7_z2_parity_origin}

The $\mathbb{Z}_2$ sign-selection mechanism (Eq.~\eqref{eq:ch7_z2_selection}) achieves
$\delta = 60°$ by flipping the sign of one effective phase in the Jarlskog product.
This subsection derives the \emph{structural} condition and proposes a \emph{geometric}
origin, replacing the ad hoc parity assignment with a principled framework.

\subsubsection{The Sign-Flip Selection Rule}

\paragraph{Jarlskog structure.}
The Jarlskog invariant involves the quartet $(V_{us}, V_{cb}, V_{ub}, V_{cs})$:
\begin{equation}
    J = \mathrm{Im}(V_{us} \cdot V_{cb} \cdot V_{ub}^* \cdot V_{cs}^*)
    \label{eq:ch7_jarlskog_def}
\end{equation}
Under pure $\mathbb{Z}_3$ discrete phases, $J$ has a definite sign corresponding
to $\delta = 120°$. A sign flip on $J$ corresponds to $\delta \to |120° - 180°| = 60°$.

\paragraph{Sign-flip arithmetic.}
If any subset of the quartet elements acquires a sign change $V_{ij} \to -V_{ij}$,
the Jarlskog invariant transforms as:
\begin{equation}
    J \to (-1)^{n_{\text{flip}}} \cdot J
    \label{eq:ch7_sign_flip_count}
\end{equation}
where $n_{\text{flip}}$ is the number of sign-flipped elements.

\begin{tcolorbox}[colback=blue!5, colframe=blue!50!black,
    title=\textbf{Z$_2$ Sign-Selection Rule}]
\textbf{Theorem} \tagDc{}: The CP phase shifts to $\delta = 60°$ if and only if
the Jarlskog quartet has an \emph{odd} number of sign flips:
\begin{equation}
    \boxed{n_{\text{flip}} \equiv 1 \pmod{2} \quad \Longleftrightarrow \quad \delta = 60°}
    \label{eq:ch7_odd_flip_rule}
\end{equation}

\textbf{Minimal case:} A single sign flip ($n_{\text{flip}} = 1$) on any one
of $\{V_{us}, V_{cb}, V_{ub}, V_{cs}\}$ produces $\delta = 60°$.
\end{tcolorbox}

\subsubsection{Geometric Origin: Brane-Reflection Parity}

\paragraph{Setup.}
In the thick-brane geometry, the CKM element $V_{ij}$ arises from an overlap
integral between up-type profile $f^u_i(\xi)$ and down-type profile $f^d_j(\xi)$:
\begin{equation}
    V_{ij} \propto \int_0^\ell f^u_i(\xi) \, f^d_j(\xi) \, w(\xi) \, d\xi
    \label{eq:ch7_ckm_overlap_z2}
\end{equation}

\paragraph{Brane-reflection operator.}
Define the reflection $\mathcal{R}: \xi \mapsto \ell - \xi$ that exchanges the
observer boundary ($\xi = 0$) with the bulk boundary ($\xi = \ell$). Under this
reflection, the overlap integral acquires a transition parity \tagP{}:
\begin{equation}
    \mathcal{R}[V_{ij}] = \eta_{ij} \cdot V_{ij}, \qquad \eta_{ij} \in \{+1, -1\}
    \label{eq:ch7_transition_parity}
\end{equation}

\paragraph{Parity from profile structure.}
The transition parity $\eta_{ij}$ depends on the node structure of the profiles:
\begin{itemize}[nosep]
    \item If $f^u_i$ and $f^d_j$ have the \emph{same} reflection symmetry
          (both even or both odd), then $\eta_{ij} = +1$.
    \item If they have \emph{opposite} symmetries, then $\eta_{ij} = -1$.
\end{itemize}
A profile with an odd number of nodes in $[0, \ell]$ is antisymmetric under
brane reflection; a profile with an even number of nodes is symmetric.

\subsubsection{Minimal Parity Assignment}

For the sign-selection rule to produce $\delta = 60°$, exactly one element
in the Jarlskog quartet must have $\eta = -1$. The two natural candidates are:

\paragraph{\texorpdfstring{Candidate 1: $V_{cb}$ odd parity.}{Candidate 1: V-cb odd parity.}}
The $c \to b$ transition connects the two heaviest non-top quarks. If the
$c$ and $b$ profiles have opposite node parities (e.g., $c$ is a first-excited
state with one node, while $b$ is a ground state), then:
\begin{equation}
    \eta_{cb} = -1, \qquad \eta_{us} = \eta_{ub} = \eta_{cs} = +1
    \label{eq:ch7_vcb_odd}
\end{equation}

\paragraph{\texorpdfstring{Candidate 2: $V_{ub}$ odd parity.}{Candidate 2: V-ub odd parity.}}
The $u \to b$ transition spans the largest ``distance'' in flavor space
(lightest up-type to heaviest down-type). This maximal separation could
cross a nodal surface:
\begin{equation}
    \eta_{ub} = -1, \qquad \eta_{us} = \eta_{cb} = \eta_{cs} = +1
    \label{eq:ch7_vub_odd}
\end{equation}

\paragraph{Status.}
Both candidates produce $\delta = 60°$. The specific choice is \tagP{}---it
requires solving the thick-brane BVP to determine actual profile parities.

\subsubsection{Connection to $(\bar\rho, \bar\eta)$}

The Wolfenstein parameters encode the complex structure of $V_{ub}$:
\begin{equation}
    V_{ub} = A\lambda^3 (\bar\rho - i\bar\eta)
    \label{eq:ch7_vub_wolfenstein_z2}
\end{equation}

With the $\mathbb{Z}_2$ parity mechanism:
\begin{itemize}[nosep]
    \item \textbf{Magnitude:} $|V_{ub}| = A\lambda^3 \sqrt{\bar\rho^2 + \bar\eta^2}$
          from overlap integral norm \tagDc{}
    \item \textbf{Phase:} $\arg(V_{ub})$ from $\mathbb{Z}_6 = \mathbb{Z}_2 \times \mathbb{Z}_3$
          discrete structure \tagDc{}
    \item \textbf{Sign/quadrant:} Determined by which element has odd parity \tagP{}
\end{itemize}

The $(\bar\rho, \bar\eta)$ values are thus structurally constrained by the
$\mathbb{Z}_6$ framework, with the sign selection reducing the discrete ambiguity.

\subsubsection{No-Smuggling Guardrail}

\begin{table}[ht]
\centering
\caption{Epistemic status of Z$_2$ parity mechanism}
\label{tab:ch7_z2_guardrail}
\begin{tabular}{p{5cm}cl}
\toprule
\textbf{Claim} & \textbf{Status} & \textbf{Note} \\
\midrule
Sign-flip count rule (Eq.~\ref{eq:ch7_odd_flip_rule}) & \tagDc{} & Arithmetic identity \\
$\delta = 60°$ from single flip & \tagDc{} & Follows from rule \\
Brane-reflection parity & \tagP{} & Geometric ansatz \\
Specific element ($V_{cb}$ or $V_{ub}$) & \tagP{} & Consistent choice \\
Profile node structure & \tagP{} & Plausible, not derived \\
PDG comparison ($\delta = 65°$) & \tagBL{} & Evaluation only \\
\midrule
\textbf{Parity from BVP solution} & \textbf{[OPEN]} & Requires profile computation \\
\bottomrule
\end{tabular}
\end{table}

\subsubsection{What Remains Open}

\begin{tcolorbox}[colback=gray!5, colframe=gray!50!black,
    title=\textbf{Open Items for Full Closure}]
\begin{enumerate}[nosep]
    \item \textbf{BVP profile parities:} Solve the thick-brane fermion BVP
          (see \S\ref{sec:ch12_bvp_workpackage}) to determine which transitions
          have odd parity from first principles.

    \item \textbf{Residual 5° discrepancy:} $\delta_{\text{pred}} = 60°$ vs.\
          $\delta_{\text{PDG}} = 65°$. May require:
          \begin{itemize}[nosep]
              \item Small continuous deformation of discrete phases
              \item Non-Abelian flavor structure (A$_4$, S$_3$)
              \item Radiative corrections
          \end{itemize}

    \item \textbf{Uniqueness:} Is there a principle that selects $V_{cb}$ over
          $V_{ub}$ (or vice versa) as the odd-parity element?
\end{enumerate}
\end{tcolorbox}

\subsubsection{Summary: Z$_2$ Parity Origin}

\begin{tcolorbox}[colback=green!5, colframe=green!50!black,
    title=\textbf{Summary: Structural Origin of Sign-Selection}]
\textbf{Established [Dc]:}
\begin{itemize}[nosep]
    \item An \emph{odd} number of sign flips in the Jarlskog quartet
          shifts $\delta$ from $120°$ to $60°$.
    \item The minimal case is a single sign flip on one CKM element.
    \item This accounts for the Attempt~4 result without ad hoc phase choice.
\end{itemize}

\textbf{Proposed [P]:}
\begin{itemize}[nosep]
    \item The sign flip arises from brane-reflection parity of overlap integrals.
    \item The odd-parity element is $V_{cb}$ or $V_{ub}$ (two candidates).
    \item Profile node structure determines the parity assignment.
\end{itemize}

\textbf{Status upgrade:}
\begin{center}
\fbox{\parbox{0.9\textwidth}{
\textbf{OPR-11} $(\bar\rho, \bar\eta)$ derivation: RED $\to$ \textbf{YELLOW [Dc]+[P]}

\emph{Established:} Structural mechanism (odd sign-flip rule) and geometric
interpretation (brane-reflection parity).

\emph{Missing:} Specific parity assignment from BVP solution; residual 5° in $\delta$.
}}
\end{center}
\end{tcolorbox}

