% ==============================================================================
% Case Study V: Electron as the Observer-Facing Charged Defect
% ==============================================================================

\subsection{Electron: The Observer-Facing Charged Defect}
\label{sec:case_electron}

\subsubsection{What Is the Electron in EDC Ontology?}

\textbf{Ontology} \tagP{}/\tagDc{}: The electron is the \emph{lightest observer-facing
charged defect} supported by the brane interface. It is the ground-mode brane
defect---the lowest-energy charged excitation of the thick-brane layer.

This is why the electron repeatedly appears as the charged output endpoint in
weak-sector relaxations: it is the lowest-energy stable charged realization
in the allowed output set.

\paragraph{Baseline observable.}
The electron is stable: no decay has ever been observed \tagBL{}. Lower limits
on the electron lifetime exceed $10^{28}$ years for various decay channels.

\subsubsection{Why the Electron Cannot Decay}

The electron's stability follows from three constraints acting together:

\paragraph{1. Charge conservation.}
Any decay must conserve electric charge. The only particles lighter than the
electron are photons and neutrinos, which are electrically neutral. Therefore,
there is no kinematically allowed charged final state \tagBL{}.

\paragraph{2. No lower-lying charged mode.}
In the thick-brane mode spectrum, the electron occupies the ground state of the
charged sector. The muon and tau are excited states of the same sector.
There is no mode below the electron \tagP{}/\tagDc{}.

\paragraph{3. Ledger closure failure.}
Any proposed electron decay would fail to close the energy-charge ledger. For
example:
\begin{itemize}[nosep]
  \item $e^- \to \gamma + \nu$: Violates charge conservation
  \item $e^- \to \nu\nu\nu$: Violates charge conservation
  \item $e^- \to$ (nothing): Violates energy conservation
\end{itemize}

\begin{tcolorbox}[mechanism, title={Electron Stability}]
\textbf{Claim} \tagDc{}: The electron is stable because:
\begin{equation}
\mathcal{P}_{\text{energy}}\big(\text{all potential } e^- \text{ decays}\big) = 0.
\end{equation}
There is no kinematically allowed channel that conserves charge and energy
with the electron as the initial state.

This is not an EDC-specific claim; it is a consequence of the mode spectrum
and conservation laws.
\end{tcolorbox}

\subsubsection{The ``No-Lower-Mode'' Gate}

The electron case introduces a new type of gate in the projection operator:
the stability gate. For the electron:
\begin{equation}
\mathcal{P}_{\text{mode}}(e^- \to X) = 0 \quad \text{for all } X,
\end{equation}
because there is no lower-lying mode $X$ that can receive the electron's charge.

\paragraph{Comparison with muon and tau.}
The muon and tau can decay because there are lower-lying modes (the electron)
to receive their charge. The electron has no such option.

\subsubsection{Role in the Generative Substrate}

The electron, as the stable ground mode, serves as the ``endpoint'' for leptonic
decays. The muon decays to electron; the tau decays to electron or muon. All
chains terminate at the electron because there is nowhere else to go.

This is the first half of what we call the \emph{Generative Closure Principle}:
a stable universe-like output sector requires a lightest charged defect that
serves as the endpoint for all charged cascades.

\subsubsection{Process Diagram: Electron Stability}

\begin{center}
\begin{tikzpicture}[
  scale=0.85,
  box/.style={rectangle, rounded corners=4pt, minimum width=2cm, minimum height=0.8cm,
              draw=black, thick, font=\footnotesize, align=center, text width=2cm},
  gate/.style={rectangle, rounded corners=2pt, minimum width=1.8cm, minimum height=0.6cm,
               draw=red!60!black, thick, fill=red!10, font=\scriptsize, align=center},
  arrow/.style={-{Stealth[length=5pt]}, thick},
  label/.style={font=\scriptsize\itshape}
]

% Electron
\node[box, fill=green!20] (e) at (0,0) {$e^-$\\ground mode};

% Potential decay arrow
\draw[arrow, dashed, gray] (e) -- (3,0);

% Gate
\node[gate] (gate) at (5,0) {No lower\\charged mode};

% Blocked output
\node[box, fill=gray!30] (blocked) at (8,0) {BLOCKED};

% Cross
\draw[ultra thick, red] (7,0.4) -- (9,-0.4);
\draw[ultra thick, red] (7,-0.4) -- (9,0.4);

% Annotation
\node[rectangle, draw=gray, rounded corners=2pt, fill=gray!5,
      font=\scriptsize, align=center, text width=3cm] at (5,-1.8)
  {$Q \neq 0$ requires charged output\\No lighter charged particle exists};

\end{tikzpicture}
\end{center}

\subsubsection{Falsifiability Hooks}

\begin{tcolorbox}[falsifiability]
\begin{itemize}[nosep]
  \item If electron decay is observed, the ``ground mode'' claim fails.
  \item If a lighter charged particle is discovered, the mode spectrum picture
        requires revision.
  \item If the electron mass cannot be connected to the ground-mode energy of
        the thick-brane potential, the ontology is incomplete \tagOpen{}.
\end{itemize}
\end{tcolorbox}

