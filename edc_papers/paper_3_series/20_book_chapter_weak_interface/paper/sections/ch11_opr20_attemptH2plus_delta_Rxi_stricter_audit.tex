%!TEX root = ../EDC_Part_II_Weak_Sector.tex
% ==============================================================================
% OPR-20b Attempt H2-plus: STRICTER Audit of $\delta$ = R_ξ Identification
% Status: AUDIT REPORT — evaluates whether $\delta$ = R_ξ can be upgraded beyond [P]
% ==============================================================================
%
% MEGA-PROMPT REQUIREMENT:
% $\delta$ = R_ξ must emerge from Part I definitions + 5D action/junction
% WITHOUT profile ansatz. Two independent routes must converge.
% If required definitions don't exist → $\delta$ = R_ξ stays [P].
%
% ==============================================================================

\subsection{Attempt H2-plus: Stricter Audit of the \texorpdfstring{$\delta = R_\xi$}{delta = Rxi} Identification}
\label{sec:ch11_opr20_attemptH2plus}

Attempt~H claimed $\delta$ = R$_\xi$ as a ``definitional identification'' [Def], upgrading
from bare postulate [P]. This audit applies \emph{stricter} criteria: the identification
must emerge from \textbf{Part~I formal definitions} via \textbf{two independent routes}
that converge to the same result. Profile ansätze or plausibility arguments do not
qualify for [Dc] status.

\begin{tcolorbox}[colback=red!5!white, colframe=red!60!black,
    title=\textbf{H2-plus Audit Criteria (No Smuggling)}]
\textbf{Required for [Dc] upgrade:}
\begin{enumerate}[nosep]
    \item Equation-anchored definition of $R_\xi$ in Part~I (label + formal statement)
    \item Formal ``boundary-layer thickness'' definition in EDC language
    \item Route A: Diffusion PDE $\to$ boundary layer theorem $\to$ $\delta$
    \item Route B: 5D action + junction $\to$ Robin BC $\to$ scale identification
    \item Convergence: Both routes yield $\delta = R_\xi$ independently
\end{enumerate}

\textbf{Forbidden (no-smuggling):}
\begin{itemize}[nosep]
    \item Using $M_Z$, $M_W$, $G_F$, or $v$ to fix $R_\xi$ value
    \item Profile ansatz $f(\xi) \propto e^{-\xi/\delta}$ assumed \emph{a priori}
    \item ``Only available scale'' arguments without formal uniqueness proof
    \item New $\mathcal{O}(1)$ parameters tuned to match result
\end{itemize}
\end{tcolorbox}

% ------------------------------------------------------------------------------
\subsubsection{H2.1: Audit of R$_\xi$ Definition Status in Part I}
\label{sec:attemptH2_Rxi_audit}

\paragraph{Search result.}
A comprehensive search of Part~I and Part~II sources yields:

\begin{tcolorbox}[colback=gray!5!white, colframe=gray!60!black,
    title=\textbf{Finding: R$_\xi$ Definition Status}]

\textbf{Where defined:}
\begin{itemize}[nosep]
    \item Framework~v2.0: ``Membrane thickness / weak-KK scale'' \tagDc{}
    \item Part~II Ch11 (Attempt~H, line 98--124): Correlation length interpretation
\end{itemize}

\textbf{Numerical parameterization:}
\begin{equation}
    R_\xi = \frac{\hbar c}{M_Z} \approx 2 \times 10^{-3} \text{ fm}
    \label{eq:H2plus_Rxi_MZ}
\end{equation}

\textbf{Critical finding:}
\begin{center}
\fcolorbox{red}{yellow!20}{\parbox{0.85\textwidth}{
\textbf{R$_\xi$ is NOT derived from EDC action.}\\[2pt]
The value is \textbf{phenomenologically constrained} by electroweak observables
(specifically $M_Z = 91.2$ GeV). Deriving $R_\xi$ from the EDC action is listed
as an \textbf{OPEN problem} in the Framework.
}}
\end{center}

\textbf{Status:} \tagP{} (Part~I physics, constrained by EW phenomenology)
\end{tcolorbox}

\paragraph{\texorpdfstring{Implication for $\delta$ = R$_\xi$.}{Implication for delta = R-xi.}}
If $R_\xi$ itself is [P] (constrained, not derived), then any identification
$\delta = R_\xi$ inherits [P] status. The chain cannot be stronger than its weakest link.

% ------------------------------------------------------------------------------
\subsubsection{H2.2: Audit of ``Unique Transverse Scale'' Claim}
\label{sec:attemptH2_unique_scale}

Attempt~H (line 162--163) asserts: ``There is no other intrinsic length available:
$R_\xi$ is the \emph{only} sub-electroweak scale from Part~I physics.''

\paragraph{Search for formal theorem.}
\begin{itemize}[nosep]
    \item Pattern: ``unique.*scale'', ``transverse.*scale'', ``only.*length''
    \item Result: \textbf{No formal theorem found}
\end{itemize}

\begin{tcolorbox}[colback=gray!5!white, colframe=gray!60!black,
    title=\textbf{Finding: Unique Transverse Scale}]
\textbf{EXISTS:} Plausibility argument (Attempt~H, line 162--163)

\textbf{DOES NOT EXIST:} Formal proof that $R_\xi$ is the unique transverse scale

\textbf{Why this matters:}
The argument ``$R_\xi$ is the only scale, therefore $\delta = R_\xi$'' requires
proving there is no other candidate. Possible alternative scales include:
\begin{itemize}[nosep]
    \item Electron Compton wavelength $\bar{\lambda}_e \approx 3.9 \times 10^{-3}$ fm
    \item Classical electron radius $r_e \approx 2.8 \times 10^{-3}$ fm
    \item Proton charge radius $r_p \approx 0.84$ fm
    \item Ratio combinations: $r_e^2/R_\xi$, $\sqrt{r_e R_\xi}$, etc.
\end{itemize}
Without a formal exclusion of these alternatives, ``unique scale'' is an assertion [P],
not a derivation [Dc].
\end{tcolorbox}

% ------------------------------------------------------------------------------
\subsubsection{H2.3: Route A — Diffusion PDE → Boundary Layer Theorem}
\label{sec:attemptH2_routeA}

\paragraph{Required structure.}
Route A would derive $\delta$ from the diffusion dynamics of the membrane:
\begin{enumerate}[nosep]
    \item Start from diffusion PDE in Part~I (e.g., $\partial_t \phi = D \nabla^2 \phi$)
    \item Apply boundary layer analysis (matched asymptotic expansions)
    \item Extract characteristic thickness $\delta_{\text{BL}}$ as function of $D$, $\tau$
    \item Show $\delta_{\text{BL}} = R_\xi$ from the definitions
\end{enumerate}

\paragraph{Search result.}

\begin{tcolorbox}[colback=red!5!white, colframe=red!50!black,
    title=\textbf{Finding: Route A Status}]
\textbf{Diffusion PDE in Part I:} Exists (frozen membrane regime)

\textbf{Boundary layer theorem:} \textcolor{BrickRed}{\textbf{DOES NOT EXIST}}

\textbf{Matched asymptotic analysis:} Not performed

\textbf{$\delta_{\text{BL}}$ derivation:} Not available

\medskip
\textbf{Route A status:} \textcolor{BrickRed}{\textbf{BLOCKED}} — required derivation missing
\end{tcolorbox}

\paragraph{What would be needed.}
A proper Route A derivation would require:
\begin{itemize}[nosep]
    \item Definition of boundary layer region $0 < z < \delta$
    \item Inner solution (near boundary) matching to outer solution (bulk)
    \item Identification $\delta = \sqrt{D \tau}$ or similar from matching conditions
    \item Proof that $\sqrt{D \tau} = R_\xi$ using Part~I definitions
\end{itemize}
This analysis is \textbf{not present} in the current repo. It represents genuine
future work, not a gap that can be filled by reformulation.

% ------------------------------------------------------------------------------
\subsubsection{H2.4: Route B — Junction → Robin BC → Scale Identification}
\label{sec:attemptH2_routeB}

\paragraph{Required structure.}
Route B would derive $\delta$ from the junction conditions:
\begin{enumerate}[nosep]
    \item Start from 5D action with junction (Israel matching or Gibbons-Hawking-York)
    \item Vary action to obtain junction conditions on fields
    \item Map junction to Robin BC: $f' + \alpha f = 0$ with $\alpha = f(\text{junction params})$
    \item Identify $\delta = 1/\alpha$ (or similar) from dimensional analysis
    \item Show this $\delta$ equals $R_\xi$ using Part~I definitions
\end{enumerate}

\paragraph{Search result.}

\begin{tcolorbox}[colback=yellow!5!white, colframe=yellow!60!black,
    title=\textbf{Finding: Route B Status}]
\textbf{Robin BC from thick-brane action:} \tagDc{} (Attempt~H, Eq.~3--5)

\textbf{$\alpha \sim \ell/\delta$ dimensional structure:} \tagDc{}

\textbf{Junction → $\delta$ derivation:} \textcolor{BrickRed}{\textbf{INCOMPLETE}}

The dimensional relation $\alpha \sim \ell/\delta$ is derived, but the critical step
--- determining $\delta$ from junction physics --- is \emph{assumed}, not derived:

\begin{quote}
``The boundary-layer thickness $\delta$ is the scale over which fields transition
from bulk-dominated to brane-dominated behavior. \textbf{In the thick-brane model,
this transition is controlled by diffusion---the same process that sets $R_\xi$.}''
(Attempt~H, lines 143--145)
\end{quote}

This is a \emph{plausibility argument}, not a derivation. The connection
``diffusion controls transition'' $\Rightarrow$ ``$\delta = R_\xi$'' is asserted,
not proven.

\medskip
\textbf{Route B status:} \textcolor{orange}{\textbf{PARTIAL}} — Robin BC derived [Dc],
but $\delta = R_\xi$ step is [P]
\end{tcolorbox}

% ------------------------------------------------------------------------------
\subsubsection{H2.5: Convergence Check — Do Routes A and B Agree?}
\label{sec:attemptH2_convergence}

\begin{tcolorbox}[colback=red!5!white, colframe=red!50!black,
    title=\textbf{Finding: Convergence Status}]
\textbf{Route A:} BLOCKED (boundary layer theorem missing)

\textbf{Route B:} PARTIAL ($\delta = R_\xi$ assumed at final step)

\textbf{Convergence test:} \textcolor{BrickRed}{\textbf{CANNOT BE PERFORMED}}

Without two independent derivations, there is nothing to compare.
The ``$\delta$ = R$_\xi$'' identification rests on a single plausibility chain, not
convergent derivations.
\end{tcolorbox}

% ------------------------------------------------------------------------------
\subsubsection{H2.6: Checklist Summary — EXISTS / DOES NOT EXIST}
\label{sec:attemptH2_checklist}

\begin{table}[ht]
\centering
\caption{H2-plus Audit Checklist}
\label{tab:H2plus_checklist}
\small
\begin{tabular}{p{6cm}cc}
\toprule
\textbf{Required Element} & \textbf{Status} & \textbf{Reference} \\
\midrule
(1) $R_\xi$ formal definition (Part~I) & \textcolor{ForestGreen}{EXISTS} & Framework~v2.0 \\
(2) $R_\xi$ equation-anchored & \textcolor{orange}{PARTIAL} & $R_\xi = \hbar c/M_Z$ \\
(3) $R_\xi$ derived from action & \textcolor{BrickRed}{DOES NOT EXIST} & [OPEN] \\
(4) Boundary-layer formal definition & \textcolor{BrickRed}{DOES NOT EXIST} & — \\
(5) ``Unique transverse scale'' theorem & \textcolor{BrickRed}{DOES NOT EXIST} & — \\
(6) Route A (diffusion → BL theorem) & \textcolor{BrickRed}{DOES NOT EXIST} & — \\
(7) Route B (junction → $\delta$ derivation) & \textcolor{orange}{PARTIAL} & Attempt~H \\
(8) Two-route convergence & \textcolor{BrickRed}{CANNOT TEST} & — \\
\bottomrule
\end{tabular}
\end{table}

% ------------------------------------------------------------------------------
\subsubsection{H2.7: Honest Verdict — $\delta$ = R$_\xi$ Remains [P]}
\label{sec:attemptH2_verdict}

\begin{tcolorbox}[
    colback=red!5!white,
    colframe=red!70!black,
    title=\textbf{OPR-20b Attempt H2-plus: Stricter Audit Verdict}
]
\begin{center}
\textbf{\large $\delta$ = R$_\xi$ REMAINS [P] (Postulated)}
\end{center}

\medskip
\textbf{Why the identification cannot be upgraded:}
\begin{enumerate}[nosep]
    \item \textbf{$R_\xi$ source:} Constrained by $M_Z$ (EW phenomenology), not derived
          from EDC action
    \item \textbf{``Unique scale'' claim:} Asserted without formal proof; alternative
          scales not rigorously excluded
    \item \textbf{Route A:} Blocked — no boundary-layer theorem exists
    \item \textbf{Route B:} Partial — final $\delta = R_\xi$ step is assumed
    \item \textbf{Convergence:} Cannot be tested with only one (incomplete) route
\end{enumerate}

\medskip
\textbf{What IS established [Dc]:}
\begin{itemize}[nosep]
    \item Robin BC emerges from thick-brane variation
    \item $\alpha \sim \ell/\delta$ dimensional structure
    \item $\ell = 2\pi R_\xi$ circumference relation
\end{itemize}

\textbf{What remains [P]:}
\begin{itemize}[nosep]
    \item $R_\xi$ value (from EW phenomenology)
    \item $\delta = R_\xi$ identification (plausibility, not derivation)
    \item Mediator mass $m_\phi$ (inherits [P] from $R_\xi$)
\end{itemize}

\textbf{Sub-gates under OPR-20b [OPEN]:}
\begin{itemize}[nosep]
    \item (i) Derive $R_\xi$ from EDC action
    \item (ii) Boundary-layer theorem from diffusion PDE
    \item (iii) ``Unique transverse scale'' formal proof
    \item (iv) \textbf{OR} demonstrate $\delta$-robustness (results insensitive to $\delta$ in wide band)
\end{itemize}
\end{tcolorbox}

% ------------------------------------------------------------------------------
\subsubsection{H2.8: Revised OPR-20b Status}
\label{sec:attemptH2_opr20_update}

\begin{tcolorbox}[colback=gray!5!white, colframe=gray!60!black,
    title=\textbf{OPR-20b Status After H2-plus Audit}]

\textbf{Before H2-plus:}
\begin{quote}
OPR-20b ($\alpha$ provenance): YELLOW [Def]+[P] \\
``The $\delta = R_\xi$ gate is now definitionally closed.''
\end{quote}

\textbf{After H2-plus (stricter audit):}
\begin{quote}
OPR-20b ($\alpha$ provenance): \textbf{YELLOW [P]+[OPEN]} \\
``The $\delta = R_\xi$ identification is plausible [P] but not derived [Dc].
Three formal gaps remain: (i) derive $R_\xi$ from action, (ii) boundary-layer
theorem, (iii) unique-scale proof.''
\end{quote}

\textbf{Change:} [Def] → [P] (downgrade: ``definitional'' claim not justified by
stricter standards)

\textbf{Sub-gates (folded under OPR-20b):}
\begin{itemize}[nosep]
    \item (i) Derive $R_\xi$ from EDC action
    \item (ii) Boundary-layer theorem from diffusion
    \item (iii) Unique transverse scale proof
    \item (iv) \textbf{OR} demonstrate $\delta$-robustness band (alternative closure)
\end{itemize}
\end{tcolorbox}

% ------------------------------------------------------------------------------
\subsubsection{H2.9: Path Forward — What Would Close the Gate?}
\label{sec:attemptH2_path_forward}

\begin{enumerate}
    \item \textbf{Sub-gate (i) — Derive $R_\xi$:} Derive $R_\xi$ from the 5D EDC
          action without using $M_Z$ or $M_W$ as input. This would require showing
          that diffusion/frozen dynamics uniquely select the scale $R_\xi \sim 10^{-18}$ m.

    \item \textbf{Sub-gate (ii) — BL theorem:} Perform matched asymptotic analysis on the diffusion
          PDE near the boundary. Extract boundary-layer thickness $\delta_{\text{BL}}$
          and show $\delta_{\text{BL}} = R_\xi$.

    \item \textbf{Sub-gate (iii) — Unique scale:} Prove that $R_\xi$ is the \emph{unique} sub-electroweak
          scale in Part~I. This requires systematically excluding alternatives
          ($\bar{\lambda}_e$, $r_e$, ratio combinations).

    \item \textbf{Sub-gate (iv) — $\delta$-robustness (alternative):} Show that
          observables depend weakly on $\delta$ in a wide band. If predictions
          are stable across $\delta \in [0.5 R_\xi, 2 R_\xi]$, then $\delta$
          provenance becomes less critical.
\end{enumerate}

\paragraph{Interim strategy.}
Until Gates 20c--20e are closed, the pragmatic approach is:
\begin{itemize}[nosep]
    \item Accept $\delta = R_\xi$ as [P] with explicit acknowledgment
    \item Track numeric predictions as \tagDc{}+\tagP{} (structure derived, scale postulated)
    \item Maintain separation between ``what is derived'' and ``what is assumed''
\end{itemize}

This is \textbf{not a failure} — it is \textbf{honest bookkeeping}. Many successful
physical theories operate with phenomenologically constrained parameters. The key
is transparency about what has and has not been derived.

\begin{tcolorbox}[colback=blue!5!white, colframe=blue!50!black,
    title=\textbf{Book-Ready Statement (Reviewer-Safe)}]
In the present Part~II, $R_\xi$ is a \emph{phenomenologically constrained}
transverse length scale (set by the electroweak KK scale), not yet derived
from the EDC action; therefore the identification $\delta = R_\xi$ is recorded
as a postulate \tagP{} and remains an explicit upgrade gate.
\end{tcolorbox}

% ------------------------------------------------------------------------------
\subsubsection{H2.10: Comparison — Attempt H vs. H2-plus}
\label{sec:attemptH2_comparison}

\begin{table}[ht]
\centering
\caption{Attempt H vs. H2-plus: Epistemic Status Comparison}
\label{tab:H_vs_H2plus}
\small
\begin{tabular}{lcc}
\toprule
\textbf{Claim} & \textbf{Attempt H} & \textbf{H2-plus Audit} \\
\midrule
$R_\xi$ definition & Exists [Dc] & Exists, but value [P] \\
$\delta = R_\xi$ & [Def] (definitional) & \textbf{[P]} (not derived) \\
``Unique scale'' & Asserted & \textbf{[OPEN]} (no proof) \\
Route A (diffusion) & Not attempted & \textbf{BLOCKED} (missing) \\
Route B (junction) & Claimed complete & \textbf{PARTIAL} (final step [P]) \\
Convergence & Not tested & \textbf{CANNOT TEST} \\
\addlinespace
\textbf{Overall OPR-20b} & YELLOW [Def]+[P] & \textbf{YELLOW [P]+[OPEN]} \\
\bottomrule
\end{tabular}
\end{table}

% ------------------------------------------------------------------------------
\subsubsection{H2.11: Guardrail Box — No-Smuggling Verification}
\label{sec:attemptH2_guardrail}

\begin{tcolorbox}[colback=green!5!white, colframe=green!50!black,
    title=\textbf{No-Smuggling Verification (H2-plus)}]

\textbf{Checked and clean:}
\begin{itemize}[nosep]
    \item[$\checkmark$] No $M_W$, $G_F$, $g_2$, $v$ used to derive $\delta$
    \item[$\checkmark$] No profile ansatz assumed \emph{a priori}
    \item[$\checkmark$] Robin BC derived from action, not postulated
    \item[$\checkmark$] $\ell = 2\pi R_\xi$ traced to Part~I
\end{itemize}

\textbf{Honestly acknowledged as [P]:}
\begin{itemize}[nosep]
    \item[$\circ$] $R_\xi$ value constrained by $M_Z$ (EW phenomenology)
    \item[$\circ$] $\delta = R_\xi$ identification (plausibility argument)
    \item[$\circ$] ``Unique scale'' claim (asserted, not proven)
\end{itemize}

\textbf{Sub-gates [OPEN] (under OPR-20b):}
\begin{itemize}[nosep]
    \item[$\times$] (i) Derive $R_\xi$ from action
    \item[$\times$] (ii) Boundary-layer theorem
    \item[$\times$] (iii) Unique-scale proof
    \item[$\times$] (iv) OR: $\delta$-robustness demonstration
\end{itemize}
\end{tcolorbox}

% ==============================================================================
% END OF H2-plus AUDIT
% ==============================================================================
