% ==============================================================================
% Section 1: How We Got Here
% ==============================================================================

This section answers the question that every careful reader should ask: ``How did we
arrive at this brane-interface picture, and why is it not merely an arbitrary
construction?''

\subsection{What We Are Trying to Explain (and What We Are Not)}

In the Standard Model, ``weak interactions'' are encoded as fundamental vertices and
gauge boson exchange. The $W^\pm$ and $Z^0$ bosons mediate weak processes, and the
interaction strength is set by the Fermi constant $G_F$ \tagBL{}.

In EDC, we pursue a different explanatory target: we aim to describe the observed
weak-sector phenomena as the \emph{observer-facing residue} of a bulk-to-brane transfer
process in a thick-brane geometry \tagP{}/\tagDc{}.

This chapter therefore does \emph{not} attempt a full Standard-Model derivation.
Instead, we build a mechanistic pipeline that is:
\begin{enumerate}[nosep]
  \item dimensionally consistent,
  \item explicit about what is baseline data versus hypothesis,
  \item structured so that future numerical closure remains well-defined rather than
        hidden inside language.
\end{enumerate}

\subsection{Why ``Weak'' Is Not a Fundamental Vertex in EDC}

The Standard Model treats weak interactions as arising from $SU(2)_L$ gauge symmetry.
The ``weakness'' comes from the large $W$ mass ($\sim 80$ GeV) appearing in propagator
denominators at low energies.

EDC offers a different interpretation \tagP{}:
\begin{quote}
\emph{Weak interactions are not fundamental vertices but the low-energy residue of
bulk$\to$brane energy transfer, geometrically suppressed by mediator gaps and
wavefunction overlaps.}
\end{quote}

This is not a claim that the Standard Model is wrong. Rather, it is a claim that
the Standard Model's effective description may have a deeper geometric origin in
a thick-brane microphysics.

\subsection{Why a Thick Brane Is Essential (Not Optional)}

A common simplification in extra-dimensional models is to treat the brane as a
delta-function localization. EDC requires a \emph{thick} brane for three essential
reasons \tagDc{}:

\paragraph{1. A reservoir for energy storage and redistribution.}
Bulk relaxation can pump energy into the brane layer, where it can be temporarily
stored and then redistributed among brane-layer modes before any 3D ``particle''
output is projected. A zero-thickness brane has no such storage capacity.

\paragraph{2. Mode overlap and localization.}
Effective couplings become overlap integrals of mode profiles across the brane
thickness. This provides a natural, geometric route to ``small effective couplings''
without inserting small numbers by hand \tagOpen{}.

\paragraph{3. A boundary/projection stage.}
The observer does not read off raw 5D fields. Instead, outputs are produced after
a ``frozen'' projection step (the observer-facing boundary condition), which can
enforce selection rules (including chirality selection) as a boundary phenomenon
\tagP{}/\tagOpen{}.

\subsection{Connecting to Observable Weak Phenomena}

The phenomena we must explain include \tagBL{}:

\begin{itemize}
  \item \textbf{Neutron $\beta$-decay}: $n \to p + e^- + \bar\nu_e$ with
        $\tau_n \approx 879$ s
  \item \textbf{Muon decay}: $\mu^- \to e^- + \bar\nu_e + \nu_\mu$ with
        $\tau_\mu \approx 2.2 \times 10^{-6}$ s
  \item \textbf{Tau decay}: Multiple channels (leptonic and hadronic) with
        $\tau_\tau \approx 2.9 \times 10^{-13}$ s
  \item \textbf{Pion decay}: $\pi^+ \to \mu^+ + \nu_\mu$ (dominant) with strong
        helicity suppression of the electron channel
  \item \textbf{Electron stability}: The lightest charged lepton does not decay
  \item \textbf{Neutrino properties}: Nearly massless, only left-handed coupling
        to weak currents
\end{itemize}

Each of these phenomena will receive a mechanistic interpretation in the case
studies (\S\ref{sec:case_studies}). The key insight is that they all share a
common interface logic: energy arrives from a bulk-facing process, is processed
in a thick-brane layer, and is then projected through boundary conditions into
an allowed set of 3D observable outputs.

\subsection{From Apparent Vertex to Coarse-Grained Residue}

In this mechanistic picture, a low-energy effective interaction term in 3D
should be read as a \emph{coarse-grained residue} of 5D transfer, not as a
fundamental interaction at a point \tagDc{}.

This is the logic behind the structural derivation of an effective four-fermion
coupling (see \S\ref{sec:GF_structural}): one couples a brane-facing current $J(x)$
to a mediator field supported in the thick brane; integrating out the mediator
produces a local contact term $J \cdot J$ with suppression controlled by mediator
gap and geometric overlap---not a tunable ``weak strength.''

The program is therefore:
\begin{enumerate}
  \item Identify the bulk-facing trigger for each weak process
  \item Describe the brane-charging (absorption) stage
  \item Characterize the mode redistribution (dissipation) stage
  \item Apply the frozen projection operator (release) to obtain 3D outputs
  \item Check that the energy ledger closes and selection rules are respected
\end{enumerate}

This is what we mean by ``mechanistic'': not a black box labeled ``weak vertex,''
but an explicit causal chain from 5D dynamics to 3D observation.
