% ==============================================================================
% Epistemic Map: What Is Known, Derived, and Open
% ==============================================================================

\subsection{Quantitative Summary: Thresholds and Gates}
\label{sec:quantitative_summary}

Before cataloging the epistemic status of each claim, we present the quantitative
data that underlies the case studies. This table is \textbf{not} EDC-specific;
it is baseline physics \tagBL{} that any framework must reproduce.

\subsubsection{Q-Gates and Kinematic Thresholds}

\begin{center}
\begin{tabular}{llccc}
\toprule
\textbf{Decay} & \textbf{Channel} & \textbf{$Q$-value} & \textbf{Gate} & \textbf{Status} \\
\midrule
\multirow{2}{*}{Neutron} & $n \to p + e^- + \bar\nu_e$ &
  $+0.782$ MeV & $\mathcal{P}_{\text{energy}}$ & OPEN \\
& $n \to p + \mu^- + \bar\nu_\mu$ &
  $-104.4$ MeV & $\mathcal{P}_{\text{energy}}$ & CLOSED \\
\addlinespace
\multirow{2}{*}{Muon} & $\mu^- \to e^- + \bar\nu_e + \nu_\mu$ &
  $+105.1$ MeV & $\mathcal{P}_{\text{energy}}$ & OPEN \\
& $\mu^- \to \text{hadrons}$ &
  --- & $\mathcal{P}_{\text{mode}}$ & FORBIDDEN \\
\addlinespace
\multirow{2}{*}{Tau} & $\tau^- \to e^-/\mu^- + \nu\bar\nu$ &
  $+1776/1671$ MeV & $\mathcal{P}_{\text{energy}}$ & OPEN \\
& $\tau^- \to \text{hadrons} + \nu_\tau$ &
  $+1637$ MeV & $\mathcal{P}_{\text{mode}}$ & OPEN \\
\addlinespace
\multirow{2}{*}{Pion} & $\pi^+ \to \mu^+ + \nu_\mu$ &
  $+33.9$ MeV & $\mathcal{P}_{\text{chir}}$ & OPEN \\
& $\pi^+ \to e^+ + \nu_e$ &
  $+139.1$ MeV & $\mathcal{P}_{\text{chir}}$ & SUPPRESSED \\
\addlinespace
Electron & $e^- \to X$ & --- & No lower mode & BLOCKED \\
\bottomrule
\end{tabular}
\end{center}

\paragraph{Reading the table.}
\begin{itemize}[nosep]
  \item $Q > 0$: kinematically allowed (energy available for products)
  \item $Q < 0$: kinematically forbidden (would violate energy conservation)
  \item SUPPRESSED: allowed but with reduced amplitude (helicity suppression)
  \item FORBIDDEN: blocked by mode mismatch, not kinematics
  \item BLOCKED: no decay channel exists
\end{itemize}

\subsubsection{Mass and Lifetime Data}

\begin{center}
\begin{tabular}{lcccc}
\toprule
\textbf{Particle} & \textbf{Mass (MeV)} & \textbf{Lifetime} &
\textbf{Ontology} & \textbf{Dominant Gate} \\
\midrule
Neutron & $939.565$ & $879.4$ s & Bulk-core junction &
$\mathcal{P}_{\text{energy}}$ \\
Muon & $105.66$ & $2.20~\mu$s & Brane-dominant &
$\mathcal{P}_{\text{mode}}$ \\
Tau & $1776.9$ & $0.290$ ps & Brane-dominant &
$\mathcal{P}_{\text{energy}}$ \\
Pion & $139.57$ & $26.0$ ns & Junction-pair &
$\mathcal{P}_{\text{chir}}$ \\
Electron & $0.511$ & $> 10^{28}$ yr & Brane defect (ground) &
None (stable) \\
Neutrino & $< 10^{-6}$ & Stable & Edge mode &
Overlap suppression \\
\bottomrule
\end{tabular}
\end{center}

All values are \tagBL{} (PDG 2024). The ``Ontology'' and ``Dominant Gate'' columns
are EDC interpretations \tagP{}/\tagDc{}.

\subsubsection{What the Table Shows}

This quantitative summary demonstrates that:
\begin{enumerate}[nosep]
  \item \textbf{Channel selection is kinematic}: Neutron $\to$ electron (not muon)
        because $Q_\beta(\mu) < 0$.
  \item \textbf{Mode overlap matters}: Muon $\to$ leptons only because mode mismatch
        forbids hadronic channels.
  \item \textbf{Chirality suppression is real}: Pion $\to$ muon dominates over
        electron by $(m_\mu/m_e)^2 \approx 4 \times 10^4$.
  \item \textbf{Electron stability is structural}: No lower charged mode exists.
\end{enumerate}

These are \emph{facts} that EDC must be consistent with; they are not EDC-derived
claims.

\vspace{1em}

This section provides a comprehensive summary of the epistemic status of each
claim made in this chapter. The goal is transparency: the reader should know
exactly what is established, what is structural interpretation, and what
remains to be computed.

\subsection{The Five Categories}

Throughout this chapter, we have used the following epistemic tags:

\begin{center}
\begin{tabular}{clp{8cm}}
\toprule
\textbf{Tag} & \textbf{Status} & \textbf{Meaning} \\
\midrule
\tagBL{} & Baseline & Established experimental fact or Standard Model result \\
\tagDef{} & Definition & Terminological convention adopted in this work \\
\tagP{} & Postulate & Structural assumption or hypothesis \\
\tagDc{} & Deduction & Derived from postulates via explicit reasoning \\
\tagOpen{} & Open & Requires further work; not yet computed or proven \\
\bottomrule
\end{tabular}
\end{center}

\subsection{Baseline Facts (What We Must Reproduce)}

The following are empirical facts that EDC must be consistent with:

\subsubsection{Particle Properties}

\begin{center}
\begin{tabular}{lll}
\toprule
\textbf{Quantity} & \textbf{Value} & \textbf{Source} \\
\midrule
Neutron mass & $m_n = 939.565$ MeV & PDG \\
Proton mass & $m_p = 938.272$ MeV & PDG \\
Electron mass & $m_e = 0.511$ MeV & PDG \\
Muon mass & $m_\mu = 105.66$ MeV & PDG \\
Tau mass & $m_\tau = 1776.9$ MeV & PDG \\
Pion mass & $m_{\pi^\pm} = 139.57$ MeV & PDG \\
\bottomrule
\end{tabular}
\end{center}

\subsubsection{Lifetimes}

\begin{center}
\begin{tabular}{lll}
\toprule
\textbf{Particle} & \textbf{Lifetime} & \textbf{Source} \\
\midrule
Neutron & $\tau_n \approx 880$ s & PDG \\
Muon & $\tau_\mu \approx 2.2 \times 10^{-6}$ s & PDG \\
Tau & $\tau_\tau \approx 2.9 \times 10^{-13}$ s & PDG \\
Pion & $\tau_\pi \approx 2.6 \times 10^{-8}$ s & PDG \\
Electron & $> 10^{28}$ years & PDG (limit) \\
\bottomrule
\end{tabular}
\end{center}

\subsubsection{Decay Channels and Branching Ratios}

\begin{center}
\begin{tabular}{lll}
\toprule
\textbf{Decay} & \textbf{Branching Ratio} & \textbf{Status} \\
\midrule
$n \to p + e^- + \bar\nu_e$ & $\approx 100\%$ & \tagBL{} \\
$\mu^- \to e^- + \bar\nu_e + \nu_\mu$ & $\approx 100\%$ & \tagBL{} \\
$\tau^- \to e^- + \bar\nu_e + \nu_\tau$ & $\approx 17.8\%$ & \tagBL{} \\
$\tau^- \to \mu^- + \bar\nu_\mu + \nu_\tau$ & $\approx 17.4\%$ & \tagBL{} \\
$\tau^- \to \text{hadrons} + \nu_\tau$ & $\approx 64.8\%$ & \tagBL{} \\
$\pi^+ \to \mu^+ + \nu_\mu$ & $\approx 99.99\%$ & \tagBL{} \\
$\pi^+ \to e^+ + \nu_e$ & $\approx 0.012\%$ & \tagBL{} \\
\bottomrule
\end{tabular}
\end{center}

\subsubsection{Coupling Constants}

\begin{center}
\begin{tabular}{lll}
\toprule
\textbf{Quantity} & \textbf{Value} & \textbf{Source} \\
\midrule
Fermi constant & $G_F = 1.166 \times 10^{-5}~\text{GeV}^{-2}$ & PDG \\
$W$ boson mass & $M_W = 80.4$ GeV & PDG \\
Fine structure const. & $\alpha \approx 1/137$ & CODATA \\
\bottomrule
\end{tabular}
\end{center}

\subsection{Postulates (Structural Assumptions)}

The following are hypotheses that define the EDC framework:

\begin{center}
\begin{tabular}{p{4cm}p{9cm}}
\toprule
\textbf{Postulate} & \textbf{Statement} \\
\midrule
Thick brane & The 3D universe is a finite-thickness layer in 5D \\
Bulk-core particles & Neutron, proton have 5D bulk structure \\
Brane-dominant modes & Leptons are excitations of the brane layer \\
Edge modes & Neutrinos are localized at the bulk-brane interface \\
Frozen projection & Observer-facing boundary is quasi-static \\
Pipeline structure & Weak decays proceed via absorption-dissipation-release \\
Mode overlap & Branching ratios depend on wavefunction overlaps \\
Chirality projection & Boundary conditions select helicity \\
\bottomrule
\end{tabular}
\end{center}

\subsection{Deductions (What Follows from Postulates)}

The following claims are derived from the postulates:

\subsubsection{Qualitative Deductions}

\begin{center}
\begin{tabular}{p{5cm}p{8cm}}
\toprule
\textbf{Claim} & \textbf{Derivation Path} \\
\midrule
Neutron decays to electron (not muon) & Kinematic threshold: $Q_\beta(\mu) < 0$ \\
Electron is stable & No lower-lying charged mode exists \\
Muon decay is purely leptonic & Mode mismatch with hadrons \\
Tau has hadronic channels & Higher mode energy opens thresholds \\
Neutrinos interact weakly & Edge-mode localization suppresses overlap \\
\bottomrule
\end{tabular}
\end{center}

\subsubsection{Quantitative Deductions}

\begin{center}
\begin{tabular}{p{4cm}p{5cm}p{4cm}}
\toprule
\textbf{Quantity} & \textbf{EDC Expression} & \textbf{Status} \\
\midrule
$Q_\beta(e)$ value & $m_n - m_p - m_e = 0.782$ MeV & \tagDc{} (arithmetic) \\
$Q_\beta(\mu)$ sign & $< 0$ (channel closed) & \tagDc{} \\
$R_{e/\mu}$ scaling & $\propto (m_e/m_\mu)^2$ & \tagBL{} + \tagP{} \\
\bottomrule
\end{tabular}
\end{center}

\subsection{Open Problems (What Remains to Be Done)}

The following require further work:

\subsubsection{Critical Open Problems}

\begin{center}
\begin{tabular}{p{5cm}p{8cm}}
\toprule
\textbf{Problem} & \textbf{What Is Needed} \\
\midrule
Neutron lifetime value & Compute tunneling rate from 5D junction dynamics \\
$G_F$ derivation & Compute overlap integral in thick-brane background \\
Helicity suppression factor & Solve Dirac equation with boundary conditions \\
Mode spectrum & Solve thick-brane eigenvalue problem \\
Neutrino mass & Compute edge-mode energy \\
\bottomrule
\end{tabular}
\end{center}

\subsubsection{Important but Non-Critical}

\begin{center}
\begin{tabular}{p{5cm}p{8cm}}
\toprule
\textbf{Problem} & \textbf{What Is Needed} \\
\midrule
Tau branching ratios & Compute mode overlaps for hadronic channels \\
$\mu/\tau$ lifetime ratio & Connect to mode energy differences \\
Generation structure & Explain three lepton generations from geometry \\
Neutrino mixing & Connect to edge-mode overlap structure \\
\bottomrule
\end{tabular}
\end{center}

\subsection{Visual Summary: The Epistemic Landscape}

\begin{center}
\begin{tikzpicture}[scale=0.9]

% Baseline region
\fill[green!15] (-5,0) rectangle (5,1.5);
\node[font=\small\bfseries, green!50!black] at (0,1.2) {BASELINE (Established)};
\node[font=\scriptsize, align=center] at (-2.5,0.5) {Masses, lifetimes,\\branching ratios};
\node[font=\scriptsize, align=center] at (2.5,0.5) {$G_F$, $M_W$,\\SM formulas};

% Postulate region
\fill[yellow!20] (-5,1.7) rectangle (5,3.2);
\node[font=\small\bfseries, yellow!60!black] at (0,2.9) {POSTULATES (Structural Assumptions)};
\node[font=\scriptsize, align=center] at (-2.5,2.2) {Thick brane,\\bulk-core particles};
\node[font=\scriptsize, align=center] at (2.5,2.2) {Frozen projection,\\pipeline structure};

% Deduction region
\fill[blue!15] (-5,3.4) rectangle (5,4.9);
\node[font=\small\bfseries, blue!50!black] at (0,4.6) {DEDUCTIONS (Derived from Postulates)};
\node[font=\scriptsize, align=center] at (-2.5,3.9) {Channel selection,\\stability conditions};
\node[font=\scriptsize, align=center] at (2.5,3.9) {Qualitative patterns,\\threshold effects};

% Open region
\fill[red!10] (-5,5.1) rectangle (5,6.6);
\node[font=\small\bfseries, red!50!black] at (0,6.3) {OPEN (To Be Computed)};
\node[font=\scriptsize, align=center] at (-2.5,5.6) {Lifetime values,\\$G_F$ derivation};
\node[font=\scriptsize, align=center] at (2.5,5.6) {Mode spectrum,\\overlap integrals};

% Arrows showing logical flow
\draw[-{Stealth}, thick, gray] (5.5,0.75) -- (5.5,2.45);
\draw[-{Stealth}, thick, gray] (5.5,2.45) -- (5.5,4.15);
\draw[-{Stealth}, thick, gray] (5.5,4.15) -- (5.5,5.85);
\node[font=\tiny, rotate=90] at (5.9,3.3) {logical dependence};

\end{tikzpicture}
\end{center}

\subsection{What This Chapter Does and Does Not Claim}

\begin{tcolorbox}[readerContract, title={Final Epistemic Statement}]
\textbf{This chapter claims}:
\begin{itemize}[nosep]
  \item A coherent structural interpretation of weak decays in thick-brane geometry
  \item Qualitative explanations for channel selection rules
  \item A well-posed framework for quantitative computation
  \item Explicit falsifiability conditions for each claim
\end{itemize}

\textbf{This chapter does not claim}:
\begin{itemize}[nosep]
  \item First-principles derivation of lifetime values
  \item Explicit computation of branching ratios
  \item Derivation of $G_F$ from the 5D action
  \item Complete solution of the mode spectrum
\end{itemize}

The gap between ``structural interpretation'' and ``derived result'' is
substantial. Closing this gap is the research program.
\end{tcolorbox}

