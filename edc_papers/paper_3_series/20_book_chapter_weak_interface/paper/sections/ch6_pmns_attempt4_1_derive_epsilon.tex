%!TEX root = ../EDC_Part_II_Weak_Sector.tex
% ============================================================================
% PMNS Attempt 4.1: Derive epsilon from lambda
% Status: YELLOW [BL->Dc] — epsilon predicted from Wolfenstein scale
% ============================================================================

\subsection{Attempt 4.1: Deriving the Reactor Perturbation}
\label{sec:pmns_attempt4_1}

Attempt~4 achieved numerical closure for all three PMNS angles via the rank-2
construction $U = R_{23}(\theta_{23}^0) \cdot R_{13}(\varepsilon) \cdot R_{12}(\theta_{12}^0)$,
but the reactor perturbation $\varepsilon \approx 0.15$~rad had no geometric origin.
Attempt~4.1 tests whether $\varepsilon$ can be \emph{predicted} from existing EDC
quantities---specifically, the Wolfenstein scale $\lambda$ from the CKM sector.

\subsubsection{No-Smuggling Constraint}

\begin{tcolorbox}[colback=red!5!white, colframe=red!50!black, title=No-Smuggling Protocol]
\begin{itemize}[nosep]
  \item $\varepsilon$ is \textbf{predicted} from $\lambda$ (and $\kappa$ ratio), \textbf{not fit to $\theta_{13}$}
  \item We compare predicted $\theta_{13}$ to PDG \emph{after} the prediction
  \item Any remaining dependence is explicitly labeled \tagI{} or \tagCal{}
\end{itemize}
\end{tcolorbox}

\subsubsection{Candidate Mechanisms}

We adopt the Wolfenstein scale $\lambda \approx 0.225$ as defined in Eq.~(\ref{eq:ch7_wolfenstein}).
Two candidate mechanisms are tested:

\paragraph{C1: $\varepsilon = \lambda / \sqrt{2}$.}
The reactor angle is suppressed relative to the Cabibbo angle by a geometric factor
$\sqrt{2}$, which appears naturally in projections from $\mathbb{Z}_6$ to the brane.
\begin{equation}
  \varepsilon_{C1} = \frac{\lambda}{\sqrt{2}} = \frac{0.225}{\sqrt{2}} \approx 0.159~\text{rad}
  \quad (9.1°)
  \label{eq:epsilon_C1}
\end{equation}
Epistemic tag: \textbf{[BL$\to$Dc]}---$\lambda$ is \tagBL{}, the $\sqrt{2}$ factor is geometric.

\paragraph{C2: $\varepsilon = \lambda \times (\kappa_q / \kappa_\ell)$.}
The reactor angle inherits the quark-lepton localization asymmetry from Eq.~(\ref{eq:ch7_kappa_ratio}):
\begin{equation}
  \varepsilon_{C2} = \lambda \times \frac{\kappa_q}{\kappa_\ell} = 0.225 \times 0.4 = 0.090~\text{rad}
  \quad (5.2°)
  \label{eq:epsilon_C2}
\end{equation}
Epistemic tag: \textbf{[I]}---the $\kappa$ ratio is identified, not derived (OPR-10).

\subsubsection{Results}

\begin{table}[ht]
\centering
\caption{Attempt 4.1: $\varepsilon$ candidates and $\theta_{13}$ predictions}
\label{tab:pmns_attempt4_1}
\begin{tabular}{lccccl}
\toprule
\textbf{Candidate} & \textbf{$\varepsilon$ (rad)} & \textbf{$\sin^2\theta_{13}$} &
\textbf{PDG} & \textbf{Error} & \textbf{Status} \\
\midrule
C1: $\lambda/\sqrt{2}$ & $0.159$ & $0.0253$ & $0.022$ & $15\%$ &
\textcolor{YellowOrange}{\textbf{YELLOW}} \\
C2: $\lambda \times \kappa$ & $0.090$ & $0.0081$ & $0.022$ & $63\%$ &
\textcolor{red}{\textbf{RED}} \\
\addlinespace
PDG target & $0.148$ & $0.022$ & --- & --- & \tagBL{} \\
\bottomrule
\end{tabular}
\end{table}

\paragraph{Full PMNS with discrete $\theta_{12}$ (no smuggling).}
Using C1 ($\varepsilon = \lambda/\sqrt{2}$) with discrete $\theta_{12}^0$ candidates
(excluding the PDG-exact value $33.7°$):

\begin{center}
\begin{tabular}{lcccc}
\toprule
\textbf{$\theta_{12}^0$} & \textbf{$\sin^2\theta_{12}$} & \textbf{$\sin^2\theta_{23}$} &
\textbf{$\sin^2\theta_{13}$} & \textbf{Status} \\
\midrule
$30°$ & $0.250$ & $0.564$ & $0.025$ & YELLOW \\
$35°$ & $0.329$ & $0.564$ & $0.025$ & \textcolor{green!60!black}{\textbf{GREEN}} \\
$45°$ & $0.500$ & $0.564$ & $0.025$ & YELLOW \\
$54.7°$ & $0.667$ & $0.564$ & $0.025$ & YELLOW \\
\addlinespace
PDG targets & $0.307$ & $0.546$ & $0.022$ & --- \\
\bottomrule
\end{tabular}
\end{center}

\textbf{Key finding:} With $\theta_{12}^0 = 35°$ (a discrete candidate, not PDG-smuggling)
and $\varepsilon = \lambda/\sqrt{2}$, the model achieves GREEN for all three angles:
\begin{itemize}[nosep]
  \item $\sin^2\theta_{12} = 0.329$ vs.\ $0.307$ (7\% error)
  \item $\sin^2\theta_{23} = 0.564$ vs.\ $0.546$ (3\% error)
  \item $\sin^2\theta_{13} = 0.025$ vs.\ $0.022$ (15\% error)
\end{itemize}

\subsubsection{Why C2 Fails}

The $\kappa$ ratio mechanism (C2) predicts $\sin^2\theta_{13} = 0.008$, which is
$\times 2.7$ smaller than PDG. For C2 to work, the $\kappa$ ratio would need to be
$\kappa_q/\kappa_\ell \approx 0.66$ instead of $0.4$---a $65\%$ increase from the
value identified in Chapter~\ref{sec:ch7_ckm}.

\textbf{C2 is closed as RED.} While the $\kappa$ mechanism remains theoretically
interesting, with 63\% error it fails to predict the reactor scale within useful
accuracy. Further investigation of the $\kappa$ ratio is tracked separately (OPR-10)
but is no longer considered a viable $\varepsilon$ mechanism.

\subsubsection{Epistemic Assessment}

\begin{tcolorbox}[colback=yellow!5!white, colframe=yellow!60!black,
    title=Attempt 4.1 Epistemic Status]
\textbf{C1 ($\varepsilon = \lambda/\sqrt{2}$): YELLOW [BL$\to$Dc]}

\begin{itemize}[nosep]
  \item $\varepsilon$ is \emph{predicted} from $\lambda$ with geometric factor $\sqrt{2}$
  \item No new \tagI{} dependency (uses only $\lambda$ \tagBL{} + geometric transformation)
  \item Predicts $\sin^2\theta_{13} = 0.025$ (15\% from PDG)---not exact, but right scale
  \item With discrete $\theta_{12}^0 = 35°$, achieves GREEN for all three angles
\end{itemize}

\textbf{What is closed:}
\begin{itemize}[nosep]
  \item $\theta_{23}$: derived from $\mathbb{Z}_6$ geometry \tagDc{}
  \item $\varepsilon$ scale: tied to Wolfenstein $\lambda$ via $\sqrt{2}$ factor [BL$\to$Dc]
\end{itemize}

\textbf{What remains open:}
\begin{itemize}[nosep]
  \item $\theta_{12}^0$: best discrete candidate is $35°$; exact value $33.7°$ is \tagI{}
  \item $\sqrt{2}$ factor: geometric motivation clear, formal derivation open
\end{itemize}
\end{tcolorbox}

\subsubsection{Verdict}

\begin{tcolorbox}[colback=green!5!white, colframe=green!50!black,
    title=Attempt 4.1 Verdict: $\varepsilon$ Mechanism Partially Closed]
\textbf{C1 ($\varepsilon = \lambda/\sqrt{2}$): Closes reactor scale without fitting.}

The reactor perturbation $\varepsilon \approx 0.16$~rad is predicted from the CKM-derived
Wolfenstein scale $\lambda = 0.225$ \tagBL{}, without calibrating to PDG $\theta_{13}$.
With discrete $\theta_{12}^0 = 35°$, all three PMNS angles match PDG within 15\%.

\textbf{OPR-05 update:}
\begin{itemize}[nosep]
  \item $\theta_{23}$: GREEN \tagDc{}
  \item $\theta_{13}$: YELLOW [BL$\to$Dc] --- $\varepsilon = \lambda/\sqrt{2}$, 15\% error
  \item $\theta_{12}$: YELLOW \tagI{} --- $35°$ discrete or $33.7°$ identified
\end{itemize}
Overall: \textbf{YELLOW [Dc/I]} with $\varepsilon$ mechanism partially closed.

\medskip
\noindent\fbox{\parbox{0.96\textwidth}{\small
\textbf{Reactor closure (Attempt 4.1):} $\varepsilon \approx \lambda/\sqrt{2}$ uses only
Wolfenstein $\lambda$ \tagBL{} and a geometric $\sqrt{2}$, predicting
$\sin^2\theta_{13} \approx 0.025$ (15\% high vs PDG) without tuning; combined with
$\mathbb{Z}_6$ $\theta_{23}$ gives a near-complete asymmetric PMNS pattern
($\theta_{23}$ GREEN, $\theta_{13}$ YELLOW, $\theta_{12}$ identified).}}
\end{tcolorbox}

\paragraph{Code.} \texttt{code/pmns\_attempt4\_1\_derive\_epsilon.py}

\paragraph{Output.} \texttt{code/output/pmns\_attempt4\_1\_results.txt}
