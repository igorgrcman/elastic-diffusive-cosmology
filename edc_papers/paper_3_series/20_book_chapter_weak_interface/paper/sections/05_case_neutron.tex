% ==============================================================================
% Case Study I: Neutron beta-Decay
% ==============================================================================

\subsection{Neutron $\beta$-Decay: Junction Relaxation to Proton Anchor}
\label{sec:case_neutron}

The neutron is the anchor case for the EDC weak program. Its decay provides the
clearest example of bulk$\to$brane transfer because the neutron, as a bulk-core
junction, has a component that extends into the bulk.

\subsubsection{What Is the Neutron in EDC Ontology?}

\textbf{Ontology} \tagP{}/\tagDc{}: The neutron is modeled as a \emph{bulk-core
junction} configuration whose relaxation can pump energy into the thick brane.
Unlike purely brane-dominant leptonic modes (muon, tau), the neutron carries a
``bulk-facing'' component: its decay is therefore used as the anchor case because
it naturally provides a bulk$\to$brane pumping stage.

\paragraph{Baseline observable.}
Experimentally, the dominant channel is \tagBL{}:
\begin{equation}
n \to p + e^- + \bar\nu_e,
\label{eq:n_channel}
\end{equation}
with a Q-value governed by the neutron--proton mass difference:
$\Delta m_{np} = m_n - m_p \approx 1.293$ MeV \tagBL{}.

EDC does not dispute the baseline; it supplies an interface mechanism that makes
the channel structure intelligible.

\subsubsection{The Mechanistic Story: The ``Film'' of Neutron Decay}

We narrate the process in three interface stages, following the unified pipeline.

\paragraph{Stage A: Absorption (brane charging by junction relaxation).}

A junction relaxation in the bulk-core sector induces a bulk-facing pumping into
the brane layer. In the effective brane coordinate picture, we model this as a
trajectory $q(t)$ in an effective potential $V(q)$, with instantaneous pumping
power \tagDef{}:
\begin{equation}
\Pi_{\text{pump}}(t) \equiv -\dot{q}(t) \cdot \partial_q V(q(t)).
\label{eq:n_pump}
\end{equation}

\textbf{Interpretation} \tagDc{}: The quantity $\Pi_{\text{pump}}$ is not a new
force; it is the power delivered into the brane reservoir by the bulk-side
relaxation mechanism.

The accumulated energy delivered into the brane up to time $t$ is the charging
integral \tagDef{}:
\begin{equation}
E_{\text{charge}}(t) \equiv \int_0^t \Pi_{\text{pump}}(t')\,dt'.
\label{eq:n_charge}
\end{equation}

This defines, at bookkeeping level, how much energy becomes available for
subsequent mode redistribution and release.

\paragraph{Stage B: Dissipation (redistribution into brane-layer modes).}

Once energy is deposited, it need not be immediately released. A thick brane
provides internal degrees of freedom (layer modes) into which the reservoir
energy can redistribute. We represent this redistribution schematically as
\tagP{}/\tagOpen{}:
\begin{equation}
E_{\text{brane}} \;\to\; \{\phi_k\}_{\text{layer modes}}.
\label{eq:n_modes}
\end{equation}

This stage is crucial: without it, one cannot explain why the observed outputs
appear as a restricted set rather than an arbitrary energy dump.

\paragraph{Stage C: Release (observer-facing projection into 3D outputs).}

The release stage begins when the system enters a regime where pumping becomes
negligible compared to release \tagDc{}:
\begin{equation}
\Xi(t) \equiv \frac{\Pi_{\text{pump}}(t)}{\Pi_{\text{release}}(t)} \ll 1
\quad\text{at } t = t_*.
\label{eq:n_trigger}
\end{equation}

\textbf{Important nuance}: We do not claim that $\dot{q}(t_*) = 0$ exactly.
Rather, the interface becomes \emph{effectively frozen} at observational
resolution: the continuous pump term is negligible, and the release can be
treated as the dominant process.

The observer-facing outputs are defined by the frozen projection operator
\tagDef{}:
\begin{equation}
\{\text{outputs}\}_{3D} = \mathcal{P}_{\text{frozen}}\big(\{\phi_k\}\big),
\qquad
\mathcal{P}_{\text{frozen}} = \mathcal{P}_{\text{energy}} \circ
\mathcal{P}_{\text{mode}} \circ \mathcal{P}_{\text{chir}}.
\label{eq:n_projection}
\end{equation}

\subsubsection{Why the Electron Channel Is Allowed but the Muon Channel Is Not}

A common confusion is to phrase this as ``EDC forbids the muon channel.'' The
correct, book-level statement is purely kinematic and belongs to
$\mathcal{P}_{\text{energy}}$.

\paragraph{Baseline kinematic gate.}
For $\beta$-decay, the available energy budget is set by \tagBL{}:
\begin{equation}
Q_\beta(\ell) \approx \Delta m_{np} - m_\ell - m_\nu \approx \Delta m_{np} - m_\ell,
\label{eq:Q_beta}
\end{equation}
where neutrino masses are negligible at this scale.

\paragraph{Electron channel.}
For $\ell = e$ one has $m_e \approx 0.511$ MeV \tagBL{}, hence $Q_\beta(e) > 0$,
so phase space exists and the channel is kinematically open:
\begin{equation}
Q_\beta(e) \approx 1.293 - 0.511 = 0.782~\text{MeV} > 0.
\end{equation}

\paragraph{Muon channel.}
For $\ell = \mu$ one has $m_\mu \approx 105.7$ MeV \tagBL{}, hence
$Q_\beta(\mu) < 0$:
\begin{equation}
Q_\beta(\mu) \approx 1.293 - 105.7 \approx -104.4~\text{MeV} < 0,
\end{equation}
meaning there is \emph{no kinematically allowed phase space} for
$n \to p + \mu^- + \bar\nu_\mu$ at rest.

\begin{tcolorbox}[guardrail, title={Q-Gate Selection Rule}]
The ``muon channel'' is rejected not by metaphysical prohibition, but because
$\mathcal{P}_{\text{energy}}$ yields zero support:
\begin{equation}
\mathcal{P}_{\text{energy}}: \quad
\text{channel allowed} \iff Q_\beta(\ell) > 0.
\label{eq:Penergy_gate}
\end{equation}
This is a kinematic fact \tagBL{}, not an EDC-specific assumption.
\end{tcolorbox}

\paragraph{EDC interpretation.}
This is exactly what we want from the pipeline language: one can separate
\emph{what is purely kinematic} (baseline gating) from \emph{what is mechanistic}
(how the brane actually processes and projects the allowed energy into specific
outputs). In neutron decay, $\mathcal{P}_{\text{energy}}$ restricts us to the
electron sector; the remaining question is then: \emph{given that the electron
channel is open, what interface mechanism produces the observed
$\{e^-, \bar\nu\}$ outputs?}

\subsubsection{Ledger Closure for Neutron Decay}

The neutron case forces discipline on bookkeeping. The brane reservoir must close
its ledger: the energy deposited by junction relaxation must appear as observable
kinetic energies plus any additional channels consistent with conservation.

We write a schematic ledger identity \tagDef{}/\tagDc{}:
\begin{equation}
\Delta E_{\text{available}} = K_p + K_e + K_{\bar\nu} + E_{\text{other}},
\label{eq:n_ledger}
\end{equation}
where:
\begin{itemize}[nosep]
  \item $\Delta E_{\text{available}}$ is the energy budget ($\sim Q_\beta$),
  \item $K_p, K_e, K_{\bar\nu}$ are the kinetic energies of the outputs,
  \item $E_{\text{other}} = E_{\text{recoil}} + E_{\text{soft}} + E_{\text{bulk,res}}$
        collects subleading channels \tagDef{}/\tagOpen{}.
\end{itemize}

The point is not to claim a numerical partition yet; the point is to enforce
that the mechanism \emph{must} account for where energy goes, and to provide
named bins that future closure must fill.

\subsubsection{Process Diagram: Neutron Decay}

\begin{center}
\begin{tikzpicture}[
  scale=0.85,
  box/.style={rectangle, rounded corners=4pt, minimum width=2cm, minimum height=0.8cm,
              draw=black, thick, font=\footnotesize, align=center},
  gate/.style={rectangle, rounded corners=2pt, minimum width=1.5cm, minimum height=0.6cm,
               draw=red!60!black, thick, fill=red!10, font=\scriptsize, align=center},
  arrow/.style={-{Stealth[length=5pt]}, thick},
  label/.style={font=\scriptsize\itshape, text=gray!60!black}
]

% Stage boxes
\node[box, fill=gray!20] (junction) at (0,0) {Junction\\relaxation};
\node[box, fill=red!15] (pump) at (3,0) {$\Pi_{\text{pump}}$\\absorption};
\node[box, fill=yellow!20] (modes) at (6,0) {$\{\phi_k\}$\\dissipation};
\node[box, fill=green!15] (project) at (9,0) {$\mathcal{P}_{\text{frozen}}$\\release};

% Q-gate
\node[gate] (qgate) at (6,-1.8) {Q-gate\\$Q_\beta(e)>0$\\$Q_\beta(\mu)<0$};

% Outputs
\node[box, fill=blue!15] (p) at (12,0.8) {$p$};
\node[box, fill=blue!15] (e) at (12,0) {$e^-$};
\node[box, fill=blue!15] (nu) at (12,-0.8) {$\bar\nu_e$};

% Arrows
\draw[arrow] (junction) -- (pump);
\draw[arrow] (pump) -- (modes);
\draw[arrow] (modes) -- (project);
\draw[arrow] (project) -- (p);
\draw[arrow] (project) -- (e);
\draw[arrow] (project) -- (nu);
\draw[arrow, dashed, red!50] (modes) -- (qgate);
\draw[arrow, dashed, red!50] (qgate) -- (project);

% Labels
\node[label, above] at (1.5,0.3) {bulk$\to$brane};
\node[label, above] at (4.5,0.3) {redistribute};
\node[label, above] at (7.5,0.3) {filter};
\node[label, above] at (10.5,0.5) {project};

% Ledger box
\node[rectangle, draw=gray, rounded corners=3pt, fill=gray!5,
      font=\scriptsize, align=left, text width=2.5cm] at (14.5,0)
  {Ledger:\\$\Delta E = K_p + K_e$\\$\phantom{\Delta E =} + K_{\bar\nu} + E_{\text{other}}$};

\end{tikzpicture}
\end{center}

\subsubsection{Falsifiability Hooks}

\begin{tcolorbox}[falsifiability]
The neutron mechanistic story can be wrong. The most direct falsifiability hooks are:
\begin{itemize}[nosep]
  \item If the mechanism predicts additional leading-order outputs beyond
        $\{p, e^-, \bar\nu_e\}$ in the neutron Q-window, it fails.
  \item If $\mathcal{P}_{\text{energy}}$ gating is not respected (i.e., if the
        model leaks into the $\mu$ channel without external energy), it fails.
  \item If the ledger cannot be closed without hidden tuning (i.e., energy
        ``disappears'' without accounted bins), it fails.
  \item If the trigger condition requires an ad hoc fitted parameter rather
        than a regime statement, it fails.
\end{itemize}
\end{tcolorbox}
