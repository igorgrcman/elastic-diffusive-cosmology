% ==============================================================================
% Section 0: Reader Contract
% ==============================================================================

\begin{tcolorbox}[readerContract]
This chapter develops a \textbf{mechanistic brane-interface framework} for weak-sector
phenomena in Elastic Diffusive Cosmology (EDC). Before proceeding, the reader should
understand what this chapter delivers and what it does not attempt.

\textbf{What this chapter provides:}
\begin{itemize}[nosep]
  \item A unified mechanistic pipeline for weak decays: Absorption $\to$ Dissipation $\to$ Release
  \item An ontological classification of particles in 5D terms
  \item Explicit falsifiability criteria for each claim
  \item A structural pathway to the Fermi constant (without numerical closure)
\end{itemize}

\textbf{What this chapter does NOT provide:}
\begin{itemize}[nosep]
  \item A derivation of particle masses from first principles
  \item Numerical predictions of lifetimes or branching ratios
  \item A replacement for Standard Model calculations
  \item Parameters tuned to fit experimental data
\end{itemize}

This chapter follows a strict no-fit policy and has been audited for dimensional
consistency, epistemic tagging, and non-overclaim language.
\end{tcolorbox}

\subsection{Reader's Guide: What Is Baseline, What Is Mechanism, What Is Open}

This chapter mixes three layers of content: (i) baseline experimental facts,
(ii) EDC mechanistic interpretations, and (iii) open derivation targets.
We tag statements accordingly:

\begin{itemize}[nosep]
  \item \textbf{[BL] Baseline:} Measured constants, established decay channels,
        lifetimes, and kinematic thresholds.
  \item \textbf{[Def] Definition:} Bookkeeping definitions (e.g.,
        $\mathcal{P}_{\text{frozen}}$, $\Pi_{\text{pump}}$, $\Pi_{\text{release}}$).
  \item \textbf{[Dc] Derived/Constrained:} Consequences that follow once definitions
        and baseline inputs are adopted (e.g., kinematic closure; ledger consistency
        checks).
  \item \textbf{[P] Postulate/Hypothesis:} Ontology assignments (brane-dominant defect,
        composite junction-pair, edge mode) and boundary-mechanism interpretations
        (chirality as BC projection).
  \item \textbf{[OPEN] Open target:} Items that require explicit 5D construction or
        numerical evaluation (mode profiles, KK spectrum, BC operator evaluation,
        mass origins, oscillations).
\end{itemize}

The goal is not to ``re-derive the Standard Model''; the goal is to present a
coherent interface mechanism that can be progressively closed by explicit computations.

\subsection{Epistemic Conventions (Table)}

Throughout this chapter, every claim is tagged with its epistemic status:

\begin{table}[ht]
\centering
\begin{tabular}{lll}
\toprule
\textbf{Tag} & \textbf{Status} & \textbf{Meaning} \\
\midrule
\tagBL{} & Baseline & Experimental value (PDG, CODATA) \\
\tagDef{} & Definition & Mathematical/operational definition \\
\tagP{} & Postulate & Hypothesis or proposed mechanism \\
\tagDc{} & Deduction & Follows from postulates via explicit steps \\
\tagOpen{} & Open & Requires future work for closure \\
\bottomrule
\end{tabular}
\caption{Epistemic tagging conventions used throughout this chapter.}
\label{tab:epistemic_tags}
\end{table}

\subsection{Three Core Terms}

The following three terms appear repeatedly and must be understood unambiguously:

\begin{enumerate}
  \item \textbf{Absorption} \tagDef{}: The process by which bulk-facing dynamics pump
        energy into the thick-brane layer. This is not ``particle creation'' but
        reservoir charging.

  \item \textbf{Dissipation} \tagDef{}: The redistribution of absorbed energy among
        brane-layer modes. This stage determines which internal degrees of freedom
        carry the energy before any 3D projection.

  \item \textbf{Release} \tagDef{}: The observer-facing projection that converts
        brane-layer excitations into 3D observable outputs. This stage is governed
        by the frozen projection operator $\mathcal{P}_{\text{frozen}}$.
\end{enumerate}

\subsection{The Mechanistic Dimension Principle}

\begin{tcolorbox}[mechanism, title={Core Interpretive Principle}]
\textbf{Mechanistic Dimension Principle} \tagDef{}:
In EDC, an ``extra dimension'' need not be interpreted merely as an additional spatial
length. It may instead encode a \emph{mechanism}---an additional degree of freedom whose
physical role is expressed through boundary conditions, coupling structure, and the
existence of distinct interface regimes.

The \textbf{brane} is not a decorative geometric hypersurface but the \emph{manifestation
of an interface mechanism} between bulk dynamics (Plenum-facing) and observer-facing 3D
outputs. The causal structure ``bulk $\to$ brane $\to$ 3D'' is not merely a picture; it
is the proposed mechanistic skeleton that this chapter formalizes.
\end{tcolorbox}

This principle has immediate consequences:
\begin{itemize}
  \item Weak interactions are not ``fundamental vertices'' but interface transfer residues.
  \item Selection rules emerge from boundary conditions, not from inserted symmetries.
  \item The ``weakness'' of weak interactions reflects geometric suppression (overlap
        integrals, mediator gaps), not a small fundamental coupling.
\end{itemize}

\subsection{What Remains Open}

The following are explicitly flagged as open problems requiring future work \tagOpen{}:

\begin{itemize}[nosep]
  \item Numerical derivation of $\tau_n$, $\tau_\mu$, $\tau_\tau$ from 5D parameters
  \item First-principles calculation of lepton mass ratios
  \item Derivation of $G_F$ magnitude (structural pathway exists; numerical closure pending)
  \item Helicity suppression factor from explicit boundary-condition computation
  \item Mode profile solutions in the thick-brane potential
\end{itemize}

These open problems define the research program's completion targets; they do not
invalidate the structural framework.
