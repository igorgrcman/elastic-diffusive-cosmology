% ==============================================================================
% Section 3: Unified Pipeline
% ==============================================================================

This section formalizes the Absorption $\to$ Dissipation $\to$ Release pipeline
that governs all weak-sector processes in EDC.

\subsection{Pipeline Overview: What Happens Physically}

We propose that weak-sector decays in EDC share a common mechanistic skeleton
\tagP{}/\tagDc{}:

\begin{center}
\fbox{\textbf{Absorption}} $\longrightarrow$
\fbox{\textbf{Dissipation}} $\longrightarrow$
\fbox{\textbf{Release}}
\end{center}

\noindent
The central claim is not that all decays have identical microphysics, but that
the \emph{interface logic} is shared: energy arrives from a bulk-facing process,
is processed in a thick-brane layer, and is then projected through boundary
conditions into an allowed set of 3D observable outputs.

\paragraph{Absorption (brane charging).}
Bulk-facing dynamics (e.g., junction relaxation, mode de-excitation) pump energy
into the brane layer. The brane acts as a reservoir that accumulates energy before
any 3D output is produced.

\paragraph{Dissipation (mode redistribution).}
The accumulated energy does not remain in its initial form. It redistributes among
the available brane-layer modes $\{\phi_k\}$. This stage is crucial: without it,
one cannot explain why the observed outputs appear as a restricted set rather than
an arbitrary energy dump.

\paragraph{Release (observer projection).}
The observer-facing boundary condition projects the brane-layer modes into 3D
outputs. This projection is \emph{not} the identity: it filters modes according
to kinematic, topological, and chirality constraints.

\subsection{Energy-Flow Bookkeeping}

We describe the brane layer as an intermediate reservoir carrying an energy
content $E_{\text{brane}}(t)$ \tagDef{}. Energy conservation at the level of
the reservoir is captured by:
\begin{equation}
\frac{dE_{\text{brane}}}{dt} \;=\; \Pi_{\text{pump}}(t) \;-\; \Pi_{\text{release}}(t)
\;-\; \Pi_{\text{other}}(t),
\label{eq:brane_energy_balance}
\end{equation}
where:
\begin{itemize}[nosep]
  \item $\Pi_{\text{pump}}$ is the bulk$\to$brane pumping power \tagDef{},
  \item $\Pi_{\text{release}}$ is the brane$\to$3D release power \tagDef{},
  \item $\Pi_{\text{other}}$ captures additional channels (recoil, soft emission,
        bulk residual) \tagDef{}/\tagOpen{}.
\end{itemize}

\paragraph{Dimensional check.}
All $\Pi$ quantities have dimensions of \textbf{energy/time} (power). The energy
balance equation is dimensionally consistent: $[E]/[t] = [E/t]$.

\subsection{Pumping Power: A Practical Model}

In the effective 1D brane-coordinate description, the pumping power is represented
as \tagDef{}:
\begin{equation}
\Pi_{\text{pump}}(t) \;\equiv\; -\dot{q}(t) \cdot \partial_q V(q(t)),
\label{eq:pump_power_def}
\end{equation}
where $q(t)$ is an effective collective coordinate (e.g., junction position) and
$V(q)$ is an effective potential. This has units of energy/time and corresponds
to the instantaneous power associated with motion along $V(q)$.

\paragraph{Physical interpretation.}
As a bulk-facing configuration relaxes toward a minimum of $V(q)$, it converts
potential energy into kinetic energy, which then pumps into the brane layer.
The pumping ceases when the system reaches the minimum ($\dot{q} \to 0$) or when
$\partial_q V \to 0$.

\subsection{Regime Parameter and Trigger Condition}

A useful dimensionless discriminator between ``still being pumped'' and
``effectively releasing'' is \tagDef{}:
\begin{equation}
\Xi(t) \;\equiv\; \frac{\Pi_{\text{pump}}(t)}{\Pi_{\text{release}}(t)}.
\label{eq:Xi_def}
\end{equation}

\paragraph{Regime interpretation.}
\begin{itemize}
  \item $\Xi \gg 1$: Pumping-dominated regime. Energy accumulates in the brane.
  \item $\Xi \sim 1$: Transition regime. Pumping and release are comparable.
  \item $\Xi \ll 1$: Release-dominated regime. The brane empties into 3D outputs.
\end{itemize}

The \textbf{freeze/release trigger} is the transition to the release-dominated
regime \tagDc{}:
\begin{equation}
t = t_*: \qquad \Xi(t_*) \ll 1.
\label{eq:trigger_condition}
\end{equation}

\paragraph{Important nuance.}
We do not claim that $\dot{q}(t_*) = 0$ exactly. Rather, the interface becomes
\emph{effectively frozen} at observational resolution: the continuous pump term
is negligible, and the release can be treated as the dominant process. This is
a \emph{regime statement}, not an exact dynamical endpoint.

\subsection{The Frozen Projection Operator}

The key conceptual move is that the observer does not ``see'' raw 5D fields.
Instead, 3D outputs are those components that survive the observer-facing
projection. We write \tagDef{}:
\begin{equation}
\mathcal{P}_{\text{frozen}} \;=\;
\mathcal{P}_{\text{energy}} \circ \mathcal{P}_{\text{mode}} \circ \mathcal{P}_{\text{chir}},
\label{eq:Pfrozen_def}
\end{equation}
where:

\paragraph{$\mathcal{P}_{\text{energy}}$: Kinematic gate.}
Enforces kinematic admissibility: only channels with positive Q-value and
available phase space are allowed \tagDef{}/\tagBL{}. This is purely kinematic
and does not require EDC-specific assumptions.

\paragraph{$\mathcal{P}_{\text{mode}}$: Mode-to-output mapping.}
Maps brane-layer excitations to allowed output species channels. Which internal
modes can produce which particles is determined by the mode structure of the
thick brane \tagP{}/\tagOpen{}.

\paragraph{$\mathcal{P}_{\text{chir}}$: Chirality filter.}
Encodes chirality/helicity selection as a boundary condition effect. The V$-$A
structure of weak interactions emerges from the geometry of the observer-facing
boundary \tagP{}/\tagOpen{}.

\subsection{Output Definition}

The 3D output set is defined at the level of the pipeline as \tagDef{}:
\begin{equation}
\{\text{outputs}\}_{3D} \;\equiv\;
\mathcal{P}_{\text{frozen}}\big(\{\phi_k\}_{\text{brane modes}}\big).
\label{eq:outputs_def}
\end{equation}

\paragraph{Interpretation.}
This formulation makes explicit where ``weak selection rules'' live in EDC:
they are not inserted as vertices but emerge as an interface phenomenon governed
by reservoir dynamics, mode structure, and observer-facing projection.

\subsection{Ledger Closure Requirement}

For any process, the energy ledger must close \tagDef{}:
\begin{equation}
\Delta E_{\text{available}} \;=\;
\sum_{\text{outputs}} K_i \;+\; E_{\text{other}},
\label{eq:ledger_closure}
\end{equation}
where $K_i$ are the kinetic energies of 3D outputs and $E_{\text{other}}$
captures subleading channels (recoil, soft modes, bulk residual).

This is not a derived result but a \emph{consistency requirement}: any mechanism
that fails to close the ledger is incomplete or incorrect.

\subsection{Pipeline Summary Diagram}

The unified pipeline can be represented schematically as:

\begin{center}
\begin{tikzpicture}[
  scale=0.85,
  box/.style={rectangle, rounded corners=5pt, minimum width=2.2cm, minimum height=1cm,
              draw=black, thick, font=\small\bfseries, align=center},
  arrow/.style={-{Stealth[length=6pt]}, thick},
  label/.style={font=\footnotesize\itshape, text=gray!70!black}
]
% Boxes
\node[box, fill=gray!20] (bulk) at (0,0) {Bulk\\Trigger};
\node[box, fill=red!15] (abs) at (3.5,0) {Absorption\\$\Pi_{\text{pump}}$};
\node[box, fill=yellow!20] (dis) at (7,0) {Dissipation\\$\{\phi_k\}$};
\node[box, fill=green!15] (rel) at (10.5,0) {Release\\$\mathcal{P}_{\text{frozen}}$};
\node[box, fill=blue!15] (out) at (14,0) {3D\\Outputs};

% Arrows
\draw[arrow] (bulk) -- (abs);
\draw[arrow] (abs) -- (dis);
\draw[arrow] (dis) -- (rel);
\draw[arrow] (rel) -- (out);

% Labels
\node[label, above] at (1.75,0.3) {energy};
\node[label, above] at (5.25,0.3) {modes};
\node[label, above] at (8.75,0.3) {filter};
\node[label, above] at (12.25,0.3) {project};

% Regime annotation
\node[label, below] at (3.5,-0.8) {$\Xi \gg 1$};
\node[label, below] at (7,-0.8) {$\Xi \sim 1$};
\node[label, below] at (10.5,-0.8) {$\Xi \ll 1$};
\end{tikzpicture}
\end{center}

This diagram applies to all weak processes considered in this chapter; the case
studies will fill in the specific triggers, modes, and outputs for each particle.
