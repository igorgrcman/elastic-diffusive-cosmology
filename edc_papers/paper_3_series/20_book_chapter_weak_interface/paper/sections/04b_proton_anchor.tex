% ==============================================================================
% Section: Proton as a Topological Anchor of the Brane--Observer Interface
% ==============================================================================

\subsection{Proton as a Topological Anchor of the Brane--Observer Interface}
\label{sec:proton_anchor}

\subsubsection{Statement (Postulate) and Consequence}

\begin{edcPostulateBox}{Proton-Anchor Stability Principle}{[P]/[Dc]}
\textbf{Postulate [P].} Our universe is stable because the proton Y-junction configuration represents
a \emph{local minimum} of an appropriate 5D energy functional under the thick-brane boundary conditions.
In EDC, the proton is not merely ``the lightest baryon''; it is a \emph{topological anchor} that stabilizes
the brane--observer interface.

\medskip
\textbf{Consequence [Dc].} If the proton were not (meta)stable as an anchored junction state,
baryonic matter would not persist, and the conditions required for complex chemistry and observers
would not be robust over macroscopic timescales.
\end{edcPostulateBox}

\subsubsection{Minimal Formal Setup: Energy Functional and Configuration Space}

We model baryonic candidates as defect configurations of the thick-brane microstructure.
Let $\mathcal{C}$ denote the space of admissible configurations (fields, embeddings, junction networks, and
their boundary data) consistent with the EDC interface conditions. We assume an effective 5D energy functional
\begin{equation}
\label{eq:5d_energy_functional}
\mathcal{E}[\Psi] \;=\; \mathcal{E}_{\mathrm{bulk}}[\Psi] \;+\; \mathcal{E}_{\mathrm{brane}}[\Psi] \;+\; \mathcal{E}_{\mathrm{BC}}[\Psi],
\end{equation}
where $\Psi\in\mathcal{C}$ encodes the relevant degrees of freedom (bulk/plenum variables, brane-layer modes,
and interface constraints). The precise microscopic form is left \textbf{OPEN} (see \S\ref{sec:proton_open_targets});
here we only require that $\mathcal{E}$ be well-defined on $\mathcal{C}$ and admits stationary points.

\subsubsection{Topological Classes and the Junction Number}

The key point is that not all deformations of a defect network are equivalent: there exist topological classes
(e.g.\ homotopy classes) that cannot be continuously deformed into one another without crossing a high-energy barrier
or violating boundary constraints. We therefore partition $\mathcal{C}$ into disjoint sectors,
\begin{equation}
\label{eq:topological_sectors}
\mathcal{C} \;=\; \bigsqcup_{\alpha\in\mathcal{I}} \mathcal{C}_\alpha,
\end{equation}
where the index $\alpha$ labels a conserved topological invariant (``junction charge'', ``winding'', or an equivalent
classification appropriate to the EDC microphysics).

\begin{edcDefinitionBox}{Y-junction sector}{[Def]}
We define the \emph{Y-junction sector} $\mathcal{C}_{Y}$ as the class of configurations whose defect network has one
trivalent junction with three legs (``arms'') satisfying the brane-interface boundary conditions on the observer-facing side.
\end{edcDefinitionBox}

\subsubsection{Local Minimality as a Stability Criterion}

Stability in this context means: within the same topological sector, small admissible perturbations cannot reduce the energy.
Formally, for a candidate configuration $\Psi_\star\in\mathcal{C}_{Y}$:

\begin{edcDefinitionBox}{Local minimum}{[Def]}
$\Psi_\star$ is a \emph{local minimum} of $\mathcal{E}$ on $\mathcal{C}_Y$ if there exists $\varepsilon>0$ such that
for all $\Psi\in\mathcal{C}_Y$ with $\|\Psi-\Psi_\star\|<\varepsilon$, one has
$\mathcal{E}[\Psi]\ge \mathcal{E}[\Psi_\star]$.
\end{edcDefinitionBox}

\subsubsection{Proposition and Proof Sketch}

\begin{edcPropositionBox}{Proton as a locally minimizing Y-junction}{[P]/[Dc]}
Assume (i) the Y-junction sector $\mathcal{C}_Y$ is topologically separated from the trivial sector $\mathcal{C}_0$ by an
energy barrier (no continuous unwinding under the BCs), and (ii) within $\mathcal{C}_Y$ the functional $\mathcal{E}$
admits a stationary configuration $\Psi_p$ whose second variation is positive for all admissible perturbations.
Then $\Psi_p$ is a local minimum of $\mathcal{E}$ in $\mathcal{C}_Y$ and represents a metastable anchored state
(the proton), providing a robust brane--observer stabilizer.
\end{edcPropositionBox}

\begin{proof}[Proof sketch (mechanistic/topological)]
\textbf{Step 1 (stationarity).} Solve $\delta\mathcal{E}[\Psi]=0$ within $\mathcal{C}_Y$ under the boundary conditions.
This yields a candidate junction configuration $\Psi_p$.

\textbf{Step 2 (topological protection).} Because $\mathcal{C}_Y$ and $\mathcal{C}_0$ are disjoint sectors,
any path from $\Psi_p$ to a trivial/no-junction state must cross configurations that violate the BCs or incur a large
energy cost. This prevents ``unwinding'' by small perturbations.

\textbf{Step 3 (local minimality).} Evaluate the second variation $\delta^2\mathcal{E}[\Psi_p;\eta]$ for admissible
perturbations $\eta$ that preserve the Y-sector constraints. If $\delta^2\mathcal{E}[\Psi_p;\eta] > 0$ for all such $\eta$,
then $\Psi_p$ is a strict local minimum within $\mathcal{C}_Y$.

\textbf{Step 4 (stability consequence).} A locally minimizing, topologically protected junction state persists against
small disturbances and acts as an anchor for the observer-facing interface; without such an anchor, baryonic composites
would not be robust. This is the mechanism-level sense in which proton stability underwrites chemistry and observers.
\end{proof}

\subsubsection{Forward Reference: The $\mathbb{Z}_6$ Program (Chapter 2)}

The proof sketch above outlines \emph{what} must be true for proton stability. In \textbf{Chapter~2}
(``The $\mathbb{Z}_6$ Program''), we provide the complete geometric derivation of \emph{why} this is true.
Specifically, Chapter~2 establishes:

\begin{itemize}[nosep]
  \item The proton Y-junction emerges as a $\mathbb{Z}_3$ fixed point of the hexagonal lattice symmetry
  \item The 120° Steiner angles are geometrically \emph{inevitable} from $\mathbb{Z}_6$-invariant boundary conditions
  \item The positive Hessian (local minimum) follows from the crystallographic structure
  \item The neutron is identified as a \emph{dislocation} in this lattice, explaining its instability
  \item Color confinement emerges from $\mathbb{Z}_3$ charge conservation
\end{itemize}

\noindent
The derivation chain in Chapter~2 transforms the [P] status of this section into [Dc] (derived consequence).
This chapter presents the \emph{physics and mechanism}; Chapter~2 provides the \emph{mathematical proof}.

\subsubsection{Connection to the Continuum of 4D Interfaces}

Among the continuum of possible 4D interfaces embedded in the 5D bulk, only those admitting a stable
topological anchor plus ledger closure yield observer-robust worlds. The proton Y-junction is one
concrete stabilizer that makes \emph{our} interface long-lived.

This connects to the broader EDC picture:
\begin{itemize}[nosep]
  \item 5D contains a continuum of 4D submanifolds (different ``interface'' choices)
  \item A viability filter selects which interfaces can be stable
  \item The proton anchor is the baryonic component of this stability
  \item The electron/neutrino pair (see \S\ref{sec:generative_closure_principle}) provides the leptonic component
\end{itemize}

\subsubsection{What Must Be Explicitly Closed (OPEN Targets)}
\label{sec:proton_open_targets}

This section is \emph{mechanism-complete} but not yet \emph{numerically closed}. The following closures remain OPEN:

\begin{itemize}[nosep]
\item \textbf{OPEN-Pa1:} Specify the concrete microphysical degrees of freedom $\Psi$ used in $\mathcal{E}[\Psi]$.
\item \textbf{OPEN-Pa2:} Derive the explicit boundary conditions at the brane--observer interface that define $\mathcal{C}_Y$.
\item \textbf{OPEN-Pa3:} Construct the topological invariant $\alpha$ (junction charge/winding) that partitions $\mathcal{C}$.
\item \textbf{OPEN-Pa4:} Compute (analytically or numerically) $\delta^2\mathcal{E}$ around $\Psi_p$ to verify positivity.
\item \textbf{OPEN-Pa5:} Estimate the barrier height between $\mathcal{C}_Y$ and $\mathcal{C}_0$ (metastability timescale).
\end{itemize}

\subsubsection{Falsifiability Hooks}

\begin{tcolorbox}[falsifiability]
\begin{itemize}[nosep]
  \item If proton decay is observed at rates inconsistent with a topologically protected minimum, the
        anchor mechanism fails.
  \item If the Y-junction configuration cannot be realized as a stationary point of any reasonable
        5D energy functional, the structural claim is falsified.
  \item If the second variation $\delta^2\mathcal{E}$ has negative eigenvalues (unstable directions),
        the local-minimum claim fails.
  \item If baryonic matter can be destabilized by small perturbations without violating conservation
        laws, the topological protection is illusory.
\end{itemize}
\end{tcolorbox}

