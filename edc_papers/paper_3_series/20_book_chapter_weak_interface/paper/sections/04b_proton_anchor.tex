% ==============================================================================
% Section: Proton as a Topological Anchor of the Brane--Observer Interface
% ==============================================================================

\subsection{Proton as a Topological Anchor of the Brane--Observer Interface}
\label{sec:proton_anchor}

\subsubsection{Statement (Postulate) and Consequence}

\begin{edcPostulateBox}{Proton-Anchor Stability Principle}{[P]/[Dc]}
\textbf{Postulate [P].} Our universe is stable because the proton Y-junction configuration represents
a \emph{local minimum} of an appropriate 5D energy functional under the thick-brane boundary conditions.
In EDC, the proton is not merely ``the lightest baryon''; it is a \emph{topological anchor} that stabilizes
the brane--observer interface.

\medskip
\textbf{Consequence [Dc].} If the proton were not (meta)stable as an anchored junction state,
baryonic matter would not persist, and the conditions required for complex chemistry and observers
would not be robust over macroscopic timescales.
\end{edcPostulateBox}

\subsubsection{Functional Role: The Proton as Energy Ledger Benchmark}

In the weak-sector pipeline, the proton serves as the \textbf{benchmark} for energy accounting.
When a neutron decays, the proton is the ``ground state'' endpoint---the stable configuration
to which the system relaxes. Without a stable proton, the entire ledger-closure mechanism
would lack a fixed reference point.

We assume the existence of an effective 5D energy functional \tagP{}:
\begin{equation}
\label{eq:5d_energy_functional}
\mathcal{E}[\Psi] \;=\; \mathcal{E}_{\mathrm{bulk}}[\Psi] \;+\; \mathcal{E}_{\mathrm{brane}}[\Psi] \;+\; \mathcal{E}_{\mathrm{BC}}[\Psi],
\end{equation}
whose configuration space $\mathcal{C}$ partitions into topologically distinct sectors.
The proton occupies a sector $\mathcal{C}_Y$ (Y-junction configurations) that is separated
from the trivial sector by an energy barrier.

\subsubsection{Core Claim (Proven in Chapter 2)}

\begin{edcPropositionBox}{Proton as a locally minimizing Y-junction}{[P] $\to$ [Dc] in Ch2}
The proton Y-junction configuration $\Psi_p$ is a \emph{local minimum} of $\mathcal{E}$
within its topological sector $\mathcal{C}_Y$, with positive-definite second variation.
This makes it a metastable anchored state that stabilizes the brane--observer interface.
\end{edcPropositionBox}

\noindent
\textbf{Note:} The full proof of this proposition is given in \textbf{Chapter~2} (The $\mathbb{Z}_6$ Program),
where we show that the proton emerges as a $\mathbb{Z}_3$ fixed point of the hexagonal lattice
symmetry with geometrically inevitable 120° Steiner angles. Here in Chapter~1, we take this
result as input and focus on its \emph{functional consequences} for the weak-decay pipeline.

\subsubsection{Forward Reference: The $\mathbb{Z}_6$ Program (Chapter 2)}

The proposition above states \emph{what} must be true for proton stability. In \textbf{Chapter~2}
(``The $\mathbb{Z}_6$ Program''), we provide the complete geometric derivation of \emph{why} this is true.
Specifically, Chapter~2 establishes:

\begin{itemize}[nosep]
  \item The proton Y-junction emerges as a $\mathbb{Z}_3$ fixed point of the hexagonal lattice symmetry
  \item The 120° Steiner angles are geometrically \emph{inevitable} from $\mathbb{Z}_6$-invariant boundary conditions
  \item The positive Hessian (local minimum) follows from the crystallographic structure
  \item The neutron is identified as a \emph{dislocation} in this lattice, explaining its instability
  \item Color confinement emerges from $\mathbb{Z}_3$ charge conservation
\end{itemize}

\noindent
The derivation chain in Chapter~2 transforms the [P] status of this section into [Dc] (derived consequence).
This chapter presents the \emph{physics and mechanism}; Chapter~2 provides the \emph{mathematical proof}.

\subsubsection{Connection to the Continuum of 4D Interfaces}

Among the continuum of possible 4D interfaces embedded in the 5D bulk, only those admitting a stable
topological anchor plus ledger closure yield observer-robust worlds. The proton Y-junction is one
concrete stabilizer that makes \emph{our} interface long-lived.

This connects to the broader EDC picture:
\begin{itemize}[nosep]
  \item 5D contains a continuum of 4D submanifolds (different ``interface'' choices)
  \item A viability filter selects which interfaces can be stable
  \item The proton anchor is the baryonic component of this stability
  \item The electron/neutrino pair (see \S\ref{sec:generative_closure_principle}) provides the leptonic component
\end{itemize}

\subsubsection{Status of Proton-Related Claims}
\label{sec:proton_open_targets}

\begin{center}
\begin{tabular}{lll}
\toprule
\textbf{Claim} & \textbf{Status} & \textbf{Reference} \\
\midrule
Proton is Y-junction minimum & [Dc] & Ch2, Thm 5.1 \\
120° Steiner angles & [Dc] & Ch2, Cor 3.1 \\
$\mathbb{Z}_3$ fixed point & [Dc] & Ch2, Prop 5.1 \\
Positive Hessian & [Dc] & Ch2, Thm 5.1 \\
\midrule
Explicit $\mathcal{E}[\Psi]$ form & [OPEN] & RT-CH3-007 \\
Barrier height (metastability) & [OPEN] & Future work \\
\bottomrule
\end{tabular}
\end{center}

\noindent
The core stability claims are now \textbf{derived} in Chapter~2 via the $\mathbb{Z}_6$ Program.
What remains open is the explicit microscopic form of the energy functional.

\subsubsection{Falsifiability Hooks}

\begin{tcolorbox}[falsifiability]
\begin{itemize}[nosep]
  \item If proton decay is observed at rates inconsistent with a topologically protected minimum, the
        anchor mechanism fails.
  \item If the Y-junction configuration cannot be realized as a stationary point of any reasonable
        5D energy functional, the structural claim is falsified.
  \item If the second variation $\delta^2\mathcal{E}$ has negative eigenvalues (unstable directions),
        the local-minimum claim fails.
  \item If baryonic matter can be destabilized by small perturbations without violating conservation
        laws, the topological protection is illusory.
\end{itemize}
\end{tcolorbox}

