% ==============================================================================
% Case Study IV: Pion Decay as the Hadron-to-Lepton Bridge
% ==============================================================================

\subsection{Pion Decay: The Hadron$\to$Lepton Bridge}
\label{sec:pion_bridge}

The pion is the first place where the EDC weak narrative must confront compositeness.
A pion is not a fundamental lepton-like excitation; it is a composite configuration.
Therefore the goal here is \emph{not} to ``derive'' $m_\pi$ or fit lifetimes, but to define a consistent
ontology and to show how the brane-interface projection can remain compatible with the
observed helicity suppression structure.

\subsubsection{Ontology (Composite Junction-Pair Hypothesis)}

\begin{edcDefinitionBox}{Pion as a composite junction-pair excitation}{[Def]/[P]}
We treat the pion as a bound pair configuration (junction-pair) in the brane/boundary system,
distinct from single-mode leptons and from bulk-core neutron excitations.
This is a postulate pending micro-ontology validation \tagOpen{}.
\end{edcDefinitionBox}

This is distinct from:
\begin{itemize}[nosep]
  \item Leptons ($e$, $\mu$, $\tau$): single brane defects (brane-dominant)
  \item Neutron: bulk-core junction with proton anchor endpoint
\end{itemize}

\paragraph{Baseline observables.}
The dominant decay channels are \tagBL{}:
\begin{align}
\pi^+ &\to \mu^+ + \nu_\mu && \text{BR} \approx 99.988\% \\
\pi^+ &\to e^+ + \nu_e && \text{BR} \approx 1.23 \times 10^{-4}
\end{align}
with lifetime $\tau_\pi \approx 2.60 \times 10^{-8}$ s \tagBL{}.

\subsubsection{Helicity Suppression: What Is Baseline and What Is EDC Interpretation}

Experimentally, the charged pion decay rates satisfy a strong lepton-mass dependence \tagBL{},
often summarized by a helicity suppression factor.

The ratio of branching ratios is \tagBL{}:
\begin{equation}
R_{e/\mu} \equiv \frac{\text{BR}(\pi \to e\nu)}{\text{BR}(\pi \to \mu\nu)}
\approx \left(\frac{m_e}{m_\mu}\right)^2 \times
\frac{(m_\pi^2 - m_e^2)^2}{(m_\pi^2 - m_\mu^2)^2}
\approx 1.28 \times 10^{-4}.
\label{eq:pi_ratio_story}
\end{equation}

In this chapter we separate:
\begin{enumerate}[nosep]
  \item[(i)] \textbf{Baseline fact} \tagBL{}: the observed scaling structure,
  \item[(ii)] \textbf{EDC interpretation} \tagP{}/\tagOpen{}: that the same chiral boundary projection
        $\mathcal{P}_{\mathrm{chir}}$ acts on composite-to-lepton release.
\end{enumerate}

\begin{edcWarningBox}{No fake derivation}{}
We do \emph{not} claim a first-principles derivation of the $m_\ell^2$ scaling here.
The correct task is to compute the boundary/projection operator for the pion composite ontology \tagOpen{}.
\end{edcWarningBox}

\subsubsection{Process-Level Narrative for $\pi \to \ell\nu$}

\begin{figure}[ht]
\centering
% figures/fig_pion_process_pipeline.tex
% Pion decay process pipeline diagram (composite junction-pair)
\begin{tikzpicture}[scale=0.88, transform shape]

% Load styles
% tikz_style_edc.tex — Reusable TikZ styles for EDC papers
% Version 1.0 — 2026-01-20
% Include via: % tikz_style_edc.tex — Reusable TikZ styles for EDC papers
% Version 1.0 — 2026-01-20
% Include via: \input{tikz_style_edc}

% ============================================================
% REQUIRED LIBRARIES (must be loaded in main document)
% ============================================================
% \usetikzlibrary{calc,angles,quotes,decorations.markings,decorations.pathmorphing,positioning}

% ============================================================
% POSITIONING DEFAULTS
% ============================================================
\tikzset{
    % Default node distances for horizontal/vertical layouts
    edc node distance/.style={node distance=1.6cm and 2.0cm},
    % Compact variant for dense diagrams
    edc compact/.style={node distance=1.2cm and 1.5cm},
    % Spread variant for clarity
    edc spread/.style={node distance=2.0cm and 2.5cm},
}

% ============================================================
% COLOR PALETTE (consistent with epistemic tags)
% ============================================================
\definecolor{edcBulk}{RGB}{220,50,50}        % Red tones for bulk/5D
\definecolor{edcBrane}{RGB}{50,150,50}       % Green tones for brane-layer
\definecolor{edcOutput}{RGB}{50,100,200}     % Blue tones for 3D outputs
\definecolor{edcNeutral}{RGB}{100,100,100}   % Gray for neutral/annotations

% ============================================================
% BOX STYLES
% ============================================================
\tikzset{
    % Generic EDC box (base style)
    edc box/.style={
        rectangle,
        draw,
        rounded corners=3pt,
        minimum width=2.2cm,
        minimum height=0.8cm,
        align=center,
        font=\small,
        inner sep=4pt,
    },
    % Bulk-core box (red family)
    bulk box/.style={
        edc box,
        fill=red!10,
        draw=edcBulk!70!black,
        text=black,
    },
    % Brane-layer box (green family)
    brane box/.style={
        edc box,
        fill=green!10,
        draw=edcBrane!70!black,
        text=black,
    },
    % 3D output box (blue family)
    output box/.style={
        edc box,
        fill=blue!10,
        draw=edcOutput!70!black,
        text=black,
    },
    % Neutral/process box
    process box/.style={
        edc box,
        fill=gray!10,
        draw=gray!60!black,
        text=black,
    },
    % Label-only box (no background)
    label box/.style={
        rectangle,
        rounded corners=2pt,
        draw=gray!40,
        fill=white,
        inner sep=2pt,
        font=\scriptsize,
    },
}

% ============================================================
% ARROW STYLES
% ============================================================
\tikzset{
    % Standard thick arrow
    edc arrow/.style={
        ->,
        >=stealth,
        thick,
    },
    % Emphasized arrow (for main flow)
    edc flow/.style={
        ->,
        >=stealth,
        very thick,
        line width=1.2pt,
    },
    % Dashed arrow (for optional/weak connections)
    edc dashed/.style={
        ->,
        >=stealth,
        thick,
        dashed,
    },
    % Double arrow (for bidirectional)
    edc bidir/.style={
        <->,
        >=stealth,
        thick,
    },
}

% ============================================================
% REGION STYLES (for background fills)
% ============================================================
\tikzset{
    % Bulk region (5D)
    bulk region/.style={
        fill=blue!8,
    },
    % Brane layer region
    brane region/.style={
        fill=yellow!25,
    },
    % Observer/3D region
    observer region/.style={
        fill=green!8,
    },
}

% ============================================================
% LABEL STYLES
% ============================================================
\tikzset{
    % Phase label (below nodes)
    phase label/.style={
        font=\scriptsize\itshape,
        text=black!70,
    },
    % Equation label (for inline math)
    eq label/.style={
        font=\scriptsize,
        fill=white,
        inner sep=1pt,
    },
    % Section annotation
    section label/.style={
        font=\footnotesize\bfseries,
        text=black,
    },
}

% ============================================================
% JUNCTION/PARTICLE STYLES
% ============================================================
\tikzset{
    % Y-junction point
    junction point/.style={
        circle,
        fill=red!60!black,
        minimum size=4pt,
        inner sep=0pt,
    },
    % Flux tube arm
    flux arm/.style={
        thick,
        blue!60!black,
    },
    % Particle dot (electron, etc.)
    particle/.style={
        circle,
        fill=black,
        minimum size=5pt,
        inner sep=0pt,
    },
    % Neutrino (smaller, gray)
    neutrino/.style={
        circle,
        fill=gray,
        minimum size=4pt,
        inner sep=0pt,
    },
}

% ============================================================
% SPRING DECORATION (for mechanical models)
% ============================================================
\tikzset{
    spring/.style={
        thick,
        decorate,
        decoration={
            coil,
            aspect=0.5,
            segment length=2mm,
            amplitude=2mm,
        },
    },
    % Wave decoration (for field modes)
    wave field/.style={
        thick,
        decorate,
        decoration={
            snake,
            amplitude=2pt,
            segment length=8pt,
        },
    },
}

% ============================================================
% BOUNDARY STYLES
% ============================================================
\tikzset{
    % Bulk-facing boundary (dashed red)
    bulk boundary/.style={
        very thick,
        red!70!black,
        dashed,
    },
    % Observer-facing boundary (solid green)
    observer boundary/.style={
        thick,
        green!50!black,
    },
    % Brane edge (orange)
    brane edge/.style={
        thick,
        orange!70!black,
    },
}

% ============================================================
% CONVENIENCE COMMANDS
% ============================================================
% Arrow label (above)
\newcommand{\arrlabel}[1]{\scriptsize #1}
% Arrow label (below)
\newcommand{\arrlabelb}[1]{\scriptsize #1}

% ============================================================
% END OF STYLE FILE
% ============================================================


% ============================================================
% REQUIRED LIBRARIES (must be loaded in main document)
% ============================================================
% \usetikzlibrary{calc,angles,quotes,decorations.markings,decorations.pathmorphing,positioning}

% ============================================================
% POSITIONING DEFAULTS
% ============================================================
\tikzset{
    % Default node distances for horizontal/vertical layouts
    edc node distance/.style={node distance=1.6cm and 2.0cm},
    % Compact variant for dense diagrams
    edc compact/.style={node distance=1.2cm and 1.5cm},
    % Spread variant for clarity
    edc spread/.style={node distance=2.0cm and 2.5cm},
}

% ============================================================
% COLOR PALETTE (consistent with epistemic tags)
% ============================================================
\definecolor{edcBulk}{RGB}{220,50,50}        % Red tones for bulk/5D
\definecolor{edcBrane}{RGB}{50,150,50}       % Green tones for brane-layer
\definecolor{edcOutput}{RGB}{50,100,200}     % Blue tones for 3D outputs
\definecolor{edcNeutral}{RGB}{100,100,100}   % Gray for neutral/annotations

% ============================================================
% BOX STYLES
% ============================================================
\tikzset{
    % Generic EDC box (base style)
    edc box/.style={
        rectangle,
        draw,
        rounded corners=3pt,
        minimum width=2.2cm,
        minimum height=0.8cm,
        align=center,
        font=\small,
        inner sep=4pt,
    },
    % Bulk-core box (red family)
    bulk box/.style={
        edc box,
        fill=red!10,
        draw=edcBulk!70!black,
        text=black,
    },
    % Brane-layer box (green family)
    brane box/.style={
        edc box,
        fill=green!10,
        draw=edcBrane!70!black,
        text=black,
    },
    % 3D output box (blue family)
    output box/.style={
        edc box,
        fill=blue!10,
        draw=edcOutput!70!black,
        text=black,
    },
    % Neutral/process box
    process box/.style={
        edc box,
        fill=gray!10,
        draw=gray!60!black,
        text=black,
    },
    % Label-only box (no background)
    label box/.style={
        rectangle,
        rounded corners=2pt,
        draw=gray!40,
        fill=white,
        inner sep=2pt,
        font=\scriptsize,
    },
}

% ============================================================
% ARROW STYLES
% ============================================================
\tikzset{
    % Standard thick arrow
    edc arrow/.style={
        ->,
        >=stealth,
        thick,
    },
    % Emphasized arrow (for main flow)
    edc flow/.style={
        ->,
        >=stealth,
        very thick,
        line width=1.2pt,
    },
    % Dashed arrow (for optional/weak connections)
    edc dashed/.style={
        ->,
        >=stealth,
        thick,
        dashed,
    },
    % Double arrow (for bidirectional)
    edc bidir/.style={
        <->,
        >=stealth,
        thick,
    },
}

% ============================================================
% REGION STYLES (for background fills)
% ============================================================
\tikzset{
    % Bulk region (5D)
    bulk region/.style={
        fill=blue!8,
    },
    % Brane layer region
    brane region/.style={
        fill=yellow!25,
    },
    % Observer/3D region
    observer region/.style={
        fill=green!8,
    },
}

% ============================================================
% LABEL STYLES
% ============================================================
\tikzset{
    % Phase label (below nodes)
    phase label/.style={
        font=\scriptsize\itshape,
        text=black!70,
    },
    % Equation label (for inline math)
    eq label/.style={
        font=\scriptsize,
        fill=white,
        inner sep=1pt,
    },
    % Section annotation
    section label/.style={
        font=\footnotesize\bfseries,
        text=black,
    },
}

% ============================================================
% JUNCTION/PARTICLE STYLES
% ============================================================
\tikzset{
    % Y-junction point
    junction point/.style={
        circle,
        fill=red!60!black,
        minimum size=4pt,
        inner sep=0pt,
    },
    % Flux tube arm
    flux arm/.style={
        thick,
        blue!60!black,
    },
    % Particle dot (electron, etc.)
    particle/.style={
        circle,
        fill=black,
        minimum size=5pt,
        inner sep=0pt,
    },
    % Neutrino (smaller, gray)
    neutrino/.style={
        circle,
        fill=gray,
        minimum size=4pt,
        inner sep=0pt,
    },
}

% ============================================================
% SPRING DECORATION (for mechanical models)
% ============================================================
\tikzset{
    spring/.style={
        thick,
        decorate,
        decoration={
            coil,
            aspect=0.5,
            segment length=2mm,
            amplitude=2mm,
        },
    },
    % Wave decoration (for field modes)
    wave field/.style={
        thick,
        decorate,
        decoration={
            snake,
            amplitude=2pt,
            segment length=8pt,
        },
    },
}

% ============================================================
% BOUNDARY STYLES
% ============================================================
\tikzset{
    % Bulk-facing boundary (dashed red)
    bulk boundary/.style={
        very thick,
        red!70!black,
        dashed,
    },
    % Observer-facing boundary (solid green)
    observer boundary/.style={
        thick,
        green!50!black,
    },
    % Brane edge (orange)
    brane edge/.style={
        thick,
        orange!70!black,
    },
}

% ============================================================
% CONVENIENCE COMMANDS
% ============================================================
% Arrow label (above)
\newcommand{\arrlabel}[1]{\scriptsize #1}
% Arrow label (below)
\newcommand{\arrlabelb}[1]{\scriptsize #1}

% ============================================================
% END OF STYLE FILE
% ============================================================


% ─────────────────────────────────────────────────────────────────────────────
% Background regions
% ─────────────────────────────────────────────────────────────────────────────
\fill[orange!8] (-5.5,1.8) rectangle (5.5,3.2);
\fill[blue!8] (-5.5,0.4) rectangle (5.5,1.8);
\fill[green!8] (-5.5,-1.2) rectangle (5.5,0.4);

% Region labels
\node[font=\scriptsize, orange!60!black] at (-4.5,3.0) {Composite (junction-pair)};
\node[font=\scriptsize, blue!50!black] at (-4.5,1.5) {Brane interface};
\node[font=\scriptsize, green!50!black] at (-4.5,0.1) {3D outputs};

% ─────────────────────────────────────────────────────────────────────────────
% Composite layer nodes
% ─────────────────────────────────────────────────────────────────────────────
\node[rectangle, draw=orange!60, fill=orange!10, rounded corners=3pt,
      text width=2.6cm, align=center, font=\small] (pion) at (-3.2,2.5)
  {Pion $\pi^+$\\{\tiny $(u\bar{d})$ junction-pair}};

\node[rectangle, draw=orange!60, fill=orange!10, rounded corners=3pt,
      text width=2.4cm, align=center, font=\small] (annihil) at (0.5,2.5)
  {Annihilation\\{\tiny $q\bar{q} \to W^*$}};

% ─────────────────────────────────────────────────────────────────────────────
% Brane layer nodes
% ─────────────────────────────────────────────────────────────────────────────
\node[brane box, text width=2.6cm] (diss) at (-1.5,1.1)
  {Energy release\\{\tiny $m_\pi c^2 \to E_{\mathrm{brane}}$}};

\node[gate box, text width=2.8cm, minimum height=0.9cm] (frozen) at (2.5,1.1)
  {$\mathcal{P}_{\mathrm{frozen}}$\\{\tiny chiral projection}};

% ─────────────────────────────────────────────────────────────────────────────
% Output layer nodes
% ─────────────────────────────────────────────────────────────────────────────
\node[output box, text width=1.8cm] (muout) at (1.2,-0.4)
  {$\mu^+$\\{\tiny 99.99\%}};

\node[output box, text width=1.8cm] (nuout) at (3.2,-0.4)
  {$\nu_\mu$};

\node[output box, text width=1.6cm, dashed, draw=gray!50, fill=gray!5] (eout) at (5.0,-0.4)
  {$e^+$\\{\tiny $10^{-4}$}};

% ─────────────────────────────────────────────────────────────────────────────
% Arrows
% ─────────────────────────────────────────────────────────────────────────────
\draw[edc flow] (pion) -- (annihil);
\draw[edc arrow] (annihil.south) -- ++(0,-0.3) -| (diss.north);
\draw[edc flow] (diss) -- (frozen);

% Release to outputs
\draw[edc arrow] (frozen.south) -- ++(0,-0.25) -| (muout.north);
\draw[edc arrow] (frozen.south) -- ++(0,-0.25) -| (nuout.north);
\draw[edc dashed] (frozen.south) -- ++(0,-0.25) -| (eout.north);

% ─────────────────────────────────────────────────────────────────────────────
% Annotations
% ─────────────────────────────────────────────────────────────────────────────

% Helicity suppression box
\node[rectangle, draw=red!50, fill=red!5, rounded corners=2pt,
      font=\tiny, align=center, text width=4.8cm] at (-2.5,-0.6)
  {\textbf{Helicity suppression} [BL]:\\
   $R_{e/\mu} \approx (m_e/m_\mu)^2 \approx 10^{-4}$\\
   EDC: from $\mathcal{P}_{\mathrm{chir}}$ boundary conditions [OPEN]};

% Composite note
\node[rectangle, draw=orange!40, fill=orange!5, rounded corners=2pt,
      font=\tiny, align=center, text width=3.0cm] at (0.5,3.0)
  {Hadron$\to$Lepton bridge\\
   (composite $\to$ fundamental)};

\end{tikzpicture}

\caption{Pion decay pipeline. The pion as a composite junction-pair annihilates and releases
energy through the brane interface. The chiral projection $\mathcal{P}_{\mathrm{chir}}$
produces helicity suppression, favoring $\mu^+$ over $e^+$ by factor $\sim 10^4$ [BL].}
\label{fig:pion_pipeline}
\end{figure}

Mechanistically, the pion must first be represented as a composite brane/boundary excitation,
then released via $\mathcal{P}_{\mathrm{frozen}}$ into lepton + neutrino outputs,
subject to kinematic allowance and chirality selection:
\begin{equation}
\Psi_\pi \;\Rightarrow\; E_{\mathrm{brane/boundary}} \;\Rightarrow\; \mathcal{P}_{\mathrm{frozen}}
\;\Rightarrow\; \ell^- + \bar\nu_\ell + \text{(recoil/soft)}.
\end{equation}

The pion case is logically inverted relative to neutron: the pion is itself a
\emph{composite interface object} which releases into lepton + neutrino under
projection constraints.

\subsubsection{EDC Consistency Target}

We do \emph{not} claim to derive $m_\ell^2$ suppression here. Instead, we require that a
boundary chirality/projection mechanism can \emph{in principle} produce a
lepton-mass sensitivity via BC structure \tagP{}/\tagOpen{}.

That is, we demand a future-computable link:
\begin{equation}
\mathcal{P}_{\mathrm{chir}} \text{ and } \mathcal{P}_{\mathrm{mode}} \quad \Rightarrow
\quad \text{suppression pattern consistent with baseline helicity selection}.
\label{eq:pion_suppression_target_story}
\end{equation}

\paragraph{What EDC adds.}
The SM calculation gives the correct numerical factor; EDC's contribution is to
interpret \emph{why} the boundary conditions have this effect. In the thick-brane
picture, the V--A structure is not inserted by hand but emerges from the
geometry of the observer-facing interface \tagP{}.

\paragraph{What remains open.}
Deriving the factor $(m_\ell/m_\pi)^2$ from first principles requires:
\begin{enumerate}[nosep]
  \item Solving the Dirac equation in the thick-brane background
  \item Imposing boundary conditions at the observer-facing edge
  \item Computing the projection onto helicity eigenstates
\end{enumerate}
This is a well-posed mathematical problem \tagOpen{}.

\subsubsection{Honest Statement About Helicity Suppression}

\begin{tcolorbox}[guardrail, title={No Numerology}]
We must be explicit about what EDC does and does not achieve here:

\textbf{What is baseline} \tagBL{}: The helicity suppression factor
$(m_e/m_\mu)^2$ and the exact branching ratio formula are established
Standard Model results.

\textbf{What EDC provides} \tagP{}: A structural interpretation that locates
helicity suppression in the observer-facing boundary conditions
($\mathcal{P}_{\mathrm{chir}}$), rather than treating it as an inexplicable
feature of the weak vertex.

\textbf{What remains open} \tagOpen{}: Deriving the factor from explicit
boundary-condition calculations.

We do not claim that EDC ``explains'' helicity suppression if by that we mean
a complete first-principles derivation. We claim that EDC provides a framework
in which such a derivation is well-posed.
\end{tcolorbox}

\subsubsection{Falsifiability Hooks}

\begin{tcolorbox}[falsifiability]
\begin{itemize}[nosep]
  \item If the framework cannot even qualitatively accommodate helicity
        suppression without ad hoc patching, the BC interpretation is likely wrong.
  \item If composite ontology contradicts earlier lepton ontology (e.g., by
        forcing redefinition of what a charged defect is), it fails.
  \item If the pion lifetime cannot be connected to the junction-pair
        annihilation rate, the mechanism fails.
  \item If the neutral pion ($\pi^0 \to \gamma\gamma$) requires a qualitatively
        different framework, the unified-pipeline claim is weakened.
\end{itemize}
\end{tcolorbox}

