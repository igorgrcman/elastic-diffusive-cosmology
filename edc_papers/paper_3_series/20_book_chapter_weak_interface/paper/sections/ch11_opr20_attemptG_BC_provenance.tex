%!TEX root = ../EDC_Part_II_Weak_Sector.tex
% ==============================================================================
% OPR-20 Attempt G_BC: Boundary Condition Provenance for the Weak Mediator
% Status: [BL] orbifold parity → BC mapping + [P] mediator field identification
% ==============================================================================

\subsection{Attempt G\_BC: Boundary Condition Provenance}
\label{sec:ch11_opr20_attemptG_BC}

The reconciliation audit (\S\ref{sec:ch11_opr20_attemptG}, commit 81de2b2) revealed
that the factor-of-2 discrepancy between attempts C/D ($m_\phi \approx 70$ GeV) and
attempt E ($m_\phi \approx 35$ GeV) arises from different boundary condition
assumptions: $x_1 = \pi$ (Dirichlet-Dirichlet) versus $x_1 = \pi/2$ (Neumann-Neumann).
\textbf{This is not an error---it is a physical fork.} This section establishes which
BC is appropriate for the weak mediator and unifies the narrative across all attempts.

% ------------------------------------------------------------------------------
\subsubsection{G\_BC.1: BC Ledger}
\label{sec:attemptG_BC_ledger}

\begin{table}[ht]
\centering
\caption{Boundary condition ledger for OPR-20 attempts}
\label{tab:bc_ledger}
\small
\begin{tabular}{lllccc}
\toprule
\textbf{Attempt} & \textbf{Field Type} & \textbf{BC Assumption} &
    \textbf{$x_1$} & \textbf{$m_\phi$} & \textbf{Status} \\
\midrule
C/D & Scalar profile $f(z)$ & DD (implicit) & $\pi$ & $\sim$70 GeV & [P] \\
E   & Scalar profile $f(z)$ & NN (explicit) & $\pi/2$ & $\sim$35 GeV & [P] \\
F   & Scalar profile $f(z)$ & Robin (scanned) & varies & 35--70 GeV & [P] \\
G   & Scalar profile $f(z)$ & Robin ($\alpha = 2\pi$) & $\sim$2.4 & $\sim$53 GeV & [P] \\
\bottomrule
\end{tabular}
\end{table}

\paragraph{Key observation.}
All attempts use the same underlying structure:
\begin{itemize}[nosep]
    \item $2\pi$ factor from circumference interpretation: \tagDc{}
    \item $\sqrt{2}$ factor from orbifold normalization: \tagDc{}
    \item Combined $2\pi\sqrt{2} \approx 8.89$: \tagDc{}
\end{itemize}
The \emph{only} difference is the eigenvalue $x_1$, which depends on the boundary
condition. Since $m_\phi = x_1/\ell$, the BC choice directly determines the
mediator mass.

% ------------------------------------------------------------------------------
\subsubsection{G\_BC.2: Orbifold Parity $\to$ BC Mapping}
\label{sec:attemptG_BC_orbifold}

The $Z_2$ orbifold $S^1/\mathbb{Z}_2$ identifies points under $z \to -z$. The
fixed points are at $z = 0$ and $z = \ell$ (or equivalently, at the boundaries
of the fundamental domain $[0, \ell]$).

\paragraph{Standard orbifold convention \tagBL{}.}
A field $\phi(z)$ must have definite parity under $Z_2$:
\begin{align}
    \textbf{Even parity:} \quad \phi(-z) &= +\phi(z)
        \quad \Rightarrow \quad \phi'(0) = 0 \text{ (Neumann)}, \\
    \textbf{Odd parity:} \quad \phi(-z) &= -\phi(z)
        \quad \Rightarrow \quad \phi(0) = 0 \text{ (Dirichlet)}.
\end{align}
This is the standard extra-dimension convention (see, e.g., Randall--Sundrum, Phys.\ Rev.\ Lett.\ \textbf{83}, 1999; Rattazzi, Int.\ J.\ Mod.\ Phys.\ A \textbf{18}, 2003).

\begin{tcolorbox}[colback=gray!5!white, colframe=gray!60!black,
    title=\textbf{Parity $\to$ BC Mapping [BL]}]
\begin{center}
\begin{tabular}{lll}
\toprule
\textbf{Parity} & \textbf{BC at Fixed Points} & \textbf{$x_1$ (Ground State)} \\
\midrule
Even (+) & Neumann ($f' = 0$) & 0 (constant mode) \\
Odd ($-$) & Dirichlet ($f = 0$) & $\pi$ (first mode) \\
\bottomrule
\end{tabular}
\end{center}
\emph{This mapping is standard brane-world physics, not EDC-specific.}
\end{tcolorbox}

\paragraph{5D gauge field decomposition.}
A 5D gauge field $A_M$ ($M = 0,1,2,3,5$) splits into 4D components:
\begin{itemize}[nosep]
    \item $A_\mu$ ($\mu = 0,1,2,3$): 4D vector, typically \textbf{even} under $Z_2$
    \item $A_5$: 4D scalar (from 5D perspective), typically \textbf{odd} under $Z_2$
\end{itemize}

This assignment ensures that the 4D gauge symmetry is preserved on the brane:
\begin{itemize}[nosep]
    \item $A_\mu$ even $\Rightarrow$ zero-mode exists (massless 4D gauge boson)
    \item $A_5$ odd $\Rightarrow$ no zero-mode (scalar eaten or decoupled)
\end{itemize}

\begin{tcolorbox}[colback=blue!5!white, colframe=blue!50!black,
    title=\textbf{5D Gauge Field Spectrum [BL]}]
\begin{center}
\begin{tabular}{lllll}
\toprule
\textbf{Component} & \textbf{4D Nature} & \textbf{Parity} & \textbf{BC} & \textbf{$x_n$} \\
\midrule
$A_\mu$ & 4D vector & Even (+) & Neumann & $n\pi$ ($n = 0,1,2,\ldots$) \\
$A_5$ & 4D scalar & Odd ($-$) & Dirichlet & $n\pi$ ($n = 1,2,3,\ldots$) \\
\bottomrule
\end{tabular}
\end{center}
\emph{The $A_\mu$ zero-mode ($n=0$) is the massless gauge boson.
Massive modes have $n \geq 1$, giving $x_n = n\pi$.}
\end{tcolorbox}

% ------------------------------------------------------------------------------
\subsubsection{G\_BC.3: Robin BC Limiting Cases}
\label{sec:attemptG_BC_robin}

The Robin boundary condition $f' + \alpha f = 0$ interpolates between Neumann
($\alpha = 0$) and Dirichlet ($\alpha \to \infty$).

\paragraph{Eigenvalue equation.}
For symmetric Robin BC on $[0, 1]$, the eigenvalue equation is \tagDc{}:
\begin{equation}
    x \tan(x) = \alpha
    \label{eq:attemptG_BC_robin_eigenvalue}
\end{equation}
for the ground-state-like solutions.

\paragraph{Limiting cases.}
\begin{itemize}[nosep]
    \item $\alpha \to 0$ (Neumann limit): $\tan(x) \to \infty$ $\Rightarrow$ $x_n = (n + \frac{1}{2})\pi$,
          but the true ground state is $x_0 = 0$ (constant mode).
    \item $\alpha \to \infty$ (Dirichlet limit): $\tan(x) \to 0$ $\Rightarrow$ $x_n = n\pi$,
          with ground state $x_0 = \pi$.
\end{itemize}

\paragraph{Interpolation.}
For intermediate $\alpha$, the eigenvalue $x_0$ smoothly transitions from 0 to $\pi$.
Attempt F found that $\alpha \in [5.5, 15]$ gives $x_1 \in [2.3, 2.8]$, which is
the intermediate regime.

\begin{table}[ht]
\centering
\caption{Robin BC eigenvalue $x_0$ as function of $\alpha$ (numerical)}
\label{tab:robin_scan}
\small
\begin{tabular}{rlll}
\toprule
$\alpha$ & $x_0$ & $x_0/\pi$ & Regime \\
\midrule
0 & 0.00 & 0.00 & Neumann (constant) \\
1 & 0.86 & 0.27 & Near-Neumann \\
5 & 1.31 & 0.42 & Intermediate \\
10 & 1.43 & 0.46 & Intermediate \\
$\infty$ & $\pi$ & 1.00 & Dirichlet \\
\bottomrule
\end{tabular}
\end{table}

\emph{Note:} The table shows the ground-state eigenvalue for the Robin equation
$x \tan x = \alpha$. For Attempt F's BVP solver (which solves on interior grid),
the values differ slightly due to finite-difference discretization.

% ------------------------------------------------------------------------------
\subsubsection{G\_BC.4: Which BC for the Weak Mediator?}
\label{sec:attemptG_BC_mediator}

\paragraph{\texorpdfstring{Option 1: $A_\mu$ zero-mode (even, Neumann).}{Option 1: A-mu zero-mode (even, Neumann).}}
The standard 4D gauge boson is the $A_\mu$ zero-mode with $x_0 = 0$ (massless).
This \textbf{cannot} be the massive W/Z without additional mass generation (Higgs).

\paragraph{\texorpdfstring{Option 2: $A_5$ component (odd, Dirichlet).}{Option 2: A5 component (odd, Dirichlet).}}
The 5D scalar component $A_5$ has $x_1 = \pi$, giving:
\begin{equation}
    m_\phi = \frac{\pi}{\ell} = \frac{\pi}{2\pi\sqrt{2} R_\xi} \approx 70 \text{ GeV}
\end{equation}
This is a geometric mass, not Higgs-generated.

\paragraph{\texorpdfstring{Option 3: KK $A_\mu$ excitation (even, Neumann, $n = 1$).}{Option 3: KK A-mu excitation (even, Neumann, n=1).}}
The first KK mode of $A_\mu$ also has $x_1 = \pi$ (same as Option 2):
\begin{equation}
    m_{\text{KK}} = \frac{\pi}{\ell} \approx 70 \text{ GeV}
\end{equation}
This is the standard KK tower picture.

\paragraph{Option 4: Brane-localized gauge fields (OPR-17).}
If SU(2)$_L$ is brane-localized (\S\ref{sec:ch9_su2_embedding}), the gauge fields
do not propagate in the bulk. The ``mediator mass'' then arises from different
physics (e.g., confinement, overlap suppression) rather than KK quantization.

\begin{tcolorbox}[colback=yellow!5!white, colframe=yellow!60!black,
    title=\textbf{BC Choice Fork [P]}]
\textbf{Baseline (canonical):} $x_1 = \pi$ (Dirichlet or first KK Neumann)

\textbf{Justification:}
\begin{itemize}[nosep]
    \item Closer to $M_W = 80$ GeV (12\% vs 56\% discrepancy)
    \item Consistent with Attempts C/D
    \item Standard for massive 5D gauge modes
\end{itemize}

\textbf{Epistemic status:} This is \tagP{} until derived from:
\begin{itemize}[nosep]
    \item Mediator field identification (what is it?)
    \item Parity assignment from gauge structure
    \item Junction/BKT physics if Robin
\end{itemize}
\end{tcolorbox}

% ------------------------------------------------------------------------------
\subsubsection{G\_BC.5: Unified Narrative}
\label{sec:attemptG_BC_unified}

\paragraph{What the attempts established.}
\begin{enumerate}[nosep]
    \item \textbf{Attempt C/D:} Best geometric factor $2\pi\sqrt{2} \approx 8.89$ \tagDc{}+\tagP{};
          implicit $x_1 = \pi$ (DD); $m_\phi \approx 70$ GeV.
    \item \textbf{Attempt E:} $2\pi$ factor from circumference \tagDc{};
          explicit $x_1 = \pi/2$ (NN); $m_\phi \approx 35$ GeV.
    \item \textbf{Attempt F:} Robin BC from junction \tagDc{}; broad $\alpha$ band;
          $x_1$ interpolates between $\pi/2$ and $\pi$.
    \item \textbf{Attempt G:} Natural $\alpha = 2\pi$ from $\ell/\delta$ \tagDc{}+\tagP{};
          $x_1 \approx 2.4$; $m_\phi \approx 53$ GeV.
    \item \textbf{Attempt G\_BC:} BC choice is a [P] fork; baseline $x_1 = \pi$
          is pragmatic but not derived.
\end{enumerate}

\paragraph{Structural synthesis.}
The KK mass formula is:
\begin{equation}
    \boxed{
    m_\phi = \frac{x_1}{\ell} = \frac{x_1}{2\pi\sqrt{2} R_\xi}
    }
    \label{eq:attemptG_BC_mass_formula}
\end{equation}
where:
\begin{itemize}[nosep]
    \item $2\pi$: circumference interpretation \tagDc{}
    \item $\sqrt{2}$: orbifold normalization \tagDc{}
    \item $x_1$: BC-dependent eigenvalue \tagP{}
    \item $R_\xi$: diffusion scale \tagP{}
\end{itemize}

% ------------------------------------------------------------------------------
\subsubsection{G\_BC.6: Epistemic Summary and OPR-20 Split}
\label{sec:attemptG_BC_epistemic}

\begin{tcolorbox}[colback=blue!5!white, colframe=blue!50!black,
    title=\textbf{OPR-20 Split into OPR-20a/20b}]

The reconciliation audit clarifies that OPR-20 contains two distinct open problems:

\medskip
\textbf{OPR-20a: BC Provenance}
\begin{itemize}[nosep]
    \item \emph{Question:} What is the physical mediator field, and what BC does it have?
    \item \emph{Options:} DD ($x_1 = \pi$), NN ($x_1 = \pi/2$), Robin (variable)
    \item \emph{Status:} \textbf{[OPEN]} --- parity/field identity not derived
    \item \emph{Upgrade condition:} Establish mediator identity + parity from gauge structure
\end{itemize}

\medskip
\textbf{OPR-20b: $\alpha$ Provenance}
\begin{itemize}[nosep]
    \item \emph{Question:} If Robin BC, where does $\alpha \sim \mathcal{O}(10)$ come from?
    \item \emph{Candidate:} $\alpha = \ell/\delta$ with $\delta = R_\xi$ gives $\alpha = 2\pi$ [P]
    \item \emph{Status:} \textbf{[OPEN]} --- $\delta = R_\xi$ identification not derived
    \item \emph{Upgrade condition:} Derive $\delta = R_\xi$ from brane microphysics
\end{itemize}

\medskip
\textbf{Main OPR-20 Status:} \textbf{RED-C [Dc]+[P]}
\begin{itemize}[nosep]
    \item Structural factors ($2\pi$, $\sqrt{2}$) now \tagDc{}
    \item BC choice and $\alpha$ provenance remain [OPEN]
    \item Clear upgrade pathway exists for both sub-problems
\end{itemize}
\end{tcolorbox}

\begin{tcolorbox}[
    colback=yellow!5!white,
    colframe=yellow!60!black,
    title=\textbf{OPR-20 Attempt G\_BC: Stoplight Verdict}
]
\begin{center}
\textbf{\large RED-C [Dc]+[P] (Structural Progress, BC Fork Identified)}
\end{center}

\medskip
\textbf{What improved:}
\begin{itemize}[nosep]
    \item Reconciled C/D vs E discrepancy (BC choice, not error)
    \item Established orbifold parity $\to$ BC mapping [BL]
    \item Identified canonical baseline: $x_1 = \pi$ [P]
    \item Split OPR-20 into OPR-20a (BC) and OPR-20b ($\alpha$)
\end{itemize}

\textbf{What remains:}
\begin{itemize}[nosep]
    \item Mediator field identity not established
    \item BC choice is pragmatic [P], not derived
    \item $\alpha$ provenance still open (see Attempt G)
\end{itemize}

\textbf{Upgrade condition:}
\begin{quote}
OPR-20a $\to$ \textbf{YELLOW [P]} when mediator field (A$_5$? KK A$_\mu$? brane scalar?)
is identified and parity follows from gauge structure.
\end{quote}
\end{tcolorbox}

