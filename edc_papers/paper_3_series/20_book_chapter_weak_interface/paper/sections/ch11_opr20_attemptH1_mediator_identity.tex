%!TEX root = ../EDC_Part_II_Weak_Sector.tex
% ==============================================================================
% OPR-20a Attempt H1: Mediator Field Identity → Boundary Condition Provenance
% Status: [P]+[Dc] — shortlist with structural discriminants
% ==============================================================================

\subsection{Attempt H1: Mediator Field Identity and BC Provenance}
\label{sec:ch11_opr20_attemptH1}

Attempt G\_BC established that the factor-of-2 discrepancy between earlier attempts
is not an error but a boundary condition choice tied to mediator identity. This
section addresses OPR-20a: \emph{What 5D field component is the weak mediator,
and what boundary condition does it carry?}

\begin{tcolorbox}[colback=gray!5!white, colframe=gray!60!black,
    title=\textbf{Attempt H1 Goal}]
Determine which 5D field component is the weak mediator in the EDC effective
$G_F$ chain, and thereby fix the correct BC class (DD vs NN vs Robin) from
orbifold parity and gauge decomposition.

\medskip
\textbf{NOT claimed:}
\begin{itemize}[nosep]
    \item Derivation of SU(2)$_L$ gauge symmetry origin
    \item W$^\pm$/Z$^0$ mass from Higgs mechanism
    \item Numerical value of $M_W$ (that remains a [BL] comparison)
\end{itemize}

\textbf{Claimed:}
\begin{itemize}[nosep]
    \item Systematic enumeration of mediator candidates
    \item Structural analysis: parity, BC, eigenvalue, coupling to LH fermions
    \item Ranked shortlist with explicit discriminants
\end{itemize}
\end{tcolorbox}

% ------------------------------------------------------------------------------
\subsubsection{H1.1: Candidate Enumeration}
\label{sec:attemptH1_candidates}

We consider five candidate interpretations for the weak mediator field:

\begin{enumerate}[label=(\roman*)]
    \item \textbf{KK zero-mode of $A_\mu$} (4D vector, even parity)
    \item \textbf{First KK mode of $A_\mu$} (4D vector, even parity, $n=1$)
    \item \textbf{$A_5$ scalar component} (4D scalar, odd parity)
    \item \textbf{Brane-localized scalar mediator} (effective 4D, junction-induced)
    \item \textbf{Mixed/junction-induced mode} (effective, Robin BC)
\end{enumerate}

\paragraph{5D gauge field decomposition \tagBL{}.}
A 5D gauge field $A_M$ ($M = 0,1,2,3,5$) on the $Z_2$ orbifold $S^1/\mathbb{Z}_2$
decomposes under $\xi \to -\xi$:
\begin{align}
    A_\mu(x, -\xi) &= +A_\mu(x, \xi) \quad \text{(even parity)} \\
    A_5(x, -\xi) &= -A_5(x, \xi) \quad \text{(odd parity)}
\end{align}
This parity assignment preserves 4D Lorentz invariance and gauge symmetry
at the fixed points.

\paragraph{Resulting KK profiles.}
On the fundamental domain $\xi \in [0, \ell]$:
\begin{align}
    A_\mu^{(n)}(\xi) &\propto \cos\left(\frac{n\pi \xi}{\ell}\right)
        \quad \text{(Neumann: } f'|_{\text{bdy}} = 0\text{)} \\
    A_5^{(n)}(\xi) &\propto \sin\left(\frac{n\pi \xi}{\ell}\right)
        \quad \text{(Dirichlet: } f|_{\text{bdy}} = 0\text{)}
\end{align}

% ------------------------------------------------------------------------------
\subsubsection{H1.2: Coupling to Boundary-Localized Fermions}
\label{sec:attemptH1_coupling}

From Ch.~9 (\S\ref{sec:ch9_va_emergence}), left-handed fermions are
localized at the observer boundary ($\xi = 0$) with profile $f_L(\xi)$ peaked
at $\xi \approx 0$. The effective 4D coupling of a mediator field $\phi(\xi)$ to
these fermions is proportional to the \emph{overlap integral}:
\begin{equation}
    g_{\text{eff}} \propto \int_0^\ell |f_L(\xi)|^2 \, |\phi(\xi)|^2 \, d\xi
    \label{eq:attemptH1_overlap}
\end{equation}

\textbf{Key observation:} The mediator profile $\phi(\xi)$ at $\xi \approx 0$
determines coupling strength. This is a \emph{structural constraint} \tagDc{},
not an assumption.

\begin{tcolorbox}[colback=blue!5!white, colframe=blue!50!black,
    title=\textbf{Coupling Selection Criterion [Dc]}]
\begin{itemize}[nosep]
    \item Mediator peaked at boundary ($\xi = 0$): $\mathcal{O}(1)$ overlap with LH fermions
    \item Mediator vanishing at boundary: suppressed coupling
\end{itemize}
\emph{The boundary value of the mediator profile controls effective weak coupling.}
\end{tcolorbox}

% ------------------------------------------------------------------------------
\subsubsection{H1.3: Systematic Candidate Analysis}
\label{sec:attemptH1_analysis}

We now analyze each candidate against the structural criteria: parity, BC,
eigenvalue $x_1$, profile at boundary, and coupling to LH fermions.

\paragraph{\texorpdfstring{(i) KK zero-mode of $A_\mu$ (even, Neumann, $n=0$).}{(i) KK zero-mode of A-mu (even, Neumann, n=0).}}
\begin{itemize}[nosep]
    \item Parity: even (+) \tagBL{}
    \item BC: Neumann ($f' = 0$ at fixed points)
    \item Eigenvalue: $x_0 = 0$ (constant profile)
    \item Mass: $m_\phi = 0$ (massless)
    \item Profile: $A_\mu^{(0)}(\xi) = \text{const}$ — flat, nonzero everywhere
    \item Coupling: $\mathcal{O}(1)$ overlap (constant profile)
\end{itemize}
\textbf{Verdict:} \textcolor{BrickRed}{\textbf{REJECTED.}} The zero-mode is
massless. To be the W$^\pm$ boson, it would require Higgs mass generation,
which is outside EDC's geometric scope (OPR-17, open).

\paragraph{\texorpdfstring{(ii) First KK mode of $A_\mu$ (even, Neumann, $n=1$).}{(ii) First KK mode of A-mu (even, Neumann, n=1).}}
\begin{itemize}[nosep]
    \item Parity: even (+) \tagBL{}
    \item BC: Neumann ($f' = 0$ at fixed points)
    \item Eigenvalue: $x_1 = \pi$ (first excited)
    \item Mass: $m_\phi = \pi/\ell \approx 70$ GeV
    \item Profile: $A_\mu^{(1)}(\xi) \propto \cos(\pi \xi/\ell)$ — \textbf{peaked at boundaries}
    \item Coupling: \textbf{$\mathcal{O}(1)$} overlap (cosine maximum at $\xi = 0$)
\end{itemize}
\textbf{Verdict:} \textcolor{OliveGreen}{\textbf{VIABLE.}} Geometric mass without
Higgs, natural coupling to boundary LH fermions. Standard KK picture.

\paragraph{\texorpdfstring{(iii) $A_5$ scalar component (odd, Dirichlet).}{(iii) A5 scalar component (odd, Dirichlet).}}
\begin{itemize}[nosep]
    \item Parity: odd ($-$) \tagBL{}
    \item BC: Dirichlet ($f = 0$ at fixed points)
    \item Eigenvalue: $x_1 = \pi$ (ground state for odd field)
    \item Mass: $m_\phi = \pi/\ell \approx 70$ GeV
    \item Profile: $A_5^{(1)}(\xi) \propto \sin(\pi \xi/\ell)$ — \textbf{ZERO at boundaries}
    \item Coupling: \textbf{Suppressed} (sine vanishes where LH fermions peak)
\end{itemize}
\textbf{Verdict:} \textcolor{BrickRed}{\textbf{DISFAVORED.}} Although the mass
is correct, the profile vanishes exactly where left-handed fermions are localized.
This creates a \emph{structural mismatch} between mediator and fermion distributions.

\begin{tcolorbox}[colback=yellow!5!white, colframe=yellow!60!black,
    title=\textbf{$A_5$ Coupling Suppression [Dc]}]
The $A_5$ mode has profile $\sin(\pi \xi/\ell)$ which satisfies:
\[
    A_5(0) = A_5(\ell) = 0 \quad \text{(Dirichlet BC)}
\]
Since left-handed fermions are localized at $\xi \approx 0$ (Ch.~9), the overlap
integral~\eqref{eq:attemptH1_overlap} is suppressed:
\[
    g_{\text{eff}}^{(A_5)} \propto \int |f_L(\xi)|^2 \sin^2(\pi \xi/\ell)\, d\xi
    \approx 0 \quad \text{(for boundary-peaked } f_L\text{)}
\]
\textbf{Conclusion:} $A_5$ is structurally disfavored as the weak mediator
in EDC's boundary-localized picture.
\end{tcolorbox}

\paragraph{(iv) Brane-localized scalar mediator (Robin BC).}
\begin{itemize}[nosep]
    \item Parity: not directly applicable (localized at brane)
    \item BC: Robin ($f' + \alpha f = 0$) from junction physics
    \item Eigenvalue: $x_1 \approx 2.4$ (for $\alpha = 2\pi$, Attempt~H)
    \item Mass: $m_\phi \approx 54$ GeV
    \item Profile: boundary-localized with decay into bulk
    \item Coupling: \textbf{$\mathcal{O}(1)$} (mediator and fermions both boundary-peaked)
\end{itemize}
\textbf{Verdict:} \textcolor{OliveGreen}{\textbf{VIABLE.}} Consistent with
brane-localized SU(2)$_L$ picture (OPR-17). Mass is 33\% below $M_W$, which
is within dimensional analysis uncertainty.

\paragraph{(v) Mixed/junction-induced mode.}
This is a variant of (iv) where the Robin BC arises from specific junction
microphysics (e.g., BKT surface terms). The analysis is subsumed by (iv).

% ------------------------------------------------------------------------------
\subsubsection{H1.4: Decision Table}
\label{sec:attemptH1_decision}

\begin{table}[ht]
\centering
\caption{Mediator candidate analysis for OPR-20a}
\label{tab:mediator_candidates}
\small
\begin{tabular}{lcccccc}
\toprule
\textbf{Candidate} & \textbf{Parity} & \textbf{BC} & \textbf{$x_1$} &
    \textbf{$m_\phi$} & \textbf{LH Overlap} & \textbf{Status} \\
\midrule
(i) $A_\mu$ zero-mode & Even & NN & 0 & 0 & $\mathcal{O}(1)$ & \textcolor{BrickRed}{REJECTED} \\
(ii) $A_\mu$ KK ($n=1$) & Even & NN & $\pi$ & 70 GeV & $\mathcal{O}(1)$ & \textcolor{OliveGreen}{VIABLE} \\
(iii) $A_5$ scalar & Odd & DD & $\pi$ & 70 GeV & Suppressed & \textcolor{BrickRed}{DISFAVORED} \\
(iv) Brane scalar & — & Robin & $\sim$2.4 & 54 GeV & $\mathcal{O}(1)$ & \textcolor{OliveGreen}{VIABLE} \\
\bottomrule
\end{tabular}
\end{table}

\paragraph{Shortlist.}
The structural analysis identifies two viable candidates:
\begin{enumerate}
    \item \textbf{KK $A_\mu$ ($n=1$):} $x_1 = \pi$, $m_\phi \approx 70$ GeV (12\% below $M_W$)
    \item \textbf{Brane-localized scalar:} $x_1 \approx 2.4$, $m_\phi \approx 54$ GeV (33\% below $M_W$)
\end{enumerate}

\paragraph{What was ruled out.}
\begin{itemize}[nosep]
    \item $A_\mu$ zero-mode: massless (needs Higgs)
    \item $A_5$ scalar: coupling suppressed at boundary (structural mismatch)
\end{itemize}

% ------------------------------------------------------------------------------
\subsubsection{H1.5: Discriminants Between Viable Candidates}
\label{sec:attemptH1_discriminants}

The two viable candidates differ in their physical picture and predictions:

\begin{table}[ht]
\centering
\caption{Discriminants between viable mediator candidates}
\label{tab:discriminants}
\small
\begin{tabular}{lcc}
\toprule
\textbf{Criterion} & \textbf{(ii) KK $A_\mu$} & \textbf{(iv) Brane Scalar} \\
\midrule
Gauge propagation & Bulk (5D) & Brane-localized \\
BC type & Neumann (pure) & Robin (junction) \\
$x_1$ value & $\pi = 3.14$ & $\sim 2.4$ \\
$m_\phi$ prediction & $\sim$70 GeV & $\sim$54 GeV \\
M$_W$ deviation & 12\% & 33\% \\
\addlinespace
KK tower? & Yes (masses $n\pi/\ell$) & No (single mode) \\
Consistency w/ OPR-17 & Requires bulk gauge & Consistent w/ brane gauge \\
\bottomrule
\end{tabular}
\end{table}

\paragraph{Key discriminant: KK tower signature.}
If the mediator is KK $A_\mu$, there should be a \emph{tower} of KK excitations
with masses $m_n = n\pi/\ell \approx n \times 70$ GeV ($n = 1, 2, 3, \ldots$).
The brane-localized picture predicts a single effective mediator without a tower.

\paragraph{\texorpdfstring{Consistency with OPR-17 (SU(2)$_L$ embedding).}{Consistency with OPR-17 (SU(2)L embedding).}}
The brane-localized SU(2)$_L$ embedding (\S\ref{sec:ch9_su2_embedding}) assumes
gauge fields do not propagate in bulk. This favors candidate (iv) over (ii).
However, OPR-17 itself is [P], so this constraint is not derived.

\begin{tcolorbox}[colback=blue!5!white, colframe=blue!50!black,
    title=\textbf{Future Discriminant: KK Tower vs Single Mediator}]
\textbf{If bulk gauge (ii):}
\begin{itemize}[nosep]
    \item Expect KK tower: $m_1 \approx 70$ GeV, $m_2 \approx 140$ GeV, etc.
    \item Phenomenologically: resonances in high-energy scattering
\end{itemize}

\textbf{If brane-localized (iv):}
\begin{itemize}[nosep]
    \item Single effective mediator, no KK tower
    \item Mass set by junction/BKT physics
\end{itemize}

\textbf{Observable:} LHC/future collider searches for KK gauge boson resonances.
Absence of tower would favor (iv); presence would favor (ii).
\end{tcolorbox}

% ------------------------------------------------------------------------------
\subsubsection{H1.6: Epistemic Summary}
\label{sec:attemptH1_epistemic}

\begin{tcolorbox}[colback=gray!5!white, colframe=gray!60!black,
    title=\textbf{OPR-20a Epistemic Ledger}]
\begin{center}
\small
\begin{tabular}{lll}
\toprule
\textbf{Item} & \textbf{Status} & \textbf{Note} \\
\midrule
5D gauge decomposition ($A_\mu$, $A_5$) & \tagBL{} & Standard orbifold physics \\
Parity $\to$ BC mapping & \tagBL{} & Even=Neumann, Odd=Dirichlet \\
KK spectrum formula & \tagDc{} & $x_n = n\pi$ for pure BC \\
LH fermion localization & \tagDc{} & From Ch.9 \\
\addlinespace
$A_5$ coupling suppression & \tagDc{} & Profile vanishes at boundary \\
Shortlist: (ii) or (iv) & \tagP{} & Structural analysis \\
Single choice & \textbf{[OPEN]} & Requires KK tower test or OPR-17 closure \\
\bottomrule
\end{tabular}
\end{center}
\end{tcolorbox}

\begin{tcolorbox}[
    colback=yellow!5!white,
    colframe=yellow!60!black,
    title=\textbf{OPR-20a Attempt H1: Stoplight Verdict}
]
\begin{center}
\textbf{\large YELLOW [Dc]+[P] (Shortlist Established, Discriminant Identified)}
\end{center}

\medskip
\textbf{What improved:}
\begin{itemize}[nosep]
    \item Enumerated all mediator candidates systematically
    \item Ruled out $A_5$ on structural grounds (coupling suppression) \tagDc{}
    \item Ruled out $A_\mu$ zero-mode (massless)
    \item Identified two viable candidates with explicit discriminants
\end{itemize}

\textbf{What remains:}
\begin{itemize}[nosep]
    \item Single mediator choice not determined
    \item Depends on OPR-17 (brane vs bulk gauge) or phenomenological test
\end{itemize}

\textbf{Upgrade condition:}
\begin{quote}
OPR-20a $\to$ \textbf{GREEN [Dc]} if:\\
(a) OPR-17 is closed with brane-localized SU(2)$_L$ $\Rightarrow$ candidate (iv), or\\
(b) OPR-17 is closed with bulk gauge $\Rightarrow$ candidate (ii), or\\
(c) Phenomenological evidence (KK tower presence/absence) discriminates.
\end{quote}
\end{tcolorbox}

% ------------------------------------------------------------------------------
\subsubsection{H1.7: Recommended Baseline and Future Work}
\label{sec:attemptH1_baseline}

\paragraph{Recommended baseline [P].}
Given the current state:
\begin{itemize}[nosep]
    \item \textbf{Conservative:} $x_1 = \pi$ (KK interpretation), $m_\phi \approx 70$ GeV
    \item \textbf{Rationale:} Closer to $M_W = 80.4$ GeV (12\% vs 33\%)
    \item \textbf{Caveat:} If OPR-17 settles on brane-localized SU(2)$_L$, switch to Robin baseline
\end{itemize}

\paragraph{Impact on OPR-20b.}
The boundary condition choice affects OPR-20b ($\alpha$ provenance):
\begin{itemize}[nosep]
    \item If candidate (ii): pure Neumann, no $\alpha$ needed (OPR-20b moot)
    \item If candidate (iv): Robin BC, $\alpha$ provenance is essential
\end{itemize}

\paragraph{Future work.}
\begin{enumerate}[nosep]
    \item Close OPR-17: derive or firmly postulate brane-localized vs bulk SU(2)$_L$
    \item If Robin: close OPR-20b ($\delta = R_\xi$ derivation, Attempt~H)
    \item Phenomenological check: KK tower signatures in precision electroweak data
\end{enumerate}
