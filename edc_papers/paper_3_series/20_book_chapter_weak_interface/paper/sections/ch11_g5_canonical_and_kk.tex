%!TEX root = ../EDC_Part_II_Weak_Sector.tex
% ==============================================================================
% Chapter 11: Canonical g_5 Normalization and KK Spectrum Tightening
% Status: Tightens OPR-19/20 closure path (no numerics yet)
% ==============================================================================

\subsection{Canonical \texorpdfstring{$g_5$}{g5} Normalization and KK Spectrum}
\label{sec:ch11_g5_kk}

This subsection tightens the $G_F$ derivation chain by:
\begin{enumerate}[nosep]
    \item Deriving the canonical normalization of the 5D gauge coupling $g_5$
    \item Establishing the KK eigenvalue structure for the mediator mass $m_\phi$
\end{enumerate}
The goal is to \emph{remove ambiguity}, not to close numerics. Both OPR-19 and OPR-20
remain RED-C, but with a mathematically concrete closure path.

% ------------------------------------------------------------------------------
\subsubsection{Canonical $g_5$ Normalization from 5D Action}
\label{sec:ch11_g5_canonical}

\paragraph{Starting point: 5D gauge action.}
Consider a gauge field $A_M$ propagating in five dimensions with action \tagP{}:
\begin{equation}
    S_{\text{5D}} = -\frac{1}{4g_5^2} \int d^4x \int_0^\ell d\xi \; F_{MN} F^{MN}
    \label{eq:ch11_5d_gauge_action}
\end{equation}
where $M,N \in \{0,1,2,3,5\}$, the extra dimension $\xi \in [0, \ell]$, and $g_5$ is the
5D gauge coupling with dimension $[g_5] = [E]^{-1/2}$ in natural units.

\paragraph{Alternative convention.}
Some authors absorb $g_5^{-2}$ into the field normalization. We use the convention
above because it makes the coupling explicit. The physics is identical.

\paragraph{KK decomposition.}
Decompose the 4D gauge field component as:
\begin{equation}
    A_\mu(x,\xi) = \sum_{n=0}^\infty A_\mu^{(n)}(x) \, \chi_n(\xi)
    \label{eq:ch11_kk_decomp}
\end{equation}
where $\{\chi_n(\xi)\}$ are orthonormal mode functions satisfying:
\begin{equation}
    \int_0^\ell d\xi \; \chi_m(\xi) \chi_n(\xi) = \delta_{mn}
    \label{eq:ch11_chi_orthonorm}
\end{equation}

\paragraph{4D effective action.}
Substituting into~\eqref{eq:ch11_5d_gauge_action} and using orthonormality:
\begin{equation}
    S_{\text{4D}} = -\frac{1}{4g_5^2} \sum_n \int d^4x \; F_{\mu\nu}^{(n)} F^{(n)\mu\nu}
    \label{eq:ch11_4d_effective}
\end{equation}
For canonical 4D kinetic terms $-\frac{1}{4} F_{\mu\nu}^{(n)} F^{(n)\mu\nu}$, we identify:
\begin{equation}
    \boxed{
    g_4^2 = g_5^2
    }
    \label{eq:ch11_g4_g5_relation}
\end{equation}
This is \emph{not} a typo: with the normalization convention~\eqref{eq:ch11_5d_gauge_action}
and orthonormal modes~\eqref{eq:ch11_chi_orthonorm}, the 4D and 5D couplings are numerically equal.

\paragraph{Where the extra dimension enters.}
The 5D nature enters through:
\begin{enumerate}[nosep]
    \item The mode spectrum $m_n$ (eigenvalues of KK equation)
    \item The overlap integrals with fermion profiles (coupling strengths)
    \item The brane kinetic terms (if present), which modify the zero-mode coupling
\end{enumerate}

\begin{tcolorbox}[colback=blue!5, colframe=blue!50!black,
    title=\textbf{Dimensional Sanity: $g_5$ and $g_4$}]
\begin{align}
    [g_5] &= [E]^{-1/2} \quad \text{(5D coupling)} \label{eq:ch11_dim_g5} \\
    [g_4] &= [E]^0 \quad \text{(dimensionless 4D coupling)} \label{eq:ch11_dim_g4} \\
    [G_F] &= [E]^{-2} \quad \text{(Fermi constant)} \label{eq:ch11_dim_GF}
\end{align}
\textbf{Consistency:} With orthonormal modes, $g_4 = g_5$ (numerically).
The mode normalization absorbs the factor of $\ell$.

\medskip
\textbf{Alternative:} If modes are normalized as $\int d\xi \, \chi^2 = \ell$, then
$g_4 = g_5/\sqrt{\ell}$. Both conventions give the same physics.
\end{tcolorbox}

\paragraph{Brane kinetic terms (optional extension).}
If brane-localized gauge kinetic terms are present \tagP{}:
\begin{equation}
    S_{\text{brane}} = -\frac{\kappa}{4} \int d^4x \; F_{\mu\nu} F^{\mu\nu} \Big|_{\xi = 0}
    \label{eq:ch11_brane_kinetic}
\end{equation}
then the effective 4D coupling is modified:
\begin{equation}
    \frac{1}{g_{\text{eff}}^2} = \frac{1}{g_5^2} + \kappa
    \quad\Rightarrow\quad
    g_{\text{eff}} \simeq g_5 \quad \text{for } \kappa \ll g_5^{-2}
    \label{eq:ch11_geff_brane}
\end{equation}
The brane kinetic term $\kappa$ is currently \textbf{[OPEN]} and not derived.

% ------------------------------------------------------------------------------
\subsubsection{KK Spectrum and Mediator Mass $m_\phi$}
\label{sec:ch11_kk_spectrum}

\paragraph{The eigenvalue problem.}
The KK mode functions $\chi_n(\xi)$ satisfy the eigenvalue equation:
\begin{equation}
    -\partial_\xi^2 \chi_n(\xi) = m_n^2 \chi_n(\xi)
    \label{eq:ch11_kk_eigenvalue}
\end{equation}
subject to boundary conditions at $\xi = 0$ and $\xi=\ell$.

\paragraph{Boundary conditions and spectrum.}
Three canonical choices yield different spectra:

\begin{center}
\begin{tabular}{llcc}
\toprule
\textbf{BC Type} & \textbf{Conditions} & \textbf{Zero Mode?} & \textbf{Spectrum $m_n$} \\
\midrule
Neumann--Neumann & $\chi'(0) = \chi'(\ell) = 0$ & Yes & $n\pi/\ell$ \\
Dirichlet--Dirichlet & $\chi(0) = \chi(\ell) = 0$ & No & $(n+1)\pi/\ell$ \\
Mixed (N--D) & $\chi'(0) = 0$, $\chi(\ell) = 0$ & No & $(n+\tfrac{1}{2})\pi/\ell$ \\
\bottomrule
\end{tabular}
\end{center}

\paragraph{The mediator mass scale.}
The first massive mode (or zero mode if present) sets the mediator mass:
\begin{equation}
    \boxed{
    m_\phi = \frac{x_1}{\ell}
    }
    \label{eq:ch11_mphi_scale}
\end{equation}
where $x_1$ is a dimensionless constant depending on boundary conditions:
\begin{itemize}[nosep]
    \item $x_1 = 0$ for N--N zero mode (massless)
    \item $x_1 = \pi$ for D--D or N--N first massive mode
    \item $x_1 = \pi/2$ for mixed BC first mode
\end{itemize}

\paragraph{Identification vs.\ derivation.}
The identification $m_\phi \sim M_W \approx 80$ GeV is currently \tagI{}:
\begin{equation}
    m_\phi \sim M_W \quad\Rightarrow\quad \ell \sim \frac{\pi}{M_W} \approx 0.04 \text{ fm}
    \label{eq:ch11_ell_estimate}
\end{equation}
This is \textbf{not} derived from EDC first principles. The derivation would require:
\begin{enumerate}[nosep]
    \item Deriving the brane layer thickness $\ell$ from membrane tension $\sigma$
    \item Specifying the boundary conditions from physical principles
    \item Computing $x_1$ for the actual BC configuration
\end{enumerate}

\begin{tcolorbox}[colback=yellow!5!white, colframe=yellow!60!black,
    title=\textbf{KK Spectrum: What Is [Dc] vs.\ [I] vs.\ [OPEN]}]
\begin{description}[nosep, font=\normalfont\bfseries]
    \item[[Dc]:] The form $m_\phi = x_1/\ell$ from KK eigenvalue equation
    \item[[I]:] The numerical value $m_\phi \sim M_W$ (calibration, not derivation)
    \item[[OPEN]:] The boundary conditions (N/D/mixed) and brane thickness $\ell$
\end{description}

\medskip
\noindent To upgrade [I] $\to$ [Dc]: derive $\ell$ from $\sigma$ and BCs from physics.
\end{tcolorbox}

% ------------------------------------------------------------------------------
\subsubsection{Chain Tightening Summary}
\label{sec:ch11_chain_tightened}

\begin{tcolorbox}[colback=green!5!white, colframe=green!50!black,
    title=\textbf{Chain Map Tightened: OPR-19/20}]
\textbf{Before (§\ref{sec:ch11_sanity_skeleton}):}
\begin{itemize}[nosep]
    \item OPR-19: ``$g_5$ postulated''
    \item OPR-20: ``$m_\phi \sim M_W$ identified''
\end{itemize}

\textbf{After (this section):}
\begin{itemize}[nosep]
    \item OPR-19: Canonical normalization from 5D action gives $g_4 = g_5$
          (with orthonormal modes). Brane kinetic terms are optional [P] extension.
          \textbf{Closure path:} compute $g_5$ from underlying 5D theory.
    \item OPR-20: KK eigenvalue problem gives $m_\phi = x_1/\ell$ where $x_1$
          depends on boundary conditions. \textbf{Closure path:} derive $\ell$ from
          membrane parameters $(\sigma, r_e)$ and BCs from physics.
\end{itemize}

\medskip
\noindent\fbox{\parbox{0.92\textwidth}{\small
\textbf{Status:} OPR-19/20 remain RED-C but with mathematically concrete closure paths.
The ``SM-help'' impression is reduced: we now have explicit derivation spines,
not just dimensional arguments.}}
\end{tcolorbox}

% ------------------------------------------------------------------------------
\subsubsection{Updated Stoplight: OPR-19/20}
\label{sec:ch11_opr19_20_stoplight}

\begin{table}[ht]
\centering
\caption{OPR-19/20 status after chain tightening}
\label{tab:ch11_opr19_20_status}
\small
\begin{tabular}{clcl}
\toprule
\textbf{OPR} & \textbf{Item} & \textbf{Status} & \textbf{Notes} \\
\midrule
19 & $g_5$ canonical normalization & RED-C & $g_4 = g_5$ [Dc]; $g_5$ value [OPEN] \\
20 & $m_\phi$ KK spectrum & RED-C & $m_\phi = x_1/\ell$ [Dc]; $\ell$, BC [OPEN] \\
\bottomrule
\end{tabular}
\end{table}

\paragraph{What ``RED-C'' means.}
The status remains RED (not derived from first principles), but the ``C'' indicates
a \emph{concrete closure path} is now defined:
\begin{itemize}[nosep]
    \item OPR-19: Need underlying 5D gauge theory to fix $g_5$ value
    \item OPR-20: Need $\ell$ from membrane physics + BCs from consistency
\end{itemize}

\paragraph{Improvement over previous state.}
Before this section, OPR-19/20 were ``open with dimensional argument.''
Now they are ``open with derivation spine.'' The mathematical structure is explicit;
only the physical inputs $(\ell, \text{BC})$ remain to be derived.

