%!TEX root = ../EDC_Part_II_Weak_Sector_rebuild.tex
% ==============================================================================
% OPR-21 BVP CLOSURE PACK
% The master key: precise specification + closure criteria + failure modes
% Status: Infrastructure specification (no numerical claims)
% ==============================================================================

\section{OPR-21: The BVP as Master Key}
\label{sec:bvp_master_key}

% ------------------------------------------------------------------------------
% EPISTEMIC STATUS
% ------------------------------------------------------------------------------

\begin{tcolorbox}[edcGuardrail, title=\textbf{Epistemic Status: Closure Pack}]
This section defines \textbf{closure criteria} for OPR-21 (the thick-brane BVP).
It does \emph{not} claim closure---it specifies what closure \emph{would mean}.

\textbf{Content type:}
\begin{itemize}[nosep]
    \item Mathematical specification: \tagM{} (pure mathematics, no physics input)
    \item Physical interpretation: \tagP{} (postulated connection to observables)
    \item Acceptance criteria: \tagDef{} (definitions, not derivations)
\end{itemize}

\textbf{NOT included:}
\begin{itemize}[nosep]
    \item Numerical solutions (deferred to implementation)
    \item Calibration to PDG values (forbidden by design)
    \item Claims about $N_{\text{gen}} = 3$ or specific $G_F$ values
\end{itemize}
\end{tcolorbox}

% ------------------------------------------------------------------------------
% FRAMEWORK 2.0 COMPLIANCE
% ------------------------------------------------------------------------------

\begin{tcolorbox}[colback=blue!3!white, colframe=blue!50!black,
    title=\textbf{Framework 2.0 Language Compliance}]
\small
\textbf{Why the BVP is the ``master key'':}
\begin{itemize}[nosep]
    \item \textbf{5D cause:} Thick-brane geometry creates an effective potential $V(z)$.
    \item \textbf{Brane process:} Schr\"odinger-like BVP determines bound states + spectrum.
    \item \textbf{3D shadow:} Eigenvalues $\to$ masses; eigenfunctions $\to$ couplings;
          bound state count $\to$ generations.
\end{itemize}

The BVP is the \emph{mathematical engine} that converts 5D postulates into 3D predictions
\emph{without} using 3D observables as input. Solving it is prerequisite for non-circular
closure of OPR-02 (generations), OPR-22 ($G_F$ spine), and CKM/PMNS overlaps.
\end{tcolorbox}

% ==============================================================================
\subsection{Why BVP is the Master Key}
\label{subsec:bvp_why_master}

Multiple open problems in Part II reduce to a single mathematical structure:
the thick-brane boundary value problem. This subsection maps the dependencies.

\paragraph{The central claim.}
If the BVP can be solved with \emph{independently fixed} parameters (from $\sigma$, $r_e$,
and membrane geometry alone), then the following become \emph{predictable}:

\begin{enumerate}[nosep]
    \item \textbf{Generation count} $N_{\text{bound}}$: number of normalizable bound states
    \item \textbf{Mass spectrum}: eigenvalue ratios (not absolute masses without scale)
    \item \textbf{Overlap integrals} $I_4$: coupling strengths from wavefunction products
    \item \textbf{Chirality suppression}: L/R asymmetry from mode localization
\end{enumerate}

\paragraph{Dependency map.}

\begin{center}
\begin{tabular}{lll}
\toprule
\textbf{Input (5D)} & \textbf{BVP Output} & \textbf{Downstream Claim} \\
\midrule
Potential $V(z)$ & Bound state count $N_{\text{bound}}$ & OPR-02: Why 3 generations? \\
Boundary conditions & Eigenvalue $x_1$ (first root) & OPR-22: $G_F$ spine scale \\
Domain $[0, \ell]$ or $[0, \infty)$ & Eigenfunctions $\psi_n(z)$ & CKM/PMNS: overlap integrals \\
Self-adjointness & Discrete spectrum guarantee & Quantization of masses \\
L/R coupling to $m(z)$ & Mode localization profiles & Ch9: V--A structure \\
\bottomrule
\end{tabular}
\end{center}

\paragraph{What closure means.}
The BVP is ``closed'' when:
\begin{enumerate}[nosep]
    \item The potential $V(z)$ is derived (not fitted) from membrane parameters
    \item The boundary conditions are derived from junction physics (not chosen ad hoc)
    \item The spectrum is computed without calibrating to PDG masses
    \item Downstream predictions match observations within stated tolerances
\end{enumerate}

% ==============================================================================
\subsection{Precise BVP Statement}
\label{subsec:bvp_precise}

% ------------------------------------------------------------------------------
% OPERATOR DEFINITION
% ------------------------------------------------------------------------------

\subsubsection{The Operator}

The thick-brane profile $f(z)$ satisfies a Schr\"odinger-type equation:
\begin{equation}
    \boxed{
    \hat{L} f \;=\; -\frac{d^2 f}{dz^2} + V(z) f \;=\; \lambda f
    }
    \label{eq:bvp:operator}
\end{equation}
where:
\begin{itemize}[nosep]
    \item $z \in \Omega$ is the extra-dimensional coordinate (domain specified below)
    \item $V(z)$ is the effective potential from membrane geometry \tagP{}
    \item $\lambda$ is the eigenvalue (related to 4D mass: $\lambda \sim m^2$)
    \item $f(z)$ is the profile function (wavefunction in extra dimension)
\end{itemize}

\paragraph{Dimensionless form.}
With $\zeta = z/\ell$ (where $\ell$ is the brane thickness scale), and
$\tilde{V}(\zeta) = \ell^2 V(\ell\zeta)$, $\tilde{\lambda} = \ell^2 \lambda$:
\begin{equation}
    -\frac{d^2 f}{d\zeta^2} + \tilde{V}(\zeta) f = \tilde{\lambda} f
    \label{eq:bvp:dimensionless}
\end{equation}
All numerical work should use this dimensionless form \tagM{}.

% ------------------------------------------------------------------------------
% DOMAIN AND INNER PRODUCT
% ------------------------------------------------------------------------------

\subsubsection{Domain and Inner Product}

\paragraph{Domain options.}
The physical domain $\Omega$ depends on brane topology:
\begin{center}
\begin{tabular}{lll}
\toprule
\textbf{Domain} & \textbf{Physical Interpretation} & \textbf{Spectrum Type} \\
\midrule
$[0, \ell]$ (finite interval) & Finite brane thickness & Pure discrete \\
$[0, \infty)$ (half-line) & Semi-infinite bulk & Discrete + continuous \\
$(-\infty, \infty)$ (full line) & Symmetric domain wall & Depends on $V$ asymptotics \\
\bottomrule
\end{tabular}
\end{center}

For Part II, the \textbf{finite interval} $\Omega = [0, \ell]$ is the primary focus
(thick brane with defined boundaries). The half-line appears in Ch9 (chirality).

\paragraph{Inner product and normalization.}
The standard $L^2$ inner product:
\begin{equation}
    \langle f, g \rangle = \int_\Omega f(z)^* g(z) \, dz
    \label{eq:bvp:inner_product}
\end{equation}
Eigenfunctions are normalized: $\langle \psi_n, \psi_n \rangle = 1$.

For weighted problems (e.g., curved extra dimension), use $\langle f, g \rangle_w =
\int_\Omega w(z) f^* g \, dz$ with positive weight $w(z) > 0$.

% ------------------------------------------------------------------------------
% BOUNDARY CONDITIONS
% ------------------------------------------------------------------------------

\subsubsection{Boundary Conditions}
\label{subsubsec:bvp_bc}

The BVP requires boundary conditions at each endpoint. The general \textbf{Robin form}
encompasses Dirichlet and Neumann as special cases:

\begin{equation}
    \boxed{
    \alpha_0 f(0) + \beta_0 f'(0) = 0, \qquad
    \alpha_\ell f(\ell) + \beta_\ell f'(\ell) = 0
    }
    \label{eq:bvp:robin_bc}
\end{equation}
where $(\alpha_0, \beta_0)$ and $(\alpha_\ell, \beta_\ell)$ are BC parameters
(not both zero at each endpoint).

\paragraph{Special cases.}
\begin{center}
\begin{tabular}{lll}
\toprule
\textbf{Name} & \textbf{Condition} & \textbf{Physical Interpretation} \\
\midrule
Dirichlet & $f = 0$ at boundary & Hard wall / infinite barrier \\
Neumann & $f' = 0$ at boundary & Reflection symmetry / no flux \\
Robin & $\alpha f + \beta f' = 0$ & Finite barrier / junction matching \\
\bottomrule
\end{tabular}
\end{center}

\paragraph{Open problem.}
The BC parameters $(\alpha, \beta)$ should be \emph{derived} from Israel junction
conditions or brane microphysics, not chosen to fit outputs. This is part of OPR-21.

% ------------------------------------------------------------------------------
% SELF-ADJOINTNESS
% ------------------------------------------------------------------------------

\subsubsection{Self-Adjointness Conditions}
\label{subsubsec:bvp_selfadjoint}

For the operator $\hat{L}$ to have real eigenvalues and orthogonal eigenfunctions,
it must be self-adjoint on the chosen domain with the chosen BCs.

\begin{theorem}[Self-Adjointness Criterion]
\label{thm:bvp:selfadjoint}
\tagM{}
The Sturm--Liouville operator $\hat{L} = -d^2/dz^2 + V(z)$ on $[0, \ell]$ with
Robin BCs~\eqref{eq:bvp:robin_bc} is self-adjoint if and only if:
\begin{enumerate}[nosep]
    \item $V(z)$ is real-valued and locally integrable on $(0, \ell)$
    \item The BC parameters are real: $\alpha_0, \beta_0, \alpha_\ell, \beta_\ell \in \mathbb{R}$
    \item The BCs are \emph{separated} (each endpoint has its own condition)
\end{enumerate}
\end{theorem}

\begin{proof}[Proof sketch]
Standard Sturm--Liouville theory~\cite{Zettl2005,Teschl2012}. Integration by parts
shows $\langle \hat{L}f, g \rangle - \langle f, \hat{L}g \rangle = [f^*g' - f'^*g]_0^\ell$,
which vanishes under separated real Robin BCs. \qed
\end{proof}

\paragraph{Consequence.}
Under self-adjointness:
\begin{itemize}[nosep]
    \item Eigenvalues $\lambda_n$ are real and form a discrete sequence (if $V$ confining)
    \item Eigenfunctions $\psi_n$ are orthogonal: $\langle \psi_m, \psi_n \rangle = \delta_{mn}$
    \item The spectrum is bounded below: $\exists \lambda_0 = \min_n \lambda_n$
\end{itemize}

% ==============================================================================
\subsection{From 5D Action to Effective Potential: What Must Be Derived}
\label{subsec:bvp_5d_skeleton}

This subsection provides a formal \textbf{skeleton} for deriving the effective
potential $V(z)$ and boundary conditions from the 5D action. This is the key
\tagOPEN{} step required for OPR-21 closure.

\begin{tcolorbox}[colback=yellow!5!white, colframe=yellow!60!black,
    title=\textbf{Epistemic Status: Derivation Skeleton}]
\label{box:vz_skeleton_status}

\textbf{Classification:} \tagOPEN{} (roadmap, not completed derivation)

\textbf{What this section provides:}
\begin{itemize}[nosep]
    \item Identification of required inputs from 5D physics
    \item Formal pipeline: action $\to$ variation $\to$ operator $\to$ V(z)
    \item Specification of what closure would require
\end{itemize}

\textbf{What this section does NOT provide:}
\begin{itemize}[nosep]
    \item Explicit form of $V(z)$ (requires completing the pipeline)
    \item Numerical values (deferred to implementation)
    \item Claims about spectrum or generation count
\end{itemize}
\end{tcolorbox}

\subsubsection{Required Inputs from 5D Physics}

The derivation of $V(z)$ requires combining several contributions from the 5D theory:

\paragraph{Bulk action.}
The bulk contribution includes the 5D Einstein--Hilbert term and any matter fields
propagating in the bulk:
\begin{equation}
    S_{\text{bulk}} = \int d^5x \sqrt{-g_5} \left[
        \frac{M_5^3}{2} R_5 + \mathcal{L}_{\text{bulk matter}}
    \right]
    \label{eq:bvp:bulk_action_skeleton}
\end{equation}
where $M_5$ is the 5D Planck mass and $R_5$ is the 5D Ricci scalar. \tagP{}

\paragraph{Brane action.}
The 3-brane (membrane) contributes tension and localized matter terms:
\begin{equation}
    S_{\text{brane}} = -\int d^4x \sqrt{-g_4} \left[
        \sigma + \mathcal{L}_{\text{brane matter}}
    \right]
    \label{eq:bvp:brane_action_skeleton}
\end{equation}
where $\sigma$ is the brane tension and $g_4$ is the induced metric. The brane
tension is related to the membrane surface tension from Part~I. \tagP{}

\paragraph{Junction terms (Gibbons--Hawking--York + Israel).}
For a brane at position $z = z_0$, the Israel junction conditions
relate the jump in extrinsic curvature $K_{ab}$ to the brane stress-energy:
\begin{equation}
    [K_{ab}] - g_{ab}[K] = -\frac{1}{M_5^3} S_{ab}
    \label{eq:bvp:israel_junction}
\end{equation}
where $S_{ab}$ is the brane stress-energy tensor and $[X] \equiv X^+ - X^-$ denotes
the jump across the brane~\cite{RandallSundrum1999a,RubakovShaposhnikov1983}.

The Gibbons--Hawking--York boundary term ensures a well-posed variational problem:
\begin{equation}
    S_{\text{GHY}} = M_5^3 \int_{\partial\mathcal{M}} d^4x \sqrt{-h}\, K
    \label{eq:bvp:ghy_term}
\end{equation}
These junction terms are essential for deriving BCs from the action. \tagM{}

\subsubsection{The Derivation Pipeline}
\label{subsubsec:vz_pipeline}

The formal pipeline from 5D action to effective 1D BVP:

\begin{enumerate}
    \item \textbf{Background solution:} Solve the Einstein equations with brane
          sources to obtain the warped metric $ds^2 = e^{2A(z)}\eta_{\mu\nu}dx^\mu dx^\nu + dz^2$.
          The warp factor $A(z)$ encodes the brane structure.

    \item \textbf{Perturbation ansatz:} Expand bulk fields around the background,
          separating 4D and extra-dimensional dependence:
          $\phi(x^\mu, z) = \sum_n \phi_n(x^\mu) f_n(z)$.

    \item \textbf{Dimensional reduction:} Integrate over the extra dimension to
          obtain an effective 4D theory. The profile functions $f_n(z)$ satisfy
          a Sturm--Liouville equation.

    \item \textbf{Identify operator and potential:} The resulting equation takes
          the form of Eq.~(\ref{eq:bvp:operator}) with:
          \begin{equation}
              V(z) = V_{\text{warp}}(z) + V_{\text{mass}}(z) + V_{\text{coupling}}(z)
              \label{eq:bvp:vz_structure}
          \end{equation}
          where each term derives from specific physics:
          \begin{itemize}[nosep]
              \item $V_{\text{warp}}$: from warp factor derivatives ($A''$, $(A')^2$)
              \item $V_{\text{mass}}$: from bulk mass terms
              \item $V_{\text{coupling}}$: from brane-localized couplings
          \end{itemize}

    \item \textbf{Derive boundary conditions:} The Israel junction conditions
          and variational principle at the brane yield Robin-type BCs with
          parameters determined by brane microphysics (not chosen ad hoc).
\end{enumerate}

\begin{tcolorbox}[colback=red!5!white, colframe=red!50!black,
    title=\textbf{OPR-21 Closure Condition: V(z) Derivation}]
\label{box:opr21_vz_closure}

\textbf{Status:} \tagOPEN{}

\textbf{Closure requires:}
\begin{enumerate}[nosep]
    \item Complete steps 1--5 above with explicit calculations
    \item Express $V(z)$ in terms of membrane parameters $(\sigma, r_e, R_\xi)$
    \item Derive BC parameters from Israel conditions (not fit to output)
    \item Show result is self-adjoint (Theorem~\ref{thm:bvp:selfadjoint})
\end{enumerate}

\textbf{Downstream unlocks (if completed):}
\begin{itemize}[nosep]
    \item OPR-02: Can compute $N_{\text{bound}}$ for physical potential
    \item OPR-22: Can compute $x_1$ for $G_F$ spine
    \item CKM/PMNS: Can compute overlap integrals from eigenfunctions
\end{itemize}

\textbf{Without this derivation:} BVP analysis remains at the ``toy potential''
level (cf.\ \S\ref{subsec:bvp_numerical_demo}).
\end{tcolorbox}

\subsubsection{Connection to Part I Membrane Physics}

The membrane parameters from Part~I provide the physical inputs:
\begin{center}
\begin{tabular}{lll}
\toprule
\textbf{Part I Parameter} & \textbf{Role in V(z)} & \textbf{Status} \\
\midrule
$\sigma$ (surface tension) & Sets energy scale of brane & \tagDc{} (Part I) \\
$r_e$ (electron radius) & Characteristic length scale & \tagDc{} (Part I) \\
$R_\xi$ (diffusion scale) & Boundary layer thickness & \tagP{} (Part I) \\
\bottomrule
\end{tabular}
\end{center}

\paragraph{Key constraint.}
The derivation must \emph{not} use 3D observables (PDG masses, $M_W$, $G_F$) to
fix $V(z)$. All parameters must trace to membrane physics. This is the
``no-smuggling'' requirement of Framework~2.0.

% ==============================================================================
\subsection{Output Objects: What We Need}
\label{subsec:bvp_outputs}

The BVP solution provides specific mathematical objects used by downstream chapters.

% ------------------------------------------------------------------------------
% EIGENFUNCTIONS
% ------------------------------------------------------------------------------

\subsubsection{Eigenfunctions $\psi_n(z)$}
\label{subsubsec:bvp_eigenfunctions}

\begin{definition}[Normalized Eigenfunctions]
\label{def:bvp:eigenfunction}
\tagDef{}
The $n$-th eigenfunction $\psi_n(z)$ satisfies:
\begin{enumerate}[nosep]
    \item $\hat{L}\psi_n = \lambda_n \psi_n$ (eigenvalue equation)
    \item $\langle \psi_n, \psi_n \rangle = 1$ (unit normalization)
    \item BCs~\eqref{eq:bvp:robin_bc} at both endpoints
\end{enumerate}
Ordering convention: $\lambda_0 < \lambda_1 < \lambda_2 < \cdots$ (ground state first).
\end{definition}

\paragraph{Physical interpretation \tagP{}.}
$|\psi_n(z)|^2$ gives the probability density for finding the $n$-th mode at depth $z$.
Modes localized near $z=0$ couple strongly to brane-localized interactions.

% ------------------------------------------------------------------------------
% FIRST EIGENVALUE
% ------------------------------------------------------------------------------

\subsubsection{First Eigenvalue $x_1$}
\label{subsubsec:bvp_x1}

\begin{definition}[First Root / Ground State Eigenvalue]
\label{def:bvp:x1}
\tagDef{}
The quantity $x_1$ is defined as the \textbf{lowest positive eigenvalue} of the
dimensionless BVP~\eqref{eq:bvp:dimensionless}:
\begin{equation}
    x_1 = \min\{\tilde{\lambda}_n : \tilde{\lambda}_n > 0\}
    \label{eq:bvp:x1_def}
\end{equation}
If the ground state has $\tilde{\lambda}_0 \leq 0$, then $x_1 = \tilde{\lambda}_1$.
\end{definition}

\paragraph{Role in $G_F$ spine.}
In Ch13, $x_1$ sets the scale for effective couplings: $G_F \propto 1/(x_1 \cdot \text{scale}^2)$.
The value of $x_1$ depends on $V(\zeta)$ and BCs---it is \emph{output}, not input.

% ------------------------------------------------------------------------------
% OVERLAP INTEGRALS
% ------------------------------------------------------------------------------

\subsubsection{Overlap Integrals $I_4$}
\label{subsubsec:bvp_overlap}

\begin{definition}[Four-Point Overlap Integral]
\label{def:bvp:I4}
\tagDef{}
The overlap integral $I_4$ for modes $(i, j, k, l)$ is:
\begin{equation}
    I_4^{(ijkl)} = \int_\Omega \psi_i(z) \psi_j(z) \psi_k(z) \psi_l(z) \, dz
    \label{eq:bvp:I4_def}
\end{equation}
For the dominant (ground state) contribution:
\begin{equation}
    I_4 \equiv I_4^{(0000)} = \int_\Omega |\psi_0(z)|^4 \, dz
    \label{eq:bvp:I4_ground}
\end{equation}
\end{definition}

\paragraph{Physical interpretation \tagP{}.}
$I_4$ measures the ``concentration'' of the ground state profile. For a mode spread
uniformly over $[0, \ell]$, $I_4 \sim 1/\ell$. For a strongly localized mode
(width $w \ll \ell$), $I_4 \sim 1/w \gg 1/\ell$.

\paragraph{Role in couplings.}
Effective 4D couplings scale as $g_{\text{eff}} \propto g_5 \cdot \sqrt{I_4}$ where
$g_5$ is the 5D coupling. Stronger localization $\to$ larger effective coupling.

% ------------------------------------------------------------------------------
% GENERATION COUNT (OPR-02 DETAILED)
% ------------------------------------------------------------------------------

\subsubsection{Generation Count $N_{\text{bound}}$ (OPR-02)}
\label{subsubsec:bvp_ngen}

This subsubsection provides the \textbf{precise criterion} for generation counting
from BVP spectral theory, moving beyond the heuristic ``$\mathbb{Z}_6/\mathbb{Z}_2 = \mathbb{Z}_3$''
to a rigorous bound-state count.

\begin{definition}[Bound State Count]
\label{def:bvp:nbound}
\tagDef{}
The generation count $N_{\text{bound}}$ is defined as the number of \textbf{normalizable
bound states} below a threshold $\lambda_{\text{th}}$:
\begin{equation}
    N_{\text{bound}} = \#\{n : \lambda_n < \lambda_{\text{th}} \text{ and }
    \psi_n \in L^2(\Omega)\}
    \label{eq:bvp:nbound_def}
\end{equation}
\end{definition}

% --- THRESHOLD DEFINITION ---

\paragraph{Threshold definition (essential spectrum onset).}
The threshold $\lambda_{\text{th}}$ must be defined \emph{intrinsically} from the BVP,
not from observed particle masses. The canonical choice is:

\begin{definition}[Essential Spectrum Threshold]
\label{def:bvp:threshold}
\tagDef{}
For a Schr\"odinger operator $\hat{L} = -d^2/dz^2 + V(z)$ on domain $\Omega$:
\begin{equation}
    \lambda_{\text{th}} = \inf \sigma_{\text{ess}}(\hat{L})
    \label{eq:bvp:threshold_def}
\end{equation}
where $\sigma_{\text{ess}}$ is the essential spectrum. For:
\begin{itemize}[nosep]
    \item \textbf{Finite interval} $[0, \ell]$: $\sigma_{\text{ess}} = \emptyset$
          (pure point spectrum); use $\lambda_{\text{th}} = V_{\max}$ or gap criterion.
    \item \textbf{Half-line} $[0, \infty)$: $\sigma_{\text{ess}} = [V_\infty, \infty)$
          where $V_\infty = \lim_{z\to\infty} V(z)$.
\end{itemize}
\end{definition}

\paragraph{Gap criterion (finite interval).}
On a finite interval, all eigenvalues are discrete. The ``generation threshold''
is then defined by a \textbf{spectral gap}:
\begin{equation}
    \lambda_{\text{th}} = \lambda_k \quad \text{where} \quad
    \frac{\lambda_{k+1} - \lambda_k}{\lambda_k - \lambda_{k-1}} \gg 1
    \label{eq:bvp:gap_criterion}
\end{equation}
This identifies a natural separation scale without using 3D mass inputs.

% --- ROBUSTNESS UNDER BC CLASS ---

\paragraph{Robustness under BC class.}
\label{para:bvp_bc_robustness}

A key requirement for OPR-02 closure is that $N_{\text{bound}}$ be \emph{robust}---not
sensitive to precise BC parameter choices within a physically admissible family.

\begin{definition}[Admissible BC Family]
\label{def:bvp:admissible_bc}
\tagDef{}
The \textbf{admissible BC family} $\mathcal{B}$ consists of all Robin-type BCs
\begin{equation}
    \alpha_0 f(0) + \beta_0 f'(0) = 0, \quad
    \alpha_\ell f(\ell) + \beta_\ell f'(\ell) = 0
    \label{eq:bvp:admissible_bc}
\end{equation}
satisfying:
\begin{enumerate}[nosep]
    \item \textbf{Self-adjointness:} $\alpha_j, \beta_j \in \mathbb{R}$, separated
    \item \textbf{Non-degeneracy:} $(\alpha_j, \beta_j) \neq (0, 0)$ at each endpoint
    \item \textbf{Physical bounds:} $|\alpha_j/\beta_j| \in [0, \infty]$
          (includes Dirichlet $\beta = 0$ and Neumann $\alpha = 0$)
\end{enumerate}
\end{definition}

\begin{lemma}[Spectral Stability under BC Deformation]
\label{lem:bvp:spectral_stability}
\tagM{}
Let $\hat{L}$ be a Sturm--Liouville operator with potential $V(z)$ on $[0, \ell]$.
For BCs in the admissible family $\mathcal{B}$:
\begin{enumerate}[nosep]
    \item Eigenvalues $\lambda_n$ depend \textbf{continuously} on BC parameters
          $(\alpha_0, \beta_0, \alpha_\ell, \beta_\ell)$.
    \item The count $N_{\text{bound}}$ is \textbf{locally constant} in $\mathcal{B}$
          except at \emph{critical points} where an eigenvalue crosses the threshold.
\end{enumerate}
\end{lemma}

\begin{proof}[Proof sketch]
Eigenvalue continuity follows from standard perturbation theory for self-adjoint
operators~\cite{Zettl2005}. The count $N_{\text{bound}}$ changes only when an eigenvalue
crosses $\lambda_{\text{th}}$; between crossings, eigenvalue ordering is preserved
(no level crossing for simple spectra). \qed
\end{proof}

\paragraph{Consequence for OPR-02.}
If $N_{\text{bound}} = 3$ for \emph{some} BC in the admissible family $\mathcal{B}$,
and no eigenvalue sits exactly at the threshold $\lambda_{\text{th}}$, then
$N_{\text{bound}} = 3$ persists under small BC deformations. This is the sense in which
the generation count can be ``robust.''

% --- OPR-02 CLOSURE CONDITION ---

\begin{tcolorbox}[colback=green!5!white, colframe=green!50!black,
    title=\textbf{OPR-02 Closure Condition}]
\label{box:opr02_closure}
OPR-02 (``Why exactly 3 generations?'') is \textbf{closed} when:
\begin{enumerate}[nosep]
    \item The potential $V(z)$ is \textbf{derived} from membrane parameters $(\sigma, r_e)$
    \item The admissible BC family $\mathcal{B}$ is specified from brane physics
    \item The threshold $\lambda_{\text{th}}$ is defined intrinsically (gap or essential spectrum)
    \item Numerical computation shows $N_{\text{bound}} = 3$ for the derived $V(z)$
    \item Robustness: $N_{\text{bound}} = 3$ persists over a \emph{finite measure}
          subset of $\mathcal{B}$, not just a single fine-tuned BC choice
\end{enumerate}

\textbf{Status:} \tagOPEN{} --- Conditions 1--2 require deriving $V(z)$ and BCs from
the 5D action (not yet done). Conditions 4--5 require numerical implementation.
\end{tcolorbox}

% --- ATTACK SURFACE BOX ---

\begin{tcolorbox}[colback=red!5!white, colframe=red!50!black,
    title=\textbf{OPR-02 Attack Surface: ``Why Not 2 or 4?''}]
\label{box:opr02_attack}

\textbf{What a reviewer will ask:}
\begin{itemize}[nosep]
    \item ``Your $\mathbb{Z}_6/\mathbb{Z}_2 = \mathbb{Z}_3$ is a slogan. Where is the proof that
          exactly 3 bound states exist?''
    \item ``What if $V(z)$ or BCs are slightly different? Does $N_{\text{bound}}$ jump to 2 or 4?''
    \item ``Is this tuned to give 3, or does 3 emerge generically?''
\end{itemize}

\textbf{What must be demonstrated to answer:}
\begin{enumerate}[nosep]
    \item \textbf{Spectral count:} Solve BVP with derived $V(z)$; count eigenvalues below
          threshold. Show $N_{\text{bound}} = 3$.
    \item \textbf{Robustness scan:} Vary BC parameters over admissible family $\mathcal{B}$;
          show $N_{\text{bound}} = 3$ persists over a finite (non-measure-zero) region.
    \item \textbf{No fine-tuning:} Demonstrate that the ``$N_{\text{bound}} = 3$ region''
          includes the physically motivated BC point, not just its boundary.
\end{enumerate}

\textbf{Current status:} \tagOPEN{}

The $\mathbb{Z}_6$ symmetry argument (Chapter~3) provides a \emph{structural reason}
why 3 is preferred, but does \emph{not} constitute a spectral proof. BVP computation
is required to close this gap.

\textbf{Failure modes:}
\begin{itemize}[nosep]
    \item $N_{\text{bound}} \neq 3$ for physically motivated $V(z)$ $\to$ model falsified
    \item $N_{\text{bound}} = 3$ only at a single BC point $\to$ fine-tuning criticism valid
    \item Threshold ambiguity $\to$ multiple ``generation counts'' possible
\end{itemize}
\end{tcolorbox}

% ------------------------------------------------------------------------------
% CHIRALITY CRITERION
% ------------------------------------------------------------------------------

\subsubsection{Chirality Suppression}
\label{subsubsec:bvp_chirality}

\begin{definition}[Chirality Asymmetry Ratio]
\label{def:bvp:chirality}
\tagDef{}
For left-handed ($f_L$) and right-handed ($f_R$) mode profiles (from Ch9), define:
\begin{equation}
    R_{\text{LR}} = \frac{|f_R(0)|^2}{|f_L(0)|^2}
    \label{eq:bvp:chirality_ratio}
\end{equation}
This measures the relative coupling of R-modes to the boundary.
\end{definition}

\paragraph{V--A criterion.}
Ch9 derived $R_{\text{LR}} \ll 1$ for the half-line domain with $m(z) > 0$.
On a finite interval, the suppression depends on $\ell$ and BCs.
A consistent picture requires $R_{\text{LR}} < 10^{-3}$ (experimental limit on
right-handed currents).

% ==============================================================================
\subsection{Acceptance Criteria and Closure Conditions}
\label{subsec:bvp_acceptance}

% ------------------------------------------------------------------------------
% CLOSURE TABLE
% ------------------------------------------------------------------------------

\begin{tcolorbox}[colback=green!5!white, colframe=green!50!black,
    title=\textbf{Closure Conditions by OPR Item}]
\small
\begin{center}
\begin{tabular}{p{2cm}p{5cm}p{5cm}}
\toprule
\textbf{OPR} & \textbf{Closure Criterion} & \textbf{What Must Be True} \\
\midrule
OPR-02 & $N_{\text{bound}} = 3$ & Robust under BC variations; derived $V(z)$ \\
OPR-21 & BVP fully specified & $V(z)$, BCs derived from $(\sigma, r_e)$; no fit \\
OPR-22 & $I_4$ converges & Eigenfunctions normalizable; overlap finite \\
Ch13 & $G_F$ spine predictive & $x_1$ computed; scale from membrane only \\
Ch9 & $R_{\text{LR}} < 10^{-3}$ & Chirality suppression from BVP (consistent) \\
\bottomrule
\end{tabular}
\end{center}
\end{tcolorbox}

% ------------------------------------------------------------------------------
% NUMERICAL CRITERIA
% ------------------------------------------------------------------------------

\paragraph{Numerical analysis criteria.}
For computational implementation (deferred):
\begin{enumerate}[nosep]
    \item \textbf{Convergence:} Eigenvalues stable under grid refinement ($< 0.1\%$ change
          when grid doubled)
    \item \textbf{Normalization:} $|\langle \psi_n, \psi_n \rangle - 1| < 10^{-6}$
    \item \textbf{Orthogonality:} $|\langle \psi_m, \psi_n \rangle| < 10^{-6}$ for $m \neq n$
    \item \textbf{Boundary residual:} BC satisfaction $< 10^{-8}$ (relative)
\end{enumerate}

% ==============================================================================
\subsection{Failure Modes}
\label{subsec:bvp_failure}

\begin{tcolorbox}[colback=red!5!white, colframe=red!50!black,
    title=\textbf{Failure Modes (F1--F6)}]
\small

\textbf{F1: Non-self-adjoint BCs.}
If BCs are not separated or real, eigenvalues may be complex $\to$ no physical spectrum.
\emph{Next step:} Derive BCs from junction physics ensuring self-adjointness.

\textbf{F2: Non-normalizable modes.}
If $V(z) \to 0$ too fast at boundaries, modes may not be $L^2$ $\to$ no bound states.
\emph{Next step:} Check potential asymptotics; confining $V$ required.

\textbf{F3: Numerical instability.}
Shooting methods can fail for stiff potentials or near-degenerate eigenvalues.
\emph{Next step:} Use stable algorithms (finite element, spectral methods); validate
against known analytic solutions.

\textbf{F4: Parameter sensitivity.}
If $N_{\text{bound}}$ changes drastically with small BC variations $\to$ no robust prediction.
\emph{Next step:} Map sensitivity; identify ``natural'' BC class from physics.

\textbf{F5: Threshold ambiguity.}
Different threshold definitions give different $N_{\text{bound}}$ $\to$ no unique count.
\emph{Next step:} Derive threshold from scale separation (gap in spectrum).

\textbf{F6: Circular calibration.}
If $V(z)$ or BCs are tuned to reproduce PDG masses $\to$ no predictive power.
\emph{Next step:} Derive parameters from membrane physics only; compare to PDG \emph{after}.

\end{tcolorbox}

% ==============================================================================
\subsection{Integration Pointers}
\label{subsec:bvp_pointers}

% ------------------------------------------------------------------------------
% POINTER BOXES
% ------------------------------------------------------------------------------

\begin{tcolorbox}[colback=gray!5!white, colframe=gray!50!black,
    title=\textbf{Used by: Ch12 OPR Register}]
\small
The BVP closure pack (this section) provides formal definitions for:
\begin{itemize}[nosep]
    \item OPR-02 closure criterion: $N_{\text{bound}} = 3$ (Definition~\ref{def:bvp:nbound})
    \item OPR-21 closure criterion: BVP fully derived (Sec.~\ref{subsec:bvp_precise})
    \item OPR-22 closure criterion: $I_4$ convergent (Definition~\ref{def:bvp:I4})
\end{itemize}
See Ch12 (Epistemic Landscape) for full OPR register.
\end{tcolorbox}

\begin{tcolorbox}[colback=gray!5!white, colframe=gray!50!black,
    title=\textbf{Feeds: Ch13 $G_F$ Spine}]
\small
The $G_F$ derivation pathway requires:
\begin{itemize}[nosep]
    \item First eigenvalue $x_1$ (Definition~\ref{def:bvp:x1})
    \item Overlap integral $I_4$ (Definition~\ref{def:bvp:I4})
    \item Scale from membrane: $\ell \sim r_e$ (thickness)
\end{itemize}
The spine becomes predictive when these are computed from derived $V(z)$.
\end{tcolorbox}

\begin{tcolorbox}[colback=gray!5!white, colframe=gray!50!black,
    title=\textbf{Feeds: Ch9 Chirality}]
\small
The V--A structure (Ch9) uses:
\begin{itemize}[nosep]
    \item Chirality ratio $R_{\text{LR}}$ (Definition~\ref{def:bvp:chirality})
    \item Mode localization profiles $f_{L/R}(z)$
    \item Half-line vs finite-interval consistency
\end{itemize}
Ch9 establishes V--A qualitatively; the closure pack provides quantitative criterion.
\end{tcolorbox}

% ==============================================================================
% SUMMARY BOX
% ==============================================================================

\begin{tcolorbox}[colback=yellow!5!white, colframe=yellow!60!black,
    title=\textbf{Summary: What This Closure Pack Provides}]
\begin{enumerate}[nosep]
    \item \textbf{Precise BVP statement:} Operator, domain, BCs, self-adjointness (Sec.~\ref{subsec:bvp_precise})
    \item \textbf{Output definitions:} $\psi_n$, $x_1$, $I_4$, $N_{\text{bound}}$, $R_{\text{LR}}$ (Sec.~\ref{subsec:bvp_outputs})
    \item \textbf{Closure criteria:} What each OPR needs to be ``closed'' (Sec.~\ref{subsec:bvp_acceptance})
    \item \textbf{Failure modes:} What can go wrong and what to do next (Sec.~\ref{subsec:bvp_failure})
    \item \textbf{Integration map:} How this feeds Ch9, Ch12, Ch13 (Sec.~\ref{subsec:bvp_pointers})
\end{enumerate}

\textbf{What it does NOT provide:}
\begin{itemize}[nosep]
    \item Derived $V(z)$ from membrane parameters (OPR-21 remains open)
    \item Derived BCs from junction physics (part of OPR-21)
    \item Claims that $N_{\text{bound}} = 3$ (requires physical $V(z)$)
\end{itemize}
\end{tcolorbox}

% ==============================================================================
% NUMERICAL PIPELINE DEMONSTRATION [M]/[Toy]
% ==============================================================================

\subsection{Numerical Pipeline Demo: Why $N_{\text{bound}}$ is a Closure Target, Not a Slogan}
\label{subsec:bvp_numerical_demo}

\noindent
\textbf{Purpose (what this demo is and is not).}
This section is a \emph{numerical sanity demonstration} of the BVP closure pipeline.
It is \emph{not} a claim that the toy potential below is the physical EDC potential $V(z)$.
Rather, it shows:
\begin{enumerate}[nosep]
    \item[(i)] how bound-state counting is performed computationally, and
    \item[(ii)] why the statement ``$N_{\text{gen}} = 3$'' must be treated as a
          \emph{spectral closure condition} tied to the \emph{derived} physical
          $V(z)$ and admissible BCs---not as an automatic consequence of
          group-quotient slogans like $Z_6/Z_2 = Z_3$.
\end{enumerate}

\begin{tcolorbox}[colback=gray!5!white, colframe=gray!50!black,
    title=\textbf{Epistemic Status: Toy Demonstration}]
\label{box:bvp_toy_demo}
\textbf{Classification:} \tagM{}/[Toy]

\textbf{What this is:}
\begin{itemize}[nosep]
    \item A \emph{pipeline test} showing that bound states can be counted numerically
    \item Uses a placeholder potential $V(z) = -V_0 \operatorname{sech}^2(z/a)$ (Pöschl--Teller)
    \item Demonstrates stability of $N_{\text{bound}}$ under truncation and BC variations
\end{itemize}

\textbf{What this is NOT:}
\begin{itemize}[nosep]
    \item NOT a derivation of $V(z)$ from 5D membrane action
    \item NOT a claim that $N_{\text{bound}} = 3$ (toy potential gives $N_{\text{bound}} = 2$)
    \item NOT OPR-21 closure
\end{itemize}

\textbf{Status:} OPR-21 remains \tagOPEN{} until $V(z)$ and BCs are derived from
membrane physics.
\end{tcolorbox}

\subsubsection{Setup: Half-Line Pöschl--Teller Potential}

We solve the dimensionless Schrödinger-type BVP on the half-line $z \in [0, \infty)$:
\begin{equation}
    -\frac{d^2\psi}{dz^2} + V(z)\psi = E\psi
    \label{eq:bvp:toy_schrodinger}
\end{equation}
with toy potential:
\begin{equation}
    V(z) = -V_0 \operatorname{sech}^2(z/a), \qquad V_0 = 10, \quad a = 1
    \label{eq:bvp:toy_potential}
\end{equation}
These parameters are chosen \emph{a priori} (no fitting to any physical data).

\paragraph{Boundary conditions.}
At $z = 0$: Robin BC $\psi'(0) + \kappa\psi(0) = 0$ with $\kappa \in \{0, 0.1, 0.5\}$.
At $z = z_{\max}$: Dirichlet $\psi(z_{\max}) = 0$ (truncation of half-line).

\paragraph{Essential spectrum threshold.}
Since $V(z) \to 0$ as $z \to \infty$, the threshold is $\lambda_{\text{th}} = 0$
(Definition~\ref{def:bvp:threshold}). Bound states satisfy $E < 0$.

\subsubsection{Robustness Demonstration}

Table~\ref{tab:bvp_toy_demo} shows the BVP outputs for varying truncation
$z_{\max}$ and Robin parameter $\kappa$.

% Include generated table
\begin{table}[htbp]
\centering
\caption{Half-line BVP toy outputs (\tagM{}/[Toy], dimensionless units).
$N_{\text{bound}} = 2$ is stable across all tested $(z_{\max}, \kappa)$ pairs,
demonstrating robustness per Lemma~\ref{lem:bvp:spectral_stability}.
Parameters: $V_0 = 10$, $a = 1$ (a priori, no calibration).}
\label{tab:bvp_toy_demo}
\small
\begin{tabular}{cccccc}
\toprule
$z_{\max}$ & $\kappa$ & $N_{\text{bound}}$ & $E_0$ & $x_1 = |E_0|$ & $I_4^{(0)}$ \\
\midrule
10.0 & 0.0 & 2 & $-7.35$ & 7.35 & 1.23 \\
10.0 & 0.1 & 2 & $-7.01$ & 7.01 & 1.17 \\
10.0 & 0.5 & 2 & $-5.96$ & 5.96 & 0.99 \\
12.0 & 0.0 & 2 & $-7.36$ & 7.36 & 1.23 \\
12.0 & 0.1 & 2 & $-7.02$ & 7.02 & 1.17 \\
12.0 & 0.5 & 2 & $-5.97$ & 5.97 & 0.99 \\
14.0 & 0.0 & 2 & $-7.36$ & 7.36 & 1.24 \\
14.0 & 0.1 & 2 & $-7.03$ & 7.03 & 1.17 \\
14.0 & 0.5 & 2 & $-5.98$ & 5.98 & 1.00 \\
\bottomrule
\end{tabular}
\end{table}

\paragraph{Key observations.}
\begin{enumerate}[nosep]
    \item \textbf{$N_{\text{bound}}$ is stable:} All $(z_{\max}, \kappa)$ combinations
          yield $N_{\text{bound}} = 2$, confirming Lemma~\ref{lem:bvp:spectral_stability}.
    \item \textbf{Eigenvalues converge:} $E_0$ varies by $< 0.3\%$ across $z_{\max}$ values.
    \item \textbf{$I_4$ is computable:} Ground-state overlap integral well-defined.
    \item \textbf{NOT three generations:} The toy potential gives 2 bound states,
          not 3. This is expected---the physical $V(z)$ (derived from membrane) may
          yield a different count.
\end{enumerate}

\begin{tcolorbox}[colback=yellow!5!white, colframe=yellow!60!black,
    title=\textbf{Interpretation: Pipeline Works, Physics Open}]
\label{box:bvp_pipeline_status}

This demonstration shows that:
\begin{itemize}[nosep]
    \item The BVP infrastructure (grid, solver, threshold, count) is operational
    \item $N_{\text{bound}}$ can be extracted robustly from the spectrum
    \item Output quantities ($x_1$, $I_4$) can be computed once profiles exist
\end{itemize}

\textbf{What remains for OPR-21 closure:}
\begin{itemize}[nosep]
    \item Derive $V(z)$ from membrane parameters $(\sigma, r_e)$
    \item Derive Robin BC parameter $\kappa$ from junction physics
    \item Compute $N_{\text{bound}}$ for the \emph{physical} potential
    \item If $N_{\text{bound}} = 3$ robustly $\Rightarrow$ OPR-02 upgrades to YELLOW
\end{itemize}

\textbf{Status:} OPR-21 and OPR-02 remain \tagOPEN{}.
\end{tcolorbox}

\subsubsection{Phase Diagram: Stepwise Spectral Counting}

The count $N_{\text{bound}}$ is an \emph{output} of the BVP, not a calibrated input.
To illustrate this, we sweep the well depth $V_0$ while keeping other parameters
fixed ($a=1$, $z_{\max}=14$, $\kappa=0$). Table~\ref{tab:bvp_phase_diagram} shows
the result.

\begin{table}[htbp]
\centering
\caption{Stepwise bound-state counting: $N_{\text{bound}}(V_0)$ for the Pöschl--Teller toy.
As the well depth increases, $N_{\text{bound}}$ grows in discrete steps.
This is generic Sturm--Liouville behavior, \emph{not} an EDC prediction.
OPR-02 closure requires deriving $V(z)$ from 5D membrane physics.}
\label{tab:bvp_phase_diagram}
\small
\begin{tabular}{cccc}
\toprule
$V_0$ & $N_{\text{bound}}$ & $E_0$ & $x_1 = |E_0|$ \\
\midrule
1.0 & 1 & $-0.39$ & 0.39 \\
3.0 & 1 & $-1.72$ & 1.72 \\
5.0 & 1 & $-3.24$ & 3.24 \\
7.0 & 2 & $-4.86$ & 4.86 \\
10.0 & 2 & $-7.36$ & 7.36 \\
15.0 & 2 & $-11.69$ & 11.69 \\
20.0 & 2 & $-16.12$ & 16.12 \\
30.0 & 3 & $-25.17$ & 25.17 \\
\bottomrule
\end{tabular}
\end{table}

\paragraph{Interpretation.}
For this toy family, $N_{\text{bound}}$ takes values $\{1, 2, 3\}$ as $V_0$ increases.
The transition points are approximately $V_0 \approx 6$ (1$\to$2) and $V_0 \approx 25$
(2$\to$3). This demonstrates the key point: \emph{three generations is not automatic}.
A specific potential depth is required.

\begin{tcolorbox}[colback=blue!5!white, colframe=blue!50!black,
    title=\textbf{Why $N_{\text{bound}} = 3$ is Non-Trivial}]
\label{box:why_three_nontrivial}

The phase diagram shows that obtaining exactly three bound states requires the
potential to lie in a specific window. For the toy Pöschl--Teller:
\[
N_{\text{bound}} = 3 \quad \Leftrightarrow \quad V_0 \gtrsim 25 \text{ (with } a=1\text{)}.
\]
The \emph{physical} membrane potential may have a completely different shape and
parameter regime. OPR-02 closure requires:
\begin{enumerate}[nosep]
    \item Derive $V(z)$ from membrane parameters $(\sigma, r_e, \ldots)$
    \item Compute $N_{\text{bound}}$ for that potential
    \item Show robustness under admissible BC deformations
\end{enumerate}
\end{tcolorbox}

\subsubsection{Visualization: Potential and Ground State}

Figure~\ref{fig:bvp_toy_profile} shows the toy potential $V(z)$ and the ground-state
wavefunction $\psi_0(z)$ for $V_0 = 10$, $a = 1$. The wavefunction is localized near
$z = 0$ where the potential well is deepest, and decays exponentially for $z \gg a$.

\begin{figure}[htbp]
\centering
\includegraphics[width=0.75\textwidth]{code/output/bvp_halfline_toy_figure.pdf}
\caption{Toy BVP profiles: potential $V(z) = -V_0 \operatorname{sech}^2(z/a)$ (blue)
and normalized ground-state wavefunction $\psi_0(z)$ (red) for $V_0 = 10$, $a = 1$.
The dashed line shows the essential spectrum threshold ($E = 0$); the dotted line
shows the ground-state energy $E_0 \approx -7.35$. Classification: \tagM{}/[Toy].}
\label{fig:bvp_toy_profile}
\end{figure}

\subsubsection{Reproducibility Box}

\begin{tcolorbox}[colback=gray!5!white, colframe=gray!70!black,
    title=\textbf{How to Reproduce}]
\label{box:bvp_reproducibility}

\textbf{Command:}
\begin{verbatim}
python3 code/bvp_halfline_toy_demo.py
\end{verbatim}

\textbf{Generated files:}
\begin{itemize}[nosep]
    \item \texttt{code/output/bvp\_halfline\_toy\_table.tex} --- robustness table
    \item \texttt{code/output/bvp\_halfline\_phase\_table.tex} --- phase diagram table
    \item \texttt{code/output/bvp\_halfline\_toy\_figure.pdf} --- $V(z)$ and $\psi_0$ plot
\end{itemize}

\textbf{Requirements:}
\begin{itemize}[nosep]
    \item Python 3.8+
    \item \texttt{numpy}, \texttt{scipy} (required)
    \item \texttt{matplotlib} (optional; for figure generation)
\end{itemize}

\textbf{Epistemic note:}
All parameters are chosen \emph{a priori}. No fitting to PDG, $M_W$, $G_F$, or
$v = 246$~GeV. The toy potential does NOT close OPR-21 or OPR-02.
\end{tcolorbox}

\subsubsection{Reader Takeaway: Why This Strengthens the Epistemic Contract}

\begin{tcolorbox}[colback=green!5!white, colframe=green!60!black,
    title=\textbf{Takeaway: $N_{\text{gen}} = 3$ is a Closure Target, Not a Slogan}]
\label{box:reader_takeaway}

This numerical demonstration makes one operational point:

\begin{itemize}[nosep]
    \item \textbf{Generation count is spectral:} $N_{\text{gen}}$ corresponds to
          $N_{\text{bound}}$ for a self-adjoint BVP with an intrinsic threshold.
    \item \textbf{Three is not automatic:} The toy phase diagram shows $N_{\text{bound}} \in \{1, 2, 3\}$
          depending on potential depth. Obtaining $N_{\text{bound}} = 3$ requires being in a
          \emph{specific parameter regime}.
    \item \textbf{Closure is conditional:} The statement ``$N_{\text{gen}} = 3$'' is a
          \emph{closure condition} that constrains the physically derived $V(z)$ and
          admissible BC family, not an automatic consequence of group-quotient arguments.
\end{itemize}

\medskip
\textbf{Consequently:} OPR-02 remains \tagOPEN{} until the physical $V(z)$ is derived from
the 5D membrane action and numerical BVP computation confirms $N_{\text{bound}} = 3$ under
admissible BCs.
\end{tcolorbox}

\paragraph{No-Calibration Guardrail.}
This demo does \emph{not} tune $V_0$ to match a desired outcome (e.g., ``force $N = 3$''),
nor does it use PDG inputs to define thresholds or counts. It is included to illustrate
the \emph{method} and \emph{logical structure} of closure:
\begin{equation}
    \text{derive } V(z) \text{ \& BCs from 5D action}
    \quad\Longrightarrow\quad
    \text{solve BVP}
    \quad\Longrightarrow\quad
    N_{\text{bound}} \stackrel{?}{=} 3.
    \label{eq:bvp:closure_chain}
\end{equation}
Until the first implication is completed, the final implication is not a prediction but
an \emph{explicitly tracked target} (OPR-02, OPR-21).
