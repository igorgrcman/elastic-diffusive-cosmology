%!TEX root = ../EDC_Part_II_Weak_Sector_rebuild.tex
% ==============================================================================
% OPR-21 BVP CLOSURE PACK
% The master key: precise specification + closure criteria + failure modes
% Status: Infrastructure specification (no numerical claims)
% ==============================================================================

\section{OPR-21: The BVP as Master Key}
\label{sec:bvp_master_key}

% ------------------------------------------------------------------------------
% EPISTEMIC STATUS
% ------------------------------------------------------------------------------

\begin{tcolorbox}[edcGuardrail, title=\textbf{Epistemic Status: Closure Pack}]
This section defines \textbf{closure criteria} for OPR-21 (the thick-brane BVP).
It does \emph{not} claim closure---it specifies what closure \emph{would mean}.

\textbf{Content type:}
\begin{itemize}[nosep]
    \item Mathematical specification: \tagM{} (pure mathematics, no physics input)
    \item Physical interpretation: \tagP{} (postulated connection to observables)
    \item Acceptance criteria: \tagDef{} (definitions, not derivations)
\end{itemize}

\textbf{NOT included:}
\begin{itemize}[nosep]
    \item Numerical solutions (deferred to implementation)
    \item Calibration to PDG values (forbidden by design)
    \item Claims about $N_{\text{gen}} = 3$ or specific $G_F$ values
\end{itemize}
\end{tcolorbox}

% ------------------------------------------------------------------------------
% FRAMEWORK 2.0 COMPLIANCE
% ------------------------------------------------------------------------------

\begin{tcolorbox}[colback=blue!3!white, colframe=blue!50!black,
    title=\textbf{Framework 2.0 Language Compliance}]
\small
\textbf{Why the BVP is the ``master key'':}
\begin{itemize}[nosep]
    \item \textbf{5D cause:} Thick-brane geometry creates an effective potential $V(z)$.
    \item \textbf{Brane process:} Schr\"odinger-like BVP determines bound states + spectrum.
    \item \textbf{3D shadow:} Eigenvalues $\to$ masses; eigenfunctions $\to$ couplings;
          bound state count $\to$ generations.
\end{itemize}

The BVP is the \emph{mathematical engine} that converts 5D postulates into 3D predictions
\emph{without} using 3D observables as input. Solving it is prerequisite for non-circular
closure of OPR-02 (generations), OPR-22 ($G_F$ spine), and CKM/PMNS overlaps.
\end{tcolorbox}

% ==============================================================================
\subsection{Why BVP is the Master Key}
\label{subsec:bvp_why_master}

Multiple open problems in Part II reduce to a single mathematical structure:
the thick-brane boundary value problem. This subsection maps the dependencies.

\paragraph{The central claim.}
If the BVP can be solved with \emph{independently fixed} parameters (from $\sigma$, $r_e$,
and membrane geometry alone), then the following become \emph{predictable}:

\begin{enumerate}[nosep]
    \item \textbf{Generation count} $N_{\text{bound}}$: number of normalizable bound states
    \item \textbf{Mass spectrum}: eigenvalue ratios (not absolute masses without scale)
    \item \textbf{Overlap integrals} $I_4$: coupling strengths from wavefunction products
    \item \textbf{Chirality suppression}: L/R asymmetry from mode localization
\end{enumerate}

\paragraph{Dependency map.}

\begin{center}
\begin{tabular}{lll}
\toprule
\textbf{Input (5D)} & \textbf{BVP Output} & \textbf{Downstream Claim} \\
\midrule
Potential $V(z)$ & Bound state count $N_{\text{bound}}$ & OPR-02: Why 3 generations? \\
Boundary conditions & Eigenvalue $x_1$ (first root) & OPR-22: $G_F$ spine scale \\
Domain $[0, \ell]$ or $[0, \infty)$ & Eigenfunctions $\psi_n(z)$ & CKM/PMNS: overlap integrals \\
Self-adjointness & Discrete spectrum guarantee & Quantization of masses \\
L/R coupling to $m(z)$ & Mode localization profiles & Ch9: V--A structure \\
\bottomrule
\end{tabular}
\end{center}

\paragraph{What closure means.}
The BVP is ``closed'' when:
\begin{enumerate}[nosep]
    \item The potential $V(z)$ is derived (not fitted) from membrane parameters
    \item The boundary conditions are derived from junction physics (not chosen ad hoc)
    \item The spectrum is computed without calibrating to PDG masses
    \item Downstream predictions match observations within stated tolerances
\end{enumerate}

% ==============================================================================
\subsection{Precise BVP Statement}
\label{subsec:bvp_precise}

% ------------------------------------------------------------------------------
% OPERATOR DEFINITION
% ------------------------------------------------------------------------------

\subsubsection{The Operator}

The thick-brane profile $f(z)$ satisfies a Schr\"odinger-type equation:
\begin{equation}
    \boxed{
    \hat{L} f \;=\; -\frac{d^2 f}{dz^2} + V(z) f \;=\; \lambda f
    }
    \label{eq:bvp:operator}
\end{equation}
where:
\begin{itemize}[nosep]
    \item $z \in \Omega$ is the extra-dimensional coordinate (domain specified below)
    \item $V(z)$ is the effective potential from membrane geometry \tagP{}
    \item $\lambda$ is the eigenvalue (related to 4D mass: $\lambda \sim m^2$)
    \item $f(z)$ is the profile function (wavefunction in extra dimension)
\end{itemize}

\paragraph{Dimensionless form.}
With $\zeta = z/\ell$ (where $\ell$ is the brane thickness scale), and
$\tilde{V}(\zeta) = \ell^2 V(\ell\zeta)$, $\tilde{\lambda} = \ell^2 \lambda$:
\begin{equation}
    -\frac{d^2 f}{d\zeta^2} + \tilde{V}(\zeta) f = \tilde{\lambda} f
    \label{eq:bvp:dimensionless}
\end{equation}
All numerical work should use this dimensionless form \tagM{}.

% ------------------------------------------------------------------------------
% DOMAIN AND INNER PRODUCT
% ------------------------------------------------------------------------------

\subsubsection{Domain and Inner Product}

\paragraph{Domain options.}
The physical domain $\Omega$ depends on brane topology:
\begin{center}
\begin{tabular}{lll}
\toprule
\textbf{Domain} & \textbf{Physical Interpretation} & \textbf{Spectrum Type} \\
\midrule
$[0, \ell]$ (finite interval) & Finite brane thickness & Pure discrete \\
$[0, \infty)$ (half-line) & Semi-infinite bulk & Discrete + continuous \\
$(-\infty, \infty)$ (full line) & Symmetric domain wall & Depends on $V$ asymptotics \\
\bottomrule
\end{tabular}
\end{center}

For Part II, the \textbf{finite interval} $\Omega = [0, \ell]$ is the primary focus
(thick brane with defined boundaries). The half-line appears in Ch9 (chirality).

\paragraph{Inner product and normalization.}
The standard $L^2$ inner product:
\begin{equation}
    \langle f, g \rangle = \int_\Omega f(z)^* g(z) \, dz
    \label{eq:bvp:inner_product}
\end{equation}
Eigenfunctions are normalized: $\langle \psi_n, \psi_n \rangle = 1$.

For weighted problems (e.g., curved extra dimension), use $\langle f, g \rangle_w =
\int_\Omega w(z) f^* g \, dz$ with positive weight $w(z) > 0$.

% ------------------------------------------------------------------------------
% BOUNDARY CONDITIONS
% ------------------------------------------------------------------------------

\subsubsection{Boundary Conditions}
\label{subsubsec:bvp_bc}

The BVP requires boundary conditions at each endpoint. The general \textbf{Robin form}
encompasses Dirichlet and Neumann as special cases:

\begin{equation}
    \boxed{
    \alpha_0 f(0) + \beta_0 f'(0) = 0, \qquad
    \alpha_\ell f(\ell) + \beta_\ell f'(\ell) = 0
    }
    \label{eq:bvp:robin_bc}
\end{equation}
where $(\alpha_0, \beta_0)$ and $(\alpha_\ell, \beta_\ell)$ are BC parameters
(not both zero at each endpoint).

\paragraph{Special cases.}
\begin{center}
\begin{tabular}{lll}
\toprule
\textbf{Name} & \textbf{Condition} & \textbf{Physical Interpretation} \\
\midrule
Dirichlet & $f = 0$ at boundary & Hard wall / infinite barrier \\
Neumann & $f' = 0$ at boundary & Reflection symmetry / no flux \\
Robin & $\alpha f + \beta f' = 0$ & Finite barrier / junction matching \\
\bottomrule
\end{tabular}
\end{center}

\paragraph{Open problem.}
The BC parameters $(\alpha, \beta)$ should be \emph{derived} from Israel junction
conditions or brane microphysics, not chosen to fit outputs. This is part of OPR-21.

% ------------------------------------------------------------------------------
% SELF-ADJOINTNESS
% ------------------------------------------------------------------------------

\subsubsection{Self-Adjointness Conditions}
\label{subsubsec:bvp_selfadjoint}

For the operator $\hat{L}$ to have real eigenvalues and orthogonal eigenfunctions,
it must be self-adjoint on the chosen domain with the chosen BCs.

\begin{theorem}[Self-Adjointness Criterion]
\label{thm:bvp:selfadjoint}
\tagM{}
The Sturm--Liouville operator $\hat{L} = -d^2/dz^2 + V(z)$ on $[0, \ell]$ with
Robin BCs~\eqref{eq:bvp:robin_bc} is self-adjoint if and only if:
\begin{enumerate}[nosep]
    \item $V(z)$ is real-valued and locally integrable on $(0, \ell)$
    \item The BC parameters are real: $\alpha_0, \beta_0, \alpha_\ell, \beta_\ell \in \mathbb{R}$
    \item The BCs are \emph{separated} (each endpoint has its own condition)
\end{enumerate}
\end{theorem}

\begin{proof}[Proof sketch]
Standard Sturm--Liouville theory~\cite{Zettl2005,Teschl2012}. Integration by parts
shows $\langle \hat{L}f, g \rangle - \langle f, \hat{L}g \rangle = [f^*g' - f'^*g]_0^\ell$,
which vanishes under separated real Robin BCs. \qed
\end{proof}

\paragraph{Consequence.}
Under self-adjointness:
\begin{itemize}[nosep]
    \item Eigenvalues $\lambda_n$ are real and form a discrete sequence (if $V$ confining)
    \item Eigenfunctions $\psi_n$ are orthogonal: $\langle \psi_m, \psi_n \rangle = \delta_{mn}$
    \item The spectrum is bounded below: $\exists \lambda_0 = \min_n \lambda_n$
\end{itemize}

% ==============================================================================
\subsection{Output Objects: What We Need}
\label{subsec:bvp_outputs}

The BVP solution provides specific mathematical objects used by downstream chapters.

% ------------------------------------------------------------------------------
% EIGENFUNCTIONS
% ------------------------------------------------------------------------------

\subsubsection{Eigenfunctions $\psi_n(z)$}
\label{subsubsec:bvp_eigenfunctions}

\begin{definition}[Normalized Eigenfunctions]
\label{def:bvp:eigenfunction}
\tagDef{}
The $n$-th eigenfunction $\psi_n(z)$ satisfies:
\begin{enumerate}[nosep]
    \item $\hat{L}\psi_n = \lambda_n \psi_n$ (eigenvalue equation)
    \item $\langle \psi_n, \psi_n \rangle = 1$ (unit normalization)
    \item BCs~\eqref{eq:bvp:robin_bc} at both endpoints
\end{enumerate}
Ordering convention: $\lambda_0 < \lambda_1 < \lambda_2 < \cdots$ (ground state first).
\end{definition}

\paragraph{Physical interpretation \tagP{}.}
$|\psi_n(z)|^2$ gives the probability density for finding the $n$-th mode at depth $z$.
Modes localized near $z=0$ couple strongly to brane-localized interactions.

% ------------------------------------------------------------------------------
% FIRST EIGENVALUE
% ------------------------------------------------------------------------------

\subsubsection{First Eigenvalue $x_1$}
\label{subsubsec:bvp_x1}

\begin{definition}[First Root / Ground State Eigenvalue]
\label{def:bvp:x1}
\tagDef{}
The quantity $x_1$ is defined as the \textbf{lowest positive eigenvalue} of the
dimensionless BVP~\eqref{eq:bvp:dimensionless}:
\begin{equation}
    x_1 = \min\{\tilde{\lambda}_n : \tilde{\lambda}_n > 0\}
    \label{eq:bvp:x1_def}
\end{equation}
If the ground state has $\tilde{\lambda}_0 \leq 0$, then $x_1 = \tilde{\lambda}_1$.
\end{definition}

\paragraph{Role in $G_F$ spine.}
In Ch13, $x_1$ sets the scale for effective couplings: $G_F \propto 1/(x_1 \cdot \text{scale}^2)$.
The value of $x_1$ depends on $V(\zeta)$ and BCs---it is \emph{output}, not input.

% ------------------------------------------------------------------------------
% OVERLAP INTEGRALS
% ------------------------------------------------------------------------------

\subsubsection{Overlap Integrals $I_4$}
\label{subsubsec:bvp_overlap}

\begin{definition}[Four-Point Overlap Integral]
\label{def:bvp:I4}
\tagDef{}
The overlap integral $I_4$ for modes $(i, j, k, l)$ is:
\begin{equation}
    I_4^{(ijkl)} = \int_\Omega \psi_i(z) \psi_j(z) \psi_k(z) \psi_l(z) \, dz
    \label{eq:bvp:I4_def}
\end{equation}
For the dominant (ground state) contribution:
\begin{equation}
    I_4 \equiv I_4^{(0000)} = \int_\Omega |\psi_0(z)|^4 \, dz
    \label{eq:bvp:I4_ground}
\end{equation}
\end{definition}

\paragraph{Physical interpretation \tagP{}.}
$I_4$ measures the ``concentration'' of the ground state profile. For a mode spread
uniformly over $[0, \ell]$, $I_4 \sim 1/\ell$. For a strongly localized mode
(width $w \ll \ell$), $I_4 \sim 1/w \gg 1/\ell$.

\paragraph{Role in couplings.}
Effective 4D couplings scale as $g_{\text{eff}} \propto g_5 \cdot \sqrt{I_4}$ where
$g_5$ is the 5D coupling. Stronger localization $\to$ larger effective coupling.

% ------------------------------------------------------------------------------
% GENERATION COUNT
% ------------------------------------------------------------------------------

\subsubsection{Generation Count $N_{\text{bound}}$}
\label{subsubsec:bvp_ngen}

\begin{definition}[Bound State Count]
\label{def:bvp:nbound}
\tagDef{}
The generation count $N_{\text{bound}}$ is defined as the number of \textbf{normalizable
bound states} below a threshold $\lambda_{\text{th}}$:
\begin{equation}
    N_{\text{bound}} = \#\{n : \lambda_n < \lambda_{\text{th}} \text{ and }
    \psi_n \in L^2(\Omega)\}
    \label{eq:bvp:nbound_def}
\end{equation}
\end{definition}

\paragraph{Threshold definition.}
The threshold $\lambda_{\text{th}}$ separates ``light'' bound states (generations)
from ``heavy'' KK modes or continuum. Possible definitions:
\begin{itemize}[nosep]
    \item \textbf{Potential barrier:} $\lambda_{\text{th}} = V(\ell)$ (top of confining well)
    \item \textbf{Scale separation:} $\lambda_{\text{th}} = (\ell^{-1})^2$ (inverse thickness squared)
    \item \textbf{Physical mass:} $\lambda_{\text{th}} = m_{\text{heavy}}^2$ (but this uses 3D input!)
\end{itemize}

\paragraph{OPR-02 closure condition.}
OPR-02 is closed if $N_{\text{bound}} = 3$ emerges \emph{robustly} from derived $V(z)$
and BCs, without tuning to get this number.

% ------------------------------------------------------------------------------
% CHIRALITY CRITERION
% ------------------------------------------------------------------------------

\subsubsection{Chirality Suppression}
\label{subsubsec:bvp_chirality}

\begin{definition}[Chirality Asymmetry Ratio]
\label{def:bvp:chirality}
\tagDef{}
For left-handed ($f_L$) and right-handed ($f_R$) mode profiles (from Ch9), define:
\begin{equation}
    R_{\text{LR}} = \frac{|f_R(0)|^2}{|f_L(0)|^2}
    \label{eq:bvp:chirality_ratio}
\end{equation}
This measures the relative coupling of R-modes to the boundary.
\end{definition}

\paragraph{V--A criterion.}
Ch9 derived $R_{\text{LR}} \ll 1$ for the half-line domain with $m(z) > 0$.
On a finite interval, the suppression depends on $\ell$ and BCs.
A consistent picture requires $R_{\text{LR}} < 10^{-3}$ (experimental limit on
right-handed currents).

% ==============================================================================
\subsection{Acceptance Criteria and Closure Conditions}
\label{subsec:bvp_acceptance}

% ------------------------------------------------------------------------------
% CLOSURE TABLE
% ------------------------------------------------------------------------------

\begin{tcolorbox}[colback=green!5!white, colframe=green!50!black,
    title=\textbf{Closure Conditions by OPR Item}]
\small
\begin{center}
\begin{tabular}{p{2cm}p{5cm}p{5cm}}
\toprule
\textbf{OPR} & \textbf{Closure Criterion} & \textbf{What Must Be True} \\
\midrule
OPR-02 & $N_{\text{bound}} = 3$ & Robust under BC variations; derived $V(z)$ \\
OPR-21 & BVP fully specified & $V(z)$, BCs derived from $(\sigma, r_e)$; no fit \\
OPR-22 & $I_4$ converges & Eigenfunctions normalizable; overlap finite \\
Ch13 & $G_F$ spine predictive & $x_1$ computed; scale from membrane only \\
Ch9 & $R_{\text{LR}} < 10^{-3}$ & Chirality suppression from BVP (consistent) \\
\bottomrule
\end{tabular}
\end{center}
\end{tcolorbox}

% ------------------------------------------------------------------------------
% NUMERICAL CRITERIA
% ------------------------------------------------------------------------------

\paragraph{Numerical analysis criteria.}
For computational implementation (deferred):
\begin{enumerate}[nosep]
    \item \textbf{Convergence:} Eigenvalues stable under grid refinement ($< 0.1\%$ change
          when grid doubled)
    \item \textbf{Normalization:} $|\langle \psi_n, \psi_n \rangle - 1| < 10^{-6}$
    \item \textbf{Orthogonality:} $|\langle \psi_m, \psi_n \rangle| < 10^{-6}$ for $m \neq n$
    \item \textbf{Boundary residual:} BC satisfaction $< 10^{-8}$ (relative)
\end{enumerate}

% ==============================================================================
\subsection{Failure Modes}
\label{subsec:bvp_failure}

\begin{tcolorbox}[colback=red!5!white, colframe=red!50!black,
    title=\textbf{Failure Modes (F1--F6)}]
\small

\textbf{F1: Non-self-adjoint BCs.}
If BCs are not separated or real, eigenvalues may be complex $\to$ no physical spectrum.
\emph{Next step:} Derive BCs from junction physics ensuring self-adjointness.

\textbf{F2: Non-normalizable modes.}
If $V(z) \to 0$ too fast at boundaries, modes may not be $L^2$ $\to$ no bound states.
\emph{Next step:} Check potential asymptotics; confining $V$ required.

\textbf{F3: Numerical instability.}
Shooting methods can fail for stiff potentials or near-degenerate eigenvalues.
\emph{Next step:} Use stable algorithms (finite element, spectral methods); validate
against known analytic solutions.

\textbf{F4: Parameter sensitivity.}
If $N_{\text{bound}}$ changes drastically with small BC variations $\to$ no robust prediction.
\emph{Next step:} Map sensitivity; identify ``natural'' BC class from physics.

\textbf{F5: Threshold ambiguity.}
Different threshold definitions give different $N_{\text{bound}}$ $\to$ no unique count.
\emph{Next step:} Derive threshold from scale separation (gap in spectrum).

\textbf{F6: Circular calibration.}
If $V(z)$ or BCs are tuned to reproduce PDG masses $\to$ no predictive power.
\emph{Next step:} Derive parameters from membrane physics only; compare to PDG \emph{after}.

\end{tcolorbox}

% ==============================================================================
\subsection{Integration Pointers}
\label{subsec:bvp_pointers}

% ------------------------------------------------------------------------------
% POINTER BOXES
% ------------------------------------------------------------------------------

\begin{tcolorbox}[colback=gray!5!white, colframe=gray!50!black,
    title=\textbf{Used by: Ch12 OPR Register}]
\small
The BVP closure pack (this section) provides formal definitions for:
\begin{itemize}[nosep]
    \item OPR-02 closure criterion: $N_{\text{bound}} = 3$ (Definition~\ref{def:bvp:nbound})
    \item OPR-21 closure criterion: BVP fully derived (Sec.~\ref{subsec:bvp_precise})
    \item OPR-22 closure criterion: $I_4$ convergent (Definition~\ref{def:bvp:I4})
\end{itemize}
See Ch12 (Epistemic Landscape) for full OPR register.
\end{tcolorbox}

\begin{tcolorbox}[colback=gray!5!white, colframe=gray!50!black,
    title=\textbf{Feeds: Ch13 $G_F$ Spine}]
\small
The $G_F$ derivation pathway requires:
\begin{itemize}[nosep]
    \item First eigenvalue $x_1$ (Definition~\ref{def:bvp:x1})
    \item Overlap integral $I_4$ (Definition~\ref{def:bvp:I4})
    \item Scale from membrane: $\ell \sim r_e$ (thickness)
\end{itemize}
The spine becomes predictive when these are computed from derived $V(z)$.
\end{tcolorbox}

\begin{tcolorbox}[colback=gray!5!white, colframe=gray!50!black,
    title=\textbf{Feeds: Ch9 Chirality}]
\small
The V--A structure (Ch9) uses:
\begin{itemize}[nosep]
    \item Chirality ratio $R_{\text{LR}}$ (Definition~\ref{def:bvp:chirality})
    \item Mode localization profiles $f_{L/R}(z)$
    \item Half-line vs finite-interval consistency
\end{itemize}
Ch9 establishes V--A qualitatively; the closure pack provides quantitative criterion.
\end{tcolorbox}

% ==============================================================================
% SUMMARY BOX
% ==============================================================================

\begin{tcolorbox}[colback=yellow!5!white, colframe=yellow!60!black,
    title=\textbf{Summary: What This Closure Pack Provides}]
\begin{enumerate}[nosep]
    \item \textbf{Precise BVP statement:} Operator, domain, BCs, self-adjointness (Sec.~\ref{subsec:bvp_precise})
    \item \textbf{Output definitions:} $\psi_n$, $x_1$, $I_4$, $N_{\text{bound}}$, $R_{\text{LR}}$ (Sec.~\ref{subsec:bvp_outputs})
    \item \textbf{Closure criteria:} What each OPR needs to be ``closed'' (Sec.~\ref{subsec:bvp_acceptance})
    \item \textbf{Failure modes:} What can go wrong and what to do next (Sec.~\ref{subsec:bvp_failure})
    \item \textbf{Integration map:} How this feeds Ch9, Ch12, Ch13 (Sec.~\ref{subsec:bvp_pointers})
\end{enumerate}

\textbf{What it does NOT provide:}
\begin{itemize}[nosep]
    \item Numerical solutions (implementation deferred)
    \item Derived $V(z)$ from membrane parameters (OPR-21 remains open)
    \item Derived BCs from junction physics (part of OPR-21)
    \item Claims that $N_{\text{bound}} = 3$ (requires computation)
\end{itemize}
\end{tcolorbox}
