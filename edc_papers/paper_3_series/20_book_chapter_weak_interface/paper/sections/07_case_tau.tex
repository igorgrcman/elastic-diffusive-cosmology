% ==============================================================================
% Subsection: Tau Decay (part of Section 1.7: Charged Leptons)
% ==============================================================================

\subsection{Tau Decay: Higher-Mode Brane Excitation}
\label{subsec:tau_story}
\label{sec:case_tau}  % alias for cross-references

% --- AT-A-GLANCE BOX (KB-CANON-002) ---
\begin{edcAtAGlance}{Tau Decay}
  \edcBaseline{
    Decay: Multiple channels with $\tau_\tau = 2.903 \times 10^{-13}$ s\\
    Leptonic: $\tau \to e\nu\bar\nu$ (17.8\%) and $\tau \to \mu\nu\bar\nu$ (17.4\%)\\
    Hadronic: $\tau \to$ hadrons + $\nu_\tau$ (64.8\% total)\\
    Energy: $m_\tau c^2 = 1777$ MeV (heaviest lepton)
  }
  \edcEDCView{
    Tau = higher-mode brane excitation (same ontology as muon)\\
    Larger energy budget opens hadronic channels via $\mathcal{P}_{\mathrm{energy}}$ threshold\\
    Same pipeline: stored brane energy $\to$ redistribution $\to$ frozen projection\\
    No new mechanism required---only higher mode number
  }
  \edcKeyInsight{
    The tau is the cleanest universality test: if the same absorption--dissipation--release
    mechanism works for both $\mu$ and $\tau$ without new ingredients, the weak-sector
    brane interface is truly universal, not specific to any single particle.
  }
  \edcFalsifiable{
    \textbullet\ If tau requires different ontological category than muon\\
    \textbullet\ If same $\mathcal{P}_{\mathrm{chir}}$ cannot apply without channel-specific tuning\\
    \textbullet\ If $\tau/\mu$ lifetime ratio contradicts mode-energy interpretation
  }
\end{edcAtAGlance}

\medskip

% ==============================================================================
% MOTIVATION: WHY TAU AFTER MUON?
% ==============================================================================

\subsubsection{Motivation: Why Tau After Muon?}

\begin{tcolorbox}[edcCornerstone, title=\textbf{Cornerstone: Tau as Mode-Spectrum Test}]
The tau lepton ($\tau^-$) is the heaviest charged lepton, with
$m_\tau \approx 1777$~MeV \tagBL{}. If the thick-brane
framework applies to muon decay (\S\ref{subsec:muon_story}), it must also accommodate
tau decay without introducing new mechanisms. The tau provides a
\emph{mode-spectrum test}: same brane-dominant ontology, different
mass/energy scale.
\end{tcolorbox}

The EDC weak-interaction program now has:
\begin{itemize}[nosep]
    \item \textbf{Neutron} (\S\ref{sec:case_neutron}): Bulk-core junction decay
    \item \textbf{Muon} (\S\ref{subsec:muon_story}): Brane-dominant leptonic decay
    \item \textbf{Tau} (this section): Heavier brane-dominant decay
\end{itemize}

If the same pipeline works for both $\mu$ and $\tau$, this validates the
\emph{brane-dominant excitation} hypothesis across the charged lepton
spectrum. The tau's larger mass probes a different region of the
brane-layer mode spectrum.

\paragraph{Scope limitation.}
This case study addresses \textbf{leptonic tau decays} primarily:
\begin{itemize}[nosep]
    \item $\tau^- \to e^- + \bar{\nu}_e + \nu_\tau$ \quad (electronic channel)
    \item $\tau^- \to \mu^- + \bar{\nu}_\mu + \nu_\tau$ \quad (muonic channel)
\end{itemize}
Hadronic tau decays (e.g., $\tau \to \pi\nu$, $\tau \to \rho\nu$) are
discussed as threshold-gated extensions (open). Full pion ontology is developed
in \S\ref{sec:case_pion}.

% ==============================================================================
% TAU ONTOLOGY
% ==============================================================================

\subsubsection{Tau Ontology: Brane-Dominant Higher Mode}

The tau is treated as the same ontological class as the muon: a brane-dominant excitation,
but at higher energy in the brane mode spectrum.

\begin{edcPostulateBox}{Tau Ontology}{[P]}
The tau lepton $\tau^-$ is a \emph{brane-dominant excitation} with a
higher mode index than the muon. Its primary degrees of freedom reside
within the brane layer, not in the bulk-core.
\end{edcPostulateBox}

\textbf{Physical Narration:}
\begin{enumerate}[nosep]
    \item \textbf{5D cause:} The tau occupies a higher-energy eigenmode of the
          brane-layer spectrum compared to the muon.
    \item \textbf{Brane response:} This mode is unstable; it can decay into
          lower-mass modes (electrons, muons, neutrinos, hadrons) via internal
          redistribution.
    \item \textbf{3D output:} The frozen projection maps allowed mode
          combinations to observable particles.
\end{enumerate}

% ==============================================================================
% MODE INDEX DEFINITION
% ==============================================================================

\subsubsection{Mode Index Hypothesis}

\begin{edcDefinitionBox}{Mode Index}{[P]}
We associate each charged lepton with a \emph{mode index} $n_\ell$
characterizing its position in the brane-layer spectrum:
\[
    n_e < n_\mu < n_\tau
\]
Higher mode index corresponds to higher mass and shorter lifetime
(greater instability).
\end{edcDefinitionBox}

\textbf{Note:} The mode index is a qualitative ordering \tagP{}.
We do not claim to derive $n_\ell$ values or the precise relationship
$m_\ell(n_\ell)$ from first principles.

% ==============================================================================
% MODE SPECTRUM FIGURE
% ==============================================================================

\begin{figure}[htbp]
\centering
\begin{tikzpicture}[scale=0.85]
    % Background regions
    \fill[bulk region] (-4.5,-2.5) rectangle (-1.5,2.5);
    \fill[brane region] (-1.5,-2.5) rectangle (1.5,2.5);
    \fill[observer region] (1.5,-2.5) rectangle (4.5,2.5);

    % Labels
    \node[section label] at (-3,2.9) {\textbf{Bulk-Core}};
    \node[section label] at (0,2.9) {\textbf{Brane-Layer}};
    \node[section label] at (3,2.9) {\textbf{3D Outputs}};

    % Boundaries
    \draw[bulk boundary] (-1.5,-2.5) -- (-1.5,2.5);
    \draw[observer boundary] (1.5,-2.5) -- (1.5,2.5);

    % Mode spectrum visualization (vertical axis = energy/mode index)
    \node[font=\scriptsize, rotate=90] at (-4.2,0) {Mode energy $\uparrow$};

    % Tau mode (higher)
    \node[circle, fill=orange!70, minimum size=10pt, inner sep=0pt] (tau) at (0,1.5) {};
    \node[right=0.15cm of tau, font=\footnotesize] {$\tau^-$ (high mode)};
    \draw[dashed, orange!60!black, thick] (-1.3,1.5) -- (1.3,1.5);

    % Muon mode (middle)
    \node[circle, fill=purple!60, minimum size=10pt, inner sep=0pt] (mu) at (0,0) {};
    \node[right=0.15cm of mu, font=\footnotesize] {$\mu^-$ (mid mode)};
    \draw[dashed, purple!60!black, thick] (-1.3,0) -- (1.3,0);

    % Electron mode (lowest)
    \node[circle, fill=blue!60, minimum size=10pt, inner sep=0pt] (e) at (0,-1.5) {};
    \node[right=0.15cm of e, font=\footnotesize] {$e^-$ (low mode)};
    \draw[dashed, blue!60!black, thick] (-1.3,-1.5) -- (1.3,-1.5);

    % Arrows showing decay directions
    \draw[->, thick, orange!70!black] (0.5,1.3) -- (0.5,0.2);
    \draw[->, thick, orange!70!black] (0.7,1.3) -- (0.7,-1.3);
    \node[font=\scriptsize, text=orange!70!black] at (1.1,0.7) {$\tau \to \mu$};
    \node[font=\scriptsize, text=orange!70!black] at (1.1,-0.5) {$\tau \to e$};

    % Bulk annotation
    \node[font=\scriptsize, text=gray] at (-3,0) {(no bulk core)};

\end{tikzpicture}
\caption{Charged lepton mode spectrum in the brane layer. The tau occupies
a higher mode than the muon, which in turn is higher than the electron.
Decay proceeds ``downward'' in the spectrum via mode redistribution.
All three are brane-dominant; none have bulk-core structure.}
\label{fig:tau-mode-spectrum}
\end{figure}

% ==============================================================================
% OBSERVATIONAL BASELINES
% ==============================================================================

\subsubsection{Observational Baselines}

The following quantities are treated as \textbf{observational inputs}
\tagBL{}, not outputs of the model.

\begin{table}[htbp]
\centering
\caption{Tau lepton properties (PDG 2024) \tagBL{}}
\label{tab:tau-baselines}
\begin{tabular}{lcc}
\toprule
\textbf{Quantity} & \textbf{Value} & \textbf{Status} \\
\midrule
Mass $m_\tau$ & $1776.86 \pm 0.12$~MeV & \tagBL{} \\
Lifetime $\tau_\tau$ & $(290.3 \pm 0.5) \times 10^{-15}$~s & \tagBL{} \\
BR($\tau \to e\nu\bar{\nu}$) & $(17.82 \pm 0.04)\%$ & \tagBL{} \\
BR($\tau \to \mu\nu\bar{\nu}$) & $(17.39 \pm 0.04)\%$ & \tagBL{} \\
BR(leptonic total) & $\approx 35\%$ & \tagBL{} \\
BR(hadronic total) & $\approx 65\%$ & \tagBL{} \\
\bottomrule
\end{tabular}
\end{table}

\begin{tcolorbox}[edcGuardrail, title=\textbf{Epistemic Guardrail: No Fitting}]
\textbf{These are not tuning targets.} The branching fractions and lifetime
in Table~\ref{tab:tau-baselines} are \emph{facts about nature} that any viable
model must be \emph{consistent with}. We do not adjust parameters to
reproduce them. Companion T provides a consistent 5D$\to$brane$\to$3D mechanism
framing \emph{without tuning parameters to match those numbers}.
\end{tcolorbox}

% ==============================================================================
% PIPELINE FOR TAU DECAY
% ==============================================================================

\subsubsection{Pipeline for Tau Decay}

The tau decay pipeline mirrors that of muon decay (\S\ref{subsec:muon_story}), with
the same three phases:

\begin{tcolorbox}[edcPPN, title=\textbf{Physical Process Narrative: Tau Leptonic Decay}]
\begin{enumerate}[nosep]
    \item[\textbf{(i)}] \textbf{Absorption/Charging:} The unstable tau mode
          redistributes energy within the brane layer.
    \item[\textbf{(ii)}] \textbf{Dissipation:} Brane-layer modes become
          populated according to the available spectrum and selection rules.
    \item[\textbf{(iii)}] \textbf{Release/Emission:} The frozen projection
          $\mathcal{P}_{\mathrm{frozen}}$ maps populated modes to 3D outputs.
\end{enumerate}
\end{tcolorbox}

At the structural level, the pipeline is identical to muon:
\begin{equation}
\Psi_\tau \;\Rightarrow\; E_{\mathrm{brane}}(t_0)\approx m_\tau c^2 \;\Rightarrow\;
\Gamma_{\mathrm{eff}} \;\Rightarrow\; \mathcal{P}_{\mathrm{frozen}} \;\Rightarrow\; \text{allowed outputs}.
\end{equation}

The key difference is that $\mathcal{P}_{\mathrm{energy}}$ and $\mathcal{P}_{\mathrm{mode}}$ now admit a
broader set of outputs because $m_\tau c^2 \approx 1777$ MeV \tagBL{} provides a much larger energy budget.

% ==============================================================================
% BULK LEAKAGE SUPPRESSION
% ==============================================================================

\subsubsection{Bulk Leakage Suppression}

\begin{edcPostulateBox}{Suppressed Bulk Leakage}{[P]}
For brane-dominant excitations (electron, muon, tau), leakage of energy
into the bulk-core is suppressed by the mode's localization within the
brane layer. At leading order, bulk leakage is treated as negligible.
\end{edcPostulateBox}

\textbf{Physical Narration:}
\begin{itemize}[nosep]
    \item \textbf{5D cause:} Brane-layer modes have exponentially small
          overlap with bulk-core wavefunctions.
    \item \textbf{Brane response:} Energy redistribution occurs predominantly
          within the brane layer.
    \item \textbf{3D output:} All released energy appears in 3D outputs
          (plus soft/residual brane modes).
\end{itemize}

% ==============================================================================
% PROCESS DIAGRAM
% ==============================================================================

\subsubsection{Process Diagram: Tau Decay}

\begin{figure}[htbp]
\centering
\begin{tikzpicture}[scale=0.85]

% Nodes
\node[draw, fill=orange!15, rounded corners, minimum width=2.2cm, minimum height=1cm, align=center] (tau) at (0,0) {$\tau^-$ mode\\(brane-layer)};
\node[draw, fill=green!10, rounded corners, minimum width=2cm, minimum height=1cm, right=1.6cm of tau, align=center] (abs) {Absorption/\\Redistribution};
\node[draw, fill=teal!10, rounded corners, minimum width=2cm, minimum height=1cm, right=1.6cm of abs, align=center] (diss) {Dissipation/\\Mode population};
\node[draw, fill=blue!10, rounded corners, minimum width=2.0cm, minimum height=1cm, right=1.6cm of diss, align=center] (out) {3D Outputs\\$\ell^-, \bar{\nu}_\ell, \nu_\tau$};

% Arrows with labels
\draw[->, thick, orange!70!black] (tau) -- node[above, font=\scriptsize] {instability} (abs);
\draw[->, thick, green!50!black] (abs) -- node[above, font=\scriptsize] {$\Gamma_{\mathrm{eff}}$} (diss);
\draw[->, thick, blue!60!black] (diss) -- node[above, font=\scriptsize] {$\mathcal{P}_{\mathrm{frozen}}$} (out);

% Phase labels below
\node[font=\scriptsize, gray] at (0,-1) {Initial state};
\node[font=\scriptsize, gray] at ($(abs.south) + (0,-0.3)$) {Charging};
\node[font=\scriptsize, gray] at ($(diss.south) + (0,-0.3)$) {Mode spectrum};
\node[font=\scriptsize, gray] at ($(out.south) + (0,-0.3)$) {Observation};

% Chiral filter annotation
\draw[dashed, red!60!black] ($(diss.east)!0.5!(out.west)$) ++(0,-0.7) -- ++(0,1.4);
\node[font=\scriptsize, text=red!60!black, below] at ($(diss.east)!0.5!(out.west) + (0,-0.9)$) {$\mathcal{P}_{\mathrm{chir}}$};

% Ledger closure annotation
\node[draw, fill=gray!5, rounded corners, font=\footnotesize] (ledger) at (5,-2) {Ledger: $m_\tau c^2 = E_\ell + E_{\bar{\nu}} + E_{\nu_\tau} + E_{\mathrm{other}}$};

\end{tikzpicture}
\caption{Energy flow in tau leptonic decay. The pipeline is identical to
muon decay (\S\ref{subsec:muon_story}), with the tau as initial brane-dominant mode.
The output $\ell^-$ can be either $e^-$ or $\mu^-$.}
\label{fig:tau_pipeline}
\end{figure}

% ==============================================================================
% ALLOWED OUTPUT SETS
% ==============================================================================

\subsubsection{Allowed Output Sets and Selection Rules}

\begin{edcDefinitionBox}{Allowed Output Sets for Tau Leptonic Decays}{[Dc]}
The allowed output sets for tau leptonic decays are:
\begin{align}
    \mathcal{A}_{\tau \to e} &= \{e^-, \bar{\nu}_e, \nu_\tau\} \\
    \mathcal{A}_{\tau \to \mu} &= \{\mu^-, \bar{\nu}_\mu, \nu_\tau\}
\end{align}
These follow from:
\begin{itemize}[nosep]
    \item Charge conservation: $Q_\tau = Q_\ell = -1$
    \item Lepton number conservation: $L_\tau = 1$ (carried by $\nu_\tau$),
          $L_\ell = 0$ (from $\ell^- + \bar{\nu}_\ell$ pair)
    \item Energy threshold: $m_\tau > m_\mu > m_e$ (both channels kinematically allowed)
\end{itemize}
\end{edcDefinitionBox}

% ==============================================================================
% CHANNEL COMPARISON TABLE
% ==============================================================================

\subsubsection{Leptonic and Forbidden Channels}

\begin{table}[htbp]
\centering
\caption{Tau leptonic channels: experimental vs.\ EDC framing}
\label{tab:tau-channels}
\begin{tabular}{lccc}
\toprule
\textbf{Channel} & \textbf{BR (exp.)} & \textbf{Status} & \textbf{EDC framing} \\
\midrule
$\tau \to e\nu\bar{\nu}$ & $17.82\%$ & \tagBL{} & Allowed by $\mathcal{A}_{\tau \to e}$ \tagDc{} \\
$\tau \to \mu\nu\bar{\nu}$ & $17.39\%$ & \tagBL{} & Allowed by $\mathcal{A}_{\tau \to \mu}$ \tagDc{} \\
$\tau \to e\gamma$ & $< 3.3 \times 10^{-8}$ & \tagBL{} & LFV; selection rule violation \tagP{} \\
$\tau \to \mu\gamma$ & $< 4.2 \times 10^{-8}$ & \tagBL{} & LFV; selection rule violation \tagP{} \\
$\tau \to eee$ & $< 2.7 \times 10^{-8}$ & \tagBL{} & Mode mismatch hypothesis \tagP{} \\
\bottomrule
\end{tabular}
\end{table}

\textbf{Note:} The near-equality of BR($\tau \to e$) and BR($\tau \to \mu$)
is an observational fact \tagBL{}. We do not claim to derive this ratio;
explaining it would require a quantitative theory of mode-spectrum
branching (open).

% ==============================================================================
% THRESHOLD GATES
% ==============================================================================

\subsubsection{Threshold Gates in the Projection Operator}

The tau case illustrates how $\mathcal{P}_{\mathrm{energy}}$ acts as a threshold gate:

\begin{table}[htbp]
\centering
\caption{Tau decay channels and energy thresholds \tagBL{}}
\label{tab:tau-thresholds}
\begin{tabular}{lccc}
\toprule
\textbf{Channel} & \textbf{Threshold} & \textbf{Status} & \textbf{BR} \\
\midrule
$\tau \to e + \nu\bar\nu$ & $m_e \approx 0.5$ MeV & Open & 17.8\% \\
$\tau \to \mu + \nu\bar\nu$ & $m_\mu \approx 106$ MeV & Open & 17.4\% \\
$\tau \to \pi + \nu$ & $m_\pi \approx 140$ MeV & Open & 10.8\% \\
$\tau \to \rho + \nu$ & $m_\rho \approx 775$ MeV & Open & 25.5\% \\
\bottomrule
\end{tabular}
\end{table}

All listed thresholds are below $m_\tau \approx 1777$ MeV, so all channels are
kinematically allowed \tagBL{}. The branching ratios then depend on phase space and mode
overlaps (open).

% ==============================================================================
% MODE-SPECTRUM BRANCHING HYPOTHESIS
% ==============================================================================

\subsubsection{Mode-Spectrum Branching Hypothesis}

\begin{edcPostulateBox}{Mode-Spectrum Branching (open)}{[P]}
The branching fractions for tau decay are determined by the
\emph{spectral overlap} between the initial tau mode and the allowed
final-state mode configurations. Schematically:
\begin{equation}
    \mathrm{BR}(\tau \to X) \propto |\langle \Psi_X | \hat{T} | \Psi_\tau \rangle|^2
    \label{eq:tau-spectral-overlap}
\end{equation}
where $\hat{T}$ is a transition operator and $\Psi_X$ represents the
final-state mode configuration.
\end{edcPostulateBox}

\textbf{Physical Narration:}
\begin{enumerate}[nosep]
    \item \textbf{5D cause:} The tau mode $\Psi_\tau$ has a specific profile
          in the brane-layer spectrum.
    \item \textbf{Brane response:} The transition operator $\hat{T}$ couples
          $\Psi_\tau$ to final-state configurations; the coupling strength
          depends on spectral overlap.
    \item \textbf{3D output:} Branching fractions reflect these overlaps,
          filtered through $\mathcal{P}_{\mathrm{frozen}}$.
\end{enumerate}

\begin{tcolorbox}[edcWarning, title=\textbf{Non-Overclaim Reminder}]
Equation~\eqref{eq:tau-spectral-overlap} is a \emph{schematic} representation
\tagP{}. We have not derived the form of $\hat{T}$ or the mode
wavefunctions from the 5D action. The claim is that branching fractions
\emph{can be understood} in terms of spectral structure—not that we have
computed them.
\end{tcolorbox}

\paragraph{\texorpdfstring{Why are BR($\tau \to e$) and BR($\tau \to \mu$) nearly equal?}{Why are BR(tau to e) and BR(tau to mu) nearly equal?}}
This is an \textbf{open question} (open). Possible framings within EDC:
\begin{itemize}[nosep]
    \item The electron and muon final states have similar spectral overlap
          with the tau initial state (modulo phase-space corrections).
    \item The mode-spectrum structure is approximately ``democratic'' for
          leptonic channels.
    \item Detailed calculation requires knowledge of $\hat{T}$ and brane-layer
          wavefunctions.
\end{itemize}

% ==============================================================================
% CHIRAL FILTER HOOK
% ==============================================================================

\subsubsection{Chiral Filter: Same Mechanism as Muon}

As in \S\ref{subsec:muon_story}, the frozen projection operator includes a chiral
filter component:

\begin{equation}
    \mathcal{P}_{\mathrm{frozen}} = \mathcal{P}_{\mathrm{energy}} \circ
    \mathcal{P}_{\mathrm{mode}} \circ \mathcal{P}_{\mathrm{chir}}
    \label{eq:tau-projection-stack}
\end{equation}

The chirality selection pattern for tau decay is identical to muon decay:
\begin{itemize}[nosep]
    \item $\ell^-$ ($e^-$ or $\mu^-$): predominantly left-handed
    \item $\nu_\tau$: left-handed
    \item $\bar{\nu}_\ell$: right-handed
\end{itemize}

This universality across $\mu$ and $\tau$ supports the hypothesis that
chirality selection is a \emph{boundary property}, not specific to the
decaying particle.

\begin{tcolorbox}[mechanism, title={Chiral Filter (Hypothesis)}]
\textbf{Hypothesis} \tagP{}\textbf{:} We propose that the observed
chirality pattern in tau leptonic decays (left-handed charged leptons,
left-handed neutrinos, right-handed antineutrinos) is \emph{consistent with}
a geometric chiral filter at the observer-facing brane boundary.

\medskip
The derivation of $\mathcal{P}_{\mathrm{chir}}$ from 5D boundary conditions
remains (open).
\end{tcolorbox}

% ==============================================================================
% LEDGER CLOSURE
% ==============================================================================

\subsubsection{Ledger Closure (Structural)}

\begin{edcLedgerBox}{Tau bookkeeping (structural)}{[Dc]}
\begin{equation}
m_\tau c^2 = \sum_i E_i + E_{\mathrm{soft}} + E_{\mathrm{recoil}} + E_{\mathrm{bulk,res}},
\end{equation}
where the sum runs over the energies of observer-facing allowed outputs produced by $\mathcal{P}_{\mathrm{frozen}}$.
\end{edcLedgerBox}

% ==============================================================================
% UNIVERSALITY CLAIM
% ==============================================================================

\subsubsection{Generalization Without New Ontology}

\begin{tcolorbox}[mechanism, title={Universality Claim}]
\textbf{Claim} \tagDc{}: The tau decay mechanism is structurally identical to
the muon decay mechanism. The only differences are:
\begin{enumerate}[nosep]
  \item Higher mode energy (larger mass)
  \item More open kinematic channels
  \item Non-zero mode overlap with hadronic sector
\end{enumerate}
The pipeline structure (absorption $\to$ dissipation $\to$ release) is unchanged.
\end{tcolorbox}

The fact that the same framework accommodates both muon (Companion M)
and tau (Companion T) decay without contradiction is a non-trivial
consistency check:

\begin{itemize}[nosep]
    \item \textbf{Same ontology:} Brane-dominant excitation (higher mode index)
    \item \textbf{Same pipeline:} Absorption$\to$Dissipation$\to$Release
    \item \textbf{Same projection:} $\mathcal{P}_{\mathrm{frozen}} =
          \mathcal{P}_{\mathrm{energy}} \circ \mathcal{P}_{\mathrm{mode}}
          \circ \mathcal{P}_{\mathrm{chir}}$
    \item \textbf{Same chirality pattern:} Universal across lepton sector
\end{itemize}

% ==============================================================================
% FALSIFIABILITY HOOKS
% ==============================================================================

\subsubsection{Falsifiability Hooks}

\begin{tcolorbox}[falsifiability, title=\textbf{Falsifiability: What Would Refute This Framing?}]
\begin{enumerate}[nosep]
    \item \textbf{Wrong allowed outputs:} If tau leptonic decay produced
          particles outside $\mathcal{A}_{\tau \to e}$ or $\mathcal{A}_{\tau \to \mu}$
          at observable rates, the selection rule mechanism fails.

    \item \textbf{Ledger non-closure:} If energy accounting showed a deficit
          not attributable to $E_{\mathrm{other}}$ (soft modes, residuals),
          the pipeline would be falsified.

    \item \textbf{Inconsistent chirality:} If tau decay showed a different
          chirality pattern than muon decay, the universal chiral-filter
          hypothesis fails.

    \item \textbf{Bulk leakage evidence:} If tau decay deposited measurable
          energy into bulk modes, the brane-dominant ontology would be
          falsified.

    \item \textbf{Pipeline failure for $\tau$ but not $\mu$:} If the same
          absorption$\to$dissipation$\to$release framework could not
          accommodate both leptons, the generalization claim fails.

    \item \textbf{Threshold violation:} If a decay channel is open that
          should be kinematically forbidden, the framework fails.
\end{enumerate}
\end{tcolorbox}

% ==============================================================================
% OPEN PROBLEMS
% ==============================================================================

\subsubsection{Open Problems}

\begin{enumerate}[nosep]
    \item \textbf{Derive $\mathcal{P}_{\mathrm{chir}}$ from boundary conditions}
          (open): Construct the chiral filter from 5D action + BC at
          $y = +\delta/2$.

    \item \textbf{Explain BR($\tau \to e$) $\approx$ BR($\tau \to \mu$)}
          (open): Derive from mode-spectrum structure without fitting.

    \item \textbf{Mode index quantification} (open): Derive the
          relationship $m_\ell(n_\ell)$ from brane-layer spectrum.

    \item \textbf{Hadronic tau decays} (open): Extend to channels like
          $\tau \to \pi\nu$, which requires pion ontology (\S\ref{sec:case_pion}).

    \item \textbf{Lifetime from first principles} (open): Currently
          $\tau_\tau$ is \tagBL{}; deriving it requires quantitative
          mode-spectrum dynamics.
\end{enumerate}

% ==============================================================================
% CANONICAL GLOSSARY
% ==============================================================================

\subsubsection{Canonical Glossary for Tau Decay}

\begin{tcolorbox}[edcCanonical, title=\textbf{Canonical Terms: Tau Decay Pipeline}]
\begin{description}[nosep, leftmargin=!, labelwidth=4cm]
\item[Mode index $n_\ell$] Qualitative ordering of charged leptons in brane spectrum
\item[$\mathcal{A}_{\tau \to e}$] Allowed output set: $\{e^-, \bar{\nu}_e, \nu_\tau\}$
\item[$\mathcal{A}_{\tau \to \mu}$] Allowed output set: $\{\mu^-, \bar{\nu}_\mu, \nu_\tau\}$
\item[Spectral overlap] Matrix element determining branching fractions
\item[Higher-mode excitation] Tau as heavier brane-dominant mode than muon
\item[Threshold gate] $\mathcal{P}_{\mathrm{energy}}$ component that opens channels
\item[Democratic branching] Hypothesis: near-equal BR for $e$ and $\mu$ channels
\end{description}
\end{tcolorbox}


