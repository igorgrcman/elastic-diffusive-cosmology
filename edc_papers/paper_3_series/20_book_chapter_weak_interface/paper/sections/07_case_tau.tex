% ==============================================================================
% Case Study III: Tau Decay as Higher-Mode Brane Excitation
% ==============================================================================

\subsection{Tau Decay: Higher-Mode Brane Excitation and Universality Check}
\label{sec:tau_story}

The tau is treated as the same ontological class as the muon: a brane-dominant excitation,
but at higher energy in the brane mode spectrum. This is the cleanest test of universality:
if the same absorption--dissipation--release mechanism works for both $\mu$ and $\tau$ without
inventing new ingredients, the weak-sector brane interface hypothesis gains structural credibility.

\subsubsection{Ontology and Minimal Hypothesis}

\begin{edcDefinitionBox}{Tau as higher-mode brane excitation}{[Def]/[P]}
We treat $\Psi_\tau$ as a brane-dominant excitation analogous to $\Psi_\mu$,
differing primarily by excitation energy scale and accessible decay channels.
No new ontological category is introduced.
\end{edcDefinitionBox}

\paragraph{Baseline observables.}
The tau has multiple decay channels \tagBL{}:
\begin{align}
\tau^- &\to e^- + \bar\nu_e + \nu_\tau
  && \text{BR} \approx 17.8\% \label{eq:tau_e_story} \\
\tau^- &\to \mu^- + \bar\nu_\mu + \nu_\tau
  && \text{BR} \approx 17.4\% \label{eq:tau_mu_story} \\
\tau^- &\to \text{hadrons} + \nu_\tau
  && \text{BR} \approx 64.8\% \label{eq:tau_had_story}
\end{align}
with lifetime $\tau_\tau \approx 2.903 \times 10^{-13}$ s \tagBL{}.

\subsubsection{Pipeline and Ledger}

\begin{figure}[ht]
\centering
% figures/fig_tau_process_pipeline.tex
% Tau decay process pipeline diagram (brane-dominant, multiple channels)
\begin{tikzpicture}[scale=0.85, transform shape]

% Load styles
% tikz_style_edc.tex — Reusable TikZ styles for EDC papers
% Version 1.0 — 2026-01-20
% Include via: % tikz_style_edc.tex — Reusable TikZ styles for EDC papers
% Version 1.0 — 2026-01-20
% Include via: \input{tikz_style_edc}

% ============================================================
% REQUIRED LIBRARIES (must be loaded in main document)
% ============================================================
% \usetikzlibrary{calc,angles,quotes,decorations.markings,decorations.pathmorphing,positioning}

% ============================================================
% POSITIONING DEFAULTS
% ============================================================
\tikzset{
    % Default node distances for horizontal/vertical layouts
    edc node distance/.style={node distance=1.6cm and 2.0cm},
    % Compact variant for dense diagrams
    edc compact/.style={node distance=1.2cm and 1.5cm},
    % Spread variant for clarity
    edc spread/.style={node distance=2.0cm and 2.5cm},
}

% ============================================================
% COLOR PALETTE (consistent with epistemic tags)
% ============================================================
\definecolor{edcBulk}{RGB}{220,50,50}        % Red tones for bulk/5D
\definecolor{edcBrane}{RGB}{50,150,50}       % Green tones for brane-layer
\definecolor{edcOutput}{RGB}{50,100,200}     % Blue tones for 3D outputs
\definecolor{edcNeutral}{RGB}{100,100,100}   % Gray for neutral/annotations

% ============================================================
% BOX STYLES
% ============================================================
\tikzset{
    % Generic EDC box (base style)
    edc box/.style={
        rectangle,
        draw,
        rounded corners=3pt,
        minimum width=2.2cm,
        minimum height=0.8cm,
        align=center,
        font=\small,
        inner sep=4pt,
    },
    % Bulk-core box (red family)
    bulk box/.style={
        edc box,
        fill=red!10,
        draw=edcBulk!70!black,
        text=black,
    },
    % Brane-layer box (green family)
    brane box/.style={
        edc box,
        fill=green!10,
        draw=edcBrane!70!black,
        text=black,
    },
    % 3D output box (blue family)
    output box/.style={
        edc box,
        fill=blue!10,
        draw=edcOutput!70!black,
        text=black,
    },
    % Neutral/process box
    process box/.style={
        edc box,
        fill=gray!10,
        draw=gray!60!black,
        text=black,
    },
    % Label-only box (no background)
    label box/.style={
        rectangle,
        rounded corners=2pt,
        draw=gray!40,
        fill=white,
        inner sep=2pt,
        font=\scriptsize,
    },
}

% ============================================================
% ARROW STYLES
% ============================================================
\tikzset{
    % Standard thick arrow
    edc arrow/.style={
        ->,
        >=stealth,
        thick,
    },
    % Emphasized arrow (for main flow)
    edc flow/.style={
        ->,
        >=stealth,
        very thick,
        line width=1.2pt,
    },
    % Dashed arrow (for optional/weak connections)
    edc dashed/.style={
        ->,
        >=stealth,
        thick,
        dashed,
    },
    % Double arrow (for bidirectional)
    edc bidir/.style={
        <->,
        >=stealth,
        thick,
    },
}

% ============================================================
% REGION STYLES (for background fills)
% ============================================================
\tikzset{
    % Bulk region (5D)
    bulk region/.style={
        fill=blue!8,
    },
    % Brane layer region
    brane region/.style={
        fill=yellow!25,
    },
    % Observer/3D region
    observer region/.style={
        fill=green!8,
    },
}

% ============================================================
% LABEL STYLES
% ============================================================
\tikzset{
    % Phase label (below nodes)
    phase label/.style={
        font=\scriptsize\itshape,
        text=black!70,
    },
    % Equation label (for inline math)
    eq label/.style={
        font=\scriptsize,
        fill=white,
        inner sep=1pt,
    },
    % Section annotation
    section label/.style={
        font=\footnotesize\bfseries,
        text=black,
    },
}

% ============================================================
% JUNCTION/PARTICLE STYLES
% ============================================================
\tikzset{
    % Y-junction point
    junction point/.style={
        circle,
        fill=red!60!black,
        minimum size=4pt,
        inner sep=0pt,
    },
    % Flux tube arm
    flux arm/.style={
        thick,
        blue!60!black,
    },
    % Particle dot (electron, etc.)
    particle/.style={
        circle,
        fill=black,
        minimum size=5pt,
        inner sep=0pt,
    },
    % Neutrino (smaller, gray)
    neutrino/.style={
        circle,
        fill=gray,
        minimum size=4pt,
        inner sep=0pt,
    },
}

% ============================================================
% SPRING DECORATION (for mechanical models)
% ============================================================
\tikzset{
    spring/.style={
        thick,
        decorate,
        decoration={
            coil,
            aspect=0.5,
            segment length=2mm,
            amplitude=2mm,
        },
    },
    % Wave decoration (for field modes)
    wave field/.style={
        thick,
        decorate,
        decoration={
            snake,
            amplitude=2pt,
            segment length=8pt,
        },
    },
}

% ============================================================
% BOUNDARY STYLES
% ============================================================
\tikzset{
    % Bulk-facing boundary (dashed red)
    bulk boundary/.style={
        very thick,
        red!70!black,
        dashed,
    },
    % Observer-facing boundary (solid green)
    observer boundary/.style={
        thick,
        green!50!black,
    },
    % Brane edge (orange)
    brane edge/.style={
        thick,
        orange!70!black,
    },
}

% ============================================================
% CONVENIENCE COMMANDS
% ============================================================
% Arrow label (above)
\newcommand{\arrlabel}[1]{\scriptsize #1}
% Arrow label (below)
\newcommand{\arrlabelb}[1]{\scriptsize #1}

% ============================================================
% END OF STYLE FILE
% ============================================================


% ============================================================
% REQUIRED LIBRARIES (must be loaded in main document)
% ============================================================
% \usetikzlibrary{calc,angles,quotes,decorations.markings,decorations.pathmorphing,positioning}

% ============================================================
% POSITIONING DEFAULTS
% ============================================================
\tikzset{
    % Default node distances for horizontal/vertical layouts
    edc node distance/.style={node distance=1.6cm and 2.0cm},
    % Compact variant for dense diagrams
    edc compact/.style={node distance=1.2cm and 1.5cm},
    % Spread variant for clarity
    edc spread/.style={node distance=2.0cm and 2.5cm},
}

% ============================================================
% COLOR PALETTE (consistent with epistemic tags)
% ============================================================
\definecolor{edcBulk}{RGB}{220,50,50}        % Red tones for bulk/5D
\definecolor{edcBrane}{RGB}{50,150,50}       % Green tones for brane-layer
\definecolor{edcOutput}{RGB}{50,100,200}     % Blue tones for 3D outputs
\definecolor{edcNeutral}{RGB}{100,100,100}   % Gray for neutral/annotations

% ============================================================
% BOX STYLES
% ============================================================
\tikzset{
    % Generic EDC box (base style)
    edc box/.style={
        rectangle,
        draw,
        rounded corners=3pt,
        minimum width=2.2cm,
        minimum height=0.8cm,
        align=center,
        font=\small,
        inner sep=4pt,
    },
    % Bulk-core box (red family)
    bulk box/.style={
        edc box,
        fill=red!10,
        draw=edcBulk!70!black,
        text=black,
    },
    % Brane-layer box (green family)
    brane box/.style={
        edc box,
        fill=green!10,
        draw=edcBrane!70!black,
        text=black,
    },
    % 3D output box (blue family)
    output box/.style={
        edc box,
        fill=blue!10,
        draw=edcOutput!70!black,
        text=black,
    },
    % Neutral/process box
    process box/.style={
        edc box,
        fill=gray!10,
        draw=gray!60!black,
        text=black,
    },
    % Label-only box (no background)
    label box/.style={
        rectangle,
        rounded corners=2pt,
        draw=gray!40,
        fill=white,
        inner sep=2pt,
        font=\scriptsize,
    },
}

% ============================================================
% ARROW STYLES
% ============================================================
\tikzset{
    % Standard thick arrow
    edc arrow/.style={
        ->,
        >=stealth,
        thick,
    },
    % Emphasized arrow (for main flow)
    edc flow/.style={
        ->,
        >=stealth,
        very thick,
        line width=1.2pt,
    },
    % Dashed arrow (for optional/weak connections)
    edc dashed/.style={
        ->,
        >=stealth,
        thick,
        dashed,
    },
    % Double arrow (for bidirectional)
    edc bidir/.style={
        <->,
        >=stealth,
        thick,
    },
}

% ============================================================
% REGION STYLES (for background fills)
% ============================================================
\tikzset{
    % Bulk region (5D)
    bulk region/.style={
        fill=blue!8,
    },
    % Brane layer region
    brane region/.style={
        fill=yellow!25,
    },
    % Observer/3D region
    observer region/.style={
        fill=green!8,
    },
}

% ============================================================
% LABEL STYLES
% ============================================================
\tikzset{
    % Phase label (below nodes)
    phase label/.style={
        font=\scriptsize\itshape,
        text=black!70,
    },
    % Equation label (for inline math)
    eq label/.style={
        font=\scriptsize,
        fill=white,
        inner sep=1pt,
    },
    % Section annotation
    section label/.style={
        font=\footnotesize\bfseries,
        text=black,
    },
}

% ============================================================
% JUNCTION/PARTICLE STYLES
% ============================================================
\tikzset{
    % Y-junction point
    junction point/.style={
        circle,
        fill=red!60!black,
        minimum size=4pt,
        inner sep=0pt,
    },
    % Flux tube arm
    flux arm/.style={
        thick,
        blue!60!black,
    },
    % Particle dot (electron, etc.)
    particle/.style={
        circle,
        fill=black,
        minimum size=5pt,
        inner sep=0pt,
    },
    % Neutrino (smaller, gray)
    neutrino/.style={
        circle,
        fill=gray,
        minimum size=4pt,
        inner sep=0pt,
    },
}

% ============================================================
% SPRING DECORATION (for mechanical models)
% ============================================================
\tikzset{
    spring/.style={
        thick,
        decorate,
        decoration={
            coil,
            aspect=0.5,
            segment length=2mm,
            amplitude=2mm,
        },
    },
    % Wave decoration (for field modes)
    wave field/.style={
        thick,
        decorate,
        decoration={
            snake,
            amplitude=2pt,
            segment length=8pt,
        },
    },
}

% ============================================================
% BOUNDARY STYLES
% ============================================================
\tikzset{
    % Bulk-facing boundary (dashed red)
    bulk boundary/.style={
        very thick,
        red!70!black,
        dashed,
    },
    % Observer-facing boundary (solid green)
    observer boundary/.style={
        thick,
        green!50!black,
    },
    % Brane edge (orange)
    brane edge/.style={
        thick,
        orange!70!black,
    },
}

% ============================================================
% CONVENIENCE COMMANDS
% ============================================================
% Arrow label (above)
\newcommand{\arrlabel}[1]{\scriptsize #1}
% Arrow label (below)
\newcommand{\arrlabelb}[1]{\scriptsize #1}

% ============================================================
% END OF STYLE FILE
% ============================================================


% ─────────────────────────────────────────────────────────────────────────────
% Background regions (brane-dominant, no bulk trigger)
% ─────────────────────────────────────────────────────────────────────────────
\fill[blue!8] (-5.5,0.6) rectangle (6.2,2.2);
\fill[green!8] (-5.5,-1.8) rectangle (6.2,0.6);

% Region labels
\node[font=\scriptsize, blue!50!black] at (-4.8,1.9) {Thick brane layer};
\node[font=\scriptsize, green!50!black] at (-4.8,0.3) {3D outputs};

% ─────────────────────────────────────────────────────────────────────────────
% Brane layer nodes
% ─────────────────────────────────────────────────────────────────────────────
\node[brane box, text width=2.6cm] (tau) at (-3.5,1.4)
  {Tau $\Psi_\tau$\\{\tiny higher-mode brane}};

\node[brane box, text width=2.6cm] (diss) at (0,1.4)
  {Dissipation\\{\tiny $\Gamma_{\mathrm{eff}} E_{\mathrm{brane}}$}};

\node[gate box, text width=2.4cm, minimum height=0.8cm] (frozen) at (3.8,1.4)
  {$\mathcal{P}_{\mathrm{frozen}}$\\{\tiny release gate}};

% ─────────────────────────────────────────────────────────────────────────────
% Output layer nodes - multiple channels
% ─────────────────────────────────────────────────────────────────────────────
% Leptonic channel (e)
\node[output box, text width=1.4cm, font=\scriptsize] (eout) at (-0.5,-0.5)
  {$e^-$};
\node[output box, text width=1.4cm, font=\scriptsize] (nue) at (0.9,-0.5)
  {$\bar\nu_e$};
\node[output box, text width=1.4cm, font=\scriptsize] (nutau1) at (2.3,-0.5)
  {$\nu_\tau$};

% Leptonic channel (mu)
\node[output box, text width=1.4cm, font=\scriptsize] (muout) at (-0.5,-1.3)
  {$\mu^-$};
\node[output box, text width=1.4cm, font=\scriptsize] (numu) at (0.9,-1.3)
  {$\bar\nu_\mu$};
\node[output box, text width=1.4cm, font=\scriptsize] (nutau2) at (2.3,-1.3)
  {$\nu_\tau$};

% Hadronic channel
\node[output box, text width=1.8cm, font=\scriptsize, fill=orange!10] (had) at (4.5,-0.9)
  {hadrons\\{\tiny $\pi,\rho,...$}};
\node[output box, text width=1.4cm, font=\scriptsize] (nutau3) at (5.8,-0.9)
  {$\nu_\tau$};

% ─────────────────────────────────────────────────────────────────────────────
% Arrows
% ─────────────────────────────────────────────────────────────────────────────
\draw[edc flow] (tau) -- (diss);
\draw[edc flow] (diss) -- (frozen);

% Release to outputs - branching
\draw[edc arrow] (frozen.south) -- ++(0,-0.2) -| (1.4,-0.1) -- (1.4,-0.3);
\draw[edc arrow] (frozen.south) -- ++(0,-0.2) -| (1.4,-1.1);
\draw[edc arrow] (frozen.south) -- ++(0,-0.2) -| (had.north);

% Connect outputs within channels
\draw[edc dashed, gray] (eout.east) -- (nue.west);
\draw[edc dashed, gray] (nue.east) -- (nutau1.west);
\draw[edc dashed, gray] (muout.east) -- (numu.west);
\draw[edc dashed, gray] (numu.east) -- (nutau2.west);
\draw[edc dashed, gray] (had.east) -- (nutau3.west);

% ─────────────────────────────────────────────────────────────────────────────
% Annotations
% ─────────────────────────────────────────────────────────────────────────────

% Channel branching note
\node[rectangle, draw=purple!40, fill=purple!5, rounded corners=2pt,
      font=\tiny, align=center, text width=3.8cm] at (-2.0,-0.9)
  {Multiple channels open:\\
   $m_\tau \gg m_\mu, m_\pi, m_\rho$\\
   \textbf{BR}: 17.8\% $e$, 17.4\% $\mu$, 64.8\% had};

% Universality note
\node[rectangle, draw=green!40, fill=green!5, rounded corners=2pt,
      font=\tiny, align=center, text width=2.8cm] at (5.2,0.3)
  {Same pipeline as $\mu$\\
   (universality test)};

\end{tikzpicture}

\caption{Tau decay pipeline. Same mechanism as muon but with multiple channels open
due to larger energy budget ($m_\tau \approx 1777$ MeV). Hadronic channels dominate (64.8\%).}
\label{fig:tau_pipeline}
\end{figure}

At the structural level, the pipeline is identical to muon:
\begin{equation}
\Psi_\tau \;\Rightarrow\; E_{\mathrm{brane}}(t_0)\approx m_\tau c^2 \;\Rightarrow\;
\Gamma_{\mathrm{eff}} \;\Rightarrow\; \mathcal{P}_{\mathrm{frozen}} \;\Rightarrow\; \text{allowed outputs}.
\end{equation}

The key difference is that $\mathcal{P}_{\mathrm{energy}}$ and $\mathcal{P}_{\mathrm{mode}}$ now admit a
broader set of outputs because $m_\tau c^2 \approx 1777$ MeV \tagBL{} provides a much larger energy budget.

\subsubsection{Channel Structure as a Projection Outcome}

Experimentally, tau decays include leptonic and hadronic channels \tagBL{}.
In EDC terms:

\begin{itemize}[nosep]
  \item \textbf{Leptonic release:} $\tau \to \ell + \nu + \bar\nu$ is the direct
        analogue of muon release but at higher energy.
  \item \textbf{Hadronic release:} Channels involving pions/mesons arise because
        the mode spectrum allows composite (junction-pair) outputs once the
        energy threshold is open \tagP{}/\tagOpen{}.
\end{itemize}

This does not yet explain branching ratios; it explains why a single interface
framework can accommodate a multi-channel pattern without inventing separate
mechanisms per channel.

\subsubsection{Threshold Gates in the Projection Operator}

The tau case illustrates how $\mathcal{P}_{\mathrm{energy}}$ acts as a threshold gate:

\begin{center}
\begin{tabular}{lccc}
\toprule
\textbf{Channel} & \textbf{Threshold} & \textbf{Status} & \textbf{BR} \\
\midrule
$\tau \to e + \nu\bar\nu$ & $m_e \approx 0.5$ MeV & Open & 17.8\% \\
$\tau \to \mu + \nu\bar\nu$ & $m_\mu \approx 106$ MeV & Open & 17.4\% \\
$\tau \to \pi + \nu$ & $m_\pi \approx 140$ MeV & Open & 10.8\% \\
$\tau \to \rho + \nu$ & $m_\rho \approx 775$ MeV & Open & 25.5\% \\
\bottomrule
\end{tabular}
\end{center}

All listed thresholds are below $m_\tau \approx 1777$ MeV, so all channels are
kinematically allowed \tagBL{}. The branching ratios then depend on phase space and mode
overlaps \tagOpen{}.

\subsubsection{Ledger Closure (Structural)}

\begin{edcLedgerBox}{Tau bookkeeping (structural)}{[Dc]}
\begin{equation}
m_\tau c^2 = \sum_i E_i + E_{\mathrm{soft}} + E_{\mathrm{recoil}} + E_{\mathrm{bulk,res}},
\end{equation}
where the sum runs over the energies of observer-facing allowed outputs produced by $\mathcal{P}_{\mathrm{frozen}}$.
\end{edcLedgerBox}

\subsubsection{Generalization Without New Ontology}

\begin{tcolorbox}[mechanism, title={Universality Claim}]
\textbf{Claim} \tagDc{}: The tau decay mechanism is structurally identical to
the muon decay mechanism. The only differences are:
\begin{enumerate}[nosep]
  \item Higher mode energy (larger mass)
  \item More open kinematic channels
  \item Non-zero mode overlap with hadronic sector
\end{enumerate}
The pipeline structure (dissipation $\to$ release) is unchanged.
\end{tcolorbox}

\subsubsection{Why Tau Tests Universality}

The tau provides the strongest test of whether the brane-dominant picture is
universal or specific to muons:

\begin{itemize}[nosep]
  \item \textbf{Same ontology}: If tau requires a different ontological category,
        the framework loses predictive power.
  \item \textbf{Same pipeline}: If the three-phase mechanism fails for tau,
        the unified weak-sector claim is falsified.
  \item \textbf{Kinematic consistency}: All open channels must be consistent with
        the threshold structure---no ``forbidden'' channels should appear.
\end{itemize}

\subsubsection{Falsifiability Hooks}

\begin{tcolorbox}[falsifiability]
\begin{itemize}[nosep]
  \item If the brane-dominant ontology cannot accommodate hadronic openings
        without contradictory assumptions, it fails.
  \item If the same $\mathcal{P}_{\mathrm{chir}}$ cannot be consistently applied
        across muon and tau without channel-dependent tuning, it fails.
  \item If a decay channel is open that should be kinematically forbidden
        (threshold violation), the framework fails.
  \item If the $\tau/\mu$ lifetime ratio is not consistent with the mode energy
        interpretation, the excited-mode picture fails.
\end{itemize}
\end{tcolorbox}

