% ==============================================================================
% Section 4: Particle Ontology in EDC
% ==============================================================================

Before diving into case studies, the reader needs a map of ``what is what'' in
EDC's 5D picture. This section classifies the particles of the weak sector by
their ontological status in the thick-brane geometry.

\subsection{Five Ontological Categories}

EDC classifies weak-sector particles into five categories based on their
geometric relationship to the thick brane \tagP{}/\tagDc{}:

\begin{table}[ht]
\centering
\begin{tabular}{llll}
\toprule
\textbf{Category} & \textbf{Examples} & \textbf{5D Character} & \textbf{Dominant Suppression} \\
\midrule
Bulk-core junction & Neutron & Extends into bulk & $\mathcal{P}_{\text{energy}}$ \\
Brane-dominant (fundamental) & $\mu$, $\tau$ & Localized in brane & $\mathcal{P}_{\text{mode}}$ \\
Brane defect & Electron & Ground-mode excitation & None (stable) \\
Edge mode & Neutrino & Interface-localized & $\mathcal{P}_{\text{chir}}$ \\
Composite & Pion & Junction pair & $\mathcal{P}_{\text{chir}}$ \\
\bottomrule
\end{tabular}
\caption{Ontological classification of weak-sector particles in EDC.}
\label{tab:ontology}
\end{table}

\subsection{Bulk-Core Junction: The Neutron}

\begin{tcolorbox}[mechanism, title={Neutron Ontology}]
\textbf{Definition} \tagP{}: The neutron is a \emph{bulk-core junction} configuration.
It is not purely brane-localized: part of its structure extends into the bulk, which
is why its decay involves relaxation of a bulk-facing component.

\textbf{Decay mechanism}: Junction relaxation pumps energy into the brane layer,
which then releases into $\{p, e^-, \bar\nu_e\}$.

\textbf{Key feature}: The neutron's bulk-facing character makes it the natural
``anchor case'' for the weak program: it provides the clearest example of
bulk$\to$brane transfer.
\end{tcolorbox}

\subsection{Brane-Dominant Fundamental: Muon and Tau}

\begin{tcolorbox}[mechanism, title={Muon/Tau Ontology}]
\textbf{Definition} \tagP{}: The muon and tau are \emph{brane-dominant excitations}.
They are localized within the thick brane layer and represent excited modes of the
same fundamental sector that has the electron as its ground state.

\textbf{Decay mechanism}: Mode de-excitation within the brane releases energy into
lower-lying modes plus neutrinos.

\textbf{Key feature}: No hadrons on leading order---the mode mismatch
$\mathcal{P}_{\text{mode}}$ prevents hadronic output for the muon. For the tau,
higher energy opens hadronic channels.
\end{tcolorbox}

The muon and tau are distinguished by their mode index: the tau occupies a higher
excited state, which explains both its larger mass and its additional decay channels.

\subsection{Brane Defect: The Electron}

\begin{tcolorbox}[mechanism, title={Electron Ontology}]
\textbf{Definition} \tagP{}: The electron is the \emph{ground-mode brane defect}.
It is the lowest-energy charged excitation of the brane layer.

\textbf{Stability mechanism}: There is no lower-energy charged state into which
the electron could decay. The ledger cannot close without violating charge
conservation.

\textbf{Key feature}: The electron is stable not because of an inserted conservation
law, but because the thick-brane mode structure has no lower-lying charged mode.
\end{tcolorbox}

\subsection{Edge Mode: The Neutrino}

\begin{tcolorbox}[mechanism, title={Neutrino Ontology}]
\textbf{Definition} \tagP{}: The neutrino is an \emph{edge mode} localized at the
bulk-brane interface. It does not penetrate deeply into either the bulk or the
brane interior.

\textbf{Weak coupling mechanism}: The neutrino's interface localization means its
overlap with bulk and brane-interior modes is suppressed. This is the geometric
origin of ``weak interactions'' for neutrinos.

\textbf{Chirality}: The interface geometry naturally selects left-handed neutrinos
for coupling. This is encoded in $\mathcal{P}_{\text{chir}}$.
\end{tcolorbox}

\subsection{Composite: The Pion}

\begin{tcolorbox}[mechanism, title={Pion Ontology}]
\textbf{Definition} \tagP{}: The charged pion is a \emph{brane-dominant composite},
modeled as a junction-pair configuration (loosely, a bound $q\bar{q}$ state in
traditional language, but here arising from brane geometry).

\textbf{Decay mechanism}: The junction pair annihilates, releasing energy into
$\ell + \nu$. The chirality projection $\mathcal{P}_{\text{chir}}$ enforces
helicity suppression.

\textbf{Key feature}: Helicity suppression factor $(m_e/m_\mu)^2$ is a baseline
fact \tagBL{}; EDC interprets it as a consequence of boundary conditions
\tagP{}/\tagOpen{}.
\end{tcolorbox}

\subsection{Ontology Map}

The following diagram shows how the five categories relate to the thick-brane
geometry:

\begin{center}
\begin{tikzpicture}[scale=0.9]
% Bulk region
\fill[gray!15] (-5,2) rectangle (5,3.5);
\node[font=\small] at (0,2.75) {Bulk (5D)};

% Brane layer
\fill[blue!10] (-5,0.5) rectangle (5,2);
\draw[thick, blue!50] (-5,0.5) -- (5,0.5);
\draw[thick, blue!50] (-5,2) -- (5,2);
\node[font=\small, blue!60!black] at (0,1.25) {Brane Layer};

% Observer region
\fill[green!10] (-5,-0.5) rectangle (5,0.5);
\node[font=\small, green!50!black] at (0,0) {Observer (3D)};

% Particles
\node[draw, circle, fill=red!30, minimum size=0.8cm, font=\tiny] (n) at (-3.5,2.5) {n};
\draw[->, thick, red!50] (n) -- (-3.5,1.25);
\node[font=\tiny, below] at (-3.5,0.8) {bulk-core};

\node[draw, circle, fill=orange!30, minimum size=0.6cm, font=\tiny] (mu) at (-1.5,1.25) {$\mu$};
\node[draw, circle, fill=orange!30, minimum size=0.6cm, font=\tiny] (tau) at (-0.5,1.25) {$\tau$};
\node[font=\tiny, below] at (-1,0.8) {brane-dom};

\node[draw, circle, fill=green!30, minimum size=0.6cm, font=\tiny] (e) at (1,1.25) {$e$};
\node[font=\tiny, below] at (1,0.8) {defect};

\node[draw, circle, fill=purple!30, minimum size=0.6cm, font=\tiny] (nu) at (2.5,1.75) {$\nu$};
\node[font=\tiny, right] at (2.9,1.75) {edge};

\node[draw, ellipse, fill=yellow!30, minimum width=1cm, minimum height=0.5cm, font=\tiny] (pi) at (4,1.25) {$\pi$};
\node[font=\tiny, below] at (4,0.8) {composite};
\end{tikzpicture}
\end{center}

Each category appears in its characteristic location: bulk-core junctions extend
into the bulk; brane-dominant modes and defects are localized within the brane;
edge modes sit at the interface; composites are structured configurations within
the brane.
