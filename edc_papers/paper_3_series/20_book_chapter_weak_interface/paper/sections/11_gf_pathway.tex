% ==============================================================================
% Section 1.10: Structural Pathway to G_F
% ==============================================================================

\section{Structural Pathway to $G_F$}
\label{sec:gf_pathway}

This section addresses a central question: \emph{where does the effective coupling
strength come from?} The previous case studies assumed that weak interactions
``happen'' with certain rates; now we examine the structural origin of the
effective coupling constant $G_F$.

\subsection{Why We Can Talk About $G_F$ Without Claiming a Derivation}

In EDC we treat the ``weak strength'' not as a fundamental vertex constant but
as a low-energy residue of mediator exchange within the thick brane. The goal
is to establish the \emph{structural pathway} without numerical closure.

\paragraph{The target.}
The Fermi constant is the empirical coupling strength of weak interactions \tagBL{}:
\begin{equation}
G_F = 1.1663787(6) \times 10^{-5}~\text{GeV}^{-2}.
\label{eq:GF_value}
\end{equation}

In the Standard Model, $G_F$ arises from $W$-boson exchange:
\begin{equation}
G_F = \frac{\sqrt{2}}{8} \frac{g^2}{M_W^2},
\label{eq:GF_SM}
\end{equation}
where $g$ is the $SU(2)_L$ gauge coupling and $M_W \approx 80.4$ GeV \tagBL{}.

The question for EDC is: what is the 5D structural interpretation of this result?

\subsection{Mediator Exchange as Thick-Brane Transfer}

Introduce a mediator field $\phi$ supported in the brane layer:
\begin{equation}
\mathcal{L}_\phi = \frac{1}{2}(\partial\phi)^2 - \frac{1}{2}m_\phi^2 \phi^2,
\end{equation}
coupled to a brane-facing current $J(x)$ at the Plenum-facing side (schematically
$y = -\delta/2$):
\begin{equation}
\mathcal{L}_{\text{int}} = g_5 \, J(x) \, \phi(x, y = -\delta/2).
\end{equation}

Integrating out $\phi$ at tree level yields the effective contact interaction:
\begin{equation}
\boxed{
\mathcal{L}_{\text{eff}} = -\frac{g_5^2}{2 m_\phi^2} \,
\mathcal{O}_{\text{overlap}} \, J(x) J(x)
}
\label{eq:Leff_contact}
\end{equation}
where $\mathcal{O}_{\text{overlap}}$ represents suppression from mode
localization/overlap in the thick brane/(open).

\subsection{Canonical Physical Interpretation}

\begin{tcolorbox}[mechanism, title={What Eq.~\eqref{eq:Leff_contact} Is and Is Not}]
Equation~\eqref{eq:Leff_contact} is \textbf{not} a fundamental ``weak vertex'';
it is the low-energy residue of a 5D bulk$\to$brane transfer process \tagDc{}.

The apparent smallness of the coupling is therefore \textbf{geometric suppression}---set
by the mediator gap $m_\phi$, mode-profile overlap $\mathcal{O}_{\text{overlap}}$,
and boundary/projection factor
$\mathcal{O}_{\text{BC}} = \mathcal{O}(\mathcal{P}_{\text{frozen}},
\mathcal{P}_{\text{chir}})$---rather than a tunable parameter \tagDc{}.
\end{tcolorbox}

\subsection{Effective Coupling Scale}

Define:
\begin{equation}
g_{\text{eff}} \equiv g_5 \times \mathcal{O}_{\text{overlap}} \times
\mathcal{O}_{\text{BC}},
\end{equation}
so that:
\begin{equation}
\boxed{
G_{\text{EDC}} \sim \frac{g_{\text{eff}}^2}{m_\phi^2}
}
\label{eq:GEDC_scaling}
\end{equation}
with $[G_{\text{EDC}}] = [E]^{-2}$ as required.

\subsection{Connecting to the Standard Model}

The EDC interpretation does not contradict the Standard Model formula
\eqref{eq:GF_SM}. Rather, it provides a structural context:

\begin{center}
\begin{tabular}{ll}
\toprule
\textbf{SM Parameter} & \textbf{EDC Structural Interpretation} \\
\midrule
$g$ (gauge coupling) & Fundamental 5D coupling $g_5$ \\
$M_W$ (boson mass) & Mediator mass $m_\phi$, set by brane thickness \\
$G_F$ (Fermi constant) & Effective 4D coupling after integration \\
\bottomrule
\end{tabular}
\end{center}

The key insight is that $G_F$ is not a fundamental constant but an \emph{effective}
parameter that emerges from 5D structure \tagP{}.

\subsection{What Remains OPEN (Numerical Closure, Not Narrative)}

The pathway becomes predictive only when the following are explicitly computed:
\begin{enumerate}[nosep]
  \item[(i)] Mode profiles in the thick brane
  \item[(ii)] KK spectrum yielding $m_\phi$
  \item[(iii)] Evaluation of $\mathcal{O}_{\text{BC}}$ from the frozen/chiral operators
\end{enumerate}
Until then, we do not claim a numerical $G_F$ derivation.

\subsection{Process Diagram: Mediator Exchange}

\begin{center}
\begin{tikzpicture}[scale=0.85]

% Coordinate labels
\node[font=\small] at (-4.5,3) {$y$ (extra dim)};
\node[font=\small] at (4.5,-0.5) {$x$ (brane)};

% Brane layer
\fill[blue!10] (-4,0) rectangle (4,2.5);
\draw[thick, blue!50] (-4,2.5) -- (4,2.5);
\draw[thick, blue!50] (-4,0) -- (4,0);

% Current 1 at y1
\draw[thick, red!70] (-2.5,0.8) ellipse (0.4 and 0.6);
\node[font=\scriptsize, red!70!black] at (-2.5,0.8) {$J_1$};
\draw[dashed, red!30] (-2.5,0) -- (-2.5,2.5);
\node[font=\tiny, red!50!black] at (-2.5,-0.3) {$y_1$};

% Current 2 at y2
\draw[thick, green!70!black] (2.5,1.6) ellipse (0.4 and 0.5);
\node[font=\scriptsize, green!60!black] at (2.5,1.6) {$J_2$};
\draw[dashed, green!30!black] (2.5,0) -- (2.5,2.5);
\node[font=\tiny, green!50!black] at (2.5,-0.3) {$y_2$};

% Mediator propagation
\draw[thick, purple, decorate, decoration={snake, amplitude=2pt, segment length=6pt}]
  (-2.1,0.8) -- (2.1,1.6);
\node[font=\scriptsize, purple!70!black] at (0,1.5) {$\phi$};

% Overlap region (shaded)
\fill[yellow!30, opacity=0.5] (-0.5,0.5) rectangle (0.5,2);
\node[font=\tiny, rotate=90] at (0.7,1.25) {overlap};

% Effective vertex after integration
\draw[-{Stealth}, thick] (5,1.25) -- (6.5,1.25);
\node[font=\scriptsize] at (5.75,1.6) {integrate};
\node[font=\scriptsize] at (5.75,0.9) {out $\phi$};

% Effective 4-fermion
\node[rectangle, rounded corners=4pt, draw=purple!60!black, fill=purple!10,
      minimum width=1.5cm, minimum height=1cm, font=\scriptsize, align=center]
  at (8,1.25) {$\mathcal{L}_{\text{eff}}$\\$\propto G_F J J$};

\end{tikzpicture}
\end{center}

\subsection{Honest Statement About $G_F$}

\begin{tcolorbox}[guardrail, title={Status of $G_F$ Derivation}]
\textbf{What is baseline} \tagBL{}: The value of $G_F$ and its expression in
terms of SM parameters ($g$, $M_W$) are established experimental and theoretical
results.

\textbf{What EDC provides} \tagP{}:
\begin{itemize}[nosep]
  \item A structural interpretation: $G_F$ emerges from integrating out a
        5D mediator
  \item A geometric origin for smallness: overlap suppression
  \item A connection to brane geometry: thickness sets the mediator mass scale
\end{itemize}

\textbf{What remains open} (open):
\begin{itemize}[nosep]
  \item Explicit computation of the overlap integral
  \item Determination of the mediator mass from first principles
  \item Derivation of the coupling $g_5$ from 5D action
  \item Quantitative recovery of the numerical value of $G_F$
\end{itemize}

We do not claim that EDC ``derives'' $G_F$ from first principles. We claim
that EDC provides a framework in which the structure of weak coupling has a
geometric interpretation.
\end{tcolorbox}

\subsection{Falsifiability Hooks}

\begin{tcolorbox}[falsifiability]
\begin{itemize}[nosep]
  \item If the overlap integral computation gives $G_F$ that differs from
        experiment by more than reasonable theoretical uncertainty, the
        framework fails.
  \item If the mediator mass required is inconsistent with other brane
        parameters (thickness, tension), the framework is overconstrained.
  \item If the $y$-profiles cannot reproduce the observed pattern of weak
        couplings across generations, the geometric interpretation fails.
  \item If the hierarchy cannot be explained without fine-tuning the 5D
        parameters, the structural advantage is lost.
\end{itemize}
\end{tcolorbox}

