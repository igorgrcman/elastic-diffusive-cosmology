% ==============================================================================
% Case Study VI: Neutrino as Edge Mode and Ledger Partner
% ==============================================================================

\subsection{Neutrino: The Edge Mode and Ledger Partner}
\label{sec:case_neutrino}

\subsubsection{What Is the Neutrino in EDC Ontology?}

\textbf{Ontology} \tagP{}/\tagDc{}: Neutrinos are treated as \emph{edge modes}
at the bulk--brane interface---neutral excitations that naturally appear as
ledger partners when charged outputs are released.

They are also the natural seat of the chirality filter interpretation: if
$\mathcal{P}_{\text{chir}}$ is a boundary phenomenon, then neutrinos are not
optional add-ons; they are the neutral channel that makes the boundary projection
physically meaningful.

\paragraph{Baseline observables.}
Neutrino properties from experiment \tagBL{}:
\begin{itemize}[nosep]
  \item Very small mass: $m_\nu \lesssim 1$ eV (oscillation data)
  \item Only left-handed neutrinos couple to weak interactions
  \item Extremely weak interactions (mean free paths of astronomical scale)
\end{itemize}

\subsubsection{Why Neutrinos Interact Weakly}

In the Standard Model, neutrinos interact only via $W^\pm$ and $Z^0$ exchange,
which is suppressed by the large gauge boson masses \tagBL{}.

In EDC, the interpretation is geometric \tagP{}/\tagDc{}:

\begin{tcolorbox}[mechanism, title={Neutrino Weak Coupling}]
\textbf{Claim}: The neutrino's edge-mode localization means its wavefunction
has suppressed overlap with bulk modes and brane-interior modes.

\textbf{Consequence}: The effective coupling of neutrinos to other particles
is controlled by overlap integrals:
\begin{equation}
g_{\nu,\text{eff}} \propto \int_{\text{brane}} \psi_\nu^*(y) \cdot
\psi_{\text{other}}(y) \cdot \phi_{\text{mediator}}(y) \, dy,
\end{equation}
which is suppressed because $\psi_\nu(y)$ is localized at the edge while
other particles are localized in the interior.
\end{tcolorbox}

This provides a geometric origin for ``weak interactions'': they are weak
because of suppressed overlap, not because of a small fundamental coupling.

\subsubsection{Chirality Selection: Left-Handed Only}

A striking feature of neutrinos is that only left-handed neutrinos (and
right-handed antineutrinos) couple to weak interactions \tagBL{}.

In EDC, this is encoded in $\mathcal{P}_{\text{chir}}$ as a boundary effect
\tagP{}/\tagOpen{}:

\paragraph{Physical picture.}
The bulk-brane interface imposes boundary conditions on spinor fields. These
boundary conditions select a particular chirality for the edge mode. The
``other'' chirality (right-handed neutrino) either:
\begin{enumerate}[nosep]
  \item Does not satisfy the boundary conditions (is projected out), or
  \item Has a different localization (propagates into the bulk) and thus
        does not appear as a 3D edge mode.
\end{enumerate}

\paragraph{Open question.}
The explicit boundary-condition calculation that produces this selection
remains \tagOpen{}. The claim is structural: $\mathcal{P}_{\text{chir}}$
encodes chirality selection.

\subsubsection{Neutrino Mass: An Edge-Mode Energy}

If neutrinos are edge modes, their mass should be related to the edge-mode
energy in the thick-brane geometry \tagP{}/\tagOpen{}.

\paragraph{Expected scaling.}
Edge modes typically have energies suppressed relative to interior modes.
This is consistent with $m_\nu \ll m_e$.

\paragraph{Open problem.}
Deriving the neutrino mass scale (sub-eV) from the edge-mode spectrum requires
solving the mode equation with appropriate boundary conditions \tagOpen{}.

\subsubsection{Role as Ledger Closure Partner}

In every weak decay, neutrinos appear as the ``missing'' particles that carry
away energy and lepton number:
\begin{itemize}[nosep]
  \item Neutron: $n \to p + e^- + \bar\nu_e$
  \item Muon: $\mu^- \to e^- + \bar\nu_e + \nu_\mu$
  \item Pion: $\pi^+ \to \mu^+ + \nu_\mu$
\end{itemize}

This pattern is not accidental. The neutrino is the ``minimal-energy neutral
partner'' required to close the ledger while conserving lepton number \tagDc{}.

\subsubsection{Generative Closure Principle (Complete)}
\label{sec:generative_closure_principle}

Together with the electron, neutrinos complete the \emph{Generative Closure
Principle}:

\begin{tcolorbox}[mechanism, title={Generative Closure Principle}]
\textbf{Postulate} \tagP{}/\tagDc{}: A stable universe-like output sector requires:
\begin{enumerate}[nosep]
  \item An electron (charged) defect sector
  \item Excited states of that sector (to allow cascades and composites)
  \item Ledger closure via neutral edge modes (neutrinos)
\end{enumerate}
so that energy can be transferred, redistributed, and released without
violating conservation or producing uncontrolled leakage.

\textbf{Guardrail}: This does not claim a full SM derivation. It asserts a
mechanism-level closure requirement and leaves explicit constructive
derivations as \tagOpen{}.
\end{tcolorbox}

\subsubsection{Process Diagram: Neutrino Localization}

\begin{center}
\begin{tikzpicture}[scale=0.85]

% Bulk region
\fill[gray!15] (-4,2) rectangle (4,3);
\node[font=\small] at (0,2.5) {Bulk};

% Brane layer
\fill[blue!10] (-4,0.5) rectangle (4,2);
\draw[thick, blue!50] (-4,2) -- (4,2);
\draw[thick, blue!50] (-4,0.5) -- (4,0.5);
\node[font=\small, blue!60!black] at (0,1.25) {Brane Interior};

% Edge region (where neutrino lives)
\fill[purple!15] (-4,1.7) rectangle (4,2);
\draw[thick, purple!50, dashed] (-4,1.7) -- (4,1.7);
\node[font=\scriptsize, purple!60!black] at (3,1.85) {Edge};

% Observer region
\fill[green!10] (-4,-0.2) rectangle (4,0.5);
\node[font=\small, green!50!black] at (0,0.15) {Observer (3D)};

% Neutrino wavefunction
\draw[thick, purple] plot[smooth, domain=-3:3] (\x, {1.85 + 0.1*exp(-\x*\x)});
\node[font=\scriptsize, purple] at (-2.5,2.2) {$\psi_\nu(y)$};

% Electron wavefunction (for comparison)
\draw[thick, green!60!black] plot[smooth, domain=-3:3]
  (\x, {1.25 + 0.3*exp(-(\x-0.5)*(\x-0.5)/2)});
\node[font=\scriptsize, green!50!black] at (2,1.5) {$\psi_e(y)$};

% Overlap region annotation
\draw[{Stealth}-{Stealth}, thick, red!50] (-1,1.7) -- (-1,1.4);
\node[font=\tiny, red!50!black, right] at (-0.9,1.55) {small overlap};

% Ledger annotation
\node[rectangle, draw=gray, rounded corners=2pt, fill=gray!5,
      font=\tiny, align=center, text width=2.5cm] at (-2.5,-0.8)
  {Neutrinos close the\\ledger in all weak decays};

\end{tikzpicture}
\end{center}

\subsubsection{Falsifiability Hooks}

\begin{tcolorbox}[falsifiability]
\begin{itemize}[nosep]
  \item If right-handed neutrinos are observed coupling to weak interactions
        at comparable strength to left-handed, the chirality-selection claim
        fails.
  \item If neutrino masses are found to be much larger than sub-eV (outside
        oscillation constraints), the edge-mode suppression picture requires
        revision.
  \item If neutrino interactions are found to be stronger than geometric
        overlap predicts, the edge-mode interpretation fails.
  \item If a decay is observed that violates lepton number without neutrinos,
        the ledger-closure role is undermined.
\end{itemize}
\end{tcolorbox}

