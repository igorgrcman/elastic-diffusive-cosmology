% ==============================================================================
% RESEARCH TARGET RT-CH3-001: V-A Structure from Plenum Inflow
% ==============================================================================
% Status: OPEN
% Priority: HIGH (foundational for weak sector)
% Dependencies: Ch1 pipeline, Ch2 Z₆ geometry
% ==============================================================================

\documentclass[11pt,a4paper]{article}
\usepackage[utf8]{inputenc}
\usepackage{amsmath,amssymb,amsthm}
\usepackage{geometry}
\usepackage{tcolorbox}
\usepackage{enumitem}
\usepackage{booktabs}

\geometry{margin=2.5cm}

\newtheorem{problem}{Problem}
\newtheorem{hypothesis}{Hypothesis}
\newtheorem{criterion}{Success Criterion}

\tcbuselibrary{skins,breakable}
\newtcolorbox{inputbox}[1][]{colback=green!5,colframe=green!50!black,title=#1}
\newtcolorbox{outputbox}[1][]{colback=blue!5,colframe=blue!50!black,title=#1}
\newtcolorbox{forbiddenbox}[1][]{colback=red!5,colframe=red!50!black,title=#1}

\title{\textbf{Research Target RT-CH3-001}\\[0.5em]
\Large Derivation of V$-$A Structure from Plenum Inflow Geometry}
\author{EDC Research Program — Chapter 3}
\date{Target formulated: January 2026}

\begin{document}
\maketitle

% ==============================================================================
\section{Problem Statement}
% ==============================================================================

\begin{problem}[V$-$A from EDC Geometry]
Derive the V$-$A (vector minus axial-vector) structure of weak interactions
as a \textbf{consequence} of 5D EDC geometry, without assuming chirality
selection as an input.
\end{problem}

\subsection{Why This Matters}

In the Standard Model, the V$-$A structure is \emph{put in by hand}:
the weak interaction couples only to left-handed fermions by construction.
This is an empirical fact encoded in the Lagrangian, not explained.

EDC claims that weak interactions are ``coarse-grained residues of thick-brane
dynamics.'' If true, V$-$A must \emph{emerge} from the geometry—not be
smuggled in as a boundary condition.

\subsection{The Trap to Avoid}

\begin{forbiddenbox}[Forbidden: Disguised Assumption]
The following is \textbf{NOT} a derivation:
\begin{quote}
``Impose $P_L \psi |_{z=0} = 0$ as boundary condition $\Rightarrow$ V$-$A emerges.''
\end{quote}
This merely \emph{relocates} the SM assumption to the boundary. The chirality
selection remains an unexplained input.
\end{forbiddenbox}

% ==============================================================================
\section{What Is INPUT vs OUTPUT}
% ==============================================================================

\begin{inputbox}[Allowed Inputs (EDC Postulates)]
The following are legitimate starting points—they are EDC's foundational
assumptions, established independently of V$-$A:

\begin{enumerate}[label=\textbf{I.\arabic*},nosep]
  \item \textbf{5D Manifold}: $M_5 = M_4 \times [0, L]$ with bulk (Plenum)
        and thick brane layer
  \item \textbf{Plenum Inflow}: Energy flows from bulk INTO brane,
        $J^z_{\text{bulk}\to\text{brane}} > 0$ (sign convention from Framework v2.0)
  \item \textbf{5D Dirac Structure}: Fermions are 5D spinors $\Psi(x^\mu, z)$
        with standard Clifford algebra $\{\gamma^A, \gamma^B\} = 2\eta^{AB}$
  \item \textbf{Thick Brane Potential}: Mass function $m(z)$ or potential $V(z)$
        that localizes modes
  \item \textbf{Frozen Boundary}: Observer-facing edge at $z=0$ is quasi-static
  \item \textbf{$\mathbb{Z}_6$ Lattice}: Hexagonal microstructure at the interface
        (from Chapter 2)
\end{enumerate}
\end{inputbox}

\begin{outputbox}[Required Output (Must Derive)]
The following must \textbf{emerge} from the inputs above:

\begin{enumerate}[label=\textbf{O.\arabic*},nosep]
  \item \textbf{Chirality Selection}: Only one chirality ($L$ or $R$) couples
        effectively to observer-facing processes
  \item \textbf{V$-$A Vertex}: The effective 4D interaction has structure
        $\bar\psi \gamma^\mu (1-\gamma^5) \psi$, not arbitrary combination
  \item \textbf{Helicity Suppression}: Pion decay ratio
        $R_{e/\mu} \propto (m_e/m_\mu)^2$ follows from the same mechanism
\end{enumerate}
\end{outputbox}

% ==============================================================================
\section{Candidate Derivation Strategy}
% ==============================================================================

\begin{hypothesis}[Inflow-Spin Coupling]
The Plenum inflow direction defines a preferred axis in 5D. Combined with
fermion spin, this selects a specific helicity at the observer boundary.
\end{hypothesis}

\subsection{Physical Picture}

\begin{enumerate}
\item \textbf{Plenum flow defines $+z$ direction}

The inflow $J^z > 0$ breaks $z \leftrightarrow -z$ symmetry. This is not
imposed for V$-$A; it's the basic EDC energy-transfer mechanism.

\item \textbf{5D spinor decomposes under flow}

A 5D Dirac spinor $\Psi$ can be decomposed relative to the $z$-direction:
\begin{equation}
\Psi = \Psi_+ + \Psi_-, \qquad \gamma^5_{(4D)} \Psi_\pm = \pm \Psi_\pm
\end{equation}
where $\gamma^5_{(4D)} = i\gamma^0\gamma^1\gamma^2\gamma^3$.

\item \textbf{Inflow couples differently to $\Psi_+$ and $\Psi_-$}

The 5D Dirac equation with $z$-dependent mass:
\begin{equation}
\left( i\gamma^\mu \partial_\mu + i\gamma^5 \partial_z - m(z) \right) \Psi = 0
\end{equation}
The $\gamma^5 \partial_z$ term couples chirality to $z$-momentum.

\item \textbf{Boundary selects surviving mode}

At the frozen boundary $z=0$, require:
\begin{itemize}
  \item Finite energy (no divergence)
  \item Consistent with inflow direction (outgoing = unphysical)
  \item Stable under small perturbations
\end{itemize}

\textbf{Conjecture}: These physical requirements, combined with inflow direction,
force one chirality to dominate.

\end{enumerate}

\subsection{Mathematical Formulation}

\textbf{Step 1: Mode equation}

Separate variables: $\Psi(x,z) = \psi(x) f(z)$. The $z$-dependent part satisfies:
\begin{equation}
\left( \pm \partial_z - m(z) \right) f_\pm(z) = \lambda f_\pm(z)
\end{equation}
where $\pm$ corresponds to chirality.

\textbf{Step 2: Inflow boundary condition}

The Plenum inflow means energy flows in $+z$ direction. For the mode to be
``carried by'' the inflow (physical) vs ``fighting against'' it (unphysical):
\begin{equation}
\text{Physical:} \quad \partial_z f \cdot J^z > 0 \quad \text{(mode flows with Plenum)}
\end{equation}

\textbf{Step 3: Check which chirality survives}

Solve the mode equation with:
\begin{itemize}
  \item Normalizability in the bulk ($z \to L$)
  \item Inflow-compatibility at boundary ($z = 0$)
\end{itemize}

\textbf{Prediction}: Only one chirality has normalizable, inflow-compatible solutions.

% ==============================================================================
\section{Success Criteria}
% ==============================================================================

\begin{criterion}[Chirality Selection]
Starting from inputs I.1--I.6 only, derive that effective 4D fermion modes
at $z=0$ satisfy:
\begin{equation}
P_R \psi_{\text{eff}} = 0 \quad \text{(or $P_L$, whichever is correct)}
\end{equation}
without imposing this as a boundary condition.
\end{criterion}

\begin{criterion}[V$-$A Vertex]
Show that the effective 4D interaction Lagrangian has the form:
\begin{equation}
\mathcal{L}_{\text{eff}} \sim G_F \, \bar\psi_1 \gamma^\mu (1 - \gamma^5) \psi_2 \cdot (\text{other fields})
\end{equation}
where the $(1-\gamma^5)$ structure emerges from mode overlaps, not from input.
\end{criterion}

\begin{criterion}[Helicity Suppression]
Derive that the coupling strength for a fermion of mass $m_\ell$ goes as:
\begin{equation}
|A(\pi \to \ell \nu)|^2 \propto m_\ell^2
\end{equation}
from the boundary mode structure.
\end{criterion}

\begin{criterion}[No Smuggling]
Verify that nowhere in the derivation is chirality selection assumed.
Explicit audit: every boundary condition must be justified from I.1--I.6,
not from ``we need V$-$A.''
\end{criterion}

% ==============================================================================
\section{Falsifiability}
% ==============================================================================

\begin{tcolorbox}[colback=orange!5,colframe=orange!50!black,title=Falsifiability Conditions]
\begin{enumerate}[nosep]
  \item If the inflow direction does NOT select chirality (both $L$ and $R$
        survive equally), the hypothesis fails.
  \item If the selected chirality is WRONG (R instead of L), the model
        is falsified by experiment.
  \item If helicity suppression does NOT emerge as $\propto m_\ell^2$,
        the boundary mechanism is incomplete.
  \item If the derivation requires ANY assumption equivalent to
        ``left-handed fermions couple to weak force,'' it's circular.
\end{enumerate}
\end{tcolorbox}

% ==============================================================================
\section{Relation to Other Open Problems}
% ==============================================================================

\begin{center}
\begin{tabular}{lll}
\toprule
\textbf{Problem} & \textbf{Relation} & \textbf{Dependency} \\
\midrule
OPEN-1 ($G_F$ derivation) & Uses same mode profiles & Parallel \\
OPEN-4 (Helicity suppression) & Direct consequence & Downstream \\
OPEN-5 (Mass hierarchy) & Same mode equation & Parallel \\
Ch2 $\mathbb{Z}_6$ lattice & May constrain spinor coupling & Input \\
\bottomrule
\end{tabular}
\end{center}

% ==============================================================================
\section{Suggested Attack Plan}
% ==============================================================================

\begin{enumerate}
\item \textbf{Phase 1: Toy model}

Solve 5D Dirac equation with simple $m(z) = m_0 \tanh(z/L)$ (domain wall).
Check if inflow direction selects chirality.

\item \textbf{Phase 2: Thick brane profile}

Use realistic thick-brane potential from EDC. Compute mode profiles
$f_L(z)$, $f_R(z)$.

\item \textbf{Phase 3: Boundary analysis}

Analyze which modes are compatible with:
\begin{itemize}
  \item Inflow direction
  \item Frozen boundary condition
  \item $\mathbb{Z}_6$ lattice coupling
\end{itemize}

\item \textbf{Phase 4: Effective vertex}

Compute effective 4D vertex by integrating over $z$:
\begin{equation}
\mathcal{L}_{\text{eff}} = \int_0^L dz \, \bar\Psi \Gamma \Psi \cdot (\text{mediator})
\end{equation}
Check if V$-$A structure emerges.

\item \textbf{Phase 5: Audit}

Independent review: verify no chirality smuggling occurred.

\end{enumerate}

% ==============================================================================
\section{Historical Note}
% ==============================================================================

This research target was formulated on January 21, 2026, during the development
of EDC Part II (The Weak Sector). The key insight was recognizing that
simply imposing chirality as a boundary condition would be ``disguised SM''
rather than a genuine EDC derivation.

The Plenum inflow direction provides a \emph{physical} mechanism for breaking
$L \leftrightarrow R$ symmetry that is independent of weak interaction
phenomenology—it comes from the basic EDC energy-transfer picture.

\vspace{1em}
\hrule
\vspace{0.5em}
\textbf{Status}: OPEN \\
\textbf{Assigned}: Chapter 3 research program \\
\textbf{Target}: Derive V$-$A from geometry, not assumption

\end{document}
