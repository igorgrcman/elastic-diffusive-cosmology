% ==============================================================================
% RESEARCH TARGET RT-CH3-003: Neutron Lifetime from Peierls Barrier
% ==============================================================================
% Status: OPEN
% Priority: HIGH (quantitative validation)
% Dependencies: Ch2 dislocation model
% ==============================================================================

\documentclass[11pt,a4paper]{article}
\usepackage[utf8]{inputenc}
\usepackage{amsmath,amssymb,amsthm}
\usepackage{geometry}
\usepackage{tcolorbox}
\usepackage{enumitem}
\usepackage{booktabs}

\geometry{margin=2.5cm}

\newtheorem{problem}{Problem}
\newtheorem{hypothesis}{Hypothesis}
\newtheorem{criterion}{Success Criterion}

\tcbuselibrary{skins,breakable}
\newtcolorbox{inputbox}[1][]{colback=green!5,colframe=green!50!black,title=#1}
\newtcolorbox{outputbox}[1][]{colback=blue!5,colframe=blue!50!black,title=#1}

\title{\textbf{Research Target RT-CH3-003}\\[0.5em]
\Large Derivation of Neutron Lifetime from Peierls Barrier}
\author{EDC Research Program --- Chapter 3}
\date{Target formulated: January 2026}

\begin{document}
\maketitle

% ==============================================================================
\section{Problem Statement}
% ==============================================================================

\begin{problem}[Neutron Lifetime from Geometry]
Derive the neutron lifetime $\tau_n \approx 880$ s from the $\mathbb{Z}_6$
lattice dislocation model, using the Peierls barrier as the tunneling mechanism.
\end{problem}

\subsection{Why This Matters}

The neutron lifetime is one of the most precisely measured quantities in
nuclear physics, yet it has no first-principles explanation. In EDC:
\begin{itemize}
  \item The neutron is a \emph{dislocation} in the $\mathbb{Z}_6$ lattice (Ch2)
  \item The proton is the stable $\mathbb{Z}_3$ fixed point
  \item $\beta$-decay = dislocation annihilation
\end{itemize}

The Peierls barrier (energy cost to move a dislocation) determines the
tunneling rate and hence the lifetime.

% ==============================================================================
\section{What Is INPUT vs OUTPUT}
% ==============================================================================

\begin{inputbox}[Allowed Inputs (EDC + Ch2)]
\begin{enumerate}[label=\textbf{I.\arabic*},nosep]
  \item \textbf{$\mathbb{Z}_6$ Lattice}: Hexagonal brane microstructure (Ch2, Theorem 3.1)
  \item \textbf{Neutron = Dislocation}: Definition 6.2 from Ch2
  \item \textbf{Burgers Vector}: $|\vec{b}| = a$ (lattice constant)
  \item \textbf{Dislocation Energy}: $\Delta E_{\text{dis}} \approx 1.29$ MeV (Ch2, calibrated)
  \item \textbf{EDC Action}: $S_{\text{EDC}}$ with thick-brane structure
\end{enumerate}
\end{inputbox}

\begin{outputbox}[Required Output (Must Derive)]
\begin{enumerate}[label=\textbf{O.\arabic*},nosep]
  \item \textbf{Peierls Barrier Height}: $V_P$ in appropriate units
  \item \textbf{Tunneling Rate}: $\Gamma = A \exp(-S_{\text{tunnel}}/\hbar)$
  \item \textbf{Lifetime}: $\tau_n = 1/\Gamma \approx 880$ s within factor of 2
\end{enumerate}
\end{outputbox}

% ==============================================================================
\section{Candidate Derivation Strategy}
% ==============================================================================

\begin{hypothesis}[Peierls-Nabarro Tunneling]
The neutron decays when the dislocation tunnels through the Peierls barrier
to annihilate with the lattice, releasing energy as $e^- + \bar\nu_e$.
\end{hypothesis}

\subsection{Physical Picture}

\textbf{Step 1: Peierls barrier from lattice}

In crystallography, the Peierls barrier height is:
\begin{equation}
V_P \approx \frac{2\mu b^2}{(1-\nu)} \exp\left( -\frac{2\pi w}{a} \right)
\end{equation}
where $\mu$ = shear modulus, $b$ = Burgers vector, $\nu$ = Poisson ratio,
$w$ = dislocation width, $a$ = lattice spacing.

\textbf{Step 2: Map to EDC parameters}

The ``shear modulus'' corresponds to flux tube tension. The lattice spacing
$a$ relates to $R_\xi$. Need to identify the mapping.

\textbf{Step 3: Tunneling action}

The semiclassical tunneling action:
\begin{equation}
S_{\text{tunnel}} = \int \sqrt{2m_{\text{eff}} (V(q) - E)} \, dq
\end{equation}

\textbf{Step 4: Compute lifetime}

\begin{equation}
\tau_n = \frac{1}{\Gamma} = \frac{1}{A} \exp\left( \frac{S_{\text{tunnel}}}{\hbar} \right)
\end{equation}

% ==============================================================================
\section{Success Criteria}
% ==============================================================================

\begin{criterion}[Order of Magnitude]
Compute $\tau_n$ and obtain:
\begin{equation}
\tau_n^{\text{EDC}} \in [400, 1800] \text{ s}
\end{equation}
(within factor of 2 of experimental value).
\end{criterion}

\begin{criterion}[No Fine-Tuning]
The barrier height should emerge from lattice parameters, not be fitted.
\end{criterion}

\begin{criterion}[Consistency with Q-value]
The released energy must match:
\begin{equation}
Q_\beta = m_n - m_p - m_e \approx 0.782 \text{ MeV}
\end{equation}
\end{criterion}

% ==============================================================================
\section{Key Challenges}
% ==============================================================================

\begin{enumerate}
\item \textbf{Scale identification}: What is the effective ``mass'' for tunneling?
\item \textbf{Prefactor $A$}: The attempt frequency---needs derivation
\item \textbf{Dislocation width}: How wide is the neutron dislocation?
\item \textbf{Quantum vs classical}: Is tunneling the correct picture?
\end{enumerate}

% ==============================================================================
\section{Connection to Experiment}
% ==============================================================================

Two experimental methods give slightly different values:
\begin{itemize}
  \item Beam method: $\tau_n = 888.0 \pm 2.0$ s
  \item Bottle method: $\tau_n = 878.4 \pm 0.5$ s
\end{itemize}

The ``neutron lifetime puzzle'' (10 s discrepancy) is itself unexplained.
EDC might offer insight if the mechanism depends on measurement geometry.

\vspace{1em}
\hrule
\vspace{0.5em}
\textbf{Status}: OPEN \\
\textbf{Assigned}: Chapter 3 research program \\
\textbf{Target}: Derive $\tau_n \approx 880$ s from Peierls barrier tunneling

\end{document}
