% ==============================================================================
% RT-CH3-003 WORKING DERIVATION: Neutron Lifetime from Peierls Barrier
% ==============================================================================
% Status: IN PROGRESS
% Phase 1: Mapping EDC parameters to elastic constants
% ==============================================================================

\documentclass[11pt,a4paper]{article}
\usepackage[utf8]{inputenc}
\usepackage{amsmath,amssymb}
\usepackage{geometry}
\usepackage{booktabs}
\usepackage{xcolor}

\geometry{margin=2.5cm}

\newcommand{\tagBL}{\textsuperscript{\textcolor{blue}{[BL]}}}
\newcommand{\tagP}{\textsuperscript{\textcolor{orange}{[P]}}}
\newcommand{\tagDc}{\textsuperscript{\textcolor{purple}{[Dc]}}}
\newcommand{\tagCal}{\textsuperscript{\textcolor{red}{[Cal]}}}
\newcommand{\tagOpen}{\textsuperscript{\textcolor{gray}{[OPEN]}}}

\title{RT-CH3-003: Neutron Lifetime Derivation\\Working Document}
\author{EDC Research Program}
\date{January 2026}

\begin{document}
\maketitle

% ==============================================================================
\section{Phase 1: Parameter Mapping}
% ==============================================================================

\subsection{Available EDC Parameters}

\begin{center}
\begin{tabular}{llll}
\toprule
\textbf{Symbol} & \textbf{Value} & \textbf{Units} & \textbf{Status} \\
\midrule
$r_e$ & $10^{-15}$ & m (1 fm) & \tagP \\
$R_\xi$ & $10^{-18}$ & m & \tagP \\
$\sigma r_e^2$ & 5.856 & MeV & \tagDc \\
$E_{\text{disl}}$ & 1.29 & MeV & \tagCal \\
$\Delta m_{np}$ & 1.293 & MeV & \tagBL \\
$\tau_n$ (target) & 879.4 & s & \tagBL \\
\bottomrule
\end{tabular}
\end{center}

\subsection{Key Identifications}

\textbf{Lattice spacing:} \tagP
\begin{equation}
a \equiv r_e \approx 1 \text{ fm} = 10^{-15} \text{ m}
\end{equation}
\textit{Justification:} The topological knot radius $r_e$ is the natural scale for
hadron structure. The flux tube lattice crystallizes at this scale.

\textbf{Burgers vector:} \tagDc
\begin{equation}
|\vec{b}| = a = r_e \approx 1 \text{ fm}
\end{equation}

\subsection{Extracting Shear Modulus from Dislocation Energy}

From Ch2, the dislocation energy formula is:
\begin{equation}
E_{\text{disl}} = \frac{\mu b^2}{4\pi(1-\nu)} \ln\left(\frac{R}{r_0}\right)
\end{equation}

We have $E_{\text{disl}} = 1.29$ MeV \tagCal{} and $b = r_e = 1$ fm.

The logarithmic factor depends on the outer/inner cutoffs:
\begin{itemize}
\item $R$ = system size (or dislocation interaction distance)
\item $r_0$ = dislocation core radius $\sim$ lattice spacing
\end{itemize}

For typical dislocations, $\ln(R/r_0) \sim 5$--$10$. Let's take $\ln(R/r_0) \approx 7$
(corresponding to $R/r_0 \sim 1000$, reasonable for nuclear scales).

Assuming $\nu \approx 0.3$ (typical elastic solid):

\begin{equation}
\mu = \frac{4\pi(1-\nu) E_{\text{disl}}}{b^2 \ln(R/r_0)}
= \frac{4\pi \times 0.7 \times 1.29 \text{ MeV}}{(1 \text{ fm})^2 \times 7}
\end{equation}

\begin{equation}
\boxed{\mu \approx 1.62 \text{ MeV/fm}^2}
\end{equation}

\textit{Cross-check:} Compare with $\sigma r_e^2 = 5.856$ MeV. We have:
\begin{equation}
\mu \cdot r_e^2 = 1.62 \text{ MeV} \quad \text{vs} \quad \sigma r_e^2 = 5.856 \text{ MeV}
\end{equation}
The ratio is $\sim 3.6$, suggesting $\mu \sim \sigma/4$, which is reasonable for
a 2D elastic medium (shear modulus is typically a fraction of bulk modulus).

% ==============================================================================
\section{Phase 2: Peierls Barrier Height}
% ==============================================================================

The Peierls stress (critical stress for dislocation motion) is:
\begin{equation}
\tau_P = \frac{2\mu}{1-\nu} \exp\left(-\frac{2\pi w}{a}\right)
\end{equation}
where $w$ is the dislocation core width.

The Peierls barrier height (energy barrier per unit length) is:
\begin{equation}
V_P = \tau_P \cdot b = \frac{2\mu b}{1-\nu} \exp\left(-\frac{2\pi w}{a}\right)
\end{equation}

\subsection{Estimating Dislocation Width $w$}

The dislocation width depends on the interplanar spacing and bonding:
\begin{itemize}
\item For narrow dislocations (covalent): $w \sim a$
\item For wide dislocations (metallic): $w \sim 5a$--$10a$
\end{itemize}

In the $\mathbb{Z}_6$ lattice, the ``bonding'' comes from flux tube tension.
The relatively high shear modulus suggests narrow dislocations.

\textbf{Ansatz:} $w = \beta \cdot a$ where $\beta \sim 1$--$2$ \tagP

For $\beta = 1.5$ (intermediate):
\begin{equation}
\exp\left(-\frac{2\pi w}{a}\right) = \exp(-3\pi) \approx 7.2 \times 10^{-5}
\end{equation}

Therefore:
\begin{equation}
V_P = \frac{2 \times 1.62 \text{ MeV/fm}^2 \times 1 \text{ fm}}{0.7} \times 7.2 \times 10^{-5}
\approx 3.3 \times 10^{-4} \text{ MeV/fm}
\end{equation}

\textbf{Total barrier energy} (for dislocation segment of length $L \sim r_e$):
\begin{equation}
\boxed{V_P^{\text{tot}} = V_P \cdot L \approx 3.3 \times 10^{-4} \text{ MeV}}
\end{equation}

% ==============================================================================
\section{Phase 3: WKB Tunneling Rate}
% ==============================================================================

The tunneling rate is:
\begin{equation}
\Gamma = \omega_0 \exp\left(-\frac{S}{\hbar}\right)
\end{equation}
where the WKB action is:
\begin{equation}
S = \int_0^d \sqrt{2 M_{\text{eff}} \, V(q)} \, dq
\end{equation}

\subsection{Effective Mass}

The effective mass for dislocation motion comes from the kinetic energy of
the deformation field:
\begin{equation}
M_{\text{eff}} \sim \rho \cdot L \cdot r_0^2
\end{equation}
where $\rho$ is the ``mass density'' of the 5D medium.

From EDC, the natural mass scale is set by $\sigma$ and $r_e$:
\begin{equation}
M_{\text{eff}} \sim \frac{\sigma r_e^2}{c^2} \sim \frac{5.86 \text{ MeV}}{c^2} \approx 6.3 \text{ MeV}/c^2
\end{equation}

This is about 0.3\% of the proton mass---a small inertia, consistent with
a localized defect.

\subsection{Tunneling Distance}

The dislocation tunnels through one Peierls valley, distance $\sim a/2$:
\begin{equation}
d \sim \frac{a}{2} = 0.5 \text{ fm}
\end{equation}

\subsection{WKB Action Calculation}

For a sinusoidal barrier $V(q) = V_0 \sin^2(\pi q/a)$, the WKB action is
approximately:
\begin{equation}
S \approx \frac{4}{\pi} \sqrt{2 M_{\text{eff}} V_0} \cdot \frac{a}{2}
\end{equation}

Let's compute with our values:
\begin{itemize}
\item $M_{\text{eff}} = 6.3$ MeV/$c^2$
\item $V_0 = V_P^{\text{tot}} = 3.3 \times 10^{-4}$ MeV
\item $a = 1$ fm = $5.068 \times 10^{-3}$ MeV$^{-1}$ (in natural units)
\end{itemize}

\begin{equation}
\sqrt{2 M_{\text{eff}} V_0} = \sqrt{2 \times 6.3 \times 3.3 \times 10^{-4}}
= \sqrt{4.16 \times 10^{-3}} \approx 0.064 \text{ MeV}
\end{equation}

\begin{equation}
S \approx \frac{4}{\pi} \times 0.064 \text{ MeV} \times 2.53 \times 10^{-3} \text{ MeV}^{-1}
\approx 2.1 \times 10^{-4} \text{ (dimensionless)}
\end{equation}

\textbf{Problem:} This gives $S/\hbar \ll 1$, meaning essentially \textit{no barrier}!

% ==============================================================================
\section{Diagnosis: Where Did We Go Wrong?}
% ==============================================================================

The calculation gives a tunneling action $S/\hbar \sim 10^{-4}$, but we need
$S/\hbar \sim 35$ to get $\tau_n \sim 880$ s (since $\ln(10^{12} \times 880) \approx 35$).

\textbf{Possible issues:}

\begin{enumerate}
\item \textbf{Peierls barrier too small:} We used $w/a = 1.5$, giving strong
      exponential suppression. If the lattice is more ``stiff'' (narrower
      dislocation core), $w/a$ should be smaller.

\item \textbf{Wrong mass scale:} Perhaps $M_{\text{eff}}$ should be larger---
      closer to the nucleon mass rather than $\sigma r_e^2$.

\item \textbf{Wrong tunneling mechanism:} Maybe it's not simple Peierls
      tunneling, but involves additional quantum effects (instantons, etc.).

\item \textbf{Different barrier:} The relevant barrier may not be Peierls
      (lateral motion) but rather the \textit{radial} barrier for dislocation
      annihilation at the brane edge.
\end{enumerate}

\subsection{Re-analysis: What Barrier Height Do We Need?}

Working backwards from $\tau_n = 880$ s:
\begin{equation}
\Gamma = \frac{1}{\tau_n} \approx 1.14 \times 10^{-3} \text{ s}^{-1}
\end{equation}

With $\omega_0 \sim 10^{12}$ Hz (membrane oscillation frequency):
\begin{equation}
\frac{S}{\hbar} = \ln(\omega_0 \tau_n) = \ln(8.8 \times 10^{14}) \approx 34.4
\end{equation}

For the action to be $34\hbar$ with $d \sim 0.5$ fm and $M_{\text{eff}} \sim 6$ MeV:
\begin{equation}
V_0 = \frac{(S \cdot \pi / 4)^2}{2 M_{\text{eff}} d^2}
\approx \frac{(34 \times \pi/4)^2}{2 \times 6 \times (2.5 \times 10^{-3})^2}
\approx \frac{710}{7.5 \times 10^{-5}} \approx 10^7 \text{ MeV}
\end{equation}

This is absurdly large!

\textbf{Alternative:} If $M_{\text{eff}} \sim m_p \sim 938$ MeV (nucleon mass):
\begin{equation}
V_0 \approx \frac{710}{2 \times 938 \times (2.5 \times 10^{-3})^2}
\approx \frac{710}{0.012} \approx 60 \text{ MeV}
\end{equation}

This is closer to nuclear scales! Perhaps the effective mass is the nucleon
mass itself, and the barrier is $\sim 60$ MeV $\sim$ nuclear potential depth.

% ==============================================================================
\section{Revised Physical Picture}
% ==============================================================================

The calculation suggests that simple Peierls barrier tunneling with
$M_{\text{eff}} \sim \sigma r_e^2$ doesn't work. Instead:

\textbf{Hypothesis:} The neutron decay involves the entire nucleon mass
($M_{\text{eff}} \sim m_p$) tunneling through a barrier of height
$V_0 \sim 60$ MeV, which is characteristic of nuclear binding.

This reframes the problem: the neutron is not just a ``dislocation moving
laterally,'' but a \textit{nucleon-scale quantum object} that must overcome
a barrier comparable to nuclear binding energies.

\textbf{Cross-check:} The nuclear potential well depth is $\sim 40$--$50$ MeV.
Our required $V_0 \sim 60$ MeV is in the right ballpark!

% ==============================================================================
\section{Next Steps}
% ==============================================================================

\begin{enumerate}
\item Justify $M_{\text{eff}} \sim m_p$ from 5D geometry (the entire junction
      must move, not just the dislocation core)
\item Derive $V_0 \sim 60$ MeV from the $\mathbb{Z}_6$ potential or brane
      confinement energy
\item Compute the tunneling distance $d$ more carefully
\item Check whether this gives the correct $\tau_n$ with reasonable parameters
\end{enumerate}

\vspace{1cm}
\hrule
\vspace{0.5em}
\textbf{Status:} Phase 1-3 complete (first pass). Barrier too small with
initial assumptions. Need revised physical picture with nucleon-scale
effective mass.

\end{document}
