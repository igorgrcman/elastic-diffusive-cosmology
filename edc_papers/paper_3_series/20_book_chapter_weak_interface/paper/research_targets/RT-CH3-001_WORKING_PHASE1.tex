% ==============================================================================
% RT-CH3-001 WORKING DOCUMENT: Phase 1 — Domain Wall Toy Model
% ==============================================================================
% Status: IN PROGRESS
% Goal: Show that Plenum inflow direction determines chirality selection
% ==============================================================================

\documentclass[11pt,a4paper]{article}
\usepackage{fontspec}
\usepackage{amsmath,amssymb,amsthm}
\usepackage{geometry}
\usepackage{tcolorbox}
\usepackage{xcolor}

\geometry{margin=2.5cm}

\tcbuselibrary{skins,breakable}

\newcommand{\tagBL}{\textsuperscript{\textcolor{blue}{[BL]}}}
\newcommand{\tagP}{\textsuperscript{\textcolor{orange}{[P]}}}
\newcommand{\tagDc}{\textsuperscript{\textcolor{purple}{[Dc]}}}
\newcommand{\tagOpen}{\textsuperscript{\textcolor{red}{[OPEN]}}}

\newtcolorbox{keybox}[1][]{colback=green!5,colframe=green!50!black,title=#1,breakable}
\newtcolorbox{problembox}[1][]{colback=red!5,colframe=red!50!black,title=#1,breakable}

\newtheorem{proposition}{Proposition}
\newtheorem{lemma}{Lemma}

\title{RT-CH3-001 Working Document\\Phase 1: Domain Wall Toy Model}
\author{EDC Research Program}
\date{January 2026}

\begin{document}
\maketitle

% ==============================================================================
\section{Setup: 5D Dirac Equation with Domain Wall}
% ==============================================================================

\subsection{The Standard Result}

Consider the 5D Dirac equation with position-dependent mass \tagBL{}:
\begin{equation}
\left( i\gamma^\mu \partial_\mu + i\gamma^5 \partial_z - m(z) \right) \Psi = 0
\label{eq:5D_dirac}
\end{equation}

For a domain wall profile:
\begin{equation}
m(z) = m_0 \tanh(z/L)
\end{equation}

\textbf{Known result} (Jackiw-Rebbi 1976, Kaplan 1992) \tagBL{}:
\begin{itemize}
\item If $m(z): -m_0 \to +m_0$ as $z: -\infty \to +\infty$, there is a
      \textbf{left-handed} zero mode localized at $z=0$
\item If $m(z): +m_0 \to -m_0$, there is a \textbf{right-handed} zero mode
\end{itemize}

The zero mode profile is:
\begin{equation}
f_L(z) \propto \exp\left(-\int_0^z m(z') dz'\right) = \text{sech}^{m_0 L}(z/L)
\label{eq:zero_mode}
\end{equation}

\begin{problembox}[The Question]
What determines the \textbf{sign} of the mass profile? In particular, what
fixes whether $m(z)$ goes from $-m_0$ to $+m_0$ or vice versa?

In EDC, this should be determined by the \textbf{Plenum inflow direction}.
\end{problembox}

% ==============================================================================
\section{EDC Constraint: Plenum Inflow}
% ==============================================================================

\subsection{Physical Picture}

In EDC, the Plenum is the 5D bulk medium that flows INTO the brane:
\begin{equation}
J^z_{\text{Plenum}} > 0 \quad \text{(flow in $+z$ direction)}
\end{equation}

This is the fundamental EDC postulate: energy flows from the bulk toward
the observer-facing boundary.

\subsection{Connection to Mass Profile}

\begin{proposition}[Inflow Determines Mass Sign] \tagP{}
The Plenum inflow creates an effective ``pressure'' on the thick brane.
This pressure has a gradient that determines the sign of the mass profile:
\begin{equation}
\text{sgn}(m(z)) = \text{sgn}\left(\frac{\partial P}{\partial z}\right)
\end{equation}
where $P$ is the Plenum pressure.
\end{proposition}

\textbf{Physical argument:}
\begin{enumerate}
\item Plenum flows in $+z$ direction (from bulk to brane)
\item This creates higher pressure on the bulk side, lower on the observer side
\item $\partial P/\partial z < 0$ (pressure decreases toward observer)
\item The effective fermion mass is induced by this pressure gradient
\end{enumerate}

\begin{keybox}[Key Insight]
If the Plenum pressure gradient determines $m(z)$, and the pressure
decreases toward the observer ($z=0$), then:
\begin{equation}
m(z) = m_0 \tanh\left(\frac{z - z_0}{L}\right)
\end{equation}
goes from $-m_0$ (observer side) to $+m_0$ (bulk side).

This is exactly the profile that localizes a \textbf{left-handed} zero mode
at the observer boundary!
\end{keybox}

% ==============================================================================
\section{Derivation: Why Left-Handed?}
% ==============================================================================

\subsection{Step 1: Inflow Stress Tensor}

The Plenum inflow has associated stress-energy:
\begin{equation}
T^{zz}_{\text{Plenum}} = \rho_{\text{Plenum}} v_z^2 > 0
\end{equation}

At the thick brane interface, this creates an effective potential for fermions.

\subsection{Step 2: Fermion Mass from Stress}

In Kaluza-Klein models, the fermion mass can arise from coupling to background
fields. In EDC, the natural coupling is to the Plenum stress:
\begin{equation}
m(z) \sim \kappa \left( T^{zz}(z) - T^{zz}(0) \right)
\end{equation}
where $\kappa$ is a coupling constant.

Since $T^{zz} > 0$ in the bulk and decreases toward the boundary:
\begin{equation}
m(z) > 0 \quad \text{for } z > 0 \quad \text{(bulk side)}
\end{equation}
\begin{equation}
m(z) \to 0 \quad \text{as } z \to 0 \quad \text{(boundary)}
\end{equation}

\subsection{Step 3: Half-Domain Wall}

The EDC geometry is a ``half'' domain wall: $z \in [0, L]$ rather than
$z \in (-\infty, +\infty)$.

The mass profile is:
\begin{equation}
m(z) = m_0 \left(1 - e^{-z/\lambda}\right) \approx m_0 \frac{z}{\lambda} \quad \text{for small } z
\end{equation}

This rises from $0$ at the boundary to $m_0$ in the bulk.

\subsection{Step 4: Zero Mode Localization}

The zero mode equation is:
\begin{equation}
\partial_z f_L = -m(z) f_L
\end{equation}

With $m(z) > 0$ for $z > 0$:
\begin{equation}
f_L(z) = f_L(0) \exp\left(-\int_0^z m(z') dz'\right)
\end{equation}

Since $m(z) > 0$, the integral is positive, so $f_L(z)$ \textbf{decreases}
as $z$ increases. The mode is localized at the boundary $z=0$.

For the right-handed mode:
\begin{equation}
\partial_z f_R = +m(z) f_R
\end{equation}

This gives:
\begin{equation}
f_R(z) = f_R(0) \exp\left(+\int_0^z m(z') dz'\right)
\end{equation}

This \textbf{increases} as $z$ increases --- not normalizable! The right-handed
mode is expelled into the bulk.

% ==============================================================================
\section{Result: V$-$A from Inflow}
% ==============================================================================

\begin{keybox}[Main Result (Phase 1)]
\textbf{Starting from:}
\begin{itemize}
\item EDC 5D geometry with thick brane
\item Plenum inflow in $+z$ direction \tagP{}
\item Fermion-Plenum coupling via stress tensor \tagP{}
\end{itemize}

\textbf{We derive:}
\begin{enumerate}
\item The effective mass $m(z) > 0$ for $z > 0$ (rises into bulk)
\item Left-handed zero mode is localized at boundary
\item Right-handed mode is expelled into bulk
\item Only \textbf{left-handed fermions} couple effectively to observer physics
\end{enumerate}

\textbf{This is V$-$A structure emerging from geometry!} \tagDc{}
\end{keybox}

% ==============================================================================
\section{Epistemic Audit}
% ==============================================================================

\begin{center}
\begin{tabular}{lll}
\hline
\textbf{Element} & \textbf{Source} & \textbf{Status} \\
\hline
5D Dirac equation & Standard QFT & \tagBL{} \\
Domain wall zero mode & Jackiw-Rebbi, Kaplan & \tagBL{} \\
Plenum inflow direction & EDC postulate & \tagP{} \\
Fermion-stress coupling & Physical hypothesis & \tagP{} \\
$m(z)$ rises into bulk & From inflow direction & \tagDc{} \\
L-mode localized & Mathematical consequence & \tagDc{} \\
R-mode expelled & Mathematical consequence & \tagDc{} \\
V$-$A structure & Emerges from above & \tagDc{} \\
\hline
\end{tabular}
\end{center}

\textbf{No chirality smuggling!} The only new physics input is:
\begin{enumerate}
\item Plenum inflow direction ($+z$) --- this is EDC's basic mechanism
\item Fermion couples to Plenum stress --- this is a physical hypothesis
\end{enumerate}

The chirality selection is a \textbf{consequence}, not an assumption.

% ==============================================================================
\section{Open Questions}
% ==============================================================================

\begin{enumerate}
\item \textbf{Quantitative:} What is the coupling constant $\kappa$? Does it
      give the correct fermion mass spectrum?

\item \textbf{Z$_2$ connection:} Does $\mathbb{Z}_2 \subset \mathbb{Z}_6$ also
      play a role in chirality selection?

\item \textbf{Helicity suppression:} Can we derive $|A|^2 \propto m_\ell^2$
      from the mode overlap integrals?

\item \textbf{G$_F$ derivation:} The weak coupling $G_F$ should come from the
      mode profile overlap. Does this work?
\end{enumerate}

% ==============================================================================
\section{Connection to $\mathbb{Z}_6$ Lattice}
% ==============================================================================

The $\mathbb{Z}_6 = \mathbb{Z}_2 \times \mathbb{Z}_3$ factorization suggests:
\begin{itemize}
\item $\mathbb{Z}_3$: Color (center of SU(3)) --- established in Ch2
\item $\mathbb{Z}_2$: Chirality/Parity?
\end{itemize}

\begin{proposition}[$\mathbb{Z}_2$ as Chirality Selector] \tagOpen{}
The $\mathbb{Z}_2$ subgroup acts on spinors as:
\begin{equation}
\mathcal{P}_{\mathbb{Z}_2}: \Psi \mapsto \gamma^5 \Psi
\end{equation}

Under this action:
\begin{itemize}
\item Left-handed modes are \textbf{odd}: $\mathcal{P}_{\mathbb{Z}_2} \psi_L = -\psi_L$
\item Right-handed modes are \textbf{even}: $\mathcal{P}_{\mathbb{Z}_2} \psi_R = +\psi_R$
\end{itemize}

If the $\mathbb{Z}_6$ lattice boundary conditions select \textbf{odd} modes
(due to the dislocation structure), then only L couples to weak processes.
\end{proposition}

This provides a \textbf{second} mechanism for chirality selection, reinforcing
the inflow argument.

\vspace{1cm}
\hrule
\vspace{0.5em}
\textbf{Status:} Phase 1 COMPLETE (toy model shows chirality selection) \\
\textbf{Next:} Phase 2 (realistic thick-brane profile) \\
\textbf{Open:} Quantitative check, helicity suppression, $G_F$ derivation

\end{document}
