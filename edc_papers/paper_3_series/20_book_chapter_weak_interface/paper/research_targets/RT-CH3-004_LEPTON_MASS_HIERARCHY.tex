% ==============================================================================
% RESEARCH TARGET RT-CH3-004: Lepton Mass Hierarchy from Mode Spectrum
% ==============================================================================
% Status: OPEN
% Priority: MEDIUM
% Dependencies: RT-CH3-001 (mode profiles)
% ==============================================================================

\documentclass[11pt,a4paper]{article}
\usepackage[utf8]{inputenc}
\usepackage{amsmath,amssymb,amsthm}
\usepackage{geometry}
\usepackage{tcolorbox}
\usepackage{enumitem}
\usepackage{booktabs}

\geometry{margin=2.5cm}

\newtheorem{problem}{Problem}
\newtheorem{hypothesis}{Hypothesis}
\newtheorem{criterion}{Success Criterion}

\tcbuselibrary{skins,breakable}
\newtcolorbox{inputbox}[1][]{colback=green!5,colframe=green!50!black,title=#1}
\newtcolorbox{outputbox}[1][]{colback=blue!5,colframe=blue!50!black,title=#1}

\title{\textbf{Research Target RT-CH3-004}\\[0.5em]
\Large Lepton Mass Hierarchy from Thick-Brane Mode Spectrum}
\author{EDC Research Program --- Chapter 3}
\date{Target formulated: January 2026}

\begin{document}
\maketitle

% ==============================================================================
\section{Problem Statement}
% ==============================================================================

\begin{problem}[Mass Hierarchy from Geometry]
Derive the charged lepton mass ratios
\begin{equation}
m_e : m_\mu : m_\tau \approx 1 : 207 : 3477
\end{equation}
from the eigenvalue spectrum of the thick-brane mode equation.
\end{problem}

\subsection{Why This Matters}

The lepton mass hierarchy is one of the deepest mysteries in particle physics:
\begin{itemize}
  \item Why is $m_\tau/m_e \approx 3477$?
  \item Why three generations (not two, not four)?
  \item Is there a pattern (cf.\ Koide formula)?
\end{itemize}

In EDC, leptons are brane-localized modes. Their masses should correspond to
eigenvalues of a Schr\"odinger-type equation in the $\xi$-direction.

% ==============================================================================
\section{What Is INPUT vs OUTPUT}
% ==============================================================================

\begin{inputbox}[Allowed Inputs (EDC Postulates)]
\begin{enumerate}[label=\textbf{I.\arabic*},nosep]
  \item \textbf{Thick Brane Potential}: $V(\xi)$ that localizes modes
  \item \textbf{5D Dirac Structure}: Fermion equation in curved 5D
  \item \textbf{Boundary Conditions}: At $\xi=0$ (frozen) and $\xi=L$ (bulk)
  \item \textbf{$\mathbb{Z}_6$ Lattice}: Discrete structure at interface (Ch2)
  \item \textbf{EDC Scales}: $R_\xi$, brane thickness $\Delta$
\end{enumerate}
\end{inputbox}

\begin{outputbox}[Required Output (Must Derive)]
\begin{enumerate}[label=\textbf{O.\arabic*},nosep]
  \item \textbf{Three bound states}: Exactly 3 normalizable modes
  \item \textbf{Mass ratios}: $m_1 : m_2 : m_3 \approx 1 : 207 : 3477$
  \item \textbf{Koide formula}: $\frac{m_e + m_\mu + m_\tau}{(\sqrt{m_e} + \sqrt{m_\mu} + \sqrt{m_\tau})^2} = \frac{2}{3}$
\end{enumerate}
\end{outputbox}

% ==============================================================================
\section{Candidate Derivation Strategy}
% ==============================================================================

\begin{hypothesis}[Mode Counting]
The thick brane supports exactly three bound states because of the potential
depth and width---determined by $R_\xi$ and the cosmological constant.
\end{hypothesis}

\subsection{Physical Picture}

\textbf{Step 1: Mode equation}

The 5D Dirac equation reduces to a 1D Schr\"odinger-type equation:
\begin{equation}
\left[ -\frac{d^2}{d\xi^2} + V_{\text{eff}}(\xi) \right] f_n(\xi) = m_n^2 f_n(\xi)
\end{equation}

\textbf{Step 2: Potential form}

Candidate potentials:
\begin{itemize}
  \item Volcano potential: $V(\xi) = V_0 \, \text{sech}^2(\xi/\Delta)$
  \item Asymmetric well: different behavior at $\xi=0$ vs $\xi=L$
  \item $\mathbb{Z}_6$-modulated: lattice corrections
\end{itemize}

\textbf{Step 3: Eigenvalue spectrum}

For the P\"oschl-Teller potential ($\text{sech}^2$), the eigenvalues are:
\begin{equation}
E_n = -\frac{\hbar^2}{2m\Delta^2}\left( \sqrt{1 + \frac{2mV_0\Delta^2}{\hbar^2}} - n - \frac{1}{2} \right)^2
\end{equation}

\textbf{Step 4: Map to lepton masses}

Identify $m_e, m_\mu, m_\tau$ with $n=0,1,2$ states.

% ==============================================================================
\section{Success Criteria}
% ==============================================================================

\begin{criterion}[Three Generations]
The potential must support exactly 3 bound states for reasonable EDC parameters.
\end{criterion}

\begin{criterion}[Mass Ratios]
With one overall scale fitted, the ratios must emerge:
\begin{equation}
\frac{m_\mu}{m_e} \approx 207 \pm 20, \qquad \frac{m_\tau}{m_e} \approx 3477 \pm 300
\end{equation}
\end{criterion}

\begin{criterion}[Koide Consistency]
The derived masses should satisfy the Koide formula within 1\%:
\begin{equation}
\frac{m_e + m_\mu + m_\tau}{(\sqrt{m_e} + \sqrt{m_\mu} + \sqrt{m_\tau})^2} = 0.6667 \pm 0.007
\end{equation}
\end{criterion}

% ==============================================================================
\section{Connection to Chapter 2}
% ==============================================================================

Chapter 2 (The $\mathbb{Z}_6$ Program) suggests:
\begin{itemize}
  \item Three generations $\leftrightarrow$ radial harmonics of Y-junction
  \item Koide phase $\delta = 2\pi/9$ from $\mathbb{Z}_3$ structure
  \item Mass formula: $m_k \propto (1 + \sqrt{2}\cos(\theta + 2\pi k/3))^2$
\end{itemize}

This target should connect the mode-equation approach (analytic) with the
$\mathbb{Z}_6$ approach (discrete symmetry).

\vspace{1em}
\hrule
\vspace{0.5em}
\textbf{Status}: OPEN \\
\textbf{Assigned}: Chapter 3 research program \\
\textbf{Target}: Derive lepton mass ratios from mode spectrum

\end{document}
