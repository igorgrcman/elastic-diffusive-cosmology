% ==============================================================================
% RESEARCH TARGET RT-CH3-005: Neutrino Mass from Edge-Mode Dynamics
% ==============================================================================
% Status: OPEN
% Priority: MEDIUM
% Dependencies: Ch1 edge-mode ontology
% ==============================================================================

\documentclass[11pt,a4paper]{article}
\usepackage[utf8]{inputenc}
\usepackage{amsmath,amssymb,amsthm}
\usepackage{geometry}
\usepackage{tcolorbox}
\usepackage{enumitem}
\usepackage{booktabs}

\geometry{margin=2.5cm}

\newtheorem{problem}{Problem}
\newtheorem{hypothesis}{Hypothesis}
\newtheorem{criterion}{Success Criterion}

\tcbuselibrary{skins,breakable}
\newtcolorbox{inputbox}[1][]{colback=green!5,colframe=green!50!black,title=#1}
\newtcolorbox{outputbox}[1][]{colback=blue!5,colframe=blue!50!black,title=#1}

\title{\textbf{Research Target RT-CH3-005}\\[0.5em]
\Large Neutrino Mass from Edge-Mode Dynamics}
\author{EDC Research Program --- Chapter 3}
\date{Target formulated: January 2026}

\begin{document}
\maketitle

% ==============================================================================
\section{Problem Statement}
% ==============================================================================

\begin{problem}[Neutrino Mass Scale]
Derive the neutrino mass scale $m_\nu \lesssim 0.1$ eV from edge-mode dynamics
at the bulk--brane interface, explaining:
\begin{enumerate}[nosep]
  \item Why $m_\nu \ll m_e$ (factor $\sim 10^6$ suppression)
  \item Why neutrinos have mass at all (nonzero, unlike photon)
  \item The origin of three-flavor mixing
\end{enumerate}
\end{problem}

\subsection{Why This Matters}

Neutrino mass is the only confirmed physics beyond the minimal Standard Model.
The tiny mass scale ($\lesssim 0.1$ eV vs $m_e = 0.511$ MeV) is deeply puzzling:
\begin{itemize}
  \item Seesaw mechanism requires $M_R \sim 10^{14}$ GeV (untestable)
  \item Dirac mass requires Yukawa $\sim 10^{-12}$ (fine-tuned)
\end{itemize}

EDC offers a geometric alternative: neutrinos are \emph{edge modes} at the
bulk--brane interface, with mass suppressed by their delocalized nature.

% ==============================================================================
\section{What Is INPUT vs OUTPUT}
% ==============================================================================

\begin{inputbox}[Allowed Inputs (EDC Postulates)]
\begin{enumerate}[label=\textbf{I.\arabic*},nosep]
  \item \textbf{Neutrino Ontology}: Edge modes at bulk--brane interface (Ch1)
  \item \textbf{Thick Brane Structure}: $V(\xi)$ with asymmetric boundaries
  \item \textbf{Plenum Inflow}: $J^\xi > 0$ defines preferred direction
  \item \textbf{Chirality Selection}: From RT-CH3-001 (left-handed survival)
  \item \textbf{$\mathbb{Z}_6$ Lattice}: Interface microstructure (Ch2)
\end{enumerate}
\end{inputbox}

\begin{outputbox}[Required Output (Must Derive)]
\begin{enumerate}[label=\textbf{O.\arabic*},nosep]
  \item \textbf{Mass Scale}: $m_\nu \sim 0.01$--$0.1$ eV from geometry
  \item \textbf{Suppression Factor}: Why $m_\nu/m_e \sim 10^{-6}$
  \item \textbf{Three Flavors}: Connection to charged lepton generations
  \item \textbf{Mixing Structure}: Origin of PMNS matrix structure
\end{enumerate}
\end{outputbox}

% ==============================================================================
\section{Candidate Derivation Strategy}
% ==============================================================================

\begin{hypothesis}[Edge Delocalization]
Neutrino mass is suppressed because edge modes extend into the bulk,
reducing their effective overlap with the brane-localized Higgs/mass mechanism.
\end{hypothesis}

\subsection{Physical Picture}

\textbf{Step 1: Edge mode profile}

At the bulk--brane boundary, the mode equation has edge solutions:
\begin{equation}
f_\nu(\xi) \sim e^{-\kappa \xi} \quad \text{for } \xi > 0 \text{ (into bulk)}
\end{equation}
where $\kappa^{-1}$ is the penetration depth.

\textbf{Step 2: Mass from overlap}

The effective 4D mass arises from overlap with the mass mechanism:
\begin{equation}
m_\nu \sim m_0 \int_0^\infty |f_\nu(\xi)|^2 \, h(\xi) \, d\xi
\end{equation}
where $h(\xi)$ is the Higgs/mass profile (brane-localized).

\textbf{Step 3: Suppression mechanism}

If $h(\xi)$ is localized at $\xi=0$ but $f_\nu(\xi)$ extends into the bulk:
\begin{equation}
\frac{m_\nu}{m_e} \sim \frac{\int |f_\nu|^2 h \, d\xi}{\int |f_e|^2 h \, d\xi} \sim \exp(-\Delta/\kappa^{-1})
\end{equation}
This gives exponential suppression.

\textbf{Step 4: Three flavors}

The three neutrino flavors correspond to three edge modes with different
penetration depths (or angular momenta around the $\mathbb{Z}_3$ axis).

% ==============================================================================
\section{Success Criteria}
% ==============================================================================

\begin{criterion}[Mass Scale]
Derive $m_\nu \in [0.01, 0.1]$ eV from EDC parameters without fine-tuning.
\end{criterion}

\begin{criterion}[Hierarchy Explanation]
The factor $10^6$ suppression ($m_\nu/m_e$) must emerge from geometry, not
be put in by hand.
\end{criterion}

\begin{criterion}[Mixing Prediction]
Make at least one quantitative prediction about PMNS angles or mass splittings.
\end{criterion}

% ==============================================================================
\section{Connection to Oscillations}
% ==============================================================================

Neutrino oscillations depend on mass-squared differences:
\begin{align}
\Delta m_{21}^2 &\approx 7.5 \times 10^{-5} \text{ eV}^2 \\
|\Delta m_{31}^2| &\approx 2.5 \times 10^{-3} \text{ eV}^2
\end{align}

If EDC predicts the absolute mass scale AND the splitting pattern, this would
be a major success.

% ==============================================================================
\section{Why Edge Modes?}
% ==============================================================================

The edge-mode hypothesis is motivated by:
\begin{itemize}
  \item Neutrinos interact only via weak force (no EM, no strong)
  \item They have tiny mass (delocalized $\Rightarrow$ small overlap)
  \item They exhibit mixing (multiple edge modes overlap)
  \item They are left-handed only (boundary selects chirality)
\end{itemize}

This ontology is not assumed---it must be derived from the 5D action.

\vspace{1em}
\hrule
\vspace{0.5em}
\textbf{Status}: OPEN \\
\textbf{Assigned}: Chapter 3 research program \\
\textbf{Target}: Derive $m_\nu \sim 0.05$ eV from edge-mode dynamics

\end{document}
