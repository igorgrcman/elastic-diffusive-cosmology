% ==============================================================================
% RT-CH3-002 WORKING DOCUMENT: Phase 1 — Asymmetric Thick-Brane Profile
% ==============================================================================
% Status: IN PROGRESS
% Goal: Replace symmetric tanh profile with realistic EDC asymmetric profile
% Prerequisite for: G_F derivation from overlap integrals
% ==============================================================================

\documentclass[11pt,a4paper]{article}
\usepackage{fontspec}
\usepackage{amsmath,amssymb,amsthm}
\usepackage{geometry}
\usepackage{tcolorbox}
\usepackage{xcolor}
\usepackage{booktabs}

\geometry{margin=2.5cm}

\tcbuselibrary{skins,breakable}

\newcommand{\tagBL}{\textsuperscript{\textcolor{blue}{[BL]}}}
\newcommand{\tagP}{\textsuperscript{\textcolor{orange}{[P]}}}
\newcommand{\tagDc}{\textsuperscript{\textcolor{purple}{[Dc]}}}
\newcommand{\tagOpen}{\textsuperscript{\textcolor{red}{[OPEN]}}}
\newcommand{\tagDer}{\textsuperscript{\textcolor{green!50!black}{[Der]}}}

\newtcolorbox{keybox}[1][]{colback=green!5,colframe=green!50!black,title=#1,breakable}
\newtcolorbox{problembox}[1][]{colback=red!5,colframe=red!50!black,title=#1,breakable}
\newtcolorbox{inputbox}[1][]{colback=blue!5,colframe=blue!50!black,title=#1,breakable}

\newtheorem{proposition}{Proposition}
\newtheorem{lemma}{Lemma}
\newtheorem{definition}{Definition}

\title{RT-CH3-002 Working Document\\Phase 1: Asymmetric Thick-Brane Profile}
\author{EDC Research Program}
\date{January 2026}

\begin{document}
\maketitle

% ==============================================================================
\section{The Problem with Symmetric Profiles}
% ==============================================================================

\begin{problembox}[Why tanh Doesn't Work for EDC]
The standard domain wall profile used in Phase 1 (RT-CH3-001):
\begin{equation}
m_{\text{tanh}}(\xi) = m_0 \tanh(\xi/L)
\end{equation}
is \textbf{symmetric} around $\xi = 0$. This describes a domain wall between two
semi-infinite bulk regions.

\textbf{But EDC geometry is different:}
\begin{itemize}
\item $\xi = 0$ is the \textbf{observer boundary} (brane surface)
\item $\xi > 0$ is the \textbf{bulk} (Plenum)
\item There is NO region $\xi < 0$
\item Plenum flows in \textbf{one direction} ($+\xi$ to $-\xi$, into the brane)
\end{itemize}

The EDC thick brane is a \textbf{half-space} problem, not a symmetric domain wall!
\end{problembox}

% ==============================================================================
\section{Physical Setup: Plenum Inflow Creates Asymmetry}
% ==============================================================================

\subsection{Coordinate Convention}

Following EDC conventions:
\begin{itemize}
\item $\xi = 0$: Observer-facing boundary (``our'' 3+1D world)
\item $\xi = \Delta$: Bulk-facing edge of thick brane
\item $\xi > \Delta$: Deep bulk (Plenum reservoir)
\item Plenum flows from bulk ($\xi > \Delta$) toward boundary ($\xi = 0$)
\end{itemize}

\subsection{Pressure Profile from Inflow}

The Plenum inflow creates a pressure gradient \tagP{}:
\begin{equation}
P(\xi) = P_\infty + \Delta P \cdot f(\xi)
\end{equation}
where $P_\infty$ is the asymptotic bulk pressure and $f(\xi)$ captures the
pressure buildup near the brane.

\begin{proposition}[Exponential Pressure Decay]
\label{prop:pressure}
For steady-state Plenum inflow with viscous dissipation, the pressure profile is:
\begin{equation}
P(\xi) = P_0 \left(1 - e^{-\xi/\lambda}\right) + P_\infty
\end{equation}
where $\lambda$ is the dissipation length scale. \tagP{}
\end{proposition}

\textbf{Physical justification:} Near the boundary, the Plenum ``piles up'' creating
higher pressure. This decays exponentially into the brane interior on the scale $\lambda$.

\subsection{Boundary Conditions}

At the observer boundary $\xi = 0$:
\begin{itemize}
\item Dirichlet-like condition: Fields must match 4D physics
\item Pressure: $P(0) = P_\infty$ (no buildup at boundary itself)
\end{itemize}

Deep in the bulk $\xi \to \infty$:
\begin{itemize}
\item Asymptotic approach to bulk values
\item $P(\infty) = P_\infty + P_0$ (full pressure)
\end{itemize}

% ==============================================================================
\section{Derivation: Asymmetric Mass Profile}
% ==============================================================================

\subsection{Fermion Mass from Pressure Coupling}

Following the RT-CH3-001 result, fermion mass couples to Plenum stress \tagP{}:
\begin{equation}
m(\xi) = \kappa \cdot \left(T^{\xi\xi}(\xi) - T^{\xi\xi}(0)\right)
\end{equation}

For the pressure profile from Proposition~\ref{prop:pressure}:
\begin{equation}
T^{\xi\xi}(\xi) \propto P(\xi)
\end{equation}

This gives:

\begin{keybox}[Asymmetric Mass Profile]
\begin{equation}
\boxed{m(\xi) = m_0 \left(1 - e^{-\xi/\lambda}\right)}
\label{eq:asymmetric_mass}
\end{equation}
where:
\begin{itemize}
\item $m_0 \equiv \kappa P_0$ is the bulk fermion mass
\item $\lambda$ is the characteristic width of the thick brane
\item $m(0) = 0$ at the boundary
\item $m(\xi) \to m_0$ as $\xi \to \infty$
\end{itemize}
\end{keybox}

\textbf{Key difference from tanh:}
\begin{center}
\begin{tabular}{lcc}
\toprule
\textbf{Property} & \textbf{tanh} & \textbf{EDC Exponential} \\
\midrule
Domain & $(-\infty, +\infty)$ & $[0, +\infty)$ \\
Symmetry & Symmetric around 0 & Asymmetric \\
Boundary at $\xi=0$ & Transition point & Hard boundary \\
$m(0)$ & 0 & 0 \\
$m'(0)$ & $m_0/L$ & $m_0/\lambda$ \\
\bottomrule
\end{tabular}
\end{center}

% ==============================================================================
\section{Mode Profiles in Asymmetric Potential}
% ==============================================================================

\subsection{The 5D Dirac Equation}

The 5D Dirac equation with position-dependent mass \tagBL{}:
\begin{equation}
\left( i\gamma^\mu \partial_\mu + i\gamma^5 \partial_\xi - m(\xi) \right) \Psi = 0
\end{equation}

Separating $\Psi = \psi(x^\mu) f(\xi)$, the zero-mode equation becomes:
\begin{equation}
\gamma^5 \partial_\xi f = m(\xi) f
\end{equation}

For chiral projections $f = f_L P_L + f_R P_R$ where $\gamma^5 P_{L/R} = \mp P_{L/R}$:
\begin{align}
\partial_\xi f_L &= -m(\xi) f_L \\
\partial_\xi f_R &= +m(\xi) f_R
\end{align}

\subsection{Solution with Asymmetric Profile}

Using $m(\xi) = m_0(1 - e^{-\xi/\lambda})$ from Eq.~\eqref{eq:asymmetric_mass}:

\begin{align}
\int_0^\xi m(\xi') d\xi' &= m_0 \int_0^\xi \left(1 - e^{-\xi'/\lambda}\right) d\xi' \\
&= m_0 \left[ \xi + \lambda e^{-\xi'/\lambda} \right]_0^\xi \\
&= m_0 \left( \xi + \lambda e^{-\xi/\lambda} - \lambda \right) \\
&= m_0 \left( \xi - \lambda (1 - e^{-\xi/\lambda}) \right)
\end{align}

Define the ``effective depth'':
\begin{equation}
\chi(\xi) \equiv \xi - \lambda (1 - e^{-\xi/\lambda})
\end{equation}

Note that:
\begin{itemize}
\item $\chi(0) = 0$
\item $\chi(\xi) \approx \xi - \lambda$ for $\xi \gg \lambda$
\item $\chi(\xi) \approx \xi^2/(2\lambda)$ for $\xi \ll \lambda$
\end{itemize}

\begin{keybox}[Left-Handed Mode Profile]
\begin{equation}
\boxed{f_L(\xi) = N_L \exp\left( -m_0 \chi(\xi) \right) = N_L \exp\left( -m_0 \xi + m_0\lambda (1 - e^{-\xi/\lambda}) \right)}
\label{eq:fL_profile}
\end{equation}
This mode is:
\begin{itemize}
\item Localized near $\xi = 0$ (boundary)
\item Normalized with $\int_0^\infty |f_L|^2 d\xi = 1$
\item Decays as $e^{-m_0 \xi}$ for $\xi \gg \lambda$
\end{itemize}
\end{keybox}

\begin{keybox}[Right-Handed Mode Profile]
\begin{equation}
f_R(\xi) = N_R \exp\left( +m_0 \chi(\xi) \right)
\end{equation}
This mode:
\begin{itemize}
\item Grows as $\xi$ increases
\item Is NOT normalizable on $[0, \infty)$
\item Must be cut off at some $\xi = L$ (brane thickness)
\end{itemize}
\end{keybox}

\textbf{Result:} Left-handed modes are naturally localized at the boundary;
right-handed modes are expelled into the bulk. This confirms the V$-$A
structure from RT-CH3-001 with the realistic profile.

% ==============================================================================
\section{Normalization and Effective Width}
% ==============================================================================

\subsection{Left-Handed Mode Normalization}

The normalization integral:
\begin{equation}
1 = \int_0^\infty |f_L(\xi)|^2 d\xi = N_L^2 \int_0^\infty e^{-2m_0\chi(\xi)} d\xi
\end{equation}

For large $m_0\lambda \gg 1$, the mode is sharply peaked near $\xi = 0$. Expanding:
\begin{equation}
\chi(\xi) \approx \frac{\xi^2}{2\lambda} \quad \text{for } \xi \ll \lambda
\end{equation}

\begin{equation}
\int_0^\infty e^{-2m_0 \xi^2/(2\lambda)} d\xi = \sqrt{\frac{\pi\lambda}{2m_0}}
\end{equation}

Therefore:
\begin{equation}
N_L \approx \left(\frac{2m_0}{\pi\lambda}\right)^{1/4}
\end{equation}

\subsection{Effective Localization Width}

The mode width is:
\begin{equation}
\boxed{\sigma_L \equiv \sqrt{\langle \xi^2 \rangle} \approx \sqrt{\frac{\lambda}{2m_0}}}
\end{equation}

\textbf{Numerical estimate:} If $\lambda \sim \Delta \sim 10^{-18}$ m (brane thickness)
and $m_0 \sim 1$ GeV (electroweak scale):
\begin{equation}
\sigma_L \sim \sqrt{\frac{10^{-18} \text{ m}}{2 \times 5 \times 10^{-18} \text{ m}}} \sim 0.3 \text{ (in units of } \lambda\text{)}
\end{equation}

The mode is localized within a fraction of the brane thickness.

% ==============================================================================
\section{Connection to $G_F$: Preview}
% ==============================================================================

\subsection{Overlap Integral for $G_F$}

The Fermi constant arises from the overlap of fermion modes with a bulk mediator:
\begin{equation}
G_F \sim \frac{g_5^2}{M_5^2} \int_0^L d\xi \, |f_L(\xi)|^2 |f_L(\xi)|^2 |\phi(\xi)|^2
\label{eq:GF_overlap}
\end{equation}

With the mediator profile $\phi(\xi) \sim 1/M_5$ (constant for a heavy mediator), this becomes:
\begin{equation}
G_F \sim \frac{g_5^2}{M_5^4} \int_0^L |f_L(\xi)|^4 d\xi
\end{equation}

\subsection{Order-of-Magnitude Estimate}

Using the Gaussian approximation for $f_L$:
\begin{equation}
\int |f_L|^4 d\xi \sim N_L^4 \cdot \sigma_L \sim \frac{1}{\sigma_L}
\end{equation}

Therefore:
\begin{equation}
G_F \sim \frac{g_5^2}{M_5^4 \sigma_L} \sim \frac{g_5^2}{M_5^4} \sqrt{\frac{2m_0}{\lambda}}
\end{equation}

\textbf{This will be the starting point for Phase 2.}

% ==============================================================================
\section{Determining Parameters from EDC}
% ==============================================================================

\subsection{What Sets $\lambda$?}

The characteristic length $\lambda$ should be related to EDC scales:
\begin{itemize}
\item \textbf{Option A:} $\lambda = \Delta$ (brane thickness) $\sim R_\xi \sim 10^{-18}$ m
\item \textbf{Option B:} $\lambda = r_e$ (lattice spacing) $\sim 10^{-15}$ m
\item \textbf{Option C:} Derived from membrane tension $\sigma$ and flow rate
\end{itemize}

\begin{proposition}[Dissipation Length from Membrane Tension]
\tagP{}
The Plenum dissipation length is set by the balance between inflow momentum
and membrane restoring force:
\begin{equation}
\lambda \sim \sqrt{\frac{\sigma}{\rho v^2}}
\end{equation}
where $\rho v^2$ is the Plenum ram pressure. \tagOpen{}
\end{proposition}

\subsection{What Sets $m_0$?}

The bulk mass scale $m_0$ determines fermion localization:
\begin{itemize}
\item For light localization: $m_0 \sim 1/\lambda$ (comparable scales)
\item For strong localization: $m_0 \lambda \gg 1$
\end{itemize}

From the $\mathbb{Z}_6$ lattice (Chapter 2), a natural mass scale is:
\begin{equation}
m_0 \sim \frac{\sigma r_e^2}{\lambda} \sim \frac{5.86 \text{ MeV}}{10^{-18} \text{ m}} \times \hbar c
\end{equation}

This needs careful dimensional analysis in Phase 2.

% ==============================================================================
\section{Epistemic Summary}
% ==============================================================================

\begin{center}
\begin{tabular}{lll}
\toprule
\textbf{Element} & \textbf{Source} & \textbf{Status} \\
\midrule
5D Dirac equation & Standard QFT & \tagBL{} \\
Asymmetric boundary conditions & EDC geometry & \tagP{} \\
Pressure profile from inflow & Physical hypothesis & \tagP{} \\
$m(\xi) = m_0(1-e^{-\xi/\lambda})$ & From pressure coupling & \tagDer{} \\
$f_L(\xi)$ localized at boundary & Mathematical consequence & \tagDer{} \\
$f_R(\xi)$ expelled to bulk & Mathematical consequence & \tagDer{} \\
$\lambda$ from membrane tension & \tagOpen{} & To derive \\
$m_0$ from EDC scales & \tagOpen{} & To derive \\
\bottomrule
\end{tabular}
\end{center}

% ==============================================================================
\section{Next Steps: Phase 2}
% ==============================================================================

\begin{enumerate}
\item \textbf{Fix $\lambda$:} Derive from EDC first principles (membrane + inflow)
\item \textbf{Fix $m_0$:} Connect to $\sigma r_e^2$ or other EDC scale
\item \textbf{Compute $\int |f_L|^4$:} Exact integral for overlap
\item \textbf{Identify $M_5$:} 5D mediator mass from EDC geometry
\item \textbf{Evaluate $G_F$:} Numerical calculation
\end{enumerate}

\vspace{1cm}
\hrule
\vspace{0.5em}
\textbf{Status:} Phase 1 COMPLETE (asymmetric profile derived) \\
\textbf{Next:} Phase 2 (fix parameters, compute $G_F$) \\
\textbf{Key result:} Asymmetric profile $m(\xi) = m_0(1-e^{-\xi/\lambda})$ replaces tanh

\end{document}
