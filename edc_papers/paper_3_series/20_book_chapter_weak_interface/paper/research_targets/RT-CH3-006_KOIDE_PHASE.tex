% ==============================================================================
% RESEARCH TARGET RT-CH3-006: Koide Phase from Z₃ Geometry
% ==============================================================================
% Status: OPEN
% Priority: LOW (elegant but not essential)
% Dependencies: Ch2 Z₃ structure, RT-CH3-004
% ==============================================================================

\documentclass[11pt,a4paper]{article}
\usepackage[utf8]{inputenc}
\usepackage{amsmath,amssymb,amsthm}
\usepackage{geometry}
\usepackage{tcolorbox}
\usepackage{enumitem}
\usepackage{booktabs}

\geometry{margin=2.5cm}

\newtheorem{problem}{Problem}
\newtheorem{hypothesis}{Hypothesis}
\newtheorem{criterion}{Success Criterion}

\tcbuselibrary{skins,breakable}
\newtcolorbox{inputbox}[1][]{colback=green!5,colframe=green!50!black,title=#1}
\newtcolorbox{outputbox}[1][]{colback=blue!5,colframe=blue!50!black,title=#1}

\title{\textbf{Research Target RT-CH3-006}\\[0.5em]
\Large Derivation of Koide Phase from $\mathbb{Z}_3$ Geometry}
\author{EDC Research Program --- Chapter 3}
\date{Target formulated: January 2026}

\begin{document}
\maketitle

% ==============================================================================
\section{Problem Statement}
% ==============================================================================

\begin{problem}[Koide Formula Derivation]
Derive the Koide formula
\begin{equation}
\frac{m_e + m_\mu + m_\tau}{(\sqrt{m_e} + \sqrt{m_\mu} + \sqrt{m_\tau})^2} = \frac{2}{3}
\end{equation}
from the $\mathbb{Z}_3$ symmetry of the Y-junction structure, specifically
deriving the phase $\delta = 2\pi/9$.
\end{problem}

\subsection{Why This Matters}

The Koide formula is a mysterious empirical relation:
\begin{itemize}
  \item Holds to 0.01\% precision for charged leptons
  \item No known theoretical explanation in the Standard Model
  \item Suggests a deeper symmetry structure
\end{itemize}

In the $\mathbb{Z}_6$ framework (Ch2), the Koide formula appears naturally if:
\begin{equation}
\sqrt{m_k} \propto 1 + \sqrt{2}\cos\left(\theta + \frac{2\pi k}{3}\right), \quad k=0,1,2
\end{equation}
with $\theta = 2\pi/9$ (the ``Koide phase'').

The question: \emph{Why is $\theta = 2\pi/9$?}

% ==============================================================================
\section{What Is INPUT vs OUTPUT}
% ==============================================================================

\begin{inputbox}[Allowed Inputs (From Ch2)]
\begin{enumerate}[label=\textbf{I.\arabic*},nosep]
  \item \textbf{$\mathbb{Z}_6$ Symmetry}: Hexagonal lattice at brane interface
  \item \textbf{$\mathbb{Z}_3$ Subgroup}: The center of emergent SU(3)
  \item \textbf{Y-junction Structure}: Proton as $\mathbb{Z}_3$ fixed point
  \item \textbf{Three Generations}: Radial/angular harmonics of junction
  \item \textbf{120° Steiner Angles}: Equal flux tube tensions
\end{enumerate}
\end{inputbox}

\begin{outputbox}[Required Output (Must Derive)]
\begin{enumerate}[label=\textbf{O.\arabic*},nosep]
  \item \textbf{Mass Parametrization}: $\sqrt{m_k} = A(1 + \sqrt{2}\cos(\theta + 2\pi k/3))$
  \item \textbf{Koide Phase}: $\theta = 2\pi/9 \approx 40°$ from geometry
  \item \textbf{Koide Formula}: The $2/3$ ratio as a consequence
\end{enumerate}
\end{outputbox}

% ==============================================================================
\section{Candidate Derivation Strategy}
% ==============================================================================

\begin{hypothesis}[$\mathbb{Z}_9$ Extension]
The Koide phase $2\pi/9$ arises from a $\mathbb{Z}_9$ symmetry, which is the
semidirect product of $\mathbb{Z}_3$ (generations) with additional structure.
\end{hypothesis}

\subsection{Mathematical Approaches}

\textbf{Approach 1: Rotational constraint}

The Y-junction has $\mathbb{Z}_3$ symmetry. If the three generations are
rotations by $2\pi/3$, then their phase might be constrained by:
\begin{equation}
3\theta \equiv \text{something} \pmod{2\pi}
\end{equation}

For $\theta = 2\pi/9$: $3\theta = 2\pi/3$ (one-third rotation).

\textbf{Approach 2: Eigenvalue constraint}

The Koide formula is equivalent to:
\begin{equation}
\text{Tr}(M) = 2 \cdot \text{Tr}(\sqrt{M})^2 / 3
\end{equation}
where $M = \text{diag}(m_e, m_\mu, m_\tau)$.

This might arise from a constraint on the mass matrix eigenvalues.

\textbf{Approach 3: Geometric constraint}

In the hexagonal lattice, the angle between nearest neighbors is $60°$.
Perhaps the Koide phase relates to lattice geometry:
\begin{equation}
\theta = \frac{60°}{3} \cdot \frac{2}{3} = \frac{40°}{1} = 2\pi/9
\end{equation}

(This is speculative---needs rigorous derivation.)

% ==============================================================================
\section{Success Criteria}
% ==============================================================================

\begin{criterion}[Phase Value]
Derive $\theta = 2\pi/9$ from $\mathbb{Z}_6$ or $\mathbb{Z}_3$ geometry alone.
\end{criterion}

\begin{criterion}[No Fitting]
The phase must emerge from group theory, not be fitted to lepton masses.
\end{criterion}

\begin{criterion}[Extension to Quarks]
If the derivation is correct, it should predict whether/how the Koide formula
extends to the quark sector.
\end{criterion}

% ==============================================================================
\section{Why This Is Low Priority}
% ==============================================================================

This target is marked LOW priority because:
\begin{enumerate}
  \item The Koide formula, while elegant, does not affect other predictions
  \item It may be a numerical coincidence (though unlikely at 0.01\% precision)
  \item Solving RT-CH3-001 through RT-CH3-004 is more fundamental
\end{enumerate}

However, if the phase derivation succeeds, it would be strong evidence that
the $\mathbb{Z}_6$ framework captures deep structure of the Standard Model.

% ==============================================================================
\section{Historical Note}
% ==============================================================================

The Koide formula was discovered empirically by Yoshio Koide in 1981. Despite
40+ years of effort, no satisfactory explanation exists. EDC's $\mathbb{Z}_3$
structure offers a natural framework, but the specific phase remains unexplained.

\vspace{1em}
\hrule
\vspace{0.5em}
\textbf{Status}: OPEN \\
\textbf{Assigned}: Chapter 3 research program \\
\textbf{Target}: Derive Koide phase $\delta = 2\pi/9$ from $\mathbb{Z}_3$ geometry

\end{document}
