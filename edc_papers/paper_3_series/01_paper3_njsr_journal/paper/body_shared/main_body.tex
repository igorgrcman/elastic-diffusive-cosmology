\section{Introduction: Reading Physics from Both Sides of the Membrane}

% ============================================================
% PAPER 3 — INTRO NARRATIVE (Popular-science, but rigorous)
% ============================================================

Most of modern particle physics is written in the language of a \emph{closed} 3+1D world:
we postulate fields, write a Lagrangian, fit parameters, and interpret experiments as
interactions \emph{within} that 4D arena. Elastic Diffusive Cosmology (EDC) deliberately
changes the viewing geometry.

EDC treats our observed universe not as the full stage of reality, but as a \emph{boundary}
(or brane) embedded in a higher-dimensional plenum. The guiding idea is simple to state:
\emph{the equations we measure in 3D/4D are not necessarily the fundamental equations of nature;
they may be effective projections of a deeper dynamics that lives in the bulk}. In the
EDC book framework (Part I), the target is not to re-encode the Standard Model with new
symbols, but to construct a mathematically controlled 5D membrane action that yields
quantum and gravitational behavior as emergent, testable consequences.%
\footnote{This ``build from a 5D membrane action'' program is explicitly stated as the
organizing aim of the EDC Part I book record.} \cite{EDCPartI_Zenodo}

\subsection{A two-sided reading rule: Left side (5D cause) vs.\ right side (3D evidence)}

To make this paper readable, we adopt a strict two-sided rule.

\paragraph{Left side (5D).}
On the left side of the membrane we describe \emph{causes}: bulk geometry, junction
constraints, and the membrane degrees of freedom. The objects we call ``particles'' on the
right side are not taken as point-like primitives. Instead, they correspond to stable or
metastable \emph{geometric/topological configurations} of the membrane-bulk system
(defects, junctions, winding, or trapped flux in the higher-dimensional description).
This is the EDC ontological move: physics is encoded in geometry and boundary conditions,
not in an a priori particle list.

\paragraph{Right side (3D/4D).}
On the right side we describe \emph{evidence}: what detectors register.
Accelerators, spectroscopy, and precision clocks do not look into the bulk; they see only
the brane-projected signatures of bulk-membrane dynamics. The role of experiment is therefore
diagnostic: it constrains which 5D hypotheses are viable, because only certain bulk geometries
and junction laws can reproduce the observed right-side regularities.

This paper is written to keep that separation clear:
\begin{itemize}
\item The 5D construction is proposed as the \emph{mechanism} (left side).
\item The neutron lifetime is used as a \emph{calibrated test observable} (right side).
\end{itemize}

\subsection{Why the neutron: a metastable clock for bulk-to-brane dynamics}

Among simple hadronic systems, the free neutron occupies a special conceptual role:
it is not permanently stable, yet it is long-lived enough to be measured with high
precision. In the EDC reading, this makes the neutron a natural ``transition object'':
a metastable configuration that can relax from one geometric/junction state to another
through a well-defined barrier in the reduced dynamics.

The key hypothesis of the present \FrameworkNameShort\ program is that the decay of a free neutron can be
modeled as an effective one-dimensional collective coordinate \(q\) describing a
bulk-to-brane reconfiguration mode at a junction/defect vertex. The 5D dynamics reduces
to an effective action
\[
S_{\mathrm{eff}}[q] \;=\; \int dt\,
\left(\tfrac12 M(q)\dot q^{\,2} - V(q)\right),
\]
and the decay rate is controlled by semiclassical tunneling across the geometric barrier
encoded in \(V(q)\). The resulting lifetime is not inserted by hand; it becomes an output
of the model once \(M(q)\) and \(V(q)\) are specified by the reduction and the boundary
conditions.

\subsection{Not ``copying Standard Model numbers''}

A frequent failure mode of alternative frameworks is accidental numerology:
one can often engineer a new formula that reproduces a known constant without adding
mechanistic explanatory power. EDC explicitly tries to avoid that trap by enforcing
a directional logic:
\begin{center}
\emph{bulk geometry and junction laws} \(\rightarrow\)
\emph{effective reduced dynamics} \(\rightarrow\)
\emph{testable right-side observables}.
\end{center}

This philosophy is illustrated already in the companion report on particle geometry,
where the fine-structure constant is presented as a derivation within a 5D membrane
cosmology framework in a frozen-regime limit, with reproducibility comparisons and
supplementary derivation indexing. \cite{EDCAlphaReport_Zenodo}
In the present paper we apply the same ``geometry-first, evidence-second'' logic to the
neutron lifetime.

\subsection{What the reader should expect}

The main text is organized for comprehension rather than for forensic completeness.
We begin with a narrative and a minimal set of definitions required to interpret the
two-sided picture, then present the effective reduction and the decay-time prediction.
All detailed derivation chains, lemma stacks, and audit artifacts are placed in appendices
so that a skeptical reader can reproduce each step without forcing every reader to start
with 100+ pages of intermediate algebra.

\paragraph{Epistemic labeling.}
Throughout, we maintain explicit epistemic tags for key statements:
\([Der]\) for fully derived results within the declared model assumptions,
\([Dc]\) for decisively constrained constructions that still contain modeling choices,
\([P]\) for proposals, and \([I]\) for identified empirical inputs. This is a practical
guardrail: it keeps the paper falsifiable, and prevents ``model rhetoric'' from blending
into claims of established physics.

\paragraph{Navigation guide.}
The document is structured for different reading depths:
\begin{itemize}
    \item \textbf{Main text} ($\sim$25 pages): narrative, key results, and conclusions---sufficient for understanding the claim and its epistemic status.
    \item \textbf{Appendices} ($\sim$110 pages): full forensic derivation chain, verification gates, and worked calculations---for readers who wish to audit every step.
    \item \textbf{Technical boxes}: self-contained derivations with explicit assumptions and status tags.
\end{itemize}
A skeptical reader should be able to verify any claim by tracing it to the relevant appendix; a casual reader can follow the main argument without drowning in algebra.

% End of intro narrative

% ============================================================================
% ORIGINAL INTRODUCTION CONTENT (preserved)
% ============================================================================

The matter-antimatter asymmetry (baryogenesis) remains one of the most
profound unsolved problems in cosmology. The Standard Model combined with
Big Bang cosmology predicts equal amounts of matter and antimatter should
have been created, yet we observe a universe dominated by matter.

This paper proposes a geometric resolution within 5D Elastic Diffusive
Cosmology (EDC), building on (i)~the \emph{EDC Theory Book} v17.49~\cite{EDC_Book}
and (ii)~the companion paper \emph{Geometric Structure of Electron and Proton
in 5D Membrane Cosmology}~\cite{EDC_Paper2}.

\subsection{Program Scope and Contribution}

This paper develops an \textbf{effective 1D collective-coordinate model} for
particle creation and stability within EDC. The scope encompasses two
interconnected phenomena:

\begin{enumerate}
    \item \textbf{Matter-Antimatter Asymmetry:} We propose that cosmological
    particle creation draws energy from the 5D Plenum (INFLOW), eliminating
    the need for antimatter as a conservation partner. The Sakharov conditions
    are not required in this framework.

    \item \textbf{Particle Stability Hierarchy:} The INFLOW/OUTFLOW mechanism
    provides a conceptual basis for understanding why some particles (proton,
    electron) are stable while others (neutron, muon) decay. Appendix~\ref{app:neutron}
    develops the neutron case as a worked example.
\end{enumerate}

\noindent\textbf{What this paper does:}
\begin{itemize}
    \item Establishes the INFLOW/OUTFLOW formalism (\S\ref{sec:inflow})
    \item Derives stability conditions under explicit assumptions A1--A6 (Appendix~\ref{app:formalization})
    \item Constructs a computable 5D$\to$1D reduction pipeline (Appendix~\ref{app:5d_reduction})
    \item Demonstrates that verification gates pass for the effective model
\end{itemize}

\noindent\textbf{What this paper does NOT do:}
\begin{itemize}
    \item Derive the effective potential $V(q)$ from the full 5D action (remains \Open)
    \item Predict the neutron lifetime $\tau_n = 879$\,s from first principles (remains \Open)
    \item Compute the baryon-to-photon ratio $\eta \sim 10^{-10}$ (no quantitative prediction)
\end{itemize}

\subsection{Roadmap of Evidence}

The logical structure of this paper follows a claim$\to$derivation$\to$verification pattern:

\begin{enumerate}
    \item \textbf{Claim C1:} INFLOW is energetically favored over OUTFLOW \\
    \textit{Derivation:} \S\ref{sec:inflow}, Appendix~\ref{app:formalization} (under A1--A6) \\
    \textit{Status:} \Dc\ (conditional on assumptions)

    \item \textbf{Claim C2:} Cosmological creation does not require antimatter \\
    \textit{Derivation:} \S\ref{sec:cosmo} (5D current conservation) \\
    \textit{Status:} \Dc\ (follows from C1)

    \item \textbf{Claim C3:} The 5D$\to$1D reduction yields computable $V(q)$, $M(q)$ \\
    \textit{Derivation:} Appendix~\ref{app:5d_reduction} \\
    \textit{Verification:} 10/10 gates pass (Table~\ref{tab:gates})

    \item \textbf{Claim C4:} Neutron instability fits the INFLOW framework \\
    \textit{Derivation:} Appendix~\ref{app:neutron} \\
    \textit{Status:} \Pp\ (conceptual fit; lifetime not derived)
\end{enumerate}

% ============================================================================
\subsection{From Baryogenesis to Stability: Why the Neutron is the First Quantitative Stress-Test}
\label{sec:narrative_bridge}
% ============================================================================

The EDC research program addresses three interlocking questions: (1) Why is there more matter
than antimatter? (2) Why are protons and electrons stable while neutrons decay? (3) Can 5D
geometry predict particle lifetimes quantitatively? This paper focuses on the third question,
but the logical arc requires situating the neutron within the broader stability hierarchy.

\paragraph{Matter-Antimatter Resolution (EDC Theory Book v17.49).}
The INFLOW/OUTFLOW framework \cite{EDC_Book} proposes that cosmological particle creation
draws energy from the 5D Plenum, eliminating the need for antimatter as a conservation partner.
If correct, the Standard Sakharov conditions are not required (\S\ref{sec:cosmo}).

\paragraph{Reviewer Question: ``Are These Defects Stable?''}
If matter consists of INFLOW defects and primordial antimatter never existed, a natural
follow-up asks: \textit{what sets the stability/metastability of these defects?} Why do
protons appear absolutely stable ($\tau_p > 10^{34}$ yr) while free neutrons decay in 879\,s?

\paragraph{Proton Stability (Paper 2 \S5.2).}
The proton is modeled as a symmetric \textbf{Y-junction} in quark flux-tube space, with
configuration space $S^3 \times S^3 \times S^3$ and geometric coefficient
$C_p = (2\pi^2)^3$ \cite{EDC_Paper2}. This configuration minimizes energy analogously to
the Steiner minimal tree (120$^\circ$ junctions). Paper~2 derives the prediction
$m_p/m_e = 6\pi^5 = 1836.118$ with 0.002\% agreement, establishing that the proton's
Y-junction geometry is energetically optimal and thus stable.

\paragraph{Why Neutron Now?}
The \textbf{neutron} is the minimal metastable baryon---an \textit{asymmetric} Y-junction
with stored configurational energy $\Delta E_{\rm config} > 0$. Unlike the proton, it can
lower its energy by decaying to proton$+$electron$+\bar{\nu}_e$. Crucially, the neutron
provides a clean observable: $\tau_n = 878.4 \pm 0.5$\,s \cite{PDG2024}. This makes the
neutron the tightest calibration target for the effective 1D WKB pipeline developed in
Appendix~\ref{app:neutron_lifetime}.

\paragraph{Why Not Electron?}
The electron belongs to the leptonic sector, which in EDC is a different topological class
(point defect vs.\ flux-tube junction). The Y-junction / Steiner framework of Paper~2 does
not directly apply. Electron stability is deferred to future work.

\paragraph{Why Not Deuteron/Helium?}
Multi-baryon systems (deuteron, ${}^3$He, ${}^4$He) require \textit{nuclear binding}---a
many-body interaction layer that involves the potential landscape of multiple defects.
This is a distinct problem from single-particle metastability. Paper~3 deliberately restricts
scope to the \textbf{single-coordinate WKB program}; nuclear binding is deferred.

\paragraph{Scope Statement.}
Paper~3 is the \textit{first quantitative metastability test} within EDC. Proton stability
(geometric optimality) is treated in Paper~2; the global 5D framework and epistemic taxonomy
are documented in Book~I \cite{EDC_Book}. Readers seeking background definitions should
consult those references.

\vspace{0.3cm}
\noindent\fbox{\parbox{0.97\textwidth}{%
\textbf{Scope \& Limitations (Reviewer Note).}
This paper does \textbf{not} attempt nuclear binding (deuteron/helium) nor leptonic stability
(electron/muon). It targets the \textbf{minimal metastable baryon}---the free neutron---as
the first quantitative closure test of the 5D$\to$1D effective model. For proton stability,
see Paper~2 \S5.2 (Y-junction geometry) \cite{EDC_Paper2}. For foundational definitions
(Plenum, topological defects, epistemic tags), see Book~I \cite{EDC_Book}.
}}

\vspace{0.3cm}
\noindent\fbox{\parbox{0.97\textwidth}{%
\textbf{Reader Pointers.}
\begin{itemize}
    \item[\textbullet] \textbf{5D framework:} Book~I \cite{EDC_Book}, Ch.~2 ($\mathcal{M}^5$, Plenum), Ch.~3 (membrane $\Sigma^3$)
    \item[\textbullet] \textbf{Topological defects:} Book~I \cite{EDC_Book}, Ch.~4 (particle = pore)
    \item[\textbullet] \textbf{INFLOW/OUTFLOW:} Book~I \cite{EDC_Book}, Ch.~5 (matter vs.\ antimatter)
    \item[\textbullet] \textbf{Epistemic tags:} Book~I \cite{EDC_Book}, Appendix~A ([P], [D], [Dc], [BL])
    \item[\textbullet] \textbf{Proton Y-junction:} Paper~2 \cite{EDC_Paper2}, \S5.2 ($S^3 \times S^3 \times S^3$, Steiner)
    \item[\textbullet] \textbf{$m_p/m_e = 6\pi^5$:} Paper~2 \cite{EDC_Paper2}, \S5.2--5.3 (0.002\% agreement)
\end{itemize}
}}

% ============================================================================
\section{EDC Framework Review}
% ============================================================================

\subsection{5D Geometry}

The EDC spacetime is a 5D manifold $\mathcal{M}^5$ with metric
(see Book~I \cite{EDC_Book}, Ch.~2 for foundational definitions):
\begin{equation}
ds^2 = g_{\mu\nu} dx^\mu dx^\nu + d\xi^2
\end{equation}
where $\xi$ is the compactified 5th dimension and $g_{\mu\nu}$ is the 4D
metric on the membrane $\Sigma^3$ (Book~I \cite{EDC_Book}, Ch.~3).

\subsection{Plenum Energy Fluid}

The bulk contains an energy fluid (Plenum) with positive pressure
(see Book~I \cite{EDC_Book}, Ch.~2 for Plenum postulates):
\begin{equation}
P_{\rm bulk} = w \rho_{\rm P} c^2, \quad w > 0
\end{equation}

This positive pressure drives energy flow toward the membrane.

\subsection{Epistemic Status Tags}

Throughout this paper, claims are tagged by derivation status
(see Book~I \cite{EDC_Book}, Appendix~A for the complete taxonomy):
\begin{table}[h]
\centering
\small
\begin{tabular}{cl}
\toprule
\textbf{Tag} & \textbf{Meaning} \\
\midrule
\Pp & Postulated / phenomenological ansatz \\
\Dc & Derived conditional on explicit assumptions \\
\Dd & Definition (convention choice) \\
\Mm & Mathematical identity or theorem \\
\Ii & Identified / fit / empirical \\
\Cal & Calibrated to data \\
\BL & Baseline (experimental value, e.g., PDG) \\
\Open & Unresolved / requires future work \\
\bottomrule
\end{tabular}
\caption{Epistemic status tags used in this paper.}
\label{tab:tags}
\end{table}

\subsection{Particle Classification: Where They Live}
\label{subsec:particle_locations}

The EDC framework assigns particles to geometric configurations with specific
brane vs.\ bulk localization properties. The following classification uses
the canonical terminology from the Framework Reference Document v2.0~\cite{grcman2026_framework_v2}:

\begin{description}
    \item[Electron] \Dc\ Membrane-confined surface defect (brane-local).
        The electron is a simple vortex with winding number $W = -1$ around the
        compact dimension, configuration space $B^3$, and mass determined by
        flux-dominated boundary conditions. Fully localized on the brane.

    \item[Baryons (proton, neutron)] \Dc\ Y-junction / volume defects with
        bulk extension via flux tubes. The proton is a symmetric Y-junction
        ($S^3 \times S^3 \times S^3$ configuration space); the neutron is an
        asymmetric Y-junction with stored configurational energy. Flux tubes
        extend into the bulk but are confined by junction topology.

    \item[Neutrino] \Pp\ Partially bulk-coupled mode. Unlike charged leptons,
        the neutrino is postulated to have low brane confinement, coupling
        primarily through weak (junction-slip) interactions. Its bulk coupling
        enables escape during $\beta$-decay.

    \item[Antiparticles] \Dc\ Ledger partners for conservation closure.
        CPT symmetry requires opposite quantum numbers; antiparticles are
        brane-localized defects with opposite winding/charge. The ``conservation
        ledger'' tracks quantum numbers across brane and bulk channels---this is
        bookkeeping language, not a new physical law.
\end{description}

\noindent\textbf{Key distinction:} Charged leptons ($e, \mu, \tau$) are brane-confined;
neutrinos are partially bulk-coupled \Pp; baryons have Y-junction topology with
flux-tube extensions. This classification determines decay channels and stability.

% ============================================================================
\section{From 5D Action to Effective 1D Barrier: What is Derived vs What is Tested}
\label{sec:5d_to_1d}
% ============================================================================

\paragraph{Reduction Roadmap.}
The full 5D action contains four components:
\begin{equation}
S_{\rm 5D} = S_{\rm bulk}[g_{AB}] + S_{\rm brane}[\Phi, h_{\mu\nu}]
           + S_{\rm GHY}[K] + S_{\rm Israel}[\Delta K]
\label{eq:S5D_components}
\end{equation}
where $S_{\rm bulk}$ is the 5D Einstein-Hilbert action with cosmological constant,
$S_{\rm brane}$ encodes membrane tension and matter fields, $S_{\rm GHY}$ is the
Gibbons-Hawking-York boundary term ensuring a well-posed variational principle,
and $S_{\rm Israel}$ enforces junction conditions across the brane.
Under a collective-coordinate ansatz $q(t)$ representing the defect's ``openness,''
dimensional reduction yields the effective 1D action:
\begin{equation}
S_{\rm eff}[q] = \int dt \left( \frac{1}{2} M(q)\, \dot{q}^2 - V(q) \right)
\label{eq:Seff_1D}
\end{equation}
where $M(q)$ is the effective mass (derived from kinetic terms) and $V(q)$ is the
effective potential (derived from energy costs of membrane deformation).
For explicit reduction steps, see Appendix~\ref{app:5d_reduction}.

\paragraph{Status Map.}
Table~\ref{tab:status_map} summarizes the epistemic status of each component in the
derivation chain, distinguishing what is rigorously derived versus what remains
postulated or calibrated.

\begin{table}[H]
\centering
\small
\begin{tabular}{lccl}
\toprule
\textbf{Quantity} & \textbf{Status} & \textbf{Source} & \textbf{Appendix} \\
\midrule
$q$ collective coordinate & \Dd & Geometric definition & App.~\ref{app:q_geometry} \\
$V(q)$ effective potential & \Dc & 5D reduction integral (under ansätze) & App.~\ref{app:worked_derivation} \\
$M(q)$ effective mass & \Dc & 5D kinetic term reduction & App.~\ref{app:worked_derivation} \\
Bulk metric $g_{AB}$ & \Pp & RS-type warped ansatz (choice) & App.~\ref{app:worked_derivation} \\
Brane profile $f(r;q)$ & \Pp & Gaussian ansatz (choice) & App.~\ref{app:worked_derivation} \\
WKB exponent $B$ & \Dd & Standard WKB from $V(q), M(q)$ & App.~\ref{app:neutron_lifetime} \\
Prefactor $A_0$ & \Pp & Functional determinant (toy form) & App.~\ref{app:5d_reduction} \\
Exponent $p = 5/16$ & \Ii & Identified from scaling analysis & Main text \S\ref{sec:roadmap} \\
Sign $\delta S_{\rm INFLOW} < 0$ & \Dc & Under assumptions A1--A6 & App.~\ref{app:sign_derivation} \\
Amplitude $V_B$ & \Open & Calibration needed & --- \\
\bottomrule
\end{tabular}
\caption{Status map: epistemic classification of key quantities in the 5D $\to$ 1D reduction.
\Dd\ = derived, \Dc\ = derived conditional, \Pp\ = postulated, \Ii\ = identified/fit, \Open\ = unresolved.}
\label{tab:status_map}
\end{table}

\paragraph{Calibration Ledger.}
To prevent ambiguity about ``1 free parameter fitting 1 datum,'' we state explicitly:
\begin{itemize}
    \item \textbf{Calibrated:} $V_B \to \tau_n$ \Cal. The barrier-height scale is the \emph{only} quantity fitted to the observed neutron lifetime.
    \item \textbf{Decisively constrained:} $V(q)$, $M(q)$ functional forms under declared ansätze \Dc---these shapes are not tuned once the ansätze are fixed.
    \item \textbf{Derived (conditional on BVP/E-L structure):} $\phitail = (1+\sqrt{5})/2$ \Dc---emerges from asymptotic analysis of the soliton ODE, not fitted to data.
    \item \textbf{Derived (method-spread systematic):} $\Rdet = 0.63 \pm 0.10$ \Dc---two independent methods yield a spread that we treat as systematic uncertainty, not as a second calibration knob.
    \item \textbf{Diagnostics/gates:} $V \geq 0$, $M \geq 0$, singularity handling, sensitivity gates \Dc---pass/fail criteria that constrain but do not tune.
\end{itemize}
The key distinction: \emph{once $V_B$ is fixed}, the remaining outputs are determined by the reduction pipeline without additional degrees of freedom.

\paragraph{Forensic Audit Rule.}
Throughout this paper, we maintain a strict epistemic discipline: \textbf{the 5D action
is the formal source of truth}; 1D effective models and their numerical implementations
serve as validation layers. Any claim marked \Dd\ must trace back to the action without
calibration. Claims marked \Dc\ or \Pp\ are explicitly flagged as requiring future upgrade.
Numerical gates (Table~\ref{tab:gates} in Appendix~\ref{app:5d_reduction}) verify
internal consistency but do not establish physical correctness.

% ============================================================================
\section{INFLOW vs OUTFLOW Defects}
\label{sec:inflow}
% ============================================================================

This section summarizes the INFLOW/OUTFLOW framework established in
Book~I \cite{EDC_Book}, Ch.~5 (see also \S\ref{sec:narrative_bridge} for context).

\subsection{Definition}

We define the energy flux in the $\xi$-direction:
\begin{equation}
J^\xi = T^\xi_{\ \mu} u^\mu
\end{equation}

\begin{itemize}
    \item \textbf{INFLOW:} $J^\xi > 0$ (energy flows from Plenum to membrane)
    \item \textbf{OUTFLOW:} $J^\xi < 0$ (energy flows from membrane to Plenum)
\end{itemize}

\medskip\noindent
\textbf{Canonical energy-exchange statement} (Framework v2.0, Remark~4.5):
\[
\nabla_A T^{AB}_{(5)}=0\ (A,B=0,\dots,4),\qquad
\nabla_\mu T^{\mu\nu}_{\mathrm{brane}}=-\,\Jbb{\nu}\ (\mu,\nu=0,\dots,3).
\]
Sign convention: $\Jbb{\nu}>0$ denotes bulk$\to$brane (INFLOW).

\emph{This block is quoted verbatim from Framework v2.0, Remark~4.5; Framework remains the canonical source.}

\subsection{Stability Analysis}

For coupling action:
\begin{equation}
S_{\rm coupling} = \int d^4x\, d\xi\, J^A \partial_A \Phi
\end{equation}

Under assumptions A1--A6 (see Appendix D), we derive \Dc:
\begin{align}
\delta S_{\rm INFLOW} &< 0 \quad \text{(energetically favored)} \\
\delta S_{\rm OUTFLOW} &> 0 \quad \text{(energetically suppressed)}
\end{align}

\textit{[Detailed derivation and assumption list in Appendix D]}

\vspace{0.3cm}
\noindent\fbox{\parbox{0.97\textwidth}{%
\textbf{KB-OPEN-009: Unresolved Sign Issue.}
The claim $\delta S_{\rm INFLOW} < 0$ is currently \Dc\ (derived conditional on
assumptions A1--A6). A rigorous derivation from the full 5D action has not been
completed. If the sign turns out to be reversed under proper calculation, the
INFLOW/OUTFLOW framework would require fundamental revision. This is flagged as
an open problem. See Roadmap (\S\ref{sec:roadmap}, Step 4) for upgrade path.
}}

% ============================================================================
\section{Cosmological vs Local Creation}
\label{sec:cosmo}
% ============================================================================

\subsection{Cosmological Creation (Big Bang)}

In an open 5D system:
\begin{equation}
\partial_A J^A = 0 \quad \Rightarrow \quad
\partial_\mu J^\mu + \partial_\xi J^\xi = 0
\end{equation}

The Plenum flux $\Phi_\xi = \int d^3x\, \partial_\xi J^\xi$ compensates
any 3D charge creation:
\begin{equation}
\Delta B_{\rm 3D} + \Phi_\xi = 0
\end{equation}

\textbf{Result:} Antimatter is \emph{not required} for charge conservation
because the Plenum provides compensation.

\subsection{Local Creation (LHC)}

In a closed 3D process with $\Phi_\xi \approx 0$:
\begin{equation}
\Delta B_{\rm 3D} = 0 \quad \Rightarrow \quad
\text{particle} + \text{antiparticle pairs required}
\end{equation}

% ============================================================================
\section{Comparison with Sakharov Conditions}
% ============================================================================

\begin{table}[h]
\centering
\small
\begin{tabular}{lccc}
\toprule
\textbf{Condition} & \textbf{Standard} & \textbf{EDC} & \textbf{EDC Replacement} \\
\midrule
B violation & Required & Not required & Plenum reservoir \Open \\
C and CP violation & Required & Not required & Asymmetric $J^\xi$ sign \Dc \\
Thermal non-equilibrium & Required & Not required & Cosmological INFLOW epoch \Pp \\
\bottomrule
\end{tabular}
\caption{Sakharov conditions in EDC: ``not required'' shifts burden to Plenum properties.
The EDC replacement requirements are flagged with their epistemic status.}
\label{tab:sakharov}
\end{table}

% ============================================================================
\section{Observational Consistency}
% ============================================================================

The INFLOW/OUTFLOW framework is \emph{consistent with} current observations, but
these observations do not uniquely distinguish EDC from other baryogenesis scenarios:

\begin{enumerate}
    \item \textbf{No primordial antimatter} \Pp: The EDC framework proposes antimatter was
    never cosmologically created. AMS-02 has detected no primordial antihelium \cite{AMS02},
    consistent with this proposal. \textit{Note: Standard cosmology with Sakharov mechanisms
    also predicts negligible primordial antimatter.}

    \item \textbf{No annihilation gamma background:} Without primordial
    antimatter domains, no massive annihilation occurred. Fermi-LAT observes
    no excess at $\sim 1$ GeV \cite{FermiLAT,FermiGamma}. \textit{Note: This is also
    consistent with standard baryogenesis.}

    \item \textbf{LHC pair production:} Local processes must create pairs ($\Delta B = 0$).
    \textit{Status: Confirmed experimentally.} This is required by both EDC and
    standard physics.
\end{enumerate}

\paragraph{What Would Falsify This Framework:}
\begin{itemize}
    \item Detection of primordial antimatter (antihelium-3 or heavier) by AMS-02
    \item Evidence of annihilation gamma background from matter-antimatter domains
    \item Derivation showing $\delta S_{\rm INFLOW} > 0$ from the full 5D action
\end{itemize}

% ============================================================================
\section{Connection to Particle Stability}
% ============================================================================

The INFLOW/OUTFLOW framework provides a conceptual basis \textbf{[P]} for
understanding particle stability hierarchies.
In particular, the neutron instability and its decay products are analyzed
in Appendix A, connecting the matter-antimatter asymmetry to weak decay
processes within the effective 1D model.

% ============================================================================
\section{Future Work}
% ============================================================================

\begin{enumerate}
    \item Complete mathematical derivation of $\delta S$ for INFLOW/OUTFLOW
    \item Derive the origin of $P_{\rm bulk} > 0$ from first principles
    \item Connect to muon/tau lepton mass hierarchy
    \item Investigate implications for dark matter candidates
\end{enumerate}

% ============================================================================
\section{Roadmap: From Effective Model to 5D Derivation}
\label{sec:roadmap}
% ============================================================================

The current framework uses phenomenological parameters. A genuine 5D derivation
would proceed as follows:

\paragraph{Step 1: Write Down Full 5D Action \Dc}
\begin{equation}
S_{\rm tot} = S_{\rm bulk}[g_{AB}] + S_{\rm membrane}[\Phi,h_{\mu\nu}]
+ S_{\rm GHY}[K] + S_{\rm defect}[q]
\end{equation}
with explicit functional forms for each term. \textit{Status: bulk and membrane terms
established in Book I; defect term requires topological specification.}

\paragraph{Step 2: Specify Defect Topology \Open}
Define the map $\phi: S^2 \to \mathcal{M}^5$ characterizing the pore structure.
This determines the winding number and energetic cost of the defect.
\textit{Status: not yet specified.}

\paragraph{Step 3: Perform KK Reduction to 1D \Open}
Integrate out angular and transverse degrees of freedom to obtain:
\begin{equation}
S_{\rm eff}[q] = \int dt \left( \frac{1}{2}M(q)\dot{q}^2 - V(q) \right)
\end{equation}
The effective potential $V(q)$ should emerge from the reduction, not be postulated.
\textit{Status: ansatz used, not derived.}

\paragraph{Step 4: Verify $\delta S_{\rm INFLOW} < 0$ from Full Action \Open}
Currently claimed under assumptions A1--A6. A true derivation would show this
sign follows from the action without additional postulates.
\textit{Status: claimed \Dc, needs upgrade to \Dd.} See KB-OPEN-009.

\paragraph{Step 5: Derive $p$ from $\mathcal{M}^5$ Topology \Open}
The suppression exponent $p = 5/16$ is currently \Ii\ (identified via scoring).
A derivation would connect $p$ to the spectral dimension or Laplacian eigenvalues
on the compactified space. \textit{Status: identified, not derived.}

\paragraph{Step 6: Compute Observables Without Calibration \Open}
A successful 5D derivation would predict:
\begin{itemize}
    \item $\tau_n = 879$ s from EDC parameters alone (currently \Cal)
    \item $V(q)$ functional form from topology (currently \Pp)
    \item $\eta \sim 10^{-10}$ baryon asymmetry (currently no prediction)
\end{itemize}

\paragraph{What Would Count as a True 5D Derivation:}
\begin{enumerate}
    \item Compute $\tau = 879$ s from EDC parameters without calibrating $V_B$
    \item Derive the functional form $V(q) = 16V_B q^2(1-q)^2 + Qq$ from the action
    \item Show $p = 5/16$ follows uniquely from geometric constraints
    \item Derive $\delta S_{\rm INFLOW} < 0$ from the action (not assumptions)
\end{enumerate}

% ----------------------------------------------------------------------------
\subsection{Neutron Upgrade Roadmap: From 5D-Induced to 5D-Computed}
\label{sec:neutron_upgrade}
% ----------------------------------------------------------------------------

The neutron decay calculation in Appendix~\ref{app:neutron} uses phenomenological
potentials $V(q)$ and $M(q)$. Here we outline how these could, in principle,
be computed from explicit 5D geometry.

\paragraph{Phase 1: Specify Computable 5D Setup \Open}
\begin{enumerate}
    \item \Dd\ \textbf{Choose bulk metric:} e.g., $\text{AdS}_5$ with
    $ds^2 = e^{-2|y|/\ell}(\eta_{\mu\nu}dx^\mu dx^\nu) + dy^2$
    where $\ell$ is the AdS curvature radius. (Standard RS-type geometry.)

    \item \Dd\ \textbf{Define brane embedding:} Specify $\Sigma^3$ as a hypersurface
    with induced metric $h_{\mu\nu}$ and extrinsic curvature $K_{\mu\nu}$.
    (See Appendix~\ref{app:5d_reduction}.)

    \item \Pp\ \textbf{Choose defect ansatz:} Parameterize the pore profile as
    $\phi(r; q) = q \cdot f_0(r)$ where $f_0(r)$ is a fixed shape function
    and $q \in [0,1]$ is the collective coordinate. The functional form of
    $f_0$ determines $M(q)$ and $V(q)$ via the reduction integrals.

    \item \Open\ \textbf{Execute reduction integrals:} Evaluate
    \begin{align}
    M(q) &= \int_{\Sigma^3} d^3x \sqrt{h}\, g_{AB}\,
    \frac{\partial X^A}{\partial q}\frac{\partial X^B}{\partial q} \\
    V(q) &= S_{\rm brane}[\phi(q)] - S_{\rm brane}[\phi(0)]
    \end{align}
    for the chosen ansatz. \textit{Status: not yet executed.}
\end{enumerate}

\paragraph{Phase 2: Numerical Verification Gates \Dc}
Before any $V(q)$, $M(q)$ can be used in WKB calculations, the following
gates must pass:
\begin{itemize}
    \item \texttt{Vq\_positive\_gate()}: $V(q) > 0$ for $q \in (0, q_{\rm max})$
    \item \texttt{Mq\_positive\_gate()}: $M(q) > 0$ for all $q$
    \item \texttt{grid\_refinement\_gate()}: Integrals converge under mesh refinement
    \item \texttt{reparam\_invariance\_gate()}: WKB rate independent of coordinate choice
    \item \texttt{reduction\_integral\_nontrivial\_gate()}: $V(q)$, $M(q)$ differ from historical model
    \item \texttt{historical\_model\_usage\_gate()}: Default uses 5D-computed functions
    \item \texttt{vm\_shape\_sanity\_gate()}: Boundary conditions and single-barrier shape
\end{itemize}
\textit{Status: All gates pass for Phase-1 ansatz. See \texttt{code/neutron\_wkb\_sensitivity.py}.}

\paragraph{Phase-1 Closure Result \Dc:}
In Phase-1 we close the reduction recipe in computable integral form under an
explicit ansatz registry \Pp. The resulting $V(q)$ and $M(q)$ are numerically
evaluated from the integrals defined in Appendix~\ref{app:5d_reduction} and pass
all refinement and reparameterization gates \Dc. This demonstrates that the
pipeline $\text{5D} \to (V(q), M(q)) \to B \to \tau$ is \emph{executable}, not
merely a narrative. However, this does \textbf{not} constitute a derivation of
the bulk geometry or defect profile---those remain \Pp\ (see Ansatz Registry,
Appendix~\ref{app:5d_reduction}). Calibration of the overall scale $V_B$ remains
\Open\ pending higher-principle input.

\paragraph{Phase-2 Prefactor Status \Dc:}
Phase-2 replaces multiple historical prefactor choices (Fermi constant, attempt
frequency) with a single 5D-motivated candidate $A_0^{\rm 5D}$ \Dc, computed from
Gel'fand--Yaglom determinant ratios under transverse-mode assumptions \Pp.
The resulting prefactor passes finiteness and $p$-stability gates (see
\texttt{code/neutron\_wkb\_sensitivity.py}). However, the exponent selection
$p \in \{5, 6, 7\}$ remains prefactor-sensitive \Open\ until the transverse
sector is derived from the full 5D action rather than postulated.

\paragraph{Numerical Results Summary:}
Table~\ref{tab:numerical_results} summarizes the key numerical outputs from
the verification pipeline. These values are computed by running
\texttt{code/neutron\_wkb\_sensitivity.py --no-phase3} and represent actual
reduction integral outputs, not calibrated fits.

\begin{table}[H]
\centering
\small
\begin{tabular}{lll}
\toprule
\textbf{Quantity} & \textbf{Value} & \textbf{Status} \\
\midrule
\multicolumn{3}{l}{\textit{Positivity \& Shape Gates (Phase-1)}} \\
$\min V(q)$, $q \in (0.01, 0.99)$ & $2.05 \times 10^{-6}$ & PASS \\
$\min M(q)$, $q \in [0, 1]$ & $2.82 \times 10^{-5}$ at $q=0.53$ & PASS \\
$V(q)$ peak location & $q = 0.52$ (single barrier) & PASS \\
\midrule
\multicolumn{3}{l}{\textit{Convergence \& Invariance Gates (Phase-2)}} \\
Grid refinement: $I_{\rm coarse}$ vs $I_{\rm fine}$ & rel.\ diff $< 2 \times 10^{-7}$ & PASS \\
Reparam invariance: $I_{\rm orig}$ vs $I_{\rm trans}$ & rel.\ diff $< 4 \times 10^{-10}$ & PASS \\
$A_0^{\rm 5D}$ stability & max rel.\ diff $< 0.01\%$ & PASS \\
WKB $B$ integral stability & rel.\ diff $< 0.02\%$ & PASS \\
\midrule
\multicolumn{3}{l}{\textit{Gate Summary}} \\
Total gates executed & 10 & -- \\
Gates passed & 10 & ALL PASS \\
\bottomrule
\end{tabular}
\caption{Numerical verification results from \texttt{neutron\_wkb\_sensitivity.py}.
For detailed gate definitions and V/M comparison tables, see Appendix~\ref{app:5d_reduction}.}
\label{tab:numerical_results}
\end{table}

\paragraph{Phase-3 Computational Closure \Dc:}
The computed profile $f^*(r;q)$ is now solved as a boundary value problem (BVP)
using \texttt{scipy.integrate.solve\_bvp}, reformulating the E-L equation as:
$y = [f, f']$, with $f'(0) = 0$ (regularity) and $f(r_{\rm max}) = 0$ (Dirichlet).
Gates 15--17 verify convergence ($\geq 4/5$ q-points), refinement stability
($N = 200 \to 800$, $<2\%$), and solver consistency. This closes the computational
pipeline; it does not fix the bulk geometry \Pp\ or derive the scale $V_B$ \Open.

\paragraph{Phase-4: Full 5D Derivation Status.}
The target is to derive $S_{\rm eff}[q] = \int dt\, (\tfrac{1}{2}M(q)\dot{q}^2 - V(q))$
directly from the 5D action $S_{\rm 5D} = S_{\rm bulk} + S_{\rm brane} + S_{\rm GHY}$
without phenomenological gluing. Three steps are required:
\begin{enumerate}
\item[(i)] \textbf{Bulk metric family}: Specify $g_{AB}(x,\xi;\theta)$ with
warp factor $a(\xi;\theta)$ and boundary conditions at $\xi=0$ (brane) and
$\xi\to\infty$ (Plenum). Status: \Pp\ (RS-type ansatz chosen).
\item[(ii)] \textbf{Brane embedding ansatz}: Define $X^A(\sigma^\mu; q)$ for
defect with collective coordinate $q \in [0,1]$. Status: \Pp\ (Gaussian profile chosen).
\item[(iii)] \textbf{Second-order expansion} \Dc: Expand $S_{\rm 5D}[q,\dot{q}]$
around static profile to $\mathcal{O}(\dot{q}^2)$, extracting $M(q)$ (supermetric)
and $V(q)$ (static energy). Status: \textbf{\Dc\ (completed under ansätze)}.
See Appendix~\ref{app:worked_derivation} for the fully worked derivation.
\end{enumerate}
\textbf{Upgrade path:} Steps (i)--(ii) remain \Pp. Deriving the bulk metric and brane
profile from EDC Plenum principles is \Open.

\paragraph{Success Criterion (Full 5D):}
The upgrade from \Pp\ to \Dd\ is complete when $V(q)$ and $M(q)$ are computable
from specified 5D geometry without fitting to the observed $\tau_n = 878.4\,$s.

\subsection{Post-calibration outputs with fixed $V_B$}
\label{sec:post_calibration}

Once the barrier-height scale $V_B$ is fixed by matching $\tau_n = 878.4$\,s, the framework
produces constrained outputs \emph{without further tuning}. These serve as discriminating
constraints rather than additional fit parameters:

\begin{itemize}
    \item \textbf{Profile-family sensitivity:} Switching from Gaussian to compact-support
    or exponential brane profiles (with $V_B$ held fixed) yields $\tau_n$ variations that
    can be compared to experimental precision. Large deviations would falsify the profile choice.
    \item \textbf{Gate robustness:} The 10/10 verification gates (Table~\ref{tab:numerical_results})
    must continue to pass under parameter perturbations; failure indicates model breakdown.
    \item \textbf{Golden-ratio stability:} The tail exponent $\phitail = (1+\sqrt{5})/2$ is
    insensitive to fit-range choices ($< 2\%$ variation over $r \in [2,5]$ to $[3,6]$), providing
    an internal consistency check independent of $V_B$.
\end{itemize}

\noindent
These post-calibration diagnostics illustrate what ``predictive once calibrated'' means operationally:
the single calibration point ($V_B \to \tau_n$) constrains the entire pipeline, and deviations in
secondary outputs would signal model inconsistency rather than additional tuning opportunities.

% ============================================================================
\section{Discussion}
% ============================================================================

The neutron-lifetime calculation developed in this paper is part of a broader program: translating bulk hypotheses into right-side (detector-accessible) evidence, one observable at a time. Before concluding, we address a natural question that arises in any non-standard baryon-asymmetry narrative.

\subsection{Antimatter in colliders as a boundary excitation: a two-sided EDC reading}

A recurring objection to any non-standard baryon-asymmetry narrative is immediate:
if the universe is matter-dominated, why do high-energy experiments routinely produce antimatter?
EDC answers this by enforcing the same two-sided rule used throughout this paper:
the \emph{right side} (3D/4D) records detector-visible event topologies, while the \emph{left side} (5D) supplies candidate mechanisms.

\paragraph{Right side (LHC facts).}
In proton--proton collisions, the detector-level statement is operational and model-agnostic:
sufficiently energetic, localized interactions produce particle--antiparticle pairs,
followed by rapid annihilation/decay in the laboratory environment. This observation does not,
by itself, imply a primordial antimatter reservoir; it demonstrates that the accessible
high-energy channel space includes configurations that register as ``anti'' quantum numbers
in 3D observables.

\paragraph{Left side (EDC mechanism proposal) \textnormal{\small [P].}}
In EDC language, a collider event is a \emph{strong, localized boundary forcing} of the brane.
Such forcing is expected to excite short-lived, high-curvature boundary configurations and
junction reconfigurations. The proposed interpretation is that what the detector labels as
``antimatter'' corresponds to a \emph{localized counter-excitation} (a conjugate defect mode)
in the bulk--brane configuration space, created in pairs due to boundary consistency and
charge-conjugation of the effective 3D projection. In this view, collider antimatter is a
\emph{laboratory-produced boundary excitation}, not evidence for a globally symmetric
matter--antimatter initial condition.

\paragraph{Relation to CP violation and baryogenesis \textnormal{\small [Dc]/[P].}}
EDC does not require that Standard-Model CP-violating mechanisms be false; rather, it shifts
the logical emphasis. The global matter dominance is hypothesized to reflect a \emph{directionality}
(or boundary condition) of bulk-to-brane energy/flux organization, while CP-violation effects
remain as effective right-side phenomenology of the brane-projected dynamics.
At the present stage, this remains a constrained proposal: the paper does not derive the full
collider production cross-sections or CP-odd observables from a 5D action. Instead, the claim is
structural: laboratory antimatter can be consistently interpreted as a local, transient excitation
channel without committing to a primordial antimatter inventory.

\paragraph{What would count as a discriminating test \textnormal{\small [P].}}
A falsifiable next step is to translate the qualitative ``counter-excitation'' picture into a
quantitative mapping between bulk/brane parameters and right-side observables: e.g., scaling laws
for pair-production yields under varying boundary forcing, or a constrained relation between
effective defect/junction degrees of freedom and measured asymmetry patterns in specific channels.
This paper focuses on a simpler diagnostic observable---the free neutron lifetime---as a first
calibrated test of the reduction pipeline.

% ============================================================================
\section{Conclusions}
% ============================================================================

\paragraph{A two-sided result.}
Elastic Diffusive Cosmology (EDC) is written under a disciplined reading rule: what we measure on the brane is the \emph{right-side evidence}, while what we hypothesize in the bulk is the \emph{left-side cause}. The value of this separation is not rhetorical. It prevents a familiar failure mode of alternative frameworks---relabeling known parameters---by enforcing a directional logic:
\[
\text{bulk geometry and junction laws} \;\rightarrow\; \text{effective reduced dynamics} \;\rightarrow\; \text{right-side observables}.
\]
Within that logic, the present work does not ``translate'' Standard Model numerics into new notation. It constructs a concrete effective description in which a precision observable, the free neutron lifetime, arises from a specific geometric barrier-crossing process in a collective coordinate $q$.

\paragraph{Why the neutron matters in this framework.}
In conventional language the neutron is simply an unstable hadron. In the EDC reading it becomes more informative: a metastable configuration whose decay time functions as a macroscopic clock for microscopic reconfiguration. The neutron is therefore not merely ``something that decays''---it is a transition object that tests whether the proposed bulk--membrane mechanism admits a controlled intermediate state and a controlled escape channel. The \FrameworkNameShort\ picture implemented here makes that statement technically sharp: the decay rate is governed by semiclassical tunneling (WKB) through a barrier encoded by an explicitly specified pair $\{V(q),M(q)\}$. The gates reported in this work (positivity of $V$, non-negativity of $M$, singularity handling, and sensitivity diagnostics) are not cosmetic checks; they are the minimal consistency requirements for treating neutron decay as a calculable barrier-crossing event rather than a postulated interaction vertex.

\paragraph{A cosmological implication (stated carefully).}
A longstanding conceptual tension in early-universe storytelling is the apparent need to ``explain'' matter by balancing it against antimatter and then invoking additional mechanisms to remove the unwanted half. EDC motivates a different framing: the first question is not ``where did antimatter go?'' but ``which bulk--membrane configurations are dynamically stable, metastable, or forbidden?'' In such a framing, a baryonic universe is not the leftover from a symmetric cancellation, but a reachable sector of configuration space. In that sense, the neutron is naturally interpreted as a \emph{gateway configuration}: without an allowed metastable junction state that mediates transitions among stable bound states, the observed baryonic sector may not be dynamically reachable at all. This paper provides a concrete technical instance of that idea: the neutron lifetime becomes the timescale of a junction-slip event in the reduced dynamics. We stress that this is an interpretive implication of the model architecture, not a completed cosmological scenario; it should be read as a hypothesis to be refined and tested rather than as a claim of established early-universe history.

\paragraph{What has been achieved here.}
The main achievement of the present paper is to show that a neutron-lifetime calculation can be organized as an explicit, auditable pipeline:
\begin{itemize}
\item a well-defined effective one-dimensional coordinate $q$ representing a reconfiguration mode,
\item an effective action $S_{\mathrm{eff}}[q]=\int dt\,(\tfrac12 M(q)\dot q^2 - V(q))$ that encodes the barrier physics,
\item a semiclassical decay estimate controlled by a WKB exponent and a tagged prefactor,
\item and a reproducible numerical sensitivity layer with explicit ``gates'' that stress-test consistency.
\end{itemize}
This delivers more than a single number: it delivers a \emph{diagnostic map} connecting assumptions to outputs. That structure is what enables falsifiability and targeted improvement.

\paragraph{Epistemic close (tags and next steps).}
We explicitly maintain epistemic tags to keep the paper falsifiable and to prevent ``model rhetoric'' from blending into claims of established physics. In the current implementation, the heavy-lift mathematics is an \emph{effective 1D realization} with decisively constrained modeling choices: the 5D action (bulk+brane+GHY+Israel) is presently a \emph{formal scaffold} rather than an input used end-to-end in the numeric pipeline, and certain elements (e.g.\ the prefactor $A_0$ and profile choices) remain \Dc\ rather than fully \Dd. Any external comparison baseline is \BL\ and must remain segregated from derivation text. The next ``true 5D step'' is therefore unambiguous: derive $M(q)$ and $V(q)$ directly from the 5D action and junction conditions, replacing phenomenological gluing with a controlled reduction. With that upgrade, neutron lifetime would move from a consistency-oriented demonstration to a sharper discriminatory prediction of the underlying bulk geometry.

% ============================================================================
% END OF MAIN BODY
% ============================================================================
% NOTE: \appendix is called by the wrapper document (main_pathB.tex or main.tex),
% NOT here, to avoid duplicate appendix counter resets.
