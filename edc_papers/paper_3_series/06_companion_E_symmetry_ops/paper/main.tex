% ============================================================================
% Companion E: Symmetry Layering and Defect Operations in EDC
% Companion E to Paper 3 (NJSR Edition)
% Version: 1.0 | Date: 2026-01-18
% Build: XeLaTeX (Unicode)
% ============================================================================

\documentclass[11pt,a4paper]{article}

% ============================================================================
% PACKAGES
% ============================================================================
\usepackage{fontspec}
\usepackage{amsmath,amssymb,amsthm}
\usepackage{physics}
\usepackage{geometry}

% FONTS (TeX Gyre Termes = Times-like with full OpenType support)
\IfFontExistsTF{TeX Gyre Termes}{%
  \setmainfont{TeX Gyre Termes}
  \setsansfont{TeX Gyre Heros}
}{%
  \setmainfont{Times New Roman}[Ligatures=TeX]
  \setsansfont{Helvetica}
}
\usepackage[colorlinks=true,linkcolor=blue,citecolor=blue,urlcolor=blue]{hyperref}
\usepackage{booktabs}
\usepackage{enumitem}
\usepackage{xcolor}
\usepackage{tcolorbox}
\usepackage{array}
\usepackage{longtable}
\usepackage{natbib}
\usepackage{gensymb}

\geometry{margin=2.5cm}

% ============================================================================
% COLORS FOR EPISTEMIC TAGS
% ============================================================================
\definecolor{tagDer}{RGB}{0,100,0}      % Green - Derived
\definecolor{tagDc}{RGB}{0,80,160}      % Blue - Deduced/Constrained
\definecolor{tagCal}{RGB}{150,100,0}    % Orange - Calibrated
\definecolor{tagI}{RGB}{100,0,100}      % Purple - Identified
\definecolor{tagP}{RGB}{150,0,0}        % Red - Postulated
\definecolor{tagOpen}{RGB}{128,128,128} % Gray - Open
\definecolor{tagBL}{RGB}{80,80,80}      % Dark gray - Baseline

% ============================================================================
% EPISTEMIC TAG COMMANDS
% ============================================================================
\newcommand{\tagDer}{\textcolor{tagDer}{\textbf{[Der]}}}
\newcommand{\tagDc}{\textcolor{tagDc}{\textbf{[Dc]}}}
\newcommand{\tagCal}{\textcolor{tagCal}{\textbf{[Cal]}}}
\newcommand{\tagI}{\textcolor{tagI}{\textbf{[I]}}}
\newcommand{\tagP}{\textcolor{tagP}{\textbf{[P]}}}
\newcommand{\tagOpen}{\textcolor{tagOpen}{\textbf{[OPEN]}}}
\newcommand{\tagBL}{\textcolor{tagBL}{\textbf{[BL]}}}

% ============================================================================
% THEOREM ENVIRONMENTS
% ============================================================================
\newtheorem{theorem}{Theorem}[section]
\newtheorem{lemma}[theorem]{Lemma}
\newtheorem{proposition}[theorem]{Proposition}
\newtheorem{corollary}[theorem]{Corollary}
\newtheorem{definition}[theorem]{Definition}
\newtheorem{postulate}{Postulate}

\theoremstyle{remark}
\newtheorem{remark}{Remark}[section]

% ============================================================================
% CUSTOM COMMANDS
% ============================================================================
\newcommand{\Rxi}{R_\xi}
\newcommand{\Lxi}{L_\xi}
\newcommand{\re}{r_e}
\newcommand{\lP}{\ell_P}
\newcommand{\Mfive}{M_5}
\newcommand{\Mfour}{M_4}
\newcommand{\Sone}{S^1}
\newcommand{\Ztri}{\mathbb{Z}_3}
\newcommand{\Ztwo}{\mathbb{Z}_2}
\newcommand{\Zsix}{\mathbb{Z}_6}
\newcommand{\Diff}{\mathrm{Diff}}
\newcommand{\Isom}{\mathrm{Isom}}
\newcommand{\Uone}{U(1)}
\newcommand{\SUthree}{SU(3)}

% Process operators
\newcommand{\opE}{\mathcal{E}}
\newcommand{\opR}{\mathcal{R}}
\newcommand{\opM}{\mathcal{M}}

% Compton wavelength
\newcommand{\lambdabar}{\bar{\lambda}}

% ============================================================================
% TITLE
% ============================================================================
\title{\textbf{Symmetry Layering and Defect Operations}\\[0.3em]
\Large in Elastic Diffusive Cosmology\\[0.5em]
\normalsize (Companion E to Paper~3: NJSR Edition)}
\author{Igor Gr\v{c}man}
\date{January 2026\\[0.5em]
\small DOI: \href{https://doi.org/10.5281/zenodo.18300199}{10.5281/zenodo.18300199}\\[0.3em]
\small Repository: \href{https://github.com/igorgrcman/elastic-diffusive-cosmology}{github.com/igorgrcman/elastic-diffusive-cosmology}\\[0.2em]
\footnotesize (Public artifacts for this paper are in the \texttt{edc\_papers} folder.)}

% ============================================================================
% DOCUMENT
% ============================================================================
\begin{document}

\maketitle

\begin{center}
\small\textbf{Related Documents:}\\[0.1cm]
\footnotesize
\emph{Neutron Lifetime from 5D Membrane Cosmology} (DOI: \href{https://doi.org/10.5281/zenodo.18262721}{10.5281/zenodo.18262721})\\[0.05cm]
\emph{Framework v2.0} (DOI: \href{https://doi.org/10.5281/zenodo.18299085}{10.5281/zenodo.18299085})\\[0.05cm]
\textbf{Companions:}\\
A: \emph{Effective Lagrangian} (\href{https://doi.org/10.5281/zenodo.18292841}{DOI}) ~$\cdot$~
B: \emph{WKB Prefactor} (\href{https://doi.org/10.5281/zenodo.18299637}{DOI})\\
C: \emph{5D Reduction} (\href{https://doi.org/10.5281/zenodo.18299751}{DOI}) ~$\cdot$~
D: \emph{Selection Rules} (\href{https://doi.org/10.5281/zenodo.18299855}{DOI})\\
F: \emph{Proton Junction} (\href{https://doi.org/10.5281/zenodo.18302953}{DOI}) ~$\cdot$~
G: \emph{Mass Difference} (\href{https://doi.org/10.5281/zenodo.18303494}{DOI})\\
H: \emph{Weak Interactions} (\href{https://doi.org/10.5281/zenodo.18307539}{DOI})
\end{center}

% ============================================================================
% ABSTRACT
% ============================================================================
\begin{abstract}
\noindent
This companion paper formalizes the symmetry structure and topological operations within Elastic Diffusive Cosmology (EDC). We present a \emph{layered} description of symmetries acting on the 5D manifold $\Mfive = \Mfour \times \Sone_\xi$: kinematic diffeomorphism invariance on the 4D base, and internal $\Uone$ isometry of the compact dimension generating charge/winding conservation. We define three topological process operators---Excitation ($\opE$), Relaxation ($\opR$), and Merging ($\opM$)---that formalize generation transitions, weak decay, and nuclear binding respectively. A conservation ledger for $\beta^-$ decay demonstrates how topological constraints require a neutral channel, identified in 4D evidence language with the antineutrino.

\textbf{What this paper does:} Provides formal definitions, consistent epistemic tagging, and falsifiability conditions for symmetry and operator structures that the main paper~\cite{paper3} may cite.

\textbf{What this paper does NOT claim:} Full derivation from first principles, replacement of Standard Model phenomenology, or proof of operator dynamics from the 5D action.
\end{abstract}

\vspace{1em}

% ============================================================================
% TAGGING STANDARD BOX
% ============================================================================
\begin{tcolorbox}[colback=green!5,colframe=green!40!black,title=\textbf{Epistemic Tagging Standard}]
All claims carry explicit tags indicating derivation status:
\begin{center}
\renewcommand{\arraystretch}{1.2}
\begin{tabular}{@{}cl@{}}
\toprule
\textbf{Tag} & \textbf{Meaning} \\
\midrule
\tagDer{} & Derived: explicit mathematical derivation from stated postulates \\
\tagDc{} & Deduced/Constrained: follows from assumptions with explicit ansatz \\
\tagCal{} & Calibrated: parameter fitted to match experimental value \\
\tagI{} & Identified: pattern matching without full derivation \\
\tagP{} & Postulated: foundational assumption; not derived \\
\tagOpen{} & Open: known gap; future work needed \\
\tagBL{} & Baseline: external empirical input (CODATA, PDG, SM) \\
\bottomrule
\end{tabular}
\end{center}
\end{tcolorbox}

\tableofcontents
\newpage

% ============================================================================
% INCLUDE SECTIONS
% ============================================================================
% ============================================================================
\section{Introduction}
\label{sec:intro}
% ============================================================================

\subsection{Purpose of This Document}

This companion paper provides formal definitions for the symmetry structures and topological operations used in Elastic Diffusive Cosmology (EDC). It serves as a reference document that the main paper~\cite{paper3} (and subsequent EDC publications) can cite without repeating foundational material.

The content here does \emph{not} claim to derive new physics. Rather, it organizes and formalizes structures that arise naturally from the EDC framework, assigns consistent epistemic tags, and identifies falsifiability conditions.

\subsection{The Two-Sided Reading Rule}

Throughout this document, we maintain a strict distinction between two levels of description:

\begin{itemize}[leftmargin=*]
    \item \textbf{5D cause} (geometric mechanism): Statements about the topology, geometry, and dynamics of defects in the 5D manifold $\Mfive$.
    \item \textbf{4D evidence} (observable prediction): Statements about quantities measurable in our 4D spacetime---masses, charges, decay rates, cross-sections.
\end{itemize}

\noindent The 5D cause is the \emph{proposal}; the 4D evidence is the \emph{test}. Neither side alone constitutes a complete claim. For example:
\begin{quote}
\emph{5D cause}: ``The $\beta^-$ decay corresponds to a $\Zsix$ sector shift in the Y-junction configuration.''\\
\emph{4D evidence}: ``The neutron lifetime and decay products match SM predictions.''
\end{quote}

\subsection{Relationship to Other EDC Documents}

This paper assumes familiarity with:
\begin{enumerate}
    \item \textbf{EDC Framework Reference} (v2.0): Defines the 5D manifold, action, and defect taxonomy.
    \item \textbf{Paper 3}: Uses the operators and ledger defined here for neutron physics.
\end{enumerate}

\noindent We do not duplicate content from the Framework Reference; instead, we cite it and build upon its definitions.

\subsection{Document Structure}

\begin{enumerate}
    \item \textbf{Section~\ref{sec:geometric-setup}}: Minimal geometric setup (manifold, parameters)
    \item \textbf{Section~\ref{sec:symmetry-layering}}: Symmetry layering (kinematic + internal)
    \item \textbf{Section~\ref{sec:defect-classification}}: Defect classification as layered structure
    \item \textbf{Section~\ref{sec:process-operators}}: Topological process operators ($\opE$, $\opR$, $\opM$)
    \item \textbf{Section~\ref{sec:beta-ledger}}: $\beta^-$ decay conservation ledger
    \item \textbf{Section~\ref{sec:discussion}}: Discussion and research roadmap
\end{enumerate}

% ============================================================================
\section{Minimal Geometric Setup}
\label{sec:geometric-setup}
% ============================================================================

This section establishes the geometric foundation required for defining symmetries and operators. We state only what is needed; full details are in the EDC Framework Reference.

\subsection{The 5D Manifold}

\begin{postulate}[5D Product Structure \tagP{}]
\label{post:5d-manifold}
The EDC spacetime is a 5-dimensional manifold with product topology:
\begin{equation}
\label{eq:M5-product}
\Mfive = \Mfour \times \Sone_\xi
\end{equation}
where $\Mfour$ is a 4-dimensional Lorentzian manifold (our observable spacetime) and $\Sone_\xi$ is a compact circle of circumference $\Lxi = 2\pi \Rxi$.
\end{postulate}

The coordinate on $\Sone_\xi$ is denoted $\xi \in [0, \Lxi)$ with periodic identification. This structure follows the Kaluza-Klein paradigm~\cite{kaluza1921,klein1926}.

\subsection{The Compact Scale Parameter}

\begin{definition}[Compact Dimension Scale]
\label{def:Lxi}
The length scale $\Lxi$ (equivalently, radius $\Rxi = \Lxi / 2\pi$) characterizes the size of the compact dimension. This is a fundamental parameter of the theory.
\end{definition}

\begin{remark}[Historical Identification with Compton Wavelength \tagCal{}/\tagP{}]
\label{rem:Lxi-identification}
In early EDC literature, the compact scale was sometimes identified with the electron Compton wavelength:
\begin{equation}
\label{eq:Lxi-Compton}
\Lxi \sim \lambdabar_C^{(e)} = \frac{\hbar}{m_e c} \approx 3.86 \times 10^{-13} \text{ m}
\end{equation}
This identification is \textbf{not derived} from first principles. It is either:
\begin{itemize}
    \item \tagCal{}: Calibrated to match observed physics (e.g., requiring $\alpha$ formula to work), or
    \item \tagP{}: Postulated as a foundational assumption linking 5D geometry to particle scales.
\end{itemize}
The derivation of $\Lxi$ from 5D dynamics remains \tagOpen{}.
\end{remark}

\begin{remark}[Correction Note: Canonical Scale Separation]
\label{rem:scale-correction}
The EDC Framework Reference (v2.0) establishes a \emph{different} canonical scale hierarchy:
\begin{align}
\Rxi &\sim 10^{-18}\,\mathrm{m} && \text{(membrane/weak-KK scale) \tagDc{}} \\
\lambdabar_C^{(e)} &\sim 3.86 \times 10^{-13}\,\mathrm{m} && \text{(reduced Compton wavelength) \tagBL{}}
\end{align}
These are \textbf{distinct scales} separated by five orders of magnitude. The identification $\Lxi \sim \lambdabar_C^{(e)}$ in Eq.~\eqref{eq:Lxi-Compton} is a \emph{historical calibration} that conflates the kinematic Compton scale with the dynamical compactification radius. The formula $\alpha = r_e/\lambdabar_C^{(e)}$ is standard QED \tagBL{}; it should \textbf{not} be written as $\alpha = r_e/\Rxi$. This paper uses the earlier convention for continuity with prior work; see the Framework Reference for the corrected treatment.
\end{remark}

\subsection{Newton's Constant Reduction}

\begin{theorem}[4D Newton Constant \tagDc{}]
\label{thm:G4-reduction}
Standard Kaluza-Klein dimensional reduction gives:
\begin{equation}
\label{eq:G4-KK}
G_4 = \frac{G_5}{2\pi \Rxi} = \frac{G_5}{\Lxi}
\end{equation}
where $G_5$ is the 5D gravitational coupling.
\end{theorem}

\begin{remark}[Epistemic Status]
Equation~\eqref{eq:G4-KK} is standard KK theory, not unique to EDC. It is tagged \tagDc{} because:
\begin{itemize}
    \item The reduction formula is mathematically derived (given the product ansatz).
    \item The value of $G_5$ is \textbf{not} derived within EDC; it remains \tagOpen{}.
\end{itemize}
Thus, while the \emph{form} of Eq.~\eqref{eq:G4-KK} is derived, the \emph{numerical prediction} for $G_4$ requires knowing $G_5$.
\end{remark}

\subsection{Scope and Non-Scope}

\begin{tcolorbox}[colback=blue!5,colframe=blue!40!black,title=\textbf{Scope \& Non-Scope: Geometric Setup}]
\textbf{This section DOES:}
\begin{itemize}[nosep,leftmargin=*]
    \item Define the product manifold structure $\Mfive = \Mfour \times \Sone_\xi$
    \item Introduce the compact scale $\Lxi$ as a parameter
    \item State the KK reduction formula for $G_4$
    \item Clearly tag the $\Lxi \sim \lambdabar_C^{(e)}$ identification as \tagCal{}/\tagP{}
\end{itemize}

\textbf{This section does NOT:}
\begin{itemize}[nosep,leftmargin=*]
    \item Derive $\Lxi$ from 5D dynamics
    \item Derive $G_5$ from EDC first principles
    \item Claim the geometry is unique or fully constrained
\end{itemize}
\end{tcolorbox}

% ============================================================================
\section{Symmetry Layering}
\label{sec:symmetry-layering}
% ============================================================================

The symmetries of the EDC manifold naturally organize into \emph{layers}: kinematic invariances of the base spacetime, and internal isometries of the compact dimension. We describe these as a layered structure rather than claiming a single unified symmetry group.

\subsection{Kinematic Invariance: Diffeomorphisms of $\Mfour$}

\begin{definition}[4D Diffeomorphism Invariance \tagDc{}]
\label{def:diff-M4}
The theory is invariant under the diffeomorphism group of the 4D base manifold:
\begin{equation}
\label{eq:Diff-M4}
\Diff(\Mfour) = \{ \phi: \Mfour \to \Mfour \mid \phi \text{ is a smooth bijection with smooth inverse} \}
\end{equation}
This is the standard general covariance of general relativity, inherited by the 4D effective theory.
\end{definition}

\begin{remark}
This invariance is \tagDc{} (not \tagDer{}) because it is \emph{assumed} as part of the geometric framework, not derived from a more fundamental principle within EDC.
\end{remark}

\subsection{Internal Isometry: $\Uone$ of the Compact Dimension}

\begin{definition}[Compact Isometry Group \tagDc{}]
\label{def:isom-S1}
The isometry group of the circle $\Sone_\xi$ is:
\begin{equation}
\label{eq:Isom-S1}
\Isom(\Sone_\xi) \cong \Uone
\end{equation}
generated by rigid translations $\xi \mapsto \xi + \epsilon$ (continuous) and the reflection $\xi \mapsto -\xi$ (discrete). The continuous part $\Uone$ corresponds to shifts around the circle.
\end{definition}

\subsection{Winding Number and Charge}

\begin{theorem}[Winding-Charge Correspondence \tagDc{}]
\label{thm:winding-charge}
In Kaluza-Klein theory, the winding number $W$ of a field configuration around $\Sone_\xi$ corresponds to electric charge:
\begin{equation}
\label{eq:charge-winding}
Q = W = \frac{1}{2\pi} \oint_\gamma d\xi
\end{equation}
where $\gamma$ is a closed loop around the defect in the $\xi$ direction.
\end{theorem}

\begin{remark}[Epistemic Status]
The correspondence $Q = W$ is:
\begin{itemize}
    \item \tagDc{} within EDC: It follows from the KK ansatz and the identification of the $\Uone$ gauge field with electromagnetism.
    \item The \emph{normalization} (charge in units of $e$) requires matching to experiment, hence involves \tagCal{} elements.
\end{itemize}
\end{remark}

\subsection{Layered Structure (Not Direct Product)}

\begin{remark}[Caution on Group Structure \tagP{}]
\label{rem:layered-caution}
It is tempting to write a ``global symmetry group'' as:
\begin{equation}
\label{eq:naive-product}
\mathcal{G}_{\text{EDC}} \stackrel{?}{=} \Diff(\Mfour) \times \Uone_\xi
\end{equation}
However, this is \textbf{not rigorously established}. The actual symmetry structure is more subtle:
\begin{enumerate}
    \item $\Diff(\Mfour)$ acts on the base; $\Uone_\xi$ acts on the fiber.
    \item The product structure of $\Mfive$ induces a \emph{semi-direct} or \emph{fiber bundle} relationship, not a simple direct product.
    \item Matter fields (defects) transform under both, but the coupling is nontrivial.
\end{enumerate}

We therefore describe the symmetries as a \textbf{layered structure}:
\begin{center}
\renewcommand{\arraystretch}{1.3}
\begin{tabular}{@{}lll@{}}
\toprule
\textbf{Layer} & \textbf{Symmetry} & \textbf{Physical Role} \\
\midrule
Kinematic (base) & $\Diff(\Mfour)$ & General covariance, gravity \\
Internal (fiber) & $\Isom(\Sone_\xi) \cong \Uone$ & Charge conservation, EM \\
\bottomrule
\end{tabular}
\end{center}
\end{remark}

\subsection{Summary}

The symmetry content of the EDC manifold is organized as:

\begin{equation}
\label{eq:symmetry-layers}
\boxed{
\text{Symmetry Layers}: \quad
\underbrace{\Diff(\Mfour)}_{\text{kinematic}} \quad \oplus \quad \underbrace{\Isom(\Sone_\xi)}_{\text{internal}}
}
\end{equation}

The ``$\oplus$'' notation indicates layering, not algebraic direct sum. The precise mathematical structure (principal bundle, gauge group action) is left for future formalization \tagOpen{}.

% ============================================================================
\section{Defect Classification as Layered Structure}
\label{sec:defect-classification}
% ============================================================================

Particles in EDC are topological defects in the 5D brane. Different defect types carry different invariants. We organize these invariants into a layered classification scheme.

\subsection{Defect Invariants}

\begin{definition}[Defect State Vector]
\label{def:defect-state}
A defect configuration is characterized by a tuple of invariants:
\begin{equation}
\label{eq:defect-tuple}
\mathcal{D} = (W, Q, \mathcal{C}, s)
\end{equation}
where:
\begin{itemize}
    \item $W \in \mathbb{Z}$ or $\mathbb{Z}/3$: Total winding number
    \item $Q \in \mathbb{Z}$ (in units of $e/3$): Electric charge
    \item $\mathcal{C} \in \{-, r, g, b\}$: Color index (``$-$'' for colorless)
    \item $s \in \Zsix$: Sector label on the transverse ring
\end{itemize}
\end{definition}

\subsection{Y-Junction Mode Algebra}

The Y-junction (three-arm configuration) supports oscillation modes that generate an algebraic structure.

\begin{theorem}[Y-Junction Mode Algebra \tagDc{}]
\label{thm:Y-algebra}
The modes of a Y-junction configuration form an 8-dimensional space with structure:
\begin{equation}
\label{eq:AY-algebra}
\mathcal{A}_Y \sim \mathfrak{su}(3)
\end{equation}
consisting of:
\begin{itemize}
    \item 6 ``exchange modes'': oscillations that swap amplitude between pairs of arms
    \item 2 ``diagonal modes'': oscillations that preserve arm identity but modulate relative phases
\end{itemize}
\end{theorem}

\begin{remark}[Epistemic Status]
The identification $\mathcal{A}_Y \sim \mathfrak{su}(3)$ is \tagDc{}:
\begin{itemize}
    \item The mode counting (8 = 6 + 2) follows from junction geometry.
    \item The Lie algebra structure (commutators) requires explicit calculation from the 5D action.
    \item Full proof that $[\cdot, \cdot]$ closes on $\mathfrak{su}(3)$ is partially shown in the Framework Reference (Thm.~5.3--5.5) but relies on the Steiner angle assumption.
\end{itemize}
We write ``$\sim$'' rather than ``$=$'' to indicate structural similarity, not proven isomorphism.
\end{remark}

\subsection{Ring Sector Labels: $\Zsix$ Structure}

The transverse ring in junction configurations admits discrete symmetry.

\begin{theorem}[$\Zsix$ Sector Decomposition \tagDer{}]
\label{thm:Z6-sectors}
The symmetry of the transverse ring configuration space factors as:
\begin{equation}
\label{eq:Z6-factor}
\Zsix = \Ztri \times \Ztwo
\end{equation}
where:
\begin{itemize}
    \item $\Ztri$: Cyclic permutation of the three Y-junction arms
    \item $\Ztwo$: Matter-antimatter conjugation (reflection symmetry)
\end{itemize}
\end{theorem}

\begin{remark}[Sector Labels and Nucleons]
The six sectors $s \in \{0, 1, 2, 3, 4, 5\}$ correspond to stable configurations:
\begin{center}
\renewcommand{\arraystretch}{1.2}
\begin{tabular}{@{}cll@{}}
\toprule
\textbf{Sector $s$} & \textbf{Angle $\theta$} & \textbf{Interpretation} \\
\midrule
0 & $0\degree$ & Proton ground state \\
1 & $60\degree$ & Neutron ground state \\
2 & $120\degree$ & (Unstable / transition) \\
3 & $180\degree$ & Antiproton ground state \\
4 & $240\degree$ & Antineutron ground state \\
5 & $300\degree$ & (Unstable / transition) \\
\bottomrule
\end{tabular}
\end{center}
The proton occupies $s = 0$; the neutron occupies $s = 1$. A $\Zsix$ step ($s \to s+1$) corresponds to a topological transition.
\end{remark}

\subsection{Layered Structure (Not Subgroups)}

\begin{remark}[On Group Containment \tagP{}]
\label{rem:not-subgroups}
It is \textbf{not claimed} that $\SUthree$ and $\Zsix$ are subgroups of a single global symmetry group. Rather, they represent:
\begin{itemize}
    \item $\mathcal{A}_Y \sim \mathfrak{su}(3)$: \emph{Local} mode algebra at junctions (dynamical degrees of freedom)
    \item $\Zsix$: \emph{Global} sector labels (topological vacuum structure)
\end{itemize}

These structures are \textbf{coupled} (a $\Zsix$ transition involves mode excitation), but the precise relationship is:
\begin{equation}
\label{eq:coupling-schematic}
\text{(Sector shift in } \Zsix \text{)} \longleftrightarrow \text{(Mode excitation in } \mathcal{A}_Y \text{)}
\end{equation}
The mathematical formalization of this coupling remains \tagOpen{}.
\end{remark}

\subsection{Defect Classification Table}

\begin{table}[h]
\centering
\caption{Defect types and their invariants}
\label{tab:defect-invariants}
\renewcommand{\arraystretch}{1.3}
\begin{tabular}{@{}lccccl@{}}
\toprule
\textbf{Particle} & $W$ & $Q$ & $\mathcal{C}$ & $s$ & \textbf{Tag} \\
\midrule
Electron ($e^-$) & $-1$ & $-1$ & $-$ & --- & \tagI{} \\
Proton ($p$) & $+1$ & $+1$ & $-$ & $0$ & \tagI{} \\
Neutron ($n$) & $+1$ & $0$ & $-$ & $1$ & \tagI{} \\
Up quark ($u$) & $+2/3$ & $+2/3$ & $r,g,b$ & --- & \tagDer{} \\
Down quark ($d$) & $-1/3$ & $-1/3$ & $r,g,b$ & --- & \tagDer{} \\
\bottomrule
\end{tabular}
\end{table}

\begin{remark}
The electron, proton, and neutron identifications are \tagI{} (pattern matching between EDC defect types and SM particles). The quark winding numbers are \tagDer{} from charge constraints (see Framework Reference, Thm.~4.4).
\end{remark}

% ============================================================================
\section{Topological Process Operators}
\label{sec:process-operators}
% ============================================================================

We define three formal operators representing topological processes that change defect configurations. These operators act on the invariant tuple $(W, Q, \mathcal{C}, s)$ and encode fundamental physical processes.

\subsection{Operator Definitions}

\begin{definition}[Excitation Operator $\opE$ \tagDc{}/\tagP{}]
\label{def:op-E}
The \textbf{excitation operator} $\opE_n$ raises a defect to a higher mode along $\Sone_\xi$:
\begin{equation}
\label{eq:op-E}
\opE_n: \mathcal{D}(n_0) \longrightarrow \mathcal{D}(n_0 + n)
\end{equation}
where $n_0$ is the initial mode number and $n \geq 1$ is the excitation level.

\textbf{Invariants:}
\begin{itemize}
    \item \emph{Preserved}: $W$, $Q$, $\mathcal{C}$, $s$
    \item \emph{Changed}: Internal mode number $n$; mass increases
\end{itemize}

\textbf{Physical interpretation}: Generation transitions (e.g., $e \to \mu \to \tau$).
\end{definition}

\begin{definition}[Relaxation Operator $\opR$ \tagDc{}/\tagP{}]
\label{def:op-R}
The \textbf{relaxation operator} $\opR$ shifts the sector label by one unit in $\Zsix$:
\begin{equation}
\label{eq:op-R}
\opR: (W, Q, \mathcal{C}, s) \longrightarrow (W', Q', \mathcal{C}', s+1 \mod 6)
\end{equation}
with compensating changes in $W$, $Q$, $\mathcal{C}$ to satisfy conservation laws.

\textbf{Invariants:}
\begin{itemize}
    \item \emph{Preserved}: Total winding (with emitted particles), total charge
    \item \emph{Changed}: Sector $s$; quark content; particle identity
\end{itemize}

\textbf{Physical interpretation}: Weak decay processes (e.g., $\beta^-$: neutron $\to$ proton).
\end{definition}

\begin{definition}[Merging Operator $\opM$ \tagP{}]
\label{def:op-M}
The \textbf{merging operator} $\opM$ combines two defects into a bound configuration:
\begin{equation}
\label{eq:op-M}
\opM: \mathcal{D}_1 \otimes \mathcal{D}_2 \longrightarrow \mathcal{D}_{\text{bound}}
\end{equation}

\textbf{Invariants:}
\begin{itemize}
    \item \emph{Preserved}: Total $W$, total $Q$
    \item \emph{Changed}: Configuration space geometry; binding energy released
\end{itemize}

\textbf{Physical interpretation}: Nuclear binding (``merged Inflow'').
\end{definition}

\subsection{Operator Properties}

\begin{table}[h]
\centering
\caption{Summary of process operators}
\label{tab:operators}
\renewcommand{\arraystretch}{1.4}
\begin{tabular}{@{}lp{3cm}p{3cm}p{2.5cm}l@{}}
\toprule
\textbf{Operator} & \textbf{Action} & \textbf{Preserved} & \textbf{Changed} & \textbf{Tag} \\
\midrule
$\opE_n$ & Raise mode by $n$ & $W, Q, \mathcal{C}, s$ & Mode $n$, mass & \tagDc{}/\tagP{} \\
$\opR$ & Shift sector $s \to s+1$ & Total $W$, total $Q$ & $s$, quark IDs & \tagDc{}/\tagP{} \\
$\opM$ & Merge two defects & Total $W$, total $Q$ & Geometry, $E_{\text{bind}}$ & \tagP{} \\
\bottomrule
\end{tabular}
\end{table}

\subsection{Formal Requirements}

For each operator, we state the mathematical requirements that a full derivation must satisfy:

\begin{enumerate}
    \item \textbf{$\opE$ (Excitation)}:
    \begin{itemize}
        \item Must follow from the spectrum of the $\Sone_\xi$ Laplacian acting on defect wavefunctions.
        \item The mass increase formula $m_n = m_0 \cdot f(n, \alpha)$ must be derived from the 5D action.
        \item Current status: The lepton mass formulas ($m_\mu/m_e$, $m_\tau/m_\mu$) are \tagI{}, not \tagDer{}.
    \end{itemize}

    \item \textbf{$\opR$ (Relaxation)}:
    \begin{itemize}
        \item Must follow from the $\Zsix$ potential landscape $V(\theta)$ and tunneling/transition dynamics.
        \item The rate formula must connect to observed weak decay rates.
        \item Current status: The mechanism is \tagP{}; the $\Zsix$ potential form is \tagDc{}.
    \end{itemize}

    \item \textbf{$\opM$ (Merging)}:
    \begin{itemize}
        \item Must follow from the ``merged Inflow'' geometry where defect configuration spaces overlap.
        \item The binding energy formula must be derived from surface area reduction.
        \item Current status: Entirely \tagP{}; no quantitative formula exists.
    \end{itemize}
\end{enumerate}

\subsection{Falsifiability Hooks}

Each operator interpretation makes implicit predictions that could falsify it:

\begin{tcolorbox}[colback=red!5,colframe=red!40!black,title=\textbf{Falsifiability Conditions}]

\textbf{$\opE$ (Excitation)}:
\begin{itemize}[nosep]
    \item If a fourth generation lepton is discovered, the ``SU(3) saturation'' argument fails.
    \item If $m_\tau/m_\mu \neq 16\pi/3$ to better than 1\%, the geometric interpretation is falsified.
\end{itemize}

\textbf{$\opR$ (Relaxation)}:
\begin{itemize}[nosep]
    \item If $\beta^-$ decay products violate the $\Zsix$ step pattern (e.g., direct $n \to \bar{p}$), the operator is falsified.
    \item If neutron lifetime deviates from the topological barrier prediction (once derived), the mechanism fails.
\end{itemize}

\textbf{$\opM$ (Merging)}:
\begin{itemize}[nosep]
    \item If nuclear binding energies show no correlation with configuration space surface area reduction, the mechanism is falsified.
    \item If light nuclei binding energies cannot be fit with a universal ``Inflow overlap'' parameter, the model fails.
\end{itemize}

\end{tcolorbox}

\subsection{Composition Rules}

\begin{remark}[Operator Algebra \tagOpen{}]
The composition of operators (e.g., $\opR \circ \opE$, $\opM \circ \opR$) is not yet formalized. Questions include:
\begin{itemize}
    \item Do operators commute? (Likely not: $\opR \circ \opE \neq \opE \circ \opR$)
    \item Is there a group structure? (Unlikely for $\opM$ due to irreversibility of binding)
    \item How do selection rules emerge? (From symmetry constraints on compositions)
\end{itemize}
This remains \tagOpen{} for future work.
\end{remark}

% ============================================================================
\section{$\beta^-$ Decay Conservation Ledger}
\label{sec:beta-ledger}
% ============================================================================

This section presents the $\beta^-$ decay process ($n \to p + e^- + \bar{\nu}_e$) in EDC language, using a conservation ledger to track topological invariants.

\subsection{The Ledger Formalism}

\begin{definition}[Conservation Ledger]
\label{def:ledger}
A \textbf{conservation ledger} is a bookkeeping table that tracks the values of conserved quantities before and after a topological transition. For each quantity $X$, we require:
\begin{equation}
\label{eq:ledger-balance}
\sum_{\text{initial}} X_i = \sum_{\text{final}} X_f
\end{equation}
\end{definition}

\subsection{$\beta^-$ Decay in EDC Language}

In EDC, the neutron and proton are Y-junction defects at different $\Zsix$ sector positions. The $\beta^-$ decay corresponds to the relaxation operator $\opR$:
\begin{equation}
\label{eq:beta-EDC}
\opR: n(s=1) \longrightarrow p(s=0) + e^- + \text{(neutral channel)}
\end{equation}

\subsection{Conservation Ledger Table}

\begin{table}[h]
\centering
\caption{$\beta^-$ decay conservation ledger}
\label{tab:beta-ledger}
\renewcommand{\arraystretch}{1.4}
\begin{tabular}{@{}l|c|ccc|c@{}}
\toprule
& \textbf{Before} & \multicolumn{3}{c|}{\textbf{After}} & \textbf{Balance} \\
\textbf{Quantity} & Neutron & Proton & Electron & Neutral & Check \\
\midrule
Winding $W$ & $+1$ & $+1$ & $-1$ & $0$ & $+1 = +1 - 1 + 0$ {\color{tagDer}\checkmark} \\
Charge $Q$ & $0$ & $+1$ & $-1$ & $0$ & $0 = +1 - 1 + 0$ {\color{tagDer}\checkmark} \\
Sector $s$ & $1$ & $0$ & --- & --- & $\Delta s = -1$ \\
Baryon \# $B$ & $+1$ & $+1$ & $0$ & $0$ & $+1 = +1$ {\color{tagDer}\checkmark} \\
Lepton \# $L$ & $0$ & $0$ & $+1$ & $-1$ & $0 = 0 + 1 - 1$ {\color{tagDer}\checkmark} \\
\bottomrule
\end{tabular}
\end{table}

\subsection{The Neutral Channel Requirement}

\begin{theorem}[Neutral Channel Constraint \tagDc{}]
\label{thm:neutral-channel}
For the conservation ledger to balance, the $\beta^-$ decay \textbf{must} produce a neutral channel with:
\begin{equation}
\label{eq:neutral-requirements}
W_{\text{neutral}} = 0, \quad Q_{\text{neutral}} = 0, \quad L_{\text{neutral}} = -1
\end{equation}
This is a topological constraint, independent of dynamics.
\end{theorem}

\begin{proof}
From the ledger (Table~\ref{tab:beta-ledger}):
\begin{align}
W_{\text{neutral}} &= W_n - W_p - W_e = 1 - 1 - (-1) - W_{\text{neutral}} \Rightarrow W_{\text{neutral}} = 0 \\
Q_{\text{neutral}} &= Q_n - Q_p - Q_e = 0 - 1 - (-1) = 0 \\
L_{\text{neutral}} &= L_n - L_p - L_e = 0 - 0 - 1 = -1
\end{align}
These are necessary conditions for ledger closure.
\end{proof}

\begin{remark}[Epistemic Status]
The \emph{requirement} for a neutral channel is \tagDc{}: it follows logically from conservation laws applied to the ledger. No dynamics or specific mechanism is invoked---only bookkeeping.
\end{remark}

\subsection{Identification with Antineutrino}

\begin{postulate}[$\xi$-Wave Identification \tagP{}]
\label{post:xi-wave}
The neutral channel required by the ledger is realized in EDC as a \textbf{$\xi$-wave}: a propagating excitation in the compact dimension $\Sone_\xi$, rather than a frozen topological defect.

In 4D evidence language, this $\xi$-wave is \textbf{identified} with the electron antineutrino $\bar{\nu}_e$.
\end{postulate}

\begin{remark}[Two-Sided Reading]
\begin{itemize}
    \item \textbf{5D cause}: The neutral channel is a $\xi$-wave (propagating mode, not localized defect).
    \item \textbf{4D evidence}: The neutral channel is identified with $\bar{\nu}_e$ (weak interaction phenomenology).
\end{itemize}
The identification $\xi\text{-wave} \leftrightarrow \bar{\nu}_e$ is \tagP{}, not derived from the 5D action.
\end{remark}

\subsection{Why $\xi$-Wave?}

The $\xi$-wave hypothesis (Postulate~\ref{post:xi-wave}) is motivated by:

\begin{enumerate}
    \item \textbf{Neutrality}: A wave in $\Sone_\xi$ carries no net winding ($W = 0$) and no charge ($Q = 0$).

    \item \textbf{Near-masslessness}: Unlike frozen defects (which have mass from configuration space volume), waves have energy $E = \hbar \omega$ with no rest mass contribution from topology.

    \item \textbf{Weak interaction}: Waves do not create ``holes'' in the brane like frozen defects do, explaining small interaction cross-sections.

    \item \textbf{Lepton number}: The $L = -1$ assignment follows from the \emph{direction} of propagation or phase winding of the wave.
\end{enumerate}

These are \textbf{plausibility arguments}, not derivations. The full dynamics of $\xi$-waves remains \tagOpen{}.

\subsection{Comparison with Standard Model}

\begin{tcolorbox}[colback=gray!5,colframe=gray!40!black,title=\textbf{Baseline Comparison \tagBL{}}]
In the Standard Model~\cite{pdg2024}, $\beta^-$ decay is mediated by $W^-$ boson exchange:
\[
n \to p + W^- \to p + e^- + \bar{\nu}_e
\]
The conservation laws (charge, baryon number, lepton number) are identical to the EDC ledger. The difference is:
\begin{itemize}
    \item \textbf{SM}: Decay mediated by gauge boson; neutrino is a fundamental fermion.
    \item \textbf{EDC}: Decay is topological relaxation; neutrino is a $\xi$-wave excitation.
\end{itemize}
Both frameworks predict the same final state and conservation laws. The distinction lies in the underlying mechanism.
\end{tcolorbox}

\subsection{Ledger Summary}

The $\beta^-$ decay ledger demonstrates:

\begin{enumerate}
    \item \tagDc{} The \emph{need} for a neutral channel follows from topological conservation laws.
    \item \tagP{} The \emph{identification} of this channel with a $\xi$-wave (and hence $\bar{\nu}_e$) is a hypothesis.
    \item \tagOpen{} The \emph{dynamics} (decay rate, energy spectrum) require deriving $\xi$-wave properties from the 5D action.
\end{enumerate}

% ============================================================================
\section{Discussion and Roadmap}
\label{sec:discussion}
% ============================================================================

\subsection{What This Paper Has Defined}

This companion paper has formalized:

\begin{enumerate}
    \item \textbf{Symmetry Layering} (Section~\ref{sec:symmetry-layering}): The EDC manifold has two symmetry layers---kinematic $\Diff(\Mfour)$ and internal $\Isom(\Sone_\xi) \cong \Uone$---described as a layered structure rather than a simple product group.

    \item \textbf{Defect Classification} (Section~\ref{sec:defect-classification}): Defects are characterized by invariants $(W, Q, \mathcal{C}, s)$. The Y-junction mode algebra ($\sim \mathfrak{su}(3)$) and ring sector labels ($\Zsix$) are related but not claimed to be subgroups of a global symmetry.

    \item \textbf{Process Operators} (Section~\ref{sec:process-operators}): Three operators---Excitation ($\opE$), Relaxation ($\opR$), Merging ($\opM$)---formalize generation transitions, weak decay, and nuclear binding respectively.

    \item \textbf{Conservation Ledger} (Section~\ref{sec:beta-ledger}): The $\beta^-$ decay is analyzed via a ledger that requires a neutral channel \tagDc{}, identified with a $\xi$-wave (antineutrino) \tagP{}.
\end{enumerate}

\subsection{Research Roadmap: Upgrading Tags}

To upgrade epistemic tags from \tagP{}/\tagDc{} to \tagDer{}, the following derivations are needed:

\begin{table}[h]
\centering
\caption{Research roadmap for tag upgrades}
\label{tab:roadmap}
\renewcommand{\arraystretch}{1.4}
\begin{tabular}{@{}p{4cm}lp{5cm}@{}}
\toprule
\textbf{Claim} & \textbf{Current} & \textbf{Required for \tagDer{}} \\
\midrule
$\Lxi \sim \lambdabar_C^{(e)}$ & \tagCal{}/\tagP{} & Derive $\Lxi$ from 5D variational principle \\
$\mathcal{A}_Y \sim \mathfrak{su}(3)$ & \tagDc{} & Prove Lie algebra closure from 5D action \\
$\Zsix$ sector structure & \tagDer{} & (Already derived from product decomposition) \\
Operator $\opE$ dynamics & \tagP{} & Derive mass spectrum from $\Sone_\xi$ Laplacian \\
Operator $\opR$ dynamics & \tagP{} & Derive tunneling rate from $V(\theta)$ potential \\
Operator $\opM$ dynamics & \tagP{} & Derive binding energy from Inflow geometry \\
$\xi$-wave $\leftrightarrow \bar{\nu}$ & \tagP{} & Derive $\xi$-wave properties; match to $\bar{\nu}$ phenomenology \\
\bottomrule
\end{tabular}
\end{table}

\subsection{Open Questions}

The following questions remain \tagOpen{}:

\begin{enumerate}
    \item \textbf{Operator algebra}: Do $\opE$, $\opR$, $\opM$ form a closed algebraic structure? What are the selection rules?

    \item \textbf{$G_5$ derivation}: Can the 5D gravitational coupling be derived from EDC geometry, or must it be an input?

    \item \textbf{$\xi$-wave dynamics}: What is the dispersion relation for $\xi$-waves? How do they interact with frozen defects?

    \item \textbf{Neutrino oscillations}: If neutrinos are $\xi$-waves, what mechanism produces mass differences and mixing?

    \item \textbf{Nuclear binding quantitative}: Can the isoperimetric inequality (surface area minimization) yield helium binding energy to percent-level accuracy?
\end{enumerate}

\subsection{How Paper 3 Will Use This Material}

Paper 3 (on neutron physics) will cite this companion paper for:

\begin{itemize}
    \item The formal definition of the $\Zsix$ sector structure
    \item The relaxation operator $\opR$ as the mechanism for $\beta^-$ decay
    \item The conservation ledger formalism for analyzing decay processes
    \item The $\xi$-wave hypothesis for the neutral channel
\end{itemize}

By citing this companion, Paper 3 can focus on neutron-specific physics without repeating foundational definitions.

% ============================================================================
\section*{Conclusion}
\label{sec:conclusion}
\addcontentsline{toc}{section}{Conclusion}
% ============================================================================

This paper has provided formal definitions for symmetry layering, defect classification, and topological process operators within Elastic Diffusive Cosmology. Every claim is tagged with its epistemic status---\tagDer{}, \tagDc{}, \tagCal{}, \tagI{}, \tagP{}, or \tagOpen{}---to maintain intellectual honesty about what is derived versus hypothesized.

The key contributions are:
\begin{enumerate}
    \item A layered (not product-group) description of EDC symmetries
    \item Formal definitions of operators $\opE$, $\opR$, $\opM$ with falsifiability conditions
    \item A conservation ledger for $\beta^-$ decay showing the topological necessity of a neutral channel
    \item A clear research roadmap for upgrading postulates to derivations
\end{enumerate}

This framework provides a foundation for Paper 3 and subsequent EDC publications to build upon without duplicating foundational material.


% ============================================================================
% APPENDICES
% ============================================================================
\appendix
% ============================================================================
\section{Notation and Conventions}
\label{app:notation}
% ============================================================================

\subsection{Manifolds and Coordinates}

\begin{center}
\renewcommand{\arraystretch}{1.3}
\begin{tabular}{@{}lp{8cm}@{}}
\toprule
\textbf{Symbol} & \textbf{Meaning} \\
\midrule
$\Mfive$ & 5-dimensional EDC manifold \\
$\Mfour$ & 4-dimensional base spacetime (Lorentzian) \\
$\Sone_\xi$ & Compact circle (fifth dimension) \\
$\xi$ & Coordinate on $\Sone_\xi$, $\xi \in [0, \Lxi)$ \\
$\Lxi$ & Circumference of compact dimension \\
$\Rxi$ & Radius of compact dimension, $\Rxi = \Lxi / 2\pi$ \\
$x^\mu$ & Coordinates on $\Mfour$, $\mu \in \{0,1,2,3\}$ \\
$x^A$ & Coordinates on $\Mfive$, $A \in \{0,1,2,3,5\}$ \\
\bottomrule
\end{tabular}
\end{center}

\subsection{Groups and Algebras}

\begin{center}
\renewcommand{\arraystretch}{1.3}
\begin{tabular}{@{}lp{8cm}@{}}
\toprule
\textbf{Symbol} & \textbf{Meaning} \\
\midrule
$\Diff(\Mfour)$ & Diffeomorphism group of $\Mfour$ \\
$\Isom(\Sone_\xi)$ & Isometry group of $\Sone_\xi$, $\cong \Uone$ \\
$\Uone$ & Unitary group $U(1)$ \\
$\SUthree$ & Special unitary group $SU(3)$ \\
$\mathfrak{su}(3)$ & Lie algebra of $\SUthree$ \\
$\Zsix$ & Cyclic group of order 6 \\
$\Ztri$ & Cyclic group of order 3 \\
$\Ztwo$ & Cyclic group of order 2 \\
$\mathcal{A}_Y$ & Mode algebra of Y-junction, $\sim \mathfrak{su}(3)$ \\
\bottomrule
\end{tabular}
\end{center}

\subsection{Defect Invariants}

\begin{center}
\renewcommand{\arraystretch}{1.3}
\begin{tabular}{@{}lp{8cm}@{}}
\toprule
\textbf{Symbol} & \textbf{Meaning} \\
\midrule
$W$ & Winding number (integer or $\mathbb{Z}/3$ for quarks) \\
$Q$ & Electric charge in units of $e$ (or $e/3$ for quarks) \\
$\mathcal{C}$ & Color index: $\{-, r, g, b\}$ \\
$s$ & Sector label in $\Zsix$, $s \in \{0,1,2,3,4,5\}$ \\
$\mathcal{D}$ & Defect state tuple $(W, Q, \mathcal{C}, s)$ \\
$B$ & Baryon number \\
$L$ & Lepton number \\
\bottomrule
\end{tabular}
\end{center}

\subsection{Process Operators}

\begin{center}
\renewcommand{\arraystretch}{1.3}
\begin{tabular}{@{}lp{8cm}@{}}
\toprule
\textbf{Symbol} & \textbf{Meaning} \\
\midrule
$\opE$ & Excitation operator (generation transition) \\
$\opE_n$ & Excitation by $n$ levels \\
$\opR$ & Relaxation operator (sector shift, weak decay) \\
$\opM$ & Merging operator (nuclear binding) \\
\bottomrule
\end{tabular}
\end{center}

\subsection{Physical Constants}

\begin{center}
\renewcommand{\arraystretch}{1.3}
\begin{tabular}{@{}llp{5cm}@{}}
\toprule
\textbf{Symbol} & \textbf{Value} & \textbf{Meaning} \\
\midrule
$\lambdabar_C^{(e)}$ & $3.86 \times 10^{-13}$ m & Reduced Compton wavelength of electron \\
$G_4$ & $6.674 \times 10^{-11}$ m$^3$kg$^{-1}$s$^{-2}$ & 4D Newton constant \\
$G_5$ & (not determined) & 5D gravitational coupling \\
$\alpha$ & $\approx 1/137$ & Fine structure constant \\
$m_e$ & $0.511$ MeV & Electron mass \\
\bottomrule
\end{tabular}
\end{center}

\subsection{Epistemic Tags}

\begin{center}
\renewcommand{\arraystretch}{1.3}
\begin{tabular}{@{}ll@{}}
\toprule
\textbf{Tag} & \textbf{Meaning} \\
\midrule
\tagDer{} & Derived from stated postulates \\
\tagDc{} & Deduced/Constrained (follows with ansatz) \\
\tagCal{} & Calibrated to experiment \\
\tagI{} & Identified (pattern match) \\
\tagP{} & Postulated (foundational assumption) \\
\tagOpen{} & Open problem \\
\tagBL{} & Baseline (external data) \\
\bottomrule
\end{tabular}
\end{center}

% ============================================================================
\section{Claim Registry}
\label{app:claim-registry}
% ============================================================================

This appendix provides a complete registry of all nontrivial claims in this document.

\subsection{Postulated Claims \tagP{}}

\begin{center}
\renewcommand{\arraystretch}{1.2}
\begin{longtable}{@{}p{4cm}p{5cm}p{3cm}@{}}
\toprule
\textbf{Claim} & \textbf{Statement} & \textbf{Reference} \\
\midrule
\endhead
5D product manifold & $\Mfive = \Mfour \times \Sone_\xi$ & Post.~\ref{post:5d-manifold} \\
$\Lxi$ identification & $\Lxi \sim \lambdabar_C^{(e)}$ & Rem.~\ref{rem:Lxi-identification} \\
Symmetry layering & Not proven to be direct product & Rem.~\ref{rem:layered-caution} \\
Defect-particle ID & Electron, proton, neutron = specific defects & Table~\ref{tab:defect-invariants} \\
$\xi$-wave hypothesis & Neutral channel = $\xi$-wave & Post.~\ref{post:xi-wave} \\
$\xi$-wave = $\bar{\nu}_e$ & Identification in 4D language & Post.~\ref{post:xi-wave} \\
Operator $\opE$ & Excitation mechanism & Def.~\ref{def:op-E} \\
Operator $\opR$ & Relaxation mechanism & Def.~\ref{def:op-R} \\
Operator $\opM$ & Merging mechanism & Def.~\ref{def:op-M} \\
\bottomrule
\end{longtable}
\end{center}

\subsection{Deduced/Constrained Claims \tagDc{}}

\begin{center}
\renewcommand{\arraystretch}{1.2}
\begin{longtable}{@{}p{4cm}p{5cm}p{3cm}@{}}
\toprule
\textbf{Claim} & \textbf{Statement} & \textbf{Reference} \\
\midrule
\endhead
$G_4$ reduction & $G_4 = G_5 / \Lxi$ (KK standard) & Thm.~\ref{thm:G4-reduction} \\
$\Diff(\Mfour)$ invariance & Kinematic covariance assumed & Def.~\ref{def:diff-M4} \\
$\Isom(\Sone_\xi) \cong \Uone$ & Compact isometry group & Def.~\ref{def:isom-S1} \\
Winding-charge $Q = W$ & KK mechanism & Thm.~\ref{thm:winding-charge} \\
$\mathcal{A}_Y \sim \mathfrak{su}(3)$ & Mode algebra similarity & Thm.~\ref{thm:Y-algebra} \\
Neutral channel needed & Ledger bookkeeping constraint & Thm.~\ref{thm:neutral-channel} \\
\bottomrule
\end{longtable}
\end{center}

\subsection{Derived Claims \tagDer{}}

\begin{center}
\renewcommand{\arraystretch}{1.2}
\begin{longtable}{@{}p{4cm}p{5cm}p{3cm}@{}}
\toprule
\textbf{Claim} & \textbf{Statement} & \textbf{Reference} \\
\midrule
\endhead
$\Zsix$ factorization & $\Zsix = \Ztri \times \Ztwo$ & Thm.~\ref{thm:Z6-sectors} \\
Quark windings & $W_u = +2/3$, $W_d = -1/3$ & Table~\ref{tab:defect-invariants} \\
\bottomrule
\end{longtable}
\end{center}

\subsection{Identified Claims \tagI{}}

\begin{center}
\renewcommand{\arraystretch}{1.2}
\begin{longtable}{@{}p{4cm}p{5cm}p{3cm}@{}}
\toprule
\textbf{Claim} & \textbf{Statement} & \textbf{Reference} \\
\midrule
\endhead
Electron = simple vortex & $W = -1$, colorless & Table~\ref{tab:defect-invariants} \\
Proton = Y-junction & $s = 0$, $W_{\text{tot}} = +1$ & Table~\ref{tab:defect-invariants} \\
Neutron = Y-junction & $s = 1$, $W_{\text{tot}} = +1$ & Table~\ref{tab:defect-invariants} \\
\bottomrule
\end{longtable}
\end{center}

\subsection{Calibrated Claims \tagCal{}}

\begin{center}
\renewcommand{\arraystretch}{1.2}
\begin{tabular}{@{}p{4cm}p{5cm}p{3cm}@{}}
\toprule
\textbf{Claim} & \textbf{Statement} & \textbf{Reference} \\
\midrule
$\Lxi$ value & $\Lxi \approx \lambdabar_C^{(e)}$ (if fitted) & Rem.~\ref{rem:Lxi-identification} \\
\bottomrule
\end{tabular}
\end{center}

\subsection{Open Problems \tagOpen{}}

\begin{enumerate}
    \item Derive $\Lxi$ from 5D variational principle
    \item Derive $G_5$ from EDC geometry
    \item Prove $\mathcal{A}_Y = \mathfrak{su}(3)$ rigorously (Lie bracket closure)
    \item Derive operator $\opE$ mass spectrum from Laplacian eigenvalues
    \item Derive operator $\opR$ transition rate from $V(\theta)$ tunneling
    \item Derive operator $\opM$ binding energy from surface area reduction
    \item Derive $\xi$-wave dispersion relation and interaction cross-sections
    \item Formalize operator composition rules and selection rules
\end{enumerate}

\subsection{Dependency Map}

The following diagram shows claim dependencies:

\begin{verbatim}
[P] 5D Manifold (Post. 2.1)
    |
    +---> [Dc] Diff(M4) invariance
    |
    +---> [Dc] Isom(S1) ~ U(1)
    |         |
    |         +---> [Dc] Q = W (winding-charge)
    |
    +---> [Der] Z6 = Z3 x Z2
    |         |
    |         +---> [Dc] Sector labels s in {0,...,5}
    |
    +---> [Dc] A_Y ~ su(3)
              |
              +---> [I] Defect-particle identification

[P] Operators E, R, M
    |
    +---> [Dc] Neutral channel required (ledger)
              |
              +---> [P] xi-wave = antineutrino
\end{verbatim}


% ============================================================================
% BIBLIOGRAPHY
% ============================================================================
\bibliographystyle{plainnat}
\bibliography{../bib/references}

\end{document}
