% ============================================================================
\section{Symmetry Layering}
\label{sec:symmetry-layering}
% ============================================================================

The symmetries of the EDC manifold naturally organize into \emph{layers}: kinematic invariances of the base spacetime, and internal isometries of the compact dimension. We describe these as a layered structure rather than claiming a single unified symmetry group.

\subsection{Kinematic Invariance: Diffeomorphisms of $\Mfour$}

\begin{definition}[4D Diffeomorphism Invariance \tagDc{}]
\label{def:diff-M4}
The theory is invariant under the diffeomorphism group of the 4D base manifold:
\begin{equation}
\label{eq:Diff-M4}
\Diff(\Mfour) = \{ \phi: \Mfour \to \Mfour \mid \phi \text{ is a smooth bijection with smooth inverse} \}
\end{equation}
This is the standard general covariance of general relativity, inherited by the 4D effective theory.
\end{definition}

\begin{remark}
This invariance is \tagDc{} (not \tagDer{}) because it is \emph{assumed} as part of the geometric framework, not derived from a more fundamental principle within EDC.
\end{remark}

\subsection{Internal Isometry: $\Uone$ of the Compact Dimension}

\begin{definition}[Compact Isometry Group \tagDc{}]
\label{def:isom-S1}
The isometry group of the circle $\Sone_\xi$ is:
\begin{equation}
\label{eq:Isom-S1}
\Isom(\Sone_\xi) \cong \Uone
\end{equation}
generated by rigid translations $\xi \mapsto \xi + \epsilon$ (continuous) and the reflection $\xi \mapsto -\xi$ (discrete). The continuous part $\Uone$ corresponds to shifts around the circle.
\end{definition}

\subsection{Winding Number and Charge}

\begin{theorem}[Winding-Charge Correspondence \tagDc{}]
\label{thm:winding-charge}
In Kaluza-Klein theory, the winding number $W$ of a field configuration around $\Sone_\xi$ corresponds to electric charge:
\begin{equation}
\label{eq:charge-winding}
Q = W = \frac{1}{2\pi} \oint_\gamma d\xi
\end{equation}
where $\gamma$ is a closed loop around the defect in the $\xi$ direction.
\end{theorem}

\begin{remark}[Epistemic Status]
The correspondence $Q = W$ is:
\begin{itemize}
    \item \tagDc{} within EDC: It follows from the KK ansatz and the identification of the $\Uone$ gauge field with electromagnetism.
    \item The \emph{normalization} (charge in units of $e$) requires matching to experiment, hence involves \tagCal{} elements.
\end{itemize}
\end{remark}

\subsection{Layered Structure (Not Direct Product)}

\begin{remark}[Caution on Group Structure \tagP{}]
\label{rem:layered-caution}
It is tempting to write a ``global symmetry group'' as:
\begin{equation}
\label{eq:naive-product}
\mathcal{G}_{\text{EDC}} \stackrel{?}{=} \Diff(\Mfour) \times \Uone_\xi
\end{equation}
However, this is \textbf{not rigorously established}. The actual symmetry structure is more subtle:
\begin{enumerate}
    \item $\Diff(\Mfour)$ acts on the base; $\Uone_\xi$ acts on the fiber.
    \item The product structure of $\Mfive$ induces a \emph{semi-direct} or \emph{fiber bundle} relationship, not a simple direct product.
    \item Matter fields (defects) transform under both, but the coupling is nontrivial.
\end{enumerate}

We therefore describe the symmetries as a \textbf{layered structure}:
\begin{center}
\renewcommand{\arraystretch}{1.3}
\begin{tabular}{@{}lll@{}}
\toprule
\textbf{Layer} & \textbf{Symmetry} & \textbf{Physical Role} \\
\midrule
Kinematic (base) & $\Diff(\Mfour)$ & General covariance, gravity \\
Internal (fiber) & $\Isom(\Sone_\xi) \cong \Uone$ & Charge conservation, EM \\
\bottomrule
\end{tabular}
\end{center}
\end{remark}

\subsection{Summary}

The symmetry content of the EDC manifold is organized as:

\begin{equation}
\label{eq:symmetry-layers}
\boxed{
\text{Symmetry Layers}: \quad
\underbrace{\Diff(\Mfour)}_{\text{kinematic}} \quad \oplus \quad \underbrace{\Isom(\Sone_\xi)}_{\text{internal}}
}
\end{equation}

The ``$\oplus$'' notation indicates layering, not algebraic direct sum. The precise mathematical structure (principal bundle, gauge group action) is left for future formalization \tagOpen{}.
