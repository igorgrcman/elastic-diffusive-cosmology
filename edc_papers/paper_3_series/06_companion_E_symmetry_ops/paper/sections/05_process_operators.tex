% ============================================================================
\section{Topological Process Operators}
\label{sec:process-operators}
% ============================================================================

We define three formal operators representing topological processes that change defect configurations. These operators act on the invariant tuple $(W, Q, \mathcal{C}, s)$ and encode fundamental physical processes.

\subsection{Operator Definitions}

\begin{definition}[Excitation Operator $\opE$ \tagDc{}/\tagP{}]
\label{def:op-E}
The \textbf{excitation operator} $\opE_n$ raises a defect to a higher mode along $\Sone_\xi$:
\begin{equation}
\label{eq:op-E}
\opE_n: \mathcal{D}(n_0) \longrightarrow \mathcal{D}(n_0 + n)
\end{equation}
where $n_0$ is the initial mode number and $n \geq 1$ is the excitation level.

\textbf{Invariants:}
\begin{itemize}
    \item \emph{Preserved}: $W$, $Q$, $\mathcal{C}$, $s$
    \item \emph{Changed}: Internal mode number $n$; mass increases
\end{itemize}

\textbf{Physical interpretation}: Generation transitions (e.g., $e \to \mu \to \tau$).
\end{definition}

\begin{definition}[Relaxation Operator $\opR$ \tagDc{}/\tagP{}]
\label{def:op-R}
The \textbf{relaxation operator} $\opR$ shifts the sector label by one unit in $\Zsix$:
\begin{equation}
\label{eq:op-R}
\opR: (W, Q, \mathcal{C}, s) \longrightarrow (W', Q', \mathcal{C}', s+1 \mod 6)
\end{equation}
with compensating changes in $W$, $Q$, $\mathcal{C}$ to satisfy conservation laws.

\textbf{Invariants:}
\begin{itemize}
    \item \emph{Preserved}: Total winding (with emitted particles), total charge
    \item \emph{Changed}: Sector $s$; quark content; particle identity
\end{itemize}

\textbf{Physical interpretation}: Weak decay processes (e.g., $\beta^-$: neutron $\to$ proton).
\end{definition}

\begin{definition}[Merging Operator $\opM$ \tagP{}]
\label{def:op-M}
The \textbf{merging operator} $\opM$ combines two defects into a bound configuration:
\begin{equation}
\label{eq:op-M}
\opM: \mathcal{D}_1 \otimes \mathcal{D}_2 \longrightarrow \mathcal{D}_{\text{bound}}
\end{equation}

\textbf{Invariants:}
\begin{itemize}
    \item \emph{Preserved}: Total $W$, total $Q$
    \item \emph{Changed}: Configuration space geometry; binding energy released
\end{itemize}

\textbf{Physical interpretation}: Nuclear binding (``merged Inflow'').
\end{definition}

\subsection{Operator Properties}

\begin{table}[h]
\centering
\caption{Summary of process operators}
\label{tab:operators}
\renewcommand{\arraystretch}{1.4}
\begin{tabular}{@{}lp{3cm}p{3cm}p{2.5cm}l@{}}
\toprule
\textbf{Operator} & \textbf{Action} & \textbf{Preserved} & \textbf{Changed} & \textbf{Tag} \\
\midrule
$\opE_n$ & Raise mode by $n$ & $W, Q, \mathcal{C}, s$ & Mode $n$, mass & \tagDc{}/\tagP{} \\
$\opR$ & Shift sector $s \to s+1$ & Total $W$, total $Q$ & $s$, quark IDs & \tagDc{}/\tagP{} \\
$\opM$ & Merge two defects & Total $W$, total $Q$ & Geometry, $E_{\text{bind}}$ & \tagP{} \\
\bottomrule
\end{tabular}
\end{table}

\subsection{Formal Requirements}

For each operator, we state the mathematical requirements that a full derivation must satisfy:

\begin{enumerate}
    \item \textbf{$\opE$ (Excitation)}:
    \begin{itemize}
        \item Must follow from the spectrum of the $\Sone_\xi$ Laplacian acting on defect wavefunctions.
        \item The mass increase formula $m_n = m_0 \cdot f(n, \alpha)$ must be derived from the 5D action.
        \item Current status: The lepton mass formulas ($m_\mu/m_e$, $m_\tau/m_\mu$) are \tagI{}, not \tagDer{}.
    \end{itemize}

    \item \textbf{$\opR$ (Relaxation)}:
    \begin{itemize}
        \item Must follow from the $\Zsix$ potential landscape $V(\theta)$ and tunneling/transition dynamics.
        \item The rate formula must connect to observed weak decay rates.
        \item Current status: The mechanism is \tagP{}; the $\Zsix$ potential form is \tagDc{}.
    \end{itemize}

    \item \textbf{$\opM$ (Merging)}:
    \begin{itemize}
        \item Must follow from the ``merged Inflow'' geometry where defect configuration spaces overlap.
        \item The binding energy formula must be derived from surface area reduction.
        \item Current status: Entirely \tagP{}; no quantitative formula exists.
    \end{itemize}
\end{enumerate}

\subsection{Falsifiability Hooks}

Each operator interpretation makes implicit predictions that could falsify it:

\begin{tcolorbox}[colback=red!5,colframe=red!40!black,title=\textbf{Falsifiability Conditions}]

\textbf{$\opE$ (Excitation)}:
\begin{itemize}[nosep]
    \item If a fourth generation lepton is discovered, the ``SU(3) saturation'' argument fails.
    \item If $m_\tau/m_\mu \neq 16\pi/3$ to better than 1\%, the geometric interpretation is falsified.
\end{itemize}

\textbf{$\opR$ (Relaxation)}:
\begin{itemize}[nosep]
    \item If $\beta^-$ decay products violate the $\Zsix$ step pattern (e.g., direct $n \to \bar{p}$), the operator is falsified.
    \item If neutron lifetime deviates from the topological barrier prediction (once derived), the mechanism fails.
\end{itemize}

\textbf{$\opM$ (Merging)}:
\begin{itemize}[nosep]
    \item If nuclear binding energies show no correlation with configuration space surface area reduction, the mechanism is falsified.
    \item If light nuclei binding energies cannot be fit with a universal ``Inflow overlap'' parameter, the model fails.
\end{itemize}

\end{tcolorbox}

\subsection{Composition Rules}

\begin{remark}[Operator Algebra \tagOpen{}]
The composition of operators (e.g., $\opR \circ \opE$, $\opM \circ \opR$) is not yet formalized. Questions include:
\begin{itemize}
    \item Do operators commute? (Likely not: $\opR \circ \opE \neq \opE \circ \opR$)
    \item Is there a group structure? (Unlikely for $\opM$ due to irreversibility of binding)
    \item How do selection rules emerge? (From symmetry constraints on compositions)
\end{itemize}
This remains \tagOpen{} for future work.
\end{remark}
