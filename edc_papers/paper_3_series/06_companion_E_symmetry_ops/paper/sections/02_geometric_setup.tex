% ============================================================================
\section{Minimal Geometric Setup}
\label{sec:geometric-setup}
% ============================================================================

This section establishes the geometric foundation required for defining symmetries and operators. We state only what is needed; full details are in the EDC Framework Reference.

\subsection{The 5D Manifold}

\begin{postulate}[5D Product Structure \tagP{}]
\label{post:5d-manifold}
The EDC spacetime is a 5-dimensional manifold with product topology:
\begin{equation}
\label{eq:M5-product}
\Mfive = \Mfour \times \Sone_\xi
\end{equation}
where $\Mfour$ is a 4-dimensional Lorentzian manifold (our observable spacetime) and $\Sone_\xi$ is a compact circle of circumference $\Lxi = 2\pi \Rxi$.
\end{postulate}

The coordinate on $\Sone_\xi$ is denoted $\xi \in [0, \Lxi)$ with periodic identification. This structure follows the Kaluza-Klein paradigm~\cite{kaluza1921,klein1926}.

\subsection{The Compact Scale Parameter}

\begin{definition}[Compact Dimension Scale]
\label{def:Lxi}
The length scale $\Lxi$ (equivalently, radius $\Rxi = \Lxi / 2\pi$) characterizes the size of the compact dimension. This is a fundamental parameter of the theory.
\end{definition}

\begin{remark}[Historical Identification with Compton Wavelength \tagCal{}/\tagP{}]
\label{rem:Lxi-identification}
In early EDC literature, the compact scale was sometimes identified with the electron Compton wavelength:
\begin{equation}
\label{eq:Lxi-Compton}
\Lxi \sim \lambdabar_C^{(e)} = \frac{\hbar}{m_e c} \approx 3.86 \times 10^{-13} \text{ m}
\end{equation}
This identification is \textbf{not derived} from first principles. It is either:
\begin{itemize}
    \item \tagCal{}: Calibrated to match observed physics (e.g., requiring $\alpha$ formula to work), or
    \item \tagP{}: Postulated as a foundational assumption linking 5D geometry to particle scales.
\end{itemize}
The derivation of $\Lxi$ from 5D dynamics remains \tagOpen{}.
\end{remark}

\begin{remark}[Correction Note: Canonical Scale Separation]
\label{rem:scale-correction}
The EDC Framework Reference (v2.0) establishes a \emph{different} canonical scale hierarchy:
\begin{align}
\Rxi &\sim 10^{-18}\,\mathrm{m} && \text{(membrane/weak-KK scale) \tagDc{}} \\
\lambdabar_C^{(e)} &\sim 3.86 \times 10^{-13}\,\mathrm{m} && \text{(reduced Compton wavelength) \tagBL{}}
\end{align}
These are \textbf{distinct scales} separated by five orders of magnitude. The identification $\Lxi \sim \lambdabar_C^{(e)}$ in Eq.~\eqref{eq:Lxi-Compton} is a \emph{historical calibration} that conflates the kinematic Compton scale with the dynamical compactification radius. The formula $\alpha = r_e/\lambdabar_C^{(e)}$ is standard QED \tagBL{}; it should \textbf{not} be written as $\alpha = r_e/\Rxi$. This paper uses the earlier convention for continuity with prior work; see the Framework Reference for the corrected treatment.
\end{remark}

\subsection{Newton's Constant Reduction}

\begin{theorem}[4D Newton Constant \tagDc{}]
\label{thm:G4-reduction}
Standard Kaluza-Klein dimensional reduction gives:
\begin{equation}
\label{eq:G4-KK}
G_4 = \frac{G_5}{2\pi \Rxi} = \frac{G_5}{\Lxi}
\end{equation}
where $G_5$ is the 5D gravitational coupling.
\end{theorem}

\begin{remark}[Epistemic Status]
Equation~\eqref{eq:G4-KK} is standard KK theory, not unique to EDC. It is tagged \tagDc{} because:
\begin{itemize}
    \item The reduction formula is mathematically derived (given the product ansatz).
    \item The value of $G_5$ is \textbf{not} derived within EDC; it remains \tagOpen{}.
\end{itemize}
Thus, while the \emph{form} of Eq.~\eqref{eq:G4-KK} is derived, the \emph{numerical prediction} for $G_4$ requires knowing $G_5$.
\end{remark}

\subsection{Scope and Non-Scope}

\begin{tcolorbox}[colback=blue!5,colframe=blue!40!black,title=\textbf{Scope \& Non-Scope: Geometric Setup}]
\textbf{This section DOES:}
\begin{itemize}[nosep,leftmargin=*]
    \item Define the product manifold structure $\Mfive = \Mfour \times \Sone_\xi$
    \item Introduce the compact scale $\Lxi$ as a parameter
    \item State the KK reduction formula for $G_4$
    \item Clearly tag the $\Lxi \sim \lambdabar_C^{(e)}$ identification as \tagCal{}/\tagP{}
\end{itemize}

\textbf{This section does NOT:}
\begin{itemize}[nosep,leftmargin=*]
    \item Derive $\Lxi$ from 5D dynamics
    \item Derive $G_5$ from EDC first principles
    \item Claim the geometry is unique or fully constrained
\end{itemize}
\end{tcolorbox}
