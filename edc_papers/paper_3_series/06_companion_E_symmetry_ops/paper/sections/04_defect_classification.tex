% ============================================================================
\section{Defect Classification as Layered Structure}
\label{sec:defect-classification}
% ============================================================================

Particles in EDC are topological defects in the 5D brane. Different defect types carry different invariants. We organize these invariants into a layered classification scheme.

\subsection{Defect Invariants}

\begin{definition}[Defect State Vector]
\label{def:defect-state}
A defect configuration is characterized by a tuple of invariants:
\begin{equation}
\label{eq:defect-tuple}
\mathcal{D} = (W, Q, \mathcal{C}, s)
\end{equation}
where:
\begin{itemize}
    \item $W \in \mathbb{Z}$ or $\mathbb{Z}/3$: Total winding number
    \item $Q \in \mathbb{Z}$ (in units of $e/3$): Electric charge
    \item $\mathcal{C} \in \{-, r, g, b\}$: Color index (``$-$'' for colorless)
    \item $s \in \Zsix$: Sector label on the transverse ring
\end{itemize}
\end{definition}

\subsection{Y-Junction Mode Algebra}

The Y-junction (three-arm configuration) supports oscillation modes that generate an algebraic structure.

\begin{theorem}[Y-Junction Mode Algebra \tagDc{}]
\label{thm:Y-algebra}
The modes of a Y-junction configuration form an 8-dimensional space with structure:
\begin{equation}
\label{eq:AY-algebra}
\mathcal{A}_Y \sim \mathfrak{su}(3)
\end{equation}
consisting of:
\begin{itemize}
    \item 6 ``exchange modes'': oscillations that swap amplitude between pairs of arms
    \item 2 ``diagonal modes'': oscillations that preserve arm identity but modulate relative phases
\end{itemize}
\end{theorem}

\begin{remark}[Epistemic Status]
The identification $\mathcal{A}_Y \sim \mathfrak{su}(3)$ is \tagDc{}:
\begin{itemize}
    \item The mode counting (8 = 6 + 2) follows from junction geometry.
    \item The Lie algebra structure (commutators) requires explicit calculation from the 5D action.
    \item Full proof that $[\cdot, \cdot]$ closes on $\mathfrak{su}(3)$ is partially shown in the Framework Reference (Thm.~5.3--5.5) but relies on the Steiner angle assumption.
\end{itemize}
We write ``$\sim$'' rather than ``$=$'' to indicate structural similarity, not proven isomorphism.
\end{remark}

\subsection{Ring Sector Labels: $\Zsix$ Structure}

The transverse ring in junction configurations admits discrete symmetry.

\begin{theorem}[$\Zsix$ Sector Decomposition \tagDer{}]
\label{thm:Z6-sectors}
The symmetry of the transverse ring configuration space factors as:
\begin{equation}
\label{eq:Z6-factor}
\Zsix = \Ztri \times \Ztwo
\end{equation}
where:
\begin{itemize}
    \item $\Ztri$: Cyclic permutation of the three Y-junction arms
    \item $\Ztwo$: Matter-antimatter conjugation (reflection symmetry)
\end{itemize}
\end{theorem}

\begin{remark}[Sector Labels and Nucleons]
The six sectors $s \in \{0, 1, 2, 3, 4, 5\}$ correspond to stable configurations:
\begin{center}
\renewcommand{\arraystretch}{1.2}
\begin{tabular}{@{}cll@{}}
\toprule
\textbf{Sector $s$} & \textbf{Angle $\theta$} & \textbf{Interpretation} \\
\midrule
0 & $0\degree$ & Proton ground state \\
1 & $60\degree$ & Neutron ground state \\
2 & $120\degree$ & (Unstable / transition) \\
3 & $180\degree$ & Antiproton ground state \\
4 & $240\degree$ & Antineutron ground state \\
5 & $300\degree$ & (Unstable / transition) \\
\bottomrule
\end{tabular}
\end{center}
The proton occupies $s = 0$; the neutron occupies $s = 1$. A $\Zsix$ step ($s \to s+1$) corresponds to a topological transition.
\end{remark}

\subsection{Layered Structure (Not Subgroups)}

\begin{remark}[On Group Containment \tagP{}]
\label{rem:not-subgroups}
It is \textbf{not claimed} that $\SUthree$ and $\Zsix$ are subgroups of a single global symmetry group. Rather, they represent:
\begin{itemize}
    \item $\mathcal{A}_Y \sim \mathfrak{su}(3)$: \emph{Local} mode algebra at junctions (dynamical degrees of freedom)
    \item $\Zsix$: \emph{Global} sector labels (topological vacuum structure)
\end{itemize}

These structures are \textbf{coupled} (a $\Zsix$ transition involves mode excitation), but the precise relationship is:
\begin{equation}
\label{eq:coupling-schematic}
\text{(Sector shift in } \Zsix \text{)} \longleftrightarrow \text{(Mode excitation in } \mathcal{A}_Y \text{)}
\end{equation}
The mathematical formalization of this coupling remains \tagOpen{}.
\end{remark}

\subsection{Defect Classification Table}

\begin{table}[h]
\centering
\caption{Defect types and their invariants}
\label{tab:defect-invariants}
\renewcommand{\arraystretch}{1.3}
\begin{tabular}{@{}lccccl@{}}
\toprule
\textbf{Particle} & $W$ & $Q$ & $\mathcal{C}$ & $s$ & \textbf{Tag} \\
\midrule
Electron ($e^-$) & $-1$ & $-1$ & $-$ & --- & \tagI{} \\
Proton ($p$) & $+1$ & $+1$ & $-$ & $0$ & \tagI{} \\
Neutron ($n$) & $+1$ & $0$ & $-$ & $1$ & \tagI{} \\
Up quark ($u$) & $+2/3$ & $+2/3$ & $r,g,b$ & --- & \tagDer{} \\
Down quark ($d$) & $-1/3$ & $-1/3$ & $r,g,b$ & --- & \tagDer{} \\
\bottomrule
\end{tabular}
\end{table}

\begin{remark}
The electron, proton, and neutron identifications are \tagI{} (pattern matching between EDC defect types and SM particles). The quark winding numbers are \tagDer{} from charge constraints (see Framework Reference, Thm.~4.4).
\end{remark}
