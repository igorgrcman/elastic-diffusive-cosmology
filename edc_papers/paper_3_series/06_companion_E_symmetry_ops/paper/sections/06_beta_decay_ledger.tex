% ============================================================================
\section{$\beta^-$ Decay Conservation Ledger}
\label{sec:beta-ledger}
% ============================================================================

This section presents the $\beta^-$ decay process ($n \to p + e^- + \bar{\nu}_e$) in EDC language, using a conservation ledger to track topological invariants.

\subsection{The Ledger Formalism}

\begin{definition}[Conservation Ledger]
\label{def:ledger}
A \textbf{conservation ledger} is a bookkeeping table that tracks the values of conserved quantities before and after a topological transition. For each quantity $X$, we require:
\begin{equation}
\label{eq:ledger-balance}
\sum_{\text{initial}} X_i = \sum_{\text{final}} X_f
\end{equation}
\end{definition}

\subsection{$\beta^-$ Decay in EDC Language}

In EDC, the neutron and proton are Y-junction defects at different $\Zsix$ sector positions. The $\beta^-$ decay corresponds to the relaxation operator $\opR$:
\begin{equation}
\label{eq:beta-EDC}
\opR: n(s=1) \longrightarrow p(s=0) + e^- + \text{(neutral channel)}
\end{equation}

\subsection{Conservation Ledger Table}

\begin{table}[h]
\centering
\caption{$\beta^-$ decay conservation ledger}
\label{tab:beta-ledger}
\renewcommand{\arraystretch}{1.4}
\begin{tabular}{@{}l|c|ccc|c@{}}
\toprule
& \textbf{Before} & \multicolumn{3}{c|}{\textbf{After}} & \textbf{Balance} \\
\textbf{Quantity} & Neutron & Proton & Electron & Neutral & Check \\
\midrule
Winding $W$ & $+1$ & $+1$ & $-1$ & $0$ & $+1 = +1 - 1 + 0$ {\color{tagDer}\checkmark} \\
Charge $Q$ & $0$ & $+1$ & $-1$ & $0$ & $0 = +1 - 1 + 0$ {\color{tagDer}\checkmark} \\
Sector $s$ & $1$ & $0$ & --- & --- & $\Delta s = -1$ \\
Baryon \# $B$ & $+1$ & $+1$ & $0$ & $0$ & $+1 = +1$ {\color{tagDer}\checkmark} \\
Lepton \# $L$ & $0$ & $0$ & $+1$ & $-1$ & $0 = 0 + 1 - 1$ {\color{tagDer}\checkmark} \\
\bottomrule
\end{tabular}
\end{table}

\subsection{The Neutral Channel Requirement}

\begin{theorem}[Neutral Channel Constraint \tagDc{}]
\label{thm:neutral-channel}
For the conservation ledger to balance, the $\beta^-$ decay \textbf{must} produce a neutral channel with:
\begin{equation}
\label{eq:neutral-requirements}
W_{\text{neutral}} = 0, \quad Q_{\text{neutral}} = 0, \quad L_{\text{neutral}} = -1
\end{equation}
This is a topological constraint, independent of dynamics.
\end{theorem}

\begin{proof}
From the ledger (Table~\ref{tab:beta-ledger}):
\begin{align}
W_{\text{neutral}} &= W_n - W_p - W_e = 1 - 1 - (-1) - W_{\text{neutral}} \Rightarrow W_{\text{neutral}} = 0 \\
Q_{\text{neutral}} &= Q_n - Q_p - Q_e = 0 - 1 - (-1) = 0 \\
L_{\text{neutral}} &= L_n - L_p - L_e = 0 - 0 - 1 = -1
\end{align}
These are necessary conditions for ledger closure.
\end{proof}

\begin{remark}[Epistemic Status]
The \emph{requirement} for a neutral channel is \tagDc{}: it follows logically from conservation laws applied to the ledger. No dynamics or specific mechanism is invoked---only bookkeeping.
\end{remark}

\subsection{Identification with Antineutrino}

\begin{postulate}[$\xi$-Wave Identification \tagP{}]
\label{post:xi-wave}
The neutral channel required by the ledger is realized in EDC as a \textbf{$\xi$-wave}: a propagating excitation in the compact dimension $\Sone_\xi$, rather than a frozen topological defect.

In 4D evidence language, this $\xi$-wave is \textbf{identified} with the electron antineutrino $\bar{\nu}_e$.
\end{postulate}

\begin{remark}[Two-Sided Reading]
\begin{itemize}
    \item \textbf{5D cause}: The neutral channel is a $\xi$-wave (propagating mode, not localized defect).
    \item \textbf{4D evidence}: The neutral channel is identified with $\bar{\nu}_e$ (weak interaction phenomenology).
\end{itemize}
The identification $\xi\text{-wave} \leftrightarrow \bar{\nu}_e$ is \tagP{}, not derived from the 5D action.
\end{remark}

\subsection{Why $\xi$-Wave?}

The $\xi$-wave hypothesis (Postulate~\ref{post:xi-wave}) is motivated by:

\begin{enumerate}
    \item \textbf{Neutrality}: A wave in $\Sone_\xi$ carries no net winding ($W = 0$) and no charge ($Q = 0$).

    \item \textbf{Near-masslessness}: Unlike frozen defects (which have mass from configuration space volume), waves have energy $E = \hbar \omega$ with no rest mass contribution from topology.

    \item \textbf{Weak interaction}: Waves do not create ``holes'' in the brane like frozen defects do, explaining small interaction cross-sections.

    \item \textbf{Lepton number}: The $L = -1$ assignment follows from the \emph{direction} of propagation or phase winding of the wave.
\end{enumerate}

These are \textbf{plausibility arguments}, not derivations. The full dynamics of $\xi$-waves remains \tagOpen{}.

\subsection{Comparison with Standard Model}

\begin{tcolorbox}[colback=gray!5,colframe=gray!40!black,title=\textbf{Baseline Comparison \tagBL{}}]
In the Standard Model~\cite{pdg2024}, $\beta^-$ decay is mediated by $W^-$ boson exchange:
\[
n \to p + W^- \to p + e^- + \bar{\nu}_e
\]
The conservation laws (charge, baryon number, lepton number) are identical to the EDC ledger. The difference is:
\begin{itemize}
    \item \textbf{SM}: Decay mediated by gauge boson; neutrino is a fundamental fermion.
    \item \textbf{EDC}: Decay is topological relaxation; neutrino is a $\xi$-wave excitation.
\end{itemize}
Both frameworks predict the same final state and conservation laws. The distinction lies in the underlying mechanism.
\end{tcolorbox}

\subsection{Ledger Summary}

The $\beta^-$ decay ledger demonstrates:

\begin{enumerate}
    \item \tagDc{} The \emph{need} for a neutral channel follows from topological conservation laws.
    \item \tagP{} The \emph{identification} of this channel with a $\xi$-wave (and hence $\bar{\nu}_e$) is a hypothesis.
    \item \tagOpen{} The \emph{dynamics} (decay rate, energy spectrum) require deriving $\xi$-wave properties from the 5D action.
\end{enumerate}
