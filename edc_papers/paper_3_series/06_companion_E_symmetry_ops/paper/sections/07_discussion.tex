% ============================================================================
\section{Discussion and Roadmap}
\label{sec:discussion}
% ============================================================================

\subsection{What This Paper Has Defined}

This companion paper has formalized:

\begin{enumerate}
    \item \textbf{Symmetry Layering} (Section~\ref{sec:symmetry-layering}): The EDC manifold has two symmetry layers---kinematic $\Diff(\Mfour)$ and internal $\Isom(\Sone_\xi) \cong \Uone$---described as a layered structure rather than a simple product group.

    \item \textbf{Defect Classification} (Section~\ref{sec:defect-classification}): Defects are characterized by invariants $(W, Q, \mathcal{C}, s)$. The Y-junction mode algebra ($\sim \mathfrak{su}(3)$) and ring sector labels ($\Zsix$) are related but not claimed to be subgroups of a global symmetry.

    \item \textbf{Process Operators} (Section~\ref{sec:process-operators}): Three operators---Excitation ($\opE$), Relaxation ($\opR$), Merging ($\opM$)---formalize generation transitions, weak decay, and nuclear binding respectively.

    \item \textbf{Conservation Ledger} (Section~\ref{sec:beta-ledger}): The $\beta^-$ decay is analyzed via a ledger that requires a neutral channel \tagDc{}, identified with a $\xi$-wave (antineutrino) \tagP{}.
\end{enumerate}

\subsection{Research Roadmap: Upgrading Tags}

To upgrade epistemic tags from \tagP{}/\tagDc{} to \tagDer{}, the following derivations are needed:

\begin{table}[h]
\centering
\caption{Research roadmap for tag upgrades}
\label{tab:roadmap}
\renewcommand{\arraystretch}{1.4}
\begin{tabular}{@{}p{4cm}lp{5cm}@{}}
\toprule
\textbf{Claim} & \textbf{Current} & \textbf{Required for \tagDer{}} \\
\midrule
$\Lxi \sim \lambdabar_C^{(e)}$ & \tagCal{}/\tagP{} & Derive $\Lxi$ from 5D variational principle \\
$\mathcal{A}_Y \sim \mathfrak{su}(3)$ & \tagDc{} & Prove Lie algebra closure from 5D action \\
$\Zsix$ sector structure & \tagDer{} & (Already derived from product decomposition) \\
Operator $\opE$ dynamics & \tagP{} & Derive mass spectrum from $\Sone_\xi$ Laplacian \\
Operator $\opR$ dynamics & \tagP{} & Derive tunneling rate from $V(\theta)$ potential \\
Operator $\opM$ dynamics & \tagP{} & Derive binding energy from Inflow geometry \\
$\xi$-wave $\leftrightarrow \bar{\nu}$ & \tagP{} & Derive $\xi$-wave properties; match to $\bar{\nu}$ phenomenology \\
\bottomrule
\end{tabular}
\end{table}

\subsection{Open Questions}

The following questions remain \tagOpen{}:

\begin{enumerate}
    \item \textbf{Operator algebra}: Do $\opE$, $\opR$, $\opM$ form a closed algebraic structure? What are the selection rules?

    \item \textbf{$G_5$ derivation}: Can the 5D gravitational coupling be derived from EDC geometry, or must it be an input?

    \item \textbf{$\xi$-wave dynamics}: What is the dispersion relation for $\xi$-waves? How do they interact with frozen defects?

    \item \textbf{Neutrino oscillations}: If neutrinos are $\xi$-waves, what mechanism produces mass differences and mixing?

    \item \textbf{Nuclear binding quantitative}: Can the isoperimetric inequality (surface area minimization) yield helium binding energy to percent-level accuracy?
\end{enumerate}

\subsection{How Paper 3 Will Use This Material}

Paper 3 (on neutron physics) will cite this companion paper for:

\begin{itemize}
    \item The formal definition of the $\Zsix$ sector structure
    \item The relaxation operator $\opR$ as the mechanism for $\beta^-$ decay
    \item The conservation ledger formalism for analyzing decay processes
    \item The $\xi$-wave hypothesis for the neutral channel
\end{itemize}

By citing this companion, Paper 3 can focus on neutron-specific physics without repeating foundational definitions.

% ============================================================================
\section*{Conclusion}
\label{sec:conclusion}
\addcontentsline{toc}{section}{Conclusion}
% ============================================================================

This paper has provided formal definitions for symmetry layering, defect classification, and topological process operators within Elastic Diffusive Cosmology. Every claim is tagged with its epistemic status---\tagDer{}, \tagDc{}, \tagCal{}, \tagI{}, \tagP{}, or \tagOpen{}---to maintain intellectual honesty about what is derived versus hypothesized.

The key contributions are:
\begin{enumerate}
    \item A layered (not product-group) description of EDC symmetries
    \item Formal definitions of operators $\opE$, $\opR$, $\opM$ with falsifiability conditions
    \item A conservation ledger for $\beta^-$ decay showing the topological necessity of a neutral channel
    \item A clear research roadmap for upgrading postulates to derivations
\end{enumerate}

This framework provides a foundation for Paper 3 and subsequent EDC publications to build upon without duplicating foundational material.
