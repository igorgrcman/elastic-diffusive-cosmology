% Companion M: Muon Decay as Thick-Brane Tomography in EDC
% Version 0.2 — 2026-01-20 (stabilization patch)
% Build: xelatex main.tex && xelatex main.tex

\documentclass[11pt,a4paper]{article}

% ============================================================
% PACKAGES
% ============================================================
\usepackage{fontspec}
\IfFontExistsTF{TeX Gyre Termes}{%
  \setmainfont{TeX Gyre Termes}
}{%
  \setmainfont{Times New Roman}[Ligatures=TeX]
}
\usepackage{amsmath,amssymb,amsthm,mathtools}
\usepackage{geometry}
\geometry{margin=2.5cm}
\usepackage{hyperref}
\hypersetup{colorlinks=true,linkcolor=blue!60!black,citecolor=green!50!black,urlcolor=blue!70!black}
\usepackage{enumitem}
\usepackage{booktabs}
\usepackage{array}
\usepackage{xcolor}
\usepackage{tcolorbox}
\tcbuselibrary{breakable}

% TikZ
\usepackage{tikz}
\usetikzlibrary{calc,angles,quotes,decorations.markings,decorations.pathmorphing,positioning}

% ============================================================
% SHARED EDC STYLE FILES
% ============================================================
% edc_style.tex — Canonical EDC Paper Style for Paper 3 Series
% Version 1.0 — 2026-01-20
%
% USAGE: Include in preamble AFTER loading packages but BEFORE \begin{document}
%   % edc_style.tex — Canonical EDC Paper Style for Paper 3 Series
% Version 1.0 — 2026-01-20
%
% USAGE: Include in preamble AFTER loading packages but BEFORE \begin{document}
%   % edc_style.tex — Canonical EDC Paper Style for Paper 3 Series
% Version 1.0 — 2026-01-20
%
% USAGE: Include in preamble AFTER loading packages but BEFORE \begin{document}
%   \input{../_shared/style/edc_style}
%   \input{../_shared/style/tikz_style_edc}  % if using TikZ figures
%
% REQUIRED PACKAGES (load these in main document before \input):
%   fontspec, amsmath, amssymb, amsthm, mathtools, geometry
%   hyperref, enumitem, booktabs, array, xcolor, tcolorbox
%
% ============================================================

% ============================================================
%  EPISTEMIC TAG COLORS
% ============================================================
\definecolor{tagDer}{RGB}{0,128,0}      % Green - Derived
\definecolor{tagDc}{RGB}{0,0,200}       % Blue - Deduced/Constrained
\definecolor{tagCal}{RGB}{200,0,0}      % Red - Calibrated
\definecolor{tagP}{RGB}{128,0,128}      % Purple - Postulated
\definecolor{tagBL}{RGB}{128,128,128}   % Gray - Baseline
\definecolor{tagI}{RGB}{255,140,0}      % Orange - Identified
\definecolor{tagOpen}{RGB}{200,100,0}   % Dark orange - Open

% ============================================================
%  EPISTEMIC TAG COMMANDS
% ============================================================
% Use these to mark claims with their epistemic status
\newcommand{\tagDer}{\textcolor{tagDer}{\textbf{[Der]}}}    % Derived from axioms
\newcommand{\tagDc}{\textcolor{tagDc}{\textbf{[Dc]}}}       % Deduced/Constrained
\newcommand{\tagCal}{\textcolor{tagCal}{\textbf{[Cal]}}}    % Calibrated (fitted)
\newcommand{\tagP}{\textcolor{tagP}{\textbf{[P]}}}          % Postulated
\newcommand{\tagBL}{\textcolor{tagBL}{\textbf{[BL]}}}       % Baseline (external fact)
\newcommand{\tagI}{\textcolor{tagI}{\textbf{[I]}}}          % Identified (pattern match)
\newcommand{\tagOpen}{\textcolor{tagOpen}{\textbf{[OPEN]}}} % Open problem
\newcommand{\tagDef}{\textcolor{tagDc}{\textbf{[Def]}}}     % Definition

% ============================================================
%  THEOREM ENVIRONMENTS
% ============================================================
\newtheorem{postulate}{Postulate}
\newtheorem{definition}{Definition}[section]
\newtheorem{theorem}{Theorem}[section]
\newtheorem{lemma}[theorem]{Lemma}
\newtheorem{corollary}[theorem]{Corollary}
\newtheorem{proposition}[theorem]{Proposition}
\newtheorem{remark}{Remark}[section]

% ============================================================
%  COMMON EDC SYMBOLS
% ============================================================
% Symmetry groups
\newcommand{\Ztwo}{\mathbb{Z}_2}
\newcommand{\Zthree}{\mathbb{Z}_3}
\newcommand{\Ztri}{\mathbb{Z}_3}    % alias
\newcommand{\Zsix}{\mathbb{Z}_6}

% Geometric objects
\newcommand{\Sthree}{S^3}           % 3-sphere
\newcommand{\Stwo}{S^2}             % 2-sphere
\newcommand{\Bthree}{B^3}           % 3-ball
\newcommand{\Mfive}{\mathcal{M}_5}  % 5D manifold
\newcommand{\Bfour}{\mathcal{B}_4}  % 4D brane

% Physical quantities
\newcommand{\tension}{\tau}         % string/flux-tube tension (E/L)
\newcommand{\re}{r_e}               % electron radius

% Operators
\newcommand{\Pfrozen}{\mathcal{P}_{\mathrm{frozen}}}  % Frozen projection operator
\newcommand{\Ebrane}{\mathcal{E}_{\mathrm{brane}}}    % Brane energy store

% Bulk-brane exchange current (canonical notation from Framework v2.0)
\newcommand{\Jbb}[1]{J^{#1}_{\mathrm{bulk}\to\mathrm{brane}}}

% ============================================================
%  TCOLORBOX STYLES FOR EDC PAPERS
% ============================================================
% Cornerstone box (blue) — key claims/foundations
\tcbset{
    edcCornerstone/.style={
        colback=blue!5,
        colframe=blue!40!black,
        fonttitle=\bfseries
    }
}

% Guardrail box (gray) — epistemic warnings/constraints
\tcbset{
    edcGuardrail/.style={
        colback=gray!5!white,
        colframe=gray!60!black,
        fonttitle=\bfseries
    }
}

% PPN box (blue, lighter) — Physical Process Narrative
\tcbset{
    edcPPN/.style={
        colback=blue!5,
        colframe=blue!50!black,
        fonttitle=\bfseries
    }
}

% Canonical box (yellow/orange) — canonical definitions/glossary
\tcbset{
    edcCanonical/.style={
        colback=yellow!5,
        colframe=orange!60!black,
        fonttitle=\bfseries
    }
}

% Conceptual box (yellow/orange, lighter) — conceptual pictures
\tcbset{
    edcConcept/.style={
        colback=yellow!5,
        colframe=orange!50!black,
        fonttitle=\bfseries
    }
}

% Pathway box (purple) — energy pathways, mechanisms
\tcbset{
    edcPathway/.style={
        colback=purple!5,
        colframe=purple!40!black,
        fonttitle=\bfseries
    }
}

% Model box (green) — mechanical analogies, heuristics
\tcbset{
    edcModel/.style={
        colback=green!5,
        colframe=green!40!black,
        fonttitle=\bfseries
    }
}

% Warning box (red) — non-overclaim, limitations
\tcbset{
    edcWarning/.style={
        colback=red!5,
        colframe=red!40!black,
        fonttitle=\bfseries
    }
}

% Framework quote box (gray) — verbatim from Framework v2.0
\tcbset{
    edcFramework/.style={
        colback=gray!5!white,
        colframe=gray!60!black,
        fonttitle=\small
    }
}

% Mechanism box (teal) — mechanistic dimension principle narrative
\tcbset{
    edcMechanism/.style={
        colback=teal!5,
        colframe=teal!50!black,
        fonttitle=\bfseries,
        title={Mechanistic Dimension Note (Canon)}
    }
}

% ============================================================
%  MECHANISTIC DIMENSION HELPER MACRO
% ============================================================
% Usage: \edcMechanismNote{bulk cause}{brane process}{3D output}
%
% Example:
%   \edcMechanismNote{Junction relaxes toward Steiner minimum}%
%                    {Energy pumps into brane-layer modes, redistributes}%
%                    {Electron, antineutrino, proton emerge on 3D side}
%
\newcommand{\edcMechanismNote}[3]{%
\begin{tcolorbox}[edcMechanism]
\begin{itemize}[nosep,leftmargin=*]
    \item \textbf{5D cause (bulk):} #1
    \item \textbf{Brane-layer process:} #2
    \item \textbf{3D observation (output):} #3
\end{itemize}
\vspace{0.3em}
\footnotesize\textit{Ledger closure must hold: bulk + brane + 3D outputs conserve energy/quantum numbers.}
\end{tcolorbox}
}

% ============================================================
%  RELATED DOCUMENTS MACRO
% ============================================================
% Usage: \edcRelatedDocs{main paper title}{main DOI}{companion list}
%
% Example:
%     A: \emph{Effective Lagrangian} (\href{...}{DOI}) $\cdot$
%     B: \emph{WKB Prefactor} (\href{...}{DOI})
%   }

% NOTE: \edcRelatedDocs macro deprecated (DOI registry consolidated)
% Use consolidated Zenodo article as primary reference instead.

% ============================================================
%  DOI REGISTRY DEPRECATED
% ============================================================
% Previous individual DOIs have been deprecated.
% All EDC Weak Sector content is now consolidated into a single
% Zenodo article. See paper_3_series/19_edc_weak_sector_zenodo_article/

% ============================================================
%  PHYSICAL NARRATION RULE REMINDER
% ============================================================
% Every key equation MUST be accompanied by a physical narrative stating:
%   1. 5D cause: What changes in the bulk-core configuration?
%   2. Brane response: How does the brane absorb/redistribute energy?
%   3. 3D observable output: What do observers detect on the 3D side?
%
% This rule eliminates "numerology smell" by ensuring every formula
% has a mechanistic interpretation.

% ============================================================
%  END OF STYLE FILE
% ============================================================

%   % tikz_style_edc.tex — Reusable TikZ styles for EDC papers
% Version 1.0 — 2026-01-20
% Include via: \input{tikz_style_edc}

% ============================================================
% REQUIRED LIBRARIES (must be loaded in main document)
% ============================================================
% \usetikzlibrary{calc,angles,quotes,decorations.markings,decorations.pathmorphing,positioning}

% ============================================================
% POSITIONING DEFAULTS
% ============================================================
\tikzset{
    % Default node distances for horizontal/vertical layouts
    edc node distance/.style={node distance=1.6cm and 2.0cm},
    % Compact variant for dense diagrams
    edc compact/.style={node distance=1.2cm and 1.5cm},
    % Spread variant for clarity
    edc spread/.style={node distance=2.0cm and 2.5cm},
}

% ============================================================
% COLOR PALETTE (consistent with epistemic tags)
% ============================================================
\definecolor{edcBulk}{RGB}{220,50,50}        % Red tones for bulk/5D
\definecolor{edcBrane}{RGB}{50,150,50}       % Green tones for brane-layer
\definecolor{edcOutput}{RGB}{50,100,200}     % Blue tones for 3D outputs
\definecolor{edcNeutral}{RGB}{100,100,100}   % Gray for neutral/annotations

% ============================================================
% BOX STYLES
% ============================================================
\tikzset{
    % Generic EDC box (base style)
    edc box/.style={
        rectangle,
        draw,
        rounded corners=3pt,
        minimum width=2.2cm,
        minimum height=0.8cm,
        align=center,
        font=\small,
        inner sep=4pt,
    },
    % Bulk-core box (red family)
    bulk box/.style={
        edc box,
        fill=red!10,
        draw=edcBulk!70!black,
        text=black,
    },
    % Brane-layer box (green family)
    brane box/.style={
        edc box,
        fill=green!10,
        draw=edcBrane!70!black,
        text=black,
    },
    % 3D output box (blue family)
    output box/.style={
        edc box,
        fill=blue!10,
        draw=edcOutput!70!black,
        text=black,
    },
    % Neutral/process box
    process box/.style={
        edc box,
        fill=gray!10,
        draw=gray!60!black,
        text=black,
    },
    % Label-only box (no background)
    label box/.style={
        rectangle,
        rounded corners=2pt,
        draw=gray!40,
        fill=white,
        inner sep=2pt,
        font=\scriptsize,
    },
}

% ============================================================
% ARROW STYLES
% ============================================================
\tikzset{
    % Standard thick arrow
    edc arrow/.style={
        ->,
        >=stealth,
        thick,
    },
    % Emphasized arrow (for main flow)
    edc flow/.style={
        ->,
        >=stealth,
        very thick,
        line width=1.2pt,
    },
    % Dashed arrow (for optional/weak connections)
    edc dashed/.style={
        ->,
        >=stealth,
        thick,
        dashed,
    },
    % Double arrow (for bidirectional)
    edc bidir/.style={
        <->,
        >=stealth,
        thick,
    },
}

% ============================================================
% REGION STYLES (for background fills)
% ============================================================
\tikzset{
    % Bulk region (5D)
    bulk region/.style={
        fill=blue!8,
    },
    % Brane layer region
    brane region/.style={
        fill=yellow!25,
    },
    % Observer/3D region
    observer region/.style={
        fill=green!8,
    },
}

% ============================================================
% LABEL STYLES
% ============================================================
\tikzset{
    % Phase label (below nodes)
    phase label/.style={
        font=\scriptsize\itshape,
        text=black!70,
    },
    % Equation label (for inline math)
    eq label/.style={
        font=\scriptsize,
        fill=white,
        inner sep=1pt,
    },
    % Section annotation
    section label/.style={
        font=\footnotesize\bfseries,
        text=black,
    },
}

% ============================================================
% JUNCTION/PARTICLE STYLES
% ============================================================
\tikzset{
    % Y-junction point
    junction point/.style={
        circle,
        fill=red!60!black,
        minimum size=4pt,
        inner sep=0pt,
    },
    % Flux tube arm
    flux arm/.style={
        thick,
        blue!60!black,
    },
    % Particle dot (electron, etc.)
    particle/.style={
        circle,
        fill=black,
        minimum size=5pt,
        inner sep=0pt,
    },
    % Neutrino (smaller, gray)
    neutrino/.style={
        circle,
        fill=gray,
        minimum size=4pt,
        inner sep=0pt,
    },
}

% ============================================================
% SPRING DECORATION (for mechanical models)
% ============================================================
\tikzset{
    spring/.style={
        thick,
        decorate,
        decoration={
            coil,
            aspect=0.5,
            segment length=2mm,
            amplitude=2mm,
        },
    },
    % Wave decoration (for field modes)
    wave field/.style={
        thick,
        decorate,
        decoration={
            snake,
            amplitude=2pt,
            segment length=8pt,
        },
    },
}

% ============================================================
% BOUNDARY STYLES
% ============================================================
\tikzset{
    % Bulk-facing boundary (dashed red)
    bulk boundary/.style={
        very thick,
        red!70!black,
        dashed,
    },
    % Observer-facing boundary (solid green)
    observer boundary/.style={
        thick,
        green!50!black,
    },
    % Brane edge (orange)
    brane edge/.style={
        thick,
        orange!70!black,
    },
}

% ============================================================
% CONVENIENCE COMMANDS
% ============================================================
% Arrow label (above)
\newcommand{\arrlabel}[1]{\scriptsize #1}
% Arrow label (below)
\newcommand{\arrlabelb}[1]{\scriptsize #1}

% ============================================================
% END OF STYLE FILE
% ============================================================
  % if using TikZ figures
%
% REQUIRED PACKAGES (load these in main document before \input):
%   fontspec, amsmath, amssymb, amsthm, mathtools, geometry
%   hyperref, enumitem, booktabs, array, xcolor, tcolorbox
%
% ============================================================

% ============================================================
%  EPISTEMIC TAG COLORS
% ============================================================
\definecolor{tagDer}{RGB}{0,128,0}      % Green - Derived
\definecolor{tagDc}{RGB}{0,0,200}       % Blue - Deduced/Constrained
\definecolor{tagCal}{RGB}{200,0,0}      % Red - Calibrated
\definecolor{tagP}{RGB}{128,0,128}      % Purple - Postulated
\definecolor{tagBL}{RGB}{128,128,128}   % Gray - Baseline
\definecolor{tagI}{RGB}{255,140,0}      % Orange - Identified
\definecolor{tagOpen}{RGB}{200,100,0}   % Dark orange - Open

% ============================================================
%  EPISTEMIC TAG COMMANDS
% ============================================================
% Use these to mark claims with their epistemic status
\newcommand{\tagDer}{\textcolor{tagDer}{\textbf{[Der]}}}    % Derived from axioms
\newcommand{\tagDc}{\textcolor{tagDc}{\textbf{[Dc]}}}       % Deduced/Constrained
\newcommand{\tagCal}{\textcolor{tagCal}{\textbf{[Cal]}}}    % Calibrated (fitted)
\newcommand{\tagP}{\textcolor{tagP}{\textbf{[P]}}}          % Postulated
\newcommand{\tagBL}{\textcolor{tagBL}{\textbf{[BL]}}}       % Baseline (external fact)
\newcommand{\tagI}{\textcolor{tagI}{\textbf{[I]}}}          % Identified (pattern match)
\newcommand{\tagOpen}{\textcolor{tagOpen}{\textbf{[OPEN]}}} % Open problem
\newcommand{\tagDef}{\textcolor{tagDc}{\textbf{[Def]}}}     % Definition

% ============================================================
%  THEOREM ENVIRONMENTS
% ============================================================
\newtheorem{postulate}{Postulate}
\newtheorem{definition}{Definition}[section]
\newtheorem{theorem}{Theorem}[section]
\newtheorem{lemma}[theorem]{Lemma}
\newtheorem{corollary}[theorem]{Corollary}
\newtheorem{proposition}[theorem]{Proposition}
\newtheorem{remark}{Remark}[section]

% ============================================================
%  COMMON EDC SYMBOLS
% ============================================================
% Symmetry groups
\newcommand{\Ztwo}{\mathbb{Z}_2}
\newcommand{\Zthree}{\mathbb{Z}_3}
\newcommand{\Ztri}{\mathbb{Z}_3}    % alias
\newcommand{\Zsix}{\mathbb{Z}_6}

% Geometric objects
\newcommand{\Sthree}{S^3}           % 3-sphere
\newcommand{\Stwo}{S^2}             % 2-sphere
\newcommand{\Bthree}{B^3}           % 3-ball
\newcommand{\Mfive}{\mathcal{M}_5}  % 5D manifold
\newcommand{\Bfour}{\mathcal{B}_4}  % 4D brane

% Physical quantities
\newcommand{\tension}{\tau}         % string/flux-tube tension (E/L)
\newcommand{\re}{r_e}               % electron radius

% Operators
\newcommand{\Pfrozen}{\mathcal{P}_{\mathrm{frozen}}}  % Frozen projection operator
\newcommand{\Ebrane}{\mathcal{E}_{\mathrm{brane}}}    % Brane energy store

% Bulk-brane exchange current (canonical notation from Framework v2.0)
\newcommand{\Jbb}[1]{J^{#1}_{\mathrm{bulk}\to\mathrm{brane}}}

% ============================================================
%  TCOLORBOX STYLES FOR EDC PAPERS
% ============================================================
% Cornerstone box (blue) — key claims/foundations
\tcbset{
    edcCornerstone/.style={
        colback=blue!5,
        colframe=blue!40!black,
        fonttitle=\bfseries
    }
}

% Guardrail box (gray) — epistemic warnings/constraints
\tcbset{
    edcGuardrail/.style={
        colback=gray!5!white,
        colframe=gray!60!black,
        fonttitle=\bfseries
    }
}

% PPN box (blue, lighter) — Physical Process Narrative
\tcbset{
    edcPPN/.style={
        colback=blue!5,
        colframe=blue!50!black,
        fonttitle=\bfseries
    }
}

% Canonical box (yellow/orange) — canonical definitions/glossary
\tcbset{
    edcCanonical/.style={
        colback=yellow!5,
        colframe=orange!60!black,
        fonttitle=\bfseries
    }
}

% Conceptual box (yellow/orange, lighter) — conceptual pictures
\tcbset{
    edcConcept/.style={
        colback=yellow!5,
        colframe=orange!50!black,
        fonttitle=\bfseries
    }
}

% Pathway box (purple) — energy pathways, mechanisms
\tcbset{
    edcPathway/.style={
        colback=purple!5,
        colframe=purple!40!black,
        fonttitle=\bfseries
    }
}

% Model box (green) — mechanical analogies, heuristics
\tcbset{
    edcModel/.style={
        colback=green!5,
        colframe=green!40!black,
        fonttitle=\bfseries
    }
}

% Warning box (red) — non-overclaim, limitations
\tcbset{
    edcWarning/.style={
        colback=red!5,
        colframe=red!40!black,
        fonttitle=\bfseries
    }
}

% Framework quote box (gray) — verbatim from Framework v2.0
\tcbset{
    edcFramework/.style={
        colback=gray!5!white,
        colframe=gray!60!black,
        fonttitle=\small
    }
}

% Mechanism box (teal) — mechanistic dimension principle narrative
\tcbset{
    edcMechanism/.style={
        colback=teal!5,
        colframe=teal!50!black,
        fonttitle=\bfseries,
        title={Mechanistic Dimension Note (Canon)}
    }
}

% ============================================================
%  MECHANISTIC DIMENSION HELPER MACRO
% ============================================================
% Usage: \edcMechanismNote{bulk cause}{brane process}{3D output}
%
% Example:
%   \edcMechanismNote{Junction relaxes toward Steiner minimum}%
%                    {Energy pumps into brane-layer modes, redistributes}%
%                    {Electron, antineutrino, proton emerge on 3D side}
%
\newcommand{\edcMechanismNote}[3]{%
\begin{tcolorbox}[edcMechanism]
\begin{itemize}[nosep,leftmargin=*]
    \item \textbf{5D cause (bulk):} #1
    \item \textbf{Brane-layer process:} #2
    \item \textbf{3D observation (output):} #3
\end{itemize}
\vspace{0.3em}
\footnotesize\textit{Ledger closure must hold: bulk + brane + 3D outputs conserve energy/quantum numbers.}
\end{tcolorbox}
}

% ============================================================
%  RELATED DOCUMENTS MACRO
% ============================================================
% Usage: \edcRelatedDocs{main paper title}{main DOI}{companion list}
%
% Example:
%     A: \emph{Effective Lagrangian} (\href{...}{DOI}) $\cdot$
%     B: \emph{WKB Prefactor} (\href{...}{DOI})
%   }

% NOTE: \edcRelatedDocs macro deprecated (DOI registry consolidated)
% Use consolidated Zenodo article as primary reference instead.

% ============================================================
%  DOI REGISTRY DEPRECATED
% ============================================================
% Previous individual DOIs have been deprecated.
% All EDC Weak Sector content is now consolidated into a single
% Zenodo article. See paper_3_series/19_edc_weak_sector_zenodo_article/

% ============================================================
%  PHYSICAL NARRATION RULE REMINDER
% ============================================================
% Every key equation MUST be accompanied by a physical narrative stating:
%   1. 5D cause: What changes in the bulk-core configuration?
%   2. Brane response: How does the brane absorb/redistribute energy?
%   3. 3D observable output: What do observers detect on the 3D side?
%
% This rule eliminates "numerology smell" by ensuring every formula
% has a mechanistic interpretation.

% ============================================================
%  END OF STYLE FILE
% ============================================================

%   % tikz_style_edc.tex — Reusable TikZ styles for EDC papers
% Version 1.0 — 2026-01-20
% Include via: % tikz_style_edc.tex — Reusable TikZ styles for EDC papers
% Version 1.0 — 2026-01-20
% Include via: \input{tikz_style_edc}

% ============================================================
% REQUIRED LIBRARIES (must be loaded in main document)
% ============================================================
% \usetikzlibrary{calc,angles,quotes,decorations.markings,decorations.pathmorphing,positioning}

% ============================================================
% POSITIONING DEFAULTS
% ============================================================
\tikzset{
    % Default node distances for horizontal/vertical layouts
    edc node distance/.style={node distance=1.6cm and 2.0cm},
    % Compact variant for dense diagrams
    edc compact/.style={node distance=1.2cm and 1.5cm},
    % Spread variant for clarity
    edc spread/.style={node distance=2.0cm and 2.5cm},
}

% ============================================================
% COLOR PALETTE (consistent with epistemic tags)
% ============================================================
\definecolor{edcBulk}{RGB}{220,50,50}        % Red tones for bulk/5D
\definecolor{edcBrane}{RGB}{50,150,50}       % Green tones for brane-layer
\definecolor{edcOutput}{RGB}{50,100,200}     % Blue tones for 3D outputs
\definecolor{edcNeutral}{RGB}{100,100,100}   % Gray for neutral/annotations

% ============================================================
% BOX STYLES
% ============================================================
\tikzset{
    % Generic EDC box (base style)
    edc box/.style={
        rectangle,
        draw,
        rounded corners=3pt,
        minimum width=2.2cm,
        minimum height=0.8cm,
        align=center,
        font=\small,
        inner sep=4pt,
    },
    % Bulk-core box (red family)
    bulk box/.style={
        edc box,
        fill=red!10,
        draw=edcBulk!70!black,
        text=black,
    },
    % Brane-layer box (green family)
    brane box/.style={
        edc box,
        fill=green!10,
        draw=edcBrane!70!black,
        text=black,
    },
    % 3D output box (blue family)
    output box/.style={
        edc box,
        fill=blue!10,
        draw=edcOutput!70!black,
        text=black,
    },
    % Neutral/process box
    process box/.style={
        edc box,
        fill=gray!10,
        draw=gray!60!black,
        text=black,
    },
    % Label-only box (no background)
    label box/.style={
        rectangle,
        rounded corners=2pt,
        draw=gray!40,
        fill=white,
        inner sep=2pt,
        font=\scriptsize,
    },
}

% ============================================================
% ARROW STYLES
% ============================================================
\tikzset{
    % Standard thick arrow
    edc arrow/.style={
        ->,
        >=stealth,
        thick,
    },
    % Emphasized arrow (for main flow)
    edc flow/.style={
        ->,
        >=stealth,
        very thick,
        line width=1.2pt,
    },
    % Dashed arrow (for optional/weak connections)
    edc dashed/.style={
        ->,
        >=stealth,
        thick,
        dashed,
    },
    % Double arrow (for bidirectional)
    edc bidir/.style={
        <->,
        >=stealth,
        thick,
    },
}

% ============================================================
% REGION STYLES (for background fills)
% ============================================================
\tikzset{
    % Bulk region (5D)
    bulk region/.style={
        fill=blue!8,
    },
    % Brane layer region
    brane region/.style={
        fill=yellow!25,
    },
    % Observer/3D region
    observer region/.style={
        fill=green!8,
    },
}

% ============================================================
% LABEL STYLES
% ============================================================
\tikzset{
    % Phase label (below nodes)
    phase label/.style={
        font=\scriptsize\itshape,
        text=black!70,
    },
    % Equation label (for inline math)
    eq label/.style={
        font=\scriptsize,
        fill=white,
        inner sep=1pt,
    },
    % Section annotation
    section label/.style={
        font=\footnotesize\bfseries,
        text=black,
    },
}

% ============================================================
% JUNCTION/PARTICLE STYLES
% ============================================================
\tikzset{
    % Y-junction point
    junction point/.style={
        circle,
        fill=red!60!black,
        minimum size=4pt,
        inner sep=0pt,
    },
    % Flux tube arm
    flux arm/.style={
        thick,
        blue!60!black,
    },
    % Particle dot (electron, etc.)
    particle/.style={
        circle,
        fill=black,
        minimum size=5pt,
        inner sep=0pt,
    },
    % Neutrino (smaller, gray)
    neutrino/.style={
        circle,
        fill=gray,
        minimum size=4pt,
        inner sep=0pt,
    },
}

% ============================================================
% SPRING DECORATION (for mechanical models)
% ============================================================
\tikzset{
    spring/.style={
        thick,
        decorate,
        decoration={
            coil,
            aspect=0.5,
            segment length=2mm,
            amplitude=2mm,
        },
    },
    % Wave decoration (for field modes)
    wave field/.style={
        thick,
        decorate,
        decoration={
            snake,
            amplitude=2pt,
            segment length=8pt,
        },
    },
}

% ============================================================
% BOUNDARY STYLES
% ============================================================
\tikzset{
    % Bulk-facing boundary (dashed red)
    bulk boundary/.style={
        very thick,
        red!70!black,
        dashed,
    },
    % Observer-facing boundary (solid green)
    observer boundary/.style={
        thick,
        green!50!black,
    },
    % Brane edge (orange)
    brane edge/.style={
        thick,
        orange!70!black,
    },
}

% ============================================================
% CONVENIENCE COMMANDS
% ============================================================
% Arrow label (above)
\newcommand{\arrlabel}[1]{\scriptsize #1}
% Arrow label (below)
\newcommand{\arrlabelb}[1]{\scriptsize #1}

% ============================================================
% END OF STYLE FILE
% ============================================================


% ============================================================
% REQUIRED LIBRARIES (must be loaded in main document)
% ============================================================
% \usetikzlibrary{calc,angles,quotes,decorations.markings,decorations.pathmorphing,positioning}

% ============================================================
% POSITIONING DEFAULTS
% ============================================================
\tikzset{
    % Default node distances for horizontal/vertical layouts
    edc node distance/.style={node distance=1.6cm and 2.0cm},
    % Compact variant for dense diagrams
    edc compact/.style={node distance=1.2cm and 1.5cm},
    % Spread variant for clarity
    edc spread/.style={node distance=2.0cm and 2.5cm},
}

% ============================================================
% COLOR PALETTE (consistent with epistemic tags)
% ============================================================
\definecolor{edcBulk}{RGB}{220,50,50}        % Red tones for bulk/5D
\definecolor{edcBrane}{RGB}{50,150,50}       % Green tones for brane-layer
\definecolor{edcOutput}{RGB}{50,100,200}     % Blue tones for 3D outputs
\definecolor{edcNeutral}{RGB}{100,100,100}   % Gray for neutral/annotations

% ============================================================
% BOX STYLES
% ============================================================
\tikzset{
    % Generic EDC box (base style)
    edc box/.style={
        rectangle,
        draw,
        rounded corners=3pt,
        minimum width=2.2cm,
        minimum height=0.8cm,
        align=center,
        font=\small,
        inner sep=4pt,
    },
    % Bulk-core box (red family)
    bulk box/.style={
        edc box,
        fill=red!10,
        draw=edcBulk!70!black,
        text=black,
    },
    % Brane-layer box (green family)
    brane box/.style={
        edc box,
        fill=green!10,
        draw=edcBrane!70!black,
        text=black,
    },
    % 3D output box (blue family)
    output box/.style={
        edc box,
        fill=blue!10,
        draw=edcOutput!70!black,
        text=black,
    },
    % Neutral/process box
    process box/.style={
        edc box,
        fill=gray!10,
        draw=gray!60!black,
        text=black,
    },
    % Label-only box (no background)
    label box/.style={
        rectangle,
        rounded corners=2pt,
        draw=gray!40,
        fill=white,
        inner sep=2pt,
        font=\scriptsize,
    },
}

% ============================================================
% ARROW STYLES
% ============================================================
\tikzset{
    % Standard thick arrow
    edc arrow/.style={
        ->,
        >=stealth,
        thick,
    },
    % Emphasized arrow (for main flow)
    edc flow/.style={
        ->,
        >=stealth,
        very thick,
        line width=1.2pt,
    },
    % Dashed arrow (for optional/weak connections)
    edc dashed/.style={
        ->,
        >=stealth,
        thick,
        dashed,
    },
    % Double arrow (for bidirectional)
    edc bidir/.style={
        <->,
        >=stealth,
        thick,
    },
}

% ============================================================
% REGION STYLES (for background fills)
% ============================================================
\tikzset{
    % Bulk region (5D)
    bulk region/.style={
        fill=blue!8,
    },
    % Brane layer region
    brane region/.style={
        fill=yellow!25,
    },
    % Observer/3D region
    observer region/.style={
        fill=green!8,
    },
}

% ============================================================
% LABEL STYLES
% ============================================================
\tikzset{
    % Phase label (below nodes)
    phase label/.style={
        font=\scriptsize\itshape,
        text=black!70,
    },
    % Equation label (for inline math)
    eq label/.style={
        font=\scriptsize,
        fill=white,
        inner sep=1pt,
    },
    % Section annotation
    section label/.style={
        font=\footnotesize\bfseries,
        text=black,
    },
}

% ============================================================
% JUNCTION/PARTICLE STYLES
% ============================================================
\tikzset{
    % Y-junction point
    junction point/.style={
        circle,
        fill=red!60!black,
        minimum size=4pt,
        inner sep=0pt,
    },
    % Flux tube arm
    flux arm/.style={
        thick,
        blue!60!black,
    },
    % Particle dot (electron, etc.)
    particle/.style={
        circle,
        fill=black,
        minimum size=5pt,
        inner sep=0pt,
    },
    % Neutrino (smaller, gray)
    neutrino/.style={
        circle,
        fill=gray,
        minimum size=4pt,
        inner sep=0pt,
    },
}

% ============================================================
% SPRING DECORATION (for mechanical models)
% ============================================================
\tikzset{
    spring/.style={
        thick,
        decorate,
        decoration={
            coil,
            aspect=0.5,
            segment length=2mm,
            amplitude=2mm,
        },
    },
    % Wave decoration (for field modes)
    wave field/.style={
        thick,
        decorate,
        decoration={
            snake,
            amplitude=2pt,
            segment length=8pt,
        },
    },
}

% ============================================================
% BOUNDARY STYLES
% ============================================================
\tikzset{
    % Bulk-facing boundary (dashed red)
    bulk boundary/.style={
        very thick,
        red!70!black,
        dashed,
    },
    % Observer-facing boundary (solid green)
    observer boundary/.style={
        thick,
        green!50!black,
    },
    % Brane edge (orange)
    brane edge/.style={
        thick,
        orange!70!black,
    },
}

% ============================================================
% CONVENIENCE COMMANDS
% ============================================================
% Arrow label (above)
\newcommand{\arrlabel}[1]{\scriptsize #1}
% Arrow label (below)
\newcommand{\arrlabelb}[1]{\scriptsize #1}

% ============================================================
% END OF STYLE FILE
% ============================================================
  % if using TikZ figures
%
% REQUIRED PACKAGES (load these in main document before \input):
%   fontspec, amsmath, amssymb, amsthm, mathtools, geometry
%   hyperref, enumitem, booktabs, array, xcolor, tcolorbox
%
% ============================================================

% ============================================================
%  EPISTEMIC TAG COLORS
% ============================================================
\definecolor{tagDer}{RGB}{0,128,0}      % Green - Derived
\definecolor{tagDc}{RGB}{0,0,200}       % Blue - Deduced/Constrained
\definecolor{tagCal}{RGB}{200,0,0}      % Red - Calibrated
\definecolor{tagP}{RGB}{128,0,128}      % Purple - Postulated
\definecolor{tagBL}{RGB}{128,128,128}   % Gray - Baseline
\definecolor{tagI}{RGB}{255,140,0}      % Orange - Identified
\definecolor{tagOpen}{RGB}{200,100,0}   % Dark orange - Open

% ============================================================
%  EPISTEMIC TAG COMMANDS
% ============================================================
% Use these to mark claims with their epistemic status
\newcommand{\tagDer}{\textcolor{tagDer}{\textbf{[Der]}}}    % Derived from axioms
\newcommand{\tagDc}{\textcolor{tagDc}{\textbf{[Dc]}}}       % Deduced/Constrained
\newcommand{\tagCal}{\textcolor{tagCal}{\textbf{[Cal]}}}    % Calibrated (fitted)
\newcommand{\tagP}{\textcolor{tagP}{\textbf{[P]}}}          % Postulated
\newcommand{\tagBL}{\textcolor{tagBL}{\textbf{[BL]}}}       % Baseline (external fact)
\newcommand{\tagI}{\textcolor{tagI}{\textbf{[I]}}}          % Identified (pattern match)
\newcommand{\tagOpen}{\textcolor{tagOpen}{\textbf{[OPEN]}}} % Open problem
\newcommand{\tagDef}{\textcolor{tagDc}{\textbf{[Def]}}}     % Definition

% ============================================================
%  THEOREM ENVIRONMENTS
% ============================================================
\newtheorem{postulate}{Postulate}
\newtheorem{definition}{Definition}[section]
\newtheorem{theorem}{Theorem}[section]
\newtheorem{lemma}[theorem]{Lemma}
\newtheorem{corollary}[theorem]{Corollary}
\newtheorem{proposition}[theorem]{Proposition}
\newtheorem{remark}{Remark}[section]

% ============================================================
%  COMMON EDC SYMBOLS
% ============================================================
% Symmetry groups
\newcommand{\Ztwo}{\mathbb{Z}_2}
\newcommand{\Zthree}{\mathbb{Z}_3}
\newcommand{\Ztri}{\mathbb{Z}_3}    % alias
\newcommand{\Zsix}{\mathbb{Z}_6}

% Geometric objects
\newcommand{\Sthree}{S^3}           % 3-sphere
\newcommand{\Stwo}{S^2}             % 2-sphere
\newcommand{\Bthree}{B^3}           % 3-ball
\newcommand{\Mfive}{\mathcal{M}_5}  % 5D manifold
\newcommand{\Bfour}{\mathcal{B}_4}  % 4D brane

% Physical quantities
\newcommand{\tension}{\tau}         % string/flux-tube tension (E/L)
\newcommand{\re}{r_e}               % electron radius

% Operators
\newcommand{\Pfrozen}{\mathcal{P}_{\mathrm{frozen}}}  % Frozen projection operator
\newcommand{\Ebrane}{\mathcal{E}_{\mathrm{brane}}}    % Brane energy store

% Bulk-brane exchange current (canonical notation from Framework v2.0)
\newcommand{\Jbb}[1]{J^{#1}_{\mathrm{bulk}\to\mathrm{brane}}}

% ============================================================
%  TCOLORBOX STYLES FOR EDC PAPERS
% ============================================================
% Cornerstone box (blue) — key claims/foundations
\tcbset{
    edcCornerstone/.style={
        colback=blue!5,
        colframe=blue!40!black,
        fonttitle=\bfseries
    }
}

% Guardrail box (gray) — epistemic warnings/constraints
\tcbset{
    edcGuardrail/.style={
        colback=gray!5!white,
        colframe=gray!60!black,
        fonttitle=\bfseries
    }
}

% PPN box (blue, lighter) — Physical Process Narrative
\tcbset{
    edcPPN/.style={
        colback=blue!5,
        colframe=blue!50!black,
        fonttitle=\bfseries
    }
}

% Canonical box (yellow/orange) — canonical definitions/glossary
\tcbset{
    edcCanonical/.style={
        colback=yellow!5,
        colframe=orange!60!black,
        fonttitle=\bfseries
    }
}

% Conceptual box (yellow/orange, lighter) — conceptual pictures
\tcbset{
    edcConcept/.style={
        colback=yellow!5,
        colframe=orange!50!black,
        fonttitle=\bfseries
    }
}

% Pathway box (purple) — energy pathways, mechanisms
\tcbset{
    edcPathway/.style={
        colback=purple!5,
        colframe=purple!40!black,
        fonttitle=\bfseries
    }
}

% Model box (green) — mechanical analogies, heuristics
\tcbset{
    edcModel/.style={
        colback=green!5,
        colframe=green!40!black,
        fonttitle=\bfseries
    }
}

% Warning box (red) — non-overclaim, limitations
\tcbset{
    edcWarning/.style={
        colback=red!5,
        colframe=red!40!black,
        fonttitle=\bfseries
    }
}

% Framework quote box (gray) — verbatim from Framework v2.0
\tcbset{
    edcFramework/.style={
        colback=gray!5!white,
        colframe=gray!60!black,
        fonttitle=\small
    }
}

% Mechanism box (teal) — mechanistic dimension principle narrative
\tcbset{
    edcMechanism/.style={
        colback=teal!5,
        colframe=teal!50!black,
        fonttitle=\bfseries,
        title={Mechanistic Dimension Note (Canon)}
    }
}

% ============================================================
%  MECHANISTIC DIMENSION HELPER MACRO
% ============================================================
% Usage: \edcMechanismNote{bulk cause}{brane process}{3D output}
%
% Example:
%   \edcMechanismNote{Junction relaxes toward Steiner minimum}%
%                    {Energy pumps into brane-layer modes, redistributes}%
%                    {Electron, antineutrino, proton emerge on 3D side}
%
\newcommand{\edcMechanismNote}[3]{%
\begin{tcolorbox}[edcMechanism]
\begin{itemize}[nosep,leftmargin=*]
    \item \textbf{5D cause (bulk):} #1
    \item \textbf{Brane-layer process:} #2
    \item \textbf{3D observation (output):} #3
\end{itemize}
\vspace{0.3em}
\footnotesize\textit{Ledger closure must hold: bulk + brane + 3D outputs conserve energy/quantum numbers.}
\end{tcolorbox}
}

% ============================================================
%  RELATED DOCUMENTS MACRO
% ============================================================
% Usage: \edcRelatedDocs{main paper title}{main DOI}{companion list}
%
% Example:
%     A: \emph{Effective Lagrangian} (\href{...}{DOI}) $\cdot$
%     B: \emph{WKB Prefactor} (\href{...}{DOI})
%   }

% NOTE: \edcRelatedDocs macro deprecated (DOI registry consolidated)
% Use consolidated Zenodo article as primary reference instead.

% ============================================================
%  DOI REGISTRY DEPRECATED
% ============================================================
% Previous individual DOIs have been deprecated.
% All EDC Weak Sector content is now consolidated into a single
% Zenodo article. See paper_3_series/19_edc_weak_sector_zenodo_article/

% ============================================================
%  PHYSICAL NARRATION RULE REMINDER
% ============================================================
% Every key equation MUST be accompanied by a physical narrative stating:
%   1. 5D cause: What changes in the bulk-core configuration?
%   2. Brane response: How does the brane absorb/redistribute energy?
%   3. 3D observable output: What do observers detect on the 3D side?
%
% This rule eliminates "numerology smell" by ensuring every formula
% has a mechanistic interpretation.

% ============================================================
%  END OF STYLE FILE
% ============================================================

% tikz_style_edc.tex — Reusable TikZ styles for EDC papers
% Version 1.0 — 2026-01-20
% Include via: % tikz_style_edc.tex — Reusable TikZ styles for EDC papers
% Version 1.0 — 2026-01-20
% Include via: % tikz_style_edc.tex — Reusable TikZ styles for EDC papers
% Version 1.0 — 2026-01-20
% Include via: \input{tikz_style_edc}

% ============================================================
% REQUIRED LIBRARIES (must be loaded in main document)
% ============================================================
% \usetikzlibrary{calc,angles,quotes,decorations.markings,decorations.pathmorphing,positioning}

% ============================================================
% POSITIONING DEFAULTS
% ============================================================
\tikzset{
    % Default node distances for horizontal/vertical layouts
    edc node distance/.style={node distance=1.6cm and 2.0cm},
    % Compact variant for dense diagrams
    edc compact/.style={node distance=1.2cm and 1.5cm},
    % Spread variant for clarity
    edc spread/.style={node distance=2.0cm and 2.5cm},
}

% ============================================================
% COLOR PALETTE (consistent with epistemic tags)
% ============================================================
\definecolor{edcBulk}{RGB}{220,50,50}        % Red tones for bulk/5D
\definecolor{edcBrane}{RGB}{50,150,50}       % Green tones for brane-layer
\definecolor{edcOutput}{RGB}{50,100,200}     % Blue tones for 3D outputs
\definecolor{edcNeutral}{RGB}{100,100,100}   % Gray for neutral/annotations

% ============================================================
% BOX STYLES
% ============================================================
\tikzset{
    % Generic EDC box (base style)
    edc box/.style={
        rectangle,
        draw,
        rounded corners=3pt,
        minimum width=2.2cm,
        minimum height=0.8cm,
        align=center,
        font=\small,
        inner sep=4pt,
    },
    % Bulk-core box (red family)
    bulk box/.style={
        edc box,
        fill=red!10,
        draw=edcBulk!70!black,
        text=black,
    },
    % Brane-layer box (green family)
    brane box/.style={
        edc box,
        fill=green!10,
        draw=edcBrane!70!black,
        text=black,
    },
    % 3D output box (blue family)
    output box/.style={
        edc box,
        fill=blue!10,
        draw=edcOutput!70!black,
        text=black,
    },
    % Neutral/process box
    process box/.style={
        edc box,
        fill=gray!10,
        draw=gray!60!black,
        text=black,
    },
    % Label-only box (no background)
    label box/.style={
        rectangle,
        rounded corners=2pt,
        draw=gray!40,
        fill=white,
        inner sep=2pt,
        font=\scriptsize,
    },
}

% ============================================================
% ARROW STYLES
% ============================================================
\tikzset{
    % Standard thick arrow
    edc arrow/.style={
        ->,
        >=stealth,
        thick,
    },
    % Emphasized arrow (for main flow)
    edc flow/.style={
        ->,
        >=stealth,
        very thick,
        line width=1.2pt,
    },
    % Dashed arrow (for optional/weak connections)
    edc dashed/.style={
        ->,
        >=stealth,
        thick,
        dashed,
    },
    % Double arrow (for bidirectional)
    edc bidir/.style={
        <->,
        >=stealth,
        thick,
    },
}

% ============================================================
% REGION STYLES (for background fills)
% ============================================================
\tikzset{
    % Bulk region (5D)
    bulk region/.style={
        fill=blue!8,
    },
    % Brane layer region
    brane region/.style={
        fill=yellow!25,
    },
    % Observer/3D region
    observer region/.style={
        fill=green!8,
    },
}

% ============================================================
% LABEL STYLES
% ============================================================
\tikzset{
    % Phase label (below nodes)
    phase label/.style={
        font=\scriptsize\itshape,
        text=black!70,
    },
    % Equation label (for inline math)
    eq label/.style={
        font=\scriptsize,
        fill=white,
        inner sep=1pt,
    },
    % Section annotation
    section label/.style={
        font=\footnotesize\bfseries,
        text=black,
    },
}

% ============================================================
% JUNCTION/PARTICLE STYLES
% ============================================================
\tikzset{
    % Y-junction point
    junction point/.style={
        circle,
        fill=red!60!black,
        minimum size=4pt,
        inner sep=0pt,
    },
    % Flux tube arm
    flux arm/.style={
        thick,
        blue!60!black,
    },
    % Particle dot (electron, etc.)
    particle/.style={
        circle,
        fill=black,
        minimum size=5pt,
        inner sep=0pt,
    },
    % Neutrino (smaller, gray)
    neutrino/.style={
        circle,
        fill=gray,
        minimum size=4pt,
        inner sep=0pt,
    },
}

% ============================================================
% SPRING DECORATION (for mechanical models)
% ============================================================
\tikzset{
    spring/.style={
        thick,
        decorate,
        decoration={
            coil,
            aspect=0.5,
            segment length=2mm,
            amplitude=2mm,
        },
    },
    % Wave decoration (for field modes)
    wave field/.style={
        thick,
        decorate,
        decoration={
            snake,
            amplitude=2pt,
            segment length=8pt,
        },
    },
}

% ============================================================
% BOUNDARY STYLES
% ============================================================
\tikzset{
    % Bulk-facing boundary (dashed red)
    bulk boundary/.style={
        very thick,
        red!70!black,
        dashed,
    },
    % Observer-facing boundary (solid green)
    observer boundary/.style={
        thick,
        green!50!black,
    },
    % Brane edge (orange)
    brane edge/.style={
        thick,
        orange!70!black,
    },
}

% ============================================================
% CONVENIENCE COMMANDS
% ============================================================
% Arrow label (above)
\newcommand{\arrlabel}[1]{\scriptsize #1}
% Arrow label (below)
\newcommand{\arrlabelb}[1]{\scriptsize #1}

% ============================================================
% END OF STYLE FILE
% ============================================================


% ============================================================
% REQUIRED LIBRARIES (must be loaded in main document)
% ============================================================
% \usetikzlibrary{calc,angles,quotes,decorations.markings,decorations.pathmorphing,positioning}

% ============================================================
% POSITIONING DEFAULTS
% ============================================================
\tikzset{
    % Default node distances for horizontal/vertical layouts
    edc node distance/.style={node distance=1.6cm and 2.0cm},
    % Compact variant for dense diagrams
    edc compact/.style={node distance=1.2cm and 1.5cm},
    % Spread variant for clarity
    edc spread/.style={node distance=2.0cm and 2.5cm},
}

% ============================================================
% COLOR PALETTE (consistent with epistemic tags)
% ============================================================
\definecolor{edcBulk}{RGB}{220,50,50}        % Red tones for bulk/5D
\definecolor{edcBrane}{RGB}{50,150,50}       % Green tones for brane-layer
\definecolor{edcOutput}{RGB}{50,100,200}     % Blue tones for 3D outputs
\definecolor{edcNeutral}{RGB}{100,100,100}   % Gray for neutral/annotations

% ============================================================
% BOX STYLES
% ============================================================
\tikzset{
    % Generic EDC box (base style)
    edc box/.style={
        rectangle,
        draw,
        rounded corners=3pt,
        minimum width=2.2cm,
        minimum height=0.8cm,
        align=center,
        font=\small,
        inner sep=4pt,
    },
    % Bulk-core box (red family)
    bulk box/.style={
        edc box,
        fill=red!10,
        draw=edcBulk!70!black,
        text=black,
    },
    % Brane-layer box (green family)
    brane box/.style={
        edc box,
        fill=green!10,
        draw=edcBrane!70!black,
        text=black,
    },
    % 3D output box (blue family)
    output box/.style={
        edc box,
        fill=blue!10,
        draw=edcOutput!70!black,
        text=black,
    },
    % Neutral/process box
    process box/.style={
        edc box,
        fill=gray!10,
        draw=gray!60!black,
        text=black,
    },
    % Label-only box (no background)
    label box/.style={
        rectangle,
        rounded corners=2pt,
        draw=gray!40,
        fill=white,
        inner sep=2pt,
        font=\scriptsize,
    },
}

% ============================================================
% ARROW STYLES
% ============================================================
\tikzset{
    % Standard thick arrow
    edc arrow/.style={
        ->,
        >=stealth,
        thick,
    },
    % Emphasized arrow (for main flow)
    edc flow/.style={
        ->,
        >=stealth,
        very thick,
        line width=1.2pt,
    },
    % Dashed arrow (for optional/weak connections)
    edc dashed/.style={
        ->,
        >=stealth,
        thick,
        dashed,
    },
    % Double arrow (for bidirectional)
    edc bidir/.style={
        <->,
        >=stealth,
        thick,
    },
}

% ============================================================
% REGION STYLES (for background fills)
% ============================================================
\tikzset{
    % Bulk region (5D)
    bulk region/.style={
        fill=blue!8,
    },
    % Brane layer region
    brane region/.style={
        fill=yellow!25,
    },
    % Observer/3D region
    observer region/.style={
        fill=green!8,
    },
}

% ============================================================
% LABEL STYLES
% ============================================================
\tikzset{
    % Phase label (below nodes)
    phase label/.style={
        font=\scriptsize\itshape,
        text=black!70,
    },
    % Equation label (for inline math)
    eq label/.style={
        font=\scriptsize,
        fill=white,
        inner sep=1pt,
    },
    % Section annotation
    section label/.style={
        font=\footnotesize\bfseries,
        text=black,
    },
}

% ============================================================
% JUNCTION/PARTICLE STYLES
% ============================================================
\tikzset{
    % Y-junction point
    junction point/.style={
        circle,
        fill=red!60!black,
        minimum size=4pt,
        inner sep=0pt,
    },
    % Flux tube arm
    flux arm/.style={
        thick,
        blue!60!black,
    },
    % Particle dot (electron, etc.)
    particle/.style={
        circle,
        fill=black,
        minimum size=5pt,
        inner sep=0pt,
    },
    % Neutrino (smaller, gray)
    neutrino/.style={
        circle,
        fill=gray,
        minimum size=4pt,
        inner sep=0pt,
    },
}

% ============================================================
% SPRING DECORATION (for mechanical models)
% ============================================================
\tikzset{
    spring/.style={
        thick,
        decorate,
        decoration={
            coil,
            aspect=0.5,
            segment length=2mm,
            amplitude=2mm,
        },
    },
    % Wave decoration (for field modes)
    wave field/.style={
        thick,
        decorate,
        decoration={
            snake,
            amplitude=2pt,
            segment length=8pt,
        },
    },
}

% ============================================================
% BOUNDARY STYLES
% ============================================================
\tikzset{
    % Bulk-facing boundary (dashed red)
    bulk boundary/.style={
        very thick,
        red!70!black,
        dashed,
    },
    % Observer-facing boundary (solid green)
    observer boundary/.style={
        thick,
        green!50!black,
    },
    % Brane edge (orange)
    brane edge/.style={
        thick,
        orange!70!black,
    },
}

% ============================================================
% CONVENIENCE COMMANDS
% ============================================================
% Arrow label (above)
\newcommand{\arrlabel}[1]{\scriptsize #1}
% Arrow label (below)
\newcommand{\arrlabelb}[1]{\scriptsize #1}

% ============================================================
% END OF STYLE FILE
% ============================================================


% ============================================================
% REQUIRED LIBRARIES (must be loaded in main document)
% ============================================================
% \usetikzlibrary{calc,angles,quotes,decorations.markings,decorations.pathmorphing,positioning}

% ============================================================
% POSITIONING DEFAULTS
% ============================================================
\tikzset{
    % Default node distances for horizontal/vertical layouts
    edc node distance/.style={node distance=1.6cm and 2.0cm},
    % Compact variant for dense diagrams
    edc compact/.style={node distance=1.2cm and 1.5cm},
    % Spread variant for clarity
    edc spread/.style={node distance=2.0cm and 2.5cm},
}

% ============================================================
% COLOR PALETTE (consistent with epistemic tags)
% ============================================================
\definecolor{edcBulk}{RGB}{220,50,50}        % Red tones for bulk/5D
\definecolor{edcBrane}{RGB}{50,150,50}       % Green tones for brane-layer
\definecolor{edcOutput}{RGB}{50,100,200}     % Blue tones for 3D outputs
\definecolor{edcNeutral}{RGB}{100,100,100}   % Gray for neutral/annotations

% ============================================================
% BOX STYLES
% ============================================================
\tikzset{
    % Generic EDC box (base style)
    edc box/.style={
        rectangle,
        draw,
        rounded corners=3pt,
        minimum width=2.2cm,
        minimum height=0.8cm,
        align=center,
        font=\small,
        inner sep=4pt,
    },
    % Bulk-core box (red family)
    bulk box/.style={
        edc box,
        fill=red!10,
        draw=edcBulk!70!black,
        text=black,
    },
    % Brane-layer box (green family)
    brane box/.style={
        edc box,
        fill=green!10,
        draw=edcBrane!70!black,
        text=black,
    },
    % 3D output box (blue family)
    output box/.style={
        edc box,
        fill=blue!10,
        draw=edcOutput!70!black,
        text=black,
    },
    % Neutral/process box
    process box/.style={
        edc box,
        fill=gray!10,
        draw=gray!60!black,
        text=black,
    },
    % Label-only box (no background)
    label box/.style={
        rectangle,
        rounded corners=2pt,
        draw=gray!40,
        fill=white,
        inner sep=2pt,
        font=\scriptsize,
    },
}

% ============================================================
% ARROW STYLES
% ============================================================
\tikzset{
    % Standard thick arrow
    edc arrow/.style={
        ->,
        >=stealth,
        thick,
    },
    % Emphasized arrow (for main flow)
    edc flow/.style={
        ->,
        >=stealth,
        very thick,
        line width=1.2pt,
    },
    % Dashed arrow (for optional/weak connections)
    edc dashed/.style={
        ->,
        >=stealth,
        thick,
        dashed,
    },
    % Double arrow (for bidirectional)
    edc bidir/.style={
        <->,
        >=stealth,
        thick,
    },
}

% ============================================================
% REGION STYLES (for background fills)
% ============================================================
\tikzset{
    % Bulk region (5D)
    bulk region/.style={
        fill=blue!8,
    },
    % Brane layer region
    brane region/.style={
        fill=yellow!25,
    },
    % Observer/3D region
    observer region/.style={
        fill=green!8,
    },
}

% ============================================================
% LABEL STYLES
% ============================================================
\tikzset{
    % Phase label (below nodes)
    phase label/.style={
        font=\scriptsize\itshape,
        text=black!70,
    },
    % Equation label (for inline math)
    eq label/.style={
        font=\scriptsize,
        fill=white,
        inner sep=1pt,
    },
    % Section annotation
    section label/.style={
        font=\footnotesize\bfseries,
        text=black,
    },
}

% ============================================================
% JUNCTION/PARTICLE STYLES
% ============================================================
\tikzset{
    % Y-junction point
    junction point/.style={
        circle,
        fill=red!60!black,
        minimum size=4pt,
        inner sep=0pt,
    },
    % Flux tube arm
    flux arm/.style={
        thick,
        blue!60!black,
    },
    % Particle dot (electron, etc.)
    particle/.style={
        circle,
        fill=black,
        minimum size=5pt,
        inner sep=0pt,
    },
    % Neutrino (smaller, gray)
    neutrino/.style={
        circle,
        fill=gray,
        minimum size=4pt,
        inner sep=0pt,
    },
}

% ============================================================
% SPRING DECORATION (for mechanical models)
% ============================================================
\tikzset{
    spring/.style={
        thick,
        decorate,
        decoration={
            coil,
            aspect=0.5,
            segment length=2mm,
            amplitude=2mm,
        },
    },
    % Wave decoration (for field modes)
    wave field/.style={
        thick,
        decorate,
        decoration={
            snake,
            amplitude=2pt,
            segment length=8pt,
        },
    },
}

% ============================================================
% BOUNDARY STYLES
% ============================================================
\tikzset{
    % Bulk-facing boundary (dashed red)
    bulk boundary/.style={
        very thick,
        red!70!black,
        dashed,
    },
    % Observer-facing boundary (solid green)
    observer boundary/.style={
        thick,
        green!50!black,
    },
    % Brane edge (orange)
    brane edge/.style={
        thick,
        orange!70!black,
    },
}

% ============================================================
% CONVENIENCE COMMANDS
% ============================================================
% Arrow label (above)
\newcommand{\arrlabel}[1]{\scriptsize #1}
% Arrow label (below)
\newcommand{\arrlabelb}[1]{\scriptsize #1}

% ============================================================
% END OF STYLE FILE
% ============================================================


% ============================================================
% DOCUMENT INFO
% ============================================================
\title{\textbf{Companion M: Muon Decay as Thick-Brane Tomography}\\[0.3cm]
\large Elastic Diffusive Cosmology — Paper 3 Series}
\author{EDC Collaboration}
\date{Draft v0.2 — January 2026}

% ============================================================
% BIBLIOGRAPHY (inline for simplicity)
% ============================================================
\usepackage{cite}

% ============================================================
\begin{document}
% ============================================================

\maketitle

\begin{abstract}
We apply the thick-brane microphysics framework developed for neutron decay
to the purely leptonic process $\mu^- \to e^- + \nu_\mu + \bar{\nu}_e$.
Unlike the neutron (a bulk-core junction state), the muon is modeled as a
\emph{brane-dominant excitation}—its primary degrees of freedom reside
within the brane layer itself. This makes muon decay an ideal ``tomographic
test'' of the brane absorption/dissipation/release pipeline without the
complications of baryonic topology. We define the energy bookkeeping ledger,
establish selection rules for allowed outputs, and introduce the
\emph{chiral filter} component of the frozen projection operator. No
numerical fits are performed; the goal is structural consistency.
\end{abstract}

\tableofcontents
\newpage

% ============================================================
\section{Motivation: Why Muon Decay?}
\label{sec:motivation}
% ============================================================

\begin{tcolorbox}[edcCornerstone, title=\textbf{Cornerstone: Muon as Brane Tomography}]
Muon decay ($\mu^- \to e^- + \nu_\mu + \bar{\nu}_e$) is a \emph{purely leptonic}
weak process. It involves no baryonic topology, no quark flavor transitions,
and no hadronic complications. If the thick-brane microphysics pipeline
works for muon decay, this provides strong evidence that the framework is
not merely ``tuned'' to neutron phenomenology.
\end{tcolorbox}

\subsection{Strategic Value}

The neutron decay analysis in \emph{Neutron Lifetime from 5D Membrane Cosmology}
(DOI: \href{https://doi.org/10.5281/zenodo.18262721}{10.5281/zenodo.18262721})
and \emph{Companion N: Neutron Junction Microphysics}
(DOI: \href{https://doi.org/10.5281/zenodo.18315110}{10.5281/zenodo.18315110})
established the bulk$\to$brane$\to$3D pipeline for a \emph{junction state}.
The neutron's complexity (three-arm Y-junction, baryonic topology, quark
flavor change) means multiple mechanisms operate simultaneously.

Muon decay strips away these complications:
\begin{itemize}[nosep]
    \item No bulk-core junction geometry \tagP{}
    \item No baryonic number conservation requirement
    \item Pure brane-layer dynamics \tagP{}
    \item Same final output structure: $e^-$ + neutrinos
\end{itemize}

\textbf{Physical Narration:}
\begin{enumerate}[nosep]
    \item \textbf{5D cause:} The muon, as a brane-dominant excitation, occupies
          an unstable mode within the brane layer.
    \item \textbf{Brane response:} Mode redistribution occurs entirely within
          the brane layer—no bulk junction relaxation.
    \item \textbf{3D output:} The frozen projection emits $e^-$, $\nu_\mu$,
          $\bar{\nu}_e$ as allowed outputs.
\end{enumerate}

\subsection{What This Document Does \emph{Not} Do}

\begin{tcolorbox}[edcGuardrail, title=\textbf{Epistemic Guardrail}]
\textbf{No numerical fits:} We do not attempt to derive $\tau_\mu = 2.197 \times 10^{-6}$~s
from first principles. The muon lifetime is treated as \tagBL{} (baseline).
Our goal is \emph{structural consistency}: showing that the same pipeline
that handles neutron decay can accommodate muon decay without contradiction.
\end{tcolorbox}

% ============================================================
\section{Thick-Brane Pipeline for Brane-Dominant Excitations}
\label{sec:pipeline}
% ============================================================

\subsection{Three-Layer Ontology (Review)}

The thick-brane framework from \emph{Framework v2.0}
(DOI: \href{https://doi.org/10.5281/zenodo.18299085}{10.5281/zenodo.18299085})
defines three layers:

\begin{definition}[Three-Layer Ontology] \tagDef{}
\label{def:three-layer}
\begin{enumerate}[nosep]
    \item \textbf{Bulk-core:} 5D interior, $y < -\delta/2$
    \item \textbf{Brane-layer:} Transition region, $y \in [-\delta/2, +\delta/2]$
    \item \textbf{3D outputs:} Observer-facing boundary, $y = +\delta/2$
\end{enumerate}
where $y$ is the extra-dimensional coordinate and $\delta$ is the brane thickness.
\end{definition}

\subsection{Muon as Brane-Dominant Excitation}

\begin{postulate}[Muon Ontology] \tagP{}
\label{post:muon-ontology}
The muon $\mu^-$ is a \emph{brane-dominant excitation}: a localized,
metastable mode whose primary degrees of freedom reside within the
brane layer, not in the bulk-core.
\end{postulate}

\textbf{Physical Narration:}
\begin{itemize}[nosep]
    \item \textbf{5D cause:} Unlike the neutron (bulk junction displaced from
          Steiner minimum), the muon is a \emph{brane-layer eigenmode} that
          happens to be unstable.
    \item \textbf{Brane response:} The instability triggers mode redistribution
          \emph{within} the brane layer—energy flows to lower-mass modes
          ($e^-$, neutrinos).
    \item \textbf{3D output:} The frozen projection maps these modes to
          observable particles.
\end{itemize}

\begin{figure}[htbp]
\centering
\begin{tikzpicture}[scale=0.9]
    % Background regions
    \fill[bulk region] (-4,-2) rectangle (-1.5,2);
    \fill[brane region] (-1.5,-2) rectangle (1.5,2);
    \fill[observer region] (1.5,-2) rectangle (4,2);

    % Labels
    \node[section label] at (-2.75,2.4) {\textbf{Bulk-Core}};
    \node[section label] at (0,2.4) {\textbf{Brane-Layer}};
    \node[section label] at (2.75,2.4) {\textbf{3D Outputs}};

    % Boundaries
    \draw[bulk boundary] (-1.5,-2) -- (-1.5,2);
    \draw[observer boundary] (1.5,-2) -- (1.5,2);

    % Muon in brane layer (not in bulk!)
    \node[circle, fill=purple!60, minimum size=12pt, inner sep=0pt] (mu) at (0,0) {};
    \node[below=0.15cm of mu, font=\footnotesize] {$\mu^-$};

    % Annotation: muon is brane-localized
    \draw[dashed, purple!60!black, thick] (-1.3,0) -- (1.3,0);
    \node[font=\scriptsize\itshape, text=purple!60!black] at (0,-0.8) {brane-localized mode};

    % Contrast: neutron would have bulk core
    \node[font=\scriptsize, text=gray] at (-2.75,0) {(no bulk core)};

    % Output particles
    \node[particle] (e) at (3,0.8) {};
    \node[neutrino] (nu1) at (3,0) {};
    \node[neutrino] (nu2) at (3,-0.8) {};
    \node[right=0.1cm of e, font=\scriptsize] {$e^-$};
    \node[right=0.1cm of nu1, font=\scriptsize] {$\nu_\mu$};
    \node[right=0.1cm of nu2, font=\scriptsize] {$\bar{\nu}_e$};

    % Flow arrow
    \draw[edc flow, purple!60!black] (0.3,0) -- (1.3,0);
    \node[font=\scriptsize, above] at (0.8,0.1) {$\mathcal{P}_{\mathrm{frozen}}$};
\end{tikzpicture}
\caption{Muon as brane-dominant excitation. Unlike the neutron (bulk junction),
the muon's degrees of freedom are localized within the brane layer itself.
Decay proceeds via internal mode redistribution, not bulk relaxation.}
\label{fig:muon-ontology}
\end{figure}

\subsection{Contrast with Neutron}

\begin{table}[htbp]
\centering
\caption{Neutron vs.\ Muon in thick-brane ontology}
\label{tab:contrast}
\begin{tabular}{lcc}
\toprule
\textbf{Property} & \textbf{Neutron} & \textbf{Muon} \\
\midrule
Primary location & Bulk-core (junction) & Brane-layer \\
Decay trigger & Junction relaxation & Mode instability \\
Baryonic topology & Yes (3-arm junction) & No \\
Quark flavor change & Yes ($d \to u$) & No \\
Output particles & $p + e^- + \bar{\nu}_e$ & $e^- + \nu_\mu + \bar{\nu}_e$ \\
Lifetime & 878.4~s \tagBL{} & $2.197 \times 10^{-6}$~s \tagBL{} \\
\bottomrule
\end{tabular}
\end{table}

% ============================================================
\section{Energy Bookkeeping Ledger}
\label{sec:ledger}
% ============================================================

\begin{tcolorbox}[edcCanonical, title=\textbf{Canonical: Energy Conservation in Muon Decay}]
The total energy released in muon decay must be accounted for:
\begin{equation}
    m_\mu c^2 = E_{e^-} + E_{\nu_\mu} + E_{\bar{\nu}_e} + E_{\mathrm{other}}
    \label{eq:ledger}
\end{equation}
where $E_{\mathrm{other}}$ captures any residual energy (recoil, soft radiation, etc.).
\end{tcolorbox}

\subsection{Ledger Structure}

Following the pattern established in Companion N, we decompose the energy flow:

\begin{definition}[Energy Partition] \tagDef{}
\label{def:partition}
\begin{align}
    \Delta E_{\mathrm{brane}} &= E_{e^-} + E_{\nu_\mu} + E_{\bar{\nu}_e} \tag{released to 3D} \\
    E_{\mathrm{other}} &= E_{\mathrm{soft}} + E_{\mathrm{brane\,residual}} \tag{not frozen}
\end{align}
\end{definition}

\textbf{Physical Narration:}
\begin{enumerate}[nosep]
    \item \textbf{5D cause:} The muon mode (energy $m_\mu c^2$) becomes unstable.
    \item \textbf{Brane response:} Energy redistributes into brane-layer modes
          compatible with the frozen projection.
    \item \textbf{3D output:} Modes satisfying the frozen criterion
          ($\hbar\omega \gg E_{\mathrm{env}}$) project to observable particles.
\end{enumerate}

\subsection{Key Difference from Neutron}

In neutron decay, there is a \emph{bulk-core contribution}:
\[
    \Delta E_{\mathrm{bulk}} \to \Delta E_{\mathrm{brane}} + E_{\mathrm{residual}}
\]
For the muon, this bulk term is absent \tagP{}:
\[
    \Delta E_{\mu\,\mathrm{mode}} \to \Delta E_{\mathrm{brane}} + E_{\mathrm{other}}
\]
The muon's energy is already ``in'' the brane layer—no bulk$\to$brane pumping
is required.

\begin{tcolorbox}[edcGuardrail, title=\textbf{Epistemic Guardrail}]
\textbf{Suppressed bulk leakage:} In muon decay, we postulate \tagP{} that
leakage of energy back into the bulk-core is suppressed by the muon's
brane-localized mode structure. At leading order, bulk leakage is
treated as negligible. This is consistent with the muon being brane-dominant:
its degrees of freedom are localized within the brane layer.
\end{tcolorbox}

% ============================================================
\section{Selection Rules and Allowed Outputs}
\label{sec:selection}
% ============================================================

\subsection{Why $e^- + \nu_\mu + \bar{\nu}_e$?}

The observed final state of muon decay is:
\[
    \mu^- \to e^- + \nu_\mu + \bar{\nu}_e
\]
This is the dominant decay channel \tagBL{} \cite{PDG2024}.
Rare radiative modes (e.g., $\mu \to e\nu\bar{\nu}\gamma$) exist but
are suppressed below $\mathcal{O}(10^{-2})$ of the total width.

In the thick-brane framework, this is not arbitrary: it is the
\emph{only allowed output configuration} satisfying:
\begin{enumerate}[nosep]
    \item \textbf{Charge conservation:} $Q_{\mu} = Q_{e} = -1$
    \item \textbf{Lepton number conservation:} $L_\mu = 1$, $L_e = 0$ initially;
          final state has $L_\mu = 1$ (via $\nu_\mu$) and $L_e = 0$
          (via $e^- + \bar{\nu}_e$ pair)
    \item \textbf{Energy threshold:} $m_\mu > m_e$ (no other charged leptons allowed)
    \item \textbf{Frozen projection compatibility:} All outputs must satisfy
          the frozen criterion $\hbar\omega \gg E_{\mathrm{env}}$
\end{enumerate}

\begin{definition}[Allowed Output Set] \tagDef{}/\tagDc{}
\label{def:allowed}
The allowed output set for muon decay is:
\[
    \mathcal{A}_\mu = \{e^-, \nu_\mu, \bar{\nu}_e\}
\]
Any other configuration (e.g., $\mu^- \to e^- + \gamma$, $\mu^- \to e^- + e^+ + e^-$)
is either forbidden by conservation laws or suppressed below $10^{-12}$ \tagBL{}.
\end{definition}

\textbf{Physical Narration:}
\begin{itemize}[nosep]
    \item \textbf{5D cause:} Muon mode instability initiates redistribution.
    \item \textbf{Brane response:} Only mode combinations in $\mathcal{A}_\mu$
          can form—the brane layer's mode spectrum constrains the possibilities.
    \item \textbf{3D output:} The frozen projection maps these to $e^-$,
          $\nu_\mu$, $\bar{\nu}_e$.
\end{itemize}

\subsection{Rare and Forbidden Channels}

\begin{table}[htbp]
\centering
\caption{Rare/forbidden muon decay channels: experimental status vs.\ EDC interpretation}
\label{tab:forbidden}
\begin{tabular}{lccc}
\toprule
\textbf{Channel} & \textbf{Exp.\ limit} & \textbf{Status} & \textbf{EDC interpretation} \\
\midrule
$\mu^- \to e^- + \gamma$ & $< 4.2 \times 10^{-13}$ \cite{PDG2024} & \tagBL{} & LFV; selection rule violation \tagP{}/\tagOpen{} \\
$\mu^- \to e^- + e^+ + e^-$ & $< 1.0 \times 10^{-12}$ \cite{PDG2024} & \tagBL{} & Mode mismatch hypothesis \tagP{}/\tagOpen{} \\
$\mu^- \to e^- + \nu_e + \bar{\nu}_\mu$ & Not observed & \tagBL{} & Wrong lepton numbers \tagDc{} \\
\bottomrule
\end{tabular}
\end{table}

\textbf{Note:} The EDC interpretations for LFV channels are hypotheses \tagP{}/\tagOpen{};
we propose that these channels violate selection rules emergent from the brane mode
spectrum, but the precise mechanism remains to be derived.

% ============================================================
\section{Frozen Projection as Chiral Filter}
\label{sec:chiral}
% ============================================================

\subsection{Refined Projection Operator}

The frozen projection operator, introduced in Companion N, maps brane-layer
modes to 3D observable particles. For weak processes, this operator must
account for the observed \emph{chirality/helicity selection} in the outputs.

\begin{definition}[Chiral Filter Decomposition] \tagDef{}/\tagP{}
\label{def:chiral}
The frozen projection operator decomposes as:
\begin{equation}
    \mathcal{P}_{\mathrm{frozen}} = \mathcal{P}_{\mathrm{energy}} \circ
    \mathcal{P}_{\mathrm{mode}} \circ \mathcal{P}_{\mathrm{chir}}
    \label{eq:chiral-decomposition}
\end{equation}
where:
\begin{itemize}[nosep]
    \item $\mathcal{P}_{\mathrm{energy}}$: Selects modes with $\hbar\omega \gg E_{\mathrm{env}}$ (frozen criterion)
    \item $\mathcal{P}_{\mathrm{mode}}$: Selects modes compatible with allowed output set $\mathcal{A}$
    \item $\mathcal{P}_{\mathrm{chir}}$: Selects preferred chirality/helicity channel
\end{itemize}
\end{definition}

\begin{tcolorbox}[edcPathway, title=\textbf{Chiral Filter Mechanism (Hypothesis)}]
\textbf{Physical picture:} The observer-facing boundary of the brane
($y = +\delta/2$) has a preferred orientation defined by its normal
vector $\hat{n}$. This orientation, combined with the brane-layer's
mode spectrum, may act as a \emph{chiral filter}: modes of one handedness
could couple more strongly to the 3D projection than the other.

\medskip
\textbf{Hypothesis} \tagP{}/\tagOpen{}\textbf{:} We propose that the weak interaction's
``maximal parity violation'' is \emph{consistent with} geometric asymmetry
of the brane boundary, rather than requiring an intrinsic vertex asymmetry.
The derivation of $\mathcal{P}_{\mathrm{chir}}$ from the 5D action and
boundary conditions remains \tagOpen{}.
\end{tcolorbox}

\textbf{Physical Narration:}
\begin{enumerate}[nosep]
    \item \textbf{5D cause:} The brane has two faces: bulk-facing and observer-facing.
    \item \textbf{Brane response:} Modes in the brane layer can be decomposed
          into chirality eigenstates. The boundary conditions at $y = +\delta/2$
          favor one chirality.
    \item \textbf{3D output:} Observers detect only the ``passed'' chirality—the
          ``blocked'' component remains in the brane or is reflected.
\end{enumerate}

\subsection{Chirality Selection in Muon Decay}

In Standard Model language, muon decay produces:
\begin{itemize}[nosep]
    \item $e^-$: Left-handed (predominantly)
    \item $\nu_\mu$: Left-handed
    \item $\bar{\nu}_e$: Right-handed
\end{itemize}

In the EDC framework, this pattern arises from $\mathcal{P}_{\mathrm{chir}}$:

\begin{equation}
    \mathcal{P}_{\mathrm{chir}}: \phi_{\mathrm{brane}} \mapsto
    \begin{cases}
        \phi_L & \text{(neutrinos)} \\
        \bar{\phi}_R & \text{(antineutrinos)} \\
        \text{mixed} & \text{(charged leptons, mass-dependent)}
    \end{cases}
    \label{eq:chiral-selection}
\end{equation}

\begin{tcolorbox}[edcWarning, title=\textbf{Non-Overclaim Reminder}]
\textbf{Status:} The chiral filter mechanism is \tagP{}/\tagOpen{}.
We have not derived the specific form of $\mathcal{P}_{\mathrm{chir}}$ from
the 5D action. The claim is that such an operator \emph{exists} and
\emph{can be geometric in origin}—not that we have constructed it explicitly.
\end{tcolorbox}

\subsection{Formal Sketch: Boundary as Chiral Projector}
\label{subsec:chiral-sketch}

We outline how boundary conditions at $y = +\delta/2$ could induce chirality
selection \tagP{}/\tagOpen{}. This is a \emph{sketch}, not a derivation.

\textbf{Setup:} Let $\hat{n}$ be the outward normal to the observer-facing
boundary. A brane-layer spinor field $\psi(x,y)$ satisfies boundary conditions
at $y = +\delta/2$ of the schematic form:
\begin{equation}
    (1 - i\gamma^5 \hat{n}\cdot\gamma)\,\psi\big|_{y=+\delta/2} = 0
    \label{eq:bc-sketch}
\end{equation}
This type of condition (analogous to MIT bag boundary conditions) projects
out one chirality component at the boundary.

\textbf{Consequence:} Modes that ``pass through'' to the 3D side satisfy
a chirality constraint imposed by the boundary geometry, not by the bulk
Lagrangian. The normal vector $\hat{n}$ breaks parity at the boundary.

\textbf{Status:} Deriving Eq.~\eqref{eq:bc-sketch} from the full 5D action
with thick-brane profile remains \tagOpen{}. The above is a plausibility
argument, not a proof.

% ============================================================
% FIGURE: Energy Flow Diagram
% ============================================================

\begin{figure}[htbp]
\centering
\begin{tikzpicture}[edc compact, scale=0.88]

% Nodes
\node[brane box, minimum width=2.5cm] (mu) at (0,0) {$\mu^-$ mode\\(brane-layer)};
\node[process box, right=1.8cm of mu] (abs) {Absorption/\\Redistribution};
\node[process box, right=1.8cm of abs] (diss) {Dissipation/\\Mode population};
\node[output box, right=1.8cm of diss, minimum width=2.2cm] (out) {3D Outputs\\$e^-, \nu_\mu, \bar{\nu}_e$};

% Arrows with labels
\draw[edc flow, purple!60!black] (mu) -- node[above, font=\scriptsize] {instability} (abs);
\draw[edc flow, green!50!black] (abs) -- node[above, font=\scriptsize] {$\Gamma_{\mathrm{eff}}$} (diss);
\draw[edc flow, blue!60!black] (diss) -- node[above, font=\scriptsize] {$\mathcal{P}_{\mathrm{frozen}}$} (out);

% Phase labels below
\node[phase label, below=0.4cm of mu] {Initial state};
\node[phase label, below=0.4cm of abs] {Charging};
\node[phase label, below=0.4cm of diss] {Mode spectrum};
\node[phase label, below=0.4cm of out] {Observation};

% Chiral filter annotation
\draw[dashed, red!60!black] ($(diss.east)!0.5!(out.west)$) ++(0,-0.8) -- ++(0,1.6);
\node[font=\scriptsize, text=red!60!black, below] at ($(diss.east)!0.5!(out.west) + (0,-1.0)$) {$\mathcal{P}_{\mathrm{chir}}$};

% Ledger closure annotation
\node[label box, below=1.5cm of abs] (ledger) {\footnotesize Ledger: $m_\mu c^2 = E_e + E_{\nu_\mu} + E_{\bar{\nu}_e} + E_{\mathrm{other}}$};
\draw[dashed, gray] (ledger.west) -- ++(-1,0);
\draw[dashed, gray] (ledger.east) -- ++(1,0);

\end{tikzpicture}
\caption{Energy flow in muon decay. The muon (brane-dominant mode) undergoes
internal redistribution within the brane layer. The chiral filter
$\mathcal{P}_{\mathrm{chir}}$ selects the helicity of outputs at the
observer-facing boundary.}
\label{fig:energy-flow}
\end{figure}

% ============================================================
\section{Falsifiability Guardrails}
\label{sec:falsifiability}
% ============================================================

\begin{tcolorbox}[edcGuardrail, title=\textbf{Falsifiability: What Would Break This Model?}]
The thick-brane tomography picture for muon decay makes the following
structural predictions. Violation of any would require fundamental revision:

\begin{enumerate}[nosep]
    \item \textbf{Ledger must close:} If energy accounting shows a deficit
          that cannot be attributed to $E_{\mathrm{other}}$ (soft radiation,
          brane residual), the model fails.

    \item \textbf{Only allowed outputs:} If muon decay produced particles
          outside $\mathcal{A}_\mu = \{e^-, \nu_\mu, \bar{\nu}_e\}$ at
          observable rates, the selection rule mechanism would be falsified.

    \item \textbf{Chirality must be geometric:} If the chirality pattern
          could not be explained by boundary geometry (i.e., required
          intrinsic vertex asymmetry), the chiral filter picture fails.

    \item \textbf{No bulk escape:} If evidence emerged that muon decay
          deposits energy into bulk modes (not brane or 3D), the
          brane-dominant ontology would be falsified.

    \item \textbf{Same pipeline, different particle:} If a structurally
          similar process (e.g., $\tau$ decay) could \emph{not} be
          accommodated by the same framework, the model loses generality.

    \item \textbf{Lifetime must remain [BL]:} If the model claimed to
          \emph{predict} $\tau_\mu$ from geometric parameters alone
          (without fitting), and the prediction disagreed with experiment,
          the quantitative mapping would be falsified.
\end{enumerate}
\end{tcolorbox}

\subsection{What We Do \emph{Not} Claim}

\begin{tcolorbox}[edcWarning, title=\textbf{Non-Overclaim Reminder}]
\begin{itemize}[nosep]
    \item $\tau_\mu = 2.197 \times 10^{-6}$~s is \tagBL{} (baseline, not derived)
    \item The precise form of $\mathcal{P}_{\mathrm{chir}}$ is \tagOpen{}
    \item The brane-dominant ontology for muon is \tagP{} (postulated, not derived)
    \item The relationship between $\Gamma_{\mathrm{eff}}$ and microphysics is \tagOpen{}
\end{itemize}
\end{tcolorbox}

% ============================================================
\section{Conclusion}
\label{sec:conclusion}
% ============================================================

Muon decay provides a ``clean room'' test of thick-brane microphysics:
\begin{itemize}[nosep]
    \item \textbf{Same pipeline:} Absorption$\to$Dissipation$\to$Release
    \item \textbf{Simpler ontology:} Brane-dominant (no bulk junction)
    \item \textbf{Same outputs:} $e^-$ + neutrinos (different flavor structure)
    \item \textbf{New element:} Chiral filter as geometric mechanism
\end{itemize}

The fact that the same framework accommodates both neutron (bulk junction)
and muon (brane-dominant) decay without contradiction is a non-trivial
consistency check. The chiral filter mechanism, while still \tagOpen{},
provides a geometric interpretation of weak interaction parity violation
that does not invoke intrinsic vertex asymmetry.

\subsection{Next Steps}

\begin{enumerate}[nosep]
    \item \textbf{Tau decay:} Apply same pipeline to $\tau \to \ell + \nu + \bar{\nu}$
    \item \textbf{Pion decay:} Test $\pi^+ \to \mu^+ + \nu_\mu$ (hadron$\to$lepton)
    \item \textbf{Chiral filter derivation:} Attempt to derive $\mathcal{P}_{\mathrm{chir}}$
          from boundary conditions at $y = +\delta/2$
\end{enumerate}

% ============================================================
% RELATED DOCUMENTS
% ============================================================

\vfill
\edcRelatedDocs{Neutron Lifetime from 5D Membrane Cosmology}{10.5281/zenodo.18262721}{%
    N: \emph{Neutron Junction} (\href{https://doi.org/10.5281/zenodo.18315110}{DOI}) $\cdot$
    H: \emph{Weak Interactions} (\href{https://doi.org/10.5281/zenodo.18307539}{DOI}) $\cdot$
    F: \emph{Proton Junction} (\href{https://doi.org/10.5281/zenodo.18302953}{DOI})
}

% ============================================================
% REFERENCES
% ============================================================
\begin{thebibliography}{9}

\bibitem{PDG2024}
R.~L.~Workman \emph{et al.} (Particle Data Group),
``Review of Particle Physics,''
Prog.\ Theor.\ Exp.\ Phys.\ \textbf{2022}, 083C01 (2022 and 2024 update).
\href{https://pdg.lbl.gov}{https://pdg.lbl.gov}.
Muon properties: $\tau_\mu = 2.1969811(22) \times 10^{-6}$~s;
$\mathrm{BR}(\mu \to e\nu\bar{\nu}) \approx 100\%$;
$\mathrm{BR}(\mu \to e\gamma) < 4.2 \times 10^{-13}$;
$\mathrm{BR}(\mu \to eee) < 1.0 \times 10^{-12}$.

\end{thebibliography}

% ============================================================
\end{document}
% ============================================================
