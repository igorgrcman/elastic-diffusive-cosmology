% Companion T: Tau Decay as a Brane Mode-Spectrum Test
% Version 0.1 — 2026-01-20
% Build: xelatex main.tex && xelatex main.tex

\documentclass[11pt,a4paper]{article}

% ============================================================
% PACKAGES
% ============================================================
\usepackage{fontspec}
\IfFontExistsTF{TeX Gyre Termes}{%
  \setmainfont{TeX Gyre Termes}
}{%
  \setmainfont{Times New Roman}[Ligatures=TeX]
}
\usepackage{amsmath,amssymb,amsthm,mathtools}
\usepackage{geometry}
\geometry{margin=2.5cm}
\usepackage{hyperref}
\hypersetup{colorlinks=true,linkcolor=blue!60!black,citecolor=green!50!black,urlcolor=blue!70!black}
\usepackage{enumitem}
\usepackage{booktabs}
\usepackage{array}
\usepackage{xcolor}
\usepackage{tcolorbox}
\tcbuselibrary{breakable}
\usepackage{cite}

% TikZ
\usepackage{tikz}
\usetikzlibrary{calc,angles,quotes,decorations.markings,decorations.pathmorphing,positioning}

% ============================================================
% SHARED EDC STYLE FILES
% ============================================================
% edc_style.tex — Canonical EDC Paper Style for Paper 3 Series
% Version 1.0 — 2026-01-20
%
% USAGE: Include in preamble AFTER loading packages but BEFORE \begin{document}
%   % edc_style.tex — Canonical EDC Paper Style for Paper 3 Series
% Version 1.0 — 2026-01-20
%
% USAGE: Include in preamble AFTER loading packages but BEFORE \begin{document}
%   % edc_style.tex — Canonical EDC Paper Style for Paper 3 Series
% Version 1.0 — 2026-01-20
%
% USAGE: Include in preamble AFTER loading packages but BEFORE \begin{document}
%   \input{../_shared/style/edc_style}
%   \input{../_shared/style/tikz_style_edc}  % if using TikZ figures
%
% REQUIRED PACKAGES (load these in main document before \input):
%   fontspec, amsmath, amssymb, amsthm, mathtools, geometry
%   hyperref, enumitem, booktabs, array, xcolor, tcolorbox
%
% ============================================================

% ============================================================
%  EPISTEMIC TAG COLORS
% ============================================================
\definecolor{tagDer}{RGB}{0,128,0}      % Green - Derived
\definecolor{tagDc}{RGB}{0,0,200}       % Blue - Deduced/Constrained
\definecolor{tagCal}{RGB}{200,0,0}      % Red - Calibrated
\definecolor{tagP}{RGB}{128,0,128}      % Purple - Postulated
\definecolor{tagBL}{RGB}{128,128,128}   % Gray - Baseline
\definecolor{tagI}{RGB}{255,140,0}      % Orange - Identified
\definecolor{tagOpen}{RGB}{200,100,0}   % Dark orange - Open

% ============================================================
%  EPISTEMIC TAG COMMANDS
% ============================================================
% Use these to mark claims with their epistemic status
\newcommand{\tagDer}{\textcolor{tagDer}{\textbf{[Der]}}}    % Derived from axioms
\newcommand{\tagDc}{\textcolor{tagDc}{\textbf{[Dc]}}}       % Deduced/Constrained
\newcommand{\tagCal}{\textcolor{tagCal}{\textbf{[Cal]}}}    % Calibrated (fitted)
\newcommand{\tagP}{\textcolor{tagP}{\textbf{[P]}}}          % Postulated
\newcommand{\tagBL}{\textcolor{tagBL}{\textbf{[BL]}}}       % Baseline (external fact)
\newcommand{\tagI}{\textcolor{tagI}{\textbf{[I]}}}          % Identified (pattern match)
\newcommand{\tagOpen}{\textcolor{tagOpen}{\textbf{[OPEN]}}} % Open problem
\newcommand{\tagDef}{\textcolor{tagDc}{\textbf{[Def]}}}     % Definition

% ============================================================
%  THEOREM ENVIRONMENTS
% ============================================================
\newtheorem{postulate}{Postulate}
\newtheorem{definition}{Definition}[section]
\newtheorem{theorem}{Theorem}[section]
\newtheorem{lemma}[theorem]{Lemma}
\newtheorem{corollary}[theorem]{Corollary}
\newtheorem{proposition}[theorem]{Proposition}
\newtheorem{remark}{Remark}[section]

% ============================================================
%  COMMON EDC SYMBOLS
% ============================================================
% Symmetry groups
\newcommand{\Ztwo}{\mathbb{Z}_2}
\newcommand{\Zthree}{\mathbb{Z}_3}
\newcommand{\Ztri}{\mathbb{Z}_3}    % alias
\newcommand{\Zsix}{\mathbb{Z}_6}

% Geometric objects
\newcommand{\Sthree}{S^3}           % 3-sphere
\newcommand{\Stwo}{S^2}             % 2-sphere
\newcommand{\Bthree}{B^3}           % 3-ball
\newcommand{\Mfive}{\mathcal{M}_5}  % 5D manifold
\newcommand{\Bfour}{\mathcal{B}_4}  % 4D brane

% Physical quantities
\newcommand{\tension}{\tau}         % string/flux-tube tension (E/L)
\newcommand{\re}{r_e}               % electron radius

% Operators
\newcommand{\Pfrozen}{\mathcal{P}_{\mathrm{frozen}}}  % Frozen projection operator
\newcommand{\Ebrane}{\mathcal{E}_{\mathrm{brane}}}    % Brane energy store

% Bulk-brane exchange current (canonical notation from Framework v2.0)
\newcommand{\Jbb}[1]{J^{#1}_{\mathrm{bulk}\to\mathrm{brane}}}

% ============================================================
%  TCOLORBOX STYLES FOR EDC PAPERS
% ============================================================
% Cornerstone box (blue) — key claims/foundations
\tcbset{
    edcCornerstone/.style={
        colback=blue!5,
        colframe=blue!40!black,
        fonttitle=\bfseries
    }
}

% Guardrail box (gray) — epistemic warnings/constraints
\tcbset{
    edcGuardrail/.style={
        colback=gray!5!white,
        colframe=gray!60!black,
        fonttitle=\bfseries
    }
}

% PPN box (blue, lighter) — Physical Process Narrative
\tcbset{
    edcPPN/.style={
        colback=blue!5,
        colframe=blue!50!black,
        fonttitle=\bfseries
    }
}

% Canonical box (yellow/orange) — canonical definitions/glossary
\tcbset{
    edcCanonical/.style={
        colback=yellow!5,
        colframe=orange!60!black,
        fonttitle=\bfseries
    }
}

% Conceptual box (yellow/orange, lighter) — conceptual pictures
\tcbset{
    edcConcept/.style={
        colback=yellow!5,
        colframe=orange!50!black,
        fonttitle=\bfseries
    }
}

% Pathway box (purple) — energy pathways, mechanisms
\tcbset{
    edcPathway/.style={
        colback=purple!5,
        colframe=purple!40!black,
        fonttitle=\bfseries
    }
}

% Model box (green) — mechanical analogies, heuristics
\tcbset{
    edcModel/.style={
        colback=green!5,
        colframe=green!40!black,
        fonttitle=\bfseries
    }
}

% Warning box (red) — non-overclaim, limitations
\tcbset{
    edcWarning/.style={
        colback=red!5,
        colframe=red!40!black,
        fonttitle=\bfseries
    }
}

% Framework quote box (gray) — verbatim from Framework v2.0
\tcbset{
    edcFramework/.style={
        colback=gray!5!white,
        colframe=gray!60!black,
        fonttitle=\small
    }
}

% Mechanism box (teal) — mechanistic dimension principle narrative
\tcbset{
    edcMechanism/.style={
        colback=teal!5,
        colframe=teal!50!black,
        fonttitle=\bfseries,
        title={Mechanistic Dimension Note (Canon)}
    }
}

% ============================================================
%  MECHANISTIC DIMENSION HELPER MACRO
% ============================================================
% Usage: \edcMechanismNote{bulk cause}{brane process}{3D output}
%
% Example:
%   \edcMechanismNote{Junction relaxes toward Steiner minimum}%
%                    {Energy pumps into brane-layer modes, redistributes}%
%                    {Electron, antineutrino, proton emerge on 3D side}
%
\newcommand{\edcMechanismNote}[3]{%
\begin{tcolorbox}[edcMechanism]
\begin{itemize}[nosep,leftmargin=*]
    \item \textbf{5D cause (bulk):} #1
    \item \textbf{Brane-layer process:} #2
    \item \textbf{3D observation (output):} #3
\end{itemize}
\vspace{0.3em}
\footnotesize\textit{Ledger closure must hold: bulk + brane + 3D outputs conserve energy/quantum numbers.}
\end{tcolorbox}
}

% ============================================================
%  RELATED DOCUMENTS MACRO
% ============================================================
% Usage: \edcRelatedDocs{main paper title}{main DOI}{companion list}
%
% Example:
%     A: \emph{Effective Lagrangian} (\href{...}{DOI}) $\cdot$
%     B: \emph{WKB Prefactor} (\href{...}{DOI})
%   }

% NOTE: \edcRelatedDocs macro deprecated (DOI registry consolidated)
% Use consolidated Zenodo article as primary reference instead.

% ============================================================
%  DOI REGISTRY DEPRECATED
% ============================================================
% Previous individual DOIs have been deprecated.
% All EDC Weak Sector content is now consolidated into a single
% Zenodo article. See paper_3_series/19_edc_weak_sector_zenodo_article/

% ============================================================
%  PHYSICAL NARRATION RULE REMINDER
% ============================================================
% Every key equation MUST be accompanied by a physical narrative stating:
%   1. 5D cause: What changes in the bulk-core configuration?
%   2. Brane response: How does the brane absorb/redistribute energy?
%   3. 3D observable output: What do observers detect on the 3D side?
%
% This rule eliminates "numerology smell" by ensuring every formula
% has a mechanistic interpretation.

% ============================================================
%  END OF STYLE FILE
% ============================================================

%   % tikz_style_edc.tex — Reusable TikZ styles for EDC papers
% Version 1.0 — 2026-01-20
% Include via: \input{tikz_style_edc}

% ============================================================
% REQUIRED LIBRARIES (must be loaded in main document)
% ============================================================
% \usetikzlibrary{calc,angles,quotes,decorations.markings,decorations.pathmorphing,positioning}

% ============================================================
% POSITIONING DEFAULTS
% ============================================================
\tikzset{
    % Default node distances for horizontal/vertical layouts
    edc node distance/.style={node distance=1.6cm and 2.0cm},
    % Compact variant for dense diagrams
    edc compact/.style={node distance=1.2cm and 1.5cm},
    % Spread variant for clarity
    edc spread/.style={node distance=2.0cm and 2.5cm},
}

% ============================================================
% COLOR PALETTE (consistent with epistemic tags)
% ============================================================
\definecolor{edcBulk}{RGB}{220,50,50}        % Red tones for bulk/5D
\definecolor{edcBrane}{RGB}{50,150,50}       % Green tones for brane-layer
\definecolor{edcOutput}{RGB}{50,100,200}     % Blue tones for 3D outputs
\definecolor{edcNeutral}{RGB}{100,100,100}   % Gray for neutral/annotations

% ============================================================
% BOX STYLES
% ============================================================
\tikzset{
    % Generic EDC box (base style)
    edc box/.style={
        rectangle,
        draw,
        rounded corners=3pt,
        minimum width=2.2cm,
        minimum height=0.8cm,
        align=center,
        font=\small,
        inner sep=4pt,
    },
    % Bulk-core box (red family)
    bulk box/.style={
        edc box,
        fill=red!10,
        draw=edcBulk!70!black,
        text=black,
    },
    % Brane-layer box (green family)
    brane box/.style={
        edc box,
        fill=green!10,
        draw=edcBrane!70!black,
        text=black,
    },
    % 3D output box (blue family)
    output box/.style={
        edc box,
        fill=blue!10,
        draw=edcOutput!70!black,
        text=black,
    },
    % Neutral/process box
    process box/.style={
        edc box,
        fill=gray!10,
        draw=gray!60!black,
        text=black,
    },
    % Label-only box (no background)
    label box/.style={
        rectangle,
        rounded corners=2pt,
        draw=gray!40,
        fill=white,
        inner sep=2pt,
        font=\scriptsize,
    },
}

% ============================================================
% ARROW STYLES
% ============================================================
\tikzset{
    % Standard thick arrow
    edc arrow/.style={
        ->,
        >=stealth,
        thick,
    },
    % Emphasized arrow (for main flow)
    edc flow/.style={
        ->,
        >=stealth,
        very thick,
        line width=1.2pt,
    },
    % Dashed arrow (for optional/weak connections)
    edc dashed/.style={
        ->,
        >=stealth,
        thick,
        dashed,
    },
    % Double arrow (for bidirectional)
    edc bidir/.style={
        <->,
        >=stealth,
        thick,
    },
}

% ============================================================
% REGION STYLES (for background fills)
% ============================================================
\tikzset{
    % Bulk region (5D)
    bulk region/.style={
        fill=blue!8,
    },
    % Brane layer region
    brane region/.style={
        fill=yellow!25,
    },
    % Observer/3D region
    observer region/.style={
        fill=green!8,
    },
}

% ============================================================
% LABEL STYLES
% ============================================================
\tikzset{
    % Phase label (below nodes)
    phase label/.style={
        font=\scriptsize\itshape,
        text=black!70,
    },
    % Equation label (for inline math)
    eq label/.style={
        font=\scriptsize,
        fill=white,
        inner sep=1pt,
    },
    % Section annotation
    section label/.style={
        font=\footnotesize\bfseries,
        text=black,
    },
}

% ============================================================
% JUNCTION/PARTICLE STYLES
% ============================================================
\tikzset{
    % Y-junction point
    junction point/.style={
        circle,
        fill=red!60!black,
        minimum size=4pt,
        inner sep=0pt,
    },
    % Flux tube arm
    flux arm/.style={
        thick,
        blue!60!black,
    },
    % Particle dot (electron, etc.)
    particle/.style={
        circle,
        fill=black,
        minimum size=5pt,
        inner sep=0pt,
    },
    % Neutrino (smaller, gray)
    neutrino/.style={
        circle,
        fill=gray,
        minimum size=4pt,
        inner sep=0pt,
    },
}

% ============================================================
% SPRING DECORATION (for mechanical models)
% ============================================================
\tikzset{
    spring/.style={
        thick,
        decorate,
        decoration={
            coil,
            aspect=0.5,
            segment length=2mm,
            amplitude=2mm,
        },
    },
    % Wave decoration (for field modes)
    wave field/.style={
        thick,
        decorate,
        decoration={
            snake,
            amplitude=2pt,
            segment length=8pt,
        },
    },
}

% ============================================================
% BOUNDARY STYLES
% ============================================================
\tikzset{
    % Bulk-facing boundary (dashed red)
    bulk boundary/.style={
        very thick,
        red!70!black,
        dashed,
    },
    % Observer-facing boundary (solid green)
    observer boundary/.style={
        thick,
        green!50!black,
    },
    % Brane edge (orange)
    brane edge/.style={
        thick,
        orange!70!black,
    },
}

% ============================================================
% CONVENIENCE COMMANDS
% ============================================================
% Arrow label (above)
\newcommand{\arrlabel}[1]{\scriptsize #1}
% Arrow label (below)
\newcommand{\arrlabelb}[1]{\scriptsize #1}

% ============================================================
% END OF STYLE FILE
% ============================================================
  % if using TikZ figures
%
% REQUIRED PACKAGES (load these in main document before \input):
%   fontspec, amsmath, amssymb, amsthm, mathtools, geometry
%   hyperref, enumitem, booktabs, array, xcolor, tcolorbox
%
% ============================================================

% ============================================================
%  EPISTEMIC TAG COLORS
% ============================================================
\definecolor{tagDer}{RGB}{0,128,0}      % Green - Derived
\definecolor{tagDc}{RGB}{0,0,200}       % Blue - Deduced/Constrained
\definecolor{tagCal}{RGB}{200,0,0}      % Red - Calibrated
\definecolor{tagP}{RGB}{128,0,128}      % Purple - Postulated
\definecolor{tagBL}{RGB}{128,128,128}   % Gray - Baseline
\definecolor{tagI}{RGB}{255,140,0}      % Orange - Identified
\definecolor{tagOpen}{RGB}{200,100,0}   % Dark orange - Open

% ============================================================
%  EPISTEMIC TAG COMMANDS
% ============================================================
% Use these to mark claims with their epistemic status
\newcommand{\tagDer}{\textcolor{tagDer}{\textbf{[Der]}}}    % Derived from axioms
\newcommand{\tagDc}{\textcolor{tagDc}{\textbf{[Dc]}}}       % Deduced/Constrained
\newcommand{\tagCal}{\textcolor{tagCal}{\textbf{[Cal]}}}    % Calibrated (fitted)
\newcommand{\tagP}{\textcolor{tagP}{\textbf{[P]}}}          % Postulated
\newcommand{\tagBL}{\textcolor{tagBL}{\textbf{[BL]}}}       % Baseline (external fact)
\newcommand{\tagI}{\textcolor{tagI}{\textbf{[I]}}}          % Identified (pattern match)
\newcommand{\tagOpen}{\textcolor{tagOpen}{\textbf{[OPEN]}}} % Open problem
\newcommand{\tagDef}{\textcolor{tagDc}{\textbf{[Def]}}}     % Definition

% ============================================================
%  THEOREM ENVIRONMENTS
% ============================================================
\newtheorem{postulate}{Postulate}
\newtheorem{definition}{Definition}[section]
\newtheorem{theorem}{Theorem}[section]
\newtheorem{lemma}[theorem]{Lemma}
\newtheorem{corollary}[theorem]{Corollary}
\newtheorem{proposition}[theorem]{Proposition}
\newtheorem{remark}{Remark}[section]

% ============================================================
%  COMMON EDC SYMBOLS
% ============================================================
% Symmetry groups
\newcommand{\Ztwo}{\mathbb{Z}_2}
\newcommand{\Zthree}{\mathbb{Z}_3}
\newcommand{\Ztri}{\mathbb{Z}_3}    % alias
\newcommand{\Zsix}{\mathbb{Z}_6}

% Geometric objects
\newcommand{\Sthree}{S^3}           % 3-sphere
\newcommand{\Stwo}{S^2}             % 2-sphere
\newcommand{\Bthree}{B^3}           % 3-ball
\newcommand{\Mfive}{\mathcal{M}_5}  % 5D manifold
\newcommand{\Bfour}{\mathcal{B}_4}  % 4D brane

% Physical quantities
\newcommand{\tension}{\tau}         % string/flux-tube tension (E/L)
\newcommand{\re}{r_e}               % electron radius

% Operators
\newcommand{\Pfrozen}{\mathcal{P}_{\mathrm{frozen}}}  % Frozen projection operator
\newcommand{\Ebrane}{\mathcal{E}_{\mathrm{brane}}}    % Brane energy store

% Bulk-brane exchange current (canonical notation from Framework v2.0)
\newcommand{\Jbb}[1]{J^{#1}_{\mathrm{bulk}\to\mathrm{brane}}}

% ============================================================
%  TCOLORBOX STYLES FOR EDC PAPERS
% ============================================================
% Cornerstone box (blue) — key claims/foundations
\tcbset{
    edcCornerstone/.style={
        colback=blue!5,
        colframe=blue!40!black,
        fonttitle=\bfseries
    }
}

% Guardrail box (gray) — epistemic warnings/constraints
\tcbset{
    edcGuardrail/.style={
        colback=gray!5!white,
        colframe=gray!60!black,
        fonttitle=\bfseries
    }
}

% PPN box (blue, lighter) — Physical Process Narrative
\tcbset{
    edcPPN/.style={
        colback=blue!5,
        colframe=blue!50!black,
        fonttitle=\bfseries
    }
}

% Canonical box (yellow/orange) — canonical definitions/glossary
\tcbset{
    edcCanonical/.style={
        colback=yellow!5,
        colframe=orange!60!black,
        fonttitle=\bfseries
    }
}

% Conceptual box (yellow/orange, lighter) — conceptual pictures
\tcbset{
    edcConcept/.style={
        colback=yellow!5,
        colframe=orange!50!black,
        fonttitle=\bfseries
    }
}

% Pathway box (purple) — energy pathways, mechanisms
\tcbset{
    edcPathway/.style={
        colback=purple!5,
        colframe=purple!40!black,
        fonttitle=\bfseries
    }
}

% Model box (green) — mechanical analogies, heuristics
\tcbset{
    edcModel/.style={
        colback=green!5,
        colframe=green!40!black,
        fonttitle=\bfseries
    }
}

% Warning box (red) — non-overclaim, limitations
\tcbset{
    edcWarning/.style={
        colback=red!5,
        colframe=red!40!black,
        fonttitle=\bfseries
    }
}

% Framework quote box (gray) — verbatim from Framework v2.0
\tcbset{
    edcFramework/.style={
        colback=gray!5!white,
        colframe=gray!60!black,
        fonttitle=\small
    }
}

% Mechanism box (teal) — mechanistic dimension principle narrative
\tcbset{
    edcMechanism/.style={
        colback=teal!5,
        colframe=teal!50!black,
        fonttitle=\bfseries,
        title={Mechanistic Dimension Note (Canon)}
    }
}

% ============================================================
%  MECHANISTIC DIMENSION HELPER MACRO
% ============================================================
% Usage: \edcMechanismNote{bulk cause}{brane process}{3D output}
%
% Example:
%   \edcMechanismNote{Junction relaxes toward Steiner minimum}%
%                    {Energy pumps into brane-layer modes, redistributes}%
%                    {Electron, antineutrino, proton emerge on 3D side}
%
\newcommand{\edcMechanismNote}[3]{%
\begin{tcolorbox}[edcMechanism]
\begin{itemize}[nosep,leftmargin=*]
    \item \textbf{5D cause (bulk):} #1
    \item \textbf{Brane-layer process:} #2
    \item \textbf{3D observation (output):} #3
\end{itemize}
\vspace{0.3em}
\footnotesize\textit{Ledger closure must hold: bulk + brane + 3D outputs conserve energy/quantum numbers.}
\end{tcolorbox}
}

% ============================================================
%  RELATED DOCUMENTS MACRO
% ============================================================
% Usage: \edcRelatedDocs{main paper title}{main DOI}{companion list}
%
% Example:
%     A: \emph{Effective Lagrangian} (\href{...}{DOI}) $\cdot$
%     B: \emph{WKB Prefactor} (\href{...}{DOI})
%   }

% NOTE: \edcRelatedDocs macro deprecated (DOI registry consolidated)
% Use consolidated Zenodo article as primary reference instead.

% ============================================================
%  DOI REGISTRY DEPRECATED
% ============================================================
% Previous individual DOIs have been deprecated.
% All EDC Weak Sector content is now consolidated into a single
% Zenodo article. See paper_3_series/19_edc_weak_sector_zenodo_article/

% ============================================================
%  PHYSICAL NARRATION RULE REMINDER
% ============================================================
% Every key equation MUST be accompanied by a physical narrative stating:
%   1. 5D cause: What changes in the bulk-core configuration?
%   2. Brane response: How does the brane absorb/redistribute energy?
%   3. 3D observable output: What do observers detect on the 3D side?
%
% This rule eliminates "numerology smell" by ensuring every formula
% has a mechanistic interpretation.

% ============================================================
%  END OF STYLE FILE
% ============================================================

%   % tikz_style_edc.tex — Reusable TikZ styles for EDC papers
% Version 1.0 — 2026-01-20
% Include via: % tikz_style_edc.tex — Reusable TikZ styles for EDC papers
% Version 1.0 — 2026-01-20
% Include via: \input{tikz_style_edc}

% ============================================================
% REQUIRED LIBRARIES (must be loaded in main document)
% ============================================================
% \usetikzlibrary{calc,angles,quotes,decorations.markings,decorations.pathmorphing,positioning}

% ============================================================
% POSITIONING DEFAULTS
% ============================================================
\tikzset{
    % Default node distances for horizontal/vertical layouts
    edc node distance/.style={node distance=1.6cm and 2.0cm},
    % Compact variant for dense diagrams
    edc compact/.style={node distance=1.2cm and 1.5cm},
    % Spread variant for clarity
    edc spread/.style={node distance=2.0cm and 2.5cm},
}

% ============================================================
% COLOR PALETTE (consistent with epistemic tags)
% ============================================================
\definecolor{edcBulk}{RGB}{220,50,50}        % Red tones for bulk/5D
\definecolor{edcBrane}{RGB}{50,150,50}       % Green tones for brane-layer
\definecolor{edcOutput}{RGB}{50,100,200}     % Blue tones for 3D outputs
\definecolor{edcNeutral}{RGB}{100,100,100}   % Gray for neutral/annotations

% ============================================================
% BOX STYLES
% ============================================================
\tikzset{
    % Generic EDC box (base style)
    edc box/.style={
        rectangle,
        draw,
        rounded corners=3pt,
        minimum width=2.2cm,
        minimum height=0.8cm,
        align=center,
        font=\small,
        inner sep=4pt,
    },
    % Bulk-core box (red family)
    bulk box/.style={
        edc box,
        fill=red!10,
        draw=edcBulk!70!black,
        text=black,
    },
    % Brane-layer box (green family)
    brane box/.style={
        edc box,
        fill=green!10,
        draw=edcBrane!70!black,
        text=black,
    },
    % 3D output box (blue family)
    output box/.style={
        edc box,
        fill=blue!10,
        draw=edcOutput!70!black,
        text=black,
    },
    % Neutral/process box
    process box/.style={
        edc box,
        fill=gray!10,
        draw=gray!60!black,
        text=black,
    },
    % Label-only box (no background)
    label box/.style={
        rectangle,
        rounded corners=2pt,
        draw=gray!40,
        fill=white,
        inner sep=2pt,
        font=\scriptsize,
    },
}

% ============================================================
% ARROW STYLES
% ============================================================
\tikzset{
    % Standard thick arrow
    edc arrow/.style={
        ->,
        >=stealth,
        thick,
    },
    % Emphasized arrow (for main flow)
    edc flow/.style={
        ->,
        >=stealth,
        very thick,
        line width=1.2pt,
    },
    % Dashed arrow (for optional/weak connections)
    edc dashed/.style={
        ->,
        >=stealth,
        thick,
        dashed,
    },
    % Double arrow (for bidirectional)
    edc bidir/.style={
        <->,
        >=stealth,
        thick,
    },
}

% ============================================================
% REGION STYLES (for background fills)
% ============================================================
\tikzset{
    % Bulk region (5D)
    bulk region/.style={
        fill=blue!8,
    },
    % Brane layer region
    brane region/.style={
        fill=yellow!25,
    },
    % Observer/3D region
    observer region/.style={
        fill=green!8,
    },
}

% ============================================================
% LABEL STYLES
% ============================================================
\tikzset{
    % Phase label (below nodes)
    phase label/.style={
        font=\scriptsize\itshape,
        text=black!70,
    },
    % Equation label (for inline math)
    eq label/.style={
        font=\scriptsize,
        fill=white,
        inner sep=1pt,
    },
    % Section annotation
    section label/.style={
        font=\footnotesize\bfseries,
        text=black,
    },
}

% ============================================================
% JUNCTION/PARTICLE STYLES
% ============================================================
\tikzset{
    % Y-junction point
    junction point/.style={
        circle,
        fill=red!60!black,
        minimum size=4pt,
        inner sep=0pt,
    },
    % Flux tube arm
    flux arm/.style={
        thick,
        blue!60!black,
    },
    % Particle dot (electron, etc.)
    particle/.style={
        circle,
        fill=black,
        minimum size=5pt,
        inner sep=0pt,
    },
    % Neutrino (smaller, gray)
    neutrino/.style={
        circle,
        fill=gray,
        minimum size=4pt,
        inner sep=0pt,
    },
}

% ============================================================
% SPRING DECORATION (for mechanical models)
% ============================================================
\tikzset{
    spring/.style={
        thick,
        decorate,
        decoration={
            coil,
            aspect=0.5,
            segment length=2mm,
            amplitude=2mm,
        },
    },
    % Wave decoration (for field modes)
    wave field/.style={
        thick,
        decorate,
        decoration={
            snake,
            amplitude=2pt,
            segment length=8pt,
        },
    },
}

% ============================================================
% BOUNDARY STYLES
% ============================================================
\tikzset{
    % Bulk-facing boundary (dashed red)
    bulk boundary/.style={
        very thick,
        red!70!black,
        dashed,
    },
    % Observer-facing boundary (solid green)
    observer boundary/.style={
        thick,
        green!50!black,
    },
    % Brane edge (orange)
    brane edge/.style={
        thick,
        orange!70!black,
    },
}

% ============================================================
% CONVENIENCE COMMANDS
% ============================================================
% Arrow label (above)
\newcommand{\arrlabel}[1]{\scriptsize #1}
% Arrow label (below)
\newcommand{\arrlabelb}[1]{\scriptsize #1}

% ============================================================
% END OF STYLE FILE
% ============================================================


% ============================================================
% REQUIRED LIBRARIES (must be loaded in main document)
% ============================================================
% \usetikzlibrary{calc,angles,quotes,decorations.markings,decorations.pathmorphing,positioning}

% ============================================================
% POSITIONING DEFAULTS
% ============================================================
\tikzset{
    % Default node distances for horizontal/vertical layouts
    edc node distance/.style={node distance=1.6cm and 2.0cm},
    % Compact variant for dense diagrams
    edc compact/.style={node distance=1.2cm and 1.5cm},
    % Spread variant for clarity
    edc spread/.style={node distance=2.0cm and 2.5cm},
}

% ============================================================
% COLOR PALETTE (consistent with epistemic tags)
% ============================================================
\definecolor{edcBulk}{RGB}{220,50,50}        % Red tones for bulk/5D
\definecolor{edcBrane}{RGB}{50,150,50}       % Green tones for brane-layer
\definecolor{edcOutput}{RGB}{50,100,200}     % Blue tones for 3D outputs
\definecolor{edcNeutral}{RGB}{100,100,100}   % Gray for neutral/annotations

% ============================================================
% BOX STYLES
% ============================================================
\tikzset{
    % Generic EDC box (base style)
    edc box/.style={
        rectangle,
        draw,
        rounded corners=3pt,
        minimum width=2.2cm,
        minimum height=0.8cm,
        align=center,
        font=\small,
        inner sep=4pt,
    },
    % Bulk-core box (red family)
    bulk box/.style={
        edc box,
        fill=red!10,
        draw=edcBulk!70!black,
        text=black,
    },
    % Brane-layer box (green family)
    brane box/.style={
        edc box,
        fill=green!10,
        draw=edcBrane!70!black,
        text=black,
    },
    % 3D output box (blue family)
    output box/.style={
        edc box,
        fill=blue!10,
        draw=edcOutput!70!black,
        text=black,
    },
    % Neutral/process box
    process box/.style={
        edc box,
        fill=gray!10,
        draw=gray!60!black,
        text=black,
    },
    % Label-only box (no background)
    label box/.style={
        rectangle,
        rounded corners=2pt,
        draw=gray!40,
        fill=white,
        inner sep=2pt,
        font=\scriptsize,
    },
}

% ============================================================
% ARROW STYLES
% ============================================================
\tikzset{
    % Standard thick arrow
    edc arrow/.style={
        ->,
        >=stealth,
        thick,
    },
    % Emphasized arrow (for main flow)
    edc flow/.style={
        ->,
        >=stealth,
        very thick,
        line width=1.2pt,
    },
    % Dashed arrow (for optional/weak connections)
    edc dashed/.style={
        ->,
        >=stealth,
        thick,
        dashed,
    },
    % Double arrow (for bidirectional)
    edc bidir/.style={
        <->,
        >=stealth,
        thick,
    },
}

% ============================================================
% REGION STYLES (for background fills)
% ============================================================
\tikzset{
    % Bulk region (5D)
    bulk region/.style={
        fill=blue!8,
    },
    % Brane layer region
    brane region/.style={
        fill=yellow!25,
    },
    % Observer/3D region
    observer region/.style={
        fill=green!8,
    },
}

% ============================================================
% LABEL STYLES
% ============================================================
\tikzset{
    % Phase label (below nodes)
    phase label/.style={
        font=\scriptsize\itshape,
        text=black!70,
    },
    % Equation label (for inline math)
    eq label/.style={
        font=\scriptsize,
        fill=white,
        inner sep=1pt,
    },
    % Section annotation
    section label/.style={
        font=\footnotesize\bfseries,
        text=black,
    },
}

% ============================================================
% JUNCTION/PARTICLE STYLES
% ============================================================
\tikzset{
    % Y-junction point
    junction point/.style={
        circle,
        fill=red!60!black,
        minimum size=4pt,
        inner sep=0pt,
    },
    % Flux tube arm
    flux arm/.style={
        thick,
        blue!60!black,
    },
    % Particle dot (electron, etc.)
    particle/.style={
        circle,
        fill=black,
        minimum size=5pt,
        inner sep=0pt,
    },
    % Neutrino (smaller, gray)
    neutrino/.style={
        circle,
        fill=gray,
        minimum size=4pt,
        inner sep=0pt,
    },
}

% ============================================================
% SPRING DECORATION (for mechanical models)
% ============================================================
\tikzset{
    spring/.style={
        thick,
        decorate,
        decoration={
            coil,
            aspect=0.5,
            segment length=2mm,
            amplitude=2mm,
        },
    },
    % Wave decoration (for field modes)
    wave field/.style={
        thick,
        decorate,
        decoration={
            snake,
            amplitude=2pt,
            segment length=8pt,
        },
    },
}

% ============================================================
% BOUNDARY STYLES
% ============================================================
\tikzset{
    % Bulk-facing boundary (dashed red)
    bulk boundary/.style={
        very thick,
        red!70!black,
        dashed,
    },
    % Observer-facing boundary (solid green)
    observer boundary/.style={
        thick,
        green!50!black,
    },
    % Brane edge (orange)
    brane edge/.style={
        thick,
        orange!70!black,
    },
}

% ============================================================
% CONVENIENCE COMMANDS
% ============================================================
% Arrow label (above)
\newcommand{\arrlabel}[1]{\scriptsize #1}
% Arrow label (below)
\newcommand{\arrlabelb}[1]{\scriptsize #1}

% ============================================================
% END OF STYLE FILE
% ============================================================
  % if using TikZ figures
%
% REQUIRED PACKAGES (load these in main document before \input):
%   fontspec, amsmath, amssymb, amsthm, mathtools, geometry
%   hyperref, enumitem, booktabs, array, xcolor, tcolorbox
%
% ============================================================

% ============================================================
%  EPISTEMIC TAG COLORS
% ============================================================
\definecolor{tagDer}{RGB}{0,128,0}      % Green - Derived
\definecolor{tagDc}{RGB}{0,0,200}       % Blue - Deduced/Constrained
\definecolor{tagCal}{RGB}{200,0,0}      % Red - Calibrated
\definecolor{tagP}{RGB}{128,0,128}      % Purple - Postulated
\definecolor{tagBL}{RGB}{128,128,128}   % Gray - Baseline
\definecolor{tagI}{RGB}{255,140,0}      % Orange - Identified
\definecolor{tagOpen}{RGB}{200,100,0}   % Dark orange - Open

% ============================================================
%  EPISTEMIC TAG COMMANDS
% ============================================================
% Use these to mark claims with their epistemic status
\newcommand{\tagDer}{\textcolor{tagDer}{\textbf{[Der]}}}    % Derived from axioms
\newcommand{\tagDc}{\textcolor{tagDc}{\textbf{[Dc]}}}       % Deduced/Constrained
\newcommand{\tagCal}{\textcolor{tagCal}{\textbf{[Cal]}}}    % Calibrated (fitted)
\newcommand{\tagP}{\textcolor{tagP}{\textbf{[P]}}}          % Postulated
\newcommand{\tagBL}{\textcolor{tagBL}{\textbf{[BL]}}}       % Baseline (external fact)
\newcommand{\tagI}{\textcolor{tagI}{\textbf{[I]}}}          % Identified (pattern match)
\newcommand{\tagOpen}{\textcolor{tagOpen}{\textbf{[OPEN]}}} % Open problem
\newcommand{\tagDef}{\textcolor{tagDc}{\textbf{[Def]}}}     % Definition

% ============================================================
%  THEOREM ENVIRONMENTS
% ============================================================
\newtheorem{postulate}{Postulate}
\newtheorem{definition}{Definition}[section]
\newtheorem{theorem}{Theorem}[section]
\newtheorem{lemma}[theorem]{Lemma}
\newtheorem{corollary}[theorem]{Corollary}
\newtheorem{proposition}[theorem]{Proposition}
\newtheorem{remark}{Remark}[section]

% ============================================================
%  COMMON EDC SYMBOLS
% ============================================================
% Symmetry groups
\newcommand{\Ztwo}{\mathbb{Z}_2}
\newcommand{\Zthree}{\mathbb{Z}_3}
\newcommand{\Ztri}{\mathbb{Z}_3}    % alias
\newcommand{\Zsix}{\mathbb{Z}_6}

% Geometric objects
\newcommand{\Sthree}{S^3}           % 3-sphere
\newcommand{\Stwo}{S^2}             % 2-sphere
\newcommand{\Bthree}{B^3}           % 3-ball
\newcommand{\Mfive}{\mathcal{M}_5}  % 5D manifold
\newcommand{\Bfour}{\mathcal{B}_4}  % 4D brane

% Physical quantities
\newcommand{\tension}{\tau}         % string/flux-tube tension (E/L)
\newcommand{\re}{r_e}               % electron radius

% Operators
\newcommand{\Pfrozen}{\mathcal{P}_{\mathrm{frozen}}}  % Frozen projection operator
\newcommand{\Ebrane}{\mathcal{E}_{\mathrm{brane}}}    % Brane energy store

% Bulk-brane exchange current (canonical notation from Framework v2.0)
\newcommand{\Jbb}[1]{J^{#1}_{\mathrm{bulk}\to\mathrm{brane}}}

% ============================================================
%  TCOLORBOX STYLES FOR EDC PAPERS
% ============================================================
% Cornerstone box (blue) — key claims/foundations
\tcbset{
    edcCornerstone/.style={
        colback=blue!5,
        colframe=blue!40!black,
        fonttitle=\bfseries
    }
}

% Guardrail box (gray) — epistemic warnings/constraints
\tcbset{
    edcGuardrail/.style={
        colback=gray!5!white,
        colframe=gray!60!black,
        fonttitle=\bfseries
    }
}

% PPN box (blue, lighter) — Physical Process Narrative
\tcbset{
    edcPPN/.style={
        colback=blue!5,
        colframe=blue!50!black,
        fonttitle=\bfseries
    }
}

% Canonical box (yellow/orange) — canonical definitions/glossary
\tcbset{
    edcCanonical/.style={
        colback=yellow!5,
        colframe=orange!60!black,
        fonttitle=\bfseries
    }
}

% Conceptual box (yellow/orange, lighter) — conceptual pictures
\tcbset{
    edcConcept/.style={
        colback=yellow!5,
        colframe=orange!50!black,
        fonttitle=\bfseries
    }
}

% Pathway box (purple) — energy pathways, mechanisms
\tcbset{
    edcPathway/.style={
        colback=purple!5,
        colframe=purple!40!black,
        fonttitle=\bfseries
    }
}

% Model box (green) — mechanical analogies, heuristics
\tcbset{
    edcModel/.style={
        colback=green!5,
        colframe=green!40!black,
        fonttitle=\bfseries
    }
}

% Warning box (red) — non-overclaim, limitations
\tcbset{
    edcWarning/.style={
        colback=red!5,
        colframe=red!40!black,
        fonttitle=\bfseries
    }
}

% Framework quote box (gray) — verbatim from Framework v2.0
\tcbset{
    edcFramework/.style={
        colback=gray!5!white,
        colframe=gray!60!black,
        fonttitle=\small
    }
}

% Mechanism box (teal) — mechanistic dimension principle narrative
\tcbset{
    edcMechanism/.style={
        colback=teal!5,
        colframe=teal!50!black,
        fonttitle=\bfseries,
        title={Mechanistic Dimension Note (Canon)}
    }
}

% ============================================================
%  MECHANISTIC DIMENSION HELPER MACRO
% ============================================================
% Usage: \edcMechanismNote{bulk cause}{brane process}{3D output}
%
% Example:
%   \edcMechanismNote{Junction relaxes toward Steiner minimum}%
%                    {Energy pumps into brane-layer modes, redistributes}%
%                    {Electron, antineutrino, proton emerge on 3D side}
%
\newcommand{\edcMechanismNote}[3]{%
\begin{tcolorbox}[edcMechanism]
\begin{itemize}[nosep,leftmargin=*]
    \item \textbf{5D cause (bulk):} #1
    \item \textbf{Brane-layer process:} #2
    \item \textbf{3D observation (output):} #3
\end{itemize}
\vspace{0.3em}
\footnotesize\textit{Ledger closure must hold: bulk + brane + 3D outputs conserve energy/quantum numbers.}
\end{tcolorbox}
}

% ============================================================
%  RELATED DOCUMENTS MACRO
% ============================================================
% Usage: \edcRelatedDocs{main paper title}{main DOI}{companion list}
%
% Example:
%     A: \emph{Effective Lagrangian} (\href{...}{DOI}) $\cdot$
%     B: \emph{WKB Prefactor} (\href{...}{DOI})
%   }

% NOTE: \edcRelatedDocs macro deprecated (DOI registry consolidated)
% Use consolidated Zenodo article as primary reference instead.

% ============================================================
%  DOI REGISTRY DEPRECATED
% ============================================================
% Previous individual DOIs have been deprecated.
% All EDC Weak Sector content is now consolidated into a single
% Zenodo article. See paper_3_series/19_edc_weak_sector_zenodo_article/

% ============================================================
%  PHYSICAL NARRATION RULE REMINDER
% ============================================================
% Every key equation MUST be accompanied by a physical narrative stating:
%   1. 5D cause: What changes in the bulk-core configuration?
%   2. Brane response: How does the brane absorb/redistribute energy?
%   3. 3D observable output: What do observers detect on the 3D side?
%
% This rule eliminates "numerology smell" by ensuring every formula
% has a mechanistic interpretation.

% ============================================================
%  END OF STYLE FILE
% ============================================================

% tikz_style_edc.tex — Reusable TikZ styles for EDC papers
% Version 1.0 — 2026-01-20
% Include via: % tikz_style_edc.tex — Reusable TikZ styles for EDC papers
% Version 1.0 — 2026-01-20
% Include via: % tikz_style_edc.tex — Reusable TikZ styles for EDC papers
% Version 1.0 — 2026-01-20
% Include via: \input{tikz_style_edc}

% ============================================================
% REQUIRED LIBRARIES (must be loaded in main document)
% ============================================================
% \usetikzlibrary{calc,angles,quotes,decorations.markings,decorations.pathmorphing,positioning}

% ============================================================
% POSITIONING DEFAULTS
% ============================================================
\tikzset{
    % Default node distances for horizontal/vertical layouts
    edc node distance/.style={node distance=1.6cm and 2.0cm},
    % Compact variant for dense diagrams
    edc compact/.style={node distance=1.2cm and 1.5cm},
    % Spread variant for clarity
    edc spread/.style={node distance=2.0cm and 2.5cm},
}

% ============================================================
% COLOR PALETTE (consistent with epistemic tags)
% ============================================================
\definecolor{edcBulk}{RGB}{220,50,50}        % Red tones for bulk/5D
\definecolor{edcBrane}{RGB}{50,150,50}       % Green tones for brane-layer
\definecolor{edcOutput}{RGB}{50,100,200}     % Blue tones for 3D outputs
\definecolor{edcNeutral}{RGB}{100,100,100}   % Gray for neutral/annotations

% ============================================================
% BOX STYLES
% ============================================================
\tikzset{
    % Generic EDC box (base style)
    edc box/.style={
        rectangle,
        draw,
        rounded corners=3pt,
        minimum width=2.2cm,
        minimum height=0.8cm,
        align=center,
        font=\small,
        inner sep=4pt,
    },
    % Bulk-core box (red family)
    bulk box/.style={
        edc box,
        fill=red!10,
        draw=edcBulk!70!black,
        text=black,
    },
    % Brane-layer box (green family)
    brane box/.style={
        edc box,
        fill=green!10,
        draw=edcBrane!70!black,
        text=black,
    },
    % 3D output box (blue family)
    output box/.style={
        edc box,
        fill=blue!10,
        draw=edcOutput!70!black,
        text=black,
    },
    % Neutral/process box
    process box/.style={
        edc box,
        fill=gray!10,
        draw=gray!60!black,
        text=black,
    },
    % Label-only box (no background)
    label box/.style={
        rectangle,
        rounded corners=2pt,
        draw=gray!40,
        fill=white,
        inner sep=2pt,
        font=\scriptsize,
    },
}

% ============================================================
% ARROW STYLES
% ============================================================
\tikzset{
    % Standard thick arrow
    edc arrow/.style={
        ->,
        >=stealth,
        thick,
    },
    % Emphasized arrow (for main flow)
    edc flow/.style={
        ->,
        >=stealth,
        very thick,
        line width=1.2pt,
    },
    % Dashed arrow (for optional/weak connections)
    edc dashed/.style={
        ->,
        >=stealth,
        thick,
        dashed,
    },
    % Double arrow (for bidirectional)
    edc bidir/.style={
        <->,
        >=stealth,
        thick,
    },
}

% ============================================================
% REGION STYLES (for background fills)
% ============================================================
\tikzset{
    % Bulk region (5D)
    bulk region/.style={
        fill=blue!8,
    },
    % Brane layer region
    brane region/.style={
        fill=yellow!25,
    },
    % Observer/3D region
    observer region/.style={
        fill=green!8,
    },
}

% ============================================================
% LABEL STYLES
% ============================================================
\tikzset{
    % Phase label (below nodes)
    phase label/.style={
        font=\scriptsize\itshape,
        text=black!70,
    },
    % Equation label (for inline math)
    eq label/.style={
        font=\scriptsize,
        fill=white,
        inner sep=1pt,
    },
    % Section annotation
    section label/.style={
        font=\footnotesize\bfseries,
        text=black,
    },
}

% ============================================================
% JUNCTION/PARTICLE STYLES
% ============================================================
\tikzset{
    % Y-junction point
    junction point/.style={
        circle,
        fill=red!60!black,
        minimum size=4pt,
        inner sep=0pt,
    },
    % Flux tube arm
    flux arm/.style={
        thick,
        blue!60!black,
    },
    % Particle dot (electron, etc.)
    particle/.style={
        circle,
        fill=black,
        minimum size=5pt,
        inner sep=0pt,
    },
    % Neutrino (smaller, gray)
    neutrino/.style={
        circle,
        fill=gray,
        minimum size=4pt,
        inner sep=0pt,
    },
}

% ============================================================
% SPRING DECORATION (for mechanical models)
% ============================================================
\tikzset{
    spring/.style={
        thick,
        decorate,
        decoration={
            coil,
            aspect=0.5,
            segment length=2mm,
            amplitude=2mm,
        },
    },
    % Wave decoration (for field modes)
    wave field/.style={
        thick,
        decorate,
        decoration={
            snake,
            amplitude=2pt,
            segment length=8pt,
        },
    },
}

% ============================================================
% BOUNDARY STYLES
% ============================================================
\tikzset{
    % Bulk-facing boundary (dashed red)
    bulk boundary/.style={
        very thick,
        red!70!black,
        dashed,
    },
    % Observer-facing boundary (solid green)
    observer boundary/.style={
        thick,
        green!50!black,
    },
    % Brane edge (orange)
    brane edge/.style={
        thick,
        orange!70!black,
    },
}

% ============================================================
% CONVENIENCE COMMANDS
% ============================================================
% Arrow label (above)
\newcommand{\arrlabel}[1]{\scriptsize #1}
% Arrow label (below)
\newcommand{\arrlabelb}[1]{\scriptsize #1}

% ============================================================
% END OF STYLE FILE
% ============================================================


% ============================================================
% REQUIRED LIBRARIES (must be loaded in main document)
% ============================================================
% \usetikzlibrary{calc,angles,quotes,decorations.markings,decorations.pathmorphing,positioning}

% ============================================================
% POSITIONING DEFAULTS
% ============================================================
\tikzset{
    % Default node distances for horizontal/vertical layouts
    edc node distance/.style={node distance=1.6cm and 2.0cm},
    % Compact variant for dense diagrams
    edc compact/.style={node distance=1.2cm and 1.5cm},
    % Spread variant for clarity
    edc spread/.style={node distance=2.0cm and 2.5cm},
}

% ============================================================
% COLOR PALETTE (consistent with epistemic tags)
% ============================================================
\definecolor{edcBulk}{RGB}{220,50,50}        % Red tones for bulk/5D
\definecolor{edcBrane}{RGB}{50,150,50}       % Green tones for brane-layer
\definecolor{edcOutput}{RGB}{50,100,200}     % Blue tones for 3D outputs
\definecolor{edcNeutral}{RGB}{100,100,100}   % Gray for neutral/annotations

% ============================================================
% BOX STYLES
% ============================================================
\tikzset{
    % Generic EDC box (base style)
    edc box/.style={
        rectangle,
        draw,
        rounded corners=3pt,
        minimum width=2.2cm,
        minimum height=0.8cm,
        align=center,
        font=\small,
        inner sep=4pt,
    },
    % Bulk-core box (red family)
    bulk box/.style={
        edc box,
        fill=red!10,
        draw=edcBulk!70!black,
        text=black,
    },
    % Brane-layer box (green family)
    brane box/.style={
        edc box,
        fill=green!10,
        draw=edcBrane!70!black,
        text=black,
    },
    % 3D output box (blue family)
    output box/.style={
        edc box,
        fill=blue!10,
        draw=edcOutput!70!black,
        text=black,
    },
    % Neutral/process box
    process box/.style={
        edc box,
        fill=gray!10,
        draw=gray!60!black,
        text=black,
    },
    % Label-only box (no background)
    label box/.style={
        rectangle,
        rounded corners=2pt,
        draw=gray!40,
        fill=white,
        inner sep=2pt,
        font=\scriptsize,
    },
}

% ============================================================
% ARROW STYLES
% ============================================================
\tikzset{
    % Standard thick arrow
    edc arrow/.style={
        ->,
        >=stealth,
        thick,
    },
    % Emphasized arrow (for main flow)
    edc flow/.style={
        ->,
        >=stealth,
        very thick,
        line width=1.2pt,
    },
    % Dashed arrow (for optional/weak connections)
    edc dashed/.style={
        ->,
        >=stealth,
        thick,
        dashed,
    },
    % Double arrow (for bidirectional)
    edc bidir/.style={
        <->,
        >=stealth,
        thick,
    },
}

% ============================================================
% REGION STYLES (for background fills)
% ============================================================
\tikzset{
    % Bulk region (5D)
    bulk region/.style={
        fill=blue!8,
    },
    % Brane layer region
    brane region/.style={
        fill=yellow!25,
    },
    % Observer/3D region
    observer region/.style={
        fill=green!8,
    },
}

% ============================================================
% LABEL STYLES
% ============================================================
\tikzset{
    % Phase label (below nodes)
    phase label/.style={
        font=\scriptsize\itshape,
        text=black!70,
    },
    % Equation label (for inline math)
    eq label/.style={
        font=\scriptsize,
        fill=white,
        inner sep=1pt,
    },
    % Section annotation
    section label/.style={
        font=\footnotesize\bfseries,
        text=black,
    },
}

% ============================================================
% JUNCTION/PARTICLE STYLES
% ============================================================
\tikzset{
    % Y-junction point
    junction point/.style={
        circle,
        fill=red!60!black,
        minimum size=4pt,
        inner sep=0pt,
    },
    % Flux tube arm
    flux arm/.style={
        thick,
        blue!60!black,
    },
    % Particle dot (electron, etc.)
    particle/.style={
        circle,
        fill=black,
        minimum size=5pt,
        inner sep=0pt,
    },
    % Neutrino (smaller, gray)
    neutrino/.style={
        circle,
        fill=gray,
        minimum size=4pt,
        inner sep=0pt,
    },
}

% ============================================================
% SPRING DECORATION (for mechanical models)
% ============================================================
\tikzset{
    spring/.style={
        thick,
        decorate,
        decoration={
            coil,
            aspect=0.5,
            segment length=2mm,
            amplitude=2mm,
        },
    },
    % Wave decoration (for field modes)
    wave field/.style={
        thick,
        decorate,
        decoration={
            snake,
            amplitude=2pt,
            segment length=8pt,
        },
    },
}

% ============================================================
% BOUNDARY STYLES
% ============================================================
\tikzset{
    % Bulk-facing boundary (dashed red)
    bulk boundary/.style={
        very thick,
        red!70!black,
        dashed,
    },
    % Observer-facing boundary (solid green)
    observer boundary/.style={
        thick,
        green!50!black,
    },
    % Brane edge (orange)
    brane edge/.style={
        thick,
        orange!70!black,
    },
}

% ============================================================
% CONVENIENCE COMMANDS
% ============================================================
% Arrow label (above)
\newcommand{\arrlabel}[1]{\scriptsize #1}
% Arrow label (below)
\newcommand{\arrlabelb}[1]{\scriptsize #1}

% ============================================================
% END OF STYLE FILE
% ============================================================


% ============================================================
% REQUIRED LIBRARIES (must be loaded in main document)
% ============================================================
% \usetikzlibrary{calc,angles,quotes,decorations.markings,decorations.pathmorphing,positioning}

% ============================================================
% POSITIONING DEFAULTS
% ============================================================
\tikzset{
    % Default node distances for horizontal/vertical layouts
    edc node distance/.style={node distance=1.6cm and 2.0cm},
    % Compact variant for dense diagrams
    edc compact/.style={node distance=1.2cm and 1.5cm},
    % Spread variant for clarity
    edc spread/.style={node distance=2.0cm and 2.5cm},
}

% ============================================================
% COLOR PALETTE (consistent with epistemic tags)
% ============================================================
\definecolor{edcBulk}{RGB}{220,50,50}        % Red tones for bulk/5D
\definecolor{edcBrane}{RGB}{50,150,50}       % Green tones for brane-layer
\definecolor{edcOutput}{RGB}{50,100,200}     % Blue tones for 3D outputs
\definecolor{edcNeutral}{RGB}{100,100,100}   % Gray for neutral/annotations

% ============================================================
% BOX STYLES
% ============================================================
\tikzset{
    % Generic EDC box (base style)
    edc box/.style={
        rectangle,
        draw,
        rounded corners=3pt,
        minimum width=2.2cm,
        minimum height=0.8cm,
        align=center,
        font=\small,
        inner sep=4pt,
    },
    % Bulk-core box (red family)
    bulk box/.style={
        edc box,
        fill=red!10,
        draw=edcBulk!70!black,
        text=black,
    },
    % Brane-layer box (green family)
    brane box/.style={
        edc box,
        fill=green!10,
        draw=edcBrane!70!black,
        text=black,
    },
    % 3D output box (blue family)
    output box/.style={
        edc box,
        fill=blue!10,
        draw=edcOutput!70!black,
        text=black,
    },
    % Neutral/process box
    process box/.style={
        edc box,
        fill=gray!10,
        draw=gray!60!black,
        text=black,
    },
    % Label-only box (no background)
    label box/.style={
        rectangle,
        rounded corners=2pt,
        draw=gray!40,
        fill=white,
        inner sep=2pt,
        font=\scriptsize,
    },
}

% ============================================================
% ARROW STYLES
% ============================================================
\tikzset{
    % Standard thick arrow
    edc arrow/.style={
        ->,
        >=stealth,
        thick,
    },
    % Emphasized arrow (for main flow)
    edc flow/.style={
        ->,
        >=stealth,
        very thick,
        line width=1.2pt,
    },
    % Dashed arrow (for optional/weak connections)
    edc dashed/.style={
        ->,
        >=stealth,
        thick,
        dashed,
    },
    % Double arrow (for bidirectional)
    edc bidir/.style={
        <->,
        >=stealth,
        thick,
    },
}

% ============================================================
% REGION STYLES (for background fills)
% ============================================================
\tikzset{
    % Bulk region (5D)
    bulk region/.style={
        fill=blue!8,
    },
    % Brane layer region
    brane region/.style={
        fill=yellow!25,
    },
    % Observer/3D region
    observer region/.style={
        fill=green!8,
    },
}

% ============================================================
% LABEL STYLES
% ============================================================
\tikzset{
    % Phase label (below nodes)
    phase label/.style={
        font=\scriptsize\itshape,
        text=black!70,
    },
    % Equation label (for inline math)
    eq label/.style={
        font=\scriptsize,
        fill=white,
        inner sep=1pt,
    },
    % Section annotation
    section label/.style={
        font=\footnotesize\bfseries,
        text=black,
    },
}

% ============================================================
% JUNCTION/PARTICLE STYLES
% ============================================================
\tikzset{
    % Y-junction point
    junction point/.style={
        circle,
        fill=red!60!black,
        minimum size=4pt,
        inner sep=0pt,
    },
    % Flux tube arm
    flux arm/.style={
        thick,
        blue!60!black,
    },
    % Particle dot (electron, etc.)
    particle/.style={
        circle,
        fill=black,
        minimum size=5pt,
        inner sep=0pt,
    },
    % Neutrino (smaller, gray)
    neutrino/.style={
        circle,
        fill=gray,
        minimum size=4pt,
        inner sep=0pt,
    },
}

% ============================================================
% SPRING DECORATION (for mechanical models)
% ============================================================
\tikzset{
    spring/.style={
        thick,
        decorate,
        decoration={
            coil,
            aspect=0.5,
            segment length=2mm,
            amplitude=2mm,
        },
    },
    % Wave decoration (for field modes)
    wave field/.style={
        thick,
        decorate,
        decoration={
            snake,
            amplitude=2pt,
            segment length=8pt,
        },
    },
}

% ============================================================
% BOUNDARY STYLES
% ============================================================
\tikzset{
    % Bulk-facing boundary (dashed red)
    bulk boundary/.style={
        very thick,
        red!70!black,
        dashed,
    },
    % Observer-facing boundary (solid green)
    observer boundary/.style={
        thick,
        green!50!black,
    },
    % Brane edge (orange)
    brane edge/.style={
        thick,
        orange!70!black,
    },
}

% ============================================================
% CONVENIENCE COMMANDS
% ============================================================
% Arrow label (above)
\newcommand{\arrlabel}[1]{\scriptsize #1}
% Arrow label (below)
\newcommand{\arrlabelb}[1]{\scriptsize #1}

% ============================================================
% END OF STYLE FILE
% ============================================================


% ============================================================
% DOCUMENT INFO
% ============================================================
\title{\textbf{Companion T: Tau Decay as a Brane Mode-Spectrum Test}\\[0.3cm]
\large Elastic Diffusive Cosmology — Paper 3 Series}
\author{Igor Grčman}
\date{Draft v0.1 — January 2026\\[0.3em]

% ============================================================
\begin{document}
% ============================================================

\maketitle

\begin{abstract}
We extend the thick-brane microphysics framework to tau lepton decay,
focusing on the leptonic channels $\tau^- \to e^- + \bar{\nu}_e + \nu_\tau$
and $\tau^- \to \mu^- + \bar{\nu}_\mu + \nu_\tau$. Like the muon, the tau
is modeled as a \emph{brane-dominant excitation}, but with a higher
``mode index'' corresponding to its greater mass. This companion tests
whether the same absorption$\to$dissipation$\to$release pipeline that
describes muon decay generalizes to the heavier lepton. We define
allowed output sets, establish the mode-spectrum hypothesis, and
identify falsifiability conditions. No numerical fits are performed;
tau lifetime and branching fractions are treated as observational
baselines \tagBL{}.
\end{abstract}

\tableofcontents
\newpage

% ============================================================
\section{Motivation: Why Tau After Muon?}
\label{sec:motivation}
% ============================================================

\begin{tcolorbox}[edcCornerstone, title=\textbf{Cornerstone: Tau as Mode-Spectrum Test}]
The tau lepton ($\tau^-$) is the heaviest charged lepton, with
$m_\tau \approx 1777$~MeV \tagBL{} \cite{PDG2024}. If the thick-brane
framework applies to muon decay (Companion M), it must also accommodate
tau decay without introducing new mechanisms. The tau provides a
\emph{mode-spectrum test}: same brane-dominant ontology, different
mass/energy scale.
\end{tcolorbox}

\subsection{Strategic Position}

The EDC weak-interaction program now has:
\begin{itemize}[nosep]
    \item \textbf{Neutron} (Companion N): Bulk-core junction decay
    \item \textbf{Muon} (Companion M): Brane-dominant leptonic decay
    \item \textbf{Tau} (this document): Heavier brane-dominant decay
\end{itemize}

If the same pipeline works for both $\mu$ and $\tau$, this validates the
\emph{brane-dominant excitation} hypothesis across the charged lepton
spectrum. The tau's larger mass probes a different region of the
brane-layer mode spectrum.

\begin{tcolorbox}[edcGuardrail, title=\textbf{Epistemic Guardrail: No Fitting}]
We treat $\tau$ lifetime and branching fractions as \tagBL{} observational
constraints. Companion T provides a consistent 5D$\to$brane$\to$3D mechanism
framing \emph{without tuning parameters to match those numbers}. The goal
is structural consistency, not numerical prediction.
\end{tcolorbox}

\subsection{Scope Limitation}

This companion addresses \textbf{leptonic tau decays only}:
\begin{itemize}[nosep]
    \item $\tau^- \to e^- + \bar{\nu}_e + \nu_\tau$ \quad (electronic channel)
    \item $\tau^- \to \mu^- + \bar{\nu}_\mu + \nu_\tau$ \quad (muonic channel)
\end{itemize}

Hadronic tau decays (e.g., $\tau \to \pi\nu$, $\tau \to \rho\nu$) are
mentioned as future extensions \tagOpen{}. Pion ontology is \emph{not}
developed here.

% ============================================================
\section{Tau Ontology: Brane-Dominant Excitation}
\label{sec:ontology}
% ============================================================

\begin{postulate}[Tau Ontology] \tagP{}
\label{post:tau-ontology}
The tau lepton $\tau^-$ is a \emph{brane-dominant excitation} with a
higher mode index than the muon. Its primary degrees of freedom reside
within the brane layer, not in the bulk-core.
\end{postulate}

\textbf{Physical Narration:}
\begin{enumerate}[nosep]
    \item \textbf{5D cause:} The tau occupies a higher-energy eigenmode of the
          brane-layer spectrum compared to the muon.
    \item \textbf{Brane response:} This mode is unstable; it can decay into
          lower-mass modes (electrons, muons, neutrinos) via internal
          redistribution.
    \item \textbf{3D output:} The frozen projection maps allowed mode
          combinations to observable particles.
\end{enumerate}

\begin{figure}[htbp]
\centering
\begin{tikzpicture}[scale=0.85]
    % Background regions
    \fill[bulk region] (-4.5,-2.5) rectangle (-1.5,2.5);
    \fill[brane region] (-1.5,-2.5) rectangle (1.5,2.5);
    \fill[observer region] (1.5,-2.5) rectangle (4.5,2.5);

    % Labels
    \node[section label] at (-3,2.9) {\textbf{Bulk-Core}};
    \node[section label] at (0,2.9) {\textbf{Brane-Layer}};
    \node[section label] at (3,2.9) {\textbf{3D Outputs}};

    % Boundaries
    \draw[bulk boundary] (-1.5,-2.5) -- (-1.5,2.5);
    \draw[observer boundary] (1.5,-2.5) -- (1.5,2.5);

    % Mode spectrum visualization (vertical axis = energy/mode index)
    \node[font=\scriptsize, rotate=90] at (-4.2,0) {Mode energy $\uparrow$};

    % Tau mode (higher)
    \node[circle, fill=orange!70, minimum size=10pt, inner sep=0pt] (tau) at (0,1.5) {};
    \node[right=0.15cm of tau, font=\footnotesize] {$\tau^-$ (high mode)};
    \draw[dashed, orange!60!black, thick] (-1.3,1.5) -- (1.3,1.5);

    % Muon mode (middle)
    \node[circle, fill=purple!60, minimum size=10pt, inner sep=0pt] (mu) at (0,0) {};
    \node[right=0.15cm of mu, font=\footnotesize] {$\mu^-$ (mid mode)};
    \draw[dashed, purple!60!black, thick] (-1.3,0) -- (1.3,0);

    % Electron mode (lowest)
    \node[circle, fill=blue!60, minimum size=10pt, inner sep=0pt] (e) at (0,-1.5) {};
    \node[right=0.15cm of e, font=\footnotesize] {$e^-$ (low mode)};
    \draw[dashed, blue!60!black, thick] (-1.3,-1.5) -- (1.3,-1.5);

    % Arrows showing decay directions
    \draw[edc arrow, orange!70!black, thick] (0.5,1.3) -- (0.5,0.2);
    \draw[edc arrow, orange!70!black, thick] (0.7,1.3) -- (0.7,-1.3);
    \node[font=\scriptsize, text=orange!70!black] at (1.1,0.7) {$\tau \to \mu$};
    \node[font=\scriptsize, text=orange!70!black] at (1.1,-0.5) {$\tau \to e$};

    % Bulk annotation
    \node[font=\scriptsize, text=gray] at (-3,0) {(no bulk core)};

\end{tikzpicture}
\caption{Charged lepton mode spectrum in the brane layer. The tau occupies
a higher mode than the muon, which in turn is higher than the electron.
Decay proceeds ``downward'' in the spectrum via mode redistribution.
All three are brane-dominant; none have bulk-core structure.}
\label{fig:mode-spectrum}
\end{figure}

\subsection{Mode Index Hypothesis}

\begin{definition}[Mode Index] \tagDef{}/\tagP{}
\label{def:mode-index}
We associate each charged lepton with a \emph{mode index} $n_\ell$
characterizing its position in the brane-layer spectrum:
\[
    n_e < n_\mu < n_\tau
\]
Higher mode index corresponds to higher mass and shorter lifetime
(greater instability).
\end{definition}

\textbf{Note:} The mode index is a qualitative ordering \tagP{}/\tagOpen{}.
We do not claim to derive $n_\ell$ values or the precise relationship
$m_\ell(n_\ell)$ from first principles.

% ============================================================
\section{Observational Baselines}
\label{sec:baselines}
% ============================================================

The following quantities are treated as \textbf{observational inputs}
\tagBL{}, not outputs of the model.

\begin{table}[htbp]
\centering
\caption{Tau lepton properties (PDG 2024) \cite{PDG2024}}
\label{tab:baselines}
\begin{tabular}{lcc}
\toprule
\textbf{Quantity} & \textbf{Value} & \textbf{Status} \\
\midrule
Mass $m_\tau$ & $1776.86 \pm 0.12$~MeV & \tagBL{} \\
Lifetime $\tau_\tau$ & $(290.3 \pm 0.5) \times 10^{-15}$~s & \tagBL{} \\
BR($\tau \to e\nu\bar{\nu}$) & $(17.82 \pm 0.04)\%$ & \tagBL{} \\
BR($\tau \to \mu\nu\bar{\nu}$) & $(17.39 \pm 0.04)\%$ & \tagBL{} \\
BR(leptonic total) & $\approx 35\%$ & \tagBL{} \\
BR(hadronic total) & $\approx 65\%$ & \tagBL{} \\
\bottomrule
\end{tabular}
\end{table}

\begin{tcolorbox}[edcGuardrail, title=\textbf{Epistemic Guardrail}]
\textbf{These are not tuning targets.} The branching fractions and lifetime
in Table~\ref{tab:baselines} are \emph{facts about nature} that any viable
model must be \emph{consistent with}. We do not adjust parameters to
reproduce them.
\end{tcolorbox}

% ============================================================
\section{Canonical Pipeline: Absorption $\to$ Dissipation $\to$ Release}
\label{sec:pipeline}
% ============================================================

The tau decay pipeline mirrors that of muon decay (Companion M), with
the same three phases:

\begin{tcolorbox}[edcPPN, title=\textbf{Physical Process Narrative: Tau Leptonic Decay}]
\begin{enumerate}[nosep]
    \item[\textbf{(i)}] \textbf{Absorption/Charging:} The unstable tau mode
          redistributes energy within the brane layer.
    \item[\textbf{(ii)}] \textbf{Dissipation:} Brane-layer modes become
          populated according to the available spectrum and selection rules.
    \item[\textbf{(iii)}] \textbf{Release/Emission:} The frozen projection
          $\mathcal{P}_{\mathrm{frozen}}$ maps populated modes to 3D outputs.
\end{enumerate}
\end{tcolorbox}

\subsection{Bulk Leakage Suppression}

\begin{postulate}[Suppressed Bulk Leakage] \tagP{}
\label{post:bulk-suppression}
For brane-dominant excitations (electron, muon, tau), leakage of energy
into the bulk-core is suppressed by the mode's localization within the
brane layer. At leading order, bulk leakage is treated as negligible.
\end{postulate}

\textbf{Physical Narration:}
\begin{itemize}[nosep]
    \item \textbf{5D cause:} Brane-layer modes have exponentially small
          overlap with bulk-core wavefunctions.
    \item \textbf{Brane response:} Energy redistribution occurs predominantly
          within the brane layer.
    \item \textbf{3D output:} All released energy appears in 3D outputs
          (plus soft/residual brane modes).
\end{itemize}

\begin{figure}[htbp]
\centering
\begin{tikzpicture}[edc compact, scale=0.85]

% Nodes
\node[brane box, minimum width=2.2cm, fill=orange!15] (tau) at (0,0) {$\tau^-$ mode\\(brane-layer)};
\node[process box, right=1.6cm of tau] (abs) {Absorption/\\Redistribution};
\node[process box, right=1.6cm of abs] (diss) {Dissipation/\\Mode population};
\node[output box, right=1.6cm of diss, minimum width=2.0cm] (out) {3D Outputs\\$\ell^-, \bar{\nu}_\ell, \nu_\tau$};

% Arrows with labels
\draw[edc flow, orange!70!black] (tau) -- node[above, font=\scriptsize] {instability} (abs);
\draw[edc flow, green!50!black] (abs) -- node[above, font=\scriptsize] {$\Gamma_{\mathrm{eff}}$} (diss);
\draw[edc flow, blue!60!black] (diss) -- node[above, font=\scriptsize] {$\mathcal{P}_{\mathrm{frozen}}$} (out);

% Phase labels below
\node[phase label, below=0.4cm of tau] {Initial state};
\node[phase label, below=0.4cm of abs] {Charging};
\node[phase label, below=0.4cm of diss] {Mode spectrum};
\node[phase label, below=0.4cm of out] {Observation};

% Chiral filter annotation
\draw[dashed, red!60!black] ($(diss.east)!0.5!(out.west)$) ++(0,-0.7) -- ++(0,1.4);
\node[font=\scriptsize, text=red!60!black, below] at ($(diss.east)!0.5!(out.west) + (0,-0.9)$) {$\mathcal{P}_{\mathrm{chir}}$};

% Ledger closure annotation
\node[label box, below=1.4cm of abs] (ledger) {\footnotesize Ledger: $m_\tau c^2 = E_\ell + E_{\bar{\nu}} + E_{\nu_\tau} + E_{\mathrm{other}}$};

\end{tikzpicture}
\caption{Energy flow in tau leptonic decay. The pipeline is identical to
muon decay (Companion M), with the tau as initial brane-dominant mode.
The output $\ell^-$ can be either $e^-$ or $\mu^-$.}
\label{fig:pipeline}
\end{figure}

% ============================================================
\section{Allowed Output Sets and Selection Rules}
\label{sec:outputs}
% ============================================================

\subsection{Leptonic Channel Definitions}

\begin{definition}[Allowed Output Sets] \tagDef{}/\tagDc{}
\label{def:allowed-outputs}
The allowed output sets for tau leptonic decays are:
\begin{align}
    \mathcal{A}_{\tau \to e} &= \{e^-, \bar{\nu}_e, \nu_\tau\} \\
    \mathcal{A}_{\tau \to \mu} &= \{\mu^-, \bar{\nu}_\mu, \nu_\tau\}
\end{align}
These follow from:
\begin{itemize}[nosep]
    \item Charge conservation: $Q_\tau = Q_\ell = -1$
    \item Lepton number conservation: $L_\tau = 1$ (carried by $\nu_\tau$),
          $L_\ell = 0$ (from $\ell^- + \bar{\nu}_\ell$ pair)
    \item Energy threshold: $m_\tau > m_\mu > m_e$ (both channels kinematically allowed)
\end{itemize}
\end{definition}

\subsection{Channel Comparison}

\begin{table}[htbp]
\centering
\caption{Tau leptonic channels: experimental vs.\ EDC framing}
\label{tab:channels}
\begin{tabular}{lccc}
\toprule
\textbf{Channel} & \textbf{BR (exp.)} & \textbf{Status} & \textbf{EDC framing} \\
\midrule
$\tau \to e\nu\bar{\nu}$ & $17.82\%$ & \tagBL{} & Allowed by $\mathcal{A}_{\tau \to e}$ \tagDc{} \\
$\tau \to \mu\nu\bar{\nu}$ & $17.39\%$ & \tagBL{} & Allowed by $\mathcal{A}_{\tau \to \mu}$ \tagDc{} \\
$\tau \to e\gamma$ & $< 3.3 \times 10^{-8}$ & \tagBL{} & LFV; selection rule violation \tagP{}/\tagOpen{} \\
$\tau \to \mu\gamma$ & $< 4.2 \times 10^{-8}$ & \tagBL{} & LFV; selection rule violation \tagP{}/\tagOpen{} \\
$\tau \to eee$ & $< 2.7 \times 10^{-8}$ & \tagBL{} & Mode mismatch hypothesis \tagP{}/\tagOpen{} \\
\bottomrule
\end{tabular}
\end{table}

\textbf{Note:} The near-equality of BR($\tau \to e$) and BR($\tau \to \mu$)
is an observational fact \tagBL{}. We do not claim to derive this ratio;
explaining it would require a quantitative theory of mode-spectrum
branching \tagOpen{}.

% ============================================================
\section{Mode-Spectrum Hypothesis}
\label{sec:mode-spectrum}
% ============================================================

\begin{postulate}[Mode-Spectrum Branching] \tagP{}/\tagOpen{}
\label{post:mode-spectrum}
The branching fractions for tau decay are determined by the
\emph{spectral overlap} between the initial tau mode and the allowed
final-state mode configurations. Schematically:
\begin{equation}
    \mathrm{BR}(\tau \to X) \propto |\langle \Psi_X | \hat{T} | \Psi_\tau \rangle|^2
    \label{eq:spectral-overlap}
\end{equation}
where $\hat{T}$ is a transition operator and $\Psi_X$ represents the
final-state mode configuration.
\end{postulate}

\textbf{Physical Narration:}
\begin{enumerate}[nosep]
    \item \textbf{5D cause:} The tau mode $\Psi_\tau$ has a specific profile
          in the brane-layer spectrum.
    \item \textbf{Brane response:} The transition operator $\hat{T}$ couples
          $\Psi_\tau$ to final-state configurations; the coupling strength
          depends on spectral overlap.
    \item \textbf{3D output:} Branching fractions reflect these overlaps,
          filtered through $\mathcal{P}_{\mathrm{frozen}}$.
\end{enumerate}

\begin{tcolorbox}[edcWarning, title=\textbf{Non-Overclaim Reminder}]
Equation~\eqref{eq:spectral-overlap} is a \emph{schematic} representation
\tagP{}/\tagOpen{}. We have not derived the form of $\hat{T}$ or the mode
wavefunctions from the 5D action. The claim is that branching fractions
\emph{can be understood} in terms of spectral structure—not that we have
computed them.
\end{tcolorbox}

\subsection{Why Are BR($\tau \to e$) and BR($\tau \to \mu$) Nearly Equal?}

This is an \textbf{open question} \tagOpen{}. Possible framings within EDC:
\begin{itemize}[nosep]
    \item The electron and muon final states have similar spectral overlap
          with the tau initial state (modulo phase-space corrections).
    \item The mode-spectrum structure is approximately ``democratic'' for
          leptonic channels.
    \item Detailed calculation requires knowledge of $\hat{T}$ and brane-layer
          wavefunctions.
\end{itemize}

We do not claim to resolve this question; it remains \tagOpen{}.

% ============================================================
\section{Chiral Filter Hook}
\label{sec:chiral}
% ============================================================

As in Companion M, the frozen projection operator includes a chiral
filter component:

\begin{equation}
    \mathcal{P}_{\mathrm{frozen}} = \mathcal{P}_{\mathrm{energy}} \circ
    \mathcal{P}_{\mathrm{mode}} \circ \mathcal{P}_{\mathrm{chir}}
    \label{eq:projection-stack}
\end{equation}

\begin{tcolorbox}[edcPathway, title=\textbf{Chiral Filter (Hypothesis)}]
\textbf{Hypothesis} \tagP{}/\tagOpen{}\textbf{:} We propose that the observed
chirality pattern in tau leptonic decays (left-handed charged leptons,
left-handed neutrinos, right-handed antineutrinos) is \emph{consistent with}
a geometric chiral filter at the observer-facing brane boundary.

\medskip
The derivation of $\mathcal{P}_{\mathrm{chir}}$ from 5D boundary conditions
remains \tagOpen{}.
\end{tcolorbox}

The chirality selection pattern for tau decay is identical to muon decay:
\begin{itemize}[nosep]
    \item $\ell^-$ ($e^-$ or $\mu^-$): predominantly left-handed
    \item $\nu_\tau$: left-handed
    \item $\bar{\nu}_\ell$: right-handed
\end{itemize}

This universality across $\mu$ and $\tau$ supports the hypothesis that
chirality selection is a \emph{boundary property}, not specific to the
decaying particle.

% ============================================================
\section{Falsifiability Guardrails}
\label{sec:falsifiability}
% ============================================================

\begin{tcolorbox}[edcGuardrail, title=\textbf{Falsifiability: What Would Refute This Framing?}]
\begin{enumerate}[nosep]
    \item \textbf{Wrong allowed outputs:} If tau leptonic decay produced
          particles outside $\mathcal{A}_{\tau \to e}$ or $\mathcal{A}_{\tau \to \mu}$
          at observable rates, the selection rule mechanism fails.

    \item \textbf{Ledger non-closure:} If energy accounting showed a deficit
          not attributable to $E_{\mathrm{other}}$ (soft modes, residuals),
          the pipeline would be falsified.

    \item \textbf{Inconsistent chirality:} If tau decay showed a different
          chirality pattern than muon decay, the universal chiral-filter
          hypothesis fails.

    \item \textbf{Bulk leakage evidence:} If tau decay deposited measurable
          energy into bulk modes, the brane-dominant ontology would be
          falsified.

    \item \textbf{Pipeline failure for $\tau$ but not $\mu$:} If the same
          absorption$\to$dissipation$\to$release framework could not
          accommodate both leptons, the generalization claim fails.
\end{enumerate}
\end{tcolorbox}

% ============================================================
\section{Open Problems}
\label{sec:open}
% ============================================================

\begin{enumerate}[nosep]
    \item \textbf{Derive $\mathcal{P}_{\mathrm{chir}}$ from boundary conditions}
          \tagOpen{}: Construct the chiral filter from 5D action + BC at
          $y = +\delta/2$.

    \item \textbf{Explain BR($\tau \to e$) $\approx$ BR($\tau \to \mu$)}
          \tagOpen{}: Derive from mode-spectrum structure without fitting.

    \item \textbf{Mode index quantification} \tagOpen{}: Derive the
          relationship $m_\ell(n_\ell)$ from brane-layer spectrum.

    \item \textbf{Hadronic tau decays} \tagOpen{}: Extend to channels like
          $\tau \to \pi\nu$, which requires pion ontology (future companion).

    \item \textbf{Lifetime from first principles} \tagOpen{}: Currently
          $\tau_\tau$ is \tagBL{}; deriving it requires quantitative
          mode-spectrum dynamics.
\end{enumerate}

% ============================================================
\section{Conclusion}
\label{sec:conclusion}
% ============================================================

Companion T demonstrates that the thick-brane microphysics pipeline
generalizes from muon to tau without modification:

\begin{itemize}[nosep]
    \item \textbf{Same ontology:} Brane-dominant excitation (higher mode index)
    \item \textbf{Same pipeline:} Absorption$\to$Dissipation$\to$Release
    \item \textbf{Same projection:} $\mathcal{P}_{\mathrm{frozen}} =
          \mathcal{P}_{\mathrm{energy}} \circ \mathcal{P}_{\mathrm{mode}}
          \circ \mathcal{P}_{\mathrm{chir}}$
    \item \textbf{Same chirality pattern:} Universal across lepton sector
\end{itemize}

This consistency across $\mu$ and $\tau$ supports the claim that the
framework captures \emph{structural features} of weak leptonic decays,
not accidental properties of a single particle.

\subsection{Position in the EDC Weak Program}

\begin{itemize}[nosep]
    \item \textbf{Companion M} ($\mu$): Established brane-dominant pipeline
    \item \textbf{Companion T} ($\tau$): Mode-spectrum generalization (this document)
    \item \textbf{Future}: Pion decay (hadron$\to$lepton bridge) \tagOpen{}
\end{itemize}

% ============================================================
% RELATED DOCUMENTS
% ============================================================

\vfill
    M: \emph{Muon Decay Tomography} (this series) $\cdot$
}

% ============================================================
% REFERENCES
% ============================================================
\begin{thebibliography}{9}

\bibitem{PDG2024}
R.~L.~Workman \emph{et al.} (Particle Data Group),
``Review of Particle Physics,''
Prog.\ Theor.\ Exp.\ Phys.\ \textbf{2022}, 083C01 (2022 and 2024 update).
\href{https://pdg.lbl.gov}{https://pdg.lbl.gov}.
Tau properties: $m_\tau = 1776.86 \pm 0.12$~MeV;
$\tau_\tau = (290.3 \pm 0.5) \times 10^{-15}$~s;
BR($\tau \to e\nu\bar{\nu}$) $= 17.82\%$;
BR($\tau \to \mu\nu\bar{\nu}$) $= 17.39\%$.

\end{thebibliography}

% ============================================================
\end{document}
% ============================================================
