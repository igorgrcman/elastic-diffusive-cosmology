%%%%%%%%%%%%%%%%%%%%%%%%%%%%%%%%%%%%%%%%%%%%%%%%%%%%%%%%%%%%%%%%%%%%%%%
%% COMPANION NOTE C: 5D Action Reduction Pipeline
%% From Brane-World Action to Effective 1D Mechanics
%%
%% Companion C to Paper 3 (NJSR Edition)
%% Build: XeLaTeX (Unicode)
%%
%% Author: Igor Grčman
%% Date: January 2026
%% Status: Reviewer-grade, epistemic cleanup
%%%%%%%%%%%%%%%%%%%%%%%%%%%%%%%%%%%%%%%%%%%%%%%%%%%%%%%%%%%%%%%%%%%%%%%

\documentclass[11pt,a4paper]{article}

%% ===== PACKAGES =====
\usepackage{fontspec}
\usepackage{amsmath,amssymb,amsthm}
\usepackage{tikz}
\usepackage{tcolorbox}
\tcbuselibrary{breakable,theorems}
\usepackage{array}
\usepackage{booktabs}
\usepackage{geometry}
\usepackage[colorlinks=true,linkcolor=blue,citecolor=blue,urlcolor=blue]{hyperref}
\usepackage{xcolor}
\usepackage{enumitem}
\usepackage[style=numeric-comp,sorting=none,backend=biber]{biblatex}
\addbibresource{refs_5d.bib}

\geometry{margin=1in}

%% ===== FONTS (fallback to default if TeX Gyre not available) =====
\IfFontExistsTF{TeX Gyre Termes}{%
  \setmainfont{TeX Gyre Termes}
  \setsansfont{TeX Gyre Heros}
}{%
  \setmainfont{Times New Roman}[Ligatures=TeX]
  \setsansfont{Helvetica}
}

%% ===== CUSTOM COLORS =====
\definecolor{derivedblue}{RGB}{0,100,180}
\definecolor{conditionalgreen}{RGB}{0,128,100}
\definecolor{proposedred}{RGB}{180,0,0}
\definecolor{mathpurple}{RGB}{120,0,120}
\definecolor{openred}{RGB}{200,0,0}

%% ===== THEOREM ENVIRONMENTS =====
\newtheorem{proposition}{Proposition}[section]
\newtheorem{theorem}[proposition]{Theorem}
\newtheorem{lemma}[proposition]{Lemma}
\newtheorem{corollary}[proposition]{Corollary}
\theoremstyle{definition}
\newtheorem{definition}[proposition]{Definition}
\newtheorem{postulate}[proposition]{Postulate}
\theoremstyle{remark}
\newtheorem*{remark}{Remark}

%% ===== EPISTEMIC TAG COMMANDS =====
\newcommand{\tagDer}{\textcolor{derivedblue}{\textbf{[Der]}}}
\newcommand{\tagDc}{\textcolor{conditionalgreen}{\textbf{[Dc]}}}
\newcommand{\tagDd}{\textcolor{derivedblue}{\textbf{[Dd]}}}
\newcommand{\tagP}{\textcolor{proposedred}{\textbf{[P]}}}
\newcommand{\tagM}{\textcolor{mathpurple}{\textbf{[M]}}}
\newcommand{\tagBL}{\textcolor{gray}{\textbf{[BL]}}}
\newcommand{\tagOPEN}{\textcolor{openred}{\textbf{[OPEN]}}}

%% ===== TITLE =====
\title{\textbf{5D Action Reduction Pipeline}\\[0.5em]
\large From Brane-World Gravity to Effective 1D Collective Dynamics\\[0.3em]
\normalsize (Companion C to Paper~3: NJSR Edition)}
\author{Igor Gr\v{c}man}
\date{January 2026\\[0.5em]
\small Repository: \href{https://github.com/igorgrcman/elastic-diffusive-cosmology}{github.com/igorgrcman/elastic-diffusive-cosmology}\\[0.2em]
\footnotesize (Public artifacts for this paper are in the \texttt{edc\_papers} folder.)}

\begin{document}

\maketitle

\begin{center}
\small\textbf{Related Documents:}\\[0.1cm]
\footnotesize
\textbf{Companions:}\\
\end{center}

%%%%%%%%%%%%%%%%%%%%%%%%%%%%%%%%%%%%%%%%%%%%%%%%%%%%%%%%%%%%%%%%%%%%%%%
%% ABSTRACT
%%%%%%%%%%%%%%%%%%%%%%%%%%%%%%%%%%%%%%%%%%%%%%%%%%%%%%%%%%%%%%%%%%%%%%%

\begin{abstract}
\noindent
This companion note documents the dimensional reduction pipeline from the 5D brane-world action to the effective 1D mechanical system $S_{\rm eff}[q] = \int dt \left(\frac{1}{2}M(q)\dot{q}^2 - V(q)\right)$. We present: (i)~the bulk geometry ansatz (warped AdS$_5$); (ii)~the brane embedding with collective coordinate $q$; (iii)~the induced geometry and extrinsic curvature; (iv)~the Israel junction conditions; and (v)~the extraction of $M(q)$ and $V(q)$ from the action. All steps are tagged with epistemic status (\tagDer, \tagDc, \tagP, \tagOPEN). The full forensic-level worked derivation is in the main paper~\cite{paper3}, Appendix~J; this note provides the conceptual roadmap and key results.
\end{abstract}

\tableofcontents
\newpage

%%%%%%%%%%%%%%%%%%%%%%%%%%%%%%%%%%%%%%%%%%%%%%%%%%%%%%%%%%%%%%%%%%%%%%%
%% SECTION 1: OVERVIEW
%%%%%%%%%%%%%%%%%%%%%%%%%%%%%%%%%%%%%%%%%%%%%%%%%%%%%%%%%%%%%%%%%%%%%%%

\section{Overview and Reduction Chain}

\subsection{The Complete Pipeline}

The dimensional reduction proceeds in stages:

\begin{equation}
\boxed{
\underbrace{S_{\rm 5D}}_{\text{bulk + GHY + brane}}
\xrightarrow{\text{embedding}}
\underbrace{S[\phi(x;q)]}_{\text{defect profile}}
\xrightarrow{\text{collective coord}}
\underbrace{S_{\rm eff}[q]}_{\text{1D mechanics}}
}
\end{equation}

\begin{center}
\renewcommand{\arraystretch}{1.2}
\begin{tabular}{cll}
\toprule
\textbf{Stage} & \textbf{Content} & \textbf{Status} \\
\midrule
0 & Bulk geometry: AdS$_5$ from Einstein equations & \tagDc \\
1 & Bulk metric ansatz: $ds^2_5 = e^{-2|y|/\ell}\eta_{\mu\nu}dx^\mu dx^\nu + dy^2$ & \tagP \\
2 & Brane embedding: $X^A(\sigma^\mu; q) = (\sigma^\mu, f(r;q))$ & \tagP \\
3 & Induced metric $h_{\mu\nu}$ and extrinsic curvature $K_{\mu\nu}$ & \tagDer \\
4 & Israel junction conditions & \tagDer \\
5 & Defect profile: Gaussian $f(r;q) = A(q) e^{-r^2/2w^2}$ & \tagP \\
6 & $M(q)$ from kinetic term & \tagDc \\
7 & $V(q)$ from static energy & \tagDc \\
\bottomrule
\end{tabular}
\end{center}

\subsection{Epistemic Legend}

\begin{tcolorbox}[colback=gray!5!white, colframe=gray!70!black, title={\textbf{Epistemic Status Tags}}]
\begin{tabular}{cl}
\tagDer & \textbf{Derived} --- explicit calculation from stated premises \\
\tagDc & \textbf{Decisively constrained} --- conditional on approximations \\
\tagDd & \textbf{Definitional} --- structural/mathematical identity \\
\tagP & \textbf{Proposed} --- ansatz, not derived from $\delta S = 0$ \\
\tagM & \textbf{Mathematics} --- pure mathematical theorem \\
\tagOPEN & \textbf{Open} --- not yet derived
\end{tabular}
\end{tcolorbox}

%%%%%%%%%%%%%%%%%%%%%%%%%%%%%%%%%%%%%%%%%%%%%%%%%%%%%%%%%%%%%%%%%%%%%%%
%% SECTION 2: BULK GEOMETRY
%%%%%%%%%%%%%%%%%%%%%%%%%%%%%%%%%%%%%%%%%%%%%%%%%%%%%%%%%%%%%%%%%%%%%%%

\section{Stage 0--1: Bulk Geometry}

\subsection{From 5D Einstein Equations to AdS$_5$ (Conditional)}

\begin{postulate}[Symmetry Assumptions] \tagDd
\begin{enumerate}[label=(A\arabic*)]
\item 5D spacetime: $\mathcal{M}^5 = \mathcal{M}^4 \times I$
\item 4D Poincaré invariance on constant-$\xi$ slices
\item Static configuration
\item Gaussian normal coordinates
\end{enumerate}
\end{postulate}

\begin{lemma}[Symmetry-Restricted Metric] \tagDd
Under (A1)--(A4), the 5D metric takes the form:
\begin{equation}
ds^2_5 = e^{-2A(\xi)} \eta_{\mu\nu} dx^\mu dx^\nu + d\xi^2
\end{equation}
\end{lemma}

\begin{postulate}[Dynamical Assumptions] \tagP
\begin{enumerate}[label=(A\arabic*), start=5]
\item 5D Einstein-Hilbert action: $S_{\rm bulk} = \frac{1}{2\kappa_5^2}\int d^5x \sqrt{-g}(R - 2\Lambda_5)$
\item Negative cosmological constant: $\Lambda_5 < 0$
\item $\mathbb{Z}_2$ orbifold symmetry
\item Thin brane with tension $\sigma$ at $\xi = 0$
\end{enumerate}
\end{postulate}

\begin{theorem}[Warp Factor Solution] \tagDc
From the $(\xi\xi)$-component of Einstein's equations:
\begin{equation}
\boxed{A(\xi) = k|\xi|, \quad k^2 = -\frac{\Lambda_5}{6}}
\end{equation}
\end{theorem}

\begin{corollary}[Working Metric] \tagP
We adopt the warped AdS$_5$ metric:
\begin{equation}
\boxed{ds^2_5 = e^{-2|y|/\ell} \eta_{\mu\nu} dx^\mu dx^\nu + dy^2}
\label{eq:bulk_metric}
\end{equation}
where $\ell = 1/k$ is the AdS radius.
\end{corollary}

\begin{remark}
The physical origin of (A5)--(A8) from EDC Plenum dynamics remains \tagOPEN. We proceed with \eqref{eq:bulk_metric} as an ansatz.
\end{remark}

%%%%%%%%%%%%%%%%%%%%%%%%%%%%%%%%%%%%%%%%%%%%%%%%%%%%%%%%%%%%%%%%%%%%%%%
%% SECTION 3: BRANE EMBEDDING
%%%%%%%%%%%%%%%%%%%%%%%%%%%%%%%%%%%%%%%%%%%%%%%%%%%%%%%%%%%%%%%%%%%%%%%

\section{Stage 2: Brane Embedding}

\subsection{Embedding Map}

\begin{postulate}[Brane Embedding] \tagP
The 3-brane worldvolume $\Sigma^4$ is embedded via:
\begin{equation}
\boxed{X^A : \Sigma^4 \to \mathcal{M}^5, \quad X^A(\sigma^\mu; q) = (\sigma^\mu, f(r; q))}
\label{eq:embedding}
\end{equation}
where:
\begin{itemize}
\item $\sigma^\mu = (t, x^1, x^2, x^3)$ are worldvolume coordinates
\item $r = \sqrt{(x^1)^2 + (x^2)^2 + (x^3)^2}$ is the radial distance
\item $q \in [0,1]$ is the collective coordinate
\item $f(r;q)$ is the defect profile function
\end{itemize}
\end{postulate}

\subsection{Collective Coordinate Definition}

\begin{definition}[Collective Coordinate] \tagP
The coordinate $q$ parameterizes the junction asymmetry:
\begin{itemize}
\item $q = 0$: Symmetric Y-junction $\Rightarrow$ proton (stable)
\item $q = 1$: Maximum asymmetry $\Rightarrow$ neutron (metastable)
\item $q \in (0,1)$: Intermediate configurations $\Rightarrow$ transition states
\end{itemize}
\end{definition}

%%%%%%%%%%%%%%%%%%%%%%%%%%%%%%%%%%%%%%%%%%%%%%%%%%%%%%%%%%%%%%%%%%%%%%%
%% SECTION 4: INDUCED GEOMETRY
%%%%%%%%%%%%%%%%%%%%%%%%%%%%%%%%%%%%%%%%%%%%%%%%%%%%%%%%%%%%%%%%%%%%%%%

\section{Stage 3: Induced Geometry}

\subsection{Tangent Vectors and Induced Metric}

\begin{proposition}[Tangent Vectors] \tagDer
\begin{equation}
e^A_\mu = \frac{\partial X^A}{\partial \sigma^\mu} = (\delta^\nu_\mu, \partial_\mu f)
\end{equation}
\end{proposition}

\begin{proposition}[Induced Metric] \tagDer
\begin{equation}
\boxed{h_{\mu\nu} = g_{AB} e^A_\mu e^B_\nu = a^2(f) \eta_{\mu\nu} + \partial_\mu f \partial_\nu f}
\label{eq:induced_metric}
\end{equation}
where $a(f) = e^{-|f|/\ell}$ is the warp factor evaluated on the brane.
\end{proposition}

\subsection{Unit Normal}

\begin{proposition}[Unit Normal] \tagDer
\begin{equation}
n^A = \frac{1}{\sqrt{1 + a^{-2}|\nabla f|^2}} \left( -a^{-2} \partial^\mu f, 1 \right)
\end{equation}
\end{proposition}

\subsection{Extrinsic Curvature}

\begin{definition}[Extrinsic Curvature] \tagDd
\begin{equation}
K_{\mu\nu} = -\nabla_\mu n_\nu = -e^A_\mu e^B_\nu \nabla_A n_B
\end{equation}
\end{definition}

\begin{proposition}[Explicit Form] \tagDer
\begin{equation}
\boxed{K_{\mu\nu} = \frac{1}{\sqrt{1 + a^{-2}|\nabla f|^2}} \left[ \nabla_\mu \nabla_\nu f - k\, {\rm sgn}(f)\, a^2 \eta_{\mu\nu} \right]}
\end{equation}
\end{proposition}

%%%%%%%%%%%%%%%%%%%%%%%%%%%%%%%%%%%%%%%%%%%%%%%%%%%%%%%%%%%%%%%%%%%%%%%
%% SECTION 5: ISRAEL JUNCTION CONDITIONS
%%%%%%%%%%%%%%%%%%%%%%%%%%%%%%%%%%%%%%%%%%%%%%%%%%%%%%%%%%%%%%%%%%%%%%%

\section{Stage 4: Israel Junction Conditions}

\begin{theorem}[Israel Junction Conditions] \tagDer
For a hypersurface with $\mathbb{Z}_2$ symmetry:
\begin{equation}
\boxed{[K_{\mu\nu}] - h_{\mu\nu}[K] = -\kappa_5^2 S_{\mu\nu}}
\label{eq:israel}
\end{equation}
where $S_{\mu\nu} = -\sigma h_{\mu\nu} + \tau_{\mu\nu}^{\rm defect}$ is the brane stress-energy.
\end{theorem}

\begin{corollary}[Fine-Tuning Condition] \tagDc
For a flat brane ($f = 0$), the Israel condition yields the RS fine-tuning:
\begin{equation}
\sigma = \frac{6k}{\kappa_5^2}
\end{equation}
\end{corollary}

%%%%%%%%%%%%%%%%%%%%%%%%%%%%%%%%%%%%%%%%%%%%%%%%%%%%%%%%%%%%%%%%%%%%%%%
%% SECTION 6: DEFECT PROFILE
%%%%%%%%%%%%%%%%%%%%%%%%%%%%%%%%%%%%%%%%%%%%%%%%%%%%%%%%%%%%%%%%%%%%%%%

\section{Stage 5: Defect Profile Ansatz}

\begin{postulate}[Gaussian Profile] \tagP
\begin{equation}
\boxed{f(r; q) = A(q) \, e^{-r^2/2w^2}}
\label{eq:gaussian_profile}
\end{equation}
where:
\begin{itemize}
\item $A(q)$ is the amplitude function (depends on collective coordinate)
\item $w$ is the characteristic width (set by defect scale)
\end{itemize}
\end{postulate}

\begin{postulate}[Amplitude Parameterization] \tagP
\begin{equation}
A(q) = A_{\rm max} \cdot q(1-q)
\label{eq:amplitude}
\end{equation}
This satisfies boundary conditions $A(0) = A(1) = 0$.
\end{postulate}

%%%%%%%%%%%%%%%%%%%%%%%%%%%%%%%%%%%%%%%%%%%%%%%%%%%%%%%%%%%%%%%%%%%%%%%
%% SECTION 7: SUPERMETRIC
%%%%%%%%%%%%%%%%%%%%%%%%%%%%%%%%%%%%%%%%%%%%%%%%%%%%%%%%%%%%%%%%%%%%%%%

\section{Stage 6: Supermetric $M(q)$}

\begin{theorem}[Supermetric Formula] \tagDc
\label{thm:supermetric}
From the kinetic term in the brane action:
\begin{equation}
\boxed{M(q) = \sigma \int d^3\sigma \, a^2(f) \sqrt{1 + a^{-2}|\nabla f|^2} \left( \frac{\partial f}{\partial q} \right)^2}
\label{eq:supermetric}
\end{equation}
\end{theorem}

\begin{proposition}[Explicit Form] \tagDc
For the Gaussian profile \eqref{eq:gaussian_profile} with amplitude \eqref{eq:amplitude}:
\begin{equation}
\boxed{M(q) = M_0 \cdot (1 - 2q)^2}
\label{eq:Mq_explicit}
\end{equation}
where $M_0 = 4\pi \sigma a_0^2 A_{\rm max}^2 w^3 \cdot \mathcal{J}$.
\end{proposition}

\begin{remark}
The supermetric vanishes at $q = 1/2$ (barrier top), reflecting the geometric singularity in configuration space.
\end{remark}

%%%%%%%%%%%%%%%%%%%%%%%%%%%%%%%%%%%%%%%%%%%%%%%%%%%%%%%%%%%%%%%%%%%%%%%
%% SECTION 8: POTENTIAL
%%%%%%%%%%%%%%%%%%%%%%%%%%%%%%%%%%%%%%%%%%%%%%%%%%%%%%%%%%%%%%%%%%%%%%%

\section{Stage 7: Potential $V(q)$}

\begin{theorem}[Static Energy Functional] \tagDc
The potential arises from the static energy:
\begin{equation}
V(q) = E_{\rm static}[f(\cdot; q)] - E_0
\end{equation}
where $E_{\rm static}$ is evaluated on the equilibrium profile.
\end{theorem}

\begin{proposition}[Quartic Barrier] \tagDc
For the Gaussian profile:
\begin{equation}
\boxed{V(q) = V_B \cdot q^2(1-q)^2}
\label{eq:Vq}
\end{equation}
with barrier height $V_B$ depending on tension and geometry parameters.
\end{proposition}

\begin{proposition}[Combined Potential with Q-Value] \tagDc
Including the endpoint energy difference:
\begin{equation}
\boxed{V(q) = 16 V_B \, q^2(1-q)^2 + Q \cdot q}
\label{eq:Vq_full}
\end{equation}
where $Q = 0.782$ MeV \tagBL{} is the Q-value of neutron decay.
\end{proposition}

%%%%%%%%%%%%%%%%%%%%%%%%%%%%%%%%%%%%%%%%%%%%%%%%%%%%%%%%%%%%%%%%%%%%%%%
%% SECTION 9: FINAL EFFECTIVE ACTION
%%%%%%%%%%%%%%%%%%%%%%%%%%%%%%%%%%%%%%%%%%%%%%%%%%%%%%%%%%%%%%%%%%%%%%%

\section{Final Result: Effective 1D Action}

\begin{theorem}[Effective Action] \tagDc
The dimensional reduction yields:
\begin{equation}
\boxed{S_{\rm eff}[q] = \int dt \left( \frac{1}{2} M(q) \dot{q}^2 - V(q) \right)}
\label{eq:seff}
\end{equation}
with $M(q)$ from \eqref{eq:Mq_explicit} and $V(q)$ from \eqref{eq:Vq_full}.
\end{theorem}

\begin{tcolorbox}[colback=blue!5!white, colframe=blue!70!black]
\textbf{Key Result:} The 5D brane-world action reduces to a 1D mechanical system
with $q$-dependent mass and a quartic barrier potential. The neutron decay is
described as tunneling through this barrier.
\end{tcolorbox}

%%%%%%%%%%%%%%%%%%%%%%%%%%%%%%%%%%%%%%%%%%%%%%%%%%%%%%%%%%%%%%%%%%%%%%%
%% SECTION 10: DERIVATION CHAIN
%%%%%%%%%%%%%%%%%%%%%%%%%%%%%%%%%%%%%%%%%%%%%%%%%%%%%%%%%%%%%%%%%%%%%%%

\section{Complete Derivation Chain}

\begin{tcolorbox}[colback=gray!5!white, colframe=gray!70!black]
\begin{center}
\begin{tabular}{rcl}
$S_{\rm 5D} = S_{\rm bulk} + S_{\rm GHY} + S_{\rm brane}$ & \tagP & (full action) \\
$\downarrow$ & & \\
Warped AdS$_5$ metric & \tagP & (geometric ansatz) \\
$\downarrow$ & & \\
Brane embedding $X^A(\sigma; q)$ & \tagP & (defect parameterization) \\
$\downarrow$ & & \\
Induced metric $h_{\mu\nu}$, normal $n^A$ & \tagDer & (differential geometry) \\
$\downarrow$ & & \\
Extrinsic curvature $K_{\mu\nu}$ & \tagDer & (second fundamental form) \\
$\downarrow$ & & \\
Israel conditions & \tagDer & (junction matching) \\
$\downarrow$ & & \\
Gaussian profile $f(r;q)$ & \tagP & (shape ansatz) \\
$\downarrow$ & & \\
$M(q) \propto (1-2q)^2$ & \tagDc & (supermetric integral) \\
$\downarrow$ & & \\
$V(q) \propto q^2(1-q)^2$ & \tagDc & (static energy) \\
$\downarrow$ & & \\
$S_{\rm eff}[q] = \int(\frac{1}{2}M\dot{q}^2 - V)dt$ & \tagDc & (final result)
\end{tabular}
\end{center}
\end{tcolorbox}

%%%%%%%%%%%%%%%%%%%%%%%%%%%%%%%%%%%%%%%%%%%%%%%%%%%%%%%%%%%%%%%%%%%%%%%
%% SECTION 11: EPISTEMIC STATUS SUMMARY
%%%%%%%%%%%%%%%%%%%%%%%%%%%%%%%%%%%%%%%%%%%%%%%%%%%%%%%%%%%%%%%%%%%%%%%

\section{Epistemic Status Summary}

\begin{center}
\renewcommand{\arraystretch}{1.3}
\begin{tabular}{lcc}
\toprule
\textbf{Element} & \textbf{Status} & \textbf{Notes} \\
\midrule
Warped AdS$_5$ metric & \tagP & Consistent with 5D GR + $\Lambda_5 < 0$ \\
Brane embedding ansatz & \tagP & Y-junction model \\
Gaussian profile & \tagP & Tractable; other shapes possible \\
Induced metric calculation & \tagDer & Standard differential geometry \\
Israel junction conditions & \tagDer & Standard result \\
Supermetric $M(q)$ formula & \tagDc & Conditional on profile \\
$M(q) \propto (1-2q)^2$ & \tagDc & From Gaussian + amplitude form \\
Potential $V(q)$ formula & \tagDc & Conditional on profile \\
$V(q) \propto q^2(1-q)^2$ & \tagDc & From static energy evaluation \\
Barrier height $V_B$ & \tagOPEN & Cannot be derived from action \\
\bottomrule
\end{tabular}
\end{center}

%%%%%%%%%%%%%%%%%%%%%%%%%%%%%%%%%%%%%%%%%%%%%%%%%%%%%%%%%%%%%%%%%%%%%%%
%% SECTION 12: RELATION TO OTHER COMPANIONS
%%%%%%%%%%%%%%%%%%%%%%%%%%%%%%%%%%%%%%%%%%%%%%%%%%%%%%%%%%%%%%%%%%%%%%%

\section{Relation to Other Companions}

\begin{center}
\begin{tabular}{lcc}
\toprule
\textbf{Aspect} & \textbf{This Note} & \textbf{Other} \\
\midrule
Reduction pipeline & Full roadmap & --- \\
$L_{\rm eff}$ derivation & Summary & Companion A (detail) \\
WKB tunneling & Referenced & Companion B \\
Selection rules & Referenced & Companion D \\
\bottomrule
\end{tabular}
\end{center}

%%%%%%%%%%%%%%%%%%%%%%%%%%%%%%%%%%%%%%%%%%%%%%%%%%%%%%%%%%%%%%%%%%%%%%%
%% SECTION 13: OPEN PROBLEMS
%%%%%%%%%%%%%%%%%%%%%%%%%%%%%%%%%%%%%%%%%%%%%%%%%%%%%%%%%%%%%%%%%%%%%%%

\section{Open Problems}

\begin{enumerate}
\item \tagOPEN{} \textbf{Derive bulk geometry from EDC Plenum:} Why $\Lambda_5 < 0$?

\item \tagOPEN{} \textbf{Derive $\mathbb{Z}_2$ symmetry:} Physical origin in EDC?

\item \tagOPEN{} \textbf{Profile optimization:} Derive optimal $f(r;q)$ from variational principle.

\item \tagOPEN{} \textbf{Barrier height $V_B$:} Requires weak interaction physics (W boson).

\item \tagOPEN{} \textbf{Higher-dimensional corrections:} Beyond leading-order warping.
\end{enumerate}

%%%%%%%%%%%%%%%%%%%%%%%%%%%%%%%%%%%%%%%%%%%%%%%%%%%%%%%%%%%%%%%%%%%%%%%
%% BIBLIOGRAPHY
%%%%%%%%%%%%%%%%%%%%%%%%%%%%%%%%%%%%%%%%%%%%%%%%%%%%%%%%%%%%%%%%%%%%%%%

\printbibliography

\end{document}
