\documentclass[../main (2).tex]{subfiles}
% For standalone compilation, the following packages/commands are used
% When included via \subfile, main document's preamble is used instead

% Standalone preamble (ignored when included as subfile)
\ifx\Dc\undefined
\usepackage{amsmath,amssymb,mathtools}
\usepackage{hyperref}
\usepackage[margin=1in]{geometry}
\usepackage{booktabs}

% Classification markers
\newcommand{\Dc}{\textcolor{green!60!black}{\textbf{[Dc]}}}
\newcommand{\Pp}{\textcolor{orange}{\textbf{[P]}}}
\newcommand{\Dd}{\textcolor{blue}{\textbf{[D]}}}
\newcommand{\Mm}{\textcolor{purple}{\textbf{[M]}}}
\newcommand{\Ii}{\textcolor{gray}{\textbf{[I]}}}
\fi

\begin{document}

% ============================================================================
% APPENDIX: SU(2)^3 SYMMETRY FROM ACTION
% ============================================================================

\section{\texorpdfstring{SU(2)$^3$}{SU(2)3} Symmetry From Action}
\label{app:su2sym}

\subsection{Part A: Configuration Space}

\textbf{Definition \Dd:} The proton orientation configuration space is:
\[
Q = S^3 \times S^3 \times S^3 \cong \mathrm{SU}(2)^3
\]
with each factor representing the internal orientation of one flux tube.

\textbf{Goal:} Show that $\varepsilon(\theta)$ is SU(2)$^3$-invariant, implying $\varepsilon(\theta) = \varepsilon_0 = \mathrm{const}$.

\subsection{Part B: Route 1 --- Plenum Isotropy}

\textbf{Postulate P-isotropy \Pp:}
\[
\text{The Plenum (5D bulk medium) has no preferred internal direction.}
\]

\textbf{Proposition (Isotropy $\Rightarrow$ Invariance) \Dc:}
If the Plenum has no preferred internal direction, then the effective action for internal DOF must be invariant under internal rotations.

\textbf{Derivation chain:}
\begin{enumerate}
\item P-isotropy: No preferred direction $\Rightarrow$ no vector field breaks symmetry \Pp
\item Action on $\theta$ cannot distinguish $\theta$ from $g \cdot \theta$ \Dc
\item Fibers $F_i$ are separate (from P-local-vertex) \Dd
\item Symmetry acts independently on each fiber \Dd
\item $S_{\mathrm{eff}}[\theta]$ is SU(2)$^3$-invariant \Dc
\item $\varepsilon(\theta)$ inherits SU(2)$^3$-invariance \Dc
\end{enumerate}

\subsection{Part C: Mathematical Completion}

\textbf{M8 \Mm:} SU(2)$^3$ acts transitively on $Q = (S^3)^3$.

\textbf{M9 \Mm:} SU(2)$^3$-invariant function on $Q$ is constant.

\textbf{Conclusion \Dc:}
\[
\varepsilon(\theta) = \varepsilon_0 = \mathrm{const}
\]

\subsection{Part D: Status Change}

\begin{center}
\begin{tabular}{llll}
\toprule
Item & v8 Status & v9 Status & Change \\
\midrule
\textbf{P-SU2-sym} & \Pp{} Postulate & \Dc{} Derived & \textbf{PROMOTED} \\
P-isotropy & --- & \Pp{} NEW & More fundamental \\
Gap 3 & Open & \textbf{CLOSED} & \\
\bottomrule
\end{tabular}
\end{center}

\vspace{0.5cm}
\noindent\textit{Extended derivation notes available in the supplementary repository.}

\end{document}
