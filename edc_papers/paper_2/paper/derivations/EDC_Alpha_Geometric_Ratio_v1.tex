\documentclass[../main (2).tex]{subfiles}
% For standalone compilation, the following packages/commands are used
% When included via \subfile, main document's preamble is used instead

% Standalone preamble (ignored when included as subfile)
\ifx\Dc\undefined
\usepackage{amsmath,amssymb,mathtools}
\usepackage{hyperref}
\usepackage[margin=1in]{geometry}
\usepackage{booktabs}

% Classification markers
\newcommand{\Dc}{\textcolor{green!60!black}{\textbf{[Dc]}}}
\newcommand{\Pp}{\textcolor{orange}{\textbf{[P]}}}
\newcommand{\Dd}{\textcolor{blue}{\textbf{[D]}}}
\newcommand{\Mm}{\textcolor{purple}{\textbf{[M]}}}
\newcommand{\Ii}{\textcolor{gray}{\textbf{[I]}}}
\fi

\begin{document}

% ============================================================================
% APPENDIX: ALPHA AS GEOMETRIC RATIO
% ============================================================================

\section{Fine Structure Constant \texorpdfstring{$\alpha$}{alpha} as Geometric Ratio}
\label{app:alpha-geo}

\subsection{Physical Picture}

\Dd{} The electron is a 5D vortex with two energy components:

\begin{center}
\begin{verbatim}
     xi = 2*pi*R_xi  ---------------------------
                     |                         |
                     |   BULK ENERGY           |
                     |   E_bulk                |
                     |                         |
     xi = 0          *------ r_e ------*       <- MEMBRANE
                     |  CORE ENERGY    |
                     |  E_core         |
                     +------------------+

     alpha = E_bulk / (E_core + E_bulk)
\end{verbatim}
\end{center}

\subsection{Step 1: Core Energy (Membrane)}

\Dc{} Using $\varepsilon_{\text{membrane}} = \sigma/r_e$ and core volume $V_{\text{core}} = \frac{4\pi}{3}r_e^3$:

\[
E_{\text{core}} = \varepsilon_{\text{membrane}} \times V_{\text{core}} = \frac{\sigma}{r_e} \times \frac{4\pi r_e^3}{3}
\]

Using $\sigma = 2\pi R_\xi^2 \rho_P$ from pressure balance derivation:

\[
E_{\text{core}} = \frac{2\pi R_\xi^2 \rho_P}{r_e} \times \frac{4\pi r_e^3}{3} = \frac{8\pi^2}{3} R_\xi^2 \rho_P r_e^2
\]

\textbf{Dimension check \Mm:}
\[
[E_{\text{core}}] = [m^2] \times \left[\frac{J}{m^4}\right] \times [m^2] = J \quad \checkmark
\]

\subsection{Step 2: Bulk Energy (Flux Tube)}

\Dc{} The vortex extends as a flux tube through the compact dimension.

\textbf{Effective 4D energy density:}
\[
\varepsilon_{4D} = \int_0^{2\pi R_\xi} d\xi \, \rho_P = 2\pi R_\xi \rho_P
\]

\textbf{3D volume (core cross-section):}
\[
V_{3D} = \frac{4\pi}{3} r_e^3
\]

\textbf{Bulk energy:}
\[
E_{\text{bulk}} = \varepsilon_{4D} \times V_{3D} = 2\pi R_\xi \rho_P \times \frac{4\pi r_e^3}{3} = \frac{8\pi^2}{3} R_\xi \rho_P r_e^3
\]

\textbf{Dimension check \Mm:}
\[
[E_{\text{bulk}}] = [m] \times \left[\frac{J}{m^4}\right] \times [m^3] = J \quad \checkmark
\]

\subsection{Step 3: Energy Ratio}

\Dc{} Define $\alpha$ as the fraction of energy in the bulk:

\[
\alpha = \frac{E_{\text{bulk}}}{E_{\text{total}}} = \frac{E_{\text{bulk}}}{E_{\text{core}} + E_{\text{bulk}}}
\]

\[
= \frac{\frac{8\pi^2}{3} R_\xi \rho_P r_e^3}{\frac{8\pi^2}{3} R_\xi^2 \rho_P r_e^2 + \frac{8\pi^2}{3} R_\xi \rho_P r_e^3}
\]

\textbf{Factor out} $\frac{8\pi^2}{3} R_\xi \rho_P r_e^2$:

\[
\alpha = \frac{r_e}{R_\xi + r_e}
\]

\subsection{Main Results}

\textbf{Result 1 --- $\alpha$ as geometric ratio \Dc:}
\[
\boxed{\alpha = \frac{r_e}{R_\xi + r_e} = \frac{1}{1 + R_\xi/r_e}}
\]

\textbf{Result 2 --- Solving for $R_\xi/r_e$ \Dc:}

For $\alpha = 1/137.036$:
\[
\frac{R_\xi}{r_e} = \frac{1-\alpha}{\alpha} = \frac{1}{\alpha} - 1 = 137.036 - 1 = 136.036
\]

\[
\boxed{R_\xi \approx 136 \times r_e}
\]

\subsection{Numerical Evaluation}

Using classical electron radius $r_e = 2.818 \times 10^{-15}$ m:

\[
R_\xi = 136 \times 2.818 \times 10^{-15} \text{ m} = 3.83 \times 10^{-13} \text{ m}
\]

This is approximately 380 femtometers --- sub-nuclear scale but accessible to high-energy experiments.

\textbf{Kaluza-Klein mass scale:}
\[
m_{KK} \sim \frac{\hbar c}{R_\xi} \sim 500 \text{ MeV}
\]

This is remarkably close to meson/pion scales!

\subsection{Physical Interpretation}

The fine structure constant $\alpha$ is the ratio of:
\begin{itemize}
\item \textbf{Numerator:} Energy stored in the bulk (5th dimension extension)
\item \textbf{Denominator:} Total vortex energy (membrane core + bulk)
\end{itemize}

\textbf{Why $\alpha \ll 1$?} Because $R_\xi \gg r_e$ --- the compact dimension is much larger than the electron core radius, so most energy is in the membrane, not the bulk.

\subsection{Consistency Check}

\begin{center}
\begin{tabular}{lll}
\toprule
Quantity & Value & Source \\
\midrule
$\alpha$ (CODATA) & 1/137.036 & BL \\
$r_e$ (classical) & $2.818 \times 10^{-15}$ m & BL \\
$R_\xi$ (predicted) & $3.83 \times 10^{-13}$ m & \Dc \\
$R_\xi/r_e$ (predicted) & 136 & \Dc \\
\bottomrule
\end{tabular}
\end{center}

\subsection{Relation to \texorpdfstring{$(4\pi + 5/6)/6\pi^5$}{(4pi + 5/6)/6pi5} Formula}

The main text derives $\alpha = (4\pi + 5/6)/6\pi^5$ from geometric considerations. This section provides an \textbf{alternative interpretation}:

\[
\alpha = \frac{r_e}{R_\xi + r_e} \quad \Leftrightarrow \quad \alpha = \frac{4\pi + D_{c2}}{6\pi^5}
\]

Both give $\alpha \approx 1/137$, suggesting:
\[
\frac{r_e}{R_\xi + r_e} = \frac{4\pi + 5/6}{6\pi^5}
\]

This is a \textbf{consistency condition} between the 5D energy ratio and the topological formula.

\subsection{Status Summary}

\begin{center}
\begin{tabular}{llll}
\toprule
Component & Formula & Status & Dependency \\
\midrule
Core energy & $E_{\text{core}} = \frac{8\pi^2}{3} R_\xi^2 \rho_P r_e^2$ & \Dc & $\sigma$ derivation \\
Bulk energy & $E_{\text{bulk}} = \frac{8\pi^2}{3} R_\xi \rho_P r_e^3$ & \Dc & --- \\
\textbf{$\alpha$ formula} & $\alpha = r_e/(R_\xi + r_e)$ & \Dc & Energy ratio \\
\textbf{$R_\xi$ prediction} & $R_\xi \approx 136 \, r_e$ & \Dc & From $\alpha = 1/137$ \\
\bottomrule
\end{tabular}
\end{center}

\vspace{0.5cm}
\noindent\textit{Extended derivation notes available in the supplementary repository.}

\end{document}
