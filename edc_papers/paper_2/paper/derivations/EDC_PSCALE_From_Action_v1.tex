\documentclass[../main (2).tex]{subfiles}
% For standalone compilation, the following packages/commands are used
% When included via \subfile, main document's preamble is used instead

% Standalone preamble (ignored when included as subfile)
\ifx\Dc\undefined
\usepackage{amsmath,amssymb,mathtools}
\usepackage{hyperref}
\usepackage[margin=1in]{geometry}
\usepackage{booktabs}

% Classification markers
\newcommand{\Dc}{\textcolor{green!60!black}{\textbf{[Dc]}}}
\newcommand{\Pp}{\textcolor{orange}{\textbf{[P]}}}
\newcommand{\Dd}{\textcolor{blue}{\textbf{[D]}}}
\newcommand{\Mm}{\textcolor{purple}{\textbf{[M]}}}
\newcommand{\Ii}{\textcolor{gray}{\textbf{[I]}}}
\fi

\begin{document}

% ============================================================================
% APPENDIX: P-SCALE AND DELTA-OMEGA ANALYSIS
% ============================================================================

\section{P-scale and \texorpdfstring{$\Delta\Omega$}{Delta-Omega} Analysis}
\label{app:pscale}

\subsection{Part A: \texorpdfstring{$\Delta\Omega$}{Delta-Omega} Cancellation}

\textbf{Theorem \Dc:} $\Delta\Omega$ does not affect the mass ratio $m_p/m_e$.

\textbf{Proof:}
\begin{enumerate}
\item \textbf{Continuum:} $\Delta\Omega$ doesn't appear; $E_p = \varepsilon_0 \times (2\pi^2)^3$
\item \textbf{Discrete:} $\varepsilon_0 = \varepsilon_{\mathrm{cell}}/\Delta\Omega$ absorbs it
\item The mass ratio depends only on $\varepsilon_0/(\sigma a^2) = 1$ (P-scale)
\end{enumerate}

\textbf{Conclusion:} Gap D1 is CLOSED --- $\Delta\Omega$ is interpretive, not physical.

\subsection{Part B: P-scale Derivation}

\textbf{P-scale \Pp:} $\tau L = \sigma a^2$

\textbf{Physical meaning:}
\begin{itemize}
\item $\tau$ = string tension, $L$ = characteristic length
\item $\sigma$ = membrane tension, $a$ = core radius
\item $\tau L = \varepsilon_0$ (energy density in configuration space)
\end{itemize}

\textbf{Route summary:}
\begin{center}
\begin{tabular}{llll}
\toprule
Route & Approach & Outcome & Assumption \\
\midrule
1 & BPS/Bogomolny & FAIL & No structure found \\
2 & Force-balance & PARTIAL & Geometric factor \\
3 & Dim. reduction & PARTIAL & $R_\xi \sim a$, $L \sim a$ \\
4 & Variational & PARTIAL & $C_s = 2C_m$ \\
\bottomrule
\end{tabular}
\end{center}

\subsection{Part C: P-common-origin}

\textbf{Postulate P-common-origin \Pp:}
The membrane tension $\sigma$ and string tension $\tau$ have common physical origin:
\[
\tau = \sigma \times a, \quad L = a
\]

\textbf{Implication:}
\[
\tau L = (\sigma a) \times a = \sigma a^2 \quad \checkmark
\]

\textbf{Physical picture:} A flux tube is a membrane wrapped around the compact dimension of size $\sim a$, giving $\tau \sim \sigma \times a$.

\subsection{Part D: Status Change}

\begin{center}
\begin{tabular}{llll}
\toprule
Gap & v7 Status & v8 Status & Change \\
\midrule
\textbf{D1 ($\Delta\Omega$)} & \Pp{} Open & \textbf{CLOSED} & Absorbed/cancels \\
\textbf{6 (P-scale)} & \Pp{} Raw & \Dc{} Conditional & P-common-origin \\
\bottomrule
\end{tabular}
\end{center}

\vspace{0.5cm}
\noindent\textit{Extended derivation notes available in the supplementary repository.}

\end{document}
