\documentclass[../main (2).tex]{subfiles}
% For standalone compilation, the following packages/commands are used
% When included via \subfile, main document's preamble is used instead

% Standalone preamble (ignored when included as subfile)
\ifx\Dc\undefined
\usepackage{amsmath,amssymb,mathtools}
\usepackage{hyperref}
\usepackage[margin=1in]{geometry}
\usepackage{booktabs}

% Classification markers
\newcommand{\Dc}{\textcolor{green!60!black}{\textbf{[Dc]}}}
\newcommand{\Pp}{\textcolor{orange}{\textbf{[P]}}}
\newcommand{\Dd}{\textcolor{blue}{\textbf{[D]}}}
\newcommand{\Mm}{\textcolor{purple}{\textbf{[M]}}}
\newcommand{\Ii}{\textcolor{gray}{\textbf{[I]}}}
\fi

\begin{document}

% ============================================================================
% APPENDIX: P-JUNCTION FROM ACTION
% ============================================================================

\section{P-junction From Action}
\label{app:pjunction}

\subsection{Part A: Fiber Bundle Structure}

\textbf{Observation \Dd:} At junction point $X_J$, there are THREE distinct fibers:
\begin{itemize}
\item $F_1 = S^3$ over tube 1, containing $\theta_1$
\item $F_2 = S^3$ over tube 2, containing $\theta_2$
\item $F_3 = S^3$ over tube 3, containing $\theta_3$
\end{itemize}
These fibers are NOT canonically identified.

\textbf{Key insight:} To compare $\theta_i \in F_i$ with $\theta_j \in F_j$, one needs holonomy (parallel transport), which is intrinsically non-local.

\subsection{Part B: Route A --- Fiber Locality}

\textbf{Postulate P-local-vertex \Pp:}
\[
S_{\mathrm{junction}} = S_{\mathrm{junction}}^{\mathrm{local}}[X_J, \{\hat{n}_i\}]
\]
No holonomy/parallel transport terms linking fibers of different tubes.

\textbf{Theorem (Locality $\Rightarrow$ No $\theta$-Coupling) \Dc:}
If P-local-vertex holds, then $S_{\mathrm{junction}}$ cannot depend on any $\theta_i$.

\textbf{Proof:}
\begin{itemize}
\item Any function $f(\theta_i)$ on a single fiber is constant (SU(2) transitivity) or breaks gauge symmetry
\item Any cross-term $f(\theta_i^{-1}\theta_j)$ requires fiber identification via holonomy
\item P-local-vertex forbids holonomy terms
\item Therefore: $S_{\mathrm{junction}}$ is $\theta$-independent
\end{itemize}

\subsection{Part C: Route B --- Gauge Invariance}

\textbf{Lemma (Invariants under SU(2)$^3$) \Mm:}
A function $f: S^3 \times S^3 \times S^3 \to \mathbb{R}$ that is SU(2)$^3$-invariant must be constant.

\textbf{Result:} If $S_{\mathrm{junction}}$ is SU(2)$^3$-invariant, it is independent of all $\theta_i$.

\subsection{Part D: Status Change}

\begin{center}
\begin{tabular}{llll}
\toprule
Item & v6 Status & v7 Status & Change \\
\midrule
\textbf{P-junction} & \Pp{} Postulate & \Dc{} Derived & \textbf{PROMOTED} \\
P-local-vertex & --- & \Pp{} NEW & More fundamental \\
Q factorization & \Dc{} on P-junction & \Dc{} on P-local-vertex & Stronger foundation \\
\bottomrule
\end{tabular}
\end{center}

\vspace{0.5cm}
\noindent\textit{Extended derivation notes available in the supplementary repository.}

\end{document}
