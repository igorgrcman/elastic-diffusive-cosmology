\documentclass[../main (2).tex]{subfiles}
% For standalone compilation, the following packages/commands are used
% When included via \subfile, main document's preamble is used instead

% Standalone preamble (ignored when included as subfile)
\ifx\Dc\undefined
\usepackage{amsmath,amssymb,mathtools}
\usepackage{hyperref}
\usepackage[margin=1in]{geometry}
\usepackage{booktabs}

% Classification markers
\newcommand{\Dc}{\textcolor{green!60!black}{\textbf{[Dc]}}}
\newcommand{\Pp}{\textcolor{orange}{\textbf{[P]}}}
\newcommand{\Dd}{\textcolor{blue}{\textbf{[D]}}}
\newcommand{\Mm}{\textcolor{purple}{\textbf{[M]}}}
\newcommand{\Ii}{\textcolor{gray}{\textbf{[I]}}}
\fi

\begin{document}

% ============================================================================
% APPENDIX: MEMBRANE TENSION FROM PRESSURE BALANCE
% ============================================================================

\section{Membrane Tension \texorpdfstring{$\sigma$}{sigma} From Pressure Balance}
\label{app:sigma}

\subsection{The Physical Picture}

\Dd{} The membrane $\Sigma^4$ separates the Plenum (bulk) from the effective vacuum below. The Plenum exerts pressure; the membrane resists with tension.

\begin{center}
\begin{verbatim}
      | | |  P_bulk = rho_P c^2  | | |
    =====================================  <- Membrane at xi=0
                                             Tension sigma
      vacuum region (effective)
\end{verbatim}
\end{center}

\subsection{Dimensional Analysis}

\textbf{Plenum energy density \Dd:}
\[
[\rho_P] = \frac{\text{Energy}}{\text{Length}^4} = \frac{J}{m^4}
\]

This is a 5D energy density (energy per 4-volume).

\textbf{Membrane tension \Dd:}
\[
[\sigma] = \frac{\text{Energy}}{\text{Area}} = \frac{J}{m^2}
\]

\subsection{Step 1: Integrated Pressure}

\Mm{} The membrane feels pressure integrated over the compact dimension $\xi$:

\[
P_{\text{effective}} = \int_0^{2\pi R_\xi} \rho_P \, d\xi = 2\pi R_\xi \rho_P
\]

\textbf{Dimension check:}
\[
[P_{\text{eff}}] = [m] \times \left[\frac{J}{m^4}\right] = \frac{J}{m^3} = \text{Pa} \quad \checkmark
\]

This is proper 3D pressure!

\subsection{Step 2: Membrane Thickness}

\Pp{} \textbf{P-membrane-thickness:} The membrane has effective thickness $\delta$ of order the compact radius:
\[
\delta \sim R_\xi
\]

\textbf{Physical justification:} The membrane ``feels'' the compact dimension; its effective thickness is set by $R_\xi$.

\subsection{Step 3: Tension Formula}

\Dc{} For a membrane of thickness $\delta$:
\[
\sigma = P_{\text{effective}} \times \delta = 2\pi R_\xi \rho_P \times R_\xi
\]

\[
\boxed{\sigma = 2\pi R_\xi^2 \rho_P}
\]

\subsection{Verification}

\textbf{Dimension check \Mm:}
\[
[\sigma] = [m^2] \times \left[\frac{J}{m^4}\right] = \frac{J}{m^2} \quad \checkmark
\]

\textbf{Circularity check:}
\begin{center}
\begin{tabular}{ll}
\toprule
Quantity & Status \\
\midrule
$R_\xi$ & Input (5D geometry) \\
$\rho_P$ & Input (Plenum property) \\
$r_e$ & NOT used $\checkmark$ \\
$\alpha$ & NOT used $\checkmark$ \\
\bottomrule
\end{tabular}
\end{center}

\subsection{Physical Interpretation}

The membrane tension $\sigma$ is determined by:
\begin{enumerate}
\item The bulk energy density $\rho_P$ pressing on the membrane
\item The compact dimension size $R_\xi$ setting both the pressure integration range AND the membrane thickness
\end{enumerate}

\textbf{Key insight:} No electron properties ($r_e$, $\alpha$, $m_e$) appear in the derivation. The membrane tension is a \textbf{bulk property}, not a particle property.

\subsection{Status Summary}

\begin{center}
\begin{tabular}{llll}
\toprule
Component & Formula & Status & Dependency \\
\midrule
Effective pressure & $2\pi R_\xi \rho_P$ & \Mm & --- \\
Membrane thickness & $\delta = R_\xi$ & \Pp & P-membrane-thickness \\
\textbf{Membrane tension} & $\sigma = 2\pi R_\xi^2 \rho_P$ & \Dc & Conditional \\
\bottomrule
\end{tabular}
\end{center}

\vspace{0.5cm}
\noindent\textit{Extended derivation notes available in the supplementary repository.}

\end{document}
