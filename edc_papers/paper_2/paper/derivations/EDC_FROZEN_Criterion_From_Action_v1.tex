\documentclass[../main (2).tex]{subfiles}
% For standalone compilation, the following packages/commands are used
% When included via \subfile, main document's preamble is used instead

% Standalone preamble (ignored when included as subfile)
\ifx\Gam\undefined
\usepackage{amsmath,amssymb,mathtools}
\usepackage{hyperref}
\usepackage[margin=1in]{geometry}
\usepackage{booktabs}
\usepackage[normalem]{ulem}
\usepackage{xcolor}

% Notation macros
\newcommand{\Gam}{\Gamma}
\newcommand{\Gammax}{\Gamma_{\mathrm{max}}}
\newcommand{\Gamzero}{\Gamma_0}
\newcommand{\taurel}{\tau_{\mathrm{relax}}}
\newcommand{\tauobs}{\tau_{\mathrm{obs}}}
\newcommand{\sig}{\sigma}
\newcommand{\DelA}{\Delta A}
\newcommand{\DelS}{\Delta S}
\newcommand{\hb}{\hbar}

% Classification markers
\newcommand{\Dc}{\textcolor{green!60!black}{\textbf{[Dc]}}}
\newcommand{\Pp}{\textcolor{orange}{\textbf{[P]}}}
\newcommand{\Dd}{\textcolor{blue}{\textbf{[D]}}}
\newcommand{\Mm}{\textcolor{purple}{\textbf{[M]}}}
\newcommand{\Ii}{\textcolor{gray}{\textbf{[I]}}}
\fi

\begin{document}

% ============================================================================
% APPENDIX: FROZEN CRITERION DERIVATION FROM ACTION v1.2
% ============================================================================

\section{Frozen Criterion From Action}
\label{app:frozen-criterion}

\subsection*{Changelog v1.2}

\begin{itemize}
\item \textbf{FIX 1:} Transition rate $\Gamma$ definition corrected from $\lim_{t\to\infty}$ to $\lim_{t\to 0^+}$.
\item \textbf{FIX 2:} Route B restructured: B1 [M] (homotopy) + B2 [P] (no topology change) + B3 [Dc] (superselection).
\item \textbf{FIX 3:} Units paragraph added to Route A (natural units $\hbar = c = 1$).
\item \textbf{FIX 4:} Part I numerical check clarified as order-of-magnitude with QCD tension caveat.
\end{itemize}

% ----------------------------------------------------------------------------
\subsection{Part A: Definitions}

\textbf{Definition A.1 ($\theta$-sector) \Dd:}
A $\theta$-sector is a connected component of configuration space $Q = S^3 \times S^3 \times S^3$ distinguished by the orientation state of the three quark flux tubes.

\textbf{Definition A.2 (Transition Rate) \Dd:}
The transition rate from sector $\theta$ to sector $\theta'$ is defined via the \textbf{small-time limit}:
\[
\Gamma(\theta \to \theta') := \lim_{t \to 0^+} \frac{P(\theta' | \theta, t)}{t}
\]
Equivalently: $\Gamma = \left.\frac{d}{dt} P(\theta'|\theta,t)\right|_{t=0}$. This matches Fermi's golden rule and master equation conventions.

\textbf{Definition A.3 (Relaxation Time) \Dd:}
\[
\tau_{\mathrm{relax}} := \frac{1}{\Gamma_{\mathrm{max}}}
\]
where $\Gamma_{\mathrm{max}} := \max_{\theta,\theta'} \Gamma(\theta \to \theta')$.

\textbf{Definition A.4 (Frozen Criterion) \Dd:}
A system is frozen if:
\[
\Gamma_{\mathrm{max}} \cdot \tau_{\mathrm{obs}} \ll 1 \quad \Leftrightarrow \quad \tau_{\mathrm{relax}} \gg \tau_{\mathrm{obs}}
\]

% ----------------------------------------------------------------------------
\subsection{Part B: Route A --- Large-\texorpdfstring{$\sigma$}{sigma} Instanton Barrier}

\subsubsection*{B.0 Units and Dimensional Conventions}

\textbf{Convention \Dd:} Throughout Route A, we work in \textbf{natural units} where $\hbar = c = 1$.

In these units:
\begin{itemize}
\item Membrane tension $\sigma$ has dimensions $[\text{Energy}^2] = [\text{Length}^{-2}]$
\item Area $\Delta A$ has dimensions $[\text{Length}^2] = [\text{Energy}^{-2}]$
\item The product $\sigma \cdot \Delta A$ is \textbf{dimensionless}
\end{itemize}

\textbf{SI reference:} For QCD string tension, $\sqrt{\sigma} \approx 440$ MeV, so $\sigma \approx 0.18$ GeV$^2 \approx 0.9$ GeV/fm.

\subsubsection*{B.1 Action for Configuration Change}

\textbf{Postulate (Membrane Action) \Pp:}
The 5D EDC action for the flux tube membrane includes:
\[
S_{\mathrm{membrane}} = \sigma \int d^2\xi \sqrt{-\det h_{ab}}
\]

\textbf{Theorem (Action Difference) \Dc:}
A transition from configuration $\theta$ to $\theta'$ requires:
\[
\Delta S = \sigma \cdot \Delta A
\]

\textbf{Theorem (Semiclassical Approximation) \Mm:}
For large action barriers:
\[
\Gamma(\theta \to \theta') \sim \Gamma_0 \, e^{-\Delta S / \hbar}
\]

\textbf{Theorem (Large-$\sigma$ Freezing) \Dc:}
If $\sigma \cdot \Delta A_{\mathrm{min}} \gg \hbar$ (i.e., $\gg 1$ in natural units), then:
\[
\Gamma \ll \Gamma_0 \quad \Rightarrow \quad \tau_{\mathrm{relax}} \to \infty
\]

\textbf{Corollary (Frozen Criterion from $\sigma$) \Dc:}
\[
\boxed{\sigma \cdot \Delta A_{\mathrm{min}} > \hbar \cdot \ln(\Gamma_0 \cdot \tau_{\mathrm{obs}})}
\]

% ----------------------------------------------------------------------------
\subsection{Part C: Route B --- Topological Protection}

Route B is structured into three logically distinct parts.

\subsubsection*{C.1 Part B1: Mathematical Invariance \Mm}

\textbf{Theorem (Homotopy Invariance) \Mm:}
Winding numbers are invariants under continuous deformations. For any homotopy $\gamma_t$, the winding number $n(\gamma_t)$ is constant.

\textbf{Corollary \Mm:} Integer-valued topological charges cannot change continuously---they either remain constant or jump discontinuously.

\subsubsection*{C.2 Part B2: Physical Admissibility Postulate \Pp}

\textbf{Definition (Flux Tube Winding) \Dd:}
Each flux tube $i \in \{1,2,3\}$ carries winding number:
\[
n_i := \frac{1}{2\pi} \oint_{\gamma_i} d\phi_i \in \mathbb{Z}
\]

\textbf{Postulate (No Topology-Changing Processes) \Pp:}
During $\tau_{\mathrm{obs}}$, EDC dynamics forbids:
\begin{enumerate}
\item Membrane cutting or tearing
\item Flux tube reconnection (``string breaking'')
\item Creation/annihilation of topological defects
\end{enumerate}

\subsubsection*{C.3 Part B3: Consequence --- Superselection \Dc}

\textbf{Theorem (Topological Freezing) \Dc:}
If Postulate B2 holds, transitions between different winding sectors are forbidden:
\[
\Gamma(\theta \to \theta') = 0 \quad \text{if} \quad \mathbf{n}(\theta) \neq \mathbf{n}(\theta')
\]

\textbf{Proof:} By B1, winding changes only via discontinuous processes. By B2, such processes are forbidden. Therefore $P(\theta'|\theta,t) = 0$ for all $t$, hence $\Gamma = \lim_{t\to 0^+} 0/t = 0$. $\square$

\subsubsection*{C.4 Route B Classification}

\begin{center}
\begin{tabular}{llll}
\toprule
Step & Statement & Status & Dependencies \\
\midrule
B1 & Winding is homotopy invariant & \Mm & Algebraic topology \\
B2 & No topology change during $\tau_{\mathrm{obs}}$ & \Pp & Physical postulate \\
B3 & Different winding $\Rightarrow \Gamma = 0$ & \Dc & B1 + B2 \\
\bottomrule
\end{tabular}
\end{center}

\textbf{Key insight:} The exact $\Gamma = 0$ depends on physical postulate B2, not pure topology.

% ----------------------------------------------------------------------------
\subsection{Part D: Synthesis}

\textbf{Theorem (Frozen Criterion from Action/Topology) \Dc:}
The frozen criterion $\tau_{\mathrm{relax}} \gg \tau_{\mathrm{obs}}$ is satisfied if EITHER:

\begin{itemize}
\item \textbf{Route A:} $\sigma \cdot \Delta A_{\mathrm{min}} \gg \hbar$
\item \textbf{Route B:} Winding numbers $\mathbf{n} = (n_1, n_2, n_3)$ are conserved (B2 holds)
\end{itemize}

\textbf{Dependency Chain (v1.2):}

\begin{center}
\begin{tabular}{llll}
\toprule
Step & Statement & Status & Dependencies \\
\midrule
F1 & $\Gamma = \Gamma_0 \exp(-\Delta S/\hbar)$ & \Mm & Instanton calculus \\
F2 & $\Delta S = \sigma \cdot \Delta A$ & \Dc & Membrane action \\
F3 & $\sigma$ large $\Rightarrow \Gamma \to 0$ & \Dc & F1 + F2 \\
F4 & $\tau_{\mathrm{relax}} = 1/\Gamma \to \infty$ & \Dd & Definition \\
\textbf{F5} & \textbf{$\tau_{\mathrm{relax}} \gg \tau_{\mathrm{obs}}$ (frozen)} & \Dc & F3 + F4 \\
B1 & Winding is homotopy invariant & \Mm & Topology \\
B2 & No topology change during $\tau_{\mathrm{obs}}$ & \Pp & Physical postulate \\
\textbf{B3} & \textbf{$\mathbf{n}$ conserved $\Rightarrow \Gamma = 0$} & \Dc & B1 + B2 \\
\bottomrule
\end{tabular}
\end{center}

% ----------------------------------------------------------------------------
\subsection{Part E: Order-of-Magnitude Numerical Check}

\textbf{Purpose:} Verify Route A gives meaningful suppression. \textbf{Caveat:} Order-of-magnitude only.

\textbf{QCD string tension:} $\sqrt{\sigma} \approx 440$ MeV $\Rightarrow T \approx 0.9$ GeV/fm.

\textbf{Minimum area:} $\Delta A \sim 0.3$ fm$^2$ (flux tube orientation flip).

\textbf{Action (natural units):} $S/\hbar \approx 0.9 \times 0.3 \times 5 \approx 1.4$.

\textbf{Suppression:} $e^{-1.4} \approx 0.25$ --- \textbf{marginal} with QCD tension.

\begin{center}
\begin{tabular}{lll}
\toprule
Scenario & $S/\hbar$ & Suppression \\
\midrule
QCD tension, small $\Delta A$ & $\sim 1$--2 & Marginal (0.1--0.4) \\
Full rotation & $\sim 5$ & Moderate ($e^{-5} \approx 0.007$) \\
Large $\sigma$ (EDC) & $\gg 10$ & Exponential freezing \\
\bottomrule
\end{tabular}
\end{center}

\textbf{Conclusion:} Robust freezing requires $\sigma \gg \sigma_{\mathrm{QCD}}$ (Route A) or topological protection via B2 (Route B).

% ----------------------------------------------------------------------------
\subsection{Status Change}

\begin{center}
\begin{tabular}{llll}
\toprule
Item & v4 Status & v1.2 Status & Change \\
\midrule
\textbf{Frozen criterion} & \Pp{} Postulate & \Dc{} Derived & \textbf{PROMOTED} \\
Route B structure & Mixed & B1\Mm{} + B2\Pp{} + B3\Dc & \textbf{CLARIFIED} \\
$\Gamma$ definition & $\lim_{t\to\infty}$ & $\lim_{t\to 0^+}$ & \textbf{FIXED} \\
Dependencies & None stated & $\sigma\Delta A \gg \hbar$ OR B2 & Explicit \\
\bottomrule
\end{tabular}
\end{center}

\vspace{0.5cm}
\noindent\textit{Extended derivation notes available in the supplementary repository.}

\end{document}
