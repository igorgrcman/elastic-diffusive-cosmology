\documentclass[11pt,a4paper]{article}
\usepackage[utf8]{inputenc}
\usepackage[T1]{fontenc}
\usepackage{amsmath, amssymb, amsfonts}
\usepackage{geometry}
\usepackage{booktabs}
\usepackage{hyperref}
\usepackage{graphicx}
\usepackage{xcolor}

\geometry{margin=2.5cm}

\title{\textbf{Technical Report: Numerical Validation of the EDC River Bridge}\\ \large Mercury's Perihelion Precession via 5D Plenum Inflow}
\author{Igor Grčman}
\date{January 11, 2026}

\begin{document}

\maketitle

\begin{abstract}
This report documents the numerical validation of the Elastic-Diffusive Cosmology (EDC) framework. Using a difference-of-runs protocol within a symplectic Verlet--Binet integrator, we demonstrate that the EDC ``River Bridge'' (mapping Plenum flow to metric forms) recovers Mercury's anomalous precession with a relative deviation of 0.022\% from the General Relativity benchmark. The results confirm $O(d\phi^2)$ convergence and Hamiltonian stability at the $10^{-10}$ level.
\end{abstract}

\noindent\textbf{Identifier:} \href{https://doi.org/10.5281/zenodo.18211854}{DOI: 10.5281/zenodo.18211854}

\section{Introduction}
In the EDC model, gravity is interpreted as a stationary radial inflow of the 5D Plenum. This study validates the ``Bridge'' phase where the inflow velocity $v(r)$ is mapped to the Painlevé--Gullstrand metric, functionally equivalent to the Schwarzschild geometry in the weak-field limit.

\section{Results and Numerical Stability}
The step-refinement ladder results are summarized in Table \ref{tab:final_ladder}. We observe a clear transition from a truncation-dominated regime to a round-off limited regime at the highest resolutions.

\begin{table}[ht]
\centering
\caption{Final results of the EDC validation ladder. The anomaly is stable in the truncation-dominated regime (first three rows), while the finest steps enter a round-off limited regime (last two rows).}
\label{tab:final_ladder}
\begin{tabular}{lccccc}
\toprule
$d\phi_{\mathrm{eff}}$ [rad] & Peri & Newton [arcsec/century] & Anomaly [arcsec/century] & $\sigma_{\text{stat}}$ (95\%) & $H$-Drift \\
\midrule
$2.0 \cdot 10^{-4}$  & 270 & $-0.897390$ & $42.981552$ & $\pm 1.2\cdot 10^{-4}$ & $5.11 \cdot 10^{-10}$ \\
$1.0 \cdot 10^{-4}$  & 270 & $-0.230477$ & $42.985448$ & $\pm 9.9\cdot 10^{-4}$ & $6.98 \cdot 10^{-10}$ \\
$7.5 \cdot 10^{-5}$  & 270 & $-0.149687$ & $43.001897$ & $\pm 1.2\cdot 10^{-3}$ & $7.51 \cdot 10^{-10}$ \\
\midrule
$5.0 \cdot 10^{-5}$  & 270 & $+0.000228$ & $42.922965$ & $\pm 5.7\cdot 10^{-4}$ & $1.29 \cdot 10^{-9}$ \\
$3.3 \cdot 10^{-5}$  & 270 & $+0.014073$ & $42.932474$ & $\pm 1.0\cdot 10^{-2}$ & $1.74 \cdot 10^{-9}$ \\
\bottomrule
\end{tabular}
\end{table}

\begin{figure}[ht]
\centering
% \includegraphics[width=\textwidth]{convergence_plot.pdf}
\caption{Step-refinement behavior of the Mercury perihelion-advance estimate. The Newtonian baseline drift (left axis) converges to zero as $O(d\phi^2)$, while the Bridge--Newton anomaly (right axis) remains anchored at the relativistic scale. Error bars denote regression-propagated 95\% confidence intervals. The stability box indicates Hamiltonian invariance at the $\sim 10^{-10}$ level.}
\label{fig:convergence}
\end{figure}

\section{Conclusion}
\noindent\textbf{Conclusion.}
In summary, the EDC ``River Bridge'' model demonstrates high-fidelity numerical consistency with the General Relativity perihelion-precession benchmark. By employing a difference-of-runs protocol (Bridge minus Newton) that cancels common-mode numerical bias, and restricting the primary estimator to the truncation-dominated regime ($d\phi_{\mathrm{eff}}\ge 7.5\times 10^{-5}\,\mathrm{rad}$), we obtain a headline anomalous advance:
\begin{equation}
\Delta\varpi_{\mathrm{anom}} = 42.9896 \pm 0.0012_{\mathrm{stat}} \pm 0.0102_{\mathrm{sys}}\ \mathrm{arcsec/century},
\end{equation}
corresponding to a relative deviation of $2.23\times 10^{-4}$ (0.022\%) from the GR reference value (42.98~arcsec/century). This subtraction removes integrator- and detector-specific phase bias common to both runs, isolating only the Bridge contribution. Upon further refinement ($d\phi_{\mathrm{eff}}\lesssim 5\times 10^{-5}\,\mathrm{rad}$), the integration enters a round-off limited regime, evidenced by non-monotonic residuals and a characteristic increase in the Bridge Binet--Hamiltonian drift (from $\sim 5\times 10^{-10}$ up to $\sim 1.7\times 10^{-9}$ across the ladder). We interpret this transition as a diagnostic of the numerical noise floor rather than a change in the physical mapping. The combination of near-constant Hamiltonian invariance and the observed $O(d\phi^2)$ convergence of the Newtonian baseline confirms that the recovered anomalous advance is not a discretization artifact, but a robust feature of the EDC dynamical framework in the classical weak-field regime.

\section*{Data and Code Availability}
The source code (\texttt{edc\_validation\_v17.51.py}) and raw logs are available at \url{https://github.com/igorgrcman/edc} and archived on Zenodo under DOI: 10.5281/zenodo.18211854.

\begin{thebibliography}{9}
\bibitem{hamilton2008} A. J. S. Hamilton and J. P. Lisle, ``The river model of black holes,'' \textit{Am. J. Phys.}, 76(6), 519-532, 2008.
\bibitem{will2014} C. M. Will, ``The Confrontation between General Relativity and Experiment,'' \textit{Living Rev. Relativ.}, 17(4), 2014.
\end{thebibliography}

\end{document}
