\documentclass[11pt,a4paper]{article}
\usepackage[utf8]{inputenc}
\usepackage[T1]{fontenc}
\usepackage{amsmath, amssymb, amsfonts}
\usepackage{geometry}
\usepackage{xcolor}
\usepackage{hyperref}
\usepackage{booktabs}
\usepackage{listings}

\geometry{margin=2.5cm}

\title{\textbf{Technical Report: Numerical Validation of the EDC River Bridge}\\ \large Perihelion Precession of Mercury via 5D Plenum Inflow}
\author{Igor Grčman}
\date{January 2026}

\begin{document}

\maketitle

\begin{abstract}
This report documents the numerical validation of the Elastic-Diffusive Cosmology (EDC) gravitational framework. By modeling gravity as a stationary radial inflow of the 5D Plenum (the ``River Model''), we derive the effective spacetime interval in the Painlevé–Gullstrand form. We demonstrate that this non-diagonal flow geometry, when diagonalized under the Newtonian baseline limit, identically recovers the Schwarzschild metric. Numerical integration using a 4th-order Runge-Kutta method yields a Mercury perihelion shift of \textbf{42.9736''/cy}, matching the General Relativity benchmark within 0.02\%. 
\end{abstract}

\section{Physical Premise: The River Model}
In EDC, a mass $M$ is defined as a topological defect that induces a continuous radial inflow of the Plenum medium. Unlike General Relativity, where space is a static curved manifold, EDC treats gravity as a dynamic flow velocity field $v(r)$.

\section{Mathematical Derivation (The Bridge)}

\subsection{The Native Flow Metric}
The natural description of a stationary radial flow in a medium where local physics remains Minkowskian is the \textbf{Painlevé–Gullstrand (PG)} metric:
\begin{equation}
ds^2 = -c^2 dt^2 + (dr + v(r)dt)^2 + r^2 d\Omega^2
\end{equation}
Expanding this yields the non-diagonal form:
\begin{equation}
ds^2 = -(c^2 - v^2(r))dt^2 + 2v(r)dt\,dr + dr^2 + r^2 d\Omega^2
\end{equation}
The cross-term $2v(r)dt\,dr$ represents the physical drift of the medium relative to a stationary observer.

\subsection{Baseline Identification}
To recover the classical limit, we identify the inflow velocity $v(r)$ with the Newtonian escape velocity:
\begin{equation}
v^2(r) = \frac{2GM}{r}
\end{equation}

\subsection{Diagonalization (Schwarzschild Equivalence)}
By redefining time to account for the flow-induced delay, $dt_S = dt + \frac{v(r)}{c^2 - v^2(r)}dr$, the metric is transformed into the standard diagonal Schwarzschild form:
\begin{equation}
ds^2 = -\left(1 - \frac{2GM}{rc^2}\right)c^2 dt_S^2 + \left(1 - \frac{2GM}{rc^2}\right)^{-1} dr^2 + r^2 d\Omega^2
\end{equation}



\section{Numerical Verification}
The orbital dynamics are governed by the Binet equation derived from the EDC PG Bridge:
\begin{equation}
\frac{d^2u}{d\phi^2} + u = \frac{GM}{L^2} + \frac{3GM}{c^2}u^2
\end{equation}
where $u = 1/r$. The term $\frac{3GM}{c^2}u^2$ accounts for the nonlinear interaction of the orbital velocity with the Plenum inflow.

\subsection{Results}
Using an RK4 integrator with a step size of $\Delta\phi = 5 \times 10^{-6}$ and quadratic peak interpolation for perihelion detection, the following results were obtained:

\begin{table}[h]
\centering
\begin{tabular}{ll}
\toprule
\textbf{Parameter} & \textbf{Value} \\
\midrule
EDC Native Result & 42.9736 arcsec/century \\
General Relativity Benchmark & 42.9800 arcsec/century \\
\textbf{Relative Accuracy} & \textbf{99.985\%} \\
\bottomrule
\end{tabular}
\caption{Comparison of EDC Native Precession against GR Benchmark.}
\end{table}



\section{Conclusion}
The EDC River Bridge provides a robust, fluid-dynamic interpretation of gravitational anomalies. The precise recovery of Mercury's precession confirms that the Plenum inflow model is a numerically sound and theoretically consistent alternative to the static curvature paradigm of standard General Relativity.

\section*{Availability}
The Python source code (\texttt{edc\_mercury\_native\_v1.3.py}) used for this validation is available in the EDC open-source repository at \href{https://github.com/igorgrcman/elastic-diffusive-cosmology}{github.com/igorgrcman/edc}.

\end{document}
