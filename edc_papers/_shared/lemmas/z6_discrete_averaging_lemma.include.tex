% AUTO-GENERATED include from edc_papers/_shared/lemmas/z6_discrete_averaging_lemma.tex
% Do not edit this file directly; edit the standalone source instead.
% Generated by generate_include_files.py

%=============================================================================
\section{Setup}
%=============================================================================

\begin{definition}[Z$_N$ Ring Points]
\label{def:zn-points}
For a discrete cyclic group $\mathbb{Z}_N$, define the $N$ uniformly spaced points on the circle:
\begin{equation}
\theta_n = \frac{2\pi n}{N}, \quad n = 0, 1, \ldots, N-1
\end{equation}
For $N = 6$ (Z$_6$): $\theta_n \in \{0, \pi/3, 2\pi/3, \pi, 4\pi/3, 5\pi/3\}$.
\end{definition}

\begin{definition}[Discrete and Continuum Averages]
\label{def:averages}
For a function $f: [0, 2\pi) \to \mathbb{R}$:
\begin{align}
\langle f \rangle_{\text{disc}} &:= \frac{1}{N} \sum_{n=0}^{N-1} f(\theta_n) \quad \text{(discrete average)} \\
\langle f \rangle_{\text{cont}} &:= \frac{1}{2\pi} \int_0^{2\pi} f(\theta)\, d\theta \quad \text{(continuum average)}
\end{align}
\end{definition}

\begin{definition}[Discrete-to-Continuum Ratio]
\label{def:ratio}
\begin{equation}
R[f] := \frac{\langle f \rangle_{\text{disc}}}{\langle f \rangle_{\text{cont}}}
\end{equation}
\end{definition}

%=============================================================================
\section{Test Function Classes}
%=============================================================================

\subsection{Class A: Z$_N$-Symmetric Fourier Mode}

The natural test function capturing $\mathbb{Z}_N$ anisotropy is:
\begin{equation}
\label{eq:test-function-A}
f_A(\theta; c, a) = c + a \cos(N\theta)
\end{equation}
where $c > 0$ is the mean level and $a$ is the anisotropy amplitude.

\textbf{Properties:}
\begin{itemize}
\item At Z$_N$ points: $\cos(N\theta_n) = \cos(2\pi n) = 1$ for all $n$
\item Between Z$_N$ points: $\cos(N\theta)$ oscillates, with minima at $\theta = \pi(2k+1)/N$
\item Full period contains exactly $N$ oscillations
\end{itemize}

\subsection{Class B: Localized Corner Bumps}

Alternative model with Gaussian bumps at corners:
\begin{equation}
\label{eq:test-function-B}
f_B(\theta; c, w, \sigma) = c + \frac{w}{N} \sum_{n=0}^{N-1} \exp\left(-\frac{(\theta - \theta_n)^2}{2\sigma^2}\right)
\end{equation}
where $w$ is the total corner weight and $\sigma$ is the bump width.

\textbf{Note:} In the limit $\sigma \to 0$, this becomes singular and requires careful regularization. We focus on Class A for the clean result.

%=============================================================================
\section{Main Calculation: Class A}
%=============================================================================

\subsection{Discrete Average}

For $f_A(\theta) = c + a\cos(N\theta)$:
\begin{align}
\langle f_A \rangle_{\text{disc}} &= \frac{1}{N} \sum_{n=0}^{N-1} \left[c + a\cos(N\theta_n)\right] \\
&= \frac{1}{N} \sum_{n=0}^{N-1} \left[c + a\cos(2\pi n)\right] \\
&= \frac{1}{N} \sum_{n=0}^{N-1} \left[c + a \cdot 1\right] \\
&= c + a
\end{align}

\subsection{Continuum Average}

\begin{align}
\langle f_A \rangle_{\text{cont}} &= \frac{1}{2\pi} \int_0^{2\pi} \left[c + a\cos(N\theta)\right] d\theta \\
&= c + \frac{a}{2\pi} \left[\frac{\sin(N\theta)}{N}\right]_0^{2\pi} \\
&= c + \frac{a}{2\pi} \cdot \frac{1}{N}[\sin(2\pi N) - \sin(0)] \\
&= c + 0 = c
\end{align}

\subsection{Ratio}

\begin{equation}
\label{eq:ratio-general}
R[f_A] = \frac{c + a}{c} = 1 + \frac{a}{c}
\end{equation}

%=============================================================================
\section{The Natural Normalization: $a/c = 1/N$}
%=============================================================================

\subsection{Equal Corner Share Condition}

\begin{definition}[Equal Corner Share Normalization]
\label{def:equal-corner}
The anisotropy amplitude $a$ satisfies the \emph{equal corner share} condition if each of the $N$ corners contributes equally to the total anisotropy, with each share being $1/N$ of the mean:
\begin{equation}
\frac{a}{c} = \frac{1}{N}
\end{equation}
\end{definition}

\textbf{Physical interpretation:} If a Z$_N$-symmetric quantity has ``corner corrections'' distributed equally over $N$ corners, the correction per corner relative to the mean is $1/N$.

\subsection{Function Values Under This Normalization}

With $a = c/N$, the test function becomes:
\begin{equation}
f_A(\theta) = c\left(1 + \frac{1}{N}\cos(N\theta)\right)
\end{equation}

Values at special points:
\begin{itemize}
\item At corners ($\theta = \theta_n$): $f = c(1 + 1/N) = c \cdot \frac{N+1}{N}$
\item At mid-edges ($\theta = \theta_n + \pi/N$): $f = c(1 - 1/N) = c \cdot \frac{N-1}{N}$
\item Corner-to-midedge ratio: $\frac{N+1}{N-1}$
\end{itemize}

For $N = 6$: corners have $f = 7c/6$, mid-edges have $f = 5c/6$, ratio = $7/5$.

%=============================================================================
\section{Main Result}
%=============================================================================

\begin{lemma}[Z$_N$ Discrete Averaging Correction]
\label{lem:zn-correction}
Let $f(\theta) = c + a\cos(N\theta)$ be a function with $\mathbb{Z}_N$ symmetry on the circle.
Under the equal corner share normalization $a/c = 1/N$, the discrete-to-continuum averaging ratio is:
\begin{equation}
\boxed{R = \frac{\langle f \rangle_{\text{disc}}}{\langle f \rangle_{\text{cont}}} = 1 + \frac{1}{N}}
\end{equation}
\end{lemma}

\begin{proof}
From equations \eqref{eq:ratio-general} and Definition~\ref{def:equal-corner}:
\[
R = 1 + \frac{a}{c} = 1 + \frac{1}{N}
\]
\end{proof}

\begin{corollary}[Z$_6$ Correction Factor]
\label{cor:z6-correction}
For $\mathbb{Z}_6$ ($N = 6$) under equal corner share normalization:
\begin{equation}
\boxed{R = 1 + \frac{1}{6} = \frac{7}{6}}
\end{equation}
\end{corollary}

%=============================================================================
\section{Discussion}
%=============================================================================

\subsection{When Does This Apply?}

The lemma applies when:
\begin{enumerate}
\item The physical quantity has Z$_N$ anisotropy (enhanced at corners, suppressed between)
\item The anisotropy is well-approximated by the fundamental $\cos(N\theta)$ mode
\item The corner excess relative to mean equals $1/N$ (equal corner share)
\end{enumerate}

\subsection{Limitations}

\begin{remark}[Normalization Dependence]
The factor $7/6$ is \emph{not universal} -- it requires the specific normalization $a/c = 1/N$. If the physical situation has a different anisotropy amplitude, the ratio will differ.
\end{remark}

\begin{remark}[Higher Harmonics]
If $f(\theta)$ contains higher harmonics $\cos(kN\theta)$ for $k > 1$, these also contribute to the discrete average but not the continuum average, potentially modifying $R$.
\end{remark}

\subsection{Application to EDC}

In EDC, the Z$_6$ ring structure appears in:
\begin{itemize}
\item Neutron-proton mass difference (Z$_6$ elastic energy)
\item Generation counting ($N_g = |$Z$_3| = 3$)
\item Weinberg angle ($\sin^2\theta_W = |$Z$_2|/|$Z$_6|$)
\end{itemize}

The pion splitting observation $r_\pi/(4\alpha) \approx 7/6$ suggests that the EM dressing of Z$_6$-symmetric quantities may carry the equal corner share correction.

\subsection{Epistemic Status}

\begin{itemize}
\item \textbf{Mathematical result:} [Der] -- the lemma is a straightforward calculation
\item \textbf{Physical application:} [Dc] -- the equal corner share normalization is a hypothesis
\item \textbf{Pion match:} [I] -- the $7/6$ factor is identified, not derived from first principles
\end{itemize}

%=============================================================================
\section{Summary}
%=============================================================================

The discrete-to-continuum correction factor $k = 1 + 1/N$ arises naturally when:
\begin{enumerate}
\item A Z$_N$-symmetric function is sampled at the $N$ corner points
\item The Z$_N$ anisotropy (fundamental Fourier mode) has amplitude $a = c/N$
\item The continuum average washes out this anisotropy; the discrete average preserves it
\end{enumerate}

For Z$_6$: $k = 7/6$, matching the pion splitting observation.

\vspace{1em}
\hrule
\vspace{0.5em}
\textit{File: edc\_papers/\_shared/lemmas/z6\_discrete\_averaging\_lemma.tex} \\
\textit{Cross-reference: docs/Z6\_CORRECTION\_FACTOR\_7over6.md}
