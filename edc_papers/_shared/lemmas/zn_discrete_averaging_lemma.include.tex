% =============================================================================
% EXTRACTED INCLUDE BODY — AUTO-GENERATED
% =============================================================================
% Source: edc_papers/_shared/lemmas/zn_discrete_averaging_lemma.tex (standalone derivation document)
% This file: Includable body extract for Book2 Derivation Library
%
% DO NOT EDIT THIS FILE DIRECTLY.
% Edit the standalone source (zn_discrete_averaging_lemma.tex) instead, then regenerate.
%
% Labels are prefixed with DL:<filename>: to avoid collisions in Book2.
% Generated by: edc_book_2/tools/generate_include_files.py
% =============================================================================

%=============================================================================
\section{Definitions}
%=============================================================================

\begin{definition}[Z$_N$ Sampling Points]
\label{DL:zn-discrete-averagin:def:zn-points}
For cyclic group $\mathbb{Z}_N$ of order $N$, define $N$ uniformly spaced points:
\begin{equation}
\theta_n = \frac{2\pi n}{N}, \quad n = 0, 1, \ldots, N-1
\end{equation}
\end{definition}

\begin{definition}[Discrete and Continuum Averages]
\label{DL:zn-discrete-averagin:def:averages}
For $f: [0, 2\pi) \to \mathbb{R}$:
\begin{align}
\langle f \rangle_{\text{disc}}^{(N)} &:= \frac{1}{N} \sum_{n=0}^{N-1} f(\theta_n) \\[6pt]
\langle f \rangle_{\text{cont}} &:= \frac{1}{2\pi} \int_0^{2\pi} f(\theta)\, d\theta
\end{align}
\end{definition}

\begin{definition}[Discrete-to-Continuum Ratio]
\label{DL:zn-discrete-averagin:def:ratio}
\begin{equation}
R_N[f] := \frac{\langle f \rangle_{\text{disc}}^{(N)}}{\langle f \rangle_{\text{cont}}}
\end{equation}
\end{definition}

%=============================================================================
\section{Main Result (Mathematical) \texorpdfstring{$[$Der$]$}{[Der]}}
%=============================================================================

\begin{lemma}[Z$_N$ Discrete Averaging — Mathematical Part]
\label{DL:zn-discrete-averagin:lem:zn-mathematical}
Let $f(\theta) = c + a\cos(N\theta)$ with $c > 0$. Then:
\begin{equation}
\boxed{R_N[f] = 1 + \frac{a}{c}}
\end{equation}
\end{lemma}

\begin{proof}
\textbf{Discrete average:}
At $\theta_n = 2\pi n/N$: $\cos(N\theta_n) = \cos(2\pi n) = 1$ for all $n \in \mathbb{Z}$.
\[
\langle f \rangle_{\text{disc}}^{(N)} = \frac{1}{N}\sum_{n=0}^{N-1}(c + a) = c + a
\]

\textbf{Continuum average:}
\[
\langle f \rangle_{\text{cont}} = \frac{1}{2\pi}\int_0^{2\pi}(c + a\cos N\theta)\,d\theta = c + \frac{a}{2\pi N}[\sin N\theta]_0^{2\pi} = c
\]

\textbf{Ratio:}
\[
R_N = \frac{c + a}{c} = 1 + \frac{a}{c}
\]
\end{proof}

%=============================================================================
\section{Normalization Hypothesis (Physical) \texorpdfstring{$[$Dc$]$}{[Dc]}}
%=============================================================================

\begin{conjecture}[Equal Corner Share Normalization]
\label{DL:zn-discrete-averagin:hyp:normalization}
For physical quantities with $\mathbb{Z}_N$ symmetry, the anisotropy amplitude satisfies:
\begin{equation}
\frac{a}{c} = \frac{1}{N}
\end{equation}
\textbf{Interpretation:} Each of the $N$ corners contributes equally, with each share being $1/N$ of the mean.
\end{conjecture}

\textbf{Status:} This hypothesis is \textbf{not derived} from the 5D action. It is supported by the pion splitting observation but remains $[$Dc$]$.

%=============================================================================
\section{Combined Result}
%=============================================================================

\begin{corollary}[Z$_N$ Correction Factor]
\label{DL:zn-discrete-averagin:cor:zn-factor}
Under Hypothesis~\ref{DL:zn-discrete-averagin:hyp:normalization}:
\begin{equation}
\boxed{k(N) := R_N = 1 + \frac{1}{N}}
\end{equation}
\end{corollary}

\textbf{Epistemic decomposition:}
\begin{itemize}
\item $R_N = 1 + a/c$ \hfill $[$Der$]$
\item $a/c = 1/N$ \hfill $[$Dc$]$
\item $k(N) = 1 + 1/N$ \hfill $[$Der$]+[$Dc$]$
\end{itemize}

%=============================================================================
\section{Specializations}
%=============================================================================

\begin{center}
\begin{tabular}{c|c|l}
$N$ & $k(N) = 1 + 1/N$ & Fraction \\ \hline
3 & $4/3 = 1.333$ & Z$_3$ (flavor) \\
4 & $5/4 = 1.250$ & Z$_4$ (Dirac) \\
6 & $7/6 = 1.167$ & Z$_6$ (ring) \\
12 & $13/12 = 1.083$ & Z$_{12}$ (HCP?) \\
\end{tabular}
\end{center}

\textbf{Z$_6$ recovery:} Setting $N = 6$ gives $k(6) = 7/6$, matching the pion observation.

%=============================================================================
\section{Physical Interpretation}
%=============================================================================

The factor $k(N)$ arises because:
\begin{enumerate}
\item The $\cos(N\theta)$ Fourier mode is \textbf{invisible} to the continuum average (integrates to zero).
\item The discrete average \textbf{samples at maxima} of this mode (where $\cos = 1$).
\item The ``excess'' seen by discrete sampling is $a/c = 1/N$ under the equal corner share hypothesis.
\end{enumerate}

\vspace{1em}
\hrule
\vspace{0.5em}
\textit{File: edc\_papers/\_shared/lemmas/zn\_discrete\_averaging\_lemma.tex} \\
\textit{Specialization: z6\_discrete\_averaging\_lemma.tex}
