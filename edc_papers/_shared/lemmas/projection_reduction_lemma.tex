% ═══════════════════════════════════════════════════════════════════════════════
% PROJECTION-REDUCTION LEMMA
% ═══════════════════════════════════════════════════════════════════════════════
%
% File: projection_reduction_lemma.tex
% Created: 2026-01-29
% Status: CANONICAL
% Epistemic: [Der] - Derived from bulk→brane geometry
%
% Usage: % ═══════════════════════════════════════════════════════════════════════════════
% PROJECTION-REDUCTION LEMMA
% ═══════════════════════════════════════════════════════════════════════════════
%
% File: projection_reduction_lemma.tex
% Created: 2026-01-29
% Status: CANONICAL
% Epistemic: [Der] - Derived from bulk→brane geometry
%
% Usage: % ═══════════════════════════════════════════════════════════════════════════════
% PROJECTION-REDUCTION LEMMA
% ═══════════════════════════════════════════════════════════════════════════════
%
% File: projection_reduction_lemma.tex
% Created: 2026-01-29
% Status: CANONICAL
% Epistemic: [Der] - Derived from bulk→brane geometry
%
% Usage: % ═══════════════════════════════════════════════════════════════════════════════
% PROJECTION-REDUCTION LEMMA
% ═══════════════════════════════════════════════════════════════════════════════
%
% File: projection_reduction_lemma.tex
% Created: 2026-01-29
% Status: CANONICAL
% Epistemic: [Der] - Derived from bulk→brane geometry
%
% Usage: \input{edc_papers/_shared/lemmas/projection_reduction_lemma.tex}
%
% Dependencies: amsmath, amsthm, physics (optional)
%
% ═══════════════════════════════════════════════════════════════════════════════

% -------------------------------------------------------------------------------
% SETUP (assumes theorem environments defined in parent document)
% -------------------------------------------------------------------------------
% If not defined, uncomment:
% \newtheorem{lemma}{Lemma}
% \newtheorem{corollary}{Corollary}
% \theoremstyle{definition}
% \newtheorem{definition}{Definition}

% -------------------------------------------------------------------------------
% DEFINITION: PROJECTION OPERATOR
% -------------------------------------------------------------------------------
\begin{definition}[Brane Projection Operator]
\label{def:projection-operator}
Let $\Phi(x,\chi)$ be a bulk field where $x \in \mathbb{R}^{3,1}$ (4D spacetime)
and $\chi$ is the extra-dimensional coordinate. Let the brane have localization
profile $w(\chi) \geq 0$ normalized as $\int d\chi\, w(\chi) = 1$.

The \textbf{projection operator} $\mathcal{P}_w$ is defined by:
\begin{equation}
\label{eq:projection-operator}
\phi(x) := (\mathcal{P}_w \Phi)(x) = \int d\chi\, w(\chi)\, \Phi(x,\chi)
\end{equation}

For any bulk quantity $F(x,\chi)$, the \textbf{projected expectation} is:
\begin{equation}
\label{eq:projected-expectation}
\langle F \rangle_w(x) := \int d\chi\, w(\chi)\, F(x,\chi)
\end{equation}
\end{definition}

% -------------------------------------------------------------------------------
% MAIN LEMMA
% -------------------------------------------------------------------------------
\begin{lemma}[Projection-Reduction Principle]
\label{lem:projection-reduction}
Bulk $\to$ brane observation is linear projection. All 4D observables are
weighted averages of bulk structure. This principle has three universal
manifestations:
\end{lemma}

% -------------------------------------------------------------------------------
% CASE (A): EFFECTIVE LAGRANGIAN
% -------------------------------------------------------------------------------
\paragraph{Case (A): Reduced Effective Lagrangian.}
\label{par:case-a-lagrangian}

If the bulk action is:
\begin{equation}
\label{eq:bulk-action}
S[\Phi] = \int d^4x\, d\chi \left[ \frac{1}{2} K(\chi)(\partial\Phi)^2 - U(\Phi,\chi) \right]
\end{equation}

Then for low-mode dynamics $\Phi(x,\chi) \approx \phi(x) f(\chi)$ with localized $f$:
\begin{equation}
\label{eq:effective-action}
S_{\text{eff}}[\phi] = \int d^4x \left[ \frac{1}{2} Z (\partial\phi)^2 - V_{\text{eff}}(\phi) \right]
\end{equation}

where the \textbf{effective coefficients} are integrals:
\begin{align}
\label{eq:wave-renormalization}
Z &= \int d\chi\, K(\chi)\, f(\chi)^2 \\[6pt]
\label{eq:effective-potential}
V_{\text{eff}}(\phi) &= \int d\chi\, U(\phi f(\chi), \chi)
\end{align}

\textit{Interpretation:} All bulk details (geometry, tension, brane thickness)
enter only through integral weights.

% -------------------------------------------------------------------------------
% CASE (B): CHIRALITY SELECTION
% -------------------------------------------------------------------------------
\paragraph{Case (B): Chirality as Projection Selection.}
\label{par:case-b-chirality}

Let bulk fermion $\Psi(x,\chi)$ have different localization profiles for
left/right components:
\begin{align}
\label{eq:left-projection}
\psi_L(x) &= \int d\chi\, w_L(\chi)\, \Psi_L(x,\chi) \\[6pt]
\label{eq:right-projection}
\psi_R(x) &= \int d\chi\, w_R(\chi)\, \Psi_R(x,\chi)
\end{align}

Define the \textbf{chirality overlap}:
\begin{equation}
\label{eq:chirality-overlap}
\varepsilon := \int d\chi\, w_L(\chi)\, w_R(\chi)
\end{equation}

\textbf{Chirality Selection Criterion:}
\begin{equation}
\label{eq:va-criterion}
\boxed{\varepsilon \ll 1 \quad \Longrightarrow \quad \text{Effective theory is dominantly chiral (V--A like)}}
\end{equation}

\textit{Interpretation:} Chirality emerges geometrically from overlap suppression,
without invoking specific gauge structure.

% -------------------------------------------------------------------------------
% CASE (C): BARRIER AND TUNNELING
% -------------------------------------------------------------------------------
\paragraph{Case (C): Effective Barrier and Tunneling.}
\label{par:case-c-barrier}

Let reaction coordinate $q$ (topological deformation) have $\chi$-dependent
potential $V(q,\chi)$.

The \textbf{projected potential} is:
\begin{equation}
\label{eq:projected-potential}
V_{\text{eff}}(q) = \int d\chi\, w(\chi)\, V(q,\chi)
\end{equation}

If bulk has pinning energy $+\kappa(\chi) q^2$:
\begin{equation}
\label{eq:pinning-potential}
V_{\text{eff}}(q) = V_0(q) + \langle\kappa\rangle_w\, q^2
\end{equation}

where the \textbf{effective pinning constant} is:
\begin{equation}
\label{eq:effective-pinning}
\kappa_{\text{eff}} := \langle\kappa\rangle_w = \int d\chi\, w(\chi)\, \kappa(\chi) > 0
\quad \Longrightarrow \quad \Delta V_{\text{barrier}} > 0
\end{equation}

\textit{Interpretation:} WKB tunneling exponents become functions of projected
parameters.

% -------------------------------------------------------------------------------
% UNIVERSAL CONSEQUENCES
% -------------------------------------------------------------------------------
\paragraph{Three Universal Consequences.}
\begin{enumerate}
    \item \textbf{Effective coefficients are integrals:} $Z$, $\kappa_{\text{eff}}$,
          $V_{\text{eff}}$ all arise from $\int d\chi\, w(\chi) \times (\text{bulk quantity})$
    \item \textbf{Chirality is geometrically selected:} $\varepsilon \ll 1$
          produces V--A without fine-tuning
    \item \textbf{Barriers are projections:} Tunneling rates depend on
          projected energy profiles
\end{enumerate}

% -------------------------------------------------------------------------------
% EDC APPLICATION COROLLARY
% -------------------------------------------------------------------------------
\begin{corollary}[EDC Breadth Mapping]
\label{cor:edc-breadth}
In Elastic Diffusive Cosmology, three distinct phenomena are manifestations
of the same Projection-Reduction Principle:

\begin{center}
\begin{tabular}{lll}
\hline
\textbf{EDC Result} & \textbf{Lemma Case} & \textbf{Key Parameter} \\
\hline
EM projection & Case (A) & $Z = \int K f^2$ \\
V--A from boundary & Case (B) with $\varepsilon \ll 1$ & $\varepsilon$ \\
Nuclear tunneling / pinning & Case (C) & $\kappa_{\text{eff}}$ \\
\hline
\end{tabular}
\end{center}

\textit{Canonical statement:} We adopt a single projection-reduction principle.
EM, chiral weak structure, and nuclear barrier tunneling appear as different
sectoral manifestations of the same bulk$\to$brane projection operator.
\end{corollary}

% -------------------------------------------------------------------------------
% CROSS-SECTOR BRIDGE
% -------------------------------------------------------------------------------
\paragraph{Cross-Sector Power.}
\label{par:cross-sector}

This lemma connects three sectors with one formalism:
\[
\text{EM} \longleftrightarrow \text{Weak} \longleftrightarrow \text{Nuclear}
\quad \text{(via } \mathcal{P}_w \text{)}
\]

Status: \textbf{[Der]} for individual cases; \textbf{[P]} for universal operator
unification (requires formal proof of uniqueness).

% ═══════════════════════════════════════════════════════════════════════════════
% END OF LEMMA
% ═══════════════════════════════════════════════════════════════════════════════

%
% Dependencies: amsmath, amsthm, physics (optional)
%
% ═══════════════════════════════════════════════════════════════════════════════

% -------------------------------------------------------------------------------
% SETUP (assumes theorem environments defined in parent document)
% -------------------------------------------------------------------------------
% If not defined, uncomment:
% \newtheorem{lemma}{Lemma}
% \newtheorem{corollary}{Corollary}
% \theoremstyle{definition}
% \newtheorem{definition}{Definition}

% -------------------------------------------------------------------------------
% DEFINITION: PROJECTION OPERATOR
% -------------------------------------------------------------------------------
\begin{definition}[Brane Projection Operator]
\label{def:projection-operator}
Let $\Phi(x,\chi)$ be a bulk field where $x \in \mathbb{R}^{3,1}$ (4D spacetime)
and $\chi$ is the extra-dimensional coordinate. Let the brane have localization
profile $w(\chi) \geq 0$ normalized as $\int d\chi\, w(\chi) = 1$.

The \textbf{projection operator} $\mathcal{P}_w$ is defined by:
\begin{equation}
\label{eq:projection-operator}
\phi(x) := (\mathcal{P}_w \Phi)(x) = \int d\chi\, w(\chi)\, \Phi(x,\chi)
\end{equation}

For any bulk quantity $F(x,\chi)$, the \textbf{projected expectation} is:
\begin{equation}
\label{eq:projected-expectation}
\langle F \rangle_w(x) := \int d\chi\, w(\chi)\, F(x,\chi)
\end{equation}
\end{definition}

% -------------------------------------------------------------------------------
% MAIN LEMMA
% -------------------------------------------------------------------------------
\begin{lemma}[Projection-Reduction Principle]
\label{lem:projection-reduction}
Bulk $\to$ brane observation is linear projection. All 4D observables are
weighted averages of bulk structure. This principle has three universal
manifestations:
\end{lemma}

% -------------------------------------------------------------------------------
% CASE (A): EFFECTIVE LAGRANGIAN
% -------------------------------------------------------------------------------
\paragraph{Case (A): Reduced Effective Lagrangian.}
\label{par:case-a-lagrangian}

If the bulk action is:
\begin{equation}
\label{eq:bulk-action}
S[\Phi] = \int d^4x\, d\chi \left[ \frac{1}{2} K(\chi)(\partial\Phi)^2 - U(\Phi,\chi) \right]
\end{equation}

Then for low-mode dynamics $\Phi(x,\chi) \approx \phi(x) f(\chi)$ with localized $f$:
\begin{equation}
\label{eq:effective-action}
S_{\text{eff}}[\phi] = \int d^4x \left[ \frac{1}{2} Z (\partial\phi)^2 - V_{\text{eff}}(\phi) \right]
\end{equation}

where the \textbf{effective coefficients} are integrals:
\begin{align}
\label{eq:wave-renormalization}
Z &= \int d\chi\, K(\chi)\, f(\chi)^2 \\[6pt]
\label{eq:effective-potential}
V_{\text{eff}}(\phi) &= \int d\chi\, U(\phi f(\chi), \chi)
\end{align}

\textit{Interpretation:} All bulk details (geometry, tension, brane thickness)
enter only through integral weights.

% -------------------------------------------------------------------------------
% CASE (B): CHIRALITY SELECTION
% -------------------------------------------------------------------------------
\paragraph{Case (B): Chirality as Projection Selection.}
\label{par:case-b-chirality}

Let bulk fermion $\Psi(x,\chi)$ have different localization profiles for
left/right components:
\begin{align}
\label{eq:left-projection}
\psi_L(x) &= \int d\chi\, w_L(\chi)\, \Psi_L(x,\chi) \\[6pt]
\label{eq:right-projection}
\psi_R(x) &= \int d\chi\, w_R(\chi)\, \Psi_R(x,\chi)
\end{align}

Define the \textbf{chirality overlap}:
\begin{equation}
\label{eq:chirality-overlap}
\varepsilon := \int d\chi\, w_L(\chi)\, w_R(\chi)
\end{equation}

\textbf{Chirality Selection Criterion:}
\begin{equation}
\label{eq:va-criterion}
\boxed{\varepsilon \ll 1 \quad \Longrightarrow \quad \text{Effective theory is dominantly chiral (V--A like)}}
\end{equation}

\textit{Interpretation:} Chirality emerges geometrically from overlap suppression,
without invoking specific gauge structure.

% -------------------------------------------------------------------------------
% CASE (C): BARRIER AND TUNNELING
% -------------------------------------------------------------------------------
\paragraph{Case (C): Effective Barrier and Tunneling.}
\label{par:case-c-barrier}

Let reaction coordinate $q$ (topological deformation) have $\chi$-dependent
potential $V(q,\chi)$.

The \textbf{projected potential} is:
\begin{equation}
\label{eq:projected-potential}
V_{\text{eff}}(q) = \int d\chi\, w(\chi)\, V(q,\chi)
\end{equation}

If bulk has pinning energy $+\kappa(\chi) q^2$:
\begin{equation}
\label{eq:pinning-potential}
V_{\text{eff}}(q) = V_0(q) + \langle\kappa\rangle_w\, q^2
\end{equation}

where the \textbf{effective pinning constant} is:
\begin{equation}
\label{eq:effective-pinning}
\kappa_{\text{eff}} := \langle\kappa\rangle_w = \int d\chi\, w(\chi)\, \kappa(\chi) > 0
\quad \Longrightarrow \quad \Delta V_{\text{barrier}} > 0
\end{equation}

\textit{Interpretation:} WKB tunneling exponents become functions of projected
parameters.

% -------------------------------------------------------------------------------
% UNIVERSAL CONSEQUENCES
% -------------------------------------------------------------------------------
\paragraph{Three Universal Consequences.}
\begin{enumerate}
    \item \textbf{Effective coefficients are integrals:} $Z$, $\kappa_{\text{eff}}$,
          $V_{\text{eff}}$ all arise from $\int d\chi\, w(\chi) \times (\text{bulk quantity})$
    \item \textbf{Chirality is geometrically selected:} $\varepsilon \ll 1$
          produces V--A without fine-tuning
    \item \textbf{Barriers are projections:} Tunneling rates depend on
          projected energy profiles
\end{enumerate}

% -------------------------------------------------------------------------------
% EDC APPLICATION COROLLARY
% -------------------------------------------------------------------------------
\begin{corollary}[EDC Breadth Mapping]
\label{cor:edc-breadth}
In Elastic Diffusive Cosmology, three distinct phenomena are manifestations
of the same Projection-Reduction Principle:

\begin{center}
\begin{tabular}{lll}
\hline
\textbf{EDC Result} & \textbf{Lemma Case} & \textbf{Key Parameter} \\
\hline
EM projection & Case (A) & $Z = \int K f^2$ \\
V--A from boundary & Case (B) with $\varepsilon \ll 1$ & $\varepsilon$ \\
Nuclear tunneling / pinning & Case (C) & $\kappa_{\text{eff}}$ \\
\hline
\end{tabular}
\end{center}

\textit{Canonical statement:} We adopt a single projection-reduction principle.
EM, chiral weak structure, and nuclear barrier tunneling appear as different
sectoral manifestations of the same bulk$\to$brane projection operator.
\end{corollary}

% -------------------------------------------------------------------------------
% CROSS-SECTOR BRIDGE
% -------------------------------------------------------------------------------
\paragraph{Cross-Sector Power.}
\label{par:cross-sector}

This lemma connects three sectors with one formalism:
\[
\text{EM} \longleftrightarrow \text{Weak} \longleftrightarrow \text{Nuclear}
\quad \text{(via } \mathcal{P}_w \text{)}
\]

Status: \textbf{[Der]} for individual cases; \textbf{[P]} for universal operator
unification (requires formal proof of uniqueness).

% ═══════════════════════════════════════════════════════════════════════════════
% END OF LEMMA
% ═══════════════════════════════════════════════════════════════════════════════

%
% Dependencies: amsmath, amsthm, physics (optional)
%
% ═══════════════════════════════════════════════════════════════════════════════

% -------------------------------------------------------------------------------
% SETUP (assumes theorem environments defined in parent document)
% -------------------------------------------------------------------------------
% If not defined, uncomment:
% \newtheorem{lemma}{Lemma}
% \newtheorem{corollary}{Corollary}
% \theoremstyle{definition}
% \newtheorem{definition}{Definition}

% -------------------------------------------------------------------------------
% DEFINITION: PROJECTION OPERATOR
% -------------------------------------------------------------------------------
\begin{definition}[Brane Projection Operator]
\label{def:projection-operator}
Let $\Phi(x,\chi)$ be a bulk field where $x \in \mathbb{R}^{3,1}$ (4D spacetime)
and $\chi$ is the extra-dimensional coordinate. Let the brane have localization
profile $w(\chi) \geq 0$ normalized as $\int d\chi\, w(\chi) = 1$.

The \textbf{projection operator} $\mathcal{P}_w$ is defined by:
\begin{equation}
\label{eq:projection-operator}
\phi(x) := (\mathcal{P}_w \Phi)(x) = \int d\chi\, w(\chi)\, \Phi(x,\chi)
\end{equation}

For any bulk quantity $F(x,\chi)$, the \textbf{projected expectation} is:
\begin{equation}
\label{eq:projected-expectation}
\langle F \rangle_w(x) := \int d\chi\, w(\chi)\, F(x,\chi)
\end{equation}
\end{definition}

% -------------------------------------------------------------------------------
% MAIN LEMMA
% -------------------------------------------------------------------------------
\begin{lemma}[Projection-Reduction Principle]
\label{lem:projection-reduction}
Bulk $\to$ brane observation is linear projection. All 4D observables are
weighted averages of bulk structure. This principle has three universal
manifestations:
\end{lemma}

% -------------------------------------------------------------------------------
% CASE (A): EFFECTIVE LAGRANGIAN
% -------------------------------------------------------------------------------
\paragraph{Case (A): Reduced Effective Lagrangian.}
\label{par:case-a-lagrangian}

If the bulk action is:
\begin{equation}
\label{eq:bulk-action}
S[\Phi] = \int d^4x\, d\chi \left[ \frac{1}{2} K(\chi)(\partial\Phi)^2 - U(\Phi,\chi) \right]
\end{equation}

Then for low-mode dynamics $\Phi(x,\chi) \approx \phi(x) f(\chi)$ with localized $f$:
\begin{equation}
\label{eq:effective-action}
S_{\text{eff}}[\phi] = \int d^4x \left[ \frac{1}{2} Z (\partial\phi)^2 - V_{\text{eff}}(\phi) \right]
\end{equation}

where the \textbf{effective coefficients} are integrals:
\begin{align}
\label{eq:wave-renormalization}
Z &= \int d\chi\, K(\chi)\, f(\chi)^2 \\[6pt]
\label{eq:effective-potential}
V_{\text{eff}}(\phi) &= \int d\chi\, U(\phi f(\chi), \chi)
\end{align}

\textit{Interpretation:} All bulk details (geometry, tension, brane thickness)
enter only through integral weights.

% -------------------------------------------------------------------------------
% CASE (B): CHIRALITY SELECTION
% -------------------------------------------------------------------------------
\paragraph{Case (B): Chirality as Projection Selection.}
\label{par:case-b-chirality}

Let bulk fermion $\Psi(x,\chi)$ have different localization profiles for
left/right components:
\begin{align}
\label{eq:left-projection}
\psi_L(x) &= \int d\chi\, w_L(\chi)\, \Psi_L(x,\chi) \\[6pt]
\label{eq:right-projection}
\psi_R(x) &= \int d\chi\, w_R(\chi)\, \Psi_R(x,\chi)
\end{align}

Define the \textbf{chirality overlap}:
\begin{equation}
\label{eq:chirality-overlap}
\varepsilon := \int d\chi\, w_L(\chi)\, w_R(\chi)
\end{equation}

\textbf{Chirality Selection Criterion:}
\begin{equation}
\label{eq:va-criterion}
\boxed{\varepsilon \ll 1 \quad \Longrightarrow \quad \text{Effective theory is dominantly chiral (V--A like)}}
\end{equation}

\textit{Interpretation:} Chirality emerges geometrically from overlap suppression,
without invoking specific gauge structure.

% -------------------------------------------------------------------------------
% CASE (C): BARRIER AND TUNNELING
% -------------------------------------------------------------------------------
\paragraph{Case (C): Effective Barrier and Tunneling.}
\label{par:case-c-barrier}

Let reaction coordinate $q$ (topological deformation) have $\chi$-dependent
potential $V(q,\chi)$.

The \textbf{projected potential} is:
\begin{equation}
\label{eq:projected-potential}
V_{\text{eff}}(q) = \int d\chi\, w(\chi)\, V(q,\chi)
\end{equation}

If bulk has pinning energy $+\kappa(\chi) q^2$:
\begin{equation}
\label{eq:pinning-potential}
V_{\text{eff}}(q) = V_0(q) + \langle\kappa\rangle_w\, q^2
\end{equation}

where the \textbf{effective pinning constant} is:
\begin{equation}
\label{eq:effective-pinning}
\kappa_{\text{eff}} := \langle\kappa\rangle_w = \int d\chi\, w(\chi)\, \kappa(\chi) > 0
\quad \Longrightarrow \quad \Delta V_{\text{barrier}} > 0
\end{equation}

\textit{Interpretation:} WKB tunneling exponents become functions of projected
parameters.

% -------------------------------------------------------------------------------
% UNIVERSAL CONSEQUENCES
% -------------------------------------------------------------------------------
\paragraph{Three Universal Consequences.}
\begin{enumerate}
    \item \textbf{Effective coefficients are integrals:} $Z$, $\kappa_{\text{eff}}$,
          $V_{\text{eff}}$ all arise from $\int d\chi\, w(\chi) \times (\text{bulk quantity})$
    \item \textbf{Chirality is geometrically selected:} $\varepsilon \ll 1$
          produces V--A without fine-tuning
    \item \textbf{Barriers are projections:} Tunneling rates depend on
          projected energy profiles
\end{enumerate}

% -------------------------------------------------------------------------------
% EDC APPLICATION COROLLARY
% -------------------------------------------------------------------------------
\begin{corollary}[EDC Breadth Mapping]
\label{cor:edc-breadth}
In Elastic Diffusive Cosmology, three distinct phenomena are manifestations
of the same Projection-Reduction Principle:

\begin{center}
\begin{tabular}{lll}
\hline
\textbf{EDC Result} & \textbf{Lemma Case} & \textbf{Key Parameter} \\
\hline
EM projection & Case (A) & $Z = \int K f^2$ \\
V--A from boundary & Case (B) with $\varepsilon \ll 1$ & $\varepsilon$ \\
Nuclear tunneling / pinning & Case (C) & $\kappa_{\text{eff}}$ \\
\hline
\end{tabular}
\end{center}

\textit{Canonical statement:} We adopt a single projection-reduction principle.
EM, chiral weak structure, and nuclear barrier tunneling appear as different
sectoral manifestations of the same bulk$\to$brane projection operator.
\end{corollary}

% -------------------------------------------------------------------------------
% CROSS-SECTOR BRIDGE
% -------------------------------------------------------------------------------
\paragraph{Cross-Sector Power.}
\label{par:cross-sector}

This lemma connects three sectors with one formalism:
\[
\text{EM} \longleftrightarrow \text{Weak} \longleftrightarrow \text{Nuclear}
\quad \text{(via } \mathcal{P}_w \text{)}
\]

Status: \textbf{[Der]} for individual cases; \textbf{[P]} for universal operator
unification (requires formal proof of uniqueness).

% ═══════════════════════════════════════════════════════════════════════════════
% END OF LEMMA
% ═══════════════════════════════════════════════════════════════════════════════

%
% Dependencies: amsmath, amsthm, physics (optional)
%
% ═══════════════════════════════════════════════════════════════════════════════

% -------------------------------------------------------------------------------
% SETUP (assumes theorem environments defined in parent document)
% -------------------------------------------------------------------------------
% If not defined, uncomment:
% \newtheorem{lemma}{Lemma}
% \newtheorem{corollary}{Corollary}
% \theoremstyle{definition}
% \newtheorem{definition}{Definition}

% -------------------------------------------------------------------------------
% DEFINITION: PROJECTION OPERATOR
% -------------------------------------------------------------------------------
\begin{definition}[Brane Projection Operator]
\label{def:projection-operator}
Let $\Phi(x,\chi)$ be a bulk field where $x \in \mathbb{R}^{3,1}$ (4D spacetime)
and $\chi$ is the extra-dimensional coordinate. Let the brane have localization
profile $w(\chi) \geq 0$ normalized as $\int d\chi\, w(\chi) = 1$.

The \textbf{projection operator} $\mathcal{P}_w$ is defined by:
\begin{equation}
\label{eq:projection-operator}
\phi(x) := (\mathcal{P}_w \Phi)(x) = \int d\chi\, w(\chi)\, \Phi(x,\chi)
\end{equation}

For any bulk quantity $F(x,\chi)$, the \textbf{projected expectation} is:
\begin{equation}
\label{eq:projected-expectation}
\langle F \rangle_w(x) := \int d\chi\, w(\chi)\, F(x,\chi)
\end{equation}
\end{definition}

% -------------------------------------------------------------------------------
% MAIN LEMMA
% -------------------------------------------------------------------------------
\begin{lemma}[Projection-Reduction Principle]
\label{lem:projection-reduction}
Bulk $\to$ brane observation is linear projection. All 4D observables are
weighted averages of bulk structure. This principle has three universal
manifestations:
\end{lemma}

% -------------------------------------------------------------------------------
% CASE (A): EFFECTIVE LAGRANGIAN
% -------------------------------------------------------------------------------
\paragraph{Case (A): Reduced Effective Lagrangian.}
\label{par:case-a-lagrangian}

If the bulk action is:
\begin{equation}
\label{eq:bulk-action}
S[\Phi] = \int d^4x\, d\chi \left[ \frac{1}{2} K(\chi)(\partial\Phi)^2 - U(\Phi,\chi) \right]
\end{equation}

Then for low-mode dynamics $\Phi(x,\chi) \approx \phi(x) f(\chi)$ with localized $f$:
\begin{equation}
\label{eq:effective-action}
S_{\text{eff}}[\phi] = \int d^4x \left[ \frac{1}{2} Z (\partial\phi)^2 - V_{\text{eff}}(\phi) \right]
\end{equation}

where the \textbf{effective coefficients} are integrals:
\begin{align}
\label{eq:wave-renormalization}
Z &= \int d\chi\, K(\chi)\, f(\chi)^2 \\[6pt]
\label{eq:effective-potential}
V_{\text{eff}}(\phi) &= \int d\chi\, U(\phi f(\chi), \chi)
\end{align}

\textit{Interpretation:} All bulk details (geometry, tension, brane thickness)
enter only through integral weights.

% -------------------------------------------------------------------------------
% CASE (B): CHIRALITY SELECTION
% -------------------------------------------------------------------------------
\paragraph{Case (B): Chirality as Projection Selection.}
\label{par:case-b-chirality}

Let bulk fermion $\Psi(x,\chi)$ have different localization profiles for
left/right components:
\begin{align}
\label{eq:left-projection}
\psi_L(x) &= \int d\chi\, w_L(\chi)\, \Psi_L(x,\chi) \\[6pt]
\label{eq:right-projection}
\psi_R(x) &= \int d\chi\, w_R(\chi)\, \Psi_R(x,\chi)
\end{align}

Define the \textbf{chirality overlap}:
\begin{equation}
\label{eq:chirality-overlap}
\varepsilon := \int d\chi\, w_L(\chi)\, w_R(\chi)
\end{equation}

\textbf{Chirality Selection Criterion:}
\begin{equation}
\label{eq:va-criterion}
\boxed{\varepsilon \ll 1 \quad \Longrightarrow \quad \text{Effective theory is dominantly chiral (V--A like)}}
\end{equation}

\textit{Interpretation:} Chirality emerges geometrically from overlap suppression,
without invoking specific gauge structure.

% -------------------------------------------------------------------------------
% CASE (C): BARRIER AND TUNNELING
% -------------------------------------------------------------------------------
\paragraph{Case (C): Effective Barrier and Tunneling.}
\label{par:case-c-barrier}

Let reaction coordinate $q$ (topological deformation) have $\chi$-dependent
potential $V(q,\chi)$.

The \textbf{projected potential} is:
\begin{equation}
\label{eq:projected-potential}
V_{\text{eff}}(q) = \int d\chi\, w(\chi)\, V(q,\chi)
\end{equation}

If bulk has pinning energy $+\kappa(\chi) q^2$:
\begin{equation}
\label{eq:pinning-potential}
V_{\text{eff}}(q) = V_0(q) + \langle\kappa\rangle_w\, q^2
\end{equation}

where the \textbf{effective pinning constant} is:
\begin{equation}
\label{eq:effective-pinning}
\kappa_{\text{eff}} := \langle\kappa\rangle_w = \int d\chi\, w(\chi)\, \kappa(\chi) > 0
\quad \Longrightarrow \quad \Delta V_{\text{barrier}} > 0
\end{equation}

\textit{Interpretation:} WKB tunneling exponents become functions of projected
parameters.

% -------------------------------------------------------------------------------
% UNIVERSAL CONSEQUENCES
% -------------------------------------------------------------------------------
\paragraph{Three Universal Consequences.}
\begin{enumerate}
    \item \textbf{Effective coefficients are integrals:} $Z$, $\kappa_{\text{eff}}$,
          $V_{\text{eff}}$ all arise from $\int d\chi\, w(\chi) \times (\text{bulk quantity})$
    \item \textbf{Chirality is geometrically selected:} $\varepsilon \ll 1$
          produces V--A without fine-tuning
    \item \textbf{Barriers are projections:} Tunneling rates depend on
          projected energy profiles
\end{enumerate}

% -------------------------------------------------------------------------------
% EDC APPLICATION COROLLARY
% -------------------------------------------------------------------------------
\begin{corollary}[EDC Breadth Mapping]
\label{cor:edc-breadth}
In Elastic Diffusive Cosmology, three distinct phenomena are manifestations
of the same Projection-Reduction Principle:

\begin{center}
\begin{tabular}{lll}
\hline
\textbf{EDC Result} & \textbf{Lemma Case} & \textbf{Key Parameter} \\
\hline
EM projection & Case (A) & $Z = \int K f^2$ \\
V--A from boundary & Case (B) with $\varepsilon \ll 1$ & $\varepsilon$ \\
Nuclear tunneling / pinning & Case (C) & $\kappa_{\text{eff}}$ \\
\hline
\end{tabular}
\end{center}

\textit{Canonical statement:} We adopt a single projection-reduction principle.
EM, chiral weak structure, and nuclear barrier tunneling appear as different
sectoral manifestations of the same bulk$\to$brane projection operator.
\end{corollary}

% -------------------------------------------------------------------------------
% CROSS-SECTOR BRIDGE
% -------------------------------------------------------------------------------
\paragraph{Cross-Sector Power.}
\label{par:cross-sector}

This lemma connects three sectors with one formalism:
\[
\text{EM} \longleftrightarrow \text{Weak} \longleftrightarrow \text{Nuclear}
\quad \text{(via } \mathcal{P}_w \text{)}
\]

Status: \textbf{[Der]} for individual cases; \textbf{[P]} for universal operator
unification (requires formal proof of uniqueness).

% ═══════════════════════════════════════════════════════════════════════════════
% END OF LEMMA
% ═══════════════════════════════════════════════════════════════════════════════
