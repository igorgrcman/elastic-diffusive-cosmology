% edc_style.tex — Canonical EDC Paper Style for Paper 3 Series
% Version 1.0 — 2026-01-20
%
% USAGE: Include in preamble AFTER loading packages but BEFORE \begin{document}
%   % edc_style.tex — Canonical EDC Paper Style for Paper 3 Series
% Version 1.0 — 2026-01-20
%
% USAGE: Include in preamble AFTER loading packages but BEFORE \begin{document}
%   % edc_style.tex — Canonical EDC Paper Style for Paper 3 Series
% Version 1.0 — 2026-01-20
%
% USAGE: Include in preamble AFTER loading packages but BEFORE \begin{document}
%   % edc_style.tex — Canonical EDC Paper Style for Paper 3 Series
% Version 1.0 — 2026-01-20
%
% USAGE: Include in preamble AFTER loading packages but BEFORE \begin{document}
%   \input{../_shared/style/edc_style}
%   \input{../_shared/style/tikz_style_edc}  % if using TikZ figures
%
% REQUIRED PACKAGES (load these in main document before \input):
%   fontspec, amsmath, amssymb, amsthm, mathtools, geometry
%   hyperref, enumitem, booktabs, array, xcolor, tcolorbox
%
% ============================================================

% ============================================================
%  EPISTEMIC TAG COLORS
% ============================================================
\definecolor{tagDer}{RGB}{0,128,0}      % Green - Derived
\definecolor{tagDc}{RGB}{0,0,200}       % Blue - Deduced/Constrained
\definecolor{tagCal}{RGB}{200,0,0}      % Red - Calibrated
\definecolor{tagP}{RGB}{128,0,128}      % Purple - Postulated
\definecolor{tagBL}{RGB}{128,128,128}   % Gray - Baseline
\definecolor{tagI}{RGB}{255,140,0}      % Orange - Identified
\definecolor{tagOpen}{RGB}{200,100,0}   % Dark orange - Open

% ============================================================
%  EPISTEMIC TAG COMMANDS
% ============================================================
% Use these to mark claims with their epistemic status
\newcommand{\tagDer}{\textcolor{tagDer}{\textbf{[Der]}}}    % Derived from axioms
\newcommand{\tagDc}{\textcolor{tagDc}{\textbf{[Dc]}}}       % Deduced/Constrained
\newcommand{\tagCal}{\textcolor{tagCal}{\textbf{[Cal]}}}    % Calibrated (fitted)
\newcommand{\tagP}{\textcolor{tagP}{\textbf{[P]}}}          % Postulated
\newcommand{\tagBL}{\textcolor{tagBL}{\textbf{[BL]}}}       % Baseline (external fact)
\newcommand{\tagI}{\textcolor{tagI}{\textbf{[I]}}}          % Identified (pattern match)
\newcommand{\tagOpen}{\textcolor{tagOpen}{\textbf{[OPEN]}}} % Open problem
\newcommand{\tagDef}{\textcolor{tagDc}{\textbf{[Def]}}}     % Definition

% ============================================================
%  THEOREM ENVIRONMENTS
% ============================================================
\newtheorem{postulate}{Postulate}
\newtheorem{definition}{Definition}[section]
\newtheorem{theorem}{Theorem}[section]
\newtheorem{lemma}[theorem]{Lemma}
\newtheorem{corollary}[theorem]{Corollary}
\newtheorem{proposition}[theorem]{Proposition}
\newtheorem{remark}{Remark}[section]

% ============================================================
%  COMMON EDC SYMBOLS
% ============================================================
% Symmetry groups
\newcommand{\Ztwo}{\mathbb{Z}_2}
\newcommand{\Zthree}{\mathbb{Z}_3}
\newcommand{\Ztri}{\mathbb{Z}_3}    % alias
\newcommand{\Zsix}{\mathbb{Z}_6}

% Geometric objects
\newcommand{\Sthree}{S^3}           % 3-sphere
\newcommand{\Stwo}{S^2}             % 2-sphere
\newcommand{\Bthree}{B^3}           % 3-ball
\newcommand{\Mfive}{\mathcal{M}_5}  % 5D manifold
\newcommand{\Bfour}{\mathcal{B}_4}  % 4D brane

% Physical quantities
\newcommand{\tension}{\tau}         % string/flux-tube tension (E/L)
\newcommand{\re}{r_e}               % electron radius

% Operators
\newcommand{\Pfrozen}{\mathcal{P}_{\mathrm{frozen}}}  % Frozen projection operator
\newcommand{\Ebrane}{\mathcal{E}_{\mathrm{brane}}}    % Brane energy store

% Bulk-brane exchange current (canonical notation from Framework v2.0)
\newcommand{\Jbb}[1]{J^{#1}_{\mathrm{bulk}\to\mathrm{brane}}}

% ============================================================
%  TCOLORBOX STYLES FOR EDC PAPERS
% ============================================================
% Cornerstone box (blue) — key claims/foundations
\tcbset{
    edcCornerstone/.style={
        colback=blue!5,
        colframe=blue!40!black,
        fonttitle=\bfseries
    }
}

% Guardrail box (gray) — epistemic warnings/constraints
\tcbset{
    edcGuardrail/.style={
        colback=gray!5!white,
        colframe=gray!60!black,
        fonttitle=\bfseries
    }
}

% PPN box (blue, lighter) — Physical Process Narrative
\tcbset{
    edcPPN/.style={
        colback=blue!5,
        colframe=blue!50!black,
        fonttitle=\bfseries
    }
}

% Canonical box (yellow/orange) — canonical definitions/glossary
\tcbset{
    edcCanonical/.style={
        colback=yellow!5,
        colframe=orange!60!black,
        fonttitle=\bfseries
    }
}

% Conceptual box (yellow/orange, lighter) — conceptual pictures
\tcbset{
    edcConcept/.style={
        colback=yellow!5,
        colframe=orange!50!black,
        fonttitle=\bfseries
    }
}

% Pathway box (purple) — energy pathways, mechanisms
\tcbset{
    edcPathway/.style={
        colback=purple!5,
        colframe=purple!40!black,
        fonttitle=\bfseries
    }
}

% Model box (green) — mechanical analogies, heuristics
\tcbset{
    edcModel/.style={
        colback=green!5,
        colframe=green!40!black,
        fonttitle=\bfseries
    }
}

% Warning box (red) — non-overclaim, limitations
\tcbset{
    edcWarning/.style={
        colback=red!5,
        colframe=red!40!black,
        fonttitle=\bfseries
    }
}

% Framework quote box (gray) — verbatim from Framework v2.0
\tcbset{
    edcFramework/.style={
        colback=gray!5!white,
        colframe=gray!60!black,
        fonttitle=\small
    }
}

% Mechanism box (teal) — mechanistic dimension principle narrative
\tcbset{
    edcMechanism/.style={
        colback=teal!5,
        colframe=teal!50!black,
        fonttitle=\bfseries,
        title={Mechanistic Dimension Note (Canon)}
    }
}

% ============================================================
%  MECHANISTIC DIMENSION HELPER MACRO
% ============================================================
% Usage: \edcMechanismNote{bulk cause}{brane process}{3D output}
%
% Example:
%   \edcMechanismNote{Junction relaxes toward Steiner minimum}%
%                    {Energy pumps into brane-layer modes, redistributes}%
%                    {Electron, antineutrino, proton emerge on 3D side}
%
\newcommand{\edcMechanismNote}[3]{%
\begin{tcolorbox}[edcMechanism]
\begin{itemize}[nosep,leftmargin=*]
    \item \textbf{5D cause (bulk):} #1
    \item \textbf{Brane-layer process:} #2
    \item \textbf{3D observation (output):} #3
\end{itemize}
\vspace{0.3em}
\footnotesize\textit{Ledger closure must hold: bulk + brane + 3D outputs conserve energy/quantum numbers.}
\end{tcolorbox}
}

% ============================================================
%  RELATED DOCUMENTS MACRO
% ============================================================
% Usage: \edcRelatedDocs{main paper title}{main DOI}{companion list}
%
% Example:
%     A: \emph{Effective Lagrangian} (\href{...}{DOI}) $\cdot$
%     B: \emph{WKB Prefactor} (\href{...}{DOI})
%   }

% NOTE: \edcRelatedDocs macro deprecated (DOI registry consolidated)
% Use consolidated Zenodo article as primary reference instead.

% ============================================================
%  DOI REGISTRY DEPRECATED
% ============================================================
% Previous individual DOIs have been deprecated.
% All EDC Weak Sector content is now consolidated into a single
% Zenodo article. See paper_3_series/19_edc_weak_sector_zenodo_article/

% ============================================================
%  PHYSICAL NARRATION RULE REMINDER
% ============================================================
% Every key equation MUST be accompanied by a physical narrative stating:
%   1. 5D cause: What changes in the bulk-core configuration?
%   2. Brane response: How does the brane absorb/redistribute energy?
%   3. 3D observable output: What do observers detect on the 3D side?
%
% This rule eliminates "numerology smell" by ensuring every formula
% has a mechanistic interpretation.

% ============================================================
%  END OF STYLE FILE
% ============================================================

%   % tikz_style_edc.tex — Reusable TikZ styles for EDC papers
% Version 1.0 — 2026-01-20
% Include via: \input{tikz_style_edc}

% ============================================================
% REQUIRED LIBRARIES (must be loaded in main document)
% ============================================================
% \usetikzlibrary{calc,angles,quotes,decorations.markings,decorations.pathmorphing,positioning}

% ============================================================
% POSITIONING DEFAULTS
% ============================================================
\tikzset{
    % Default node distances for horizontal/vertical layouts
    edc node distance/.style={node distance=1.6cm and 2.0cm},
    % Compact variant for dense diagrams
    edc compact/.style={node distance=1.2cm and 1.5cm},
    % Spread variant for clarity
    edc spread/.style={node distance=2.0cm and 2.5cm},
}

% ============================================================
% COLOR PALETTE (consistent with epistemic tags)
% ============================================================
\definecolor{edcBulk}{RGB}{220,50,50}        % Red tones for bulk/5D
\definecolor{edcBrane}{RGB}{50,150,50}       % Green tones for brane-layer
\definecolor{edcOutput}{RGB}{50,100,200}     % Blue tones for 3D outputs
\definecolor{edcNeutral}{RGB}{100,100,100}   % Gray for neutral/annotations

% ============================================================
% BOX STYLES
% ============================================================
\tikzset{
    % Generic EDC box (base style)
    edc box/.style={
        rectangle,
        draw,
        rounded corners=3pt,
        minimum width=2.2cm,
        minimum height=0.8cm,
        align=center,
        font=\small,
        inner sep=4pt,
    },
    % Bulk-core box (red family)
    bulk box/.style={
        edc box,
        fill=red!10,
        draw=edcBulk!70!black,
        text=black,
    },
    % Brane-layer box (green family)
    brane box/.style={
        edc box,
        fill=green!10,
        draw=edcBrane!70!black,
        text=black,
    },
    % 3D output box (blue family)
    output box/.style={
        edc box,
        fill=blue!10,
        draw=edcOutput!70!black,
        text=black,
    },
    % Neutral/process box
    process box/.style={
        edc box,
        fill=gray!10,
        draw=gray!60!black,
        text=black,
    },
    % Label-only box (no background)
    label box/.style={
        rectangle,
        rounded corners=2pt,
        draw=gray!40,
        fill=white,
        inner sep=2pt,
        font=\scriptsize,
    },
}

% ============================================================
% ARROW STYLES
% ============================================================
\tikzset{
    % Standard thick arrow
    edc arrow/.style={
        ->,
        >=stealth,
        thick,
    },
    % Emphasized arrow (for main flow)
    edc flow/.style={
        ->,
        >=stealth,
        very thick,
        line width=1.2pt,
    },
    % Dashed arrow (for optional/weak connections)
    edc dashed/.style={
        ->,
        >=stealth,
        thick,
        dashed,
    },
    % Double arrow (for bidirectional)
    edc bidir/.style={
        <->,
        >=stealth,
        thick,
    },
}

% ============================================================
% REGION STYLES (for background fills)
% ============================================================
\tikzset{
    % Bulk region (5D)
    bulk region/.style={
        fill=blue!8,
    },
    % Brane layer region
    brane region/.style={
        fill=yellow!25,
    },
    % Observer/3D region
    observer region/.style={
        fill=green!8,
    },
}

% ============================================================
% LABEL STYLES
% ============================================================
\tikzset{
    % Phase label (below nodes)
    phase label/.style={
        font=\scriptsize\itshape,
        text=black!70,
    },
    % Equation label (for inline math)
    eq label/.style={
        font=\scriptsize,
        fill=white,
        inner sep=1pt,
    },
    % Section annotation
    section label/.style={
        font=\footnotesize\bfseries,
        text=black,
    },
}

% ============================================================
% JUNCTION/PARTICLE STYLES
% ============================================================
\tikzset{
    % Y-junction point
    junction point/.style={
        circle,
        fill=red!60!black,
        minimum size=4pt,
        inner sep=0pt,
    },
    % Flux tube arm
    flux arm/.style={
        thick,
        blue!60!black,
    },
    % Particle dot (electron, etc.)
    particle/.style={
        circle,
        fill=black,
        minimum size=5pt,
        inner sep=0pt,
    },
    % Neutrino (smaller, gray)
    neutrino/.style={
        circle,
        fill=gray,
        minimum size=4pt,
        inner sep=0pt,
    },
}

% ============================================================
% SPRING DECORATION (for mechanical models)
% ============================================================
\tikzset{
    spring/.style={
        thick,
        decorate,
        decoration={
            coil,
            aspect=0.5,
            segment length=2mm,
            amplitude=2mm,
        },
    },
    % Wave decoration (for field modes)
    wave field/.style={
        thick,
        decorate,
        decoration={
            snake,
            amplitude=2pt,
            segment length=8pt,
        },
    },
}

% ============================================================
% BOUNDARY STYLES
% ============================================================
\tikzset{
    % Bulk-facing boundary (dashed red)
    bulk boundary/.style={
        very thick,
        red!70!black,
        dashed,
    },
    % Observer-facing boundary (solid green)
    observer boundary/.style={
        thick,
        green!50!black,
    },
    % Brane edge (orange)
    brane edge/.style={
        thick,
        orange!70!black,
    },
}

% ============================================================
% CONVENIENCE COMMANDS
% ============================================================
% Arrow label (above)
\newcommand{\arrlabel}[1]{\scriptsize #1}
% Arrow label (below)
\newcommand{\arrlabelb}[1]{\scriptsize #1}

% ============================================================
% END OF STYLE FILE
% ============================================================
  % if using TikZ figures
%
% REQUIRED PACKAGES (load these in main document before \input):
%   fontspec, amsmath, amssymb, amsthm, mathtools, geometry
%   hyperref, enumitem, booktabs, array, xcolor, tcolorbox
%
% ============================================================

% ============================================================
%  EPISTEMIC TAG COLORS
% ============================================================
\definecolor{tagDer}{RGB}{0,128,0}      % Green - Derived
\definecolor{tagDc}{RGB}{0,0,200}       % Blue - Deduced/Constrained
\definecolor{tagCal}{RGB}{200,0,0}      % Red - Calibrated
\definecolor{tagP}{RGB}{128,0,128}      % Purple - Postulated
\definecolor{tagBL}{RGB}{128,128,128}   % Gray - Baseline
\definecolor{tagI}{RGB}{255,140,0}      % Orange - Identified
\definecolor{tagOpen}{RGB}{200,100,0}   % Dark orange - Open

% ============================================================
%  EPISTEMIC TAG COMMANDS
% ============================================================
% Use these to mark claims with their epistemic status
\newcommand{\tagDer}{\textcolor{tagDer}{\textbf{[Der]}}}    % Derived from axioms
\newcommand{\tagDc}{\textcolor{tagDc}{\textbf{[Dc]}}}       % Deduced/Constrained
\newcommand{\tagCal}{\textcolor{tagCal}{\textbf{[Cal]}}}    % Calibrated (fitted)
\newcommand{\tagP}{\textcolor{tagP}{\textbf{[P]}}}          % Postulated
\newcommand{\tagBL}{\textcolor{tagBL}{\textbf{[BL]}}}       % Baseline (external fact)
\newcommand{\tagI}{\textcolor{tagI}{\textbf{[I]}}}          % Identified (pattern match)
\newcommand{\tagOpen}{\textcolor{tagOpen}{\textbf{[OPEN]}}} % Open problem
\newcommand{\tagDef}{\textcolor{tagDc}{\textbf{[Def]}}}     % Definition

% ============================================================
%  THEOREM ENVIRONMENTS
% ============================================================
\newtheorem{postulate}{Postulate}
\newtheorem{definition}{Definition}[section]
\newtheorem{theorem}{Theorem}[section]
\newtheorem{lemma}[theorem]{Lemma}
\newtheorem{corollary}[theorem]{Corollary}
\newtheorem{proposition}[theorem]{Proposition}
\newtheorem{remark}{Remark}[section]

% ============================================================
%  COMMON EDC SYMBOLS
% ============================================================
% Symmetry groups
\newcommand{\Ztwo}{\mathbb{Z}_2}
\newcommand{\Zthree}{\mathbb{Z}_3}
\newcommand{\Ztri}{\mathbb{Z}_3}    % alias
\newcommand{\Zsix}{\mathbb{Z}_6}

% Geometric objects
\newcommand{\Sthree}{S^3}           % 3-sphere
\newcommand{\Stwo}{S^2}             % 2-sphere
\newcommand{\Bthree}{B^3}           % 3-ball
\newcommand{\Mfive}{\mathcal{M}_5}  % 5D manifold
\newcommand{\Bfour}{\mathcal{B}_4}  % 4D brane

% Physical quantities
\newcommand{\tension}{\tau}         % string/flux-tube tension (E/L)
\newcommand{\re}{r_e}               % electron radius

% Operators
\newcommand{\Pfrozen}{\mathcal{P}_{\mathrm{frozen}}}  % Frozen projection operator
\newcommand{\Ebrane}{\mathcal{E}_{\mathrm{brane}}}    % Brane energy store

% Bulk-brane exchange current (canonical notation from Framework v2.0)
\newcommand{\Jbb}[1]{J^{#1}_{\mathrm{bulk}\to\mathrm{brane}}}

% ============================================================
%  TCOLORBOX STYLES FOR EDC PAPERS
% ============================================================
% Cornerstone box (blue) — key claims/foundations
\tcbset{
    edcCornerstone/.style={
        colback=blue!5,
        colframe=blue!40!black,
        fonttitle=\bfseries
    }
}

% Guardrail box (gray) — epistemic warnings/constraints
\tcbset{
    edcGuardrail/.style={
        colback=gray!5!white,
        colframe=gray!60!black,
        fonttitle=\bfseries
    }
}

% PPN box (blue, lighter) — Physical Process Narrative
\tcbset{
    edcPPN/.style={
        colback=blue!5,
        colframe=blue!50!black,
        fonttitle=\bfseries
    }
}

% Canonical box (yellow/orange) — canonical definitions/glossary
\tcbset{
    edcCanonical/.style={
        colback=yellow!5,
        colframe=orange!60!black,
        fonttitle=\bfseries
    }
}

% Conceptual box (yellow/orange, lighter) — conceptual pictures
\tcbset{
    edcConcept/.style={
        colback=yellow!5,
        colframe=orange!50!black,
        fonttitle=\bfseries
    }
}

% Pathway box (purple) — energy pathways, mechanisms
\tcbset{
    edcPathway/.style={
        colback=purple!5,
        colframe=purple!40!black,
        fonttitle=\bfseries
    }
}

% Model box (green) — mechanical analogies, heuristics
\tcbset{
    edcModel/.style={
        colback=green!5,
        colframe=green!40!black,
        fonttitle=\bfseries
    }
}

% Warning box (red) — non-overclaim, limitations
\tcbset{
    edcWarning/.style={
        colback=red!5,
        colframe=red!40!black,
        fonttitle=\bfseries
    }
}

% Framework quote box (gray) — verbatim from Framework v2.0
\tcbset{
    edcFramework/.style={
        colback=gray!5!white,
        colframe=gray!60!black,
        fonttitle=\small
    }
}

% Mechanism box (teal) — mechanistic dimension principle narrative
\tcbset{
    edcMechanism/.style={
        colback=teal!5,
        colframe=teal!50!black,
        fonttitle=\bfseries,
        title={Mechanistic Dimension Note (Canon)}
    }
}

% ============================================================
%  MECHANISTIC DIMENSION HELPER MACRO
% ============================================================
% Usage: \edcMechanismNote{bulk cause}{brane process}{3D output}
%
% Example:
%   \edcMechanismNote{Junction relaxes toward Steiner minimum}%
%                    {Energy pumps into brane-layer modes, redistributes}%
%                    {Electron, antineutrino, proton emerge on 3D side}
%
\newcommand{\edcMechanismNote}[3]{%
\begin{tcolorbox}[edcMechanism]
\begin{itemize}[nosep,leftmargin=*]
    \item \textbf{5D cause (bulk):} #1
    \item \textbf{Brane-layer process:} #2
    \item \textbf{3D observation (output):} #3
\end{itemize}
\vspace{0.3em}
\footnotesize\textit{Ledger closure must hold: bulk + brane + 3D outputs conserve energy/quantum numbers.}
\end{tcolorbox}
}

% ============================================================
%  RELATED DOCUMENTS MACRO
% ============================================================
% Usage: \edcRelatedDocs{main paper title}{main DOI}{companion list}
%
% Example:
%     A: \emph{Effective Lagrangian} (\href{...}{DOI}) $\cdot$
%     B: \emph{WKB Prefactor} (\href{...}{DOI})
%   }

% NOTE: \edcRelatedDocs macro deprecated (DOI registry consolidated)
% Use consolidated Zenodo article as primary reference instead.

% ============================================================
%  DOI REGISTRY DEPRECATED
% ============================================================
% Previous individual DOIs have been deprecated.
% All EDC Weak Sector content is now consolidated into a single
% Zenodo article. See paper_3_series/19_edc_weak_sector_zenodo_article/

% ============================================================
%  PHYSICAL NARRATION RULE REMINDER
% ============================================================
% Every key equation MUST be accompanied by a physical narrative stating:
%   1. 5D cause: What changes in the bulk-core configuration?
%   2. Brane response: How does the brane absorb/redistribute energy?
%   3. 3D observable output: What do observers detect on the 3D side?
%
% This rule eliminates "numerology smell" by ensuring every formula
% has a mechanistic interpretation.

% ============================================================
%  END OF STYLE FILE
% ============================================================

%   % tikz_style_edc.tex — Reusable TikZ styles for EDC papers
% Version 1.0 — 2026-01-20
% Include via: % tikz_style_edc.tex — Reusable TikZ styles for EDC papers
% Version 1.0 — 2026-01-20
% Include via: \input{tikz_style_edc}

% ============================================================
% REQUIRED LIBRARIES (must be loaded in main document)
% ============================================================
% \usetikzlibrary{calc,angles,quotes,decorations.markings,decorations.pathmorphing,positioning}

% ============================================================
% POSITIONING DEFAULTS
% ============================================================
\tikzset{
    % Default node distances for horizontal/vertical layouts
    edc node distance/.style={node distance=1.6cm and 2.0cm},
    % Compact variant for dense diagrams
    edc compact/.style={node distance=1.2cm and 1.5cm},
    % Spread variant for clarity
    edc spread/.style={node distance=2.0cm and 2.5cm},
}

% ============================================================
% COLOR PALETTE (consistent with epistemic tags)
% ============================================================
\definecolor{edcBulk}{RGB}{220,50,50}        % Red tones for bulk/5D
\definecolor{edcBrane}{RGB}{50,150,50}       % Green tones for brane-layer
\definecolor{edcOutput}{RGB}{50,100,200}     % Blue tones for 3D outputs
\definecolor{edcNeutral}{RGB}{100,100,100}   % Gray for neutral/annotations

% ============================================================
% BOX STYLES
% ============================================================
\tikzset{
    % Generic EDC box (base style)
    edc box/.style={
        rectangle,
        draw,
        rounded corners=3pt,
        minimum width=2.2cm,
        minimum height=0.8cm,
        align=center,
        font=\small,
        inner sep=4pt,
    },
    % Bulk-core box (red family)
    bulk box/.style={
        edc box,
        fill=red!10,
        draw=edcBulk!70!black,
        text=black,
    },
    % Brane-layer box (green family)
    brane box/.style={
        edc box,
        fill=green!10,
        draw=edcBrane!70!black,
        text=black,
    },
    % 3D output box (blue family)
    output box/.style={
        edc box,
        fill=blue!10,
        draw=edcOutput!70!black,
        text=black,
    },
    % Neutral/process box
    process box/.style={
        edc box,
        fill=gray!10,
        draw=gray!60!black,
        text=black,
    },
    % Label-only box (no background)
    label box/.style={
        rectangle,
        rounded corners=2pt,
        draw=gray!40,
        fill=white,
        inner sep=2pt,
        font=\scriptsize,
    },
}

% ============================================================
% ARROW STYLES
% ============================================================
\tikzset{
    % Standard thick arrow
    edc arrow/.style={
        ->,
        >=stealth,
        thick,
    },
    % Emphasized arrow (for main flow)
    edc flow/.style={
        ->,
        >=stealth,
        very thick,
        line width=1.2pt,
    },
    % Dashed arrow (for optional/weak connections)
    edc dashed/.style={
        ->,
        >=stealth,
        thick,
        dashed,
    },
    % Double arrow (for bidirectional)
    edc bidir/.style={
        <->,
        >=stealth,
        thick,
    },
}

% ============================================================
% REGION STYLES (for background fills)
% ============================================================
\tikzset{
    % Bulk region (5D)
    bulk region/.style={
        fill=blue!8,
    },
    % Brane layer region
    brane region/.style={
        fill=yellow!25,
    },
    % Observer/3D region
    observer region/.style={
        fill=green!8,
    },
}

% ============================================================
% LABEL STYLES
% ============================================================
\tikzset{
    % Phase label (below nodes)
    phase label/.style={
        font=\scriptsize\itshape,
        text=black!70,
    },
    % Equation label (for inline math)
    eq label/.style={
        font=\scriptsize,
        fill=white,
        inner sep=1pt,
    },
    % Section annotation
    section label/.style={
        font=\footnotesize\bfseries,
        text=black,
    },
}

% ============================================================
% JUNCTION/PARTICLE STYLES
% ============================================================
\tikzset{
    % Y-junction point
    junction point/.style={
        circle,
        fill=red!60!black,
        minimum size=4pt,
        inner sep=0pt,
    },
    % Flux tube arm
    flux arm/.style={
        thick,
        blue!60!black,
    },
    % Particle dot (electron, etc.)
    particle/.style={
        circle,
        fill=black,
        minimum size=5pt,
        inner sep=0pt,
    },
    % Neutrino (smaller, gray)
    neutrino/.style={
        circle,
        fill=gray,
        minimum size=4pt,
        inner sep=0pt,
    },
}

% ============================================================
% SPRING DECORATION (for mechanical models)
% ============================================================
\tikzset{
    spring/.style={
        thick,
        decorate,
        decoration={
            coil,
            aspect=0.5,
            segment length=2mm,
            amplitude=2mm,
        },
    },
    % Wave decoration (for field modes)
    wave field/.style={
        thick,
        decorate,
        decoration={
            snake,
            amplitude=2pt,
            segment length=8pt,
        },
    },
}

% ============================================================
% BOUNDARY STYLES
% ============================================================
\tikzset{
    % Bulk-facing boundary (dashed red)
    bulk boundary/.style={
        very thick,
        red!70!black,
        dashed,
    },
    % Observer-facing boundary (solid green)
    observer boundary/.style={
        thick,
        green!50!black,
    },
    % Brane edge (orange)
    brane edge/.style={
        thick,
        orange!70!black,
    },
}

% ============================================================
% CONVENIENCE COMMANDS
% ============================================================
% Arrow label (above)
\newcommand{\arrlabel}[1]{\scriptsize #1}
% Arrow label (below)
\newcommand{\arrlabelb}[1]{\scriptsize #1}

% ============================================================
% END OF STYLE FILE
% ============================================================


% ============================================================
% REQUIRED LIBRARIES (must be loaded in main document)
% ============================================================
% \usetikzlibrary{calc,angles,quotes,decorations.markings,decorations.pathmorphing,positioning}

% ============================================================
% POSITIONING DEFAULTS
% ============================================================
\tikzset{
    % Default node distances for horizontal/vertical layouts
    edc node distance/.style={node distance=1.6cm and 2.0cm},
    % Compact variant for dense diagrams
    edc compact/.style={node distance=1.2cm and 1.5cm},
    % Spread variant for clarity
    edc spread/.style={node distance=2.0cm and 2.5cm},
}

% ============================================================
% COLOR PALETTE (consistent with epistemic tags)
% ============================================================
\definecolor{edcBulk}{RGB}{220,50,50}        % Red tones for bulk/5D
\definecolor{edcBrane}{RGB}{50,150,50}       % Green tones for brane-layer
\definecolor{edcOutput}{RGB}{50,100,200}     % Blue tones for 3D outputs
\definecolor{edcNeutral}{RGB}{100,100,100}   % Gray for neutral/annotations

% ============================================================
% BOX STYLES
% ============================================================
\tikzset{
    % Generic EDC box (base style)
    edc box/.style={
        rectangle,
        draw,
        rounded corners=3pt,
        minimum width=2.2cm,
        minimum height=0.8cm,
        align=center,
        font=\small,
        inner sep=4pt,
    },
    % Bulk-core box (red family)
    bulk box/.style={
        edc box,
        fill=red!10,
        draw=edcBulk!70!black,
        text=black,
    },
    % Brane-layer box (green family)
    brane box/.style={
        edc box,
        fill=green!10,
        draw=edcBrane!70!black,
        text=black,
    },
    % 3D output box (blue family)
    output box/.style={
        edc box,
        fill=blue!10,
        draw=edcOutput!70!black,
        text=black,
    },
    % Neutral/process box
    process box/.style={
        edc box,
        fill=gray!10,
        draw=gray!60!black,
        text=black,
    },
    % Label-only box (no background)
    label box/.style={
        rectangle,
        rounded corners=2pt,
        draw=gray!40,
        fill=white,
        inner sep=2pt,
        font=\scriptsize,
    },
}

% ============================================================
% ARROW STYLES
% ============================================================
\tikzset{
    % Standard thick arrow
    edc arrow/.style={
        ->,
        >=stealth,
        thick,
    },
    % Emphasized arrow (for main flow)
    edc flow/.style={
        ->,
        >=stealth,
        very thick,
        line width=1.2pt,
    },
    % Dashed arrow (for optional/weak connections)
    edc dashed/.style={
        ->,
        >=stealth,
        thick,
        dashed,
    },
    % Double arrow (for bidirectional)
    edc bidir/.style={
        <->,
        >=stealth,
        thick,
    },
}

% ============================================================
% REGION STYLES (for background fills)
% ============================================================
\tikzset{
    % Bulk region (5D)
    bulk region/.style={
        fill=blue!8,
    },
    % Brane layer region
    brane region/.style={
        fill=yellow!25,
    },
    % Observer/3D region
    observer region/.style={
        fill=green!8,
    },
}

% ============================================================
% LABEL STYLES
% ============================================================
\tikzset{
    % Phase label (below nodes)
    phase label/.style={
        font=\scriptsize\itshape,
        text=black!70,
    },
    % Equation label (for inline math)
    eq label/.style={
        font=\scriptsize,
        fill=white,
        inner sep=1pt,
    },
    % Section annotation
    section label/.style={
        font=\footnotesize\bfseries,
        text=black,
    },
}

% ============================================================
% JUNCTION/PARTICLE STYLES
% ============================================================
\tikzset{
    % Y-junction point
    junction point/.style={
        circle,
        fill=red!60!black,
        minimum size=4pt,
        inner sep=0pt,
    },
    % Flux tube arm
    flux arm/.style={
        thick,
        blue!60!black,
    },
    % Particle dot (electron, etc.)
    particle/.style={
        circle,
        fill=black,
        minimum size=5pt,
        inner sep=0pt,
    },
    % Neutrino (smaller, gray)
    neutrino/.style={
        circle,
        fill=gray,
        minimum size=4pt,
        inner sep=0pt,
    },
}

% ============================================================
% SPRING DECORATION (for mechanical models)
% ============================================================
\tikzset{
    spring/.style={
        thick,
        decorate,
        decoration={
            coil,
            aspect=0.5,
            segment length=2mm,
            amplitude=2mm,
        },
    },
    % Wave decoration (for field modes)
    wave field/.style={
        thick,
        decorate,
        decoration={
            snake,
            amplitude=2pt,
            segment length=8pt,
        },
    },
}

% ============================================================
% BOUNDARY STYLES
% ============================================================
\tikzset{
    % Bulk-facing boundary (dashed red)
    bulk boundary/.style={
        very thick,
        red!70!black,
        dashed,
    },
    % Observer-facing boundary (solid green)
    observer boundary/.style={
        thick,
        green!50!black,
    },
    % Brane edge (orange)
    brane edge/.style={
        thick,
        orange!70!black,
    },
}

% ============================================================
% CONVENIENCE COMMANDS
% ============================================================
% Arrow label (above)
\newcommand{\arrlabel}[1]{\scriptsize #1}
% Arrow label (below)
\newcommand{\arrlabelb}[1]{\scriptsize #1}

% ============================================================
% END OF STYLE FILE
% ============================================================
  % if using TikZ figures
%
% REQUIRED PACKAGES (load these in main document before \input):
%   fontspec, amsmath, amssymb, amsthm, mathtools, geometry
%   hyperref, enumitem, booktabs, array, xcolor, tcolorbox
%
% ============================================================

% ============================================================
%  EPISTEMIC TAG COLORS
% ============================================================
\definecolor{tagDer}{RGB}{0,128,0}      % Green - Derived
\definecolor{tagDc}{RGB}{0,0,200}       % Blue - Deduced/Constrained
\definecolor{tagCal}{RGB}{200,0,0}      % Red - Calibrated
\definecolor{tagP}{RGB}{128,0,128}      % Purple - Postulated
\definecolor{tagBL}{RGB}{128,128,128}   % Gray - Baseline
\definecolor{tagI}{RGB}{255,140,0}      % Orange - Identified
\definecolor{tagOpen}{RGB}{200,100,0}   % Dark orange - Open

% ============================================================
%  EPISTEMIC TAG COMMANDS
% ============================================================
% Use these to mark claims with their epistemic status
\newcommand{\tagDer}{\textcolor{tagDer}{\textbf{[Der]}}}    % Derived from axioms
\newcommand{\tagDc}{\textcolor{tagDc}{\textbf{[Dc]}}}       % Deduced/Constrained
\newcommand{\tagCal}{\textcolor{tagCal}{\textbf{[Cal]}}}    % Calibrated (fitted)
\newcommand{\tagP}{\textcolor{tagP}{\textbf{[P]}}}          % Postulated
\newcommand{\tagBL}{\textcolor{tagBL}{\textbf{[BL]}}}       % Baseline (external fact)
\newcommand{\tagI}{\textcolor{tagI}{\textbf{[I]}}}          % Identified (pattern match)
\newcommand{\tagOpen}{\textcolor{tagOpen}{\textbf{[OPEN]}}} % Open problem
\newcommand{\tagDef}{\textcolor{tagDc}{\textbf{[Def]}}}     % Definition

% ============================================================
%  THEOREM ENVIRONMENTS
% ============================================================
\newtheorem{postulate}{Postulate}
\newtheorem{definition}{Definition}[section]
\newtheorem{theorem}{Theorem}[section]
\newtheorem{lemma}[theorem]{Lemma}
\newtheorem{corollary}[theorem]{Corollary}
\newtheorem{proposition}[theorem]{Proposition}
\newtheorem{remark}{Remark}[section]

% ============================================================
%  COMMON EDC SYMBOLS
% ============================================================
% Symmetry groups
\newcommand{\Ztwo}{\mathbb{Z}_2}
\newcommand{\Zthree}{\mathbb{Z}_3}
\newcommand{\Ztri}{\mathbb{Z}_3}    % alias
\newcommand{\Zsix}{\mathbb{Z}_6}

% Geometric objects
\newcommand{\Sthree}{S^3}           % 3-sphere
\newcommand{\Stwo}{S^2}             % 2-sphere
\newcommand{\Bthree}{B^3}           % 3-ball
\newcommand{\Mfive}{\mathcal{M}_5}  % 5D manifold
\newcommand{\Bfour}{\mathcal{B}_4}  % 4D brane

% Physical quantities
\newcommand{\tension}{\tau}         % string/flux-tube tension (E/L)
\newcommand{\re}{r_e}               % electron radius

% Operators
\newcommand{\Pfrozen}{\mathcal{P}_{\mathrm{frozen}}}  % Frozen projection operator
\newcommand{\Ebrane}{\mathcal{E}_{\mathrm{brane}}}    % Brane energy store

% Bulk-brane exchange current (canonical notation from Framework v2.0)
\newcommand{\Jbb}[1]{J^{#1}_{\mathrm{bulk}\to\mathrm{brane}}}

% ============================================================
%  TCOLORBOX STYLES FOR EDC PAPERS
% ============================================================
% Cornerstone box (blue) — key claims/foundations
\tcbset{
    edcCornerstone/.style={
        colback=blue!5,
        colframe=blue!40!black,
        fonttitle=\bfseries
    }
}

% Guardrail box (gray) — epistemic warnings/constraints
\tcbset{
    edcGuardrail/.style={
        colback=gray!5!white,
        colframe=gray!60!black,
        fonttitle=\bfseries
    }
}

% PPN box (blue, lighter) — Physical Process Narrative
\tcbset{
    edcPPN/.style={
        colback=blue!5,
        colframe=blue!50!black,
        fonttitle=\bfseries
    }
}

% Canonical box (yellow/orange) — canonical definitions/glossary
\tcbset{
    edcCanonical/.style={
        colback=yellow!5,
        colframe=orange!60!black,
        fonttitle=\bfseries
    }
}

% Conceptual box (yellow/orange, lighter) — conceptual pictures
\tcbset{
    edcConcept/.style={
        colback=yellow!5,
        colframe=orange!50!black,
        fonttitle=\bfseries
    }
}

% Pathway box (purple) — energy pathways, mechanisms
\tcbset{
    edcPathway/.style={
        colback=purple!5,
        colframe=purple!40!black,
        fonttitle=\bfseries
    }
}

% Model box (green) — mechanical analogies, heuristics
\tcbset{
    edcModel/.style={
        colback=green!5,
        colframe=green!40!black,
        fonttitle=\bfseries
    }
}

% Warning box (red) — non-overclaim, limitations
\tcbset{
    edcWarning/.style={
        colback=red!5,
        colframe=red!40!black,
        fonttitle=\bfseries
    }
}

% Framework quote box (gray) — verbatim from Framework v2.0
\tcbset{
    edcFramework/.style={
        colback=gray!5!white,
        colframe=gray!60!black,
        fonttitle=\small
    }
}

% Mechanism box (teal) — mechanistic dimension principle narrative
\tcbset{
    edcMechanism/.style={
        colback=teal!5,
        colframe=teal!50!black,
        fonttitle=\bfseries,
        title={Mechanistic Dimension Note (Canon)}
    }
}

% ============================================================
%  MECHANISTIC DIMENSION HELPER MACRO
% ============================================================
% Usage: \edcMechanismNote{bulk cause}{brane process}{3D output}
%
% Example:
%   \edcMechanismNote{Junction relaxes toward Steiner minimum}%
%                    {Energy pumps into brane-layer modes, redistributes}%
%                    {Electron, antineutrino, proton emerge on 3D side}
%
\newcommand{\edcMechanismNote}[3]{%
\begin{tcolorbox}[edcMechanism]
\begin{itemize}[nosep,leftmargin=*]
    \item \textbf{5D cause (bulk):} #1
    \item \textbf{Brane-layer process:} #2
    \item \textbf{3D observation (output):} #3
\end{itemize}
\vspace{0.3em}
\footnotesize\textit{Ledger closure must hold: bulk + brane + 3D outputs conserve energy/quantum numbers.}
\end{tcolorbox}
}

% ============================================================
%  RELATED DOCUMENTS MACRO
% ============================================================
% Usage: \edcRelatedDocs{main paper title}{main DOI}{companion list}
%
% Example:
%     A: \emph{Effective Lagrangian} (\href{...}{DOI}) $\cdot$
%     B: \emph{WKB Prefactor} (\href{...}{DOI})
%   }

% NOTE: \edcRelatedDocs macro deprecated (DOI registry consolidated)
% Use consolidated Zenodo article as primary reference instead.

% ============================================================
%  DOI REGISTRY DEPRECATED
% ============================================================
% Previous individual DOIs have been deprecated.
% All EDC Weak Sector content is now consolidated into a single
% Zenodo article. See paper_3_series/19_edc_weak_sector_zenodo_article/

% ============================================================
%  PHYSICAL NARRATION RULE REMINDER
% ============================================================
% Every key equation MUST be accompanied by a physical narrative stating:
%   1. 5D cause: What changes in the bulk-core configuration?
%   2. Brane response: How does the brane absorb/redistribute energy?
%   3. 3D observable output: What do observers detect on the 3D side?
%
% This rule eliminates "numerology smell" by ensuring every formula
% has a mechanistic interpretation.

% ============================================================
%  END OF STYLE FILE
% ============================================================

%   % tikz_style_edc.tex — Reusable TikZ styles for EDC papers
% Version 1.0 — 2026-01-20
% Include via: % tikz_style_edc.tex — Reusable TikZ styles for EDC papers
% Version 1.0 — 2026-01-20
% Include via: % tikz_style_edc.tex — Reusable TikZ styles for EDC papers
% Version 1.0 — 2026-01-20
% Include via: \input{tikz_style_edc}

% ============================================================
% REQUIRED LIBRARIES (must be loaded in main document)
% ============================================================
% \usetikzlibrary{calc,angles,quotes,decorations.markings,decorations.pathmorphing,positioning}

% ============================================================
% POSITIONING DEFAULTS
% ============================================================
\tikzset{
    % Default node distances for horizontal/vertical layouts
    edc node distance/.style={node distance=1.6cm and 2.0cm},
    % Compact variant for dense diagrams
    edc compact/.style={node distance=1.2cm and 1.5cm},
    % Spread variant for clarity
    edc spread/.style={node distance=2.0cm and 2.5cm},
}

% ============================================================
% COLOR PALETTE (consistent with epistemic tags)
% ============================================================
\definecolor{edcBulk}{RGB}{220,50,50}        % Red tones for bulk/5D
\definecolor{edcBrane}{RGB}{50,150,50}       % Green tones for brane-layer
\definecolor{edcOutput}{RGB}{50,100,200}     % Blue tones for 3D outputs
\definecolor{edcNeutral}{RGB}{100,100,100}   % Gray for neutral/annotations

% ============================================================
% BOX STYLES
% ============================================================
\tikzset{
    % Generic EDC box (base style)
    edc box/.style={
        rectangle,
        draw,
        rounded corners=3pt,
        minimum width=2.2cm,
        minimum height=0.8cm,
        align=center,
        font=\small,
        inner sep=4pt,
    },
    % Bulk-core box (red family)
    bulk box/.style={
        edc box,
        fill=red!10,
        draw=edcBulk!70!black,
        text=black,
    },
    % Brane-layer box (green family)
    brane box/.style={
        edc box,
        fill=green!10,
        draw=edcBrane!70!black,
        text=black,
    },
    % 3D output box (blue family)
    output box/.style={
        edc box,
        fill=blue!10,
        draw=edcOutput!70!black,
        text=black,
    },
    % Neutral/process box
    process box/.style={
        edc box,
        fill=gray!10,
        draw=gray!60!black,
        text=black,
    },
    % Label-only box (no background)
    label box/.style={
        rectangle,
        rounded corners=2pt,
        draw=gray!40,
        fill=white,
        inner sep=2pt,
        font=\scriptsize,
    },
}

% ============================================================
% ARROW STYLES
% ============================================================
\tikzset{
    % Standard thick arrow
    edc arrow/.style={
        ->,
        >=stealth,
        thick,
    },
    % Emphasized arrow (for main flow)
    edc flow/.style={
        ->,
        >=stealth,
        very thick,
        line width=1.2pt,
    },
    % Dashed arrow (for optional/weak connections)
    edc dashed/.style={
        ->,
        >=stealth,
        thick,
        dashed,
    },
    % Double arrow (for bidirectional)
    edc bidir/.style={
        <->,
        >=stealth,
        thick,
    },
}

% ============================================================
% REGION STYLES (for background fills)
% ============================================================
\tikzset{
    % Bulk region (5D)
    bulk region/.style={
        fill=blue!8,
    },
    % Brane layer region
    brane region/.style={
        fill=yellow!25,
    },
    % Observer/3D region
    observer region/.style={
        fill=green!8,
    },
}

% ============================================================
% LABEL STYLES
% ============================================================
\tikzset{
    % Phase label (below nodes)
    phase label/.style={
        font=\scriptsize\itshape,
        text=black!70,
    },
    % Equation label (for inline math)
    eq label/.style={
        font=\scriptsize,
        fill=white,
        inner sep=1pt,
    },
    % Section annotation
    section label/.style={
        font=\footnotesize\bfseries,
        text=black,
    },
}

% ============================================================
% JUNCTION/PARTICLE STYLES
% ============================================================
\tikzset{
    % Y-junction point
    junction point/.style={
        circle,
        fill=red!60!black,
        minimum size=4pt,
        inner sep=0pt,
    },
    % Flux tube arm
    flux arm/.style={
        thick,
        blue!60!black,
    },
    % Particle dot (electron, etc.)
    particle/.style={
        circle,
        fill=black,
        minimum size=5pt,
        inner sep=0pt,
    },
    % Neutrino (smaller, gray)
    neutrino/.style={
        circle,
        fill=gray,
        minimum size=4pt,
        inner sep=0pt,
    },
}

% ============================================================
% SPRING DECORATION (for mechanical models)
% ============================================================
\tikzset{
    spring/.style={
        thick,
        decorate,
        decoration={
            coil,
            aspect=0.5,
            segment length=2mm,
            amplitude=2mm,
        },
    },
    % Wave decoration (for field modes)
    wave field/.style={
        thick,
        decorate,
        decoration={
            snake,
            amplitude=2pt,
            segment length=8pt,
        },
    },
}

% ============================================================
% BOUNDARY STYLES
% ============================================================
\tikzset{
    % Bulk-facing boundary (dashed red)
    bulk boundary/.style={
        very thick,
        red!70!black,
        dashed,
    },
    % Observer-facing boundary (solid green)
    observer boundary/.style={
        thick,
        green!50!black,
    },
    % Brane edge (orange)
    brane edge/.style={
        thick,
        orange!70!black,
    },
}

% ============================================================
% CONVENIENCE COMMANDS
% ============================================================
% Arrow label (above)
\newcommand{\arrlabel}[1]{\scriptsize #1}
% Arrow label (below)
\newcommand{\arrlabelb}[1]{\scriptsize #1}

% ============================================================
% END OF STYLE FILE
% ============================================================


% ============================================================
% REQUIRED LIBRARIES (must be loaded in main document)
% ============================================================
% \usetikzlibrary{calc,angles,quotes,decorations.markings,decorations.pathmorphing,positioning}

% ============================================================
% POSITIONING DEFAULTS
% ============================================================
\tikzset{
    % Default node distances for horizontal/vertical layouts
    edc node distance/.style={node distance=1.6cm and 2.0cm},
    % Compact variant for dense diagrams
    edc compact/.style={node distance=1.2cm and 1.5cm},
    % Spread variant for clarity
    edc spread/.style={node distance=2.0cm and 2.5cm},
}

% ============================================================
% COLOR PALETTE (consistent with epistemic tags)
% ============================================================
\definecolor{edcBulk}{RGB}{220,50,50}        % Red tones for bulk/5D
\definecolor{edcBrane}{RGB}{50,150,50}       % Green tones for brane-layer
\definecolor{edcOutput}{RGB}{50,100,200}     % Blue tones for 3D outputs
\definecolor{edcNeutral}{RGB}{100,100,100}   % Gray for neutral/annotations

% ============================================================
% BOX STYLES
% ============================================================
\tikzset{
    % Generic EDC box (base style)
    edc box/.style={
        rectangle,
        draw,
        rounded corners=3pt,
        minimum width=2.2cm,
        minimum height=0.8cm,
        align=center,
        font=\small,
        inner sep=4pt,
    },
    % Bulk-core box (red family)
    bulk box/.style={
        edc box,
        fill=red!10,
        draw=edcBulk!70!black,
        text=black,
    },
    % Brane-layer box (green family)
    brane box/.style={
        edc box,
        fill=green!10,
        draw=edcBrane!70!black,
        text=black,
    },
    % 3D output box (blue family)
    output box/.style={
        edc box,
        fill=blue!10,
        draw=edcOutput!70!black,
        text=black,
    },
    % Neutral/process box
    process box/.style={
        edc box,
        fill=gray!10,
        draw=gray!60!black,
        text=black,
    },
    % Label-only box (no background)
    label box/.style={
        rectangle,
        rounded corners=2pt,
        draw=gray!40,
        fill=white,
        inner sep=2pt,
        font=\scriptsize,
    },
}

% ============================================================
% ARROW STYLES
% ============================================================
\tikzset{
    % Standard thick arrow
    edc arrow/.style={
        ->,
        >=stealth,
        thick,
    },
    % Emphasized arrow (for main flow)
    edc flow/.style={
        ->,
        >=stealth,
        very thick,
        line width=1.2pt,
    },
    % Dashed arrow (for optional/weak connections)
    edc dashed/.style={
        ->,
        >=stealth,
        thick,
        dashed,
    },
    % Double arrow (for bidirectional)
    edc bidir/.style={
        <->,
        >=stealth,
        thick,
    },
}

% ============================================================
% REGION STYLES (for background fills)
% ============================================================
\tikzset{
    % Bulk region (5D)
    bulk region/.style={
        fill=blue!8,
    },
    % Brane layer region
    brane region/.style={
        fill=yellow!25,
    },
    % Observer/3D region
    observer region/.style={
        fill=green!8,
    },
}

% ============================================================
% LABEL STYLES
% ============================================================
\tikzset{
    % Phase label (below nodes)
    phase label/.style={
        font=\scriptsize\itshape,
        text=black!70,
    },
    % Equation label (for inline math)
    eq label/.style={
        font=\scriptsize,
        fill=white,
        inner sep=1pt,
    },
    % Section annotation
    section label/.style={
        font=\footnotesize\bfseries,
        text=black,
    },
}

% ============================================================
% JUNCTION/PARTICLE STYLES
% ============================================================
\tikzset{
    % Y-junction point
    junction point/.style={
        circle,
        fill=red!60!black,
        minimum size=4pt,
        inner sep=0pt,
    },
    % Flux tube arm
    flux arm/.style={
        thick,
        blue!60!black,
    },
    % Particle dot (electron, etc.)
    particle/.style={
        circle,
        fill=black,
        minimum size=5pt,
        inner sep=0pt,
    },
    % Neutrino (smaller, gray)
    neutrino/.style={
        circle,
        fill=gray,
        minimum size=4pt,
        inner sep=0pt,
    },
}

% ============================================================
% SPRING DECORATION (for mechanical models)
% ============================================================
\tikzset{
    spring/.style={
        thick,
        decorate,
        decoration={
            coil,
            aspect=0.5,
            segment length=2mm,
            amplitude=2mm,
        },
    },
    % Wave decoration (for field modes)
    wave field/.style={
        thick,
        decorate,
        decoration={
            snake,
            amplitude=2pt,
            segment length=8pt,
        },
    },
}

% ============================================================
% BOUNDARY STYLES
% ============================================================
\tikzset{
    % Bulk-facing boundary (dashed red)
    bulk boundary/.style={
        very thick,
        red!70!black,
        dashed,
    },
    % Observer-facing boundary (solid green)
    observer boundary/.style={
        thick,
        green!50!black,
    },
    % Brane edge (orange)
    brane edge/.style={
        thick,
        orange!70!black,
    },
}

% ============================================================
% CONVENIENCE COMMANDS
% ============================================================
% Arrow label (above)
\newcommand{\arrlabel}[1]{\scriptsize #1}
% Arrow label (below)
\newcommand{\arrlabelb}[1]{\scriptsize #1}

% ============================================================
% END OF STYLE FILE
% ============================================================


% ============================================================
% REQUIRED LIBRARIES (must be loaded in main document)
% ============================================================
% \usetikzlibrary{calc,angles,quotes,decorations.markings,decorations.pathmorphing,positioning}

% ============================================================
% POSITIONING DEFAULTS
% ============================================================
\tikzset{
    % Default node distances for horizontal/vertical layouts
    edc node distance/.style={node distance=1.6cm and 2.0cm},
    % Compact variant for dense diagrams
    edc compact/.style={node distance=1.2cm and 1.5cm},
    % Spread variant for clarity
    edc spread/.style={node distance=2.0cm and 2.5cm},
}

% ============================================================
% COLOR PALETTE (consistent with epistemic tags)
% ============================================================
\definecolor{edcBulk}{RGB}{220,50,50}        % Red tones for bulk/5D
\definecolor{edcBrane}{RGB}{50,150,50}       % Green tones for brane-layer
\definecolor{edcOutput}{RGB}{50,100,200}     % Blue tones for 3D outputs
\definecolor{edcNeutral}{RGB}{100,100,100}   % Gray for neutral/annotations

% ============================================================
% BOX STYLES
% ============================================================
\tikzset{
    % Generic EDC box (base style)
    edc box/.style={
        rectangle,
        draw,
        rounded corners=3pt,
        minimum width=2.2cm,
        minimum height=0.8cm,
        align=center,
        font=\small,
        inner sep=4pt,
    },
    % Bulk-core box (red family)
    bulk box/.style={
        edc box,
        fill=red!10,
        draw=edcBulk!70!black,
        text=black,
    },
    % Brane-layer box (green family)
    brane box/.style={
        edc box,
        fill=green!10,
        draw=edcBrane!70!black,
        text=black,
    },
    % 3D output box (blue family)
    output box/.style={
        edc box,
        fill=blue!10,
        draw=edcOutput!70!black,
        text=black,
    },
    % Neutral/process box
    process box/.style={
        edc box,
        fill=gray!10,
        draw=gray!60!black,
        text=black,
    },
    % Label-only box (no background)
    label box/.style={
        rectangle,
        rounded corners=2pt,
        draw=gray!40,
        fill=white,
        inner sep=2pt,
        font=\scriptsize,
    },
}

% ============================================================
% ARROW STYLES
% ============================================================
\tikzset{
    % Standard thick arrow
    edc arrow/.style={
        ->,
        >=stealth,
        thick,
    },
    % Emphasized arrow (for main flow)
    edc flow/.style={
        ->,
        >=stealth,
        very thick,
        line width=1.2pt,
    },
    % Dashed arrow (for optional/weak connections)
    edc dashed/.style={
        ->,
        >=stealth,
        thick,
        dashed,
    },
    % Double arrow (for bidirectional)
    edc bidir/.style={
        <->,
        >=stealth,
        thick,
    },
}

% ============================================================
% REGION STYLES (for background fills)
% ============================================================
\tikzset{
    % Bulk region (5D)
    bulk region/.style={
        fill=blue!8,
    },
    % Brane layer region
    brane region/.style={
        fill=yellow!25,
    },
    % Observer/3D region
    observer region/.style={
        fill=green!8,
    },
}

% ============================================================
% LABEL STYLES
% ============================================================
\tikzset{
    % Phase label (below nodes)
    phase label/.style={
        font=\scriptsize\itshape,
        text=black!70,
    },
    % Equation label (for inline math)
    eq label/.style={
        font=\scriptsize,
        fill=white,
        inner sep=1pt,
    },
    % Section annotation
    section label/.style={
        font=\footnotesize\bfseries,
        text=black,
    },
}

% ============================================================
% JUNCTION/PARTICLE STYLES
% ============================================================
\tikzset{
    % Y-junction point
    junction point/.style={
        circle,
        fill=red!60!black,
        minimum size=4pt,
        inner sep=0pt,
    },
    % Flux tube arm
    flux arm/.style={
        thick,
        blue!60!black,
    },
    % Particle dot (electron, etc.)
    particle/.style={
        circle,
        fill=black,
        minimum size=5pt,
        inner sep=0pt,
    },
    % Neutrino (smaller, gray)
    neutrino/.style={
        circle,
        fill=gray,
        minimum size=4pt,
        inner sep=0pt,
    },
}

% ============================================================
% SPRING DECORATION (for mechanical models)
% ============================================================
\tikzset{
    spring/.style={
        thick,
        decorate,
        decoration={
            coil,
            aspect=0.5,
            segment length=2mm,
            amplitude=2mm,
        },
    },
    % Wave decoration (for field modes)
    wave field/.style={
        thick,
        decorate,
        decoration={
            snake,
            amplitude=2pt,
            segment length=8pt,
        },
    },
}

% ============================================================
% BOUNDARY STYLES
% ============================================================
\tikzset{
    % Bulk-facing boundary (dashed red)
    bulk boundary/.style={
        very thick,
        red!70!black,
        dashed,
    },
    % Observer-facing boundary (solid green)
    observer boundary/.style={
        thick,
        green!50!black,
    },
    % Brane edge (orange)
    brane edge/.style={
        thick,
        orange!70!black,
    },
}

% ============================================================
% CONVENIENCE COMMANDS
% ============================================================
% Arrow label (above)
\newcommand{\arrlabel}[1]{\scriptsize #1}
% Arrow label (below)
\newcommand{\arrlabelb}[1]{\scriptsize #1}

% ============================================================
% END OF STYLE FILE
% ============================================================
  % if using TikZ figures
%
% REQUIRED PACKAGES (load these in main document before \input):
%   fontspec, amsmath, amssymb, amsthm, mathtools, geometry
%   hyperref, enumitem, booktabs, array, xcolor, tcolorbox
%
% ============================================================

% ============================================================
%  EPISTEMIC TAG COLORS
% ============================================================
\definecolor{tagDer}{RGB}{0,128,0}      % Green - Derived
\definecolor{tagDc}{RGB}{0,0,200}       % Blue - Deduced/Constrained
\definecolor{tagCal}{RGB}{200,0,0}      % Red - Calibrated
\definecolor{tagP}{RGB}{128,0,128}      % Purple - Postulated
\definecolor{tagBL}{RGB}{128,128,128}   % Gray - Baseline
\definecolor{tagI}{RGB}{255,140,0}      % Orange - Identified
\definecolor{tagOpen}{RGB}{200,100,0}   % Dark orange - Open

% ============================================================
%  EPISTEMIC TAG COMMANDS
% ============================================================
% Use these to mark claims with their epistemic status
\newcommand{\tagDer}{\textcolor{tagDer}{\textbf{[Der]}}}    % Derived from axioms
\newcommand{\tagDc}{\textcolor{tagDc}{\textbf{[Dc]}}}       % Deduced/Constrained
\newcommand{\tagCal}{\textcolor{tagCal}{\textbf{[Cal]}}}    % Calibrated (fitted)
\newcommand{\tagP}{\textcolor{tagP}{\textbf{[P]}}}          % Postulated
\newcommand{\tagBL}{\textcolor{tagBL}{\textbf{[BL]}}}       % Baseline (external fact)
\newcommand{\tagI}{\textcolor{tagI}{\textbf{[I]}}}          % Identified (pattern match)
\newcommand{\tagOpen}{\textcolor{tagOpen}{\textbf{[OPEN]}}} % Open problem
\newcommand{\tagDef}{\textcolor{tagDc}{\textbf{[Def]}}}     % Definition

% ============================================================
%  THEOREM ENVIRONMENTS
% ============================================================
\newtheorem{postulate}{Postulate}
\newtheorem{definition}{Definition}[section]
\newtheorem{theorem}{Theorem}[section]
\newtheorem{lemma}[theorem]{Lemma}
\newtheorem{corollary}[theorem]{Corollary}
\newtheorem{proposition}[theorem]{Proposition}
\newtheorem{remark}{Remark}[section]

% ============================================================
%  COMMON EDC SYMBOLS
% ============================================================
% Symmetry groups
\newcommand{\Ztwo}{\mathbb{Z}_2}
\newcommand{\Zthree}{\mathbb{Z}_3}
\newcommand{\Ztri}{\mathbb{Z}_3}    % alias
\newcommand{\Zsix}{\mathbb{Z}_6}

% Geometric objects
\newcommand{\Sthree}{S^3}           % 3-sphere
\newcommand{\Stwo}{S^2}             % 2-sphere
\newcommand{\Bthree}{B^3}           % 3-ball
\newcommand{\Mfive}{\mathcal{M}_5}  % 5D manifold
\newcommand{\Bfour}{\mathcal{B}_4}  % 4D brane

% Physical quantities
\newcommand{\tension}{\tau}         % string/flux-tube tension (E/L)
\newcommand{\re}{r_e}               % electron radius

% Operators
\newcommand{\Pfrozen}{\mathcal{P}_{\mathrm{frozen}}}  % Frozen projection operator
\newcommand{\Ebrane}{\mathcal{E}_{\mathrm{brane}}}    % Brane energy store

% Bulk-brane exchange current (canonical notation from Framework v2.0)
\newcommand{\Jbb}[1]{J^{#1}_{\mathrm{bulk}\to\mathrm{brane}}}

% ============================================================
%  TCOLORBOX STYLES FOR EDC PAPERS
% ============================================================
% Cornerstone box (blue) — key claims/foundations
\tcbset{
    edcCornerstone/.style={
        colback=blue!5,
        colframe=blue!40!black,
        fonttitle=\bfseries
    }
}

% Guardrail box (gray) — epistemic warnings/constraints
\tcbset{
    edcGuardrail/.style={
        colback=gray!5!white,
        colframe=gray!60!black,
        fonttitle=\bfseries
    }
}

% PPN box (blue, lighter) — Physical Process Narrative
\tcbset{
    edcPPN/.style={
        colback=blue!5,
        colframe=blue!50!black,
        fonttitle=\bfseries
    }
}

% Canonical box (yellow/orange) — canonical definitions/glossary
\tcbset{
    edcCanonical/.style={
        colback=yellow!5,
        colframe=orange!60!black,
        fonttitle=\bfseries
    }
}

% Conceptual box (yellow/orange, lighter) — conceptual pictures
\tcbset{
    edcConcept/.style={
        colback=yellow!5,
        colframe=orange!50!black,
        fonttitle=\bfseries
    }
}

% Pathway box (purple) — energy pathways, mechanisms
\tcbset{
    edcPathway/.style={
        colback=purple!5,
        colframe=purple!40!black,
        fonttitle=\bfseries
    }
}

% Model box (green) — mechanical analogies, heuristics
\tcbset{
    edcModel/.style={
        colback=green!5,
        colframe=green!40!black,
        fonttitle=\bfseries
    }
}

% Warning box (red) — non-overclaim, limitations
\tcbset{
    edcWarning/.style={
        colback=red!5,
        colframe=red!40!black,
        fonttitle=\bfseries
    }
}

% Framework quote box (gray) — verbatim from Framework v2.0
\tcbset{
    edcFramework/.style={
        colback=gray!5!white,
        colframe=gray!60!black,
        fonttitle=\small
    }
}

% Mechanism box (teal) — mechanistic dimension principle narrative
\tcbset{
    edcMechanism/.style={
        colback=teal!5,
        colframe=teal!50!black,
        fonttitle=\bfseries,
        title={Mechanistic Dimension Note (Canon)}
    }
}

% ============================================================
%  MECHANISTIC DIMENSION HELPER MACRO
% ============================================================
% Usage: \edcMechanismNote{bulk cause}{brane process}{3D output}
%
% Example:
%   \edcMechanismNote{Junction relaxes toward Steiner minimum}%
%                    {Energy pumps into brane-layer modes, redistributes}%
%                    {Electron, antineutrino, proton emerge on 3D side}
%
\newcommand{\edcMechanismNote}[3]{%
\begin{tcolorbox}[edcMechanism]
\begin{itemize}[nosep,leftmargin=*]
    \item \textbf{5D cause (bulk):} #1
    \item \textbf{Brane-layer process:} #2
    \item \textbf{3D observation (output):} #3
\end{itemize}
\vspace{0.3em}
\footnotesize\textit{Ledger closure must hold: bulk + brane + 3D outputs conserve energy/quantum numbers.}
\end{tcolorbox}
}

% ============================================================
%  RELATED DOCUMENTS MACRO
% ============================================================
% Usage: \edcRelatedDocs{main paper title}{main DOI}{companion list}
%
% Example:
%     A: \emph{Effective Lagrangian} (\href{...}{DOI}) $\cdot$
%     B: \emph{WKB Prefactor} (\href{...}{DOI})
%   }

% NOTE: \edcRelatedDocs macro deprecated (DOI registry consolidated)
% Use consolidated Zenodo article as primary reference instead.

% ============================================================
%  DOI REGISTRY DEPRECATED
% ============================================================
% Previous individual DOIs have been deprecated.
% All EDC Weak Sector content is now consolidated into a single
% Zenodo article. See paper_3_series/19_edc_weak_sector_zenodo_article/

% ============================================================
%  PHYSICAL NARRATION RULE REMINDER
% ============================================================
% Every key equation MUST be accompanied by a physical narrative stating:
%   1. 5D cause: What changes in the bulk-core configuration?
%   2. Brane response: How does the brane absorb/redistribute energy?
%   3. 3D observable output: What do observers detect on the 3D side?
%
% This rule eliminates "numerology smell" by ensuring every formula
% has a mechanistic interpretation.

% ============================================================
%  END OF STYLE FILE
% ============================================================
