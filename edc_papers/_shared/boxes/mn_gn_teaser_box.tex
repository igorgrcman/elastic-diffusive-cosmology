% =============================================================================
% mn_gn_teaser_box.tex
%
% Compact box: Frustration-Corrected Geiger-Nuttall Law (teaser for papers)
%
% Include with: % =============================================================================
% mn_gn_teaser_box.tex
%
% Compact box: Frustration-Corrected Geiger-Nuttall Law (teaser for papers)
%
% Include with: % =============================================================================
% mn_gn_teaser_box.tex
%
% Compact box: Frustration-Corrected Geiger-Nuttall Law (teaser for papers)
%
% Include with: % =============================================================================
% mn_gn_teaser_box.tex
%
% Compact box: Frustration-Corrected Geiger-Nuttall Law (teaser for papers)
%
% Include with: \input{../../edc_papers/_shared/boxes/mn_gn_teaser_box}
%
% Status: [I/Cal]
% Date: 2026-01-29
% =============================================================================

\begin{tcolorbox}[colback=green!5!white, colframe=green!75!black,
                  title={Frustration-Corrected Geiger-Nuttall Law [I/Cal]}]

\textbf{Formula:}
\begin{equation*}
    \log_{10}(t_{1/2}) = 1.63 \cdot \frac{Z}{\sqrt{Q_\alpha}} - 2.40 \cdot \varepsilon_f - 42.1
\end{equation*}

\textbf{Key insight:} The optimal coordination $n = 43$ for nuclear matter is
\emph{topologically forbidden} ($43$ is prime $> 3$), creating geometric frustration
that drives alpha decay.

\textbf{Result:} $R^2 = 0.994$, \textbf{44.7\% improvement} in MAE over standard Geiger-Nuttall
(21 nuclei, Po--Cf).

\textbf{Metric:} MAE on $\log_{10}(t_{1/2})$ [Cal]; improvement = $(0.56 - 0.31)/0.56 \times 100\%$.

\textbf{Epistemic:}
\begin{itemize}
    \item[\textbf{[I]}] Correlation identified
    \item[\textbf{[Cal]}] Coefficients fitted to data
    \item[\textbf{[P]}] $n_{\text{eff}}(A)$ interpolation phenomenological
\end{itemize}

\textit{Full derivation in Book 3: Nuclear Structure from 5D Topology.}

\end{tcolorbox}

%
% Status: [I/Cal]
% Date: 2026-01-29
% =============================================================================

\begin{tcolorbox}[colback=green!5!white, colframe=green!75!black,
                  title={Frustration-Corrected Geiger-Nuttall Law [I/Cal]}]

\textbf{Formula:}
\begin{equation*}
    \log_{10}(t_{1/2}) = 1.63 \cdot \frac{Z}{\sqrt{Q_\alpha}} - 2.40 \cdot \varepsilon_f - 42.1
\end{equation*}

\textbf{Key insight:} The optimal coordination $n = 43$ for nuclear matter is
\emph{topologically forbidden} ($43$ is prime $> 3$), creating geometric frustration
that drives alpha decay.

\textbf{Result:} $R^2 = 0.994$, \textbf{44.7\% improvement} in MAE over standard Geiger-Nuttall
(21 nuclei, Po--Cf).

\textbf{Metric:} MAE on $\log_{10}(t_{1/2})$ [Cal]; improvement = $(0.56 - 0.31)/0.56 \times 100\%$.

\textbf{Epistemic:}
\begin{itemize}
    \item[\textbf{[I]}] Correlation identified
    \item[\textbf{[Cal]}] Coefficients fitted to data
    \item[\textbf{[P]}] $n_{\text{eff}}(A)$ interpolation phenomenological
\end{itemize}

\textit{Full derivation in Book 3: Nuclear Structure from 5D Topology.}

\end{tcolorbox}

%
% Status: [I/Cal]
% Date: 2026-01-29
% =============================================================================

\begin{tcolorbox}[colback=green!5!white, colframe=green!75!black,
                  title={Frustration-Corrected Geiger-Nuttall Law [I/Cal]}]

\textbf{Formula:}
\begin{equation*}
    \log_{10}(t_{1/2}) = 1.63 \cdot \frac{Z}{\sqrt{Q_\alpha}} - 2.40 \cdot \varepsilon_f - 42.1
\end{equation*}

\textbf{Key insight:} The optimal coordination $n = 43$ for nuclear matter is
\emph{topologically forbidden} ($43$ is prime $> 3$), creating geometric frustration
that drives alpha decay.

\textbf{Result:} $R^2 = 0.994$, \textbf{44.7\% improvement} in MAE over standard Geiger-Nuttall
(21 nuclei, Po--Cf).

\textbf{Metric:} MAE on $\log_{10}(t_{1/2})$ [Cal]; improvement = $(0.56 - 0.31)/0.56 \times 100\%$.

\textbf{Epistemic:}
\begin{itemize}
    \item[\textbf{[I]}] Correlation identified
    \item[\textbf{[Cal]}] Coefficients fitted to data
    \item[\textbf{[P]}] $n_{\text{eff}}(A)$ interpolation phenomenological
\end{itemize}

\textit{Full derivation in Book 3: Nuclear Structure from 5D Topology.}

\end{tcolorbox}

%
% Status: [I/Cal]
% Date: 2026-01-29
% =============================================================================

\begin{tcolorbox}[colback=green!5!white, colframe=green!75!black,
                  title={Frustration-Corrected Geiger-Nuttall Law [I/Cal]}]

\textbf{Formula:}
\begin{equation*}
    \log_{10}(t_{1/2}) = 1.63 \cdot \frac{Z}{\sqrt{Q_\alpha}} - 2.40 \cdot \varepsilon_f - 42.1
\end{equation*}

\textbf{Key insight:} The optimal coordination $n = 43$ for nuclear matter is
\emph{topologically forbidden} ($43$ is prime $> 3$), creating geometric frustration
that drives alpha decay.

\textbf{Result:} $R^2 = 0.994$, \textbf{44.7\% improvement} in MAE over standard Geiger-Nuttall
(21 nuclei, Po--Cf).

\textbf{Metric:} MAE on $\log_{10}(t_{1/2})$ [Cal]; improvement = $(0.56 - 0.31)/0.56 \times 100\%$.

\textbf{Epistemic:}
\begin{itemize}
    \item[\textbf{[I]}] Correlation identified
    \item[\textbf{[Cal]}] Coefficients fitted to data
    \item[\textbf{[P]}] $n_{\text{eff}}(A)$ interpolation phenomenological
\end{itemize}

\textit{Full derivation in Book 3: Nuclear Structure from 5D Topology.}

\end{tcolorbox}
