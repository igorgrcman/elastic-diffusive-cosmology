% ═══════════════════════════════════════════════════════════════════════════════
% DELTA FROM 5D ACTION BOX
% ═══════════════════════════════════════════════════════════════════════════════
%
% File: delta_from_5d_action_box.tex
% Created: 2026-01-29
% Issue: OPR-21c — δ derivation from 5D action
% Status: YELLOW [Dc] — Principled but model-dependent
%
% Usage: % ═══════════════════════════════════════════════════════════════════════════════
% DELTA FROM 5D ACTION BOX
% ═══════════════════════════════════════════════════════════════════════════════
%
% File: delta_from_5d_action_box.tex
% Created: 2026-01-29
% Issue: OPR-21c — δ derivation from 5D action
% Status: YELLOW [Dc] — Principled but model-dependent
%
% Usage: % ═══════════════════════════════════════════════════════════════════════════════
% DELTA FROM 5D ACTION BOX
% ═══════════════════════════════════════════════════════════════════════════════
%
% File: delta_from_5d_action_box.tex
% Created: 2026-01-29
% Issue: OPR-21c — δ derivation from 5D action
% Status: YELLOW [Dc] — Principled but model-dependent
%
% Usage: % ═══════════════════════════════════════════════════════════════════════════════
% DELTA FROM 5D ACTION BOX
% ═══════════════════════════════════════════════════════════════════════════════
%
% File: delta_from_5d_action_box.tex
% Created: 2026-01-29
% Issue: OPR-21c — δ derivation from 5D action
% Status: YELLOW [Dc] — Principled but model-dependent
%
% Usage: \input{../../edc_papers/_shared/boxes/delta_from_5d_action_box}
%
% Dependencies: tcolorbox, amsmath, amssymb
%
% ═══════════════════════════════════════════════════════════════════════════════

\begin{tcolorbox}[breakable, enhanced, colback=yellow!5!white, colframe=yellow!60!black,
    title={\textbf{Brane Thickness from 5D Action: $\delta = \hbar/(2m_p c)$}}, width=\linewidth]

\textbf{Result:}
\begin{equation}
\boxed{\delta = \frac{\hbar}{2 m_p c} = 0.533\,\text{GeV}^{-1} = 0.105\,\text{fm}}
\end{equation}
\textit{The brane thickness equals half the proton Compton wavelength.}

\vspace{0.5em}
\textbf{Derivation Chain} \tagDc{}:
\begin{enumerate}[nosep]
    \item \textbf{5D action $\to$ mode equation} \tagDer{}: KK reduction gives
    \[
    -\frac{d^2 w}{d\chi^2} + V(\chi)\,w = \lambda\,w
    \]
    \item \textbf{Potential scaling} \tagDc{}: Well depth $V_0 \sim 1/\delta^2$, width $\sim \delta$
    \item \textbf{Ground state energy}: $E_0 = c_E/\delta$ where $c_E = O(1)$
    \begin{itemize}[nosep]
        \item \textbf{Scaling} \tagDer{}: $E_0 \propto 1/\delta$ is dimensional/robust
        \item \textbf{Coefficient} \tagDc{}: $c_E$ depends on $\omega = \sqrt{V''/M_{\text{eff}}}$
    \end{itemize}
    \item \textbf{Proton identification} \tagP{}: Postulate: proton = localized fermionic mode
    \item \textbf{Energy matching} \tagDc{}: $m_p = E_0 \Rightarrow \delta = c_\delta/m_p$
\end{enumerate}

\vspace{0.5em}
\textbf{Coefficient origin:} The factor $c_\delta = 1/2$ arises from harmonic approximation:
\[
\omega = \sqrt{\frac{V''(0)}{M_{\text{eff}}}} \sim \frac{1}{\delta}, \quad
E_0 = \frac{\omega}{2} = \frac{1}{2\delta} = m_p
\quad\Rightarrow\quad \delta = \frac{1}{2m_p}
\]
\textbf{Open:} Exact $c_\delta$ requires $V''(0)$ from bulk EOM.

\vspace{0.5em}
\textbf{Explicit assumptions:}
\begin{itemize}[nosep]
    \item Thick brane structure with localized potential \tagP{}
    \item Proton identified as lightest bound fermionic mode \tagP{}
    \item Single-scale dominance (no hierarchy of widths) \tagDc{}
    \item Harmonic approximation for coefficient \tagDc{}
\end{itemize}

\vspace{0.5em}
\textbf{What this achieves:}
\begin{itemize}[nosep]
    \item \textbf{Scaling $\delta \propto 1/m_p$}: Derived (dimensional) \checkmark
    \item \textbf{Coefficient $c_\delta = 1/2$}: From harmonic approx (model-dependent) \checkmark
    \item Connects brane geometry to hadron physics \checkmark
\end{itemize}

\vspace{0.5em}
\textbf{What remains open:}
\begin{itemize}[nosep]
    \item Potential shape $V(\chi)$ from bulk EOM (currently ansatz) $\times$
    \item Exact $c_\delta$: requires $V''(0)$ — why $1/2$ vs $1/\pi$? $\times$
    \item Proton mass itself (input, not output) $\times$
    \item Topological justification for proton = bound mode $\times$
\end{itemize}

\vspace{0.5em}
\begin{center}
\fbox{\parbox{0.85\linewidth}{\centering
\textbf{Status: YELLOW [Dc]} --- Principled derivation, model-dependent coefficient\\[0.3em]
\textbf{Upgrade path:} Derive $V(\chi)$ from 5D action; compute exact bound state; justify proton ID from topology
}}
\end{center}

\vspace{0.3em}
\textit{Full derivation:} \texttt{edc\_papers/\_shared/derivations/delta\_from\_5d\_action\_proton\_scale.tex}

\end{tcolorbox}

% ═══════════════════════════════════════════════════════════════════════════════
% END OF BOX
% ═══════════════════════════════════════════════════════════════════════════════

%
% Dependencies: tcolorbox, amsmath, amssymb
%
% ═══════════════════════════════════════════════════════════════════════════════

\begin{tcolorbox}[breakable, enhanced, colback=yellow!5!white, colframe=yellow!60!black,
    title={\textbf{Brane Thickness from 5D Action: $\delta = \hbar/(2m_p c)$}}, width=\linewidth]

\textbf{Result:}
\begin{equation}
\boxed{\delta = \frac{\hbar}{2 m_p c} = 0.533\,\text{GeV}^{-1} = 0.105\,\text{fm}}
\end{equation}
\textit{The brane thickness equals half the proton Compton wavelength.}

\vspace{0.5em}
\textbf{Derivation Chain} \tagDc{}:
\begin{enumerate}[nosep]
    \item \textbf{5D action $\to$ mode equation} \tagDer{}: KK reduction gives
    \[
    -\frac{d^2 w}{d\chi^2} + V(\chi)\,w = \lambda\,w
    \]
    \item \textbf{Potential scaling} \tagDc{}: Well depth $V_0 \sim 1/\delta^2$, width $\sim \delta$
    \item \textbf{Ground state energy}: $E_0 = c_E/\delta$ where $c_E = O(1)$
    \begin{itemize}[nosep]
        \item \textbf{Scaling} \tagDer{}: $E_0 \propto 1/\delta$ is dimensional/robust
        \item \textbf{Coefficient} \tagDc{}: $c_E$ depends on $\omega = \sqrt{V''/M_{\text{eff}}}$
    \end{itemize}
    \item \textbf{Proton identification} \tagP{}: Postulate: proton = localized fermionic mode
    \item \textbf{Energy matching} \tagDc{}: $m_p = E_0 \Rightarrow \delta = c_\delta/m_p$
\end{enumerate}

\vspace{0.5em}
\textbf{Coefficient origin:} The factor $c_\delta = 1/2$ arises from harmonic approximation:
\[
\omega = \sqrt{\frac{V''(0)}{M_{\text{eff}}}} \sim \frac{1}{\delta}, \quad
E_0 = \frac{\omega}{2} = \frac{1}{2\delta} = m_p
\quad\Rightarrow\quad \delta = \frac{1}{2m_p}
\]
\textbf{Open:} Exact $c_\delta$ requires $V''(0)$ from bulk EOM.

\vspace{0.5em}
\textbf{Explicit assumptions:}
\begin{itemize}[nosep]
    \item Thick brane structure with localized potential \tagP{}
    \item Proton identified as lightest bound fermionic mode \tagP{}
    \item Single-scale dominance (no hierarchy of widths) \tagDc{}
    \item Harmonic approximation for coefficient \tagDc{}
\end{itemize}

\vspace{0.5em}
\textbf{What this achieves:}
\begin{itemize}[nosep]
    \item \textbf{Scaling $\delta \propto 1/m_p$}: Derived (dimensional) \checkmark
    \item \textbf{Coefficient $c_\delta = 1/2$}: From harmonic approx (model-dependent) \checkmark
    \item Connects brane geometry to hadron physics \checkmark
\end{itemize}

\vspace{0.5em}
\textbf{What remains open:}
\begin{itemize}[nosep]
    \item Potential shape $V(\chi)$ from bulk EOM (currently ansatz) $\times$
    \item Exact $c_\delta$: requires $V''(0)$ — why $1/2$ vs $1/\pi$? $\times$
    \item Proton mass itself (input, not output) $\times$
    \item Topological justification for proton = bound mode $\times$
\end{itemize}

\vspace{0.5em}
\begin{center}
\fbox{\parbox{0.85\linewidth}{\centering
\textbf{Status: YELLOW [Dc]} --- Principled derivation, model-dependent coefficient\\[0.3em]
\textbf{Upgrade path:} Derive $V(\chi)$ from 5D action; compute exact bound state; justify proton ID from topology
}}
\end{center}

\vspace{0.3em}
\textit{Full derivation:} \texttt{edc\_papers/\_shared/derivations/delta\_from\_5d\_action\_proton\_scale.tex}

\end{tcolorbox}

% ═══════════════════════════════════════════════════════════════════════════════
% END OF BOX
% ═══════════════════════════════════════════════════════════════════════════════

%
% Dependencies: tcolorbox, amsmath, amssymb
%
% ═══════════════════════════════════════════════════════════════════════════════

\begin{tcolorbox}[breakable, enhanced, colback=yellow!5!white, colframe=yellow!60!black,
    title={\textbf{Brane Thickness from 5D Action: $\delta = \hbar/(2m_p c)$}}, width=\linewidth]

\textbf{Result:}
\begin{equation}
\boxed{\delta = \frac{\hbar}{2 m_p c} = 0.533\,\text{GeV}^{-1} = 0.105\,\text{fm}}
\end{equation}
\textit{The brane thickness equals half the proton Compton wavelength.}

\vspace{0.5em}
\textbf{Derivation Chain} \tagDc{}:
\begin{enumerate}[nosep]
    \item \textbf{5D action $\to$ mode equation} \tagDer{}: KK reduction gives
    \[
    -\frac{d^2 w}{d\chi^2} + V(\chi)\,w = \lambda\,w
    \]
    \item \textbf{Potential scaling} \tagDc{}: Well depth $V_0 \sim 1/\delta^2$, width $\sim \delta$
    \item \textbf{Ground state energy}: $E_0 = c_E/\delta$ where $c_E = O(1)$
    \begin{itemize}[nosep]
        \item \textbf{Scaling} \tagDer{}: $E_0 \propto 1/\delta$ is dimensional/robust
        \item \textbf{Coefficient} \tagDc{}: $c_E$ depends on $\omega = \sqrt{V''/M_{\text{eff}}}$
    \end{itemize}
    \item \textbf{Proton identification} \tagP{}: Postulate: proton = localized fermionic mode
    \item \textbf{Energy matching} \tagDc{}: $m_p = E_0 \Rightarrow \delta = c_\delta/m_p$
\end{enumerate}

\vspace{0.5em}
\textbf{Coefficient origin:} The factor $c_\delta = 1/2$ arises from harmonic approximation:
\[
\omega = \sqrt{\frac{V''(0)}{M_{\text{eff}}}} \sim \frac{1}{\delta}, \quad
E_0 = \frac{\omega}{2} = \frac{1}{2\delta} = m_p
\quad\Rightarrow\quad \delta = \frac{1}{2m_p}
\]
\textbf{Open:} Exact $c_\delta$ requires $V''(0)$ from bulk EOM.

\vspace{0.5em}
\textbf{Explicit assumptions:}
\begin{itemize}[nosep]
    \item Thick brane structure with localized potential \tagP{}
    \item Proton identified as lightest bound fermionic mode \tagP{}
    \item Single-scale dominance (no hierarchy of widths) \tagDc{}
    \item Harmonic approximation for coefficient \tagDc{}
\end{itemize}

\vspace{0.5em}
\textbf{What this achieves:}
\begin{itemize}[nosep]
    \item \textbf{Scaling $\delta \propto 1/m_p$}: Derived (dimensional) \checkmark
    \item \textbf{Coefficient $c_\delta = 1/2$}: From harmonic approx (model-dependent) \checkmark
    \item Connects brane geometry to hadron physics \checkmark
\end{itemize}

\vspace{0.5em}
\textbf{What remains open:}
\begin{itemize}[nosep]
    \item Potential shape $V(\chi)$ from bulk EOM (currently ansatz) $\times$
    \item Exact $c_\delta$: requires $V''(0)$ — why $1/2$ vs $1/\pi$? $\times$
    \item Proton mass itself (input, not output) $\times$
    \item Topological justification for proton = bound mode $\times$
\end{itemize}

\vspace{0.5em}
\begin{center}
\fbox{\parbox{0.85\linewidth}{\centering
\textbf{Status: YELLOW [Dc]} --- Principled derivation, model-dependent coefficient\\[0.3em]
\textbf{Upgrade path:} Derive $V(\chi)$ from 5D action; compute exact bound state; justify proton ID from topology
}}
\end{center}

\vspace{0.3em}
\textit{Full derivation:} \texttt{edc\_papers/\_shared/derivations/delta\_from\_5d\_action\_proton\_scale.tex}

\end{tcolorbox}

% ═══════════════════════════════════════════════════════════════════════════════
% END OF BOX
% ═══════════════════════════════════════════════════════════════════════════════

%
% Dependencies: tcolorbox, amsmath, amssymb
%
% ═══════════════════════════════════════════════════════════════════════════════

\begin{tcolorbox}[breakable, enhanced, colback=yellow!5!white, colframe=yellow!60!black,
    title={\textbf{Brane Thickness from 5D Action: $\delta = \hbar/(2m_p c)$}}, width=\linewidth]

\textbf{Result:}
\begin{equation}
\boxed{\delta = \frac{\hbar}{2 m_p c} = 0.533\,\text{GeV}^{-1} = 0.105\,\text{fm}}
\end{equation}
\textit{The brane thickness equals half the proton Compton wavelength.}

\vspace{0.5em}
\textbf{Derivation Chain} \tagDc{}:
\begin{enumerate}[nosep]
    \item \textbf{5D action $\to$ mode equation} \tagDer{}: KK reduction gives
    \[
    -\frac{d^2 w}{d\chi^2} + V(\chi)\,w = \lambda\,w
    \]
    \item \textbf{Potential scaling} \tagDc{}: Well depth $V_0 \sim 1/\delta^2$, width $\sim \delta$
    \item \textbf{Ground state energy}: $E_0 = c_E/\delta$ where $c_E = O(1)$
    \begin{itemize}[nosep]
        \item \textbf{Scaling} \tagDer{}: $E_0 \propto 1/\delta$ is dimensional/robust
        \item \textbf{Coefficient} \tagDc{}: $c_E$ depends on $\omega = \sqrt{V''/M_{\text{eff}}}$
    \end{itemize}
    \item \textbf{Proton identification} \tagP{}: Postulate: proton = localized fermionic mode
    \item \textbf{Energy matching} \tagDc{}: $m_p = E_0 \Rightarrow \delta = c_\delta/m_p$
\end{enumerate}

\vspace{0.5em}
\textbf{Coefficient origin:} The factor $c_\delta = 1/2$ arises from harmonic approximation:
\[
\omega = \sqrt{\frac{V''(0)}{M_{\text{eff}}}} \sim \frac{1}{\delta}, \quad
E_0 = \frac{\omega}{2} = \frac{1}{2\delta} = m_p
\quad\Rightarrow\quad \delta = \frac{1}{2m_p}
\]
\textbf{Open:} Exact $c_\delta$ requires $V''(0)$ from bulk EOM.

\vspace{0.5em}
\textbf{Explicit assumptions:}
\begin{itemize}[nosep]
    \item Thick brane structure with localized potential \tagP{}
    \item Proton identified as lightest bound fermionic mode \tagP{}
    \item Single-scale dominance (no hierarchy of widths) \tagDc{}
    \item Harmonic approximation for coefficient \tagDc{}
\end{itemize}

\vspace{0.5em}
\textbf{What this achieves:}
\begin{itemize}[nosep]
    \item \textbf{Scaling $\delta \propto 1/m_p$}: Derived (dimensional) \checkmark
    \item \textbf{Coefficient $c_\delta = 1/2$}: From harmonic approx (model-dependent) \checkmark
    \item Connects brane geometry to hadron physics \checkmark
\end{itemize}

\vspace{0.5em}
\textbf{What remains open:}
\begin{itemize}[nosep]
    \item Potential shape $V(\chi)$ from bulk EOM (currently ansatz) $\times$
    \item Exact $c_\delta$: requires $V''(0)$ — why $1/2$ vs $1/\pi$? $\times$
    \item Proton mass itself (input, not output) $\times$
    \item Topological justification for proton = bound mode $\times$
\end{itemize}

\vspace{0.5em}
\begin{center}
\fbox{\parbox{0.85\linewidth}{\centering
\textbf{Status: YELLOW [Dc]} --- Principled derivation, model-dependent coefficient\\[0.3em]
\textbf{Upgrade path:} Derive $V(\chi)$ from 5D action; compute exact bound state; justify proton ID from topology
}}
\end{center}

\vspace{0.3em}
\textit{Full derivation:} \texttt{edc\_papers/\_shared/derivations/delta\_from\_5d\_action\_proton\_scale.tex}

\end{tcolorbox}

% ═══════════════════════════════════════════════════════════════════════════════
% END OF BOX
% ═══════════════════════════════════════════════════════════════════════════════
