% =============================================================================
% Derivation: Fermion Width fw from Stability and Spectrum Constraints
% =============================================================================
% File: fw_from_stability_and_spectrum.tex
% Created: 2026-01-29
% Purpose: Derive/constrain fw (fermion localization width) from principled arguments
%
% COMPANION: docs/FW_FROM_STABILITY_NOTE.md
% GOAL: Explain why fw ≈ 0.8δ or identify what remains [Cal]
% =============================================================================

\documentclass[11pt]{article}
\usepackage{amsmath,amssymb,amsthm}
\usepackage[margin=1in]{geometry}
\usepackage{tcolorbox}
\usepackage{booktabs}
\usepackage{xcolor}

% Note: XeLaTeX compiles without font warnings for Greek letters in math mode

\newtheorem{theorem}{Theorem}[section]
\newtheorem{lemma}[theorem]{Lemma}
\newtheorem{proposition}[theorem]{Proposition}
\newtheorem{corollary}[theorem]{Corollary}
\newtheorem{postulate}[theorem]{Postulate}
\theoremstyle{definition}
\newtheorem{definition}[theorem]{Definition}

% Epistemic tags
\newcommand{\tagDer}{\textcolor{blue}{\textbf{[Der]}}}
\newcommand{\tagDc}{\textcolor{green!60!black}{\textbf{[Dc]}}}
\newcommand{\tagP}{\textcolor{orange}{\textbf{[P]}}}
\newcommand{\tagI}{\textcolor{purple}{\textbf{[I]}}}
\newcommand{\tagBL}{\textcolor{gray}{\textbf{[BL]}}}
\newcommand{\tagCal}{\textcolor{red}{\textbf{[Cal]}}}

\title{Fermion Localization Width from Stability and Spectrum\\[0.5em]
\large Constraining $f_w$ from First Principles}
\author{EDC Project}
\date{2026-01-29}

\begin{document}
\maketitle

\begin{abstract}
We analyze the fermion localization width parameter $f_w$ (in units of
brane thickness $\delta$) from stability and spectral constraints.
We derive a physical window $f_w \in [0.5, 1.2]$ from normalizability,
ground-state dominance, and variational bounds. The BVP-tuned value
$f_w \approx 0.8$ lies comfortably inside this window, supporting the
physical interpretation that $f_w \sim O(1)$ is natural. However, the
\emph{specific} value 0.8 (vs 0.7 or 0.9) remains \tagCal{} --- determined
by the $I_4$ gate fit, not derived from first principles.
\end{abstract}

\tableofcontents
\newpage

%=============================================================================
\section{Definition of Fermion Width}
%=============================================================================

\subsection{In the BVP Framework}

\begin{definition}[Fermion Width Parameter $f_w$] \tagDer
\label{def:fw}
The fermion width parameter $f_w$ is defined as:
\begin{equation}
f_w = \frac{\sigma_\psi}{\delta}
\label{eq:fw_def}
\end{equation}
where $\sigma_\psi$ is the characteristic width of the fermion mode profile
$w(\chi)$ in the extra dimension, and $\delta$ is the brane thickness.
\end{definition}

Several equivalent definitions of $\sigma_\psi$ are possible:

\begin{enumerate}
\item \textbf{Second moment (variance):}
\begin{equation}
\sigma_\psi^2 = \frac{\int_{-\infty}^{\infty} \chi^2 |w(\chi)|^2\, d\chi}
                     {\int_{-\infty}^{\infty} |w(\chi)|^2\, d\chi}
- \left( \frac{\int \chi |w|^2\, d\chi}{\int |w|^2\, d\chi} \right)^2
\label{eq:sigma_second_moment}
\end{equation}

\item \textbf{FWHM (Full Width at Half Maximum):}
\begin{equation}
\sigma_\psi \approx \frac{\mathrm{FWHM}}{2\sqrt{2\ln 2}} \approx 0.425 \times \mathrm{FWHM}
\label{eq:sigma_fwhm}
\end{equation}

\item \textbf{Exponential decay length:} For $w(\chi) \propto e^{-|\chi|/\sigma}$:
\begin{equation}
\sigma_\psi = \sigma
\label{eq:sigma_exp}
\end{equation}
\end{enumerate}

In the BVP code, $f_w$ parameterizes the domain-wall mass profile:
\begin{equation}
m(\chi) = m_0 \tanh\left(\frac{\chi - \chi_c}{f_w \delta}\right)
\label{eq:mass_profile}
\end{equation}

%=============================================================================
\section{Constraint 1: Normalizability}
%=============================================================================

\subsection{Bound-State Existence}

\begin{proposition}[Normalizability Bound] \tagDer
\label{prop:normalizable}
For a fermion zero mode to be normalizable on a domain wall:
\begin{equation}
\int_{-\infty}^{\infty} |w(\chi)|^2\, d\chi < \infty
\label{eq:norm_condition}
\end{equation}
This requires the mode to decay faster than $|\chi|^{-1/2}$ at large $|\chi|$.
\end{proposition}

For the domain-wall mass profile \eqref{eq:mass_profile}, the zero mode is:
\begin{equation}
w_L(\chi) \propto \exp\left(-\int_0^\chi m(\chi')\, d\chi'\right)
= \left[\cosh\left(\frac{\chi - \chi_c}{f_w \delta}\right)\right]^{-m_0 f_w \delta}
\label{eq:wL_solution}
\end{equation}

\begin{corollary}[Width Constraint from Normalizability] \tagDer
\label{cor:norm_constraint}
For \eqref{eq:wL_solution} to be normalizable:
\begin{equation}
m_0 f_w \delta > \frac{1}{2}
\label{eq:norm_inequality}
\end{equation}
With $m_0 \sim 1/\delta$ (from dimensional analysis):
\begin{equation}
f_w > \frac{1}{2}
\label{eq:fw_lower}
\end{equation}
\end{corollary}

This gives our first constraint: $\boxed{f_w > 0.5}$.

%=============================================================================
\section{Constraint 2: Confinement to Brane}
%=============================================================================

\subsection{Physical Requirement}

\begin{proposition}[Confinement Bound] \tagDc
\label{prop:confinement}
For the fermion to be effectively 4-dimensional (not leaking into the bulk),
its localization width should not exceed the brane extent:
\begin{equation}
\sigma_\psi \lesssim L_{\mathrm{brane}}
\label{eq:confinement}
\end{equation}
With $L_{\mathrm{brane}} \sim \text{few} \times \delta$, this gives:
\begin{equation}
f_w \lesssim 2\text{--}3
\label{eq:fw_upper_conf}
\end{equation}
\end{proposition}

A stricter bound comes from requiring the mode to be well-localized:

\begin{corollary}[Strict Localization] \tagDc
\label{cor:strict_loc}
For the fermion to be localized within $\pm 2\delta$ of the wall center
with 95\% probability:
\begin{equation}
\int_{-2\delta}^{+2\delta} |w|^2\, d\chi > 0.95
\quad \Rightarrow \quad
f_w \lesssim 1.2
\label{eq:fw_strict}
\end{equation}
\end{corollary}

This gives: $\boxed{f_w < 1.2}$ for tight confinement.

%=============================================================================
\section{Constraint 3: Ground-State Dominance}
%=============================================================================

\subsection{Spectral Gap}

\begin{proposition}[Spectral Gap Requirement] \tagDc
\label{prop:gap}
For the ground-state mode to dominate (excited modes suppressed), we need
a sufficient gap between eigenvalues:
\begin{equation}
\lambda_1 - \lambda_0 \gg T_{\mathrm{eff}}
\label{eq:gap_condition}
\end{equation}
where $T_{\mathrm{eff}}$ is an effective temperature scale.
\end{proposition}

For a potential well of width $f_w \delta$ and depth $V_0 \sim 1/\delta^2$:
\begin{equation}
\lambda_n \sim \frac{n^2}{(f_w \delta)^2}
\quad \Rightarrow \quad
\Delta \lambda = \lambda_1 - \lambda_0 \sim \frac{3}{(f_w \delta)^2}
\label{eq:gap_estimate}
\end{equation}

\begin{corollary}[Gap-Based Constraint] \tagDc
\label{cor:gap_constraint}
Requiring $\Delta \lambda > 1/\delta^2$ (gap comparable to ground state energy):
\begin{equation}
\frac{3}{(f_w)^2} > 1 \quad \Rightarrow \quad f_w < \sqrt{3} \approx 1.7
\label{eq:fw_from_gap}
\end{equation}
\end{corollary}

This is weaker than the localization bound but confirms $f_w \sim O(1)$.

%=============================================================================
\section{Constraint 4: Variational Principle}
%=============================================================================

\subsection{Energy Functional}

\begin{proposition}[Variational Lower Bound] \tagDer
\label{prop:variational}
The ground-state energy satisfies:
\begin{equation}
E_0 = \min_{w} \frac{\langle w | H | w \rangle}{\langle w | w \rangle}
\label{eq:variational}
\end{equation}
For a trial function $w(\chi) \propto \exp(-|\chi|/(f_w \delta))$, the
energy functional is:
\begin{equation}
E[f_w] = \frac{1}{(f_w \delta)^2} + V_{\mathrm{eff}}
\label{eq:E_functional}
\end{equation}
where the first term is kinetic and the second is potential energy.
\end{proposition}

\subsection{Optimal Width}

\begin{proposition}[Variational Optimum] \tagDc
\label{prop:optimal_fw}
Minimizing $E[f_w]$ with respect to $f_w$ for a harmonic-like potential:
\begin{equation}
V(\chi) \approx \frac{1}{2} \omega^2 \chi^2, \quad \omega \sim \frac{1}{\delta}
\end{equation}
gives the optimal width:
\begin{equation}
\sigma_{\mathrm{opt}} = \sqrt{\frac{\hbar}{m_{\mathrm{eff}} \omega}}
\label{eq:sigma_opt}
\end{equation}
With $m_{\mathrm{eff}} \sim 1/\delta$ and $\omega \sim 1/\delta$:
\begin{equation}
\sigma_{\mathrm{opt}} \sim \delta \quad \Rightarrow \quad f_w^{\mathrm{opt}} \sim 1
\label{eq:fw_opt}
\end{equation}
\end{proposition}

The variational principle predicts $f_w \sim O(1)$, consistent with the tuned
value 0.8.

%=============================================================================
\section{Combined Constraints}
%=============================================================================

\subsection{The Allowed Window}

Combining all constraints:

\begin{center}
\renewcommand{\arraystretch}{1.3}
\begin{tabular}{|l|c|c|p{5cm}|}
\hline
\textbf{Constraint} & \textbf{Bound} & \textbf{Status} & \textbf{Physical Origin} \\
\hline
\hline
Normalizability & $f_w > 0.5$ & \tagDer & Zero mode must be square-integrable \\
\hline
Strict localization & $f_w < 1.2$ & \tagDc & 95\% within $\pm 2\delta$ \\
\hline
Confinement & $f_w < 2$ & \tagDc & Mode stays within brane \\
\hline
Spectral gap & $f_w < 1.7$ & \tagDc & Ground state dominates \\
\hline
Variational & $f_w \sim 1$ & \tagDc & Energy minimization \\
\hline
\end{tabular}
\end{center}

\begin{tcolorbox}[colback=green!10!white, colframe=green!60!black,
title=\textbf{Derived Window}]
\begin{equation}
\boxed{f_w \in [0.5, 1.2]}
\label{eq:fw_window}
\end{equation}
The tuned value $f_w = 0.8$ is \textbf{inside} this window.
\end{tcolorbox}

\subsection{Why Larger $f_w$ Affects Overlap}

\begin{proposition}[Overlap Sensitivity] \tagDer
\label{prop:overlap_sensitivity}
The overlap integral $I_4 = \int w_L^2 w_R^2 w_\phi^2\, d\chi$ scales as:
\begin{equation}
I_4 \propto (f_w)^p \times \exp\left(-\frac{d_{\mathrm{LR}}^2}{2\sigma^2}\right)
\label{eq:I4_scaling}
\end{equation}
where $p > 0$ (polynomial prefactor) and the exponential depends on $d_{\mathrm{LR}}$.

Larger $f_w$ increases the polynomial prefactor but does NOT affect the
exponential suppression (which depends on $d_{\mathrm{LR}}/\sigma$).
\end{proposition}

From the sensitivity analysis (OPR-21c):
\begin{itemize}
\item $f_w$ elasticity: $+1.3$ (positive, polynomial)
\item $d_{\mathrm{LR}}$ elasticity: $-6.5$ (negative, exponential)
\end{itemize}

The value $f_w = 0.8$ is where the polynomial prefactor is large enough
to give $I_4 \sim 10^{-3}$ GeV while staying within the physical window.

%=============================================================================
\section{What Is NOT Derived}
%=============================================================================

\subsection{Why Specifically 0.8?}

The constraints give a \emph{window} $[0.5, 1.2]$, not a unique value.
The specific value $f_w = 0.8$ comes from:

\begin{enumerate}
\item \textbf{BVP parameter scan:} Scanning over the window to find where
$X_{\mathrm{EDC}} \approx X_{\mathrm{target}}$.

\item \textbf{$I_4$ gate matching:} The value 0.8 gives the right overlap
suppression combined with $d_{\mathrm{LR}} = 8\delta$.
\end{enumerate}

\textbf{Status:} The value $f_w = 0.8$ is \tagCal{} (calibrated), not derived.

\subsection{What Would Upgrade to [Der]}

To derive the specific value, we would need:
\begin{enumerate}
\item The exact potential shape $V(\chi)$ from the 5D action
\item The exact fermion mass profile $m(\chi)$ from bulk Yukawa structure
\item Solution of the BVP eigenvalue problem with these exact potentials
\end{enumerate}

Currently, these are modeling choices (Gaussian, tanh, etc.), not derived.

%=============================================================================
\section{Stoplight Verdict}
%=============================================================================

\begin{tcolorbox}[colback=yellow!10!white, colframe=yellow!70!black,
title=\textbf{Epistemic Status: YELLOW [Dc/Cal]}]

\textbf{What is derived [Der]:}
\begin{itemize}
\item Normalizability requires $f_w > 0.5$
\item Overlap scales polynomially with $f_w$
\item Second-moment definition of mode width
\end{itemize}

\textbf{What is derived conditional [Dc]:}
\begin{itemize}
\item Strict localization gives $f_w < 1.2$
\item Variational principle predicts $f_w \sim 1$
\item Physical window $f_w \in [0.5, 1.2]$
\end{itemize}

\textbf{What is calibrated [Cal]:}
\begin{itemize}
\item Specific value $f_w = 0.8$ from BVP scan
\item This value passes $I_4$ gate with $d_{\mathrm{LR}} = 8\delta$
\end{itemize}

\textbf{Upgrade path YELLOW $\to$ GREEN:}
\begin{enumerate}
\item Derive $V(\chi)$ and $m(\chi)$ from 5D bulk action
\item Solve exact BVP to get ground-state width
\item Show $\sigma_\psi/\delta \approx 0.8$ emerges from first principles
\end{enumerate}

\end{tcolorbox}

%=============================================================================
\section{Conclusion}
%=============================================================================

We have established:

\begin{enumerate}
\item \textbf{Physical window:} $f_w \in [0.5, 1.2]$ from stability and
localization constraints.

\item \textbf{Tuned value inside window:} $f_w = 0.8$ lies comfortably
within this window, supporting its physical plausibility.

\item \textbf{Not arbitrary:} The value is constrained by physics, not
randomly chosen. But the specific value comes from the $I_4$ fit.

\item \textbf{Polynomial sensitivity:} $f_w$ affects overlap polynomially
(elasticity $+1.3$), unlike $d_{\mathrm{LR}}$ which controls exponentially.
\end{enumerate}

\textbf{Defendable claim:} The value $f_w \approx 0.8$ is \textbf{physically
natural} (within derived bounds) but not \textbf{uniquely derived} (still
calibrated within the window).

\end{document}
