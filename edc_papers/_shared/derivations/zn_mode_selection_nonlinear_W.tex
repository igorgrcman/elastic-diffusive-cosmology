% Z_N Mode Selection Robustness Under Non-Quadratic W(u)
% File: edc_papers/_shared/derivations/zn_mode_selection_nonlinear_W.tex
% Created: 2026-01-29
% Purpose: Prove mode selection (m=N) is robust when W(u) is not purely quadratic
%
% EPISTEMIC STATUS:
%   [Der] - Second variation theorem for mode index selection
%   [Dc]  - Amplitude corrections from higher-order terms
%
% PREREQUISITE: edc_papers/_shared/derivations/zn_ring_delta_pinning_modes.tex
% COMPANION: docs/ZN_NONQUADRATIC_W_ROBUSTNESS_NOTE.md

\documentclass[11pt]{article}
\usepackage{amsmath,amssymb,amsthm}
\usepackage[margin=1in]{geometry}
\usepackage{tcolorbox}
\usepackage{booktabs}
\usepackage{array}

\newtheorem{theorem}{Theorem}[section]
\newtheorem{lemma}[theorem]{Lemma}
\newtheorem{proposition}[theorem]{Proposition}
\newtheorem{corollary}[theorem]{Corollary}
\theoremstyle{definition}
\newtheorem{definition}[theorem]{Definition}

% Epistemic tags
\newcommand{\tagDer}{\textcolor{blue}{\textbf{[Der]}}}
\newcommand{\tagDc}{\textcolor{green!60!black}{\textbf{[Dc]}}}
\newcommand{\tagP}{\textcolor{orange}{\textbf{[P]}}}

\title{Z$_N$ Mode Selection Robustness\\[0.5em]
\large Under Non-Quadratic Anchor Potential $W(u)$}
\author{EDC Project}
\date{2026-01-29}

\begin{document}
\maketitle

\begin{abstract}
We prove that the mode selection result---$\cos(N\theta)$ is the leading
anisotropic mode under Z$_N$ delta-pinning---is \textbf{robust} when the
anchor potential $W(u)$ is not purely quadratic. The key insight is that
mode \textit{index} selection is determined by the Hessian (second variation)
of the energy functional, which depends only on $W''(u_0)$. Nonlinear terms
$W'''$, $W''''$, etc., generate higher harmonics and amplitude corrections
but do not change the leading mode index near equilibrium.
\end{abstract}

\tableofcontents
\newpage

%============================================================
\section{Setup: General Smooth Potential}
%============================================================

\subsection{Energy Functional with General $W$}

\begin{definition}[General Anchor Potential] \tagP
\label{def:general_W}
Let $W: \mathbb{R} \to \mathbb{R}$ be a $C^4$ function with a stable minimum at $u_0$:
\begin{align}
W'(u_0) &= 0 \quad \text{(equilibrium)} \\
W''(u_0) &= \kappa > 0 \quad \text{(stability)}
\end{align}
\end{definition}

\begin{definition}[Full Energy Functional] \tagP
\label{def:E_full}
\begin{equation}
E[u] = \frac{T}{2} \int_0^{2\pi} (u')^2 \, d\theta + \lambda \sum_{n=0}^{N-1} W(u(\theta_n))
\label{eq:E_full}
\end{equation}
where $\theta_n = 2\pi n / N$ are the Z$_N$ fixed points.
\end{definition}

\subsection{Taylor Expansion of $W$}

\begin{proposition}[Potential Expansion] \tagDer
\label{prop:W_taylor}
Setting $\eta = u - u_0$ (perturbation around equilibrium):
\begin{equation}
W(u_0 + \eta) = W_0 + \frac{\kappa}{2} \eta^2 + \frac{g}{6} \eta^3 + \frac{h}{24} \eta^4 + O(\eta^5)
\label{eq:W_taylor}
\end{equation}
where:
\begin{align}
W_0 &= W(u_0) \quad \text{(constant, irrelevant for dynamics)} \\
\kappa &= W''(u_0) > 0 \quad \text{(quadratic stiffness)} \\
g &= W'''(u_0) \quad \text{(cubic coupling)} \\
h &= W''''(u_0) \quad \text{(quartic coupling)}
\end{align}
\end{proposition}

%============================================================
\section{Second Variation and Mode Selection}
%============================================================

\subsection{First and Second Variations}

\begin{theorem}[Second Variation Controls Mode Selection] \tagDer
\label{thm:second_variation}
Let $u = u_0 + \eta$ with $\eta$ small. The energy expands as:
\begin{equation}
E[u_0 + \eta] = E_0 + \delta E[\eta] + \frac{1}{2}\delta^2 E[\eta, \eta] + O(\eta^3)
\label{eq:E_expansion}
\end{equation}

\textbf{First variation:}
\begin{equation}
\delta E[\eta] = \lambda \sum_{n=0}^{N-1} W'(u_0) \cdot \eta(\theta_n) = 0
\label{eq:first_var}
\end{equation}
(vanishes at equilibrium since $W'(u_0) = 0$).

\textbf{Second variation (Hessian):}
\begin{equation}
\boxed{\delta^2 E[\eta, \eta] = T \int_0^{2\pi} (\eta')^2 \, d\theta
+ \lambda \kappa \sum_{n=0}^{N-1} \eta(\theta_n)^2}
\label{eq:second_var}
\end{equation}
\end{theorem}

\begin{proof}
From \eqref{eq:E_full} with $u = u_0 + \eta$:
\begin{align}
E[u_0 + \eta] &= \frac{T}{2} \int (\eta')^2 \, d\theta + \lambda \sum_n W(u_0 + \eta_n) \\
&= \frac{T}{2} \int (\eta')^2 \, d\theta + \lambda \sum_n \left[ W_0 + \frac{\kappa}{2}\eta_n^2
+ \frac{g}{6}\eta_n^3 + \ldots \right]
\end{align}
The quadratic part is exactly $\frac{1}{2}\delta^2 E$. Higher powers contribute to $O(\eta^3)$ and beyond.
\end{proof}

\subsection{The Central Observation}

\begin{tcolorbox}[colback=blue!5!white, colframe=blue!50!black,
                  title=\textbf{Key Insight: Mode Index Selection is Linear}]
The second variation \eqref{eq:second_var} has \textbf{exactly the same form}
as the quadratic energy functional analyzed in the previous derivation:
\begin{equation}
\delta^2 E[\eta, \eta] = T \int (\eta')^2 \, d\theta + \lambda\kappa \sum_n \eta_n^2
\end{equation}

\textbf{This depends only on $\kappa = W''(u_0)$, not on $g = W'''(u_0)$, $h = W''''(u_0)$, etc.}

Therefore:
\begin{itemize}
\item The eigenmode structure of $\delta^2 E$ is identical to the quadratic case
\item The Selection Lemma still holds: only $m = kN$ modes couple to anchors
\item The gradient ordering still holds: $m = N$ has lowest gradient energy among coupled modes
\item \textbf{The leading anisotropic mode remains $\cos(N\theta)$}
\end{itemize}
\end{tcolorbox}

\begin{theorem}[Mode Index Robustness] \tagDer
\label{thm:robustness}
For any $C^2$ potential $W$ with stable minimum at $u_0$ ($W'(u_0) = 0$, $W''(u_0) > 0$),
the leading anisotropic eigenmode of the Hessian $\delta^2 E$ is $\cos(N\theta)$.

The mode \textit{index} $m = N$ is determined by:
\begin{enumerate}
\item Z$_N$ symmetry of anchor positions (Selection Lemma)
\item Gradient energy ordering ($E_{\text{grad}} \propto m^2$)
\end{enumerate}
Neither depends on $W'''$, $W''''$, or any higher derivatives.
\end{theorem}

%============================================================
\section{Nonlinear Corrections: Amplitude and Harmonics}
%============================================================

\subsection{Leading-Order Solution}

\begin{proposition}[Linear Solution] \tagDer
\label{prop:linear_sol}
At leading order, the perturbation takes the form:
\begin{equation}
\eta^{(1)}(\theta) = A \cos(N\theta)
\label{eq:eta_1}
\end{equation}
where $A$ is the amplitude, determined by the source/forcing.
\end{proposition}

\subsection{Cubic Correction}

\begin{proposition}[First Nonlinear Correction] \tagDc
\label{prop:cubic}
The cubic term in $W$ generates corrections at order $A^2$:
\begin{equation}
W_{\text{cubic}} = \frac{g}{6} \eta^3 = \frac{g}{6} A^3 \cos^3(N\theta)
\end{equation}

Using $\cos^3 x = \frac{3}{4}\cos x + \frac{1}{4}\cos 3x$:
\begin{equation}
\cos^3(N\theta) = \frac{3}{4}\cos(N\theta) + \frac{1}{4}\cos(3N\theta)
\label{eq:cos3_expansion}
\end{equation}

\textbf{Result:} The cubic term:
\begin{itemize}
\item Modifies the coefficient of $\cos(N\theta)$ (amplitude shift)
\item Generates a new harmonic $\cos(3N\theta)$ at order $A^3$
\end{itemize}
\end{proposition}

\subsection{Quartic Correction}

\begin{proposition}[Second Nonlinear Correction] \tagDc
\label{prop:quartic}
The quartic term generates corrections at order $A^3$:
\begin{equation}
W_{\text{quartic}} = \frac{h}{24} \eta^4 = \frac{h}{24} A^4 \cos^4(N\theta)
\end{equation}

Using $\cos^4 x = \frac{3}{8} + \frac{1}{2}\cos 2x + \frac{1}{8}\cos 4x$:
\begin{equation}
\cos^4(N\theta) = \frac{3}{8} + \frac{1}{2}\cos(2N\theta) + \frac{1}{8}\cos(4N\theta)
\label{eq:cos4_expansion}
\end{equation}

\textbf{Result:} The quartic term:
\begin{itemize}
\item Shifts the mean value (constant term)
\item Generates harmonics $\cos(2N\theta)$ and $\cos(4N\theta)$ at order $A^4$
\end{itemize}
\end{proposition}

\subsection{Harmonic Content Table}

\begin{tcolorbox}[colback=yellow!10!white, colframe=orange!70!black,
                  title=\textbf{Harmonic Content vs Amplitude Order}]

\renewcommand{\arraystretch}{1.4}
\begin{tabular}{|c|c|c|l|}
\hline
\textbf{Order} & \textbf{Source} & \textbf{Harmonics Generated} & \textbf{Amplitude} \\
\hline
\hline
$O(A)$ & Linear ($\kappa$) & $\cos(N\theta)$ & $A$ \\
\hline
$O(A^2)$ & --- & (none at this order) & --- \\
\hline
$O(A^3)$ & Cubic ($g$) & $\cos(N\theta)$, $\cos(3N\theta)$ & $\sim gA^3/\kappa$ \\
\hline
$O(A^4)$ & Quartic ($h$) & const, $\cos(2N\theta)$, $\cos(4N\theta)$ & $\sim hA^4/\kappa$ \\
\hline
\end{tabular}

\vspace{0.5em}
\textbf{Key observation:} All generated harmonics are multiples of $N$: $(N, 2N, 3N, 4N, \ldots)$.

No harmonics with $m < N$ are generated by nonlinear terms!
\end{tcolorbox}

%============================================================
\section{Regime of Validity}
%============================================================

\subsection{Smallness Conditions}

\begin{definition}[Nonlinearity Parameters] \tagDer
\label{def:eps_nl}
Define dimensionless nonlinearity measures:
\begin{align}
\varepsilon_3 &= \frac{|g| A}{\kappa} \quad \text{(cubic strength)} \\
\varepsilon_4 &= \frac{|h| A^2}{\kappa} \quad \text{(quartic strength)}
\end{align}
\end{definition}

\begin{proposition}[Perturbative Regime] \tagDer
\label{prop:regime}
The linear mode selection result holds when:
\begin{equation}
\boxed{\varepsilon_3 \ll 1 \quad \text{and} \quad \varepsilon_4 \ll 1}
\label{eq:regime}
\end{equation}

Equivalently, the amplitude must satisfy:
\begin{equation}
|A| \ll \min\left( \frac{\kappa}{|g|}, \sqrt{\frac{\kappa}{|h|}} \right)
\label{eq:A_bound}
\end{equation}
\end{proposition}

\subsection{Physical Interpretation}

\begin{proposition}[Scale of Validity] \tagDc
\label{prop:scale}
The perturbative regime corresponds to:
\begin{equation}
|u - u_0| \ll L_W
\end{equation}
where $L_W$ is the scale over which $W(u)$ deviates significantly from quadratic:
\begin{equation}
L_W \sim \min\left( \frac{\kappa}{|g|}, \sqrt{\frac{\kappa}{|h|}} \right)
\end{equation}

For a ``typical'' smooth potential, $L_W$ is comparable to the width of the potential well.
\end{proposition}

%============================================================
\section{The Robustness Theorem}
%============================================================

\begin{tcolorbox}[colback=green!10!white, colframe=green!60!black,
                  title=\textbf{Robustness Theorem: Mode Index Selection Under General $W$}]

\begin{theorem}[Mode Selection Robustness] \tagDer
\label{thm:main}
Let $W: \mathbb{R} \to \mathbb{R}$ be a $C^2$ function with:
\begin{enumerate}
\item A stable equilibrium at $u_0$: $W'(u_0) = 0$, $W''(u_0) = \kappa > 0$
\item Identical anchors at Z$_N$ fixed points $\theta_n = 2\pi n/N$
\end{enumerate}

Then for sufficiently small amplitude $|A| \ll L_W$:
\begin{equation}
\boxed{\text{The leading anisotropic mode is } \cos(N\theta)}
\end{equation}

\textbf{Nonlinear effects:}
\begin{itemize}
\item Modify the amplitude relationship (energy vs $A$)
\item Generate higher harmonics $(2N\theta, 3N\theta, \ldots)$ at higher orders
\item Do NOT change the mode index $m = N$
\item Do NOT introduce harmonics with $m < N$
\end{itemize}
\end{theorem}

\vspace{0.5em}
\textbf{Failure modes (when theorem does not apply):}
\begin{enumerate}
\item \textbf{Non-smooth $W$:} If $W$ is not $C^2$, Hessian may not exist
\item \textbf{Metastability:} If $W''(u_0) \leq 0$, equilibrium is unstable
\item \textbf{Large amplitude:} If $|A| \gtrsim L_W$, perturbation theory fails
\item \textbf{Symmetry breaking:} If anchors are not identical or not at Z$_N$ positions
\item \textbf{Multiple minima:} If system jumps between different equilibria
\end{enumerate}
\end{tcolorbox}

%============================================================
\section{Summary: What Changes vs What Doesn't}
%============================================================

\renewcommand{\arraystretch}{1.3}
\begin{tabular}{|l|c|c|}
\hline
\textbf{Property} & \textbf{Quadratic $W$} & \textbf{General $W$} \\
\hline
\hline
Mode index ($m = N$) & Fixed & \textbf{Unchanged} \\
\hline
Selection Lemma (coupling) & Exact & \textbf{Unchanged} \\
\hline
Gradient ordering & Exact & \textbf{Unchanged} \\
\hline
Amplitude relation & Linear in source & Nonlinear corrections \\
\hline
Harmonic content & Pure $\cos(N\theta)$ & $\cos(N\theta)$ + higher $(2N, 3N, \ldots)$ \\
\hline
Shape near equilibrium & Pure cosine & Slightly distorted \\
\hline
\end{tabular}

\vspace{1em}

\textbf{Bottom line:}
\begin{quote}
\textit{Mode index selection is a \textbf{linear} property, determined by the Hessian.
Nonlinearities modify amplitude and shape but do not change the leading harmonic
near equilibrium.}
\end{quote}

%============================================================
\section{Epistemic Status}
%============================================================

\renewcommand{\arraystretch}{1.3}
\begin{tabular}{|l|l|p{6cm}|}
\hline
\textbf{Result} & \textbf{Status} & \textbf{Comment} \\
\hline
\hline
Second variation formula & \tagDer & Standard calculus of variations \\
\hline
Hessian depends only on $W''$ & \tagDer & Direct calculation \\
\hline
Mode index robustness theorem & \tagDer & Follows from Hessian structure \\
\hline
Harmonic generation table & \tagDc & Perturbation expansion; exact coefficients not computed \\
\hline
Regime of validity bounds & \tagDer & Dimensional analysis \\
\hline
\end{tabular}

\vspace{1em}

The central result---mode index $m = N$ is robust under non-quadratic $W$---is \textbf{fully derived} [Der].

\end{document}
