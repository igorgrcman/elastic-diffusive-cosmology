% 5D Action to Toy Functional Mapping for Z_N Anisotropy
% File: edc_papers/_shared/derivations/zn_toy_functional_from_5d_action.tex
% Created: 2026-01-29
% Purpose: Map 5D brane-world action to toy functional E[u] used in Z_N normalization
%
% COMPANION: docs/ZN_5D_TO_TOY_MAPPING_NOTE.md
% PREREQUISITE: edc_papers/_shared/derivations/zn_anisotropy_normalization_from_action.tex
% REFERENCE: edc_papers/paper_3_series/04_companion_C_5d_reduction/paper/main.tex

\documentclass[11pt]{article}
\usepackage{amsmath,amssymb,amsthm}
\usepackage[margin=1in]{geometry}
\usepackage{tcolorbox}
\usepackage{booktabs}
\usepackage{array}

\newtheorem{theorem}{Theorem}[section]
\newtheorem{lemma}[theorem]{Lemma}
\newtheorem{proposition}[theorem]{Proposition}
\newtheorem{postulate}[theorem]{Postulate}
\theoremstyle{definition}
\newtheorem{definition}[theorem]{Definition}

% Epistemic tags
\newcommand{\tagDer}{\textcolor{blue}{\textbf{[Der]}}}
\newcommand{\tagDc}{\textcolor{green!60!black}{\textbf{[Dc]}}}
\newcommand{\tagP}{\textcolor{orange}{\textbf{[P]}}}

\title{5D Action to Toy Functional Mapping\\[0.5em]
\large For Z$_N$ Anisotropy Normalization}
\author{EDC Project}
\date{2026-01-29}

\begin{document}
\maketitle

\begin{abstract}
We establish the explicit mapping from the 5D brane-world action
$S_{5D} = S_{\text{bulk}} + S_{\text{brane}} + S_{\text{GHY}}$ to the toy functional
$E[u] = (T/2)\int(u')^2 d\theta + \lambda \sum_n W(u(\theta_n))$
used in the Z$_N$ anisotropy normalization derivation.
The mapping identifies the tension parameter $T$, anchor coupling $\lambda$,
anchor potential $W$, and angular coordinate $\theta$.
All steps are tagged with epistemic status.
\end{abstract}

\tableofcontents
\newpage

%============================================================
\section{Overview: The Mapping Chain}
%============================================================

\subsection{Starting Points}

\textbf{5D Action (established in EDC):}
\begin{equation}
\boxed{S_{5D} = S_{\text{bulk}} + S_{\text{brane}} + S_{\text{GHY}}}
\label{eq:S5D}
\end{equation}

where (following Companion C notation):
\begin{align}
S_{\text{bulk}} &= \frac{1}{2\kappa_5^2} \int d^5x \sqrt{-g} \, (R - 2\Lambda_5) \\
S_{\text{brane}} &= -\sigma \int d^4\xi \sqrt{-h} + S_{\text{matter}} \\
S_{\text{GHY}} &= \frac{1}{\kappa_5^2} \oint d^4\xi \sqrt{-h} \, K
\end{align}

\textbf{Toy Functional (target):}
\begin{equation}
\boxed{E[u] = \frac{T}{2} \int_0^{2\pi} (u')^2 \, d\theta + \lambda \sum_{n=0}^{N-1} W(u(\theta_n))}
\label{eq:E_toy}
\end{equation}

\subsection{The Reduction Chain}

\begin{equation}
\boxed{
\underbrace{S_{5D}}_{\text{5D action}}
\xrightarrow[\text{Z}_N\text{ ansatz}]{\text{reduction}}
\underbrace{S_{4D}[u]}_{\text{4D on ring}}
\xrightarrow[\text{static}]{\text{integration}}
\underbrace{E[u]}_{\text{1D functional}}
}
\end{equation}

%============================================================
\section{Stage 1: Geometry Setup}
%============================================================

\subsection{Z$_N$ Ring Embedding}

\begin{postulate}[Ring Geometry] \tagP
\label{post:ring}
Consider a circular submanifold $\mathcal{R} \subset \Sigma^4$ parameterized by
angular coordinate $\theta \in [0, 2\pi)$.
\end{postulate}

\begin{definition}[Angular Coordinate] \tagP
\label{def:theta}
The coordinate $\theta$ is defined via the transverse embedding such that:
\begin{itemize}
\item The Z$_N$ fixed points are located at $\theta_n = 2\pi n / N$
\item The metric on $\mathcal{R}$ is $ds^2_{\mathcal{R}} = R^2 d\theta^2$
\item $R$ is the characteristic ring radius
\end{itemize}
\end{definition}

\begin{postulate}[Field Variable] \tagP
\label{post:field}
Let $u(\theta)$ be a scalar field on the ring representing:
\begin{itemize}
\item \textbf{Option A:} Brane displacement in the $\xi$-direction (transverse excursion)
\item \textbf{Option B:} Mode envelope of a fermion zero mode
\item \textbf{Option C:} Thickness variation of the membrane
\end{itemize}
We proceed with the general $u(\theta)$ without committing to a specific identification.
\end{postulate}

\subsection{Z$_N$ Symmetry Constraint}

\begin{definition}[Z$_N$ Symmetry] \tagDer
\label{def:ZN}
The field $u(\theta)$ has Z$_N$ symmetry:
\begin{equation}
u\left(\theta + \frac{2\pi}{N}\right) = u(\theta)
\end{equation}
\end{definition}

\begin{proposition}[Fourier Restriction] \tagDer
\label{prop:fourier}
Under Z$_N$ symmetry, only modes with period $2\pi/N$ survive:
\begin{equation}
u(\theta) = u_0 + \sum_{m=1}^{\infty} a_m \cos(mN\theta)
\end{equation}
\end{proposition}

%============================================================
\section{Stage 2: Brane Action Reduction}
%============================================================

\subsection{Brane Tension Contribution}

\begin{postulate}[Brane Action Ansatz] \tagP
\label{post:brane_action}
The brane action on the ring takes the form:
\begin{equation}
S_{\text{brane}}[\text{on }\mathcal{R}] = -\sigma \int d\theta \sqrt{h_{\theta\theta}}
\end{equation}
where $h_{\theta\theta}$ is the induced metric component.
\end{postulate}

\begin{proposition}[Induced Metric with Deformation] \tagDc
\label{prop:induced}
For a brane with transverse displacement $u(\theta)$:
\begin{equation}
h_{\theta\theta} = R^2 + (u')^2
\label{eq:h_theta}
\end{equation}
where $u' = du/d\theta$.
\end{proposition}

\begin{proof}
From the embedding $X^A(\theta) = (\theta, u(\theta))$, the tangent vector is
$e^A = (1, u')$ and the induced metric is
$h_{\theta\theta} = g_{AB} e^A e^B = g_{\theta\theta} + g_{uu} (u')^2 = R^2 + (u')^2$.
\end{proof}

\subsection{Gradient Energy Extraction}

\begin{theorem}[Gradient Energy Term] \tagDc
\label{thm:gradient}
Expanding to quadratic order in $u'$:
\begin{equation}
\sqrt{h_{\theta\theta}} = R \sqrt{1 + (u'/R)^2} \approx R + \frac{(u')^2}{2R}
\end{equation}

The brane action becomes:
\begin{equation}
S_{\text{brane}} \approx -\sigma R \cdot 2\pi - \frac{\sigma}{2R} \int_0^{2\pi} (u')^2 \, d\theta
\end{equation}

Dropping the constant, the \textbf{gradient energy} is:
\begin{equation}
\boxed{E_{\text{grad}} = \frac{\sigma}{2R} \int_0^{2\pi} (u')^2 \, d\theta}
\label{eq:E_grad}
\end{equation}
\end{theorem}

\begin{tcolorbox}[colback=blue!5!white, colframe=blue!50!black,
                  title=\textbf{Mapping: Tension Parameter}]
\begin{equation}
\boxed{T = \frac{\sigma}{R}}
\label{eq:T_mapping}
\end{equation}

\textbf{5D origin:} $T$ is the ratio of brane tension $\sigma$ to ring radius $R$.

\textbf{Epistemic status:} \tagDc --- derived under small-deformation approximation.
\end{tcolorbox}

%============================================================
\section{Stage 3: Israel Junction Conditions at Z$_N$ Points}
%============================================================

\subsection{Localized Sources at Fixed Points}

\begin{postulate}[Discrete Defects] \tagP
\label{post:defects}
At each Z$_N$ fixed point $\theta_n$, there exists a localized contribution
to the stress-energy:
\begin{equation}
\tau_{\mu\nu}^{\text{defect}} = \sum_{n=0}^{N-1} \tau_n \, h_{\mu\nu} \, \delta(\theta - \theta_n)
\label{eq:defect_stress}
\end{equation}
\end{postulate}

\begin{theorem}[Israel Condition at Fixed Points] \tagDer
\label{thm:israel}
The Israel junction condition \cite{companion_C}:
\begin{equation}
[K_{\mu\nu}] - h_{\mu\nu} [K] = -\kappa_5^2 S_{\mu\nu}
\end{equation}

At a Z$_N$ fixed point with localized defect stress $\tau_n$:
\begin{equation}
[K]_{\theta_n} = -\kappa_5^2 \tau_n
\label{eq:israel_local}
\end{equation}
\end{theorem}

\subsection{Anchor Energy from Localized Extrinsic Curvature}

\begin{postulate}[Extrinsic Curvature Coupling] \tagDc
\label{post:K_coupling}
The extrinsic curvature at $\theta_n$ depends on the field value:
\begin{equation}
K(\theta_n) = K_0 + K_1 \cdot u(\theta_n) + O(u^2)
\end{equation}
where $K_0, K_1$ are geometric coefficients.
\end{postulate}

\begin{theorem}[Discrete Anchor Energy] \tagDc
\label{thm:anchor}
Integrating the GHY + defect contributions over the $N$ fixed points:
\begin{equation}
E_{\text{disc}} = \sum_{n=0}^{N-1} \left[ \frac{1}{\kappa_5^2} K(\theta_n) + \tau_n u(\theta_n) \right]
\end{equation}

Defining $\lambda \equiv \tau_n$ (identical defects) and $W(u) \equiv u + \text{const}$:
\begin{equation}
\boxed{E_{\text{disc}} = \lambda \sum_{n=0}^{N-1} W(u(\theta_n))}
\label{eq:E_disc}
\end{equation}
\end{theorem}

\begin{tcolorbox}[colback=blue!5!white, colframe=blue!50!black,
                  title=\textbf{Mapping: Anchor Coupling}]
\begin{equation}
\boxed{\lambda = \tau_n = \kappa_5^2 \cdot (\text{defect stress per anchor})}
\label{eq:lambda_mapping}
\end{equation}

\textbf{5D origin:} $\lambda$ is the localized stress-energy at each Z$_N$ fixed point,
arising from topological defects (Y-junctions, vortices, or pinning sites).

\textbf{Key assumption:} All $N$ anchors have \textit{identical} $\tau_n = \lambda$.

\textbf{Epistemic status:} \tagDc --- assumes Z$_N$ symmetry of defect configuration.
\end{tcolorbox}

%============================================================
\section{Stage 4: Combined Functional}
%============================================================

\begin{theorem}[Total Energy Functional] \tagDc
\label{thm:total}
Combining Eq.~\eqref{eq:E_grad} and Eq.~\eqref{eq:E_disc}:
\begin{equation}
\boxed{E[u] = \frac{T}{2} \int_0^{2\pi} (u')^2 \, d\theta + \lambda \sum_{n=0}^{N-1} W(u(\theta_n))}
\label{eq:E_total}
\end{equation}

This is exactly the toy functional used in the anisotropy normalization derivation.
\end{theorem}

%============================================================
\section{Mapping Dictionary}
%============================================================

\begin{tcolorbox}[colback=yellow!10!white, colframe=orange!70!black,
                  title=\textbf{5D $\to$ Toy Mapping Dictionary}]

\renewcommand{\arraystretch}{1.4}
\begin{tabular}{|l|l|l|l|}
\hline
\textbf{Toy Symbol} & \textbf{5D Origin} & \textbf{Formula} & \textbf{Status} \\
\hline
$\theta$ & Angular coordinate on ring & $\theta \in [0, 2\pi)$ & \tagP \\
$u(\theta)$ & Brane displacement / mode envelope & --- & \tagP \\
$T$ & Brane tension / ring radius & $T = \sigma / R$ & \tagDc \\
$\lambda$ & Defect stress at Z$_N$ fixed points & $\lambda = \kappa_5^2 \tau_n$ & \tagDc \\
$W(u)$ & Extrinsic curvature coupling & $W(u) \approx u$ & \tagDc \\
$\theta_n$ & Z$_N$ fixed points & $\theta_n = 2\pi n / N$ & \tagDer \\
$N$ & Order of discrete symmetry & $|Z_N|$ & \tagDer \\
\hline
\end{tabular}
\end{tcolorbox}

%============================================================
\section{The Complete Chain}
%============================================================

\begin{center}
\begin{tcolorbox}[colback=green!5!white, colframe=green!50!black, width=0.95\textwidth,
                  title=\textbf{Complete Derivation Chain: 5D $\to$ $a/c = 1/N$}]

\textbf{ASCII Diagram:}
\begin{verbatim}
S_5D = S_bulk + S_brane + S_GHY
              |
              | [Dc] Ring embedding, Z_N symmetry
              v
S_4D[u] on ring R
              |
              | [Dc] Small deformation, static limit
              v
E[u] = (T/2) ∫(u')^2 dθ + λ Σ_n W(u(θ_n))
              |
              | [Der] Fourier expansion, Euler-Lagrange
              v
u(θ) = u_0 + a_1 cos(Nθ),  a_1 ∝ 1/N
              |
              | [Der] Identification c = u_0, a = a_1
              v
a/c = 1/N  (equal corner share)
              |
              | [Der] Averaging ratio formula
              v
k(N) = 1 + 1/N
\end{verbatim}

\end{tcolorbox}
\end{center}

%============================================================
\section{Epistemic Status Summary}
%============================================================

\renewcommand{\arraystretch}{1.3}
\begin{tabular}{|l|l|p{7cm}|}
\hline
\textbf{Step} & \textbf{Status} & \textbf{What Would Upgrade to [Der]} \\
\hline
5D action structure & \tagP & Derive from Plenum dynamics \\
Ring embedding & \tagP & Show ring arises from Y-junction topology \\
$T = \sigma/R$ identification & \tagDc & Explicit dimensional reduction with warp factor \\
Identical anchors $\lambda = \tau_n$ & \tagDc & Prove from Z$_N$ symmetry of brane \\
$W(u) \approx u$ linear form & \tagDc & Derive from GHY + Israel terms \\
\hline
Fourier restriction & \tagDer & (Already derived) \\
$a_1 \propto 1/N$ from E-L & \tagDer & (Already derived) \\
$k(N) = 1 + 1/N$ & \tagDer & (Already derived) \\
\hline
\end{tabular}

\vspace{1em}

\textbf{Summary:} The toy functional form is \textit{motivated} from 5D physics but not \textit{derived}
from first principles. The key [Dc] assumptions are:
\begin{enumerate}
\item Ring embedding exists and is Z$_N$-symmetric
\item Brane deformation gives gradient energy $\propto (u')^2$
\item Localized defects at Z$_N$ points give discrete anchors with identical $\lambda$
\end{enumerate}

Once these assumptions are granted, the rest follows by standard variational mechanics [Der].

%============================================================
\section{Conclusion}
%============================================================

We have established the explicit mapping:
\begin{equation}
\underbrace{S_{5D}}_{\text{5D brane-world}} \xrightarrow{\quad [Dc] \quad}
\underbrace{E[u] = \frac{T}{2}\int(u')^2 d\theta + \lambda \sum_n W(u_n)}_{\text{toy functional}}
\xrightarrow{\quad [Der] \quad}
\underbrace{a/c = 1/N}_{\text{normalization}}
\end{equation}

\textbf{What this upgrades:}
\begin{itemize}
\item The toy functional is no longer an \textit{ad hoc} construct
\item Its form has a clear physical interpretation from 5D brane physics
\item The ``equal corner share'' hypothesis ($a/c = 1/N$) has a derivation chain back to 5D
\end{itemize}

\textbf{What remains open:}
\begin{itemize}
\item Full derivation of ring embedding from Y-junction topology
\item Explicit calculation of $T$ and $\lambda$ from 5D parameters
\item BVP verification of Z$_N$ mode structure
\end{itemize}

\end{document}
