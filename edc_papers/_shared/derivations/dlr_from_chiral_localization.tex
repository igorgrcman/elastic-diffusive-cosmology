% =============================================================================
% Derivation: L-R Separation from Chiral Localization
% =============================================================================
% File: dlr_from_chiral_localization.tex
% Created: 2026-01-29
% Purpose: Derive d_LR from 5D Dirac equation; connect to OPR-21 tuned value
%
% COMPANION: docs/DLR_FROM_CHIRAL_LOCALIZATION_NOTE.md
% GOAL: Upgrade d_LR ≈ 8δ from [Cal] toward [Dc]
% =============================================================================

\documentclass[11pt]{article}
\usepackage{amsmath,amssymb,amsthm}
\usepackage[margin=1in]{geometry}
\usepackage{tcolorbox}
\usepackage{booktabs}
\usepackage{xcolor}
% Note: XeLaTeX shows "Missing χ in cmmi10" warning but renders correctly via fallback

\newtheorem{theorem}{Theorem}[section]
\newtheorem{lemma}[theorem]{Lemma}
\newtheorem{proposition}[theorem]{Proposition}
\newtheorem{corollary}[theorem]{Corollary}
\newtheorem{postulate}[theorem]{Postulate}
\theoremstyle{definition}
\newtheorem{definition}[theorem]{Definition}

% Epistemic tags
\newcommand{\tagDer}{\textcolor{blue}{\textbf{[Der]}}}
\newcommand{\tagDc}{\textcolor{green!60!black}{\textbf{[Dc]}}}
\newcommand{\tagP}{\textcolor{orange}{\textbf{[P]}}}
\newcommand{\tagI}{\textcolor{purple}{\textbf{[I]}}}
\newcommand{\tagBL}{\textcolor{gray}{\textbf{[BL]}}}
\newcommand{\tagCal}{\textcolor{red}{\textbf{[Cal]}}}

\title{L-R Separation from Chiral Localization\\[0.5em]
\large Connecting $d_{\mathrm{LR}}$ to 5D Domain-Wall Fermions}
\author{EDC Project}
\date{2026-01-29}

\begin{document}
\maketitle

\begin{abstract}
We analyze the chiral localization of fermions in a 5D brane-world setup
to understand the origin of the L-R center separation $d_{\mathrm{LR}}$.
For a single symmetric domain wall, both chiralities localize at the wall
center, giving $d_{\mathrm{LR}} = 0$. Non-zero separation requires additional
structure: either two walls, asymmetric profiles, or junction geometry.
The OPR-21 gate constraint $I_4 \sim 10^{-3}$ GeV forces
$d_{\mathrm{LR}}/\delta \sim 5$--$10$. The coincidence $d_{\mathrm{LR}} \approx r_p$
(proton radius) suggests junction geometry as the underlying mechanism, but
this remains a hypothesis \tagP{}, not a derivation.
\end{abstract}

\tableofcontents
\newpage

%=============================================================================
\section{The 5D Dirac Equation}
%=============================================================================

\subsection{Setup}

\begin{definition}[5D Dirac Action] \tagDer
\label{def:dirac_action}
The 5D Dirac action for a fermion $\Psi$ coupled to a domain-wall mass is:
\begin{equation}
S = \int d^5x \sqrt{-g} \left[ \bar{\Psi} \left( i \Gamma^M D_M - m(\chi) \right) \Psi \right]
\label{eq:dirac_action}
\end{equation}
where $\chi$ is the extra-dimensional coordinate, $\Gamma^M$ are 5D gamma matrices,
and $m(\chi)$ is a position-dependent mass (domain-wall profile).
\end{definition}

\subsection{Chiral Decomposition}

In 5D, there is no intrinsic chirality. However, for a 4D observer, we can
decompose the 5D spinor into left- and right-handed 4D components:
\begin{equation}
\Psi(x^\mu, \chi) = \psi_L(x^\mu) w_L(\chi) + \psi_R(x^\mu) w_R(\chi)
\label{eq:chiral_decomp}
\end{equation}
where $\gamma_5 \psi_{L,R} = \mp \psi_{L,R}$ (4D chirality).

\begin{proposition}[Mode Equations] \tagDer
\label{prop:mode_eq}
The transverse profiles satisfy coupled first-order equations:
\begin{align}
\partial_\chi w_L + m(\chi) w_L &= \lambda w_R \label{eq:wL_eq} \\
-\partial_\chi w_R + m(\chi) w_R &= \lambda w_L \label{eq:wR_eq}
\end{align}
For massless 4D modes ($\lambda = 0$):
\begin{align}
\partial_\chi w_L^{(0)} &= -m(\chi) w_L^{(0)} \label{eq:wL0} \\
\partial_\chi w_R^{(0)} &= +m(\chi) w_R^{(0)} \label{eq:wR0}
\end{align}
\end{proposition}

\begin{proof}
Standard dimensional reduction of the 5D Dirac equation.
\end{proof}

%=============================================================================
\section{Single Domain Wall: $d_{\mathrm{LR}} = 0$}
%=============================================================================

\subsection{The Tanh Profile}

\begin{definition}[Domain-Wall Mass] \tagDc
\label{def:wall_mass}
A standard domain-wall profile is:
\begin{equation}
m(\chi) = \mu \tanh\left(\frac{\chi}{\delta}\right)
\label{eq:tanh_wall}
\end{equation}
where $\mu$ is the bulk mass scale and $\delta$ is the wall thickness.
\end{definition}

\subsection{Zero Mode Solutions}

\begin{theorem}[Zero Mode Profiles] \tagDer
\label{thm:zero_modes}
For the domain-wall mass \eqref{eq:tanh_wall}, the normalizable zero modes are:
\begin{align}
w_L^{(0)}(\chi) &\propto \exp\left( -\int_0^\chi m(\chi')\, d\chi' \right)
= \left[ \cosh\left(\frac{\chi}{\delta}\right) \right]^{-\mu\delta}
\label{eq:wL_solution} \\
w_R^{(0)}(\chi) &\propto \exp\left( +\int_0^\chi m(\chi')\, d\chi' \right)
= \left[ \cosh\left(\frac{\chi}{\delta}\right) \right]^{+\mu\delta}
\label{eq:wR_solution}
\end{align}
\end{theorem}

\begin{proof}
Direct integration of \eqref{eq:wL0} and \eqref{eq:wR0} using
$\int \tanh(x)\, dx = \ln\cosh(x)$.
\end{proof}

\begin{corollary}[Normalizability] \tagDer
\label{cor:normalizable}
For $\mu > 0$:
\begin{itemize}
\item $w_L^{(0)}$ is normalizable: $\int |w_L^{(0)}|^2 d\chi < \infty$
\item $w_R^{(0)}$ is NOT normalizable: diverges as $|\chi| \to \infty$
\end{itemize}
For $\mu < 0$: roles are reversed.
\end{corollary}

\textbf{Key result:} A single domain wall supports ONE chiral zero mode, not both.

\subsection{Peak Position Analysis}

\begin{proposition}[Peak at Wall Center] \tagDer
\label{prop:peak_center}
For the normalizable mode $w_L^{(0)}$ (with $\mu > 0$):
\begin{equation}
\frac{d|w_L^{(0)}|^2}{d\chi}\bigg|_{\chi = 0} = 0
\end{equation}
The mode peaks at $\chi = 0$ (wall center).
\end{proposition}

\begin{proof}
$|w_L^{(0)}|^2 \propto [\cosh(\chi/\delta)]^{-2\mu\delta}$ which is maximized
at $\chi = 0$ where $\cosh(0) = 1$ is minimal.
\end{proof}

\begin{tcolorbox}[colback=red!10!white, colframe=red!60!black,
title=\textbf{Critical Result: Single Wall Gives $d_{\mathrm{LR}} = 0$}]
For a single symmetric domain wall:
\begin{itemize}
\item Only ONE chirality has a normalizable zero mode
\item That mode peaks at the wall center $\chi = 0$
\item The opposite chirality has no localized mode (or peaks elsewhere)
\end{itemize}

\textbf{Therefore:} A single domain wall CANNOT produce separated L and R
zero modes. To get $d_{\mathrm{LR}} > 0$, we need additional structure.
\end{tcolorbox}

%=============================================================================
\section{Mechanisms for $d_{\mathrm{LR}} > 0$}
%=============================================================================

To achieve L-R separation, several mechanisms are possible:

\subsection{Mechanism A: Two Domain Walls}

\begin{definition}[Two-Wall Configuration] \tagDc
\label{def:two_walls}
Place two domain walls at $\chi = \pm d/2$:
\begin{equation}
m(\chi) = \mu \left[ \tanh\left(\frac{\chi - d/2}{\delta}\right)
- \tanh\left(\frac{\chi + d/2}{\delta}\right) - 1 \right]
\label{eq:two_wall}
\end{equation}
\end{definition}

In this configuration:
\begin{itemize}
\item $w_L$ localizes on wall at $\chi = +d/2$
\item $w_R$ localizes on wall at $\chi = -d/2$ (or vice versa)
\item Separation: $d_{\mathrm{LR}} = d$
\end{itemize}

\textbf{Status:} This is the ``brane-localized fermions'' approach of
Randall-Sundrum and similar models. The separation $d$ is a free parameter
of the geometry --- it must be determined by other physics.

\subsection{Mechanism B: Yukawa Coupling to Higgs Profile}

\begin{definition}[Yukawa Mechanism] \tagDc
\label{def:yukawa}
If there is a Higgs-like field $\phi(\chi)$ with Yukawa coupling:
\begin{equation}
\mathcal{L}_{\mathrm{Yuk}} = -y \phi(\chi) \bar{\psi}_L \psi_R + \mathrm{h.c.}
\label{eq:yukawa}
\end{equation}
and $\phi(\chi)$ has a non-trivial profile, this modifies the effective
localization positions.
\end{definition}

If $\phi(\chi) \propto \chi$ (linear VEV gradient), the L and R modes
experience different effective potentials and localize at different positions.

\textbf{Status:} This requires specifying the Higgs profile $\phi(\chi)$,
which introduces additional parameters.

\subsection{Mechanism C: EDC Junction Geometry}

\begin{postulate}[Y-Junction Localization] \tagP
\label{post:junction}
In the EDC framework, the proton is a Y-junction with three arms meeting
at 120$^\circ$. The characteristic size is the proton charge radius:
\begin{equation}
r_p \approx 0.84\,\mathrm{fm}
\label{eq:rp}
\end{equation}
\end{postulate}

\begin{proposition}[Junction-Induced Separation] \tagP
\label{prop:junction_sep}
If L-modes localize at the junction center and R-modes localize at the
arm tips (or vice versa), the separation is:
\begin{equation}
d_{\mathrm{LR}} \sim r_p \approx 0.84\,\mathrm{fm}
\label{eq:dlr_rp}
\end{equation}
\end{proposition}

\textbf{Status:} This is a hypothesis \tagP{}, not a derivation. It explains
the $d_{\mathrm{LR}} \approx r_p$ coincidence but requires showing that
junction geometry actually produces this localization pattern.

%=============================================================================
\section{Constraint from OPR-21 Gates}
%=============================================================================

\subsection{The I$_4$ Gate}

\begin{definition}[Overlap Integral] \tagDer
\label{def:I4}
The four-point overlap controlling $G_F$ is:
\begin{equation}
I_4 = \int d\chi\, w_L^2(\chi)\, w_R^2(\chi)\, w_\phi^2(\chi)
\label{eq:I4_def}
\end{equation}
where $w_\phi$ is the mediator mode profile.
\end{definition}

\begin{proposition}[Exponential Sensitivity] \tagDer
\label{prop:exp_sensitivity}
For Gaussian-like modes separated by $d_{\mathrm{LR}}$ with width $\sigma$:
\begin{equation}
I_4 \propto \exp\left( -\frac{d_{\mathrm{LR}}^2}{2\sigma^2} \right)
\label{eq:I4_exp}
\end{equation}
The overlap is exponentially suppressed by the separation.
\end{proposition}

\begin{proof}
For $w_L \propto \exp(-(χ-\chi_L)^2/(2\sigma^2))$ and similar for $w_R$
centered at $\chi_R$:
\begin{align}
w_L^2 w_R^2 &\propto \exp\left(-\frac{(\chi-\chi_L)^2 + (\chi-\chi_R)^2}{\sigma^2}\right) \\
&= \exp\left(-\frac{2\chi^2 - 2\chi(\chi_L+\chi_R) + \chi_L^2 + \chi_R^2}{\sigma^2}\right)
\end{align}
Integrating over $\chi$ gives exponential suppression $\propto \exp(-d_{\mathrm{LR}}^2/(2\sigma^2))$.
\end{proof}

\subsection{Gate 1 Constraint}

\begin{proposition}[Required Suppression] \tagDc
\label{prop:gate1}
From OPR-21, Gate 1 requires:
\begin{equation}
I_4 \approx 2 \times 10^{-3}\,\mathrm{GeV}
\label{eq:I4_required}
\end{equation}
to match the target $X_{\mathrm{EDC}} \approx X_{\mathrm{target}} = G_F m_e^2$.
\end{proposition}

\begin{corollary}[Required Separation] \tagDc
\label{cor:required_sep}
With $\sigma \sim \delta \approx 0.5\,\mathrm{GeV}^{-1}$, achieving
$I_4 \sim 10^{-3}$ GeV requires:
\begin{equation}
\frac{d_{\mathrm{LR}}^2}{2\sigma^2} \sim \ln(10^3) \approx 7
\quad \Rightarrow \quad
\frac{d_{\mathrm{LR}}}{\delta} \sim \sqrt{14} \approx 4
\label{eq:dlr_estimate}
\end{equation}
Including polynomial prefactors and the actual mode shapes from BVP:
\begin{equation}
\boxed{\frac{d_{\mathrm{LR}}}{\delta} \approx 5\text{--}10}
\label{eq:dlr_range}
\end{equation}
\end{corollary}

\textbf{Key insight:} The $G_F$ constraint imposes $d_{\mathrm{LR}}/\delta \sim O(10)$.
The BVP scan found $d_{\mathrm{LR}}/\delta = 8$ passes all gates. This is
\textbf{not arbitrary} --- it is the value that produces the required
$I_4$ suppression.

\subsection{Why 8$\delta$ Specifically?}

The sensitivity analysis (OPR-21c) showed:
\begin{itemize}
\item LR elasticity = $-6.5$ (dominant control)
\item 10\% change in $d_{\mathrm{LR}}$ → 65\% change in $X_{\mathrm{ratio}}$
\end{itemize}

The window $d_{\mathrm{LR}}/\delta \in [7, 9]$ passes Gate 1. The center
of this window is $\approx 8$. This is:
\begin{equation}
d_{\mathrm{LR}} = 8\delta = 8 \times 0.533\,\mathrm{GeV}^{-1} = 4.26\,\mathrm{GeV}^{-1} = 0.84\,\mathrm{fm}
\end{equation}

%=============================================================================
\section{The $r_p$ Coincidence}
%=============================================================================

\subsection{Numerical Match}

\begin{center}
\begin{tabular}{lcc}
\toprule
\textbf{Quantity} & \textbf{Value} & \textbf{Source} \\
\midrule
$d_{\mathrm{LR}}$ (from BVP scan) & 0.84 fm & \tagCal{} (Gate 1) \\
$r_p$ (proton charge radius) & 0.841 fm & \tagBL{} (CODATA) \\
Match & 0.1\% & --- \\
\bottomrule
\end{tabular}
\end{center}

\subsection{Possible Interpretations}

\begin{enumerate}
\item \textbf{Coincidence:} Both are O(QCD scale), so agreement within
factors of 2 is not surprising. Many QCD lengths are $\sim 1$ fm.

\item \textbf{Common origin:} If the proton's structure (Y-junction) also
determines the chiral localization geometry, both would be set by the
same length scale.

\item \textbf{Constraint propagation:} Perhaps $r_p$ is \emph{defined} by
this chiral scale, not vice versa. The proton radius emerges from the
same 5D geometry that sets $d_{\mathrm{LR}}$.
\end{enumerate}

\begin{tcolorbox}[colback=yellow!10!white, colframe=yellow!70!black,
title=\textbf{Status of $r_p$ Coincidence}]
The match $d_{\mathrm{LR}} \approx r_p$ is:
\begin{itemize}
\item \textbf{NOT derived} --- we don't have a mechanism connecting them
\item \textbf{NOT accidental} --- the O(10\%) precision is suggestive
\item \textbf{Hypothesis-forming} --- suggests Y-junction geometry
\end{itemize}

\textbf{To upgrade:} Derive $r_p$ and $d_{\mathrm{LR}}$ from the same
5D junction geometry. Show that the localization positions depend on
the junction arm length, which equals $r_p$.
\end{tcolorbox}

%=============================================================================
\section{Derivation Summary}
%=============================================================================

\begin{center}
\renewcommand{\arraystretch}{1.3}
\begin{tabular}{|p{5cm}|c|p{6cm}|}
\hline
\textbf{Result} & \textbf{Status} & \textbf{Comment} \\
\hline
\hline
Single domain wall → $d_{\mathrm{LR}} = 0$ & \tagDer & Standard result \\
\hline
$d_{\mathrm{LR}} > 0$ requires additional structure & \tagDer & Two walls, Yukawa, or junction \\
\hline
Gate 1 imposes $d_{\mathrm{LR}}/\delta \sim 5$--10 & \tagDc & From $I_4 \sim 10^{-3}$ GeV \\
\hline
BVP scan gives $d_{\mathrm{LR}}/\delta = 8$ & \tagCal & Fit to target $X_{\mathrm{EDC}}$ \\
\hline
$d_{\mathrm{LR}} = 0.84$ fm $\approx r_p$ & \tagI & Coincidence (suggestive) \\
\hline
Junction geometry → $d_{\mathrm{LR}} \sim r_p$ & \tagP & Hypothesis \\
\hline
\end{tabular}
\end{center}

%=============================================================================
\section{Stoplight Verdict}
%=============================================================================

\begin{tcolorbox}[colback=yellow!10!white, colframe=yellow!70!black,
title=\textbf{Epistemic Status: YELLOW [Dc/Cal]}]

\textbf{What is derived [Der]:}
\begin{itemize}
\item Single symmetric domain wall gives $d_{\mathrm{LR}} = 0$
\item Non-zero separation requires two walls, Yukawa, or junction geometry
\item Overlap integral $I_4$ is exponentially sensitive to $d_{\mathrm{LR}}$
\end{itemize}

\textbf{What is derived conditional [Dc]:}
\begin{itemize}
\item Gate 1 constraint imposes $d_{\mathrm{LR}}/\delta \sim 5$--10
\item Two-wall or junction mechanism can produce this separation
\end{itemize}

\textbf{What is calibrated [Cal]:}
\begin{itemize}
\item Specific value $d_{\mathrm{LR}}/\delta = 8$ from parameter scan
\end{itemize}

\textbf{What is a hypothesis [P]:}
\begin{itemize}
\item Y-junction geometry produces $d_{\mathrm{LR}} \sim r_p$
\item Proton charge radius equals chiral separation scale
\end{itemize}

\textbf{Upgrade path YELLOW → GREEN:}
\begin{enumerate}
\item Derive the two-wall separation (or junction geometry) from 5D action
\item Show that $d_{\mathrm{LR}} = r_p$ follows from proton = Y-junction
\item Compute $r_p$ itself from the junction geometry
\end{enumerate}

\end{tcolorbox}

%=============================================================================
\section{Conclusion}
%=============================================================================

We have established:

\begin{enumerate}
\item \textbf{A single domain wall cannot produce L-R separation.} Both
chiralities (when present) localize at the wall center.

\item \textbf{To get $d_{\mathrm{LR}} > 0$, we need:}
\begin{itemize}
\item Two domain walls at different positions, OR
\item Yukawa coupling to a non-trivial Higgs profile, OR
\item Y-junction geometry (EDC-specific)
\end{itemize}

\item \textbf{The $G_F$ gate imposes $d_{\mathrm{LR}}/\delta \sim 5$--10.}
This is a constraint from requiring $I_4 \sim 10^{-3}$ GeV. The BVP scan
found $d_{\mathrm{LR}}/\delta = 8$ satisfies all gates.

\item \textbf{The coincidence $d_{\mathrm{LR}} \approx r_p$ is suggestive
but not derived.} It motivates the Y-junction interpretation but does not
prove it.
\end{enumerate}

\textbf{Defendable claim:} The value $d_{\mathrm{LR}}/\delta \approx 8$ is
\textbf{not arbitrary tuning} --- it is constrained by the $G_F$ gate to
lie in the range $[5, 10]$. The specific value $8$ is the center of this
window. The coincidence with $r_p$ suggests deeper structure but remains
a hypothesis.

\end{document}
