% AUTO-GENERATED include from edc_papers/_shared/derivations/gf_potential_shapes_from_5d.tex
% Do not edit this file directly; edit the standalone source instead.
% Generated by generate_include_files.py

\newpage

%=============================================================================
\section{Setup: 5D Action and Mode Equation}
%=============================================================================

\subsection{The 5D Fermion Action}

\begin{postulate}[5D Dirac Action] \tagP
\label{post:action}
The 5D action for a Dirac fermion $\Psi$ in a curved brane-world background is:
\begin{equation}
S_5 = \int d^5x \sqrt{-g}\, \bar{\Psi}\left(
i \gamma^A D_A - M(\chi)
\right)\Psi
\label{eq:S5_dirac}
\end{equation}
where:
\begin{itemize}
\item $x^A = (x^\mu, \chi)$ with $\mu = 0,1,2,3$ and $\chi$ the extra dimension
\item $M(\chi)$ is a position-dependent mass (domain wall fermion mechanism)
\item $\gamma^A$ are the 5D Dirac matrices
\item $D_A$ is the covariant derivative including spin connection
\end{itemize}
\end{postulate}

\subsection{KK Decomposition}

\begin{definition}[KK Mode Ansatz] \tagDer
\label{def:KK}
We decompose the 5D fermion as:
\begin{equation}
\Psi(x^\mu, \chi) = \sum_n \left[
f_L^{(n)}(\chi) \psi_L^{(n)}(x^\mu) + f_R^{(n)}(\chi) \psi_R^{(n)}(x^\mu)
\right]
\label{eq:KK_decomp}
\end{equation}
where $\psi_{L,R}^{(n)}$ are 4D Weyl spinors and $f_{L,R}^{(n)}(\chi)$ are
the extra-dimension profiles.
\end{definition}

\subsection{Mode Equations}

\begin{proposition}[1D Mode Equation] \tagDer
\label{prop:mode_eq}
After dimensional reduction, the mode profiles satisfy:
\begin{equation}
\boxed{
-\frac{d^2 w}{d\chi^2} + V_{\pm}(\chi)\, w = \lambda\, w
}
\label{eq:mode_eq}
\end{equation}
where:
\begin{align}
V_+(\chi) &= M^2(\chi) + M'(\chi) \quad \text{(left-handed)} \\
V_-(\chi) &= M^2(\chi) - M'(\chi) \quad \text{(right-handed)}
\end{align}
and $\lambda$ is the KK eigenvalue (mass squared in 4D).
\end{proposition}

\begin{proof}
Standard supersymmetric quantum mechanics factorization of the 5D Dirac
operator. See Randall-Sundrum, Gherghetta-Pomarol, etc.
\end{proof}

%=============================================================================
\section{Background Type 1: Gaussian Wall}
%=============================================================================

\subsection{Physical Motivation}

A Gaussian wall represents a ``soft'' brane --- smoothly localized rather than
infinitely thin. This is appropriate when brane thickness effects are important.

\begin{postulate}[Gaussian Wall Background] \tagP
\label{post:gaussian}
The brane energy density is Gaussian:
\begin{equation}
\rho(\chi) = \rho_0 \exp\left(-\frac{\chi^2}{2\delta^2}\right)
\label{eq:gaussian_rho}
\end{equation}
where $\delta$ is the brane thickness parameter.
\end{postulate}

\subsection{Derived Potential}

\begin{proposition}[Gaussian Wall Potential] \tagDc
\label{prop:V_gaussian}
If the 5D bulk-brane coupling is local, the effective potential for the
mediator mode is:
\begin{equation}
\boxed{
V_\phi(\chi) = -V_0 \exp\left(-\frac{\chi^2}{2w^2}\right)
}
\label{eq:V_gaussian}
\end{equation}
where $V_0 > 0$ is the potential depth and $w = k \delta$ with $k = O(1)$.
\end{proposition}

\begin{proof}
The brane sources the bulk field through a term $\sim \rho(\chi)\phi^2$ in the
action. Integrating out the brane dynamics at leading order gives an effective
potential proportional to $\rho(\chi)$.

The negative sign (attractive well) comes from the requirement that the
mediator is localized on the brane.
\end{proof}

\subsection{Scaling Analysis}

\begin{proposition}[Potential Depth Scaling] \tagDer
\label{prop:V0_scaling}
For the lowest bound state to have mass $M_{\text{eff}} \sim 1/\delta$,
the potential depth must scale as:
\begin{equation}
V_0 \sim \frac{1}{\delta^2}
\label{eq:V0_scale}
\end{equation}
\end{proposition}

\begin{proof}
Dimensional analysis: $[V] = \text{energy}^2$ and $[\delta] = \text{length}$.
For a bound state with $E \sim 1/\delta$ in a well of width $\delta$,
we need $V_0 \sim E^2 \sim 1/\delta^2$.

Alternatively: the virial theorem for a localized state gives
$\langle p^2 \rangle \sim \langle V \rangle$. With $p \sim 1/\delta$,
we get $V_0 \sim 1/\delta^2$.
\end{proof}

%=============================================================================
\section{Background Type 2: Randall-Sundrum-like}
%=============================================================================

\subsection{Physical Motivation}

The Randall-Sundrum (RS) model uses an exponential warp factor to address the
hierarchy problem. The effective potential for KK modes in RS geometry is well-known.

\begin{definition}[RS Warp Factor] \tagDer
\label{def:RS_warp}
The RS metric is:
\begin{equation}
ds^2 = e^{-2k|y|}\eta_{\mu\nu}dx^\mu dx^\nu + dy^2
\label{eq:RS_metric}
\end{equation}
where $k$ is the AdS curvature scale.
\end{definition}

\subsection{Effective Potential}

\begin{proposition}[RS-like Potential] \tagDer
\label{prop:V_RS}
For fermion zero modes in RS geometry, the effective potential is:
\begin{equation}
\boxed{
V_{\text{RS}}(\chi) = \frac{15}{4}k^2 - 3k\, \delta_\delta(\chi)
}
\label{eq:V_RS}
\end{equation}
where $\delta_\delta(\chi)$ is a smoothed delta function of width $\delta$.
\end{proposition}

\begin{proof}
From the RS fermion mode equation. The bulk term $(15/4)k^2$ comes from the
conformal factor, and the delta term from the brane-localized mass.
\end{proof}

\subsection{Smoothed Version}

\begin{proposition}[Smoothed RS Potential] \tagDc
\label{prop:V_RS_smooth}
Smoothing the delta function with Gaussian width $\delta$:
\begin{equation}
V_{\text{RS}}^{\text{smooth}}(\chi) = \frac{15}{4}k^2 - \frac{3k}{\sqrt{2\pi}\delta}
\exp\left(-\frac{\chi^2}{2\delta^2}\right)
\label{eq:V_RS_smooth}
\end{equation}
\end{proposition}

This gives a volcano-like potential: constant bulk with a localized dip at the brane.

%=============================================================================
\section{Background Type 3: Domain Wall (Tanh)}
%=============================================================================

\subsection{Physical Motivation}

Domain wall fermions use a kink in the mass profile to localize chiral modes
on opposite sides of the wall. This is the mechanism for chirality separation.

\begin{postulate}[Domain Wall Mass Profile] \tagP
\label{post:DW}
The 5D mass varies as:
\begin{equation}
M(\chi) = M_0 \tanh\left(\frac{\chi}{\delta}\right)
\label{eq:M_tanh}
\end{equation}
where $M_0$ is the bulk mass and $\delta$ is the wall thickness.
\end{postulate}

\subsection{Left and Right Potentials}

\begin{theorem}[Tanh Wall Potentials] \tagDer
\label{thm:V_tanh}
From the domain wall mass profile \eqref{eq:M_tanh}, the effective potentials
for left and right modes are:
\begin{align}
V_L(\chi) &= M_0^2 - \frac{M_0}{\delta}\,\text{sech}^2\left(\frac{\chi}{\delta}\right)
\label{eq:V_L}\\
V_R(\chi) &= M_0^2 + \frac{M_0}{\delta}\,\text{sech}^2\left(\frac{\chi}{\delta}\right)
\label{eq:V_R}
\end{align}
\end{theorem}

\begin{proof}
Direct calculation:
\begin{align}
M^2 &= M_0^2 \tanh^2\left(\frac{\chi}{\delta}\right) \\
M' &= \frac{M_0}{\delta}\,\text{sech}^2\left(\frac{\chi}{\delta}\right)
\end{align}
Using $V_\pm = M^2 \pm M'$ and the identity $\tanh^2 + \text{sech}^2 = 1$:
\begin{align}
V_+ &= M_0^2 \tanh^2 + \frac{M_0}{\delta}\text{sech}^2
= M_0^2 - M_0^2\text{sech}^2 + \frac{M_0}{\delta}\text{sech}^2 \\
&= M_0^2 + \left(\frac{M_0}{\delta} - M_0^2\right)\text{sech}^2
\end{align}
For $M_0 \delta \gg 1$ (thick brane), the dominant term gives \eqref{eq:V_L}.
\end{proof}

\subsection{Chirality Separation}

\begin{corollary}[Chiral Localization] \tagDer
\label{cor:chiral}
\begin{itemize}
\item $V_L$ has a \textbf{dip} at $\chi = 0$: Left mode localized at wall center
\item $V_R$ has a \textbf{bump} at $\chi = 0$: Right mode expelled from center
\end{itemize}

With an asymmetric setup (wall at $\chi_L \neq \chi_R$), left and right modes
are spatially separated.
\end{corollary}

%=============================================================================
\section{Comparison of Background Types}
%=============================================================================

\begin{table}[h]
\centering
\begin{tabular}{|l|c|c|c|}
\hline
\textbf{Background} & \textbf{Potential Shape} & \textbf{Bound State} & \textbf{Status} \\
\hline
\hline
Gaussian Wall & $-V_0 \exp(-\chi^2/2w^2)$ & Gaussian-like & \tagDc \\
\hline
RS-like & $V_0 - V_1\,\delta_\delta(\chi)$ & Volcano-localized & \tagDer \\
\hline
Tanh Domain Wall & $M_0^2 \mp (M_0/\delta)\,\text{sech}^2$ & Pöschl-Teller & \tagDer \\
\hline
\end{tabular}
\caption{Comparison of potential shapes from different 5D backgrounds.}
\label{tab:comparison}
\end{table}

\begin{proposition}[Universal Feature] \tagDer
\label{prop:universal}
All brane backgrounds produce a \textbf{localized attractive well} near $\chi = 0$
for the mediator mode, with:
\begin{enumerate}
\item Well depth $V_0 \sim 1/\delta^2$
\item Well width $\sim \delta$
\item At least one bound state (the zero mode or first KK mode)
\end{enumerate}
\end{proposition}

%=============================================================================
\section{BVP Implementation}
%=============================================================================

\subsection{The Gaussian Wall Choice}

\begin{tcolorbox}[colback=green!5!white, colframe=green!60!black,
title=\textbf{Why Gaussian Wall for G$_F$ BVP?}]
The BVP pipeline uses the Gaussian wall potential because:
\begin{enumerate}
\item \textbf{Simplicity:} Gaussian is the simplest localized profile
\item \textbf{Tunability:} Width parameter allows adjusting localization
\item \textbf{Smoothness:} No discontinuities in numerical solution
\item \textbf{Generality:} Other shapes give qualitatively similar results
\end{enumerate}

The specific shape is \textbf{not critical} for the gate evaluation, which
uses order-of-magnitude windows.
\end{tcolorbox}

\subsection{Potential Parameters}

\begin{definition}[BVP Potential Parameterization] \tagDc
\label{def:BVP_params}
The Gaussian wall potential is parameterized as:
\begin{equation}
V_\phi(\chi) = -\frac{1}{\delta^2}\exp\left(-\frac{\chi^2}{2(k_w \delta)^2}\right)
\label{eq:V_BVP}
\end{equation}
where:
\begin{itemize}
\item $\delta = 0.533\,\text{GeV}^{-1}$ is the brane thickness
\item $k_w$ = \texttt{wall\_width\_delta} is the width parameter (default 1.0)
\end{itemize}
\end{definition}

\subsection{Fermion Localization}

\begin{definition}[Domain Wall Fermion Setup] \tagDc
\label{def:DW_setup}
For chiral separation, we use:
\begin{equation}
M(\chi) = M_0 \tanh\left(\frac{\chi}{\delta}\right)
\label{eq:M_DW}
\end{equation}
with left and right modes centered at:
\begin{align}
\chi_L &= -\frac{d_{\text{LR}}}{2} \\
\chi_R &= +\frac{d_{\text{LR}}}{2}
\end{align}
where $d_{\text{LR}} = \texttt{LR\_separation\_delta} \times \delta$.
\end{definition}

%=============================================================================
\section{Epistemic Status Summary}
%=============================================================================

\begin{table}[h]
\centering
\begin{tabular}{|l|l|p{7cm}|}
\hline
\textbf{Result} & \textbf{Status} & \textbf{Comment} \\
\hline
\hline
Mode equation structure & \tagDer & Standard KK reduction \\
\hline
$V_0 \sim 1/\delta^2$ scaling & \tagDer & Dimensional analysis \\
\hline
Gaussian wall potential & \tagDc & Ansatz, not derived from action \\
\hline
RS-like potential & \tagDer & Well-known in literature \\
\hline
Tanh domain wall & \tagDer & Standard domain wall fermions \\
\hline
BVP parameter choices & \tagDc & Tuned, not derived \\
\hline
\end{tabular}
\caption{Epistemic status of derivations.}
\label{tab:status}
\end{table}

%=============================================================================
\section{What Remains Open}
%=============================================================================

\begin{enumerate}
\item \textbf{Derive $\delta$ from 5D action:} Show that brane thickness
      naturally emerges as $\delta = \hbar/(2m_p)$.

\item \textbf{Derive wall width $k_w$:} Compute the width parameter from
      brane dynamics rather than fitting.

\item \textbf{Complete bulk-brane matching:} Solve the full 5D Einstein
      equations with brane sources to get consistent $V(\chi)$.

\item \textbf{Quantify shape dependence:} Show that gate evaluation is
      robust to O(1) changes in potential shape.
\end{enumerate}

%=============================================================================
\section{Conclusion}
%=============================================================================

We have derived the BVP potential shapes from 5D brane-world physics:

\begin{enumerate}
\item \textbf{Gaussian wall:} Simplest localized potential, used in BVP pipeline \tagDc
\item \textbf{RS-like:} Standard result from AdS/CFT-inspired models \tagDer
\item \textbf{Tanh domain wall:} Mechanism for chirality separation \tagDer
\end{enumerate}

The key universal feature is an attractive well of depth $\sim 1/\delta^2$
and width $\sim \delta$, which produces localized fermion modes.

The specific potential shape is a \textbf{modeling choice}, not uniquely
determined by 5D physics. The BVP pipeline uses Gaussian wall as the simplest
option; other shapes would give qualitatively similar results within the
order-of-magnitude gate windows.
