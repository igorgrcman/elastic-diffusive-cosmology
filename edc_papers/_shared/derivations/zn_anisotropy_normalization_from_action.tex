% Z_N Anisotropy Normalization from Action
% File: edc_papers/_shared/derivations/zn_anisotropy_normalization_from_action.tex
% Created: 2026-01-29
% Purpose: Derive a/c ~ 1/N from Z_N symmetric energy minimization
%
% EPISTEMIC STATUS:
%   [Der] - Toy model derivation (energy functional on ring)
%   [Dc]  - Mapping to true 5D membrane action
%
% COMPANION: docs/ZN_NORMALIZATION_FROM_ACTION_NOTE.md

\documentclass[11pt]{article}
\usepackage{amsmath,amssymb,amsthm}
\usepackage[margin=1in]{geometry}
\usepackage{tcolorbox}

\newtheorem{theorem}{Theorem}
\newtheorem{lemma}[theorem]{Lemma}
\newtheorem{proposition}[theorem]{Proposition}

\title{Z$_N$ Anisotropy Normalization from Energy Minimization}
\author{EDC Project}
\date{2026-01-29}

\begin{document}
\maketitle

\begin{abstract}
We derive the ``equal corner share'' normalization $a/c = 1/N$ for Z$_N$-symmetric
profiles from energy minimization on a ring with discrete anchor couplings.
The result follows from the balance between gradient energy (scaling as $N^2$)
and discrete anchor contributions (scaling as $N$). This provides physical
justification for the hypothesis used in the $k$-channel correction formula.
\end{abstract}

%============================================================
\section{Setup}
%============================================================

Consider a ring parameterized by $\theta \in [0, 2\pi)$ with Z$_N$ symmetry.
Let $u(\theta)$ be a scalar field representing membrane displacement, thickness
variation, or mode envelope.

\textbf{Z$_N$ Symmetry Constraint:}
\begin{equation}
u\left(\theta + \frac{2\pi}{N}\right) = u(\theta)
\end{equation}

The corners (fixed points of Z$_N$) are located at:
\begin{equation}
\theta_n = \frac{2\pi n}{N}, \quad n = 0, 1, \ldots, N-1
\end{equation}

%============================================================
\section{Energy Functional}
%============================================================

We consider a minimal energy functional with two components:

\begin{equation}
\boxed{E[u] = E_{\text{cont}}[u] + E_{\text{disc}}[u]}
\end{equation}

\subsection{Continuum (Gradient) Energy}

The continuum contribution penalizes spatial variations:
\begin{equation}
E_{\text{cont}}[u] = \frac{T}{2} \int_0^{2\pi} \left(\frac{du}{d\theta}\right)^2 d\theta
\label{eq:E_cont}
\end{equation}
where $T > 0$ is the tension/stiffness parameter.

\textit{Physical origin [Dc]:} In the 5D membrane picture, this arises from
the brane tension $\sigma$ integrated over the ring.

\subsection{Discrete (Anchor) Energy}

The discrete contribution couples the field at each corner:
\begin{equation}
E_{\text{disc}}[u] = \lambda \sum_{n=0}^{N-1} W(u(\theta_n))
\label{eq:E_disc}
\end{equation}
where:
\begin{itemize}
\item $\lambda > 0$ is the coupling strength \textbf{per anchor}
\item $W(u)$ is a potential (e.g., $W(u) = u$ or $W(u) = u^2/2$)
\item The sum runs over all $N$ corners
\end{itemize}

\textit{Physical origin [Dc]:} In the 5D picture, discrete anchors may arise from:
\begin{itemize}
\item Topological pinning at Y-junction positions
\item Discrete gauge field couplings at Z$_N$ fixed points
\item Localized sources in the GHY boundary terms
\end{itemize}

\textbf{Key assumption:} All $N$ anchors are \textit{identical} (same $\lambda$, same $W$).

%============================================================
\section{Fourier Expansion and Z$_N$ Constraint}
%============================================================

For a Z$_N$-symmetric field, the Fourier expansion contains only modes
with period $2\pi/N$:
\begin{equation}
u(\theta) = u_0 + \sum_{m=1}^{\infty} \left[ a_m \cos(mN\theta) + b_m \sin(mN\theta) \right]
\end{equation}

The dominant anisotropic mode is $m = 1$:
\begin{equation}
u(\theta) \approx u_0 + a_1 \cos(N\theta)
\label{eq:u_expansion}
\end{equation}

We identify:
\begin{itemize}
\item $c \equiv u_0$ = isotropic baseline (mean value)
\item $a \equiv a_1$ = anisotropy amplitude
\end{itemize}

%============================================================
\section{Energy Evaluation}
%============================================================

\subsection{Continuum Energy}

For $u(\theta) = u_0 + a_1 \cos(N\theta)$:
\begin{align}
\frac{du}{d\theta} &= -a_1 N \sin(N\theta) \\
\left(\frac{du}{d\theta}\right)^2 &= a_1^2 N^2 \sin^2(N\theta)
\end{align}

Using $\int_0^{2\pi} \sin^2(N\theta) \, d\theta = \pi$:
\begin{equation}
\boxed{E_{\text{cont}} = \frac{T}{2} \cdot a_1^2 N^2 \cdot \pi = \frac{\pi T N^2}{2} a_1^2}
\label{eq:E_cont_result}
\end{equation}

\textbf{Key observation:} Gradient energy scales as $N^2$ for the $\cos(N\theta)$ mode.

\subsection{Discrete Energy}

At corners $\theta_n = 2\pi n/N$:
\begin{equation}
\cos(N \theta_n) = \cos(2\pi n) = 1 \quad \forall n
\end{equation}

Therefore:
\begin{equation}
u(\theta_n) = u_0 + a_1 \cdot 1 = u_0 + a_1 \quad \forall n
\end{equation}

The discrete energy becomes:
\begin{equation}
\boxed{E_{\text{disc}} = \lambda N \cdot W(u_0 + a_1)}
\label{eq:E_disc_result}
\end{equation}

\textbf{Key observation:} Discrete energy scales as $N$ (number of anchors).

%============================================================
\section{Minimization}
%============================================================

\subsection{Linearization for Small Anisotropy}

Expand $W$ around the baseline $u_0$:
\begin{equation}
W(u_0 + a_1) \approx W(u_0) + W'(u_0) \cdot a_1 + \frac{1}{2} W''(u_0) \cdot a_1^2
\end{equation}

Total energy:
\begin{equation}
E = \frac{\pi T N^2}{2} a_1^2 + \lambda N \left[ W(u_0) + W'(u_0) a_1 + \frac{1}{2} W''(u_0) a_1^2 \right]
\end{equation}

\subsection{Equilibrium Condition}

Setting $\partial E / \partial a_1 = 0$:
\begin{equation}
\pi T N^2 \cdot a_1 + \lambda N W'(u_0) + \lambda N W''(u_0) \cdot a_1 = 0
\end{equation}

Solving for $a_1$:
\begin{equation}
a_1 = -\frac{\lambda N W'(u_0)}{\pi T N^2 + \lambda N W''(u_0)}
= -\frac{\lambda W'(u_0)}{\pi T N + \lambda W''(u_0)}
\label{eq:a1_solution}
\end{equation}

\subsection{Large-$N$ or Tension-Dominated Regime}

When $\pi T N \gg \lambda W''(u_0)$ (tension dominates):
\begin{equation}
\boxed{a_1 \approx -\frac{\lambda W'(u_0)}{\pi T N} \propto \frac{1}{N}}
\label{eq:a1_scaling}
\end{equation}

\textbf{Result:} The anisotropy amplitude scales as $1/N$.

%============================================================
\section{The Ratio $a/c$}
%============================================================

With $c = u_0$ and $a = a_1$:
\begin{equation}
\frac{a}{c} = \frac{a_1}{u_0} \approx -\frac{\lambda W'(u_0)}{\pi T N u_0}
\end{equation}

Define the dimensionless ratio:
\begin{equation}
\xi \equiv \frac{\lambda W'(u_0)}{\pi T u_0}
\end{equation}

Then:
\begin{equation}
\boxed{\frac{a}{c} = -\frac{\xi}{N} \sim \frac{1}{N}}
\label{eq:ac_scaling}
\end{equation}

\textbf{Equal corner share interpretation:} Each of the $N$ anchors contributes
$\xi/N$ to the relative anisotropy. The total anisotropy is divided equally
among $N$ identical anchors.

%============================================================
\section{Connection to $k$-Channel Formula}
%============================================================

\begin{theorem}[Anisotropy Normalization]
\label{thm:anisotropy_norm}
For a Z$_N$-symmetric profile $|f(\theta)|^4 = c + a \cos(N\theta)$ arising from
energy minimization with $N$ identical discrete anchors, the ratio satisfies:
\begin{equation}
\frac{a}{c} \sim \frac{1}{N}
\end{equation}
in the tension-dominated regime.
\end{theorem}

\textbf{Application to $k$-channel:} The discrete-to-continuum averaging ratio is:
\begin{equation}
R = \frac{\langle f \rangle_{\text{disc}}}{\langle f \rangle_{\text{cont}}} = 1 + \frac{a}{c}
\end{equation}

With $a/c = 1/N$:
\begin{equation}
\boxed{R = k(N) = 1 + \frac{1}{N}}
\end{equation}

For Z$_6$: $k(6) = 7/6 = 1.1667$.

%============================================================
\section{Physical Interpretation}
%============================================================

\begin{tcolorbox}[colback=blue!5!white, colframe=blue!50!black,
                  title=\textbf{Equal Corner Share Principle}]
When $N$ identical discrete couplings are distributed on a Z$_N$-symmetric ring,
the equilibrium anisotropy amplitude is:
\begin{equation}
\frac{a}{c} = \frac{1}{N}
\end{equation}

\textbf{Mechanism:} The gradient energy penalizes the $\cos(N\theta)$ mode
with a factor $N^2$, while the discrete anchors contribute linearly as $N$.
The balance gives amplitude $\propto 1/N$.

\textbf{Intuition:} Each corner ``owns'' $1/N$ of the total anisotropy.
\end{tcolorbox}

%============================================================
\section{Epistemic Status}
%============================================================

\begin{tabular}{|l|l|l|}
\hline
\textbf{Component} & \textbf{Status} & \textbf{Note} \\
\hline
Energy functional structure & [Der] & Standard variational mechanics \\
Gradient energy $\propto N^2$ & [Der] & Fourier mode analysis \\
Discrete sum $\propto N$ & [Der] & Direct counting \\
Balance $\Rightarrow a_1 \propto 1/N$ & [Der] & Euler-Lagrange equation \\
\hline
Identification of $E_{\text{cont}}$ with 5D brane tension & [Dc] & Plausible, not proven \\
Identification of $E_{\text{disc}}$ with anchor couplings & [Dc] & Requires GHY/Israel terms \\
Tension-dominated regime assumption & [Dc] & Needs verification from 5D \\
\hline
\end{tabular}

\subsection{What Would Upgrade [Dc] $\rightarrow$ [Der]}

\begin{enumerate}
\item \textbf{Explicit 5D reduction:} Show that dimensional reduction of
      $S_{5D} = S_{\text{bulk}} + S_{\text{brane}} + S_{\text{GHY}}$
      on a Z$_N$-symmetric background produces the functional Eq.~\eqref{eq:E_cont}--\eqref{eq:E_disc}.
\item \textbf{Israel junction conditions:} Verify that boundary matching at
      the brane gives identical anchor couplings at Z$_N$ fixed points.
\item \textbf{BVP mode profiles:} Compute $f(\theta)$ from 5D BVP and verify
      the $\cos(N\theta)$ structure with amplitude $\sim 1/N$.
\end{enumerate}

%============================================================
\section{Summary}
%============================================================

\begin{enumerate}
\item The ``equal corner share'' normalization $a/c = 1/N$ is \textbf{derived}
      from energy minimization in a toy model with Z$_N$ symmetric anchors.
\item The physical mechanism is the \textbf{competition} between:
      \begin{itemize}
      \item Gradient energy $\sim N^2$ (penalizes high-frequency anisotropy)
      \item Discrete anchors $\sim N$ (drive anisotropy)
      \end{itemize}
\item The result justifies the $k$-channel formula $k(N) = 1 + 1/N$.
\item Mapping to the full 5D membrane action remains [Dc] (open).
\end{enumerate}

\end{document}
