% AUTO-GENERATED include from edc_papers/_shared/derivations/zn_ring_delta_pinning_modes.tex
% Do not edit this file directly; edit the standalone source instead.
% Generated by generate_include_files.py

\newpage

%============================================================
\section{Model Setup}
%============================================================

\subsection{Linearized Energy Functional}

\begin{definition}[Quadratic Anchor Potential] \tagDer
\label{def:quadratic_W}
For linearization around equilibrium $u_0$, we take:
\begin{equation}
W(u) = \frac{\kappa}{2}(u - u_0)^2
\label{eq:W_quadratic}
\end{equation}
where $\kappa = W''(u_0) > 0$ is the local stiffness.
\end{definition}

\begin{definition}[Linearized Energy Functional] \tagDer
\label{def:E_linear}
The energy functional for small perturbations $v = u - u_0$:
\begin{equation}
\boxed{E[v] = \frac{T}{2} \int_0^{2\pi} (v')^2 \, d\theta
+ \frac{\lambda\kappa}{2} \sum_{n=0}^{N-1} v(\theta_n)^2}
\label{eq:E_linear}
\end{equation}
where $\theta_n = 2\pi n/N$ are the Z$_N$ fixed points.
\end{definition}

\subsection{Euler-Lagrange Equation}

\begin{proposition}[Static Equation with Delta Sources] \tagDer
\label{prop:EL}
The Euler-Lagrange equation for $E[v]$ is:
\begin{equation}
-T v''(\theta) + \lambda\kappa \sum_{n=0}^{N-1} v(\theta_n) \, \delta(\theta - \theta_n) = 0
\label{eq:EL_static}
\end{equation}
\end{proposition}

\begin{proof}
Variation of \eqref{eq:E_linear}:
\begin{align}
\delta E &= T \int_0^{2\pi} v' \delta v' \, d\theta
+ \lambda\kappa \sum_{n=0}^{N-1} v(\theta_n) \, \delta v(\theta_n) \\
&= -T \int_0^{2\pi} v'' \delta v \, d\theta
+ \lambda\kappa \sum_{n=0}^{N-1} v(\theta_n) \int_0^{2\pi} \delta(\theta-\theta_n) \delta v \, d\theta
\end{align}
Setting $\delta E = 0$ for all $\delta v$ gives \eqref{eq:EL_static}.
\end{proof}

\subsection{Operator Form}

\begin{definition}[Linear Operator] \tagDer
\label{def:L_op}
Define the linear operator:
\begin{equation}
\mathcal{L} = -T \frac{d^2}{d\theta^2} + \lambda\kappa \sum_{n=0}^{N-1} \delta(\theta - \theta_n) \, P_n
\label{eq:L_op}
\end{equation}
where $P_n$ is the projection (evaluation) at $\theta_n$: $(P_n f) = f(\theta_n)$.
\end{definition}

The eigenvalue problem is:
\begin{equation}
\mathcal{L} v = \mu v
\label{eq:eigenvalue}
\end{equation}

%============================================================
\section{Group-Theoretic Decomposition}
%============================================================

\subsection{Z$_N$ Symmetry of the Operator}

\begin{lemma}[Z$_N$ Invariance] \tagDer
\label{lem:ZN_invariance}
The operator $\mathcal{L}$ commutes with the Z$_N$ rotation $R_{2\pi/N}$:
\begin{equation}
[\mathcal{L}, R_{2\pi/N}] = 0
\label{eq:commutator}
\end{equation}
\end{lemma}

\begin{proof}
The Laplacian $-d^2/d\theta^2$ is rotationally invariant.
The pinning term $\sum_n \delta(\theta-\theta_n)$ is invariant under
$\theta \to \theta + 2\pi/N$ because the set $\{\theta_n\}$ maps to itself.
\end{proof}

\begin{corollary}[Mode Classification] \tagDer
\label{cor:irreps}
Eigenfunctions of $\mathcal{L}$ can be classified by Z$_N$ irreducible representations.
For $\mathbb{Z}_N$, the irreps are labeled by $\ell = 0, 1, \ldots, N-1$:
\begin{equation}
R_{2\pi/N} \, \psi_\ell = e^{2\pi i \ell / N} \, \psi_\ell
\label{eq:irrep_label}
\end{equation}
\end{corollary}

\subsection{Fourier Basis and Z$_N$ Sectors}

\begin{proposition}[Fourier Decomposition into Z$_N$ Sectors] \tagDer
\label{prop:fourier_sectors}
The Fourier mode $e^{im\theta}$ belongs to Z$_N$ sector $\ell = m \mod N$:
\begin{equation}
R_{2\pi/N} \, e^{im\theta} = e^{im(\theta + 2\pi/N)} = e^{2\pi i m/N} \, e^{im\theta}
\label{eq:fourier_transform}
\end{equation}

Therefore, modes in sector $\ell$ have the form:
\begin{equation}
e^{i(\ell + kN)\theta}, \quad k \in \mathbb{Z}
\label{eq:sector_modes}
\end{equation}
\end{proposition}

\begin{tcolorbox}[colback=blue!5!white, colframe=blue!50!black,
                  title=\textbf{Key Insight: Sector Structure}]
The Z$_N$ symmetry decomposes the Hilbert space into $N$ sectors:
\begin{align}
\text{Sector } \ell = 0: \quad &\{1, e^{\pm iN\theta}, e^{\pm 2iN\theta}, \ldots\}
= \{1, \cos(N\theta), \cos(2N\theta), \ldots\} \\
\text{Sector } \ell = 1: \quad &\{e^{\pm i\theta}, e^{\pm i(N+1)\theta}, \ldots\} \\
&\vdots \\
\text{Sector } \ell = N-1: \quad &\{e^{\pm i(N-1)\theta}, e^{\pm i(2N-1)\theta}, \ldots\}
\end{align}

\textbf{The $\ell = 0$ sector contains all Z$_N$-symmetric functions.}
\end{tcolorbox}

%============================================================
\section{Pinning Term Coupling Analysis}
%============================================================

\subsection{Pinning Strength for Different Modes}

\begin{theorem}[Mode Coupling to Pinning Sites] \tagDer
\label{thm:coupling}
For a Fourier mode $e^{im\theta}$, the coupling to the pinning term is:
\begin{equation}
\sum_{n=0}^{N-1} e^{im\theta_n} = \sum_{n=0}^{N-1} e^{2\pi i m n/N} =
\begin{cases}
N & \text{if } m \equiv 0 \pmod{N} \\
0 & \text{otherwise}
\end{cases}
\label{eq:coupling_sum}
\end{equation}
\end{theorem}

\begin{proof}
This is the standard geometric series:
\begin{equation}
\sum_{n=0}^{N-1} \omega^{mn} = \frac{1 - \omega^{mN}}{1 - \omega^m}
\quad \text{where } \omega = e^{2\pi i/N}
\end{equation}
If $m \equiv 0 \pmod{N}$, then $\omega^m = 1$ and all terms equal 1.
Otherwise, $\omega^{mN} = 1$ but $\omega^m \neq 1$, so the sum vanishes.
\end{proof}

\begin{corollary}[Selective Coupling] \tagDer
\label{cor:selective}
The pinning term \textbf{couples only to Z$_N$-symmetric modes} (sector $\ell = 0$).

For real modes:
\begin{itemize}
\item $\cos(mN\theta)$: maximally coupled (all sites contribute coherently)
\item $\cos(m\theta)$ with $m \not\equiv 0 \pmod{N}$: zero net coupling
\end{itemize}
\end{corollary}

%============================================================
\section{Selection Lemma: $\cos(N\theta)$ as Leading Mode}
%============================================================

\subsection{Unpinned Spectrum}

\begin{proposition}[Free Ring Spectrum] \tagDer
\label{prop:free_spectrum}
Without pinning ($\lambda = 0$), the eigenvalues of $-T d^2/d\theta^2$ are:
\begin{equation}
\mu_m^{(0)} = T m^2, \quad m = 0, 1, 2, \ldots
\label{eq:free_spectrum}
\end{equation}
with eigenfunctions $\cos(m\theta)$, $\sin(m\theta)$ (degenerate for $m > 0$).
\end{proposition}

\subsection{Effect of Pinning: Perturbative Analysis}

\begin{theorem}[Mode Lifting by Pinning] \tagDer
\label{thm:mode_lifting}
Consider the dimensionless pinning strength:
\begin{equation}
\rho = \frac{\lambda\kappa}{T}
\label{eq:rho}
\end{equation}

\textbf{(a) Non-Z$_N$-symmetric modes} ($m \not\equiv 0 \pmod{N}$):
These are \textbf{unaffected} by pinning (zero net coupling).
Their eigenvalues remain $\mu_m = Tm^2$.

\textbf{(b) Z$_N$-symmetric modes} ($m = kN$):
These are \textbf{lifted} by the pinning term.
\end{theorem}

\begin{proof}
From Corollary \ref{cor:selective}, modes with $m \not\equiv 0 \pmod{N}$
have zero matrix element with the pinning perturbation.

For Z$_N$-symmetric modes, the pinning term acts as a positive definite
perturbation, raising their eigenvalues.
\end{proof}

\subsection{The Selection Lemma}

\begin{lemma}[Selection Lemma] \tagDer
\label{lem:selection}
Among all anisotropic modes (excluding the constant $m = 0$), the mode
with the \textbf{lowest gradient energy} that couples to the pinning sites is:
\begin{equation}
\boxed{v(\theta) = \cos(N\theta)}
\label{eq:leading_mode}
\end{equation}

This is the \textbf{first} Z$_N$-symmetric harmonic above the constant.
\end{lemma}

\begin{proof}
Z$_N$-symmetric harmonics have $m = kN$ for $k = 0, 1, 2, \ldots$
\begin{itemize}
\item $k = 0$: constant mode (isotropic, $m = 0$)
\item $k = 1$: $\cos(N\theta)$ with gradient energy $\propto N^2$
\item $k = 2$: $\cos(2N\theta)$ with gradient energy $\propto 4N^2$
\item etc.
\end{itemize}

The lowest-energy anisotropic mode that couples to the anchors is $\cos(N\theta)$.
\end{proof}

\subsection{Physical Interpretation}

\begin{tcolorbox}[colback=green!5!white, colframe=green!50!black,
                  title=\textbf{Why $\cos(N\theta)$ Dominates}]
\textbf{Two competing effects:}
\begin{enumerate}
\item \textbf{Gradient energy} favors low harmonics: $E_{\text{grad}} \propto m^2$
\item \textbf{Pinning coupling} requires $m \equiv 0 \pmod{N}$
\end{enumerate}

\textbf{Resolution:}
\begin{itemize}
\item Modes with $m < N$ (e.g., $\cos\theta$, $\cos 2\theta$) have lower gradient energy
      but \textbf{zero coupling} to the pinning sites.
\item The first mode that \textbf{both} couples to pinning \textbf{and} is anisotropic
      is $m = N$, i.e., $\cos(N\theta)$.
\end{itemize}

\textbf{Conclusion:} $\cos(N\theta)$ is the unique leading anisotropic mode.
\end{tcolorbox}

%============================================================
\section{Explicit Eigenvalue Calculation}
%============================================================

\subsection{Ansatz in Z$_N$-Symmetric Sector}

\begin{proposition}[Z$_N$-Symmetric Mode Eigenvalue] \tagDer
\label{prop:ZN_eigenvalue}
For the mode $v(\theta) = \cos(N\theta)$:

\textbf{Gradient contribution:}
\begin{equation}
-T v'' = TN^2 \cos(N\theta)
\end{equation}

\textbf{Pinning contribution:}
At $\theta_n = 2\pi n/N$: $\cos(N\theta_n) = \cos(2\pi n) = 1$.

The equation $\mathcal{L}v = \mu v$ in distributional form:
\begin{equation}
TN^2 \cos(N\theta) + \lambda\kappa \sum_{n=0}^{N-1} \delta(\theta - \theta_n) = \mu \cos(N\theta)
\end{equation}

Taking the inner product with $\cos(N\theta)$:
\begin{equation}
TN^2 \cdot \pi + \lambda\kappa \cdot N = \mu \cdot \pi
\end{equation}

Therefore:
\begin{equation}
\boxed{\mu_{N} = TN^2 + \frac{\lambda\kappa N}{\pi} = TN^2 \left(1 + \frac{\rho}{\pi N}\right)}
\label{eq:mu_N}
\end{equation}
\end{proposition}

\subsection{Comparison with Lower Modes}

\begin{proposition}[Non-Coupled Mode Eigenvalues] \tagDer
\label{prop:non_coupled}
For $m \not\equiv 0 \pmod{N}$, the eigenvalue is unchanged:
\begin{equation}
\mu_m = Tm^2 \quad (m = 1, 2, \ldots, N-1)
\label{eq:mu_m}
\end{equation}
\end{proposition}

\begin{tcolorbox}[colback=yellow!10!white, colframe=orange!70!black,
                  title=\textbf{Eigenvalue Ordering}]
\renewcommand{\arraystretch}{1.3}
\begin{tabular}{|l|l|l|l|}
\hline
\textbf{Mode} & \textbf{Eigenvalue} & \textbf{Z$_N$ Sector} & \textbf{Pinning} \\
\hline
$m = 0$ (const) & $0$ & $\ell = 0$ & coupled \\
$m = 1$ & $T$ & $\ell = 1$ & \textbf{not} coupled \\
$m = 2$ & $4T$ & $\ell = 2$ & \textbf{not} coupled \\
$\vdots$ & $\vdots$ & $\vdots$ & $\vdots$ \\
$m = N-1$ & $T(N-1)^2$ & $\ell = N-1$ & \textbf{not} coupled \\
$m = N$ & $TN^2(1 + \rho/\pi N)$ & $\ell = 0$ & \textbf{coupled} \\
$m = N+1$ & $T(N+1)^2$ & $\ell = 1$ & \textbf{not} coupled \\
\hline
\end{tabular}
\end{tcolorbox}

%============================================================
\section{Regime Analysis}
%============================================================

\subsection{Weak Pinning Regime}

\begin{definition}[Weak Pinning] \tagDer
\label{def:weak}
\begin{equation}
\rho = \frac{\lambda\kappa}{T} \ll N^2
\label{eq:weak_pinning}
\end{equation}
\end{definition}

\begin{proposition}[Weak Pinning Result] \tagDer
\label{prop:weak_result}
In the weak pinning regime:
\begin{itemize}
\item The mode structure is dominated by gradient energy.
\item $\cos(N\theta)$ is the \textbf{only} Z$_N$-symmetric anisotropic mode
      that couples to anchors at leading order.
\item Lower harmonics ($m < N$) do not couple, so cannot be excited by
      anchor forcing.
\end{itemize}
\end{proposition}

\subsection{Strong Pinning Regime}

\begin{definition}[Strong Pinning] \tagDer
\label{def:strong}
\begin{equation}
\rho = \frac{\lambda\kappa}{T} \gg N^2
\label{eq:strong_pinning}
\end{equation}
\end{definition}

\begin{proposition}[Strong Pinning Result] \tagDer
\label{prop:strong_result}
In the strong pinning regime:
\begin{itemize}
\item The field is pinned near $v = 0$ at all anchor sites.
\item Between anchors, the field interpolates smoothly.
\item The dominant mode structure \textbf{still} has Z$_N$ periodicity.
\item $\cos(N\theta)$ remains the leading anisotropic component.
\end{itemize}
\end{proposition}

%============================================================
\section{Conclusion}
%============================================================

\begin{tcolorbox}[colback=green!10!white, colframe=green!60!black,
                  title=\textbf{VERDICT: PASS}]
\textbf{Result:} Under the Z$_N$ delta-pinning model, $\cos(N\theta)$ is the
\textbf{unique leading anisotropic mode}.

\textbf{Proof summary:}
\begin{enumerate}
\item Pinning couples \textbf{only} to Z$_N$-symmetric modes (Theorem \ref{thm:coupling})
\item The lowest Z$_N$-symmetric anisotropic harmonic is $m = N$ (Lemma \ref{lem:selection})
\item Therefore, $\cos(N\theta)$ is the first mode that is both anisotropic
      and coupled to the anchors.
\end{enumerate}

\textbf{Conditions:}
\begin{itemize}
\item Identical anchors at Z$_N$ fixed points (verified in Israel derivation)
\item Quadratic W(u) near equilibrium (standard linearization)
\item Ring geometry with periodic boundary conditions
\end{itemize}

\textbf{This validates the ansatz} $u(\theta) = u_0 + a_1 \cos(N\theta)$
used in the $a/c = 1/N$ derivation.
\end{tcolorbox}

%============================================================
\section{Epistemic Status}
%============================================================

\renewcommand{\arraystretch}{1.3}
\begin{tabular}{|l|l|p{6cm}|}
\hline
\textbf{Result} & \textbf{Status} & \textbf{Comment} \\
\hline
\hline
Z$_N$ sector decomposition & \tagDer & Group theory \\
\hline
Pinning couples only to $\ell = 0$ sector & \tagDer & Geometric series identity \\
\hline
$\cos(N\theta)$ is lowest anisotropic in $\ell = 0$ & \tagDer & By construction \\
\hline
Selection Lemma & \tagDer & Follows from above \\
\hline
Eigenvalue formula \eqref{eq:mu_N} & \tagDer & Direct calculation \\
\hline
\hline
Mapping to 5D brane physics & \tagDc & Requires bulk solution \\
\hline
W(u) being quadratic near equilibrium & \tagDc & Assumed, not derived \\
\hline
\end{tabular}

\vspace{1em}

\textbf{Summary:} Within the delta-pinning ring model, all results are \tagDer.
The mapping to full 5D physics remains \tagDc.
